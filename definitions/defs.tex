\newcommand*\bang[1]{! #1}
%\newcommand{\ifthenelse}[3]{\texttt{if}\; #1 \texttt{ then}\; #2 \texttt{ else}\; #3}
\newcommand{\match}[4]{\texttt{match}\; #1\; \{ 0 \mapsto #2;\; s(#3) \mapsto #4 \}}

\newcommand{\outputch}[4]{\overline{#1}\langle #2\rangle^{#3}\!.#4}
\newcommand{\inputch}[4]{#1(#2)^{#3}.#4}

\newcommand{\asyncoutputch}[3]{\overline{#1}\!\langle #2\rangle^{#3}}
\newcommand{\asyncinputch}[3]{#1\!\left( #2\right)^{#3}}
\newcommand{\asyncrepinputch}[3]{\bang{\asyncinputch{#1}{#2}{#3}}}


\newcommand{\splitatcommas}[1]{%
  \begingroup
  \begingroup\lccode`~=`, \lowercase{\endgroup
    \edef~{\mathchar\the\mathcode`, \penalty0 \noexpand\hspace{0pt plus 1em}}%
  }\mathcode`,="8000 #1%
  \endgroup
}



\newcommand{\parcomp}[2]{#1 \mid #2}
\newcommand{\parcompthree}[3]{\parcomp{\parcomp{#1}{#2}}{#3}}
\newcommand{\parcompfour}[4]{\parcompthree{#1}{#2}{\parcomp{#3}{#4}}}
\newcommand{\parcompseven}[7]{#1 \mid #2 \mid #3 \mid #4 \mid #5 \mid #6 \mid #7}

\newcommand{\uparcomp}[2]{\parcomp{#1}{#2}}
\newcommand{\uparcompthree}[3]{\uparcomp{\uparcomp{#1}{#2}}{#3}}
\newcommand{\uparcompfour}[4]{\uparcompthree{#1}{#2}{\uparcomp{#3}{#4}}}
\newcommand{\uparcompfive}[5]{\uparcompfour{#1}{#2}{#3}{\uparcomp{#4}{#5}}}

\newcommand{\newvar}[2]{(\nu #1) #2}
\newcommand{\newvarU}[1]{\left(\nu #1\right)}

\newcommand{\nil}{\mathbf{0}}

\newcommand{\freemodule}[0]{\mathbb{Z}[i_1,\dots,i_n]}

\newcommand{\true}{\textit{true}}
\newcommand{\false}{\textit{false}}

\newcommand{\succeeds}{\mathbf{c}}

\newcommand{\succc}[1]{s(#1)}

\newcommand{\tick}[1]{\texttt{tick}.#1}


\newcommand{\dasfwr}[1]{[#1]^{-1}_R}


\newcommand{\subst}[2]{#1\!\left[#2\right]}

\newcommand{\substi}[2]{\{#1/#2\}}

% Type stuff
\newcommand{\withusage}[2]{#1/#2}
\newcommand{\withtype}[2]{#1:#2}
\newcommand{\channeltype}[1]{\texttt{ch}(#1)}
\newcommand{\channeltypeusage}[2]{\withusage{\channeltype{#1}}{#2}}
\newcommand{\inchanneltypeS}[2]{\texttt{in}_{#1}(#2)}
\newcommand{\outchanneltypeS}[2]{\texttt{out}_{#1}(#2)}
\newcommand{\channeltypeS}[2]{\texttt{ch}_{#1}(#2)}
\newcommand{\tparcomp}[2]{\parcomp{#1}{#2}}
\newcommand{\withdelay}[2]{\uparrow^{#1}\!\!#2}
\newcommand{\usagepref}[3]{#1^{#2}_{#3}}
\newcommand{\inusagesym}[0]{\texttt{In}}
\newcommand{\outusagesym}[0]{\texttt{Out}}
\newcommand{\inusagepref}[2]{\usagepref{\inusagesym}{#1}{#2}}
\newcommand{\outusagepref}[2]{\usagepref{\outusagesym}{#1}{#2}}
\newcommand{\repinusagepref}[2]{\bang{\usagepref{\inusagesym}{#1}{#2}}}
\newcommand{\usagerep}[1]{\;*\!#1}
\newcommand{\errres}[0]{\textbf{\texttt{err}}}
\newcommand{\comlabel}[0]{\textbf{com}}

\newcommand{\withcomplex}[2]{#1 \triangleleft #2}

\newcommand{\typenat}[0]{\texttt{Nat}}
\newcommand{\typechanusage}[2]{\withusage{\channeltype{#1}}{#2}}
\newcommand{\kinterval}[2]{\left[#1,#2\right]}
\newcommand{\kintervalsingle}[1]{\left[#1\right]}
\newcommand{\natinterval}[2]{\typenat\!\kinterval{#1}{#2}}
\newcommand{\natintervalsingle}[1]{\typenat\!\left[#1\right]}

\newcommand{\encoding}[1]{\left[\!\left[#1\right]\!\right]}


\newcommand\defeq{\stackrel{\mathclap{\normalfont\tiny\mbox{def}}}{=}}


\newcommand{\servt}[5]{\forall_{#1} #2.\texttt{serv}^{#3}_{#4}(#5)}
\newcommand{\chant}[3]{\texttt{ch}^{#1}_{#2}(#3)}

\newcommand{\servU}[4]{\withusage{\forall #1.\texttt{serv}^{#2}(#3)}{#4}}

\newcommand{\reliableU}[1]{#1\;\text{reliable}\;}
\newcommand{\reliableT}[1]{#1\;\text{reliable}\;}

%\newcommand{\susume}[4]{\langle #1 \rangle^{#2;#3}_{#4}} % old
\newcommand{\susume}[4]{\downarrow^{#2;#3}_{#4}\!\!(#1)} % new
\newcommand{\susumesim}[2]{\downarrow_{#2}\!\!#1}
\newcommand{\tforward}[4]{\susume{#1}{#2}{#3}{#4}}
\newcommand{\tforwardsim}[2]{\susumesim{#1}{#2}}

\newcommand{\vect}[1]{\texttt{(#1)}}
\newcommand{\evect}[2][]{\vect{#2}_{\!#1}}
\newcommand{\cvect}[2][\varphi]{\vect{#2}_{\!#1}}

\newcommand{\normlinearindex}[3][\mathcal{E}(I)]{#2 + \sum_{\alpha\in #1} #3_\alpha i_\alpha}

\newcommand{\monus}[0]{\dot -}
\newcommand{\monusE}[1][\varphi;\Phi]{\monus_{#1}}

% Fede mængder

\newcommand{\skat}[1]{\textbf{#1}\xspace}
\newcommand{\EnvV}{\skat{EnvV}}
\newcommand{\Var}{\skat{Var}}
\newcommand{\Store}{\skat{Store}}
\newcommand{\Loc}{\skat{Loc}}

% De hele tal

\newcommand{\Z}{\ensuremath{\mathbb{Z}}}

% Inferensregler

\newcommand{\condinfrule}[3]
          {\parbox{5.5cm}{$$ {\frac{#1}{#2}}{\qquad
            #3} \hfill  $$}}

\newcommand{\infrule}[2]
          {\parbox{4.5cm}{$$ \frac{#1}{#2}\hspace{.5cm}$$}}

% Regelnavne
           
\newcommand{\runa}[1]{\mbox{\textsc{\protect{(#1})}}}
\newcommand{\runatt}[2]{$[{\mbox{\textsc{#1}}}_{\mbox{\textsc{\small
        #2}}}]$\xspace}

% Pile

\newcommand{\ra}[1][\relax]{\ensuremath \rightarrow_{#1}}
\newcommand{\lra}{\longrightarrow}
\newcommand{\Ra}{\Rightarrow}
\newcommand{\pra}{\ensuremath \rightharpoonup }

% Kantede parenteser

\newcommand{\lag}{\langle}
\newcommand{\rag}{\rangle}
\newcommand{\conf}[1]{\ensuremath{\lag #1 \rag}}

% Mængder

\newcommand{\setof}[2]{\ensuremath{\{ #1 \mid #2 \}}}
\newcommand{\set}[1]{\ensuremath{\{ #1 \}}}

% Kommandoer i Bims

\newcommand{\skib}{\texttt{skip}}
%\newcommand{\ifthenelse}[3]{\texttt{if}\; #1 \; \texttt{then}\; #2 \;
%  \texttt{else}\; #3}
%\newcommand{\whiledo}[2]{\texttt{while}\; #1 \; \texttt{do}\; #2}

% Neu commands
\newcommand{\hole}{(\!\mid\mid\!)}
\newcommand{\nehole}[1]{\mathord{(\!\mid\!\!#1\!\!\mid\!)}}
\newcommand{\prerep}{[^{m}\!\!\//_{\!\!D}]}
\newcommand{\replace}[1]{\lbrbrak #1 \rbrbrak}
\newcommand{\cursor}[1]{\mathord{
    \lBrack\mspace{1mu}#1\mspace{1mu}\rBrack
}}
\newcommand{\breakpoint}[1]{
\mathord{
    \lAngle\mspace{1mu}#1\mspace{1mu}\rAngle
}}
\newcommand{\wellformed}[1]{\letvdash{wf}#1}
\newcommand{\complete}[1]{\letvdash{c}#1}
\newcommand{\cursorexc}[1]{\letvdash{ce}#1}
\newcommand{\nodevalid}[1]{\letvdash{nv}#1}
\newcommand{\cursorctx}[1]{C\!\left[#1\right]}
\newcommand{\cursorctxhole}{\left[\cdot\right]}
\newcommand{\recursion}[2]{#1\!\left(#2\right)}
% Next

\newcommand{\nexte}{\textrm{next}\xspace}

\newcommand{\editelig}[1]{\mathcal{E}#1}
\newcommand{\noteditelig}[1]{\overline{\mathcal{E}#1}}

\newcommand{\consistent}[2]{#1 \sim #2}

% Omvendt \vdash
\usepackage{graphicx}

\makeatletter
\providecommand*{\dashv}{%
  \mathrel{%
    \mathpalette\@dashv\vdash
  }%
}
\newcommand*{\@dashv}[2]{%
  \reflectbox{$\m@th#1#2$}%
}

\newcommand{\f}{\mkern-2mu f\mkern-3mu}%fix f i mathmode

\newcommand{\cmdchild}[1]{\texttt{child}\; #1}
\newcommand{\cmdparent}{\texttt{parent}}
\newcommand{\cmdlambda}[1]{\texttt{lambda}\; #1}
\newcommand{\cmdhole}{\texttt{hole}}
\newcommand{\cmdrec}[1]{\texttt{rec}\; #1.}
\newcommand{\cmdapp}{\texttt{app}}
\newcommand{\cmdbreak}{\texttt{break}}


\newcommand{\condexp}[2]{#1 \Rightarrow #2}
\newcommand{\condexpto}[3]{#1 \Rightarrow #2 \vert #3}

\definecolor{codegreen}{rgb}{0,0.6,0}
\definecolor{codegray}{rgb}{0.5,0.5,0.5}
\definecolor{codepurple}{rgb}{0.58,0,0.82}
\definecolor{backcolour}{rgb}{0.95,0.95,0.92}

\lstdefinestyle{mystyle}{
    backgroundcolor=\color{backcolour},   
    commentstyle=\color{codegreen},
    keywordstyle=\color{magenta},
    numberstyle=\tiny\color{codegray},
    stringstyle=\color{codepurple},
    basicstyle=\ttfamily\footnotesize,
    breakatwhitespace=false,         
    breaklines=true,                 
    captionpos=b,                    
    keepspaces=true,                 
    numbers=left,                    
    numbersep=5pt,                  
    showspaces=false,                
    showstringspaces=false,
    showtabs=false,                  
    tabsize=2
}

\lstset{style=mystyle}

\makeatother
