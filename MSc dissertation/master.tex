%\documentclass[fleqn, 11pt]{article}
\documentclass[11pt,a4paper,oldfontcommands]{memoir}
%\usepackage[danish]{babel}
%\usepackage{boisik}
%\usepackage[T1]{fontenc}
\usepackage{mathtools}
\usepackage{stackengine}
\newcommand\letvdash[1]{\mathrel{
  \stackengine{1.2ex}{\vdash}{\;\;\scriptscriptstyle#1}{O}{c}{F}{T}{L}}}
\stackMath
%\usepackage{stix}
\usepackage{upgreek}%alpha symbol
\usepackage{adjmulticol}
\usepackage[utf8]{inputenc}
% xspace sørger for rigtigt mellemrum efter en kommando-anvendelse
\setlength\parindent{0pt}
\usepackage{xspace}
\usepackage{comment}
\usepackage{multicol}
\usepackage{wasysym}
\usepackage{float}
\usepackage{array}

% Memoir stuff
\usepackage[T1]{fontenc}
\usepackage{microtype}
\usepackage[dvips]{graphicx}
\usepackage[dvipsnames]{xcolor}
\usepackage{times}

\usepackage{listings}
\usepackage{tikz}
\usepackage{pgfplots}
\usepackage{mathrsfs}
\usepgfplotslibrary{external}
\usetikzlibrary{external}
\tikzexternalize[prefix=tikz/]

\usepackage[
breaklinks=true,colorlinks=true,
%linkcolor=blue,urlcolor=blue,citecolor=blue,% PDF VIEW
linkcolor=black,urlcolor=black,citecolor=black,% PRINT
bookmarks=true,bookmarksopenlevel=2]{hyperref}

\usepackage{geometry}
% PDF VIEW
% \geometry{total={210mm,297mm},
% left=25mm,right=25mm,%
% bindingoffset=0mm, top=25mm,bottom=25mm}
% PRINT
\geometry{total={210mm,297mm},
left=20mm,right=20mm,
bindingoffset=10mm, top=25mm,bottom=25mm}

\OnehalfSpacing
%\linespread{1.3}

\usepackage{bm}

%%% CHAPTER'S STYLE
\chapterstyle{bianchi}
%\chapterstyle{ger}
%\chapterstyle{madsen}
%\chapterstyle{ell}
%%% STYLE OF SECTIONS, SUBSECTIONS, AND SUBSUBSECTIONS
\setsecheadstyle{\Large\bfseries\sffamily\raggedright}
\setsubsecheadstyle{\large\bfseries\sffamily\raggedright}
\setsubsubsecheadstyle{\bfseries\sffamily\raggedright}


%%% STYLE OF PAGES NUMBERING
%\pagestyle{companion}\nouppercaseheads 
%\pagestyle{headings}
%\pagestyle{Ruled}
\pagestyle{plain}
\makepagestyle{plain}
\makeevenfoot{plain}{\thepage}{}{}
\makeoddfoot{plain}{}{}{\thepage}
\makeevenhead{plain}{}{}{}
\makeoddhead{plain}{}{}{}


\maxsecnumdepth{subsection} % chapters, sections, and subsections are numbered
\maxtocdepth{subsection} % chapters, sections, and subsections are in the Table of Contents

%Memoir stuff end

%\usepackage{algorithm}
%\usepackage{algpseudocode}
\usepackage{algorithm2e}

%\usepackage{breqn}

% amsmath giver align-environments og meget mere
\usepackage{amsmath}
\newcommand\numberthis{\addtocounter{equation}{1}\tag{\theequation}}
\usepackage{amsthm}

\usepackage{stmaryrd}

\theoremstyle{proposition}
\newtheorem{prop}{Proposition}

\theoremstyle{proposition}
\newtheorem{conj}{Conjecture}[section]


\theoremstyle{definition}
\newtheorem{definition}{Definition}[section]
\newtheorem{exmp}{Example}[section]
\newtheorem{theorem}{Theorem}
\newtheorem{corollary}{Corollary}[theorem]

\theoremstyle{plain}
\newtheorem{lemma}[theorem]{Lemma}

\usepackage{framed}
\theoremstyle{remark}
\newtheorem*{remark}{Remark}

\theoremstyle{definition}
\newtheorem{defi/}{Definition}[section]

\newenvironment{defi}
  {\renewcommand{\qedsymbol}{$\triangleleft$}%
   \pushQED{\qed}\begin{defi/}}
  {\popQED\end{defi/}}

%\theoremstyle{definition}
\newtheorem{examp/}{Example}[section]
\newenvironment{examp}
  {\renewcommand{\qedsymbol}{$\triangleright$}%
   \pushQED{\qed}\begin{examp/}}
  {\popQED\end{examp/}}



% \addtolength{\oddsidemargin}{-.475in}
% \addtolength{\evensidemargin}{-.475in}
% \addtolength{\textwidth}{1.05in}

% amssymb giver mulighed for "fede" symboler i mathmode til N, Z mm.

\usepackage{amssymb}

\usepackage{enumerate}
\usepackage{ebproof}
% Fede mængder

\newcommand{\skat}[1]{\textbf{#1}\xspace}
\newcommand{\EnvV}{\skat{EnvV}}
\newcommand{\Var}{\skat{Var}}
\newcommand{\Store}{\skat{Store}}
\newcommand{\Loc}{\skat{Loc}}

% De hele tal

\newcommand{\Z}{\ensuremath{\mathbb{Z}}}

% Inferensregler

\newcommand{\condinfrule}[3]
          {\parbox{5.5cm}{$$ {\frac{#1}{#2}}{\qquad
            #3} \hfill  $$}}

\newcommand{\infrule}[2]
          {\parbox{4.5cm}{$$ \frac{#1}{#2}\hspace{.5cm}$$}}

% Regelnavne
           
\newcommand{\runa}[1]{\mbox{\textsc{\protect{(#1})}}}
\newcommand{\runatt}[2]{$[{\mbox{\textsc{#1}}}_{\mbox{\textsc{\small
        #2}}}]$\xspace}

% Pile

\newcommand{\ra}[1][\relax]{\ensuremath \rightarrow_{#1}}
\newcommand{\lra}{\longrightarrow}
\newcommand{\Ra}{\Rightarrow}
\newcommand{\pra}{\ensuremath \rightharpoonup }

% Kantede parenteser

\newcommand{\lag}{\langle}
\newcommand{\rag}{\rangle}
\newcommand{\conf}[1]{\ensuremath{\lag #1 \rag}}

% Mængder

\newcommand{\setof}[2]{\ensuremath{\{ #1 \mid #2 \}}}
\newcommand{\set}[1]{\ensuremath{\{ #1 \}}}

% Kommandoer i Bims

\newcommand{\skib}{\texttt{skip}}
%\newcommand{\ifthenelse}[3]{\texttt{if}\; #1 \; \texttt{then}\; #2 \;
%  \texttt{else}\; #3}
%\newcommand{\whiledo}[2]{\texttt{while}\; #1 \; \texttt{do}\; #2}

% Neu commands
\newcommand{\hole}{(\!\mid\mid\!)}
\newcommand{\nehole}[1]{\mathord{(\!\mid\!\!#1\!\!\mid\!)}}
\newcommand{\prerep}{[^{m}\!\!\//_{\!\!D}]}
\newcommand{\replace}[1]{\lbrbrak #1 \rbrbrak}
\newcommand{\cursor}[1]{\mathord{
    \lBrack\mspace{1mu}#1\mspace{1mu}\rBrack
}}
\newcommand{\breakpoint}[1]{
\mathord{
    \lAngle\mspace{1mu}#1\mspace{1mu}\rAngle
}}
\newcommand{\wellformed}[1]{\letvdash{wf}#1}
\newcommand{\complete}[1]{\letvdash{c}#1}
\newcommand{\cursorexc}[1]{\letvdash{ce}#1}
\newcommand{\nodevalid}[1]{\letvdash{nv}#1}
\newcommand{\cursorctx}[1]{C\!\left[#1\right]}
\newcommand{\cursorctxhole}{\left[\cdot\right]}
\newcommand{\recursion}[2]{#1\!\left(#2\right)}
% Next

\newcommand{\nexte}{\textrm{next}\xspace}

\newcommand{\editelig}[1]{\mathcal{E}#1}
\newcommand{\noteditelig}[1]{\overline{\mathcal{E}#1}}

\newcommand{\consistent}[2]{#1 \sim #2}

% Omvendt \vdash
\usepackage{graphicx}

\makeatletter
\providecommand*{\dashv}{%
  \mathrel{%
    \mathpalette\@dashv\vdash
  }%
}
\newcommand*{\@dashv}[2]{%
  \reflectbox{$\m@th#1#2$}%
}

\newcommand{\f}{\mkern-2mu f\mkern-3mu}%fix f i mathmode

\newcommand{\cmdchild}[1]{\texttt{child}\; #1}
\newcommand{\cmdparent}{\texttt{parent}}
\newcommand{\cmdlambda}[1]{\texttt{lambda}\; #1}
\newcommand{\cmdhole}{\texttt{hole}}
\newcommand{\cmdrec}[1]{\texttt{rec}\; #1.}
\newcommand{\cmdapp}{\texttt{app}}
\newcommand{\cmdbreak}{\texttt{break}}


\newcommand{\condexp}[2]{#1 \Rightarrow #2}
\newcommand{\condexpto}[3]{#1 \Rightarrow #2 \vert #3}

\definecolor{codegreen}{rgb}{0,0.6,0}
\definecolor{codegray}{rgb}{0.5,0.5,0.5}
\definecolor{codepurple}{rgb}{0.58,0,0.82}
\definecolor{backcolour}{rgb}{0.95,0.95,0.92}

\lstdefinestyle{mystyle}{
    backgroundcolor=\color{backcolour},   
    commentstyle=\color{codegreen},
    keywordstyle=\color{magenta},
    numberstyle=\tiny\color{codegray},
    stringstyle=\color{codepurple},
    basicstyle=\ttfamily\footnotesize,
    breakatwhitespace=false,         
    breaklines=true,                 
    captionpos=b,                    
    keepspaces=true,                 
    numbers=left,                    
    numbersep=5pt,                  
    showspaces=false,                
    showstringspaces=false,
    showtabs=false,                  
    tabsize=2
}

\lstset{style=mystyle}

\makeatother

% HER KOMMER TITLEN

\usepackage{authblk}

\title{Implementations of Sized Types for Parallel Complexity of Message-passing Processes}

%\usepackage{newpxmath} % math font is Palatino compatible

%\author{Herrmann, Thomas. Lauridsen, Mikkel Korup}

%\def\email#1{{\tt#1}}

\author{Thomas Herrmann}
\author{Mikkel Korup Lauridsen}
\affil{\{therrm17,mkla17\}@student.aau.dk}
%\affil{mkla17@student.aau.dk}
\affil{Department of Computer Science, Aalborg University}
%\affil{Selma Lagerlöfs Vej 300}
\affil{Aalborg, Denmark}
\date{June 17, 2022}
%\affil{Denmark}
%\email{therrm17@student.aau.dk}
%\authornotemark[1]
%\affiliation{%
%  \institution{Department of Computer Science, Aalborg University}
%  \streetaddress{Selma Lagerlöfs Vej 300}
%  \city{Aalborg}
%  \country{Denmark}}
%
%\author[2]{Lauridsen, Mikkel Korup}
%\email{mkla17@student.aau.dk}
%\authornotemark[1]
%\affiliation{%
%  \institution{Department of Computer Science, Aalborg University}
%  \streetaddress{Selma Lagerlöfs Vej 300}
%  \city{Aalborg}
%  \country{Denmark}}

\newcommand*\bang[1]{! #1}
%\newcommand{\ifthenelse}[3]{\texttt{if}\; #1 \texttt{ then}\; #2 \texttt{ else}\; #3}
\newcommand{\match}[4]{\texttt{match}\; #1\; \{ 0 \mapsto #2;\; s(#3) \mapsto #4 \}}

\newcommand{\outputch}[4]{\overline{#1}\langle #2\rangle^{#3}\!.#4}
\newcommand{\inputch}[4]{#1(#2)^{#3}.#4}

\newcommand{\asyncoutputch}[3]{\overline{#1}\!\langle #2\rangle^{#3}}
\newcommand{\asyncinputch}[3]{#1\!\left( #2\right)^{#3}}
\newcommand{\asyncrepinputch}[3]{\bang{\asyncinputch{#1}{#2}{#3}}}


\newcommand{\splitatcommas}[1]{%
  \begingroup
  \begingroup\lccode`~=`, \lowercase{\endgroup
    \edef~{\mathchar\the\mathcode`, \penalty0 \noexpand\hspace{0pt plus 1em}}%
  }\mathcode`,="8000 #1%
  \endgroup
}



\newcommand{\parcomp}[2]{#1 \mid #2}
\newcommand{\parcompthree}[3]{\parcomp{\parcomp{#1}{#2}}{#3}}
\newcommand{\parcompfour}[4]{\parcompthree{#1}{#2}{\parcomp{#3}{#4}}}
\newcommand{\parcompseven}[7]{#1 \mid #2 \mid #3 \mid #4 \mid #5 \mid #6 \mid #7}

\newcommand{\uparcomp}[2]{\parcomp{#1}{#2}}
\newcommand{\uparcompthree}[3]{\uparcomp{\uparcomp{#1}{#2}}{#3}}
\newcommand{\uparcompfour}[4]{\uparcompthree{#1}{#2}{\uparcomp{#3}{#4}}}
\newcommand{\uparcompfive}[5]{\uparcompfour{#1}{#2}{#3}{\uparcomp{#4}{#5}}}

\newcommand{\newvar}[2]{(\nu #1) #2}
\newcommand{\newvarU}[1]{\left(\nu #1\right)}

\newcommand{\nil}{\mathbf{0}}

\newcommand{\freemodule}[0]{\mathbb{Z}[i_1,\dots,i_n]}

\newcommand{\true}{\textit{true}}
\newcommand{\false}{\textit{false}}

\newcommand{\succeeds}{\mathbf{c}}

\newcommand{\succc}[1]{s(#1)}

\newcommand{\tick}[1]{\texttt{tick}.#1}


\newcommand{\dasfwr}[1]{[#1]^{-1}_R}


\newcommand{\subst}[2]{#1\!\left[#2\right]}

\newcommand{\substi}[2]{\{#1/#2\}}

% Type stuff
\newcommand{\withusage}[2]{#1/#2}
\newcommand{\withtype}[2]{#1:#2}
\newcommand{\channeltype}[1]{\texttt{ch}(#1)}
\newcommand{\channeltypeusage}[2]{\withusage{\channeltype{#1}}{#2}}
\newcommand{\inchanneltypeS}[2]{\texttt{in}_{#1}(#2)}
\newcommand{\outchanneltypeS}[2]{\texttt{out}_{#1}(#2)}
\newcommand{\channeltypeS}[2]{\texttt{ch}_{#1}(#2)}
\newcommand{\tparcomp}[2]{\parcomp{#1}{#2}}
\newcommand{\withdelay}[2]{\uparrow^{#1}\!\!#2}
\newcommand{\usagepref}[3]{#1^{#2}_{#3}}
\newcommand{\inusagesym}[0]{\texttt{In}}
\newcommand{\outusagesym}[0]{\texttt{Out}}
\newcommand{\inusagepref}[2]{\usagepref{\inusagesym}{#1}{#2}}
\newcommand{\outusagepref}[2]{\usagepref{\outusagesym}{#1}{#2}}
\newcommand{\repinusagepref}[2]{\bang{\usagepref{\inusagesym}{#1}{#2}}}
\newcommand{\usagerep}[1]{\;*\!#1}
\newcommand{\errres}[0]{\textbf{\texttt{err}}}
\newcommand{\comlabel}[0]{\textbf{com}}

\newcommand{\withcomplex}[2]{#1 \triangleleft #2}

\newcommand{\typenat}[0]{\texttt{Nat}}
\newcommand{\typechanusage}[2]{\withusage{\channeltype{#1}}{#2}}
\newcommand{\kinterval}[2]{\left[#1,#2\right]}
\newcommand{\kintervalsingle}[1]{\left[#1\right]}
\newcommand{\natinterval}[2]{\typenat\!\kinterval{#1}{#2}}
\newcommand{\natintervalsingle}[1]{\typenat\!\left[#1\right]}

\newcommand{\encoding}[1]{\left[\!\left[#1\right]\!\right]}


\newcommand\defeq{\stackrel{\mathclap{\normalfont\tiny\mbox{def}}}{=}}


\newcommand{\servt}[5]{\forall_{#1} #2.\texttt{serv}^{#3}_{#4}(#5)}
\newcommand{\chant}[3]{\texttt{ch}^{#1}_{#2}(#3)}

\newcommand{\servU}[4]{\withusage{\forall #1.\texttt{serv}^{#2}(#3)}{#4}}

\newcommand{\reliableU}[1]{#1\;\text{reliable}\;}
\newcommand{\reliableT}[1]{#1\;\text{reliable}\;}

%\newcommand{\susume}[4]{\langle #1 \rangle^{#2;#3}_{#4}} % old
\newcommand{\susume}[4]{\downarrow^{#2;#3}_{#4}\!\!(#1)} % new
\newcommand{\susumesim}[2]{\downarrow_{#2}\!\!#1}
\newcommand{\tforward}[4]{\susume{#1}{#2}{#3}{#4}}
\newcommand{\tforwardsim}[2]{\susumesim{#1}{#2}}

\newcommand{\vect}[1]{\texttt{(#1)}}
\newcommand{\evect}[2][]{\vect{#2}_{\!#1}}
\newcommand{\cvect}[2][\varphi]{\vect{#2}_{\!#1}}

\newcommand{\normlinearindex}[3][\mathcal{E}(I)]{#2 + \sum_{\alpha\in #1} #3_\alpha i_\alpha}

\newcommand{\monus}[0]{\dot -}
\newcommand{\monusE}[1][\varphi;\Phi]{\monus_{#1}}

% Fede mængder

\newcommand{\skat}[1]{\textbf{#1}\xspace}
\newcommand{\EnvV}{\skat{EnvV}}
\newcommand{\Var}{\skat{Var}}
\newcommand{\Store}{\skat{Store}}
\newcommand{\Loc}{\skat{Loc}}

% De hele tal

\newcommand{\Z}{\ensuremath{\mathbb{Z}}}

% Inferensregler

\newcommand{\condinfrule}[3]
          {\parbox{5.5cm}{$$ {\frac{#1}{#2}}{\qquad
            #3} \hfill  $$}}

\newcommand{\infrule}[2]
          {\parbox{4.5cm}{$$ \frac{#1}{#2}\hspace{.5cm}$$}}

% Regelnavne
           
\newcommand{\runa}[1]{\mbox{\textsc{\protect{(#1})}}}
\newcommand{\runatt}[2]{$[{\mbox{\textsc{#1}}}_{\mbox{\textsc{\small
        #2}}}]$\xspace}

% Pile

\newcommand{\ra}[1][\relax]{\ensuremath \rightarrow_{#1}}
\newcommand{\lra}{\longrightarrow}
\newcommand{\Ra}{\Rightarrow}
\newcommand{\pra}{\ensuremath \rightharpoonup }

% Kantede parenteser

\newcommand{\lag}{\langle}
\newcommand{\rag}{\rangle}
\newcommand{\conf}[1]{\ensuremath{\lag #1 \rag}}

% Mængder

\newcommand{\setof}[2]{\ensuremath{\{ #1 \mid #2 \}}}
\newcommand{\set}[1]{\ensuremath{\{ #1 \}}}

% Kommandoer i Bims

\newcommand{\skib}{\texttt{skip}}
%\newcommand{\ifthenelse}[3]{\texttt{if}\; #1 \; \texttt{then}\; #2 \;
%  \texttt{else}\; #3}
%\newcommand{\whiledo}[2]{\texttt{while}\; #1 \; \texttt{do}\; #2}

% Neu commands
\newcommand{\hole}{(\!\mid\mid\!)}
\newcommand{\nehole}[1]{\mathord{(\!\mid\!\!#1\!\!\mid\!)}}
\newcommand{\prerep}{[^{m}\!\!\//_{\!\!D}]}
\newcommand{\replace}[1]{\lbrbrak #1 \rbrbrak}
\newcommand{\cursor}[1]{\mathord{
    \lBrack\mspace{1mu}#1\mspace{1mu}\rBrack
}}
\newcommand{\breakpoint}[1]{
\mathord{
    \lAngle\mspace{1mu}#1\mspace{1mu}\rAngle
}}
\newcommand{\wellformed}[1]{\letvdash{wf}#1}
\newcommand{\complete}[1]{\letvdash{c}#1}
\newcommand{\cursorexc}[1]{\letvdash{ce}#1}
\newcommand{\nodevalid}[1]{\letvdash{nv}#1}
\newcommand{\cursorctx}[1]{C\!\left[#1\right]}
\newcommand{\cursorctxhole}{\left[\cdot\right]}
\newcommand{\recursion}[2]{#1\!\left(#2\right)}
% Next

\newcommand{\nexte}{\textrm{next}\xspace}

\newcommand{\editelig}[1]{\mathcal{E}#1}
\newcommand{\noteditelig}[1]{\overline{\mathcal{E}#1}}

\newcommand{\consistent}[2]{#1 \sim #2}

% Omvendt \vdash
\usepackage{graphicx}

\makeatletter
\providecommand*{\dashv}{%
  \mathrel{%
    \mathpalette\@dashv\vdash
  }%
}
\newcommand*{\@dashv}[2]{%
  \reflectbox{$\m@th#1#2$}%
}

\newcommand{\f}{\mkern-2mu f\mkern-3mu}%fix f i mathmode

\newcommand{\cmdchild}[1]{\texttt{child}\; #1}
\newcommand{\cmdparent}{\texttt{parent}}
\newcommand{\cmdlambda}[1]{\texttt{lambda}\; #1}
\newcommand{\cmdhole}{\texttt{hole}}
\newcommand{\cmdrec}[1]{\texttt{rec}\; #1.}
\newcommand{\cmdapp}{\texttt{app}}
\newcommand{\cmdbreak}{\texttt{break}}


\newcommand{\condexp}[2]{#1 \Rightarrow #2}
\newcommand{\condexpto}[3]{#1 \Rightarrow #2 \vert #3}

\definecolor{codegreen}{rgb}{0,0.6,0}
\definecolor{codegray}{rgb}{0.5,0.5,0.5}
\definecolor{codepurple}{rgb}{0.58,0,0.82}
\definecolor{backcolour}{rgb}{0.95,0.95,0.92}

\lstdefinestyle{mystyle}{
    backgroundcolor=\color{backcolour},   
    commentstyle=\color{codegreen},
    keywordstyle=\color{magenta},
    numberstyle=\tiny\color{codegray},
    stringstyle=\color{codepurple},
    basicstyle=\ttfamily\footnotesize,
    breakatwhitespace=false,         
    breaklines=true,                 
    captionpos=b,                    
    keepspaces=true,                 
    numbers=left,                    
    numbersep=5pt,                  
    showspaces=false,                
    showstringspaces=false,
    showtabs=false,                  
    tabsize=2
}

\lstset{style=mystyle}

\makeatother


\begin{document}

\section*{Summary}

Type systems have been studied extensively in the domain of static complexity analysis, to formalize rules that can describe the relationships between a program and its resource use in terms of time and memory (space). Formal methods such as type systems have the advantage that they are typically proved sound. In this thesis, we explore the challenges of implementing both type checking and type inference for a type system for parallel complexity of message-passing processes introduced by Baillot and Ghyselen \cite{BaillotGhyselen2021} that has until now not been implemented. Such implementations of type checking and type inference, paired with corresponding soundness proofs, enable verification or inference of correct complexity bounds, respectively. The type system builds on sized types to express parametric complexity, combined with input/output types to bound synchronizations on channels. \\

After briefly presenting the variant of the pi-calculus considered by Baillot and Ghyselen as well as a more formal description of parallel complexity, we provide an overview of their type system. An important concept is that of \textit{constraint judgements}. These judgements allow us to compare parametric sizes and complexity bounds under a set of known constraints on otherwise unknown sizes. We show both types and type rules of the type system, as well as present examples of typings for processes.\\

Constructing a type checker for the type system by Baillot and Ghyselen introduces a number of challenges. In creating algorithmic type rules for their type system, a notable challenge is maintaining the subject reduction property of the type system. That is, if we are not careful, the reduction relation of $\pi$-calculus processes may not be type-preserving under our modified type rules. We solve this problem by introducing the idea of \textit{combined complexities} consisting of a set of intersecting parametric complexity bounds. As such, a combined complexity represents the maximum of all the complexities contained within. We also define the accompanying function \textit{basis} that keeps a combined complexity as a minimal set by removing complexities that never contribute to the actual bound. Finally, we prove our algorithmic type rules sound, to this effort proving a weaker subject reduction property. We also show how the type checker may be extended with more practical constructs, and present an example of a parallel merge sort encoding that would then be typable using our modified rules.\\

We next show how constraint judgements may be verified. As these judgements are universally quantified over size variables (referred to as index variables), comparison of sizes and complexity bounds is a partial order. We first limit ourselves to constraint judgements over linear functions that may be verified by reduction to integer programs, or alternatively by over-approximation using linear programming. To increase expressiveness, we then show how we can reduce constraint judgements on monotonic monovariate polynomial functions to linear constraint judgements.\\

% After shortly introducing the work by Baillot and Ghyselen and the variant of the pi-calculus they consider, we focus on so-called \textit{constraint judgements} that are particularly relevant for their type system. We give different interpretations of the judgements to give both a formal and an intuitive understanding. Judgements are first limited to linear judgements and are verified by reducing the problem to an integer programming problem. To increase expressiveness, we then show how we can reduce monotonic monovariate polynomial constraints to linear constraints.\\

% Constructing a type checker for the type system by Baillot and Ghyselen introduces a number of challenges. One such challenge is checking for the existence of a specific substitution needed during type checking of outputs on \textit{servers}. We prove that this is NP-complete by reducing it to the NP-complete 3-SAT problem. To get around this, we limit ourselves to type checking processes with expressions of a particular form and introduce a function \textit{instantiate} that uses a greedy strategy to find substitutions.\\

% Another problem encountered during construction of the type checker is that of soundness of its type rules. More specifically, if we are not careful, reduction of processes checked by the type checker may no longer type check and the subject reduction property is therefore lost. We solve this problem by introducing the idea of \textit{combined complexities} consisting of a set of intersecting parametric complexity bounds. We also define the accompanying function \textit{basis} that keeps a combined complexity as a minimal set. Finally, we introduce algorithmic type rules for the type checker and prove them sound, including subject reduction. We show how an example process corresponding to the sequential merge function can be type checked using our type checker.\\

Our type inference algorithm is constraint-based, and as such generates constraints that enforce premises of the type rules are satisfied, based on a provided process. Once these constraints have been generated, they are reduced to simpler constraints that may be checked using an off-the-shelf SMT solver. As many of the premises in the type rules are universally quantified constraint judgements, our generated constraints contain both existential quantifiers over unknown coefficients and universal quantifiers over index variables. Such constraints are often not solvable using SMT solvers. As such, we perform a number of over-approximations during constraint reduction that put some limitations on which processes we may infer bounds on the parallel complexity for. We implement type inference in Haskell using the Z3 SMT solver. We find that we can bound the parallel complexity of some linear time and many constant time servers in reasonable time.\\

In conclusion, we have implemented type checking and type inference for the type system by Baillot and Ghyselen, and find that we can type check some polynomial and linear time primitive recursive processes, as well as infer precise bounds on the parallel complexity of some linear time processes and many constant time processes. To increase the expressiveness of our type inference algorithm, future work may include relaxing the over-approximations made during constraint reduction, such that they more closely reflect the original constraints, while still being satisfiable in reasonable time. Furthermore, it may be interesting to see how our type inference algorithm extends to \textit{usage} types, as in the type system by Baillot et al. \cite{BaillotEtAl2021}, which is a generalization of the type system by Baillot and Ghyselen that increases the expressiveness and precision.
\maketitle

\begin{abstract}
    Type systems using sized types have been studied extensively in the context of complexity analysis of functional and parallel programs to formally express, verify and infer complexity bounds on programs. Recent contributions have extended this study to message-passing processes using behavioral types to bound channel synchronizations, providing a sound framework for parallel complexity of pi-calculus processes. We explore the challenges of implementing this work, and present a type checker and a type inference algorithm. Our type checker can verify complexity analyses of some polynomial- and linear time primitive recursive functions, encoded as replicated channel inputs (servers), by using integer programming to bound channel synchronizations. Comparison of parametric complexities is a partial order, so to bound parallel complexities, we introduce the notion of combined complexity: A set of intersecting parametric complexity bounds. Our type inference algorithm first generates a constraint satisfaction problem on sized input/output types that we reduce to a set of polynomial inequality constraints, a solution to which provides a parametric bound on the parallel complexity of a server. We show how our constraint satisfaction problems can be over-approximated and provide a Haskell implementation using the Z3 SMT solver that can provide reasonable bounds on the parallel complexity of some linear time servers.
    %
    % Complexity analysis has long been a central part of algorithm design and is paramount to determine the efficiency of algorithms and other systems. Existing type systems formalize rules determining the complexity of both functional and parallel programs. We extend the work by Baillot and Ghyselen \cite{BaillotGhyselen2021} and implement both type checking and type inference for their type system for parallel complexity with message-passing processes in the pi-calculus. We show how we determine if a linear constraint is always satisfied given some other constraints using linear programming, which is an important step during type checking. To increase expressiveness, we allow certain polynomial constraints that we can reduce to linear ones. We also introduce algorithmic type rules, where we introduce combined complexities to preserve the subject reduction property. Soundness of the algorithmic type rules is proved. We use a constraint-based system to infer types representing linear bounds. This system is two-phased and sets up constraints based on linear templates that are later reduced into simple constraints on coefficients from the templates. The constraints on coefficients are then checked for satisfiability by an off-the-shelf SMT solver, and types are inferred based on an interpretation of the constraints. We implement type inference in Haskell using Z3 as the SMT solver. We find that we can type check many linear processes and infer bounds for constant and linear time processes in a reasonable amount of time.
\end{abstract}
\clearpage

\text{ }
\vspace{50em}
\section*{Acknowledgments}
We give special thanks to our supervisor Hans Hüttel for his excellent supervision and perspectives. We also thank the people at Kobayashi Laboratory for their hospitality during our stay at the University of Tokyo, and especially Naoki Kobayashi for his excellent guidance and passionate input on type inference.
\clearpage

\setcounter{tocdepth}{1}
\tableofcontents*
\clearpage


\section{Introduction}\label{ch:introduction}

Static analysis of computational complexity has long been a central part of algorithm design and computer science as a whole. One way of performing such a static analysis on a program is by means of type-based techniques. Traditionally, soundness properties have pertained to the absence of certain run-time errors, i.e. \textit{well-typed programs do not go wrong} \cite{Milner1978} but  with the advent of behavioral type disciplines, soundness properties that for instance ensure bounds on the resource use of programs have been proved. % If combined with an implementation, i.e. an algorithm that specifies how the rules of a type system are to be used, we may either verify that a specified type based complexity analysis of a program is correct (type-checking), or automate the complexity analysis and infer a complexity bound for a program, if it can be bounded by the type system (type inference).

Research on type systems for computational complexity has originally focused on sequential programs, using a notion of types that can express sizes of terms in a program, referred to as sized types, which have been both formalized and implemented \cite{HofmannAndJost2003,HofmannAndHoffmann2010,HoffmannEtAl2012,LagoGaboardi2012,AvanziniLago2017}. However, there is particularly interest in static complexity analysis of parallel and concurrent computation, following the trend for programs to increase in size, as distributed systems scale better. Moreover, parallel and especially concurrent computation is significantly more difficult to analyze, and so the work on type systems for static complexity analysis has been extended to parallel computation \cite{HoffmannShao2015}, and more recently to the more intricate domain of message-passing processes, using behavioral type disciplines to bound message-passing \cite{BaillotGhyselen2021,BaillotEtAl2021}.

Baillot and Ghyselen \cite{BaillotGhyselen2021} introduce a type system for parallel computational complexity of $\pi$-calculus processes, extended with naturals and pattern matching as a computational model, combining sized types and input/output types to bound synchronizations on channels, and thereby bound the parallel time complexity of a process. Baillot et al. \cite{BaillotEtAl2021} generalize the type system using the more expressive behavioral type discipline usage types \cite{Kobayashi1998,KobayashiEtAl2000}. However, these type systems are quite abstract and build on a notion of sized types that has yet to be implemented in the context of message-passing, and so neither type checking nor type inference has until now been realized for either type system.

In this thesis, we explore the challenges of implementing type checking and type inference for the type system by Baillot and Ghyselen \cite{BaillotGhyselen2021}. An important part of this type system is the concept of indices. That is, arithmetic expressions that may contain index variables that represent unknown sizes, thereby enabling a notion of size polymorphism. Indices appear in sized types, to for instance express the timesteps at which a channel must synchronize, which may depend on the size of a value received on a replicated input. To compute an upper bound on the parallel complexity, a partial order on channel synchronizations is required, represented as index comparisons that we refer to as constraint judgements and read as: \textit{Provided a set of constraints on valuations of index variables, is one index always less than or equal to another?} Many of the challenges that arise for both type checking and type inference are related to either verification or satisfaction of such judgements.%Finally, the type system relies on on a special form of minus that can never give negative results, which breaks many useful mathematical properties.\\

We implement a type checker for the type system by Baillot and Ghyselen. This effort is two-fold: We define algorithmic type rules and show how constraint judgements on linear indices can be verified using integer programming or alternatively be over-approximated as linear programs. The type system makes heavy use of subtyping, which we partially account for using \textit{combined complexities} that are effectively sets of indices with a number of associated functions that enable us to discard indices we can guarantee to be bounded by other indices. Combined complexities have the advantage that we can defer finding a single index representing a least upper bound until a later time, and we can in fact show that the combined complexity of a closed process can always be reduced to a singleton, i.e. a singular complexity bound.

We prove that our type checker is sound with regards to time complexity, and to this effort we prove a subject reduction property. Our soundness results guarantee that the bounds assigned to well-typed processes by our type checker are indeed upper bounds on the parallel complexity. To increase the expressiveness of the type checker, we also show how constraints on monotonic univariate polynomial indices can be reduced to linear constraints.

We also define a type inference algorithm for the type system by Baillot and Ghyselen. We take a constraint based approach akin to that of \cite{HofmannAndJost2003,HofmannAndHoffmann2010,HoffmannEtAl2012,KobayashiEtAl2000,Kobayashi2005,Lhoussaine2004}, where unknown indices are represented by \textit{templates}: linear functions over a set of known index variables with unknown coefficients represented by coefficient variables. Inspired by  Kobayashi et al. \cite{KobayashiEtAl2000}, we first infer simple types, which are then used to infer a constraint satisfaction problem on use-capabilities and subtyping which we then reduce to constraints of the form $\exists\alpha_1,\dots,\alpha_n.\forall i_1,\dots,i_m.C_1\land\cdots\land C_k \implies I \leq J$ where $\alpha_1,\dots,\alpha_n$ are coefficient variables, $i_1,\dots,i_m$ are index variables, $C_1,\dots,C_k$ are inequality constraints on indices and $I$ and $J$ are indices.

We provide a Haskell implementation of our type inference algorithm using the Z3 SMT solver \cite{Z3}. We naively eliminate universal quantifiers by over-approximating our constraints using coefficient-wise inequality constraints. We account for antecedents $C_1,\dots,C_k$ by substitution. For instance, if we can deduce that coefficient $c$ is positive in the constraint $\exists\alpha_1,\dots,\alpha_n.\forall i_1,\dots,i_j,\dots,i_m.K \leq L + ci_j \land C_1 \land \cdots \land C_k \implies I \leq J$, then we can simulate the antecedent by substituting $\frac{K-L}{c} + i_j$ for $i_j$, i.e. $\exists\alpha_1,\dots,\alpha_n.\forall i_1,\dots,i_j,\dots,i_m.C_1\{\frac{K-L}{c} + i_j/i_j\} \land \cdots \land C_k\{\frac{K-L}{c} + i_j/i_j\} \implies I\{\frac{K-L}{c} + i_j/i_j\} \leq J\{\frac{K-L}{c} + i_j/i_j\}$. Using these over-approximations, our implementation is able to infer precise bounds on several processes containing replicated inputs with linear time complexity.


%%% Local Variables:
%%% mode: latex
%%% TeX-master: "../esop2023"
%%% End:

\chapter{The $\pi$-calculus}\label{ch:picalc}

In this chapter, we introduce an extended $\pi$-calculus definition, and introduce a cost model for time that uses process annotations to represent incurred reduction costs. Finally, we formalize the parallel complexity of a process.

%%

\section{Syntax and semantics}
We consider an asynchronous polyadic $\pi$-calculus extended with naturals as algebraic terms and a pattern matching construct that enables deconstruction of such terms. The languages of processes and expressions are defined by the syntax 
%
\begin{align*}
    P,Q \text{ (processes) } ::=&\; \nil \mid \left(\parcomp{P}{Q} \right) \mid \inputch{a}{\widetilde{v}}{}{P} \mid \asyncoutputch{a}{\widetilde{e}}{} \mid\; \bang{\inputch{a}{\widetilde{v}}{}{P}} \mid \newvar{a}{P} \mid \tick{P} \mid \\
    &\; \match{e}{P}{x}{Q}\\
    e \text{ (expressions) } ::=&\; 0 \mid \succc{e} \mid v
\end{align*}
%
where we assume a countably infinite set of names $\textbf{Var}$, such that $a,b,c\in\textbf{Var}$ represent channels, $x,y,z\in\textbf{Var}$ are bound to algebraic terms and the meta-variable $v\in\textbf{Var}$ may be bound to any expression. As usual, we have inaction $\mathbf{0}$ representing a terminated process, the parallel composition of two processes $P \mid Q$ and restrictions $\newvar{a}{P}$. 
%Here, $T$ is a type annotation the implications of which we shall defer until Section \ref{sec:typesandsubs}.
For polyadic inputs and outputs $\inputch{a}{\widetilde{v}}{}{P}$ and $\asyncoutputch{a}{\widetilde{e}}{}$, we use the notation $\widetilde{v}$ and $\widetilde{e}$, respectively, to denote sequences of names $v_1,\dots,v_n$ and expressions $e_1,\dots,e_n$, and we write $P[\widetilde{v}\mapsto\widetilde{e}]$ to denote the substitution $P[v_1\mapsto e_1,\dots,v_n\mapsto e_n]$ for names in process $P$. Similarly, for nested restrictions $\newvar{a_1}{\cdots \newvar{a_n}{P}}$ we may write $\newvar{\widetilde{a}}{P}$. For technical convenience, we only enable replication on inputs, i.e. $!\inputch{a}{\widetilde{v}}{}{P}$, such that we can more easily determine which channels induce recursive behavior. Note that any replicated process $!P$ can be simulated using a replicated input $!\inputch{a}{}{}{(P \mid \asyncoutputch{a}{}{})} \mid \asyncoutputch{a}{}{})$.\\ 

Naturals are represented by a zero constructor $0$, representing the smallest natural number, and a successor constructor $\succc{e}$ that represents the successor of the natural number $e$. Thus, algebraic terms have a clearly distinguishable base case. We use this for the pattern matching constructor that deconstructs such terms, by branching based on the shapes of the expressions. This will be useful later, as this enables us to analyze termination conditions for some processes with recursive behavior. The \texttt{tick} constructor represents a cost in time complexity of one. We provide a detailed account of this constructor in Section \ref{sec:parcomplex}.\\

In Definition \ref{def:structcong1}, we introduce the usual structural congruence relation \cite{Milner1993}. We omit rules for replication, as these will be covered by the reduction relation. The congruence relation essentially introduces associative and commutative properties to parallel compositions and enables widening or narrowing of the scopes of restrictions, thereby simplifying the semantics.
%
\begin{defi}[Structural congruence]
We define structural congruence $\equiv$ as the least congruence relation that satisfies the rules
%
\begin{align*}
    &\kern6em\runa{SC-nil}\;\;\parcomp{P}{\nil} \equiv P\kern2em \runa{SC-commu}\;\; \parcomp{P}{Q} \equiv \parcomp{Q}{P}\\ &\kern7em\runa{SC-assoc}\;\; \parcomp{P}{\left(\parcomp{Q}{R}\right)} \equiv \parcomp{\left(\parcomp{P}{Q}\right)}{R}\\
    %
    &\kern2em\runa{SC-scope}\;\;\newvar{a}{\left(\parcomp{P}{Q}\right)} \equiv \parcomp{\newvar{a}{P}}{Q}\;\; \text{ if } a \text{ is not free in } Q\\
    %
    &\kern2.5em\runa{SC-par}\;\;\infrule{P \equiv P'}{\parcomp{P}{Q} \equiv \parcomp{P'}{Q}} \kern2em
    \runa{SC-res}\;\;\infrule{P \equiv Q}{\newvar{a}{P} \equiv \newvar{a}{Q}}
\end{align*}
\label{def:structcong1}
\end{defi}
%
We extend the usual definition of structural congruence with rules that enable us to group restrictions. This allows us to introduce a normal form for processes that simplifies the definition of parallel complexity or span, and consequently makes it easier to prove various properties for typed $\pi$-calculi. We formalize the normal form in Definition \ref{def:canonform1}, referring to it as the canonical form. Essentially, we can view a process in canonical form as a sequence of bound names and a multiset of guarded processes. In Lemma \ref{lemma:cannform}, we prove that an arbitrary process is structurally congruent to a process in this form.
%
\begin{defi}[Canonical form]
We say that a process $P$ is in canonical form if is has the shape $\newvar{\widetilde{a}}{\left(\parcomp{G_1}{\parcomp{G_2}{\parcomp{\dots}{G_n}}}\right)}$
 where $G_1,G_2,\dots,G_n$ are referred to as guarded processes which may be of any of the forms
\begin{equation*}
    G ::=\; \bang{\inputch{a}{\widetilde{y}}{}{P}} \mid \inputch{a}{\widetilde{v}}{}{P} \mid \asyncoutputch{a}{\widetilde{e}}{} \mid \match{e}{P}{x}{Q} \mid \tick P
\end{equation*}
If $n = 0$ then the canonical form is $\newvar{\widetilde{a}}{\nil}$.
\label{def:canonform1}
\end{defi}
%
\begin{lemma}[Existence of a canonical form]\label{lemma:cannform}
Let $P$ be a process. Then there exists a process $Q$ in canonical form such that $P \equiv Q$.
\begin{proof}
Suppose by $\alpha$-renaming that all names are unique, then by using rule $\runa{SC-res}$ from left to right and $\runa{SC-scope}$ from right to left, we can widen the scope of all unguarded restrictions, such that all unguarded restrictions are outmost. Then using rule $\runa{SC-res}$ and $\runa{SC-zero}$ from left to right, we remove all unguarded inactions. If the whole process is unguarded, one inaction will remain and we have the canonical form $\newvar{\widetilde{a} : \widetilde{T}}{\mathbf{0}}$.
\end{proof}
\end{lemma}
%
In Table \ref{tab:redurules}, we define the reduction relation $\longrightarrow$ for $\pi$-calculus processes, such that $P \longrightarrow Q$ denotes that process $P$ reduces to $Q$ in one reduction step \cite{Milner1993}. We have the usual rules for the asynchronous polyadic $\pi$-calculus, enriched with rules for replicated inputs, pattern matching and temporal reductions. The former is covered by rule $\runa{R-rep}$ that synchronizes a replicated input with an output, preserving the replicated input for subsequent synchronizations. Rule $\runa{R-zero}$ considers the base case for pattern matching on naturals. For pattern matches on successors, we substitute the \textit{deconstructed} terms for variables bound in the patterns of the pattern match constructors. Finally, rule $\runa{R-tick}$ reduces a tick prefix, representing a cost of one in time complexity.
%
\begin{table*}[ht]
    \centering
    \begin{framed}\vspace{-1em}\begin{tabular}{l}
        \kern0em\runa{R-rep}\;\;\infrule{}{\parcomp{\;\bang{\inputch{a}{\widetilde{v}}{}{P}}}{\asyncoutputch{a}{\widetilde{e}}{}} \longrightarrow \parcomp{\;\bang{\inputch{a}{\widetilde{v}}{}{P}}}{\subst{P}{\widetilde{v}\mapsto \widetilde{e}}}}
        %
        \kern7em\runa{R-comm}\;\;\infrule{}{\parcomp{\inputch{a}{\widetilde{v}}{}{P}}{\asyncoutputch{a}{\widetilde{e}}{}} \longrightarrow \subst{P}{\widetilde{v}\mapsto \widetilde{e}}}\\[-1em]
        %
        \kern0em\runa{R-zero}\;\;\infrule{}{\match{0}{P}{x}{Q} \longrightarrow P}
        %
        \kern7em\runa{R-par}\;\;\infrule{P \longrightarrow Q}{\parcomp{P}{R} \longrightarrow \parcomp{Q}{R}}
        \\[-1em]
        %
        \kern0em\runa{R-succ}\;\;\infrule{}{\match{\succc{e}}{P}{x}{Q} \longrightarrow \subst{Q}{x \mapsto e}} \kern9.5em \runa{R-tick}\kern-1.5em\infrule{}{\tick P \longrightarrow P} \\[-1em]
        %
        %\kern0em\runa{R-empty}\;\;\infrule{}{\texttt{match}\; []\; \{ [] \mapsto P; x :: y \mapsto Q \} \longrightarrow P}
        %
        \kern-0em \runa{R-res}\kern-1em\infrule{P \longrightarrow Q}{\newvar{a}{P} \longrightarrow \newvar{a}{Q}}
        %
        %\kern7em\runa{R-cons}\;\;\infrule{}{\texttt{match}\; e :: e'\; \{ [] \mapsto P; x :: y \mapsto Q \} \longrightarrow Q[x \mapsto e,y \mapsto e']}\\[-1em]
        %
        %\kern3em\runa{R-par}\;\;\infrule{P \longrightarrow Q}{\parcomp{P}{R} \longrightarrow \parcomp{Q}{R}} \kern-0em \runa{R-res}\;\;\infrule{P \longrightarrow Q}{\newvar{a}{P} \longrightarrow \newvar{a}{Q}}\\
        %
        \kern2em\runa{R-struct}\infrule{P \equiv P'\quad P' \longrightarrow Q'\quad Q' \equiv Q}{P \longrightarrow Q} 
    \end{tabular}\end{framed}
    \smallskip
    \caption{The reduction rules defining $\longrightarrow$.}
    \label{tab:redurules}
\end{table*}
%
%
\section{Parallel complexity}\label{sec:parcomplex}
There are several tried approaches to modeling time complexity for the $\pi$-calculus. One way is to incur a cost for every axiomatic reduction, in our case whenever a channel synchronizes or a pattern match branches. A widely used alternative, including the one chosen by Baillot and Ghyselen, is to introduce an explicit process prefix constructor $\tick P$ denoting that the continuation $P$ is preceded by a cost in time complexity \cite{BaillotGhyselen2021,BaillotEtAl2021,DasEtAl2018}. This approach is more flexible, as we can simulate many different cost models, based on placement patterns of tick prefixes. We now present some of the properties of the tick constructor with respect to parallel complexity. We are interested in maximizing the parallelism, and to do so we must reduce ticks in parallel. Consider the process
\begin{align*}
    \overbrace{\tick\nil \mid \tick\nil \mid \cdots \mid \tick\nil}^{n}
\end{align*}
The sequential complexity or work is $n$, whereas the parallel complexity or span is $1$, as we can reduce the $n$ ticks in parallel. To keep track of the span during reduction, we introduce integer annotations to processes, such that process $P$ can be represented as $m : P$ where $m$ represents the time already incurred. We refer to such processes as \textit{annotated processes}, and we enrich the definition of structural congruence with four additional rules that include time annotations in Definition \ref{def:structuralcongruenceanno}. Intuitively, this means that process annotations can be moved outward and may be summed. Any process is also structurally congruent to one with an extra annotation of $0$.

\begin{defi}\label{def:structuralcongruenceanno}
    We enrich the definition of structural congruence with the four rules
    %
    \begin{align*}
        &\runa{SC-dis}\;\; m : (P \mid Q) \equiv (m : P) \mid (m : Q) \kern3em \runa{SC-ares}\;\; m : \newvar{a}{P} \equiv \newvar{a}{(m : P)} \\
        &\kern4em\runa{SC-sum}\;\; m : (n : P) \equiv (m + n) : P \kern3em \runa{SC-zero}\;\; 0 : P \equiv P
    \end{align*}
\end{defi}

Annotations on processes introduce another notion of canonical process that can be seen in Definition \ref{def:annotatedcanonical}. That is, an annotated process in canonical form is a sequence of nested restrictions on a parallel composition of annotated guarded processes.

% canonical form for annotated processes
\begin{defi}[Annotated canonical form]\label{def:annotatedcanonical}
An annotated process is in canonical form if it is of the form
\begin{align*}
    \newvar{\widetilde{a}}{(n_1 : G_1 \mid \cdots \mid n_m : G_m)}
\end{align*}
where $G_1, \dots, G_m$ are guarded annotated processes. If $m=0$, its canonical form is $\newvar{\widetilde{a}}{(0 : \nil)}$.
\end{defi}

\begin{lemma}[Existence of an annotated canonical form]\label{lemma:anncannform}
Let $P$ be an annotated process. Then there exists an annotated process $Q$ in annotated canonical form such that $P \equiv Q$.
\begin{proof}
Suppose by $\alpha$-renaming that all names are unique, then by using rule $\runa{SC-res}$, $\runa{SC-ares}$, $\runa{SC-dis}$ and $\runa{SC-sum}$ from left to right, and $\runa{SC-scope}$ from right to left, we can widen the scope of all unguarded restrictions, such that all unguarded restrictions are outmost, and all non-guarded annotations are prefixes to guarded processes. Then using rule $\runa{SC-res}$ and $\runa{SC-zero}$ from left to right, we remove all unguarded inactions. For all remaining guarded processes that are not prefixed with an annotation, we use $\runa{SC-res}$, $\runa{SC-par}$ and $\runa{SC-zero}$ to introduce prefixing $0$-annotations. If the whole process is unguarded, one inaction will remain and we have the canonical form $\newvar{\widetilde{a} : \widetilde{T}}{\mathbf{0}}$.
\end{proof}
\end{lemma}

With annotated processes, we can define the parallel reduction relation $\Longrightarrow$ in Table \ref{tab:redurulesanno}. Most notably, we can see that the tick constructor reduces to an annotation of $1$, and during communication we choose the maximum annotation amongst the two endpoints.\\

\begin{table*}[ht]
    \centering
    \begin{framed}\begin{tabular}{l}
        \vspace{-1.0em}
        \kern0em\runa{PR-rep}\;\infrule{}{\parcomp{(n :\;\bang{\inputch{a}{\widetilde{v}}{}{P}})}{(m : \asyncoutputch{a}{\widetilde{e}}{})} \Longrightarrow \parcomp{(n :\;\bang{\inputch{a}{\widetilde{v}}{}{P}})}{(\text{max}(n,m) : \subst{P}{\widetilde{v}\mapsto \widetilde{e}}})}\\ 
        %
        \kern1.5em\runa{PR-comm}\;\infrule{}{\parcomp{(n : \inputch{a}{\widetilde{v}}{}{P})}{(m :\asyncoutputch{a}{\widetilde{e}}{})} \Longrightarrow \text{max}(n,m) : \subst{P}{\widetilde{v}\mapsto \widetilde{e}}}        \vspace{-1em}\\
        %
        \kern-1em\runa{PR-tick}\;\infrule{}{\tick{P} \Longrightarrow 1 : P}
        %
        %\vspace{-1.5em}
        \kern-2em\runa{PR-zero}\;\infrule{}{\match{0}{P}{x}{Q} \Longrightarrow P}\vspace{-1em}\\
        %
        \kern4em\runa{PR-succ}\;\infrule{}{\match{\succc{e}}{P}{x}{Q} \Longrightarrow \subst{Q}{x \mapsto e}}\vspace{-1.0em}\\
        %
        \kern1em\runa{PR-par}\;\infrule{P \Longrightarrow Q}{\parcomp{P}{R} \Longrightarrow \parcomp{Q}{R}} \kern0em \runa{PR-res}\;\infrule{P \Longrightarrow Q}{\newvar{a}{P} \Longrightarrow \newvar{a}{Q}}\\
        \kern1em\runa{PR-annot}\;\infrule{P \Longrightarrow Q}{n : P \Longrightarrow n : Q}
        %
        \kern-1em\runa{PR-struct}\;\infrule{P \equiv P'\quad P' \Longrightarrow Q'\quad Q' \equiv Q}{P \Longrightarrow Q}
    \end{tabular}\end{framed}
    \smallskip
    \caption{The reduction rules defining $\Rightarrow$.}
    \label{tab:redurulesanno}
\end{table*}

In Definition \ref{def:bglcsim}, we define the local parallel complexity $C_\ell(P)$ of a process $P$. Intuitively, the local complexity is the maximal integer annotation in the canonical form of $P$.

\begin{defi}[Local complexity]\label{def:bglcsim}
    We define the local parallel complexity $\mathcal{C}_\ell(P)$ of a process $P$ by the following rules
    \begin{align*}
        \mathcal{C}_\ell(n:P) &= n + \mathcal{C}_\ell(P)\quad\quad
        \mathcal{C}_\ell(P \mid Q) = \text{max}(\mathcal{C}_\ell(P), \mathcal{C}_\ell(Q))\\
        \mathcal{C}_\ell(\newvar{a}{P}) &= \mathcal{C}_\ell(P)\quad\quad
        \mathcal{C}_\ell(G) = 0 \text{ if } G \text{ is a guarded process}
    \end{align*}
\end{defi}

We now formalize the parallel complexity or span of a process in Definition \ref{def:spancomp}. To account for the non-determinism of the $\pi$-calculus, the parallel complexity of a process is defined as the maximal integer annotation in any reduction sequence. To see why this is necessary, consider the process
%
\begin{align*}
    \inputch{a}{}{}{\tick} \mid \inputch{a}{}{}{\tick\asyncoutputch{a}{}{}} \mid \asyncoutputch{a}{}{}
\end{align*}
%
We have two possible reduction sequences with different integer annotations. That is, if we were to reduce the left-most input first, we have a single time reduction, as the second tick will be guarded. If we instead reduce the second input first, then we can synchronize of channel $a$ again after one time reduction, thus yielding two time reductions.
%
\begin{defi}[Parallel complexity]\label{def:spancomp}
We define the parallel complexity (or span) $\mathcal{C}_{\mathcal{P}}(P)$ of process $P$ as the maximal local complexity of any reduction sequence from $P$.
\begin{align*}
    \mathcal{C}_{\mathcal{P}}(P) = \text{max}\{n \mid P \Longrightarrow^* Q \land \mathcal{C}_\ell(Q) = n\}
\end{align*}
where $\Longrightarrow^*$ is the reflexive and transitive closure of $\Longrightarrow$.
\end{defi}


% As the reduction relation $\longrightarrow$ is sequential, we define a parallel reduction relation $\Longrightarrow^{-1}\subseteq\longrightarrow^*$ that reduces all outer-most ticks in parallel. Here, $\longrightarrow^*$ is the transitive and reflexive closure of $\longrightarrow$, and $P \Longrightarrow^{-1} Q$ is called a \textit{time reduction}. We formalize the relation in Definition \ref{def:timereduction}.
% %
% %
% \begin{defi}[Time reduction]\label{def:timereduction}
% Let $P$ be a process with canonical form $P \equiv \newvar{\widetilde{a}}{(G_1\mid\dots\mid G_n \mid \tick R_1 \mid\dots\mid\tick R_m)}$ such that $G_1\dots G_n$ are not guarded by ticks. We refer to the reduction sequence from $P$ to $Q\equiv \newvar{\widetilde{a}}{(G_1\mid\dots\mid G_n \mid R_1 \mid\dots\mid R_m)}$ $P\longrightarrow^* Q$ as a \textit{time reduction}, denoted $P \Longrightarrow^{-1} Q$. We say that a time reduction $P\Longrightarrow^{-1} Q$ is \textit{productive} if $P\neq Q$, i.e. $P$ is not invariant to $\Longrightarrow^{-1}$.
% \end{defi}
% %
% Another concern with respect to maximal parallelism is the reduction order. Consider the process
% %
% \begin{align*}
%     \tick \nil \mid \inputch{a}{}{}{\tick\nil} \mid \asyncoutputch{a}{}{}
% \end{align*}
% %
% We have two valid reduction sequences, we can either reduce the left-most tick first yielding a reduction sequence with two time reductions, or synchronize on channel $a$ first, such that the second tick is unguarded, thereby enabling a single time reduction. Thus, to maximize the parallelism, we must prioritize reductions on channels and pattern matches, to maximize the number of ticks in parallel, before performing a time reduction. Therefore, we introduce a reduction relation $\leadsto\subseteq\longrightarrow$ that represents \textit{non-temporal reductions}, i.e. reductions that do not use rule $\runa{R-tick}$. We formalize this relation in Definition \ref{def:nontempreduction}.
% %
% \begin{defi}[Non-temporal reduction]\label{def:nontempreduction}
% We define a relation $\leadsto\subseteq\longrightarrow$ such that $P\leadsto Q$ if $P \longrightarrow Q$ without using $\runa{R-tick}$. If there exists $Q$ such that $P\leadsto Q$, we write $P\!\!\leadsto$, and conversely, $P\!\not\!\leadsto$ if no such $Q$ exists. We write $P \leadsto^* Q$ for the transitive and reflexive closure of $\leadsto$.
% \end{defi}
% %
% Much inspired by a similar reduction strategy in Baillot and Ghyselen \cite{BaillotGhyselen2021}, we introduce the tick-last reduction strategy in Definition \ref{def:ticklaststrat}. To maximize the parallelism of reduction of process $P$, we perform non-temporal reductions until no non-temporal reduction is defined, and then we perform a single time reduction. If a tick was reduced, we proceed to perform non-temporal reduction again. If no tick was reduced, we terminate reduction, as the process then cannot reduce with reduction relation $\longrightarrow$.
% %
% \begin{defi}[Tick-last strategy]\label{def:ticklaststrat}
% We define a reduction strategy for processes called the tick-last strategy. We reduce a process $P$ in two steps
% \begin{enumerate}
%     \item We perform a sequence of non-temporal reductions on $P$ such that $P\leadsto^* Q$  and $Q\!\not\!\leadsto$.
    
%     \item We perform a single time reduction on $Q$ such that if
%     \begin{itemize}
%         \item $Q \Longrightarrow^{-1} Q$ we stop, as $Q$ cannot reduce any further.
%         \item $Q \Longrightarrow^{-1} R$ with $Q \neq R$ we proceed to step one starting from $R$.  
%     \end{itemize}
% \end{enumerate}
% We say that a reduction sequence $P_1 \longrightarrow^* P_{n+1}$ adheres to the tick-last strategy if it is of the form
% \begin{align*}
%     P_1 \leadsto^{*} P_1' \Longrightarrow^{-1} P_2 \leadsto^{*} \cdots \Longrightarrow^{-1} P_n \leadsto^{*} P_n' \Longrightarrow^{-1} P_{n+1}
% \end{align*}
% such that $P_1'\!\not\!\leadsto$, $P_1' \neq P_2$, $P_n'\!\not\!\leadsto$ and $P_n' = P_{n+1}$. Moreover, we write $P_1 \hookrightarrow^n P_{n+1}$ to denote that $P_1$ reduces to $P_{n+1}$ by the tick-last strategy using $n$ productive time reductions.
% \end{defi}
% %
% We now formalize the parallel complexity or span of a process in Definition \ref{def:spancomp}. To account for the non-determinism of the $\pi$-calculus, the parallel complexity of a process is defined as the maximal number of productive time reductions in any reduction sequence that adheres to the tick-last strategy. To see why this is necessary, consider the process
% %
% \begin{align*}
%     \inputch{a}{}{}{\tick} \mid \inputch{a}{}{}{\tick\asyncoutputch{a}{}{}} \mid \asyncoutputch{a}{}{}
% \end{align*}
% %
% We have two reduction sequences that adhere to the tick-last strategy, yet have different numbers of productive time reductions. That is, if we were to reduce the left-most input first, we have a single time reduction, as the second tick will be guarded. If we instead reduce the second input first, then we can synchronize of channel $a$ again after one time reduction, thus yielding two time reductions.
% %
% \begin{defi}[Parallel complexity]\label{def:spancomp}
% Let $P$ be a process. The parallel complexity (or span) of $P$ is given as the maximum number of productive time reductions in any reduction sequence from $P$ that adheres to the tick-last strategy 
% \begin{align*}
%     \mathcal{C}(P) = \text{max}\{n \mid P \hookrightarrow^n Q\}
% \end{align*}
% \end{defi}

%%
\chapter{Type checking sized types for parallel complexity}\label{ch:typecheck}
As mentioned in Chapter \ref{ch:bgts}, Baillot and Ghyselen \cite{BaillotGhyselen2021} bound sizes of algebraic terms and synchronizations on channels using indices, leading to a partial order on indices. For instance, for a process of the form $\inputch{a}{v}{}{\asyncoutputch{b}{v}{}} \mid P$ (assuming synchronizations induce a cost in time complexity of one) we must enforce that the bound on $a$ is strictly smaller than the bound on $b$. Thus, we must impose constraints on the interpretations of indices. Another concern in the typing of the process above is the parallel complexity. Granted separate bounds on the complexities of $\inputch{a}{v}{}{\asyncoutputch{b}{v}{}}$ and $P$, how do we establish a tight bound on their parallel composition? This turns out to be another major challenge, as bounds may be parametric, such that comparison of bounds is a partial order. Finally, for a process of the form $!\inputch{a}{v}{}{P} \mid \asyncoutputch{a}{e}{}$ we must \textit{instantiate} the parametric complexity of $!\inputch{a}{v}{}{P}$ based on the deducible size bounds of $e$, which quickly becomes difficult as indices become more complex.\\ % As the type system is otherwise fairly standard, for instance using input/output types for channels, the challenge in introducing type check is to ensure constraints on indices are not violated.\\
%
%Type inference for the type system introduced in Baillot and Ghyselen \cite{BaillotGhyselen2021} is complicated by similar challenges to that of type checking, such as constraint satisfaction. Another concern with respect to sized types is that we must infer indices. Here, it is relevant to consider existing work on sized type inference, such as Hughes et al. \cite{HughesEtAl1996} and Avanzini and Dal Lago \cite{AvanziniLago2017}. The set of function symbols used to form indices should be be more strictly defined, to make inference tractable. We also must be careful with respect to recursion, predominantly with how primitive recursion can be identified. In this chapter, we address some of these challenges.
%
% The type system for parallel complexity of message-passing processes introduced in Baillot and Ghyselen \cite{BaillotGhyselen2021} uses sized types to express parametric complexity of invoking replicated inputs, and thereby achieve precise bounds on primitively recursive processes. This requires a notion of polymorphism in the message types of replicated inputs. Baillot and Ghyselen introduce size polymorphism by bounding sizes of algebraic terms and synchronizations on channels with algebraic expressions referred to as indices that may contain index variables representing unknown sizes. We may interpret an index with an index valuation that maps its index variables to naturals, such that the index may be evaluated.\\ %
%
% The bounds on sizes and synchronizations lead to a partial order on indices. For instance, for a process of the form $\inputch{a}{v}{}{\asyncoutputch{b}{v}{}} \mid P$ (assuming synchronizations induce a cost in time complexity of one) we must enforce that the bound on $a$ is strictly smaller than the bound on $b$. Thus, we must induce constraints on the interpretations of indices. As the type system is otherwise fairly standard, for instance using input/output types for channels, the challenge in introducing type check is to ensure constraints on indices are not violated.

The purpose of this section is to present a version of the type system by Baillot and Ghyselen that is algorithmic in the sense that its type rules can be easily implemented in a programming language, and so we must address the challenges described above. For the type checker, we assume we are given a set of constraints $\Phi$ on index variables in $\varphi$ and a type environment $\Gamma$. We first present the types of the type system as well as subtyping. Afterwards, we present auxiliary functions and type rules. For the type rules we also present the concept of combined complexities that we use to bound parallel complexities by deferring comparisons of indices when these are not defined. We then prove the soundness of the type checker and show how it can be extended accompanied by examples. Finally, we show how we can verify the constraint judgements that show up in the type rules.

\section{Auxiliary functions}
We first present two functions that will be used in the type rules. As the continuation of a replicated input may be invoked an arbitrary number of times at different time steps, we need to ensure that names used within the continuation are of types that are invariant to time as defined in Definition \ref{def:timeinvariance}, i.e. they may be used at any time step. In Definition \ref{def:readyfunc}, we define a function $\text{ready}(\varphi,\Phi,T)$ that discards use-capabilities from a type to obtain time invariance. For a server type $\forall_I\widetilde{i}.\texttt{serv}^\sigma_K(\widetilde{T)}$, outputs are well-typed whenever $\varphi;\Phi\vDash I \leq 0$, and so for names of such types, we only discard input capabilities, whenever we can guarantee the constraint judgement $\varphi;\Phi\vDash I \leq 0$. We return to how to guarantee constraint judgements in section \ref{sec:verifyinglinearjudgements}.
%
\begin{defi}\label{def:readyfunc}
We inductively define a function \textit{ready} that transforms a type into one that is time invariant.
\begin{align*}
    %\text{ready}(\varphi,\Phi,\epsilon) =&\; \epsilon\\
    %
    \text{ready}(\varphi,\Phi,\natinterval{I}{J}) =&\; \natinterval{I}{J}\\
    %
    \text{ready}(\varphi,\Phi,\forall_I\widetilde{i}.\texttt{serv}^{\sigma}_K(\widetilde{T})) =&\; \left\{ \begin{matrix}
        \forall_I\widetilde{i}.\texttt{serv}^{\sigma \cap \{\texttt{out}\}}_K(\widetilde{T}) & \text{if}\; \varphi;\Phi\vDash I \leq 0\\
        \forall_0\widetilde{i}.\texttt{serv}^{\emptyset}_K(\widetilde{T}) & \text{if}\; \varphi;\Phi\nvDash I \leq 0
    \end{matrix} \right.\\
    %
    %\text{ready}(\varphi,\Phi,\Gamma,a:\oservS{I}{\widetilde{i}}{K}{\widetilde{T}}) =&\; \left\{ \begin{matrix}
    %    \text{ready}(\varphi,\Phi,\Gamma), a:\oservS{I}{\widetilde{i}}{K}{\widetilde{T}} & \text{if}\; %\varphi;\Phi\vDash I \leq 0\\
    %    \text{ready}(\varphi,\Phi,\Gamma) & \text{if}\; \varphi;\Phi\nvDash I \leq 0
    %\end{matrix} \right.\\
    %%
    %\text{ready}(\varphi,\Phi,\Gamma,a:\iservS{I}{\widetilde{i}}{K}{\widetilde{T}}) =&\; %\text{ready}(\varphi,\Phi,\Gamma)\\
    %
    \text{ready}(\varphi,\Phi,\texttt{ch}^\sigma_I(\widetilde{T})) =&\;\texttt{ch}^\emptyset_0(\widetilde{T})%\\
    %
    %\text{ready}(\varphi,\Phi,\Gamma,a:\outchanneltypeS{I}{\widetilde{T}}) =&\; \text{ready}(\varphi,\Phi,\Gamma)\\
    %
    %\text{ready}(\varphi,\Phi,\Gamma,a:\inchanneltypeS{I}{\widetilde{T}}) =&\; \text{ready}(\varphi,\Phi,\Gamma)
\end{align*}
We extend \textit{ready} to type contexts such that for $v\in\texttt{dom}(\Gamma)$ we have that $\text{ready}(\varphi,\Phi,\Gamma)(v)=\text{ready}(\varphi,\Phi,\Gamma(v))$.
\end{defi}

% In Definition \ref{def:joinbase}, we introduce a binary function on base types $\uplus_{\varphi;\Phi}$ that computes a base type that is a subtype of both argument types, if such a base type exists. We do this by selecting the least lower bound and the greatest upper bound amongst the two argument base types as the new size bounds. This function will be useful for typing list expressions, as the elements of a list may be typed with different size bounds that we will need a common subtype of.

% \begin{defi}[Joining base types]\label{def:joinbase}

% \begin{align*}
%     \texttt{Nat}[I,J] \uplus_{\varphi;\Phi} \texttt{Nat}[I',J'] =&\; \left\{
%     \begin{matrix}
%         \texttt{Nat}[I,J] & \varphi;\Phi\vDash I \leq I'\;\text{and};\varphi;\Phi\vDash J' \leq J\\
%         \texttt{Nat}[I',J] & \varphi;\Phi\vDash I' \leq I\;\text{and};\varphi;\Phi\vDash J' \leq J\\
%         \texttt{Nat}[I,J'] & \varphi;\Phi\vDash I \leq I'\;\text{and};\varphi;\Phi\vDash J \leq J'\\
%         \texttt{Nat}[I',J'] & \varphi;\Phi\vDash I' \leq I\;\text{and};\varphi;\Phi\vDash J \leq J'
%     \end{matrix}
%     \right.\\
    
%     \texttt{List}[I,J](\mathcal{B}) \uplus_{\varphi;\Phi} \texttt{List}[I',J'](\mathcal{B}') =&\; \left\{
%     \begin{matrix}
%         \texttt{List}[I,J](\mathcal{B} \uplus_{\varphi;\Phi} \mathcal{B}') & \varphi;\Phi\vDash I \leq I'\;\text{and};\varphi;\Phi\vDash J' \leq J\\
%         \texttt{List}[I',J](\mathcal{B} \uplus_{\varphi;\Phi} \mathcal{B}') & \varphi;\Phi\vDash I' \leq I\;\text{and};\varphi;\Phi\vDash J' \leq J\\
%         \texttt{List}[I,J'](\mathcal{B} \uplus_{\varphi;\Phi} \mathcal{B}') & \varphi;\Phi\vDash I \leq I'\;\text{and};\varphi;\Phi\vDash J \leq J'\\
%         \texttt{List}[I',J'](\mathcal{B} \uplus_{\varphi;\Phi} \mathcal{B}') & \varphi;\Phi\vDash I' \leq I\;\text{and};\varphi;\Phi\vDash J \leq J'
%     \end{matrix}
%     \right.
% \end{align*}
% \end{defi}

% \begin{defi}[Removing servers]
%     Given a type context $\Gamma$, the function \textit{noserv} removes all server types from the context.
%     \begin{align*}
%         \text{noserv}(\emptyset) &= \emptyset\\
%         \text{noserv}(\Gamma, \natinterval{I}{J}) &= \text{noserv}(\Gamma),\natinterval{I}{J}\\
%         \text{noserv}(\Gamma, \chant{I}{\sigma}{\widetilde{T}}) &= \text{noserv}(\Gamma),\chant{I}{\sigma}{\widetilde{T}})\\
%         \text{noserv}(\Gamma, \servt{I}{\widetilde{i}}{\sigma}{K}{\widetilde{T}}) &= \text{noserv}(\Gamma)\\
%     \end{align*}
% \end{defi}


In Definition \ref{def:instantiatef}, we introduce a function $\text{instantiate}(\widetilde{i},\widetilde{T})$ that assigns the index variables in sequence $\widetilde{i}$ to indices in types of the sequence $\widetilde{T}$, by means of a substitution of indices for index variables. Note that the function is only defined for sequences such that the number of index variables equals the number of indices in the types. This function will be useful for outputs on servers, where the parametric types $\widetilde{S}$ of a server type $\forall_I\widetilde{i}.\texttt{serv}^{\{\texttt{out}\}}_K(\widetilde{S)}$ must match the types of expressions $\widetilde{T}$ to be output. More specifically, there must exist a substitution $\{\widetilde{J}/\widetilde{i}\}$ such that $\widetilde{T} \sqsubseteq \widetilde{S}\{\widetilde{J}/\widetilde{i}\}$. We return to this in Section \ref{section:typeruless}.

\begin{defi}[Server invocation]\label{def:instantiatef}
We inductively define a function \textit{instantiate} that constructs a substitution of indices for index variables, provided a sequence of index variables and a sequence of types%. The function is only defined for sequences such that the number of index variables equals the number of indices in the types.
\begin{align*}
    \text{instantiate}(\epsilon,\epsilon) =&\; \{\}\\
    \text{instantiate}((\widetilde{i},\widetilde{j}),(T,\widetilde{S})) =&\; \text{instantiate}(\widetilde{i},T),\text{instantiate}(\widetilde{j},\widetilde{S})\\
    \text{instantiate}((i,j),\texttt{Nat}[I,J]) =&\; \{I/i,J/j\}\\
    %\text{instantiate}((i,j,\widetilde{k}),\texttt{List}[I,J](\mathcal{B})) =&\; \{I/i,J/j\}, \text{instantiate}(\widetilde{k},\mathcal{B})\\
    \text{instantiate}((i,\widetilde{j}),\texttt{ch}_I^\sigma(\widetilde{T})) =&\; \{I/i\},\text{instantiate}(\widetilde{j},\widetilde{T})\\
    \text{instantiate}((i,j,\widetilde{k}),\forall_I\widetilde{l}.\texttt{serv}^\sigma_K(\widetilde{T})) =&\; \{I/i,K/j\},\text{instantiate}(\widetilde{k},\widetilde{T})
\end{align*}
\end{defi}
\section{Algorithmic type rules}\label{section:typeruless}
%We are now ready to introduce type rules for a type checker of the type system in Baillot and Ghyselen \cite{BaillotGhyselen2021}. %TODO 
%
%
%A piecewise complexity $\kappa$ is a set of pairs $(\Phi_i, K_i)$ where $K_i$ is an index describing a complexity that is valid within the feasible region described by the set of constraints $\Phi_i$. As such, a piecewise complexity $\kappa = \{(\Phi_1, K_1), \cdots, (\Phi_n, K_n)\}$ describes a complexity bound within the feasible region $\mathcal{M}_\varphi(\Phi_1) \cup \cdots \cup \mathcal{M}_\varphi(\Phi_n)$ for some $\varphi$ such that $\Phi_1, \cdots, \Phi_n$ use index variables in $\varphi$. In the case where $m$ feasible regions $\mathcal{M}_\varphi(\Phi_{i_1})$, $\mathcal{M}_\varphi(\Phi_{i_{m-1}})$ and $\mathcal{M}_\varphi(\Phi_{i_m})$ intersect, we choose the maximal complexity of the corresponding complexities for any valuation $\rho$ in the intersecting region.\\

When typing a process, we often need to find an index that is an upper bound on two other indices, for which there may be many options. To allow for the type checker to be as precise as possible, we want to find the minimum complexity that is a bound of two other complexities, which will depend on the representation of complexity, and as such, instead of representing complexity bounds using indices, we opt to use sets of indices which we refer to as \textit{combined complexities}. Intuitively, given any point in the space spanned by some index variables, the combined complexity at that point is the maximum of the complexities making up the combined complexity at that point. This is illustrated in Figure \ref{fig:combined_complexity} which shows a combined complexity consisting of three indices. The red dashed line represents the bound on the combined complexity. Representing complexities as sets of indices has the effect of \textit{externalizing} the process of finding bounds of complexities by deferring this until a later time. We will later define the algorithm \textit{basis} that removes superfluous indices of a combined complexity. In Figure \ref{fig:combined_complexity} the index $K$ is superfluous as it never contributes to the bound of the combined complexity.

\begin{figure}
    \centering
    \begin{tikzpicture}
\begin{axis}[
    axis lines = left,
    xlabel = \(i\),
    ylabel = {},
    domain = 0:2.5,
    xtick={\empty},ytick={\empty},
    ymin=0,
    ymax=5.2,
    xmax=2.7,
    restrict y to domain=0:5,
]
    \addplot[thick, color=orange]{x^2} node[above,pos=1] {I};
    \addplot[thick, color=blue]{x+1} node[above,pos=1] {J};
    \addplot[thick, color=green]{ln(x+1)*2} node[above,pos=1] {K};
    \draw [ultra thick, dashed, draw=red] (axis cs:0,1) -- (axis cs:1.62,2.62);
    \addplot[ultra thick, color=green, dashed, color=red, domain=1.62:2.5]{x^2};
\end{axis}
\end{tikzpicture}
    \caption{Combined complexities illustrated. The combined complexity consists of the three indices I, J, K of the single index variable $i$. The dashed red line shows the bound of the combined complexity. $K$ is a superfluous index in the combined complexity as it never contributes to the bound of the combined complexity.}
    \label{fig:combined_complexity}
\end{figure}


\begin{defi}[Combined complexity]\label{def:combinedcomp} 
    We refer to a set $\kappa$ of complexities as a \textit{combined complexity}. We extend constraint judgements to include combined complexities such that
    \begin{enumerate}
        \item $\varphi;\Phi\vDash \kappa \leq \kappa'$ if for all $K \in \kappa$ there exists $K'\in \kappa'$ such that $\varphi;\Phi\vDash K \leq K'$.
        % 
        \item $\varphi;\Phi\vDash \kappa = \kappa'$ if $\varphi;\Phi\vDash \kappa \leq \kappa'$ and $\varphi;\Phi\vDash \kappa' \leq \kappa$.
        \item $\kappa + I = \{K + I \mid K \in \kappa\}$.
        %
        \item $\kappa\{J/i\} = \{ K\{J/i\} \mid K\in\kappa \}$.
    \end{enumerate}
    In the above, we may substitute an index for a combined complexity. In such judgements, the index represents a singleton set. For instance, $\varphi;\Phi\vDash \kappa \leq K$ represents $\varphi;\Phi\vDash \kappa \leq \{K\}$.
    %$\varphi;\Phi \vDash \kappa \bowtie \kappa' \quad\text{ if }\quad \forall K \in \kappa. (\exists K' \in \kappa'. \varphi;\Phi \vDash K \bowtie K')$.
\end{defi}

More specifically, when considering a combined complexity $\kappa$, we are interested in the maximal complexity given some valuation $\rho$, which we find by simply comparing the different values for the complexities within $\kappa$ given $\rho$. Note that the complexity $K \in \kappa$ that is maximal may be different for different valuations. In Definition \ref{def:combinedcomp} we extend the binary relations in $\bowtie$ on indices to combined complexities, such that we can compare two combined complexities such as $\varphi;\Phi \vDash \kappa \bowtie \kappa'$ and a combined complexity and complexity such as $\varphi;\Phi \vDash \kappa \bowtie K$. Definition \ref{def:combinedcompbasis} defines the function \textit{basis} that discards any $K \in \kappa$ that can never be the maximal complexity given some set of constraints $\Phi$ (i.e. the complexities that are bounded by other complexities in the set), such that we can always keep the number of complexities in a combined complexity to a minimum. %Finally, we may also be interested in adding an index onto a combined complexity, and so we define the addition of indices onto combined complexities in Definition \ref{def:combinedcompadd}.
%
\begin{defi}\label{def:combinedcompbasis}
    We define the function \textit{basis} that takes a set of index variables $\varphi$, a set of constraints $\Phi$ and a combined complexity $\kappa$, and returns a new combined complexity without superfluous complexities (The \textit{basis} of $\kappa$)
    \begin{align*}
        \text{basis}(\varphi,\Phi,\kappa) = \bigcap\left\{ \kappa' \subseteq \kappa \mid \forall K\in\kappa.\exists K'\in\kappa'.\varphi;\Phi\vDash K \leq K' \right\}
    \end{align*}
    Moreover, the algorithm below computes the basis
    % \begin{align*}
    %     \text{basis}(\varphi, \kappa) = \{(\Phi, K) \in \kappa \mid \varphi;\Phi \not \vDash K < K' \text{ for all } (\Phi', K') \in \kappa\}
    % \end{align*}
    \begin{align*}
        &\text{basis}(\varphi, \Phi, \kappa) = \text{do}\\[-0.5em]
        &\quad \kappa' \leftarrow \kappa\\[-0.5em]
        &\quad \text{for } K \in \kappa \text{ do}\\[-0.5em]
        &\quad\quad \text{ if } \exists K' \in \kappa' \text{ with } K \not = K' \text{ and } \varphi;\Phi \vDash K \leq K' \text{ then}\\[-0.5em]
        &\quad\quad\quad \kappa' \leftarrow \kappa' \setminus \{K\}\\[-0.5em]
        &\quad \text{return } \kappa'
    \end{align*}
\end{defi}
%
% \begin{defi}[]\label{def:combinedcompadd}
%     We define the the addition of a combined complexity and index as
%     \begin{align*}
%         \kappa + I = \{K + I \mid K \in \kappa\}
%     \end{align*}

% \end{defi}
%
For typing expressions, we use the rules presented in Table \ref{tab:sizedtypedexpressiontypes}, excluding the rule $\runa{BG-sub}$. In Table \ref{tab:sizedprocesstypingrules} we show the type rules for processes. Type judgements are of the form $\varphi;\Phi;\Gamma \vdash P \triangleleft \kappa$ where $\kappa$ denotes the complexity of process $P$. The rule $\runa{S-tick}$ types a \texttt{tick} prefix and incurs a cost of one in time complexity. We advance the time of all types in the context accordingly when typing the continuation. Rule $\runa{S-annot}$ is similar but may incur a cost of $n$. Matches on naturals are typed with rule $\runa{S-nmatch}$. Most notably, we extend the set of known constraints when typing the two continuations. That is, we can deduce constraints on the lower and upper bounds on the size of the expression we match on. For instance, for the zero pattern we can deduce that the lower bound $I$ must be equal to $0$ (or equivalently $I \leq 0$), and for the successor pattern, we can guarantee that the upper bound $J$ must be greater than or equal to $1$. For the complexity of pattern matches and parallel composition, we take advantage of the fact that we represent complexities using combined complexities. As such, we include complexities in both $P$ and $Q$ in the result. To remove redundancy from the set $\kappa \cup \kappa'$, we use the basis function.\\

%
% \begin{table*}[!ht]
%     \begin{framed}\vspace{-1em}\begin{align*}
%         &\kern15em\\[-2em] % Stretch frame
%         &\kern0em\runa{S-nil}\infrule{}{\varphi;\Phi;\Gamma \vdash \withcomplex{\nil}{0}} \kern1em\runa{S-tick}\;\infrule{\varphi;\Phi;\susumesim{\Gamma}{1}\vdash P \triangleleft K}{\varphi;\Phi;\Gamma\vdash \tick P \triangleleft K + 1} \kern3em\runa{S-nu}\;\infrule{\varphi;\Phi;\Gamma,\withtype{a}{T} \vdash \withcomplex{P}{K}}{\varphi;\Phi;\Gamma \vdash \newvar{a: T}{\withcomplex{P}{K}}}\\[-1em]
%         %
%         &\kern-0em\runa{S-nmatch}\;\condinfrule{
%         \begin{matrix}
%             \varphi;\Phi;\Gamma \vdash \withtype{e}{\natinterval{I}{J}}\quad \varphi;\Phi, I \leq 0;\Gamma \vdash \withcomplex{P}{K} \\
%             \varphi;\Phi, J \geq 1;\Gamma, \withtype{x}{\natinterval{I-1}{J-1}} \vdash \withcomplex{Q}{K'}
%         \end{matrix}}{\varphi;\Phi;\Gamma \vdash \withcomplex{\match{e}{P}{x}{Q}}{L}}{\text{where}\quad L = \left\{
% \begin{matrix}
%     K & \text{if}\; \varphi;\Phi\vDash K' \leq K   \\
%     K' & \text{if}\; \varphi;\Phi\vDash K \leq K'  \\
%     K+K' & \text{otherwise}
% \end{matrix}
% \right.}\\[-1em]
%         %
%         %&\kern-0em\runa{S-nmatch-2}\;\infrule{
%         %\begin{matrix}
%         %    \varphi;\Phi;\Gamma \vdash \withtype{e}{\natinterval{I}{J}} \quad \varphi;\Phi\vDash K \leq K' \\
%         %    \varphi;\Phi, I \leq 0;\Gamma \vdash \withcomplex{P}{K} \quad \varphi;\Phi, J \geq 1;\Gamma, \withtype{x}{\natinterval{I-1}{J-1}} \vdash \withcomplex{Q}{K'}
%         %\end{matrix}}{\varphi;\Phi;\Gamma \vdash \withcomplex{\match{e}{P}{x}{Q}}{K'}}\\[-1em]
%         %
%         &\kern-0em\runa{S-lmatch}\;\condinfrule{
%         \begin{matrix}
%             \varphi;\Phi;\Gamma \vdash \withtype{e}{\texttt{List}[I,J](\mathcal{B})} \quad \varphi;\Phi, I \leq 0;\Gamma \vdash \withcomplex{P}{K} \\
%             \varphi;\Phi, J \geq 1;\Gamma, \withtype{x}{\mathcal{B}},y : \texttt{List}[I-1,J-1](\mathcal{B}) \vdash \withcomplex{Q}{K'}
%         \end{matrix}}{\varphi;\Phi;\Gamma \vdash \withcomplex{\texttt{match}\;e\;\{ [] \mapsto P;\; x :: y \mapsto Q \}}{L}}{\text{where}\quad L = \left\{
% \begin{matrix}
%     K & \text{if}\; \varphi;\Phi\vDash K' \leq K   \\
%     K' & \text{if}\; \varphi;\Phi\vDash K \leq K'  \\
%     K+K' & \text{otherwise}
% \end{matrix}
% \right.}\\[-1em]
%         %
%         %&\kern-0em\runa{S-lmatch-2}\;\infrule{
%         %\begin{matrix}
%         %    \varphi;\Phi;\Gamma \vdash \withtype{e}{\texttt{List}[I,J](\mathcal{B})} \quad \varphi;\Phi\vDash K \leq K' \\
%         %    \varphi;\Phi, I \leq 0;\Gamma \vdash \withcomplex{P}{K} \quad \varphi;\Phi, J \geq 1;\Gamma, \withtype{x}{\mathcal{B}},y : \texttt{List}[I-1,J-1](\mathcal{B}) \vdash \withcomplex{Q}{K'}
%       % \end{matrix}}{\varphi;\Phi;\Gamma \vdash \withcomplex{\texttt{match}\;e\;\{ [] \mapsto P;\; x :: y \mapsto Q \}}{K'}}\\[-1em]
%         %
%         &\kern4em\runa{S-par}\;\condinfrule{\varphi;\Phi;\Gamma\vdash P \triangleleft K\quad \varphi;\Phi;\Gamma\vdash Q \triangleleft K'}{\varphi;\Phi;\Gamma\vdash \parcomp{P}{Q} \triangleleft L}{\text{where}\quad L = \left\{
% \begin{matrix}
%     K & \text{if}\; \varphi;\Phi\vDash K' \leq K   \\
%     K' & \text{if}\; \varphi;\Phi\vDash K \leq K'  %\\
%     %K+K' & \text{otherwise}
% \end{matrix}
% \right.}\\[-1em]
%         %
%         %&\kern4em\runa{S-par-2}\;\infrule{\varphi;\Phi;\Gamma\vdash P \triangleleft K\quad \varphi;\Phi;\Gamma\vdash Q \triangleleft K'\quad \varphi;\Phi\vDash K \leq K'}{\varphi;\Phi;\Gamma\vdash \parcomp{P}{Q} \triangleleft K'}\\[-1em]
%         %
%         &\kern-0em\runa{S-iserv}\;\infrule{\texttt{in}\in\sigma\quad \varphi,\widetilde{i};\Phi;\text{ready}(\varphi,\Phi,\susumesim{\Gamma}{I}),a:\forall_0\widetilde{i}.\texttt{serv}^{\sigma\cap\{\texttt{out}\}}_K(\widetilde{T}),\widetilde{v} : \widetilde{T}\vdash P \triangleleft K'\quad \varphi,\widetilde{i};\Phi\vDash K' \leq K}{\varphi;\Phi;\Gamma,a:\forall_I\widetilde{i}.\texttt{serv}^\sigma_K(\widetilde{T})\vdash\; \bang\inputch{a}{\widetilde{v}}{}{P}\triangleleft I}\\[-1em]
%         %
%         &\kern-0em\runa{S-ich}\;\infrule{\texttt{in}\in\sigma\quad \varphi;\Phi;\susumesim{\Gamma}{I},a:\texttt{ch}_0^\sigma(\widetilde{T}),\widetilde{v} : \widetilde{T}\vdash P \triangleleft K}{\varphi;\Phi;\Gamma,a:\texttt{ch}_I^\sigma(\widetilde{T})\vdash \inputch{a}{\widetilde{v}}{}{P}\triangleleft K + I}
%         %
%         \kern8.5em \runa{S-och}\;\infrule{\texttt{out}\in \sigma\quad \varphi;\Phi;\susumesim{\Gamma}{I}\vdash \widetilde{e} : \widetilde{T}\quad \varphi;\Phi\vdash\widetilde{T}\sqsubseteq\widetilde{S}}{\varphi;\Phi;\Gamma,a:\texttt{ch}^{\sigma}_I(\widetilde{S})\vdash \asyncoutputch{a}{\widetilde{e}}{} \triangleleft I}\\[-1em]
%         %
%         &\kern0em\runa{S-oserv}\;\infrule{\texttt{out} \in \sigma \quad \varphi;\Phi;\susumesim{\Gamma}{I}\vdash \widetilde{e} : \widetilde{T}\quad \text{instantiate}(\widetilde{i},\widetilde{T})=\{\widetilde{J}/\widetilde{i}\}\quad  \varphi;\Phi\vdash\widetilde{T}\sqsubseteq\widetilde{S}\{\widetilde{J}/\widetilde{i}\}}{\varphi;\Phi;\Gamma,a:\forall_I\widetilde{i}.\texttt{serv}_K^\sigma(\widetilde{S})\vdash \asyncoutputch{a}{\widetilde{e}}{} \triangleleft K\!\substi{\widetilde{J}}{\widetilde{i}} + I}
%         %
%     \end{align*}\vspace{-1em}\end{framed}
%     \smallskip
%     \caption{Sized typing rules for parallel complexity of processes.}
%     \label{tab:sizedprocesstypingrules}
% \end{table*}

\begin{table*}[!ht]
    \begin{framed}\vspace{-1em}\begin{align*}
        %
        % S-nil
        &\runa{S-nu}\infrule{\varphi;\Phi;\Gamma, a:T \vdash P \triangleleft \kappa}{\varphi;\Phi;\Gamma \vdash \newvar{a:T}{P} \triangleleft \kappa}
        % S-par
        \kern1em\runa{S-par}\infrule{\varphi;\Phi;\Gamma \vdash P \triangleleft \kappa \quad \varphi;\Phi;\Gamma \vdash Q \triangleleft \kappa'}{\varphi;\Phi;\Gamma \vdash P \mid Q \triangleleft \text{basis}(\varphi, \Phi,\kappa \cup \kappa')}\\[-1em]
        %
        &\runa{S-tick}\infrule{\varphi;\Phi;\tforwardsim{\Gamma}{1} \vdash P \triangleleft \kappa}{\varphi;\Phi;\Gamma \vdash \tick P \triangleleft \kappa + 1}\kern2em
        %
        \runa{S-annot}\infrule{\varphi;\Phi;\tforwardsim{\Gamma}{n}\vdash P \triangleleft \kappa}{\varphi;\Phi;\Gamma\vdash n:P \triangleleft \kappa + n}\\[-1em]
        % S-match
        &\runa{S-match}\infrule{
        \begin{matrix}
            \varphi;\Phi;\Gamma \vdash e:\natinterval{I}{J} \quad \varphi;\Phi, I \leq 0;\Gamma \vdash P \triangleleft \kappa\\
            \varphi;\Phi, J \geq 1;\Gamma, x:\natinterval{I-1}{J-1} \vdash Q \triangleleft \kappa'
        \end{matrix}}{\varphi;\Phi;\Gamma \vdash \match{e}{P}{x}{Q} \triangleleft \text{basis}(\varphi, \Phi, \kappa \cup \kappa')}\\[-1em]
        % S-iserv
        &\runa{S-iserv}\infrule{\begin{matrix}
            \texttt{in} \in \sigma\quad \varphi;\Phi;\Gamma\vdash a:\servt{I}{i}{\sigma}{K}{\widetilde{T}}\\
            \varphi, \widetilde{i}; \Phi; \text{ready}(\varphi,\Phi,\tforwardsim{\Gamma}{I}), \widetilde{v} : \widetilde{T} \vdash P \triangleleft \kappa \quad \varphi,\widetilde{i};\Phi\vDash\kappa \leq K
        \end{matrix}}
        {\varphi;\Phi;\Gamma \vdash \;\bang\inputch{a}{\widetilde{v}}{}{P}\triangleleft \{I\}}
        %
        \kern14em\runa{S-nil}\kern-1em\infrule{}{\varphi;\Phi;\Gamma \vdash \nil \triangleleft \{0\}}\kern-3em\text{ }\\[-1em]
        % S-oserv
        &\runa{S-oserv}\infrule{\begin{matrix}
            \texttt{out} \in \sigma\quad \varphi;\Phi;\Gamma\vdash a:\servt{I}{i}{\sigma}{K}{\widetilde{T}}\\
            \varphi; \Phi;\tforwardsim{\Gamma}{I} \vdash \widetilde{e}:\widetilde{S} \quad \text{instantiate}(\widetilde{i}, \widetilde{S}) = \{\widetilde{J}/\widetilde{i}\} \quad \varphi;\Phi \vDash \widetilde{S} \sqsubseteq \widetilde{T}
        \end{matrix}}
        {\varphi;\Phi;\Gamma \vdash \asyncoutputch{a}{\widetilde{e}}{}\triangleleft \{K\{\widetilde{J}/\widetilde{i}\} + I\}}\\[-1em]
        % S-annot
        &\runa{S-ich}\infrule{\begin{matrix}
            \texttt{in} \in \sigma\quad \varphi;\Phi;\Gamma \vdash a:\chant{\sigma}{I}{\widetilde{T}}\\
            \varphi; \Phi; \tforwardsim{\Gamma}{I}, \widetilde{v}:\widetilde{T} \vdash P \triangleleft \kappa
        \end{matrix}}
        {\varphi;\Phi;\Gamma \vdash \inputch{a}{\widetilde{v}}{}{P} \triangleleft \kappa + I}\kern3em
        %
        \runa{S-och}\infrule{\begin{matrix}
            \texttt{out} \in \sigma\quad \varphi;\Phi;\Gamma \vdash a:\chant{\sigma}{I}{\widetilde{T}}\\
            \varphi; \Phi; \tforwardsim{\Gamma}{I} \vdash \widetilde{e}:\widetilde{S} \quad \varphi;\Phi \vDash \widetilde{S} \sqsubseteq \widetilde{T}
        \end{matrix}}
        {\varphi;\Phi;\Gamma \vdash \asyncoutputch{a}{\widetilde{e}}{} \triangleleft \{I\}}\\[-1em]
    \end{align*}\vspace{-1em}\end{framed}
    \smallskip
    \caption{Sized typing rules for parallel complexity of processes.}
    \label{tab:sizedprocesstypingrules}
\end{table*}

%
Rule $\runa{S-iserv}$ types a replicated input on a name $a$, and so $a$ must be bound to a server type with input capability. As the index $I$ in the server type denotes the time steps remaining before the server is ready to synchronize, we advance the time by $I$ units of time complexity when typing the continuation $P$. To ensure that bounds on synchronizations in $\downarrow^{\varphi;\Phi}_I\!\Gamma$ are not violated, we type $P$ under the time invariant part of $\downarrow^{\varphi;\Phi}_I\!\Gamma$, i.e. $\text{ready}(\varphi,\Phi,\downarrow_I\!\Gamma)$. Note that the bound on the span of the replicated input is the bound on the time remaining before the server is ready to synchronize. As the replicated input may be invoked many times, the cost of invoking the server is accounted for in rule $\runa{S-oserv}$ using the complexity bound $K$ in the server type. Therefore, we enforce that $K$ is in fact an upper bound on the span of the continuation $P$.\\

The rule $\runa{S-oserv}$ types outputs on names bound to server types. Here, as stated above, we must account for the cost of invoking a server, and as a replicated input on a server is parametric, we must \textit{instantiate} it based on the types of the expressions we are to output. Recall that in the type rule for outputs on servers from Chapter \ref{ch:bgts}, this is to be done by finding a substitution that satisfies the premise $\widetilde{T} \sqsubseteq \widetilde{S}\{\widetilde{J}/\widetilde{i}\}$. However, this turns out to be a difficult problem, and we can in fact prove it NP-complete for types of polynomial indices even if we disregard subtyping. However, note that it might not be necessary to use the full expressive power of polynomial indices, and so this may not necessarily affect type checking. Nevertheless, we over-approximate finding such a substitution, by using the function $\textit{instantiate}$. That is, we \textit{zip} together the index variables $\widetilde{i}$ with indices in types $\widetilde{T}$. Remark that Baillot and Ghyselen \cite{BaillotGhyselen2021} propose types for inference in their technical report, where the problem is simplified substantially, by forcing naturals to have lower bounds of $0$ and upper bounds with exactly one index variable and a constant. Our approach admits more expressive lower bounds and multiplications, while imposing no direct restrictions on the number of index variables in an index, and is thus more suitable for a type-checker.\\

We now prove the NP-completeness of the smaller problem of checking whether there exists a substitution $\{\widetilde{J}/\widetilde{i}\}$ that satisfies $T = S\{\widetilde{J}/\widetilde{i}\}$ where $T$ and $S$ are types with polynomial indices. The main idea is a reduction proof from the NP-complete 3-SAT problem, i.e. the satisfiability problem of a boolean formula in conjunctive normal form with exactly three literals in each clause \cite{Karp1972}. We first define a translation from a 3-SAT formula to a polynomial index in Definition \ref{def:3satredu}. This is a polynomial time computable reduction, as we simply replace each logical-and with a multiplication, each logical-or with an addition and each negation with a subtraction from 1. In Lemma \ref{lemma:soundtranslation}, we prove that the reduction is faithful with respect to satisfiability of a boolean formula. Finally, in Lemma \ref{lemma:npcompletesubst}, we prove that it is an NP-complete decision problem to verify the existence of a substitution that satisfies $T = S\{\widetilde{J}/\widetilde{i}\}$ for types $T$ and $S$.
%
\begin{defi}[3-SAT reduction]\label{def:3satredu}
We assume a one-to-one mapping $f$ from unknowns to index variables. Let $\phi$ be a 3-SAT formula
\begin{align*}
    \phi = \bigwedge_{i=1}^n \left(\ell_{i1} \lor \ell_{i2} \lor \ell_{i3}\right)% \land \cdots \land (A_n \lor B_n \lor C_n)
\end{align*}
where $\ell_{i1}$, $\ell_{i2}$ and $\ell_{i3}$ are of the forms $x$ or $\neg x$ for some variable $x$. We define a translation of $\phi$ to a polynomial index %$[\![\phi]\!]_{\text{3-SAT}}$
\begin{align*}
    [\![\phi]\!]_{\text{3-SAT}} = \prod_{i=1}^n \left([\![\ell_{i1}]\!]_{\text{3-SAT}} + [\![\ell_{i2}]\!]_{\text{3-SAT}} + [\![\ell_{i3}]\!]_{\text{3-SAT}}\right) %\cdots ([\![A_n]\!]_{\text{3-SAT}} + [\![B_n]\!]_{\text{3-SAT}} + [\![C_n]\!]_{\text{3-SAT}})
\end{align*}
where $[\![x]\!]_{\text{3-SAT}} = f(x)$ and $[\![\neg x]\!]_{\text{3-SAT}} = (1 - f(x))$.
\end{defi}


\begin{lemma}\label{lemma:soundtranslation}
Let $\phi$ be a 3-SAT formula. Then $\phi$ is satisfiable if and only if there exists a substitution $\{\widetilde{n}/\widetilde{i}\}$ such that $1\leq [\![\phi]\!]_{\text{3-SAT}}\{\widetilde{n}/\widetilde{i}\}$.
\begin{proof}
We consider the implications separately
\begin{enumerate}
    \item Assume that $\phi$ is satisfiable. Then there exists a truth assignment $\tau$ such that each clause of $\phi$ is true. Correspondingly, as $[\![\phi]\!]_{\text{3-SAT}}$ is a product of non-negative factors, we for some substitution $\{\widetilde{n}/\widetilde{i}\}$ have that $1 \leq [\![\phi]\!]_{\text{3-SAT}}\{\widetilde{n}/\widetilde{i}\}$ if and only if each factor in the product is positive. We compare the conditions for a clause to be true in $\phi$ to those for a corresponding factor in $[\![\phi]\!]_{\text{3-SAT}}$ to be positive, and show that a substitution $\{\widetilde{n}/\widetilde{i}\}$ exists such that $1 \leq [\![\phi]\!]_{\text{3-SAT}}\{\widetilde{n}/\widetilde{i}\}$. A clause in $\phi$ is a disjunction of three literals of either the form $x$ or $\neg x$ for some unknown $x$. Thus, for a clause to be true, we must have at least one literal $\tau(x) = tt$ or $\neg \tau(x) = tt$ with $\tau(x) = f\!f$. The corresponding factor in $[\![\phi]\!]_{\text{3-SAT}}$ is a sum of three terms of the forms $f(x)$ or $(1 - f(x))$ for some unknown $x$, where $f$ is a one-to-one mapping from unknowns to index variables. Here, we utilize that in the type system by Baillot and Ghyselen \cite{BaillotGhyselen2021}, we have $(1 - i\{\widetilde{n}/\widetilde{i}\}) = 0$ when $i\{\widetilde{n}/\widetilde{i}\} \geq 1$ and $(1 - i\{\widetilde{n}/\widetilde{i}\}) = 1$ when $i\{\widetilde{n}/\widetilde{i}\} = 0$. Thus, for a factor to be positive, it suffices that one term is positive, and so we can construct a substitution that guarantees this from the interpretation of $\phi$. That is, if $\tau(x) = tt$, we substitute $1$ for $f(x)$, and if $\tau(x) = f\!f$, we substitute 0 for $f(x)$. Then, whenever a literal is true in $\phi$, the corresponding term in $[\![\phi]\!]_{\text{3-SAT}}$ is positive, and so if $\phi$ is satisfiable then there exists a substitution $\{\widetilde{n}/\widetilde{i}\}$ such that $1 \leq [\![\phi]\!]_{\text{3-SAT}}\{\widetilde{n}/\widetilde{i}\}$.
     
    \item Assume that there exists a substitution $\{\widetilde{n}/\widetilde{i}\}$ such that $1 \leq [\![\phi]\!]_{\text{3-SAT}}\{\widetilde{n}/\widetilde{i}\}$. Then, as $[\![\phi]\!]_{\text{3-SAT}}$ is a product of non-negative factors, each factor must be positive. Correspondingly, if $\Phi$ is satisfiable, then there exists a truth assignment such that each clause of $\phi$ is true. We compare the conditions for a factor in $[\![\phi]\!]_{\text{3-SAT}}\{\widetilde{n}/\widetilde{i}\}$ to be positive to those for a corresponding clause in $\phi$ to be true, and show that $\phi$ is satisfiable. A factor in $[\![\phi]\!]_{\text{3-SAT}}$ is a sum of at most three terms of the forms $f(x)\{\widetilde{n}/\widetilde{i}\}$ or $(1 - f(x)\{\widetilde{n}/\widetilde{i}\})$. Here we again utilize that in the type system by Baillot and Ghyselen \cite{BaillotGhyselen2021}, we have $(1 - f(x)\{\widetilde{n}/\widetilde{i}\}) = 0$ when $f(x)\{\widetilde{n}/\widetilde{i}\} \geq 1$ and $(1 - f(x)\{\widetilde{n}/\widetilde{i}\}) = 1$ when $f(x)\{\widetilde{n}/\widetilde{i}\} = 0$, and so it must be that in the factor, we have at least one term $f(x)\{\widetilde{n}/\widetilde{i}\} \geq 1$ or $(1 - f(x)\{\widetilde{n}/\widetilde{i}\}) \geq 1$. Correspondingly, for the clause in $\phi$ to be true, at least one literal must be true. We show that there exists a truth assignment $\tau$ such that if a term in $[\![\phi]\!]_{\text{3-SAT}}$ is positive, then the corresponding literal in $\phi$ is true. If $f(x)\{\widetilde{n}/\widetilde{i}\}\geq 1$ then we set $\tau(x) = tt$, and if $f(x)\{\widetilde{n}/\widetilde{i}\} = 0$ we set $\tau(x) = f\!f$, as $[\![x]\!]_{\text{3-SAT}}\{\widetilde{n}/\widetilde{i}\} \geq 1$ when $f(x)\{\widetilde{n}/\widetilde{i}\} \geq 1$ and $[\![\neg x]\!]_{\text{3-SAT}} \geq 1$ when $f(x)\{\widetilde{n}/\widetilde{i}\}=0$. Then, whenever a term is positive in $[\![\phi]\!]_{\text{3-SAT}}\{\widetilde{n}/\widetilde{i}\}$, the corresponding literal in $\phi$ is true, and so if there exists a substitution $\{\widetilde{n}/\widetilde{i}\}$ such that $1 \leq [\![\phi]\!]_{\text{3-SAT}}\{\widetilde{n}/\widetilde{i}\}$, then $\phi$ is satisfiable.
    
\end{enumerate}
\end{proof}
\end{lemma}


\begin{lemma}\label{lemma:npcompletesubst}
Let $T$ and $S$ be types with polynomial indices. Then checking whether there exists a substitution $\{\widetilde{J}/\widetilde{i}\}$ such that $T = S\{\widetilde{J}/\widetilde{i}\}$ is an NP-complete problem.
\begin{proof}
By reduction from the 3-SAT problem. Assume that we have some algorithm that can verify the existence of a substitution $\{\widetilde{J}/\widetilde{i}\}$ such that $T = S\{\widetilde{J}/\widetilde{i}\}$, and let $\phi$ be a 3-SAT formula. Then using the algorithm, we can check whether $\phi$ is satisfiable by verifying whether there exists $\{\widetilde{J}/\widetilde{i}\}$ such that the following holds
\begin{align*}
    \texttt{Nat}[0,1] = \texttt{Nat}[0,(1 - (1 - [\![\phi]\!]_{\text{3-SAT}}))]\{\widetilde{J}/\widetilde{i}\}
\end{align*}
That is, $1 = (1 - (1 - [\![\phi]\!]_{\text{3-SAT}}\{\widetilde{J}/\widetilde{i}\}))$ implies $1 \leq [\![\phi]\!]_{\text{3-SAT}}\{\widetilde{J}/\widetilde{i}\}$, as $(1 - [\![\phi]\!]_{\text{3-SAT}}\{\widetilde{J}/\widetilde{i}\}) = 0$ when $[\![\phi]\!]_{\text{3-SAT}}\{\widetilde{J}/\widetilde{i}\} \geq 1$ and $(1 - [\![\phi]\!]_{\text{3-SAT}}\{\widetilde{J}/\widetilde{i}\}) = 1$ when $[\![\phi]\!]_{\text{3-SAT}}\{\widetilde{J}/\widetilde{i}\} = 0$. Furthermore, for $1 \leq [\![\phi]\!]_{\text{3-SAT}}\{\widetilde{J}/\widetilde{i}\}$ to hold, the indices in the sequence $\widetilde{J}$ cannot contain index variables, and so there must exist an equivalent substitution of naturals for index variables $\{\widetilde{n}/\widetilde{i}\}$. Then, by Lemma \ref{lemma:soundtranslation} we have that $\phi$ is satisfiable if and only if there exists a substitution $\{\widetilde{n}/\widetilde{i}\}$ such that $1\leq [\![\phi]\!]_{\text{3-SAT}}\{\widetilde{n}/\widetilde{i}\}$. Thus, as 3-SAT is an NP-complete problem, the reduction from 3-SAT is computable in polynomial time and as polynomial reduction is a transitive relation, i.e. any NP-problem is polynomial time reducible to verifying the existence of a substitution $\{\widetilde{J}/\widetilde{i}\}$ that satisfies the equation $T = S\{\widetilde{J}/\widetilde{i}\}$, it follows that the problem is NP-hard. To show that it is an NP-complete problem, we show that a \textit{certificate} can be verified in polynomial time. That is, given some substitution $\{\widetilde{J}/\widetilde{i}\}$, we can in linear time check whether $T=S\{\widetilde{J}/\widetilde{i}\}$ by substituting indices $\widetilde{J}$ for indices $\widetilde{i}$ in type $S$ and by then comparing the two types.\\
%
%
%Utilizing that $n - m = 0$ for $m\geq n$ in the type system of Baillot and Ghyselen \cite{BaillotGhyselen2021}, we can simulate any boolean formula using a polynomial index. By denoting $J = 0$ false and $I > 0$ true, we have the translation
% \begin{align*}
%     [\![a \land b]\!]_\phi =&\; [\![a]\!]_\phi [\![b]\!]_\phi\\
%     [\![a \lor b]\!]_\phi =&\; [\![a]\!]_\phi + [\![b]\!]_\phi\\
%     [\![\neg a]\!]_\phi =&\; (1 - [\![a]\!]_\phi)\\
%     [\![x]\!]_\phi =&\; i
% \end{align*}
% Then assuming some algorithm that checks whether there exists a substitution $\{\widetilde{J}/\widetilde{i}\}$ such that $T \sqsubseteq S\{\widetilde{J}/\widetilde{i}\}$, we can solve the boolean satisfiability problem. Let $\phi_0$ be any boolean formula and let $\widetilde{i}$ be the index variables in $[\![\phi_0]\!]_\phi$, and assume that there exists a substitution $\{\widetilde{J}/\widetilde{i}\}$ that satisfies the judgement
% \begin{align*}
%     \emptyset;\emptyset\vDash\texttt{Nat}[0,1] \sqsubseteq \texttt{Nat}[0,[\![\phi_0]\!]_\phi]\{\widetilde{J}/\widetilde{i}\}
% \end{align*}
% Then by rule $\runa{SS-nweak}$ we have that $\emptyset;\emptyset\vDash 1 \leq [\![\phi_0]\!]_\phi\{\widetilde{J}/\widetilde{i}\}$, and as $\varphi = \emptyset$, the indices $\widetilde{J}$ must be constants. Thus, $\emptyset;\emptyset\vDash 1 \leq [\![\phi_0]\!]_\phi\{\widetilde{J}/\widetilde{i}\}$ is equivalent to $1 \leq [\![\phi_0]\!]_\phi\{\widetilde{J}/\widetilde{i}\}$, and so $\phi_0$ must have a solution. If instead no such substitution exists, then for any $\{\widetilde{J}/\widetilde{i}\}$, it must be that $[\![\phi_0]\!]_\phi\{\widetilde{J}/\widetilde{i}\} = 0$ implying that $\phi_0$ is a contradiction. Therefore, as the boolean satisfiability problem is NP-complete, the algorithm we assumed must be NP-complete as well.
\end{proof}
\end{lemma}

In Example \ref{example:addition}, we show how a process implementing addition of naturals can be typed using our type rules, yielding a precise bound on the parallel complexity.
%

\begin{examp}\label{example:addition}
As an example of a process that is typable using our type rules, we show how the addition operator for naturals can be written as a process and subsequently be typed. We use a server to encode the addition operator
\begin{align*}
    !\inputch{\text{add}}{x,y,r}{}{\match{x}{\asyncoutputch{r}{y}{}}{z}{\tick{\asyncoutputch{\text{add}}{z,\succc y,r}{}}}}
\end{align*}
such that channel $r$ is used to output the addition of naturals $x$ and $y$. To type the process, we use the following contexts and set of index variables
\begin{align*}
    \Gamma\defeq&\; \text{add} : \forall_0 i,j,k,l,m,n,o.\texttt{serv}^{\{\texttt{in},\texttt{out}\}}_j(\texttt{Nat}[0,j],\texttt{Nat}[0,l],\texttt{ch}^{\{\texttt{out}\}}_j(\texttt{Nat}[0,j+l])) \\
    \Delta\defeq&\; \text{ready}(\cdot,\cdot,\Gamma), x : \texttt{Nat}[0,j], y: \texttt{Nat}[0,l], r:\texttt{ch}^{\{\texttt{out}\}}_j(\texttt{Nat}[0,j+l])\\
    \varphi \defeq&\; \{i,j,k,l,m,n,o\}
\end{align*}
%
We now derive a type for the encoding of the addition operator, yielding a precise bound of $j$, corresponding to an upper bound on the size of $x$, as we pattern match at most $j$ times on natural $x$. Notably we have that $\text{instantiate}((i,j,k,l,m,n,o),\texttt{Nat}[0,j\monus 1],\texttt{Nat}[1,l+1],\texttt{ch}^{\{\texttt{out}\}}_j(\texttt{Nat}[0,j+l]))=\{0/i,j\monus 1/j,0/k,l+1/l,j/m,0/n,j+l/o\}$.
%
{\small
\begin{align*}
    \begin{prooftree}
        %
        \infer0{\varphi;\cdot,0\leq 0;\Delta\vdash \asyncoutputch{r}{y}{} \triangleleft \{j\}}
        %
        % \infer0{\texttt{Nat}[0,j\monus 1] \sqsubseteq \texttt{Nat}[0,j]\{j\monus 1/j\}}
        % %
        % \infer0{\texttt{Nat}[0,l+1] \sqsubseteq \texttt{Nat}[0,l]\{l+1/l\}}
        % %
        % \infer0{\texttt{ch}^{\{\texttt{out}\}}_{j\monus 1}(\texttt{Nat}[0,j+l] \sqsubseteq \texttt{ch}^{\{\texttt{out}\}}_{j\monus 1}(\texttt{Nat}[0,j+l)\{j\monus 1/j,l+1/l\}}
        %
        \infer0{
        \begin{matrix}
        \varphi;\cdot,1\leq j\vdash\texttt{Nat}[0,j\monus 1] \sqsubseteq \texttt{Nat}[0,j]\{j\monus 1/j\}\\
        \varphi;\cdot,1\leq j\vdash\texttt{Nat}[1,l+1] \sqsubseteq \texttt{Nat}[0,l]\{l+1/l\}\\
        \varphi;\cdot,1\leq j\vdash\texttt{ch}^{\{\texttt{out}\}}_{j\monus 1}(\texttt{Nat}[0,j+l] \sqsubseteq \texttt{ch}^{\{\texttt{out}\}}_{j\monus 1}(\texttt{Nat}[0,j+l)\{j\monus 1/j,l+1/l\}
        \end{matrix}
        }
        %
        \infer1{\varphi;\cdot,1\leq j;\susumesim{\Delta}{1},z : \texttt{Nat}[0,j\monus 1]\vdash \asyncoutputch{\text{add}}{z,\succc y, r}{} \triangleleft \{j\monus 1\}}
        %
        \infer1{\varphi;\cdot,1\leq j;\Delta,z : \texttt{Nat}[0,j\monus 1]\vdash \tick{\asyncoutputch{\text{add}}{z,\succc y, r}{}} \triangleleft \{j\}}
        %
        \infer2{\varphi;\cdot;\Delta\vdash \match{x}{\asyncoutputch{r}{y}{}}{z}{\tick{\asyncoutputch{\text{add}}{z,\succc y,r}{}}} \triangleleft \{j\}}
        %
        \infer1{\cdot;\cdot;\Gamma\vdash\; !\inputch{\text{add}}{x,y,r}{}{\match{x}{\asyncoutputch{r}{y}{}}{z}{\tick{\asyncoutputch{\text{add}}{z,\succc y,r}{}}}}\triangleleft \{0\}}
    \end{prooftree}
\end{align*}}
%
\end{examp}

% \subsection{Undecidability of judgements}
% Verifying whether a polynomial constraint with integer coefficients imposes further restrictions onto the model set of index valuations of natural codomain of some set of known constraints can be reduced to Hilbert's tenth problem \cite{Davis1973}. That is, the problem of verifying whether a diophantine equation has an integer solution.\\

% We first assume some algorithm that can verify a judgement of the form $\varphi;\Phi\vDash C$ where $\varphi$ is a set of index variables and $C$ and $C'\in\Phi$ are binary constraints on polynomials of integer coefficients over relations from any subset of $\{\neq,\leq, <\}$. Recall that such a judgement holds exactly when for each index valuation $\rho : \varphi \longrightarrow \mathbb{N}$ over $\varphi$ for which $\rho \vDash C'$ for $C'\in\Phi$ we also have $\rho\vDash C$, i.e. $C$ does not impose further restrictions on interpretations of indices.\\

% We can then verify whether any diophantine equation has an integer solution. Let $p$ be an arbitrary polynomial of integer coefficients such that $p = 0$ is a diophantine equation. As only non-negative integers substitute for index variables, we first transform $p = 0$ to a new diophantine equation $p' = 0$ that has a non-negative integer solution exactly when $p = 0$ has an integer solution. To do this, we simply replace each index variable $i$ in $p$ with two new index variables $i_1 - i_2$. Then the judgement $\varphi;\emptyset\vDash p' \neq 0$ holds exactly when $p=0$ has no integer solution. That is, if $p=0$ has an integer solution, then there must exist a valuation $\rho_0$ such that $\rho_0\vDash \emptyset$ with $[\![p']\!]_{\rho_0} = 0$ and so $\rho_0\nvDash p' \neq 0$. Moreover, we need not rely on the relation $\neq$, as the judgements below are equivalent
% \begin{align*}
%     \varphi;\{p' \leq 0\} \vDash p' < 0\\
%     \varphi;\{p' \leq 0\} \vDash p' \leq 1
% \end{align*}
\section{Soundness}
\section{Verification of constraint judgements}\label{sec:verifyinglinearjudgements}
Until now we have not considered how we can verify constraint judgements in the type rules. The expressiveness of implementations of the type system by Baillot and Ghyselen \cite{BaillotGhyselen2021} depends on both the expressiveness of indices and whether judgements on the corresponding constraints are decidable. Naturally, we are interested in both of these properties, and so in this section, we show how judgements on linear constraints can be verified using algorithms. Later, we show how this can be extended to certain groups of polynomial constraints. We first make some needed changes to how the type checker uses subtraction.
%
\subsection{Subtraction of naturals}
The constraint judgements rely on a special minus operator ($\monus$) for indices such that $n \monus m=0$ when $m \geq n$, which we refer to as the \textit{monus} operator. This is apparent in the pattern match constructor type rule from Chapter \ref{ch:bgts}. Without this behavior, we may encounter problems when checking subtype premises in match processes. This has the consequence that equations such as $2\monus 3+3=3$ hold, such that indices form a semiring rather than a ring, as we are no longer guaranteed an additive inverse. In general, semirings lack many properties of rings that are desirable. For example, given two seemingly equivalent constraints $i \leq 5$ and $i \monus 5 \leq 0$, we see that by adding any constant to their left-hand sides, the constraints are no longer equivalent. Adding the constant 2 to their left-hand sides, we obtain $i + 2 \leq 5$ and $i \monus 5 + 2 \leq 0$, however, we see that the first constraint is satisfied given the valuation $i = 3$ but the second is not. In general the associative property of $+$ is lost.\\

Unfortunately, this is not an easy problem to solve implementation-wise, as indices are not actually evaluated but rather represent whole feasible regions. Thus, instead of trying to implement this operator exactly, we limit the number of processes typable by the type system. Removing the operator entirely is not an option as it us used by the type rules themselves. Instead, we ensure that one cannot \textit{exploit} the special behavior of monus by introducing additional conditions to the type rules of the type system. More precisely, any time the type system uses the monus operator such as $I \monus J$, we require the premise $\varphi;\Phi \vDash I \geq J$, in which case the monus operator is safe to treat as a regular minus. This, however, puts severe restrictions on the number of processes typable, and so we relax the restriction a bit by also checking the judgement $\varphi;\Phi \vDash I \leq J$, in which case we can conclude that the result is definitely $0$. If neither $\varphi;\Phi\vDash I \geq J$ nor $\varphi;\Phi\vDash I \leq 0$ hold, which is possible as $\leq$ and $\geq$ do not form a total order on indices, the result is undefined. We refer to this variant of monus as the \textit{partial} monus operator, as formalized in Definition \ref{def:partialmonus}. Note that this definition of monus allows us to obtain identical behavior to minus on a constraint $I \bowtie J$ by moving terms between the LHS and RHS, i.e. $I - K \bowtie J \Rightarrow I \bowtie J + K$, and so we can assume we have a standard minus operator when verifying judgements on constraints. For the remainder of this section, we assume this definition is used in the type rules instead of the usual monus. We may omit $\varphi;\Phi$ if it is clear from the context.%\\
%
%Definition \ref{def:partialmonus} defines the \textit{partial} monus operator that is undefined if we cannot determine if the result is either always positive or always zero. For the remainder of this thesis, we assume this definition is used in the type rules instead of the usual monus. We may omit $\varphi;\Phi$ if it is clear from the context.
%
\begin{defi}[Partial monus]\label{def:partialmonus}
Let $\Phi$ be a set of constraints in index variables $\varphi$. The partial monus operator is defined for two indices $I$ and $J$ as
\begin{equation*}
    I \monusE J = \begin{cases}
    I - J &\text{if $\varphi;\Phi \vDash J \leq I$}\\
    0 &\text{if $\varphi;\Phi \vDash I \leq J$}\\
    \textit{undefined} & \textit{otherwise}
    \end{cases}
\end{equation*}
\end{defi}

To ensure soundness of the algorithmic type rules after switching to the partial monus operator, we must make some changes to advancement of time. Consider the typing
\begin{align*}
    (\cdot,i);(\cdot,i\leq 3);\Gamma\vdash\; !\inputch{a}{}{}{\nil}  \mid 5 : \asyncoutputch{a}{}{} \triangleleft \{5\}
\end{align*}
where $\Gamma = \cdot,a : \forall_{3-i}\epsilon.\texttt{serv}^{\{\texttt{in},\texttt{out}\}}_0()$. Upon typing the time annotation, we advance the time of the server type by $5$ yielding the type $\forall_{3-i-5}\epsilon.\texttt{serv}^{\{\texttt{out}\}}_0()$ as $(\cdot,i);(\cdot,i\leq 3)\nvDash 3-i \geq 5$, which is defined as $(\cdot,i);(\cdot,i\leq 3)\vDash 3-i \leq 5$. However, if we apply the congruence rule $\runa{SC-sum}$ from right to left we obtain
\begin{align*}
    !\inputch{a}{}{}{\nil}  \mid 2 : 3 : \asyncoutputch{a}{}{}\equiv\;!\inputch{a}{}{}{\nil}  \mid 5 : \asyncoutputch{a}{}{}
\end{align*}
Then, we get a problem upon typing the first annotation. That is, as $(\cdot,i);(\cdot,i\leq 3)\nvDash 3-i \leq 2$ (i.e. when for some valuation $\rho$ we have $\rho(i) = 0$) the operation $(3-i) \monusE[(\cdot,i);(\cdot,i\leq 3)] 2$ is undefined. Thus, the type system loses its subject congruence property, and subsequently its subject reduction property. There are, however, several ways to address this. One option is to modify the type rules to perform a single advancement of time for a sequence of annotations. A more contained option is to remove monus from the definition of advancement of time, by enriching the formation rules of types with the constructor $\forall_{I}\widetilde{i}.\texttt{serv}^\sigma_K(\widetilde{T})^{-J}$ and by augmenting the definition of advancement as so
\begin{align*}
    \downarrow_I^{\varphi;\Phi}\!\!(\forall_J\widetilde{i}.\texttt{serv}^\sigma_K(\widetilde{T})) =&\; \left\{
\begin{matrix}
\forall_{J-I}\widetilde{i}.\texttt{serv}^\sigma_K(\widetilde{T}) & \text{ if } \varphi;\Phi\vDash I \leq J \\
\forall_0\widetilde{i}.\texttt{serv}^{\sigma\cap\{\texttt{out}\}}_K(\widetilde{T}) & \text{ if } \varphi;\Phi\vDash J \leq I \\
\forall_{J}\widetilde{i}.\texttt{serv}^{\sigma\cap\{\texttt{out}\}}_K(\widetilde{T})^{-I} & \text{ if } \varphi;\Phi\nvDash I \leq J \text{ and } \varphi;\Phi\nvDash J \leq I
\end{matrix}
\right.\\
%
\downarrow_I^{\varphi;\Phi}\!\!(\forall_{J}\widetilde{i}.\texttt{serv}^\sigma_K(\widetilde{T})^{-L}) =&\; \left\{
\begin{matrix}
\forall_{0}\widetilde{i}.\texttt{serv}^{\sigma\cap\{\texttt{out}\}}_K(\widetilde{T}) & \text{ if } \varphi;\Phi\vDash J \leq L+I \\
\forall_{J}\widetilde{i}.\texttt{serv}^{\sigma\cap\{\texttt{out}\}}_K(\widetilde{T})^{-(L+I)} & \text{ if } \varphi;\Phi\nvDash J \leq L+I
\end{matrix}
\right.
%
\end{align*}
This in essence introduces a form of \textit{lazy} time advancement, where time is not advanced until partial monus allows us to do so. Then, as the type rules for servers require a server type of the form $\forall_J\widetilde{i}.\texttt{serv}^\sigma_K(\widetilde{T})$, the summed advancement of time must always be less than or equal, or always greater than or equal to the time of the server, and so typing is invariant to the use of congruence rule $\runa{SC-sum}$. Revisiting the above example, we have that $\susume{\forall_{3-i}\epsilon.\texttt{serv}^{\{\texttt{in},\texttt{out}\}}_0()}{(\cdot,i)}{(\cdot,i\leq 3)}{5} =\; \susume{\susume{\forall_{3-i}\epsilon.\texttt{serv}^{\{\texttt{in},\texttt{out}\}}_0()}{(\cdot,i)}{(\cdot,i\leq 3)}{2}}{(\cdot,i)}{(\cdot,i\leq 3)}{3}$, and so we obtain the original typing
\begin{align*}
    (\cdot,i);(\cdot,i\leq 3);\Gamma\vdash\; !\inputch{a}{}{}{\nil}  \mid 2 : 3 : \asyncoutputch{a}{}{} \triangleleft \{5\}
\end{align*}



% \begin{remark}

%     Baillot and Ghyselen \cite{BaillotGhyselen2021} assume that the minus operator ($-$) for indices is defined such that $n-m=0$ when $m \geq n$. This has the consequence that expressions such as $2-3+3=3$ apply, such that indices form a semiring instead of a ring as we no longer have an additive inverse. In this work we lift this assumption by arguing that any index $I$ using a ring-centric definition for $-$ such that $I \leq 0$, can be simulated using another index $J$ using a semiring-centric definition for $-$ such that $J \leq 0$. For $I$, the order of summation of terms does not matter, and so we can freely change this. By moving any terms with a negative coefficient to the end of the summation, we obtain an expression of the form $c_1 i_1 + \cdots + c_n i_n - c_{n+1} i_{n+1} - \cdots - c_m i_m$ where $c_j$ are positive numbers and $i_j$ are index variables for $j = 0\dots m$. When evaluating this expression from left to right, the result will be increasing until $c_{n + 1} i_{n+1}$, as both the coefficients and index variables are positive, after which it will be decreasing. This results in an expression that is indifferent to the two definitions of $-$ when considering constraints of the form $I \leq 0$. Thus, a normalized constraint using a ring-centric definition of $-$ can be simulated using a normalized constraint using a semiring-centric definition of $-$.

% \end{remark}

\subsection{Undecidability of polynomial constraint judgements}
As we have seen, verifying that a constraint imposes no further restrictions onto index valuations amounts to checking whether all possible index valuations that satisfy a set of known constraints are also contained in the model space of our new constraint. It also amounts to checking whether the feasible region of the constraint contains the feasible region of a known system of inequality constraints, or checking whether the feasible region of the inverse constraint does not intersect the feasible region of a known system of inequality constraints. This turns out to be a difficult problem, and we can in fact prove it undecidable for diophantine constraints, i.e. multivariate polynomial inequalities with integer coefficients, when index variables must have natural (or integer) interpretations. The main idea is to reduce Hilbert's tenth problem \cite{Hilbert1902} to that of verification of judgements on constraints, as this problem has been proven undecidable \cite{Davis1973}. That is, we show that assuming some complete algorithm that verifies judgements on constraints, we can verify whether an arbitrary diophantine equation has a solution with all unknowns taking integer values. We show this result in Lemma \ref{lemma:judgementUndecidable}.
%
\begin{lemma}\label{lemma:judgementUndecidable}
Let $C$ and $C'\in \Phi$ be diophantine inequalities with unknowns in $\varphi$ and coefficients in $\mathbb{N}$. Then the judgement $\varphi;\Phi\vDash C$ is undecidable.
\begin{proof}
By reduction from Hilbert's tenth problem. Let $p=0$ be an arbitrary diophantine equation. We show that assuming some algorithm that can verify a judgement of the form $\varphi;\Phi\vDash C$, we can determine whether $p=0$ has an integer solution. We must pay special attention to the non-standard definition of subtraction in the type system by Baillot and Ghyselen \cite{BaillotGhyselen2021} and to the fact that only non-negative integers substitute for index variables. We first replace each integer variable $x$ in $p$ with two non-negative variables $i_x - j_x$, referring to the modified polynomial as $p'$. We can quickly verify that $p'=0$ has a non-negative integer solution if and only if $p=0$ has an integer solution
\begin{enumerate}
    \item Assume that $p'=0$ has a non-negative integer solution. Then for each variable $x$ in $p$ we assign $x = i_x - j_x$ reaching an integer solution to $p$.
    
    \item Assume that $p=0$ has an integer solution. Then for each pair $i_x$ and $j_x$ in $p'$ we assign $i_x = x$ and $j_x = 0$ when $x \geq 0$ and $i_x = 0$ and $j_x = |x|$ when $x < 0$ reaching a non-negative integer solution to $p'$.
\end{enumerate}
%
Then, by the distributive property of integer multiplication and the associative property of integer addition, we can utilize that $p'$ has an equivalent expanded form 
\begin{align*}
p' = n_1 t_1 + \cdots + n_k t_k + n_{k+1} t_{k+1} + \cdots + n_{k+l} t_{k+l}    
\end{align*}
such that $n_1,\dots,n_k\in\mathbb{N}$, $n_{k+1},\dots,n_{k+l} \in \mathbb{Z}^{\leq 0}$ and $t_1,\dots,t_k,t_{k+1},\dots,t_{k+l}$ are power products over the set of all index variables in $p'$ denoted $\varphi_{p'}$. We can then factor the negative coefficients
\begin{align*}
    p' \;&= n_1 t_1 + \cdots + n_k t_k + n_{k+1} t_{k+1} + \cdots + n_{k+l} t_{k+l}\\ 
    \;&= (n_1 t_1 + \cdots + n_k t_k) + (-1)(|n_{k+1}| t_{k+1} + \cdots + |n_{k+l}| t_{k+l})\\
    \;&= (n_1 t_1 + \cdots + n_k t_k) - (|n_{k+1}| t_{k+1} + \cdots + |n_{k+l}| t_{k+l})
\end{align*}
We use this to show that $p'=0$ has a non-negative integer solution if and only if the following judgement does not hold 
{\small
\begin{align*}
    \varphi_{p'};\{|n_{k+1}| t_{k+1} + \cdots + |n_{k+l}| t_{k+l} \leq n_1 t_1 + \cdots + n_k t_k\}\vDash 1 \leq (n_1 t_1 + \cdots + n_k t_k) - (|n_{k+1}| t_{k+1} + \cdots + |n_{k+l}| t_{k+l}) 
\end{align*}}
We consider the implications separately
\begin{enumerate}
    \item Assume that $p'=0$ has a non-negative integer solution. Then we have that $n_1 t_1 + \cdots + n_k t_k = |n_{k+1}| t_{k+1} + \cdots + |n_{k+l}| t_{k+l}$, and so there must exist a valuation $\rho : \varphi_{p'} \longrightarrow \mathbb{N}$ such that $[\![n_1 t_1 + \cdots + n_k t_k]\!]_\rho = [\![|n_{k+1}| t_{k+1} + \cdots + |n_{k+l}| t_{k+l}]\!]_\rho$. We trivially have that $\rho$ satisfies $[\![|n_{k+1}| t_{k+1} + \cdots + |n_{k+l}| t_{k+l}]\!]_\rho \leq [\![n_1 t_1 + \cdots + n_k t_k]\!]_\rho$. But $\rho$ is not in the model space of the constraint $1 \leq (n_1 t_1 + \cdots + n_k t_k) - (|n_{k+1}| t_{k+1} + \cdots + |n_{k+l}| t_{k+l})$, and so the judgement does not hold.
    
    \item Assume that the judgement does not hold. Then there must exist a valuation $\rho : \varphi_{p'} \longrightarrow \mathbb{N}$ that satisfies $[\![|n_{k+1}| t_{k+1} + \cdots + |n_{k+l}| t_{k+l}]\!]_\rho \leq [\![n_1 t_1 + \cdots + n_k t_k]\!]_\rho$, but that is not in the model space of the constraint $1 \leq (n_1 t_1 + \cdots + n_k t_k) - (|n_{k+1}| t_{k+1} + \cdots + |n_{k+l}| t_{k+l})$. This implies that $[\![n_1 t_1 + \cdots + n_k t_k]\!]_\rho = [\![|n_{k+1}| t_{k+1} + \cdots + |n_{k+l}| t_{k+l}]\!]_\rho$, and so $p'$ has a non-negative integer solution.
\end{enumerate}
% Then the subtraction operator in Baillot and Ghyselen $\cite{BaillotGhyselen2021}$ only has non-standard behavior when $[\![n_{k+1} t_{k+1} + \cdots + n_{k+l} t_{k+l}]\!]_\rho > [\![n_1 t_1 + \cdots + n_k t_k]\!]_\rho$ for some interpretation $\rho : \varphi_{p'} \longrightarrow \mathbb{N}$ where $\varphi_{p'}$ is the set of all index variables in $p'$. Thus, we have that the judgement
% \begin{align*}
%     \varphi_{p'};\{n_{k+1} t_{k+1} + \cdots + n_{k+l} t_{k+l} \leq n_1 t_1 + \cdots + n_k t_k\}\vDash 1 \leq (n_1 t_1 + \cdots + n_k t_k) - (n_{k+1} t_{k+1} + \cdots + n_{k+l} t_{k+l}) 
% \end{align*}
% holds exactly when there exists no index valuation $\rho$ over $\varphi_{p'}$ that simultaneously satisfies $[\![n_{k+1} t_{k+1} + \cdots + n_{k+l} t_{k+l}]\!]_\rho \leq [\![n_1 t_1 + \cdots + n_k t_k]\!]_\rho$ and $[\![p']\!]_\rho = 0$. 
As such, we can verify that the above judgement does not hold if and only if $p'$ has a non-negative integer solution, and by extension if and only if $p$ has an integer solution. Thus, we would have a solution to Hilbert's tenth problem, which is undecidable.
\end{proof}
\end{lemma}

As an unfortunate consequence of Lemma \ref{lemma:judgementUndecidable}, we are forced into considering approximate algorithms for verification of judgements over polynomial constraints (in general). However, this result does not imply that type checking is undecidable. It may well be that problematic judgements are not required to type check any process, as computational complexity has certain properties, such as monotonicity. Note that the freedom of type checking, i.e. we can specify an arbitrary type context as well as type annotations, enables us to select indices that lead to undecidable judgements. To prove that type checking is undedidable, however, a more reasonable result would be that there exists a process that is typable if and only if an undecidable judgement is satisfied. This is out of the scope of this thesis, and so we leave it as future work. % Remark that Baillot and Ghyselen \cite{BaillotGhyselen2021} introduce a notion of type inference in their technical report, where the set of constraints $\Phi$ is empty for any judgement on constraints, and so they are able to bypass some of the problems associated with checking such judgements. However, this comes at the price of expressiveness, as natural types are forced to have lower bounds of $0$ and upper bounds with exactly one index variable and constant. Such indices are arguably sufficient for describing the sizes of simple terms when all operations on these terms in a program can be correspondingly described with a single index variable and constant. However, this quickly becomes too restrictive, as we are unable to type servers that implement simple arithmetic operations such as addition and subtraction.

\subsection{Normalization of linear indices}

To make checking of judgements on constraints tractable, we reduce the set of function symbols on which indices are defined, such that indices may only contain integers and index variables, as well as addition, subtraction and scalar multiplication operators, such that we restrict ourselves to linear functions.
\begin{align*}
        I,J ::= n \mid i \mid I + J \mid I - J \mid n I
    \end{align*}
% \begin{defi}[Indices]
%     \begin{align*}
%         I,J ::= n \mid i \mid I + J \mid I - J \mid I \cdot J
%     \end{align*}
% \end{defi}


Such indices can be written in a \textit{normal} form, presented in Definition \ref{def:normlinindex}.

\begin{defi}[Normalized linear index]\label{def:normlinindex}
    Let $I$ be an index in index variables $\varphi = i_1,\dots,i_n$. We say that $I$ is a \textit{normalized} index when it is a linear combination of index variables $i_1, ..., i_n$. Let $m$ be an integer constant and $I_\alpha\in\mathbb{Z}$ the coefficient of variable $i_\alpha$, we then define normalized indices as
    %
    \begin{align*}
        I = \normlinearindex{m}{I}
    \end{align*}
    
    
    We use the notation $\mathcal{B}(I)$ and $\mathcal{E}(I)$ to refer to the constant and unique identifiers of index variables of $I$, respectively.
\end{defi}

Any index can be transformed to an equivalent normalized index (i.e. it is a normal form) through expansion with the distributive law, reordering by the commutative and associative laws and then by regrouping terms that share variables. Therefore, the set of normalized indices in index variables $i_1,\dots,i_n$ and with coefficients in $\mathbb{Z}$, denoted $\mathbb{Z}[i_1,\dots,i_n]$, is a free module with the variables as basis, as the variables are linearly independent. In Definition \ref{def:operationsmodule}, we show how scalar multiplication, addition and multiplication of normalized indices (i.e. linear combinations of monomials) can be defined. Definition \ref{def:normalizationindex} shows how an equivalent normalized index can be computed from an arbitrary linear index using these operations.
%
\begin{defi}[Operations in $\freemodule$]\label{def:operationsmodule}
Let $I = \normlinearindex[\varphi_1]{n}{I}$ and $J = \normlinearindex[\varphi_2]{m}{J}$ be normalized indices in index variables $i_1,\dots,i_n$. We define addition and scalar addition of such indices. Given a scalar $n\in\mathbb{Z}$, the scalar multiplication $n I$ is
%
\begin{align*}
    n I = \normlinearindex[\mathcal{E}(I)]{n \cdot m}{n I}
\end{align*}
When $d$ is a common divisor of all coefficients in $I$, i.e. $I_\alpha / n \in \mathbb{Z}$ for all $\alpha\in\varphi$, the inverse operation is defined
\begin{align*}
    \frac{I}{d} = \frac{n}{d} + \sum_{\alpha\in \mathcal{E}(I)} \frac{I_\alpha}{d} i_\alpha\quad\text{if}\;\frac{I_\alpha}{d} \in \mathbb{Z}\;\text{for all}\;\alpha\in\mathcal{E}(I)
\end{align*}

The addition of $I$ and $J$ is the sum of constants plus the sum of scaled variables where coefficients $I_\alpha$ and $J_\alpha$ are summed when $\alpha\in\varphi_1 \cap \varphi_2$
\begin{align*}
    I + J = n + m + \sum_{\alpha \in \mathcal{E}(I) \cup \mathcal{E}(J)}(I_\alpha + J_\alpha)i_\alpha
\end{align*}

where for any $\alpha\in \varphi_1 \cup \varphi_2$ such that $I_\alpha + J_\alpha = 0$ we omit the corresponding zero term. The inverse of addition is always defined for elements of a polynomial ring
%
\begin{align*}
    I - J = n - m + \sum_{\alpha \in \mathcal{E}(I) \cup \mathcal{E}(J)}(I_\alpha - J_\alpha)i^\alpha
\end{align*}
\end{defi}
%
%We now formalize the transformation of an index $I$ to an equivalent normalized index in Definition \ref{def:normalizationindex}. An integer constant $n$ corresponds to scaling the monomial identified by the exponent vector of all zeroes by $n$. An index variable $i$ represents the monomial consisting of exactly one $i$ scaled by $1$. For addition, subtraction and multiplication we simply normalize the two subindices and and use the corresponding operators for normalized indices.
\begin{defi}[Index normalization]\label{def:normalizationindex}
The normalization of some index $I$ in index variables $i_1,\dots,i_n$ into an equivalent normalized index $\mathcal{N}(I)\in \mathbb{Z}[i_1,\dots,i_n]$ is a homomorphism defined inductively
    \begin{align*}
        \mathcal{N}(n) =&\; n i_1^0\cdots i_n^0\\
        \mathcal{N}(i_j) =&\; 1 i_1^0 \cdots i_j^1 \cdots i_n^0\\
        \mathcal{N}(I + J) =&\; \mathcal{N}(I) + \mathcal{N}(J)\\
        \mathcal{N}(I - J) =&\; \mathcal{N}(I) - \mathcal{N}(J)\\
        \mathcal{N}(n I) =&\; n \mathcal{N}(I)
    \end{align*}
\end{defi}

% \subsubsection{Normalization of constraints}
% A constraint may provide stronger or weaker restrictions on index variables compared to another constraint, or it may provide entirely different restrictions that are neither stronger nor weaker. For example, assuming some index $J$, if we have the constraint $3 \cdot i \leq J$, the constraint $2 \cdot i \leq J$ is redundant as index variables can only be assigned natural numbers, and thus $3 \cdot i \leq J$ implies $n \cdot i \leq J$ for any $n \leq 3$. Similarly, $I \leq n \cdot j$ implies $I \leq m \cdot j$ for any $n \leq m$. We thus define the subconstraint relation $\sqsubseteq$, and by extension the subindex relation $\sqsubseteq_\text{Index}$, in Definition \ref{def:subconstraint}. If $C_1 \sqsubseteq C_2$ we say that $C_2$ is a subconstraint of $C_1$.


% \begin{defi}[Subindices and subconstraints] \label{def:subconstraint}
%     We define the subindex relation $\sqsubseteq_\text{Index}$ by the following rule
%     \begin{align*}
%         &I \sqsubseteq_\text{Index} J \quad \text{ if} \\
%         &\quad (\mathcal{B}(I) \leq \mathcal{B}(J)) \land\\
%         &\quad (\forall \alpha \in \mathcal{E}(I) \cap \mathcal{E}(J) : I_\alpha \leq J_\alpha)\land\\
%         &\quad (\forall \alpha \in \mathcal{E}(J) \setminus \mathcal{E}(I) : J_\alpha \geq 0)\land\\
%         &\quad (\forall \alpha \in \mathcal{E}(I) \setminus \mathcal{E}(J) : I_\alpha \leq 0)
%     \end{align*}
%     % \begin{align*}
%     %     &(\varphi, F) \sqsubseteq_\text{Index} (\varphi', F') \text{ if} \\
%     %     &\quad (\forall V \in \varphi \cap \varphi' : F(V) \leq F'(V)) \land\\
%     %     &\quad (\forall V \in \varphi' \setminus \varphi : F'(V) \geq 0) \land\\
%     %     &\quad (\forall V \in \varphi \setminus \varphi' : F'(V) \leq 0)
%     % \end{align*}
    
%     We define the subconstraint relation $\sqsubseteq$ by the following rule
%     \begin{align*}
%       &\infrule{I' \sqsubseteq_\text{Index} I \quad J \sqsubseteq_\text{Index} J'}{I \leq J \sqsubseteq I' \leq J'}
%       %
%       %
%       %&\infrule{I \leq J \sqsubseteq I' \leq J' \quad I' \leq J' \sqsubseteq I'' \leq J''}{I \leq J \sqsubseteq I'' \leq J''}
%     \end{align*}
% \end{defi}

We extend normalization to constraints. We first note that an equality constraint $I = J$ is satisfied if and only if $I \leq J$ and $J \leq I$ are both satisfied. Thus, it suffices to only consider inequality constraints. A normalized constraint is of the form $I \leq 0$ for some normalized index $I$, as formalized in Definition \ref{def:normconst}.
%
\begin{defi}[Normalized constraints]\label{def:normconst}
    Let $C = I \leq J$ be an inequality constraint such that $I$ and $J$ are normalized indices. We say that $I-J \leq 0$ is the normalization of $C$ denoted $\mathcal{N}(C)$, and we refer to constraints in this form as \textit{normalized} constraints.
    %We represent normalized constraints $C$ using a single normalized constraint $I$, such that $C$ is of the form
    %\begin{align*}
    %    C = I \leq 0
    %\end{align*}
%
\end{defi}
%
% We now show how any constraint $J \bowtie K$ can be represented using a set of normalized constraints of the form $I \leq 0$ where $I$ is a normalized index. To do this, we first represent the constraint $J \bowtie K$ using a set of constraints of the form $J \leq K$ using the function $\mathcal{N_R}$. We then finalize the normalization using the function $\mathcal{N}$ by first moving all indices to the left-hand side of the constraint.
%
% \begin{defi}
%     Given a constraint $I \bowtie J$ $(\bowtie\; \in \{\leq, \geq, =\})$, the function $\mathcal{N_R}$ converts $I \bowtie J$ to a set of constraints of the form $I \leq J$
%     %
%     \begin{align*}
%         \mathcal{N_R}(I \leq J) &= \{I \leq J\}\\
%         \mathcal{N_R}(I \geq J) &= \{J \leq I\}\\
%         %\mathcal{N_R}(I < J) &= \{I+1 \leq J\}\\
%         %\mathcal{N_R}(I > J) &= \{J+1 \leq I\}\\
%         \mathcal{N_R}(I = J) &= \{I \leq J, J \leq I\}
%     \end{align*}
% \end{defi}
%
% \begin{defi}
%     Given a constraint $C$, the function $\mathcal{N}$ converts $C$ into a set of normalized constraints of the form $I \leq 0$
%     %
%     \begin{align*}
%         \mathcal{N}(C) &= \left\{I-J \leq 0 \mid (I \leq J) \in \mathcal{N}_R(C)\right\}
%     \end{align*}
% \end{defi}
%
%Normalized constraints have the key property that, given any two constraints $I \leq 0$ and $J \leq 0$, we can combine these to obtain a new constraint $J + I \leq 0$. This is possible as we know that both $I$ and $J$ are both non-positive, and so their sum must also be non-positive. In general, given $n$ normalized constraints $I_1 \leq 0, ..., I_n \leq 0$, we can infer any linear combination $a_1 \cdot I_1 \leq 0 + ... + a_n \cdot I_n \leq 0$ where $a_i \geq 0$ for $i = 1..n$ as new constraints that can be inferred based on the constraints $I_1 \leq 0, ..., I_n \leq 0$. Linear combinations where all coefficients are non-negative are also called \textit{conical combinations}.
Normalizing constraints has a number of benefits. First of all, it ensures that equivalent constraints are always expressed the same way. Secondly, having all constraints in a common form where variables only appear once means we can easily reason about individual variables of a constraint, which will be useful later when we verify constraint judgements.
%
\subsection{Checking for emptiness of model space}
As explained in Section \ref{sec:cjalternativeform}, we can verify a constraint judgement $\varphi;\Phi \vDash C_0$ by letting $C_0'$ be the inverse of $C_0$ and checking if $\mathcal{M}_\varphi(\Phi \cup \{C_0'\}) = \emptyset$ holds. Being able to check for non-emptiness of a model space is therefore paramount for verifying constraint judgements. For convenience, given a finite ordered set of index variables $\varphi = \{i_1, i_2, \dots, i_n\}$, we represent a normalized constraint $I \leq 0$ as a vector $\left( \mathcal{B}(I), I_1\; I_2\; \cdots\; I_n \right)_{\varphi}$. As such, the constraint $-5i + -2j + -4k \leq 0$ can be represented by the vector $\cvect[\varphi_1]{0 {-5} {-2} {-4}}$ where $\varphi_1=\left\{i, j, k\right\}$. Another way to represent that same constraint is with the vector $\cvect[\varphi_2]{0 {-5} {-2} 0 {-4}}$ where $\varphi_2 = \left\{i,j,l,k\right\}$. We denote the vector representation of a constraint $C$ over a finite ordered set of index variables $\varphi$ by $\mathbf{C}_{\varphi}$. We extend this notation to sets of constraints, such that $\Phi_{\varphi}$ denotes the set of vector representations over $\varphi$ of normalized constraints in $\Phi$\\

Recall that the model space of any set of constraints $\Phi$ is the set of all valuations satisfying all constraints in $\Phi$. Thus, to show that $\mathcal{M}_\varphi(\Phi)$ is empty, we must show that no valuation $\rho$ exists satisfying all constraints in $\Phi$. This is a linear constraint satisfaction problem (CSP) with an infinite domain. One method for solving such is by optimization using the simplex algorithm. If the linear program of the CSP has a feasible solution, the model space is non-empty and if it does not have a feasible solution, the model space is empty.\\

As is usual for linear constraints, our linear constraints can be thought of as hyper-planes dividing some n-dimensional space in two, with one side constituting the feasible region and the other side the non-feasible region. By extension, for a set of constraints their shared feasible region is the intersection of all of their individual feasible regions. Since the feasible region of a set of constraints is defined by a set of hyper-planes, the feasible region consists of a convex polytope. This fact is used by the simplex algorithm when performing optimization.\\

The simplex algorithm has some requirements to the form of the linear program it is presented, i.e. that it must be in \textit{standard} form. The standard form is a linear program expressed as 
\begin{align*}
    \text{minimize}&\quad \mathbf{c}^T\mathbf{a}\\
    \text{subject to}&\quad M\mathbf{a} = \mathbf{b}\\
    &\quad\mathbf{a} \geq \mathbf{0}
\end{align*}
where $M$ is a matrix representing constraints, $\mathbf{a}$ is a vector of scalars, and $\mathbf{b}$ is a vector of constants. As such, we first need all our constraints to be of the form $a_0 \cdot i_0 + ... + a_n \cdot i_n \leq b$, after which we must convert them into equality constraints by introducing \textit{slack} variables that allow the equality to also take on lower values. Since all of our constraints are normalized and of the form $I \leq 0$, all of our slack variables will have negative coefficients. In our specific case, we let row $i$ of $M$ consist of $(\mathbf{C}^i_\varphi)_{-1}$, where $(\cdot)_{-1}$ removes the first element of the vector (the constant term here). We must also include our slack variables, and so we augment row $i$ of $M$ with the n-vector with all zeroes except at position $i$ where it is $-1$. We let $\mathbf{a}$ be a column vector containing our variables in $\varphi$ as well as our slack variables, and finally we let $\mathbf{b}_i = -(\mathbf{C}^i_\varphi)_1$. $\mathbf{c}$ may be an arbitrary vector.\\

Checking feasibility of the above linear program can itself be formulated as a linear program that is guaranteed to be feasible, enabling us to use efficient polynomial time linear programming algorithms, such as interior point methods, to check whether constraints are covered. Let $\mathbf{s}$ be a new vector, then we have the linear program
%
\begin{align*}
    \text{minimize}&\quad \mathbf{1}^T\mathbf{s}\\
    \text{subject to}&\quad M\mathbf{a} + \mathbf{s} = \mathbf{b}\\
    &\quad\mathbf{a},\mathbf{s} \geq \mathbf{0}
\end{align*}
where $\mathbf{1}$ is the vector of all ones. We can verify the feasibility of this problem with the certificate $(\mathbf{a},\mathbf{s})=(\mathbf{0},\mathbf{b})$. Then the original linear program is feasible if and only if the augmented problem has an optimal solution $(\mathbf{x}^*,\mathbf{s}^*)$ such that $\mathbf{s}^* = \mathbf{0}$.\\

Given a constraint judgement $\varphi;\Phi \vDash C_0$, it should be noted that while the simplex algorithm can be used to check if a solution exists to the constraints $\Phi \cup \{C_0'\}$, there is no guarantee that the solution is an integer solution nor that an integer solution exists at all. Thus, in the case that a non-integer solution exists but no integer solution, this method will over-approximate. An example of such is the two constraints $3i - 1 \leq 0$ and $-2i + 1 \leq 0$ yielding the feasible region where $\frac{1}{3} \leq i \leq \frac{1}{2}$, containing no integers. For an exact solution, we may use integer programming.

% \subsubsection{Conical combinations of constraints}
% We now show how constraints can be conically combined. For convenience, given a finite ordered set of index variables $\varphi = \{i_1, i_2, \dots, i_n\}$, we represent a normalized constraint $I \leq 0$ as a vector $\left( \mathcal{B}(I), I_1\; I_2\; \cdots\; I_n \right)_{\varphi}$. As such, the constraint $-5i + -2j + -4k \leq 0$ can be represented by the vector $\cvect[\varphi_1]{0 {-5} {-2} {-4}}$ where $\varphi_1=\left\{i, j, k\right\}$. Another way to represent that same constraint is with the vector $\cvect[\varphi_2]{0 {-5} {-2} 0 {-4}}$ where $\varphi_2 = \left\{i,j,l,k\right\}$. We denote the vector representation of a constraint $C$ over a finite ordered set of index variables $\varphi$ by $\mathbf{C}_{\varphi}$. We extend this notation to sets of constraints, such that $\Phi_{\varphi}$ denotes the set of vector representations over $\varphi$ of normalized constraints in $\Phi$. Then for a finite ordered set of exponent vectors $\varphi$ and a set of normalized constraints $\Phi$, we can infer any constraint $C$ represented by a vector $\mathbf{C}_\varphi\in \text{coni}(\Phi_\varphi)$ where $\text{coni}(\Phi_\varphi)$ is the \textit{conical hull} of $\Phi_\varphi$. That is, $\text{coni}(\Phi_\varphi)$ is the set of conical combinations with non-negative integer coefficients of vectors in $\Phi_\varphi$
% %
% \begin{align*}
%   \text{coni}(\Phi_\varphi) = \left\{\sum^k_{i=1} a_i {\mathbf{C}^i_\varphi} : {\mathbf{C}^i_\varphi} \in \Phi_\varphi,\; a_i,k \in \mathbb{N}\right\}  
% \end{align*}
% %
% Then, to check if a constraint $C^{new}$ is covered by the set of normalized constraints $\Phi = \{C_1,C_2,\dots, C_n\}$, we can test if $\mathbf{C}^{new}_\varphi$ is a member of the conical hull $\text{coni}(\Phi_\varphi)$. However, by itself, this does not take into account subconstraints of constraints in $\Phi$, as these may not necessarily be written as conical combinations of $\Phi_\varphi$. To account for these, we can include $m=|\varphi|$ vectors of size $m$ of the form $\cvect{-1 0 $\cdots$ 0}, \cvect{0 {-1} 0 $\cdots$ 0}, \dots, \cvect{0 $\cdots$ 0 {-1})}$ in $\Phi_\varphi$. As the conical hull $\text{coni}(\Phi_\varphi)$ is infinite when there exists $\mathbf{C}_\varphi \in \Phi_\varphi$ such that $\mathbf{C}_\varphi \neq \mathbf{0}$ where $\mathbf{0}$ is the vector of all zeroes, when checking for the existence of a conical combination of vectors in $\Phi_\varphi$ equal to $\mathbf{C}^\textit{new}_\varphi$, we can instead solve the following system of linear equations
% %
% \begin{align*}
%     a_1 {\mathbf{C}^1_\varphi}_1 + a_2 {\mathbf{C}^2_\varphi}_1 + \cdots + a_n {\mathbf{C}^n_\varphi}_1 =&\; {\mathbf{C}^{new}_\varphi}_1\\
%     a_1 {\mathbf{C}^1_\varphi}_2 + a_2 {\mathbf{C}^2_\varphi}_2 + \cdots + a_n {\mathbf{C}^n_\varphi}_2 =&\; {\mathbf{C}^{new}_\varphi}_2\\
%     &\!\!\!\vdots\\
%     a_1 {\mathbf{C}^1_\varphi}_m + a_2 {\mathbf{C}^2_\varphi}_m + \cdots + a_n {\mathbf{C}^n_\varphi}_m =&\; {\mathbf{C}^{new}_\varphi}_m
% \end{align*}
% %
% where $a_1,a_2,\dots,a_m\in\mathbb{Z}_{\geq 0}$ are non-negative integer numbers. However, this is an integer programming problem, and so it is NP-hard. We can relax the requirement for $a_1,a_2,\dots,a_m$ to be integers, as the equality relation is preserved under multiplication by any positive real number. We can then view the above system as a linear program, with additional constraints $a_i \geq 0$ for $1 \geq i \geq n$. That is, let $M = \vect{$\mathbf{C}^1_\varphi$ $\mathbf{C}^2_\varphi$ $\cdots$ $\mathbf{C}^n_\varphi$}$ be a matrix with column vectors representing constraints and $\mathbf{a} = \vect{$a_1$ $a_2$ $\cdots$ $a_n$}$ be a row vector of scalars, then checking whether $\mathbf{C}^{new}_\varphi\in\text{coni}(\Phi_\varphi)$ amounts to determining if the following linear program is feasible
% %
% \begin{align*}
%     \text{minimize}&\quad \mathbf{c}^T\mathbf{a}\\
%     \text{subject to}&\quad M\mathbf{a} = \mathbf{C}^{new}_\varphi\\
%     &\quad\mathbf{a} \geq \mathbf{0}
% \end{align*}
% %
% where $\mathbf{c}$ is an arbitrary vector of length $n$ and $\mathbf{0}$ is the vector of all zeroes of length $n$. Checking feasibility of the above linear program can itself be formulated as a linear program that is guaranteed to be feasible, enabling us to use efficient polynomial time linear programming algorithms, such as interior point methods, to check whether constraints are covered. Let $\mathbf{s}$ be a new vector of length $m$, then we have the linear program
% %
% \begin{align*}
%     \text{minimize}&\quad \mathbf{1}^T\mathbf{s}\\
%     \text{subject to}&\quad M\mathbf{a} + \mathbf{s} = \mathbf{C}^{new}_\varphi\\
%     &\quad\mathbf{a},\mathbf{s} \geq \mathbf{0}
% \end{align*}
% where $\mathbf{1}$ is the vector of all ones of length $m$. We can verify the feasibility of this problem with the certificate $(\mathbf{a},\mathbf{s})=(\mathbf{0},\mathbf{C}^{new}_\varphi)$. Then the original linear program is feasible if and only if the augmented problem has an optimal solution $(\mathbf{x}^*,\mathbf{s}^*)$ such that $\mathbf{s}^* = \mathbf{0}$.
% %

\begin{examp}
    Given the constraints
    \begin{align*}
        C^1 &= 3i - 3 \leq 0\\
        C^2 &= j + 2k - 2 \leq 0\\
        C^3 &= -k \leq 0\\
        C^{new} &= i + j - 3 \leq 0
    \end{align*}
    
    we want to check if the constraint judgement $\{i, j, k\};\{C^1, C^2, C^3\} \vDash C^{new}$ is satisfied\\
    
    
    We first let $C^{newinv}$ be the inversion of constraint $C^{new}$.
    \begin{align*}
        C^{newinv} &= 1i + 1j - 2 \geq 0
    \end{align*}
    
    We now want to check if the feasible region $\mathcal{M}_\varphi(\{C^1, C^2, C^3, C^{newinv}\})$ is nonempty. To do so, we construct a linear program with the four constraints. To convert all inequality constraints into equality constraints, we add the slack variables $s_1, s_2, s_3, s_4$ 
    
    \begin{align*}
        \text{minimize}&\quad i + j + k\\
        \text{subject to}&\quad 3i + 0j + 0k + s_1 = 3\\
        &\quad 0i + 1j + 2k + s_2 = 2\\
        &\quad 0i + 0j - 1k + s_3 = 0\\
        &\quad 1i + 1j + 0k - s_4 = 2\\
        &\quad i, j, k, s_1, s_2, s_3, s_4 \geq 0
    \end{align*}
    
    Using an algorithm such as the simplex algorithm, we see that there is no feasible solution, and so we conclude that the constraint judgement $\{i, j, k\};\{C^1, C^2, C^3\} \vDash C^{new}$ is satisfied.
    
    
    %%%%%%%%%%%%%%%%%%%%%%%
    
    % We first represent the four constraints as vectors in terms of some ordered set $\varphi$ of index variables and some ordered set $\varphi$ of exponent vectors.\\
    
    % Let $\varphi = \{i, j, k\}$ and $\varphi = \{\evect{1 0 0}, \evect{0 1 0}, \evect{0 0 1}, \evect{0 0 0}\}$. The constraints $C^1, C^2, C^3, C^{new}$ can now be written as the following vectors
    % %
    % \begin{align*}
    %     \mathbf{C}^1_\varphi &= \cvect{1 0 0 -3}\\
    %     \mathbf{C}^2_\varphi &= \cvect{0 1 1 -2}\\
    %     \mathbf{C}^3_\varphi &= \cvect{0 0 -1 0}\\
    %     \mathbf{C}^{new}_\varphi &= \cvect{2 3 2 -15}
    % \end{align*}
    
    % With the constraints now represented as vectors, we can prepare the equation $M\mathbf{a} = \mathbf{b}$ representing the conical combination, for which we wish to check if a solution exists given given the requirement that $\mathbf{a} \geq \mathbf{0}$. We first prepare the matrix $M$, where we represent the constraint vectors as column vectors
    % %
    % \begin{align*}
    %     &M = \vect{$\mathbf{C}^1_\varphi$ $\mathbf{C}^2_\varphi$ $\mathbf{C}^3_\varphi$ $\bm{\beta}_1$ $\bm{\beta}_2$ $\bm{\beta}_3$ $\bm{\beta}_4$}\\
    %     &\quad \text{where } \bm{\beta}_1 = \cvect{-1 0 0 0}, \bm{\beta}_2 = \cvect{0 {-1} 0 0}, \bm{\beta}_3 = \cvect{0 0 {-1} 0}, \bm{\beta}_4 = \cvect{0 0 0 {-1}}
    % \end{align*}
    % %
    % We include vectors $\bm{\beta}_i, i \in \{1, 2, 3, 4\}$ to ensure we can also use subconstraints of $\mathbf{C}^i, i \in \{1, 2, 3\}$ when checking if we can construct $\mathbf{C}^{new}_\varphi$. To check if a solution exists to the aforementioned equation, we solve the following linear program to check if $\mathbf{s} = \mathbf{0}$
    % \begin{align*}
    %     \text{minimize}&\quad \mathbf{1}^T\mathbf{s}\\
    %     \text{subject to}&\quad M\mathbf{a} + \mathbf{s} = \mathbf{C}^{new}_\varphi\\
    %     &\quad\mathbf{a},\mathbf{s} \geq \mathbf{0}
    % \end{align*}
    
    % This is possible given $\mathbf{a} = \vect{2 3 1 0 0 0 3}$, and so a solution exists to the canonical combination. Notice that we had to use the additional $\bm{\beta}$ vectors when constructing the conical combination. This shows the importance of subconstraints when checking type judgements.
\end{examp}
%
% \section{Soundness}
% %


% \begin{theorem}[Subject reduction]\label{theorem:srbg}
% If $\varphi;\Phi;\Gamma\vdash P \triangleleft K$ and $P \leadsto Q$ then $\varphi;\Phi;\Gamma\vdash Q \triangleleft K'$ with $\varphi;\Phi\vDash k' \leq K$.
% \begin{proof} by induction on the rules defining $\leadsto$.
%     \begin{description}
%     \item[$\runa{R-rep}$]
%     %
%     \item[$\runa{R-comm}$]
%     %
%     \item[$\runa{R-zero}$]
%     %
%     \item[$\runa{R-par}$]
%     %
%     \item[$\runa{R-succ}$]
%     %
%     \item[$\runa{R-empty}$]
%     %
%     \item[$\runa{R-res}$]
%     %
%     \item[$\runa{R-cons}$]
%     %
%     \item[$\runa{R-struct}$]
%     %
%     %\item[$\runa{R-tick}$] We have that $P = \tick{P'}$ and $Q=P'$. Then by $\runa{S-tick}$ we have $\varphi;\Phi;\Gamma\vdash $
%     \end{description}
% \end{proof}
% \end{theorem}

% \begin{lemma}\label{lemma:timeredtype}
% If $\varphi;\Phi;\Gamma\vdash P \triangleleft K$ with $P\!\not\!\leadsto$ and $P \Longrightarrow^{-1} Q$ then $\varphi;\Phi;\downarrow_1\!\Gamma\vdash Q \triangleleft K'$ with $\varphi;\Phi\vDash K' \leq K + 1$.
% \begin{proof}
    
% \end{proof}
% \end{lemma}

% \begin{theorem}\label{theorem:ubbg}
% If $\varphi;\Phi;\Gamma\vdash P \triangleleft K$ and $P \hookrightarrow^n Q$ then $\varphi;\Phi\vDash n \leq K$.
% \begin{proof} by induction on the number of time reductions $n$ in the sequence $P \hookrightarrow^n Q$.
    
% \end{proof}
% \end{theorem}


% % \begin{description}
% %     \item[$\runa{S-nil}$]
% %     %
% %     \item[$\runa{S-tick}$]
% %     %
% %     \item[$\runa{S-nu}$]
% %     %
% %     \item[$\runa{S-nmatch}$]
% %     %
% %     \item[$\runa{S-lmatch}$]
% %     %
% %     \item[$\runa{S-par}$]
% %     %
% %     \item[$\runa{S-iserv}$]
% %     %
% %     \item[$\runa{S-ich}$]
% %     %
% %     \item[$\runa{S-och}$]
% %     %
% %     \item[$\runa{S-oserv}$]
% %     \end{description}


% %
% It is worth noting that the Simplex algorithm does not guarantee an integer solution, and so we may get indices in constraints where the coefficients may be non-integer values. However, we can use the fact that any feasible linear programming problem with rational coefficients also has an (optimal) solution with rational values \cite{keller2016applied}. We use this fact and Lemma \ref{lemma:constraintcommonden} and \ref{lemma:constraintscaling} to show that we need not to worry about the solution given by the Simplex algorithm, given a rational linear programming problem. Definition \ref{def:constraintequivalence} defines what it means for constraints to be equivalent.

% \begin{defi}[Conditional constraint equivalence]\label{def:constraintequivalence}
%     Let $C_1$, $C_2$ and $C\in\Phi$ be linear constraints with integer coefficients and unknowns in $\varphi$. We say that $C_1$ and $C_2$ are equivalent with respect to $\varphi$ and $\Phi$, denoted $C_1 =_{\varphi;\Phi} C_2$, if we have that
%     \begin{equation*}
%     \mathcal{M}_\varphi(\{C_1\} \cup \Phi) = \mathcal{M}_\varphi(\{C_2\} \cup \Phi) %\mathcal{M}_\varphi(\{C_0\})
% \end{equation*}
% where $\mathcal{M}_{\varphi'}(\Phi')=\{\rho : \varphi' \rightarrow \mathbb{N} \mid \rho \vDash C\;\text{for}\; C \in \Phi'\}$ is the model space of a set of constraints $\Phi'$ over a set of index variables $\varphi'$.
%     %
%     %
%     %$\varphi;\Phi\vDash C_1$ if and only if $\varphi;\Phi\vDash C_2$.
%     %Two normalized constraints $C_1$ and $C_2$ are said to be \textit{equivalent} if for any index valuation $\rho$, we have that $\rho \vDash C_1$ if and only if $\rho \vDash C_2$.
% \end{defi}

% \begin{lemma}\label{lemma:constraintscaling}
% Let $I \leq 0$ be a linear constraint with unknowns in $\varphi$. Then $I \leq 0 =_{\varphi;\Phi} n I \leq 0$ for any $n>0$ and set of constraints $\Phi$.
% \begin{proof}
%     This follows from the fact that if $I \leq 0$ is satisfied, then the sign of $I$ must be non-positive, and so the sign of $n I$ must also be non-positive as $n > 0$. Conversely, if $I \leq 0$ is not satisfied, then the sign of $I$, must be positive and so the sign of $n I$ must also be positive.
% \end{proof}
% \end{lemma}

% \begin{lemma}\label{lemma:constraintcommonden}
% Let $I \leq 0$ be a normalized linear constraint with rational coefficients and unknowns in $\varphi$. Then there exists a normalized linear constraint $I' \leq 0$ with integer coefficients and unknowns in $\varphi$ such that $I \leq 0 =_{\varphi;\Phi} I' \leq 0$ for any set of constraints $\Phi$.% there exists an equivalent constraint $I' = \normlinearindex{n'}{I'}$ where $n', I'_{\alpha_1}, \dots,I'_{\alpha_{m}}$ are integers.
% \begin{proof}
%     It is well known that any set of rationals has a common denominator, whose multiplication with any rational in the set yields an integer. One is found by multiplying the denominators of all rationals in the set. As the coefficients of $I$ are non-negative, this common denominator must be positive. By Lemma \ref{lemma:constraintscaling}, we have that $I\leq 0 =_{\varphi;\Phi} n I \leq 0$ where $n$ is a positive number and $\Phi$ is any set of constraints.% the constraint $I \leq 0$ is equivalent to $d I \leq 0$.
% \end{proof}
% \end{lemma}

% %%
% % \begin{lemma}
% % Let $I \leq J$ and $C\in\Phi$ be a linear constraints with integer coefficients and unknowns in $\varphi$. Then $I \leq J =_{\varphi;\Phi} \mathcal{N}(I\leq J)$ if for any subtraction $K - L$ in $I$ or $J$, we have $\varphi;\Phi\vDash L \leq K$. 
% % \begin{proof}
    
% % \end{proof}
% % \end{lemma}
% % %

% % % \begin{lemma}
% % % Let $C$ and $C'\in\Phi$ be normalized linear constraints with integer coefficients and unknowns in $\varphi$. Then $\varphi;\Phi\nvDash C$ if there does not exist $\mathbf{C}^{new}_\varphi\in\text{coni}(\Phi_\varphi \cup \{\mathbf{0}\})$ with $\mathbf{C}_\varphi\leq \mathbf{C}^{new}_\varphi$.
% % % \begin{proof}
    
% % % \end{proof}
% % % \end{lemma}


% % %
% % \begin{theorem}
% % Let $C$ and $C'\in\Phi$ be normalized linear constraints with integer coefficients and unknowns in $\varphi$. Then $\varphi;\Phi\vDash_{\mathbb{R}^{\geq 0}} C$ if and only if there exists $\textbf{C}^{new}_\varphi\in\text{coni}(\Phi_\varphi \cup \{\mathbf{0}\})$ with $\textbf{C}_\varphi\leq \textbf{C}^{new}_\varphi$.
% % \begin{proof}
% %     We consider the implications separately
% %     \begin{enumerate}
% %         \item Assume that $\varphi;\Phi\vDash_{\mathbb{R}^{\geq 0}} C$. Then for all valuations $\rho : \varphi \longrightarrow \mathbb{R}^{\geq 0}$ such that $\rho\vDash \Phi$ we also have $\rho\vDash C$, or equivalently $([\![I_1]\!]_\rho \leq 0) \land \cdots \land ([\![I_n]\!]_\rho \leq 0) \implies [\![I]\!]_\rho \leq 0$, where $C = I_0 \leq 0$ and $C_i = I_i \leq 0$ for $C_i\in \Phi$. We show by contradiction that this implies there exists $\textbf{C}^{new}_\varphi\in\text{coni}(\Phi_\varphi \cup \{\mathbf{0}\})$ with $\textbf{C}_\varphi\leq \textbf{C}^{new}_\varphi$. Assume that such a conical combination does not exist. Then for all $\mathbf{C}'_\varphi\in\text{coni}(\Phi_\varphi \cup \{\mathbf{0}\})$ there is at least one coefficient ${\mathbf{C}'_\varphi}_k$ for some $0\leq k \leq |\varphi|$ such that ${\mathbf{C}'_\varphi}_k < {\mathbf{C}_\varphi}_k$. We show that this implies there exists $\rho\in\mathcal{M}_\varphi(\Phi)$ such that $\rho\nvDash C$.\\ 
        
        
% %         and so there must exist $\rho : \varphi \longrightarrow \mathbb{R}^{\geq 0}$ such that $\rho\vDash C'$ and $\rho\nvDash C$. However, as $\varphi;\Phi\vDash_{\mathbb{R}^{\geq 0}} C$ holds there must be some constraint $C''\in\Phi$ such that $\rho\nvDash C''$.\\
        
        
% %         Assume that there does not exist a conical combination $\textbf{C}^{new}_\varphi\in\text{coni}(\Phi_\varphi \cup \{\mathbf{0}\})$ with $\textbf{C}_\varphi\leq \textbf{C}^{new}_\varphi$. \\
        
        
        
% %         We have that $C = n + \sum_{\alpha\in\mathcal{E}(I)} I_\alpha i_\alpha$, and so for $\rho\vDash n + \sum_{\alpha\in\mathcal{E}(I)} I_\alpha i_\alpha$ to hold, it must be that $[\![\sum_{\alpha\in\mathcal{E}(I)} I_\alpha i_\alpha]\!]_\rho \leq -n$. This implies that $\Phi$ contains constraints that collectively bound the sizes of index variables that appear in $\sum_{\alpha\in\mathcal{E}(I)} I_\alpha i_\alpha$, such that $\sum_{\alpha\in\mathcal{E}(I)} I_\alpha i_\alpha$ cannot exceed $-n$. We now show by contradiction that there must then exist $\textbf{C}^{new}_\varphi\in\text{coni}(\Phi_\varphi \cup \{\mathbf{0}\})$ such that $\textbf{C}_\varphi\leq \textbf{C}^{new}_\varphi$. Assume that such a conical combination does not exist. Then it must be that for any $C'_\varphi\in\text{coni}(\Phi \cup \{\mathbf{0}\})$, at least one coefficient in $C'_\varphi$ is smaller than the corresponding coefficient in $C_\varphi$, implying that $C$ imposes a new restriction on valuations. Thus, there must exist a valuation $\rho$ such that $\rho\vDash\Phi$ but $\rho\nvDash C$, but then we have that $\varphi;\Phi\nvDash C$, and so we have a contradiction.
        
% %         \item Assume that there exists $\mathbf{C}^{new}_\varphi\in\text{coni}(\Phi_\varphi \cup \{\mathbf{0}\})$ with $\mathbf{C}_\varphi \leq \mathbf{C}^{new}_\varphi$. Then by Lemma \ref{TODO}, we have that $\varphi;\phi\vDash C^{new}$ and by Lemma \ref{TODO} it follows from $\mathbf{C}_\varphi \leq \mathbf{C}^{new}_\varphi$ that also $\varphi;\Phi\vDash C$.
% %     \end{enumerate}
% % \end{proof}
% % \end{theorem}
\subsection{Reducing polynomial constraints to linear constraints}\label{sec:verifyingpolynomial}

Many programs do not run in linear time, and so we cannot type them if we are constrained to just verifying linear constraint judgements. In this section we show how we can reduce certain polynomial constraints to linear constraints, enabling us to use the techniques described above. We first extend our definition of indices such that they can be used to express multivariate polynomials. We assume a normal form for polynomial indices akin to that of linear indices. Terms are now monomials with integer coefficients.
%
\begin{align*}
        I,J ::= n \mid i \mid I + J \mid I - J \mid I J
\end{align*}
%
When reducing normalized constraints with polynomial indices to normalized constraints with linear indices, we wish to construct new linear constraints that are only satisfied if the original polynomial constraint is satisfied. For example, given the constraint ${-i^2 + 10 \leq 0}$, one can see that this polynomial constraint can be simulated using the constraint $-i + \sqrt{10} \leq 0$. We notice that the reason this is possible is that when ${-i^2 + 10 \leq 0}$ holds, i.e. when $i \geq \sqrt{10}$, the value of $i$ can always be increased without violating the constraint. Similarly, when ${-i^2 + 10 \geq 0}$ holds, the value of $i$ can always be decreased until reaching its minimum value of 0 without violating the constraint. We can thus introduce a new simpler constraint with the same properties, i.e. $-i + \sqrt{10} \leq 0$. More specifically, the polynomials of the left-hand side of the two constraints share the same positive real-valued roots as well as the same sign for any value of $i$. In general, limiting ourselves to univariate polynomials, for any constraint whose left-hand side polynomial only has a single positive root, we can simulate such a constraint using a constraint of the form $a \cdot i + c \leq 0$. For describing complexities of programs, we expect to mostly encounter monotonic polynomials with at most a single positive real-valued root.\\

Note that the above has the consequence that we may end up with irrational coefficients for indices of constraints. While such constraints are not usually allowed, they can still represent valid bounds for index variables. As such, we can allow them as constraints in this context. Furthermore, any irrational coefficient can be approximated to an arbitrary precision using a rational coefficient.\\

For finding the roots of a specific polynomial we can use either analytical or numerical methods. Using analytical methods has the advantage of being able to determine all roots with exact values, however, we are limited to polynomials of degree at most four as stated by the Abel-Ruffini theorem \cite{abelruffinitheorem}. With numerical methods, we are not limited to polynomials of a specific degree, however, numerical methods often require a given interval to search for a root and do not guarantee to find all roots. Introducing constraints with false restrictions may lead to an under-approximating type system, and so we must be careful not to introduce such. We must therefore ensure we find all roots to avoid constraints with false restrictions. We can use Descartes rule of signs to get an upper bound on the number of positive real roots of a polynomial. Descarte's rule of signs states that the number of roots in a polynomial is at most the number of sign-changes in its sequence of coefficients.\\

For our application, we decide to only consider constraints whose left-hand side polynomial is univariate and monotonic with a single root. In some cases we may remove safely positive monomials in a normalized constraint to obtain such constraints. We limit ourselves to these constraints both to keep complexity down, as well as because we expect to mainly encounter such polynomials when considering complexity analysis of programs. We use Laguerre's method as a numerical method to find the root of the polynomial, which has the advantage that it does not require any specified interval when performing root-finding. Assuming we can find a root $r$, we add an additional constraint $\pm (i - r) \leq 0$ where the sign depends on whether the original polynomial is increasing or decreasing.\\

Additionally, given a non-linear monomial, we may also treat this as a single unit and construct linear combinations of this by treating the monomial as a single fresh variable. For example, given normalized constraints $C_1 = i^2 + 4 \leq 0$, $C_2 = i - 2 \leq 0$, and $C_3 = 2ij \leq 0$, we may view these as the linear constraints $C_1' = k + 4 \leq 0$, $C_2' = i - 2 \leq 0$, and $C_3' = 2l \leq 0$ where $k = i^2$ and $l = ij$. This may make the feasible region larger than it actually is, meaning we over-approximate when verifying judgements on polynomial constraints.

\begin{examp}
    We want to check if the following judgement holds
    $$\{i, j\}; \{-2i \leq 0, -1i^2 + 1j + 1 \leq 0\} \vDash -2i + 2 \leq 0$$
    
    %We notice that $-2i + 2$ cannot be written as a conical combination of the polynomials $-2i$ and $-1i^2 + 1j + 1$.\\
    
    We first try to generate new constraints of the form $-a \cdot i + r \leq 0$ for some index variable $i$ and some constants $a$ and $r$ based on our two existing constraints using the root-finding method. The first constraint is already of such form, so we can only consider the second. For the second, we first use the subconstraint relation to remove the term $1j$ obtaining $-1i^2 + 1 \leq 0$. Next, we note that $-1i^2 + 1$ is a monotonically decreasing polynomial as every coefficient excluding the constant term is negative. We then find the root $r = 1$ of the polynomial and add a new constraint $-1i + 1 \leq 0$ to our set of constraints. Finally, we can invert the constraint $-2i + 2 \leq 0$ obtaining the constraint $-2i + 1 \geq 0$. We now need to check if the feasible region $\mathcal{M}_{\{i, j, i^2\}}(\{-2i \leq 0, -1i^2 + 1j + 1 \leq 0, -1i + 1 \leq 0, -2i + 1 \geq 0\})$ is empty. Treating the monomial $i^2$ as its own separate variable and solving this as a linear program using an algorithm such as the simplex algorithm, we see that there is indeed no solution, and so the constraint is satisfied. 
\end{examp}

\subsection{Trivial judgements}
We now show how some judgements may be verified without neither transforming constraints into linear constraints nor solving any integer programs. To do so, we consider an example provided by Baillot and Ghyselen \cite{BaillotGhyselen2021}, where we exploit the fact that all coefficients in the normalized constraints are non-positive. Judgements with such constraints can be answered in linear time with respect to the number of monomials in the normalized equivalent of constraint $C$. That is if all coefficients in the normalized constraint are non-positive, we can guarantee that the constraint is always satisfied, recalling that only naturals substitute for index variables. Similarly, if there are no negative coefficients and at least one positive coefficient, we can guarantee that the constraint is never satisfied.\\

In practice, it turns out that we can type check many processes by simply over-approximating constraint judgements using pair-wise coefficient inequality constraints. In Example \ref{example:baillotghyssimple}, we show how all constraint judgements in the typings of both a linear and a polynomial time replicated input can be verified using this approach. 
%
\begin{examp}\label{example:baillotghyssimple}
Baillot and Ghyselen \cite{BaillotGhyselen2021} provide an example of how their type system for parallel complexity of message-passing processes can be used to bound the time complexity of a linear, a polynomial and an exponential time replicated input process. We show that we can verify all judgements on constraints in the typings of the first two processes using normalized constraints. We first define the processes $P_1$ and $P_2$
\begin{align*}
    P_i \defeq\; !\inputch{a}{n,r}{}{\tick\match{n}{\asyncoutputch{r}{}{}}{m}{\newvar{r'}{\newvar{r''}{Q_i}}}}
\end{align*}
for the corresponding definitions of $Q_1$ and $Q_2$
\begin{align*}
    Q_1 \defeq&\; \asyncoutputch{a}{m,r'}{} \mid \asyncoutputch{a}{m,r''}{} \mid \inputch{r'}{}{}{\inputch{r''}{}{}{\asyncoutputch{r}{}{}}}\\
    Q_2 \defeq&\; \asyncoutputch{a}{m,r'}{} \mid \inputch{r'}{}{}{(\asyncoutputch{a}{m,r''}{} \mid \asyncoutputch{r}{}{})} \mid \asyncinputch{r''}{}{}
\end{align*}
We type $Q_1$ and $Q_2$ under the respective contexts $\Gamma_1$ and $\Gamma_2$
\begin{align*}
    \Gamma_1 \defeq&\; a : \forall_0 i.\texttt{oserv}^{i+1}(\texttt{Nat}[0,i],\texttt{ch}_{i+1}()), n : \texttt{Nat}[0,i], m : \texttt{Nat}[0,i-1],\\ &\; r : \texttt{ch}_{i}(),
     r' : \texttt{ch}_{i}(), r'' : \texttt{ch}_i()\\
    %
    \Gamma_2 \defeq&\; a : \forall_0 i.\texttt{oserv}^{i^2+3i+2}(\texttt{Nat}[0,i],\texttt{ch}_{i+1}()), n : \texttt{Nat}[0,i], m : \texttt{Nat}[0,i-1],\\ &\; r : \texttt{ch}_{i}(),
     r' : \texttt{ch}_i(), r'' : \texttt{ch}_{2i-1}()
\end{align*}
Note that in the original work, the bound on the complexity of server $a$ in context $\Gamma_2$ is $(i^2+3i+2)/2$. However, we are forced to use a less precise bound, as the multiplicative inverse is not always defined for our view of indices. Upon typing process $P_1$, we amass the judgements on the left-hand side, with corresponding judgements with normalized constraints on the right-hand side
%
\begin{align*}
    &\{i\};\emptyset\vDash i + 1 \geq 1\kern7.5em\Longleftrightarrow &  \{i\};\emptyset\vDash -i \leq 0\\
   % &\{i\};\{i \geq 1\} \vDash i-1 \leq i \kern6em\Longleftrightarrow & \{i\};\{1-i \leq 0\} \vDash -1 \leq 0\\
    &\{i\};\{i \geq 1\} \vDash i \leq i + 1\kern5em\Longleftrightarrow &  \{i\};\{1-i \leq 0\} \vDash -1 \leq 0\\
    %
    &\{i\};\{i \geq 1\} \vDash i \geq i\kern6.7em\Longleftrightarrow &  \{i\};\{1-i \leq 0\} \vDash 0 \leq 0\\
    &\{i\};\{i \geq 1\} \vDash 0 \geq 0\kern6.45em\Longleftrightarrow &  \{i\};\{1-i \leq 0\} \vDash 0 \leq 0
\end{align*}
%
As all coefficients in the normalized constraints are non-positive, each judgement is trivially satisfied, and we can verify the bound $i + 1$ on server $a$. For process $P_2$ we correspondingly have the trivially satisfied judgements
\begin{align*}
    &\{i\};\emptyset\vDash i + 1 \geq 1\kern10.3em\Longleftrightarrow &  \{i\};\emptyset\vDash -i \leq 0\\
    &\{i\};\{0 \leq 0\}\vDash i \leq i^2 + 3i + 2\kern5.2em\Longleftrightarrow &  \{i\};\{0\leq 0\}\vDash -i^2-2i-2 \leq 0\\
    %
    &\{i\};\{i \geq 1\} \vDash i \leq i + 1\kern7.9em\Longleftrightarrow &  \{i\};\{1-i \leq 0\} \vDash -1 \leq 0\\
    %
    &\{i\};\{i \geq 1\} \vDash 2i-1 \leq i^2 + 3i + 2\kern3.2em\Longleftrightarrow &  \{i\};\{1-i \leq 0\} \vDash -i^2-i-3 \leq 0\\
    &\{i\};\{i \geq 1\} \vDash i \geq i\kern9.7em\Longleftrightarrow &  \{i\};\{1-i \leq 0\} \vDash 0 \leq 0\\
    %
    &\{i\};\{i \geq 1\} \vDash 0 \geq i^2+i\kern7.5em\Longleftrightarrow &  \{i\};\{1-i \leq 0\} \vDash -i^2-i \leq 0\\
    %
    &\{i\};\{i \geq 1\} \vDash i^2+2i \geq i^2+3i+2\kern2.9em\Longleftrightarrow &  \{i\};\{1-i \leq 0\} \vDash -i-2 \leq 0
\end{align*}

\end{examp}

% In the general sense, however, verifying whether judgements on polynomial constraints are satisfied is a difficult problem, as it amounts to verifying that a constraint is satisfied under all interpretations that satisfy our set of known constraints. In Example \ref{example:needconic}, we show how a constraint that is not satisfied for all interpretations can be shown to be covered by a set of two constraints, by utilizing the transitive, multiplicative and additive properties of inequalities to combine the two constraints. More specifically, we can exploit the fact that we can generate new constraints from any set of normalized constraints by taking a \textit{conical} combination of their left-hand side indices, as we shall formalize in the following section.
% %
% \begin{examp}\label{example:needconic}
%     Given the judgement
%     \begin{align*}
%         \{i\};\{i \leq 3, 5 \leq i^2\} \vDash 5i \leq 3i^2
%     \end{align*}
%     we want to verify that constraint $5i \leq 3i^2$ is covered by the set of constraints $\{i\leq 3, 5 \leq i^2\}$. This constraint is not satisfied by all interpretations, as substituting $1$ for $i$ yields $5 \not\leq 3$. However, we can rearrange and scale the constraints $i\leq 3$ and $5\leq i^2$ as follows
%     \begin{align*}
%         i \leq 3 \iff i-3\leq 0 \implies 5i - 15 \leq 0\\
%         %
%         5\leq i^2 \iff 0 \leq i^2-5 \implies 0 \leq 3i^2-15
%     \end{align*}
%     Then it follows that
%     \begin{align*}
%         5i-15 \leq 3i^2-15 \iff 5i \leq 3i^2
%     \end{align*}
%     %More generally, we can use the transitive, multiplicative and additive properties of the inequality relation to construct new constraints from a set of known constraints, thereby verifying that some constraint does not impose new restrictions on interpretations. 
% \end{examp}
% %
% \subsection{An alternative method for verifying univariate polynomial constraints}
% In this section we restrict ourselves to constraints whose left-hand side normalized indices are monotonic univariate polynomials. These constraints have a number of convenient properties we can take advantage of to greatly simplify the process of verifying whether a constraint judgement holds. Namely, these constraints perfectly divide the index variable $i$ of the polynomial into two intervals $[-\infty,n[$ and $[n, \infty]$ where for any value of $i$ in either the first or the second interval, the constraint is satisfied and for any $i$ in the other interval, it is not. The only exception to this is for constraints where its left-hand side index is constant, in which case $n = \pm\infty$. As such, we can describe the behavior of a monotonic univariate polynomial constraint using a single value as well as a sign denoting whether the polynomial is increasing or decreasing.\\

% The point $n$ that divides the range that satisfies a constraint from the range that does not satisfy the constraint, corresponds to the root of the left-hand side of the normalized constraint. This is similar to the method described in Section \ref{sec:verifyingpolynomial}, except we now only store the range that satisfies the constraint. Here we take advantage of the fact that the polynomial is monotonic, to ensure that there is only a single root. Verifying whether a constraint $I \leq 0$ covers another constraint $J \leq 0$ then amounts to comparing the roots of $I$ and $J$. This method can be extended to non-monotonic polynomials by simply considering sequences of intervals satisfying constraints and comparing sequences of intervals satisfying a constraint.

% \begin{examp}
%     Given the three constraints
%     %
%     \begin{align*}
%         C_1&: -i^2 + 10 \leq 0\\
%         C_2&: -i + 2 \leq 0\\
%         C_3&: -5 i^3 + 80 i^2 - 427 i + 758 \leq 0
%     \end{align*}
    
%     we want to check if the judgement $\{i\};\{C_1, C_2, C_3\} \vDash -i + 4 \leq 0$ holds. We first find the non-negative roots $r_1$, $r_2$ and $r_3$ of $C_1$, $C_2$ and $C_3$. This can be done either numerically or analytically as all polynomials are of degree $\leq 4$. The roots are $r_1 = \sqrt{10} \approx 3.16$, $r_2 = 2$ and $r_3 \approx 4.59$. The root of $-i + 4$ is $4$, and so we must check if any of the roots $r_1$, $r_2$ and $r_3$ are greater than or equal $4$. In this case, $r_3 \geq 4$, and so the judgement $\{i\};\{C_1, C_2, C_3\} \vDash -i + 4 \leq 0$ holds.
% \end{examp}

%If our indices are univariate polynomials, we can express the feasible region of a constraint as a sequence of disjoint intervals. Then for a constraint $I \leq 0$ such that $I$ is in index variable $i$, the interpretation $I\{n/i\}$ with $n\in\mathbb{N}$ is satisfied when $n$ is within one of the intervals representing the feasible region of $I\leq 0$. We can utilize this to determine whether the feasible region of one constraint contains the feasible region of another by computing their intersection. This can be generalized to judgements on constraints, such that we can verify whether one constraint imposes new restrictions on possible interpretations. Then the question remains \textit{how do we find a sequence of disjoint intervals that corresponds to the feasible region of constraint $I \leq 0$?}\\ 

%We can find such a sequence of intervals for a normalized constraint, by computing the roots of the corresponding index. For polynomials of degree $4$ or less, there exists exact analytical methods to compute the roots, and we can approximate them in the general case using numerical methods.

% \begin{lstlisting}[escapeinside={(*}{*)}]
% intersectIntervals(is1, is2):
%     if is1 or is2 is empty:
%         return empty list
        
%     i1 (*$\longleftarrow$*) head(is1)
%     i2 (*$\longleftarrow$*) head(is2)
%     ires (*$\longleftarrow$*) i1 (*$\cap$*) i2
    
%     if max(i1) > max(i2):
%         is2' (*$\longleftarrow$*) tail(is2)
%         intersectedIntervals (*$\longleftarrow$*) intersectIntervals(is1, is2')
%     else:
%         is1' (*$\longleftarrow$*) tail(is1)
%         intersectedIntervals (*$\longleftarrow$*) intersectIntervals(is1', is2)
    
%     if ires is not the empty interval:
%         add ires to intersectedIntervals as the head
    
%     return intersectedIntervals
% \end{lstlisting}


% \begin{lstlisting}[escapeinside={(*}{*)}]
% containsIntervals(is1, is2):
%     if is2 is empty:
%         return true
%     else if is1 is empty:
%         return false
    
%     i1 (*$\longleftarrow$*) head(is1)
%     i2 (*$\longleftarrow$*) head(is2)
%     ires (*$\longleftarrow$*) i1 (*$\cap$*) i2
    
%     if ires = i2:
%         is2' (*$\longleftarrow$*) tail(is2)
%         return containsIntervals(is1, is2')
%     else:
%         is1' (*$\longleftarrow$*) tail(is1)
%         return containsIntervals(is1', is2)
% \end{lstlisting}


% \begin{lstlisting}[escapeinside={(*}{*)}]
% findIntervals((*$\{i\}$*), (*$I \leq 0$*)):
%     roots (*$\longleftarrow$*) sorted list of all positive roots of (*$I$*)
    
%     if roots is empty:
%         if (*$I\{0/i\}$*) (*$\leq$*) 0:
%         return singleton list of [0, (*$\infty$*)]
%     else:
%         return empty list
    
%     low (*$\longleftarrow$*) 0
%     satisfiedIntervals (*$\longleftarrow$*) empty list

%     if head(roots) = 0:
%         roots (*$\longleftarrow$*) tail(roots)

%     while roots is not empty:
%         high (*$\longleftarrow$*) head(roots)
%         mid (*$\longleftarrow$*) (*$(\texttt{low} + \texttt{high})/2$*)
        
%         if (*$I\{\texttt{mid}/i\}$*) (*$\leq$*) 0:
%             append [low,high] to satisfiedIntervals
            
%         low (*$\longleftarrow$*) high
%         roots (*$\longleftarrow$*) tail(roots)
    
%     if (*$I\{(\texttt{low}+1)/i\}$*) (*$\leq$*) 0:
%         append [low, (*$\infty$*)] to satisfiedIntervals
    
%     return satisfiedIntervals
% \end{lstlisting}


% \begin{lstlisting}[escapeinside={(*}{*)}]
% checkJudgement((*$\{i\}$*), (*$\Phi$*), (*$I\leq 0$*)):
%     satisfiedIntervals (*$\longleftarrow$*) singleton list of [0, (*$\infty$*)]
    
%     for (*$(J \leq 0) \in \Phi$*):
%         isJ (*$\longleftarrow$*) findIntervals((*$\{i\}$*), (*$J\leq 0$*))
%         satisfiedIntervals (*$\longleftarrow$*) intersectIntervals(satisfiedIntervals, isJ)
        
%     isI (*$\longleftarrow$*) findIntervals((*$\{i\}$*), (*$I \leq 0$*))
        
%     return containsInterval(isI, satisfiedIntervals)
% \end{lstlisting}
\section{Examples of invalid configurations}
The following examples are written in the format $\conf{E, a}$, where $E$ is an editor expression and $a$ is the AST on which we apply the editor expression. \\

In equation \ref{condsubproblem} we show how conditioned substitution can cause problems.
\begin{equation}
    \conf{\left(@\texttt{break} \Rightarrow \replace{\texttt{break}}\right) \ggg \texttt{child}\; 1,\; \lambda x.\hole\; \cursor{\breakpoint{c}}} \label{condsubproblem}
\end{equation}
 In the example we check if the cursor is at a breakpoint, and since the check is true we \textit{toggle} the breakpoint thereby making the following \texttt{child} 1 command problematic. The constant c cannot have a child which means this configuration would cause a run-time error. \\
 
In equation \ref{parentproblem} we show how using the \texttt{parent} command can cause problem when the root is unknown.
\begin{equation}
    \conf{\left(\lozenge\texttt{hole} \Rightarrow \texttt{parent}\right) \ggg \texttt{parent},\; \cursor{\lambda x.\hole}\; c} \label{parentproblem}
\end{equation}
In the example we first check if there is a hole in some subtree of the current cursor. This condition holds and we therefore evaluate the \texttt{parent} command resulting in the AST $\cursor{\lambda x.\hole\; c}$. When the next \texttt{parent} command is evaluated we have a run-time error since we are already situated at the root.\\

In equation \ref{astproblem} we show how an editor expression can result in an AST that would cause a run-time error when evaluated.
\begin{equation}
    \conf{\left(\neg\Box(\texttt{lambda}\; x) \Rightarrow \texttt{child}\; 1\right) \ggg \replace{\texttt{var}\; x}.\texttt{eval},\; \cursor{\lambda x.\hole}\; c} \label{astproblem}
\end{equation}
In the example we first check if it is \textbf{not} necessary that the subtree of the cursor contains a lambda expression. This condition does not hold since it is necessary. Since the condition does not hold we do not evaluate the \texttt{child} 1 command, which means the following substitution of \texttt{var} x is problematic. The substitution results in the AST $\cursor{\texttt{var}\; x}\; c$, which causes a run-time error when the command \texttt{eval} is evaluated, since the left child of the function application is no longer a function.
%
\section{Over-approximations}
As we cannot determine statically whether a condition holds, we establish over-approximations to ensure run-time errors cannot occur in well-typed configurations. As equation \ref{parentproblem} shows, conditioned expressions can result in loss of information about the cursor location. As such, we enforce the cursor \textit{depth} in the tree to be the same before and after a conditioned expression. Furthermore, the first cursor movement in a conditioned expression must be a \texttt{child} prefix. As equation \ref{condsubproblem} shows, conditioned substitution also results in loss of information. Thus, we can no longer guarantee that subsequent substitution at a deeper level is well-typed. Similarly, we no longer know of the structure of the subtree, such that we must condition \texttt{child} prefixes.\\

The above discussion leads to the following list of over-approximations:
\begin{itemize}
    \item In conditioned and recursive expressions, the cursor depth must be the same before and after.
    \item In conditioned and recursive expressions, only the subtree encapsulated by the cursor is accessible.
    \item After conditioned substitution, subsequent substitution at a deeper level is no longer valid, and the \texttt{child} prefix command must be conditioned.
\end{itemize}
%
\section{AST type rules}
\begin{table*}[htp]
    \centering
    \begin{align*}
        \runa{t-var} &\; \infrule{\Gamma_a\left(x\right)=\tau}{\Gamma_a \vdash x : \tau}\\
        %
        \runa{t-const} &\; \infrule{}{\Gamma_a \vdash c : b}\\
        %
        \runa{t-app} &\; \infrule{\Gamma_a \vdash a_1 : \tau_1 \rightarrow \tau_2 \quad \Gamma_a \vdash a_2 : \tau_1}{\Gamma_a \vdash a_1\; a_2 : \tau_2}\\
        %
        \runa{t-lambda} &\; \infrule{\Gamma_a\left[x \mapsto \tau_1\right] \vdash a : \tau_2}{\Gamma_a \vdash \lambda x:\tau_1.a : \tau_1 \rightarrow \tau_2} \\
        %
        \runa{t-break} &\; \infrule{\Gamma_a \vdash a : \tau}{\Gamma_a \vdash \breakpoint{a} : \tau} \\
        %
        \runa{t-hole} &\; \infrule{}{\Gamma_a \vdash \left(\hole : \tau\right) : \tau}
        %
    \end{align*}
    \caption{Type rules for abstract syntax trees.}
    \label{tab:typerules}
\end{table*}

%\section{Type context format}
%Here, we propose a format for type contexts of editor expressions. The context of an editor expression could be a triple $\Psi = (\Gamma_a, \tau, \Gamma)$, where $\Gamma_a$ is the type context for the subtree encapsulated by the cursor, $\tau$ is the type of the subtree and $\Gamma$ is a function or map from prefix command types to editor expression contexts. That is, contexts for editor expressions are recursive. Say we have context $(\Gamma_a, \tau, \Gamma)$. Upon a $\texttt{child}\; 1$ prefix, we \textit{look up} $\texttt{one}$ in $\Gamma$. If $\Gamma(\texttt{one}) = undef$, the expression is not well-typed. Otherwise, we evaluate the prefixed expression in the new context $\Gamma(\texttt{one})$.\\

%We construct the initial context based on the AST in the configuration $\conf{E,\; a}$. Upon a substitution prefix, we modify the context, upon a child or parent prefix, we \textit{move} in the context, and upon a conditioned or recursive expression, we set some of the bindings to $undef$: $\Gamma(T)=undef$.\\

%$\Gamma = T_1 : \Psi_1,...,T_n : \Psi_n$ \\
%$\Psi = (\Gamma_a, \tau, \Gamma)$
%Γ = T1 : Ψ1,..,Tn : Ψn
%Ψ = (Γa, τ, Γ)

\section{Experimental type system}

In this section, we introduce a type system for our editor-calculus. For the type system, we introduce the syntactic categories $\tau \in \mathbf{ATyp}$ to denote types of AST nodes, $T \in \mathbf{CTyp}$ to denote \textit{child} types, and p $\in \mathbf{Pth}$ to denote AST paths.
%
\begin{align*}
    \tau ::=&\; b \mid \tau_1 \rightarrow \tau_2 \mid \breakpoint{\tau} \mid \texttt{indet}\\
    T ::=&\; \texttt{one} \mid \texttt{two}\\
    p ::=&\; p\; T \mid \epsilon
\end{align*}

In addition to the basic and arrow types in $\mathbf{ATyp}$, we include a type for breakpoints, $\breakpoint{\tau}$, and a type to denote indeterminate types, \texttt{indet}. We use $\mathbf{Pth}$ to denote paths in an AST by storing a sequence of \textbb{one} and \textbb{two} which denote if the path goes through the first or second child.\\

We define two sets for contexts in our type system. The first context, $\mathbf{ACtx}$, stores type bindings for variables in the AST. The second context, $\mathbf{ECtx}$, stores, for all available paths so far, a pair of an AST context and the type of the node at the end of the path. We use $\Gamma_a \in \mathbf{ACtx}$ and $\Gamma_e \in \mathbf{ECtx}$ as metavariables for the two contexts. To check if a path $p$ is available in a context $\Gamma_e$, we use the notion $\Gamma_e(p) \neq \text{undef}$. $\mathbf{ACtx}$ and $\mathbf{ECtx}$ are thus defined as the following.
%
\begin{align*}
\mathbf{ACtx} &= \mathbf{Var} \rightharpoonup \mathbf{ATyp}\\
\mathbf{ECtx} &= \mathbf{Pth} \rightharpoonup \left(\mathbf{ACtx} \times \mathbf{ATyp}\right)
\end{align*}

To support our type system, we modify the syntax for AST node modifications by including type annotations for application, abstraction and holes. The new syntax thus becomes the following.
%
\begin{align*}
  D ::= \; & \texttt{var}\;x \mid \texttt{const}\;c \mid \texttt{app} : \tau_1 \rightarrow \tau_2, \tau_1 \mid \texttt{lambda}\; x : \tau_1 \rightarrow \tau_2 \mid \texttt{break} \mid \texttt{hole} : \tau
\end{align*}

To support breakpoint types, we introduce the notion of type consistency into our typesystem. The purpose of consistency in our type system is to ensure breakpoints types are consistent with their respective type, as defined below.
%
\begin{definition}{(Type consistency)}
    We define two types $\tau_1, \tau_2$ to be \textit{consistent}, denoted $\tau_1 \sim \tau_2$, by the following rules.
    \begin{align*}
        \runa{cons-1} \hspace{-1cm}
        \infrule{}{\tau \sim \tau} \hspace{-1cm}
        \runa{cons-2} \hspace{-1cm}
        \infrule{}{\breakpoint{\tau} \sim \tau} \hspace{-1cm}
        \runa{cons-3} \hspace{-1cm}
        \infrule{}{\tau \sim \breakpoint{\tau}} \hspace{-1cm}
        \runa{cons-4}
        \infrule{\tau_1 \sim \tau_1' \quad \tau_2 \sim \tau_2'}{(\tau_1 \rightarrow \tau_2) \sim (\tau_1' \rightarrow \tau_2')}
    \end{align*}
\end{definition}


\begin{table*}[htp]
    \centering
    \begin{align*}
        \runa{ctx-split-1}&\; \infrule{}{\emptyset = p \left(\emptyset\; \circ\; \emptyset\right)}\\
        \runa{ctx-split-2}&\; \infrule{\Gamma_e = p \left({\Gamma_e}_1\; \circ\; {\Gamma_e}_2\right)}{\Gamma_e,\; p\; T_1..T_n: (\Gamma_a,\; \tau) = p \left(\left({\Gamma_e}_1,\; p\; T_1..T_n: (\Gamma_a,\; \tau)\right)\; \circ\; {\Gamma_e}_2\right)}\\
        \runa{ctx-split-3}&\; \infrule{p_1 \neq p_2 \quad \Gamma_e = p_2 \left({\Gamma_e}_1\; \circ\; {\Gamma_e}_2\right)}{\Gamma_e,\; p_1\; T_1..T_n: (\Gamma_a,\; \tau) = p_2 \left({\Gamma_e}_1\; \circ\; \left({\Gamma_e}_2,\; p_1\; T_1..T_n: (\Gamma_a,\; \tau)\right)\right)}\\
        %
        \runa{ctx-update-1}&\; \infrule{}{\Gamma_e = \Gamma_e + \emptyset}\\
        \runa{ctx-update-2}&\; \infrule{\Gamma_e = \left({\Gamma_e}_1,\; p: ({\Gamma_a}_2,\; \tau_2)\right) + {\Gamma_e}_2}{\Gamma_e,\; p: ({\Gamma_a}_1,\; \tau_1) = \left({\Gamma_e}_1,\; p: ({\Gamma_a}_2,\; \tau_2)\right) + {\Gamma_e}_2}\\
        \runa{ctx-update-3}&\; \infrule{\Gamma_e = {\Gamma_e}_1 + {\Gamma_e}_2}{\Gamma_e,\; p: (\Gamma_a,\; \tau) = {\Gamma_e}_1 + \left({\Gamma_e}_2,\; p: (\Gamma_a,\; \tau)\right)}
    \end{align*}
    \caption{Context split and context update for editor contexts.}
    \label{tab:context}
\end{table*}
% We define \textit{type contexts}, $\Gamma_e$ in Table \ref{tab:context} as a mapping from a path $p$ to a pair consisting of an AST context $\Gamma_a$ and AST type $\tau$. We denote the $\Gamma_e, p : (\Gamma_a, \tau)$ as the type context equal to the paths not in the domain of map $\Gamma_e$ except for $p$, where $\Gamma_e(p) = (\Gamma_a, \tau)$. For type contexts we introduce the concept of \textit{context splitting} on a path in terms of $\Gamma_e$ maintained through two sub-contexts $\Gamma_{e1}$ and $\Gamma_{e2}$. For this we require a split-operation $\circ$, defined for two sub-contexts on a path as $\Gamma_e = p(\Gamma_{e1}\; \circ \; \Gamma_{e2})$. Notice the empty context is defined with the symbol $\emptyset$ as in \runa{ctx-split-1}. In rule \runa{ctx-split-2} we have that $p$ is in $\Gamma_{e1}$, but not in $\Gamma_{e2}$. Thus, $p$ is not in $\Gamma = \Gamma_{e1}\; \circ \; \Gamma_{e2}$, which is similarly done for the \runa{ctx-split-3} in terms of $\Gamma_{e1}$.\\

Next we introduce the notion of \textit{context updates} to update bindings in a context with new types for the associated path $p$. We use the addition operator $+$, to denote sum-context $\Gamma$ of two compatible type contexts $\Gamma_{e1}$ and $\Gamma_{e2}$. The rules require linear paths to not have bindings exist in another context. Thus, we can only update a context $\Gamma_{e2}$ iff no bindings for a given path is in context $\Gamma_{e1}$. In rule \runa{ctx-update-2} we have bindings in $\Gamma_{e1}$, which means we cannot add bindings to $\Gamma_{e2}$. However, in rule \runa{ctx-update-3} we allow path bindings in $\Gamma_{e2}$ since no such bindings are in context $\Gamma_{e1}$.

% \begin{equation}
%     depth(e) = \left\{
%         \begin{array}{ll}
%             depth(E) + 1            & \quad if e = (\texttt{child}\; n).E \\
%             depth(E) - 1            & \quad if e = \texttt{parent}.E\\
%             depth(E_1) + depth(E_2) & \quad if e = E_1 \ggg E_2\\
%             depth(E)                & \quad if e = \texttt{rec}\; x.E\\
%             depth(E)                & \quad if e = \pi.E\\
%             0                       & \quad otherwise
%         \end{array}
%     \right.
% \end{equation}

\begin{definition}{(Relative cursor depth)}
    We define the function $depth : \mathbf{Edt} \rightarrow \mathbb{Z}$, from the set of atomic editor expression to the set of integers.
    \begin{align*}
    depth((\texttt{child}\; n).E) &= depth(E) + 1 \\
    depth(\texttt{parent}.E) &= depth(E) - 1 \\
    depth(E_1 \ggg E_2) &= depth(E_1) + depth(E_2) \\
    depth(\texttt{rec}\; x.E) &= depth(E) \\
    depth(\pi.E) &= depth(E) \\
    depth(E) &= 0 
\end{align*}
\end{definition}
The $depth$ function statically analyses the structure of an editor expression to determine the relative depth of the cursor after evaluation of the expression. This function is used to make sure the position of the cursor before and after evaluation of an expression is the same. As the function performs a static analysis, we do not consider conditioned subexpressions. Later, in the type rules, we will see why we can safely ignore conditioned subexpressions. \\


% Next we define the function $match : \mathbf{Aam} \times \mathbf{ACtx} \times \mathbf{ATyp} \rightarrow \{tt, f\!\!f\}$. This function returns true if the type of the given AST modification $D$, is equal to the given AST type $\tau$.  
% \begin{align*}
%     match(\texttt{var}\; x,\;\Gamma_a,\;\tau) &= \left\{\begin{matrix}
%  tt & \text{if}\; \Gamma_a(x) = \tau\\ 
%  f\!\!f & \text{otherwise}
% \end{matrix}\right.\\
%     match(\texttt{const}\; c,\;\Gamma_a,\; b) &= tt\\
%     match(\texttt{app} : \tau_1 \rightarrow \tau_2,\; \tau_1,\;\Gamma_a,\; \tau_2) &= tt\\
%     match(\texttt{lambda}\; x : \tau_1 \rightarrow \tau_2,\;\Gamma_a,\; \tau_1 \rightarrow \tau_2) &= tt\\
%     match(\texttt{break},\;\Gamma_a,\; \tau) &= tt\\
%     match(\texttt{hole} : \tau,\;\Gamma_a,\; \tau) &= tt\\
%     match(D,\; \Gamma_a,\; \tau) &= f\!\!f
% \end{align*}

%\begin{equation*}
%    %context : \left(\mathbf{Aam} \times \mathbf{ACtx}\right) \rightharpoonup %\left(\left(\mathbf{Pth} \rightarrow \left(\left(\mathbf{Var} \rightharpoonup %\mathbf{ATyp}\right) \times \mathbf{ATyp}\right)\right) \cup \{error\}\right)
    %context : \left(\mathbf{Aam} \times \mathbf{ACtx} \times \mathbf{Pth} \right) %\rightharpoonup \mathbf{ECtx}
%\end{equation*}
%\begin{align*}
% context(\texttt{const}\; c,\; \Gamma_a,\; p) =&\; \emptyset\\
%  context(\texttt{hole} : \tau,\; \Gamma_a,\; p) =&\; \emptyset\\
%context(\texttt{var}\; x,\; \Gamma_a,\; p) =&\; \emptyset\\
 %context((\texttt{app} : \tau_1 \rightarrow \tau2,\; \tau_1),\; \Gamma_a,\; p) =&\; %\emptyset,\; p\; \texttt{one} : (\Gamma_a,\; \tau_1 \rightarrow \tau_2),\; p\; \texttt{two} : %(\Gamma_a,\; \tau_1)\\
 %context(\texttt{lambda}\; x : \tau_1 \rightarrow \tau_2,\; \Gamma_a,\; p) =&\; \emptyset,\; %p\; \texttt{one} : ((\Gamma_a,\; x : \tau_1),\; \tau_2)
%\end{align*}
%
%

We define functions \textit{limits} and \textit{follows} to analyze which cursor movement is safe given a condition holds. \textit{limits} finds the set of possible AST node modifiers, on which the cursor may sit, given the condition holds. \textit{follows} gives a set of editor type context bindings guaranteed to be safe, given the cursor sits on AST node modifier $D$. Note that the AST type context is empty and that the node type is $\texttt{indet}$, as we cannot determine such information based on a condition. Thus, besides toggling of breakpoints, substitution is not well-typed at path $p$ if $\Gamma_e(p)=(\emptyset,\; \texttt{indet})$. We can combine functions \textit{limits} and \textit{follows} to provide additional bindings to the editor type context of a conditioned expression $\phi \Rightarrow E$. The intersection of \textit{follows} applied to each AST node modifier $D$ in the set $limits(\phi)$ is the set of bindings guaranteed to be safe, given $\phi$ holds.

\theoremstyle{definition}
\begin{definition}{(Condition constraints)}
We define a function $limits: \mathbf{Eed} \rightarrow \mathcal{P}(\mathbf{Aam})$ from the set of conditions to the power set of the set of AST node modifiers. We assume conditions are in conjunctive normal form.
\begin{align*}
    limits(@D)=&\;\{D\}\\
    limits(\neg @D)=&\;\mathbf{Aam}\setminus \{D\}\\
    limits(\lozenge D)=&\;\{D\} \cup \{\texttt{app},\; \texttt{lambda}\; x,\; \texttt{break}\}\\
    limits(\neg \lozenge D)=&\;\mathbf{Aam}\setminus \{D\}\\
    limits(\Box D)=&\;\{D\} \cup \{\texttt{app},\; \texttt{lambda}\; x,\; \texttt{break}\}\\
    limits(\neg \Box D)=&\;\mathbf{Aam}\setminus \{D\}\\
    limits(\phi_1 \land \phi_2)=&\;limits(\phi_1) \cap limits(\phi_2)\\
    limits(\phi_1 \lor \phi_2)=&\;limits(\phi_1) \cup limits(\phi_2)
\end{align*}
\end{definition}


\theoremstyle{definition}
\begin{definition}{(Safe movement)}
We define a function $follows: \mathbf{Aam} \times \mathbf{Pth} \rightarrow \mathcal{P}\left(\mathbf{Pth} \times \left(\mathbf{ACtx} \times \mathbf{ATyp}\right)\right)$ from the set of pairs of AST node modifiers and paths to the power set of editor context bindings.
\begin{align*}
    \textit{follows}(\texttt{var}\; x,\; p)=&\; \emptyset\\
    \textit{follows}(\texttt{const}\; c,\; p)=&\; \emptyset\\
    \textit{follows}(\texttt{app},\; p)=&\; \{p\; \texttt{one} : (\emptyset,\; \texttt{indet}),\; p\; \texttt{two} : (\emptyset,\; \texttt{indet})\}\\
    \textit{follows}(\texttt{lambda}\; x,\; p)=&\; \{p\; \texttt{one} : (\emptyset,\; \texttt{indet})\}\\
    \textit{follows}(\texttt{break},\; p)=&\; \{p\; \texttt{one} : (\emptyset,\; \texttt{indet})\}\\
    \textit{follows}(\texttt{hole},\; p)=&\; \emptyset
\end{align*}
\end{definition}

%
%
We now introduce the type rules for editor expressions. Type rules for substitution are shown in table \ref{tab:typerulesv2sub} and the remaining rules are shown in table \ref{tab:typerulesv2}. The \texttt{child} n prefix is handled by \runa{t-child-1} and \runa{t-child-2}. Here we check that the cursor movement is viable by looking up the new path in $\Gamma_e$. Notice that the remaining editor expression $E$, is evaluated using the new path. The \texttt{parent} prefix is handled similarly in \runa{t-parent} with the exception being that we deconstruct the path instead of building it. When using recursion we require that the depth of the cursor is unchanged after evaluating the expression. We ensure this in \runa{t-rec} with the side condition $depth(E) = 0$. Similarly, \runa{t-cond} utilizes the same side condition to ensure that the cursor is unaffected by whether the condition holds or not. Notice here that evaluation of the conditioned expression is limited by what can follow the condition if it holds, denoted by $\delta$. Sequential composition is handled by the type rule \runa{t-seq}. Here we split the type context into $\Gamma_{e1}$, which contains information about the current subtree, and $\Gamma{e2}$, which contains information about the rest of the tree. This split ensures that the potentially hazardous evaluation of $E_1$ is kept separate from the evaluation of $E_2$.\\

\begin{table*}[htp]
    \centering
    \begin{align*}
        %
        \runa{t-eval} &\; \infrule{p,\; \Gamma_e \vdash E : ok}{p,\; \Gamma_e \vdash \texttt{eval}.E : ok}\\
        %
        \runa{t-child-1}&\; \infrule{\Gamma_e(p\; \texttt{one}) \neq \text{undef} \quad p\; \texttt{one},\; \Gamma_e \vdash E : ok}{p,\; \Gamma_e \vdash \left(\texttt{child}\; 1\right).E : ok}\\
        %
        \runa{t-child-2}&\; \infrule{\Gamma_e(p\; \texttt{two}) \neq \text{undef} \quad p\; \texttt{one},\; \Gamma_e \vdash E : ok}{p,\; \Gamma_e \vdash \left(\texttt{child}\; 2\right).E : ok}\\
        %
        \runa{t-parent}&\; \infrule{\Gamma_e(p) \neq \text{undef} \quad p,\; \Gamma_e \vdash E : ok}{p\; T,\; \Gamma_e \vdash \texttt{parent}.E : ok}\\
        %
        \runa{t-rec} &\; \condinfrule{p,\; \Gamma_e \vdash E : ok}{p,\; \Gamma_e \vdash \texttt{rec} x.E : ok}{\text{if}\; depth(E) = 0}\\
        %
        \runa{t-cond} &\; \condinfrule{p,\; \Gamma_e + \delta \vdash E : ok}{p,\; \Gamma_e \vdash \phi \Rightarrow E : ok}{\begin{align*}
            \text{if}\; &depth(E) = 0\;\\
            \text{and}\; &\delta = \bigcap_{D \in limits(\phi)}follows(D,\; p)\\
        \end{align*}}\\
        %
        \runa{t-seq} &\; \condinfrule{p,\; {\Gamma_e}_1 \vdash E_1 : ok \quad p,\; {\Gamma_e}_2 \vdash E_2 : ok}{p,\; \Gamma_e \vdash E_1 \ggg E_2 : ok}{\text{where}\; \Gamma_e = p\; ({\Gamma_e}_1\; \circ\; {\Gamma_e}_2)}\\
        %
        \runa{t-ref} &\; \infrule{}{p,\;\Gamma_e \vdash x : ok}\\
        %
        \runa{t-nil} &\; \infrule{}{p,\;\Gamma_e \vdash \mathbf{0} : ok}
    \end{align*}
    \caption{Type rules for editor expressions.}
    \label{tab:typerulesv2}
\end{table*}
%
%
Table \ref{tab:typerulesv2sub} shows the type rules for substitution. For substitution to be well-typed, the AST node type $\tau$ in the type context binding associated with the current path $p$ must be consistent with the type of the AST node modifier to be inserted. In \runa{t-sub-var}, we handle the special case where we insert a variable reference $x$. For this to be well-typed, a binding $\Gamma_a(x)=\tau'$ must exist, such that $\consistent{\tau}{\tau'}$. Note that substitution replaces a subtree of the AST. Thus, the bindings in the editor type context with paths starting with $p$ are no longer valid. Therefore, we split the type context on path $p$, such that $\Gamma_e = p\left({\Gamma_e}_1\;\circ\;{\Gamma_e}_2\right)$, and evaluate the prefixed expression $E$ in the type context ${\Gamma_e}_2$. That is, the type context containing all bindings of $\Gamma_e$ not starting with $p$. Note that the binding with path exactly $p$ is in both ${\Gamma_e}_1$ and ${\Gamma_e}_2$, however. We add bindings to ${\Gamma_e}_2$ in rules $\runa{t-sub-app}$ and $\runa{t-sub-abs}$. Particularly, we expand the AST type context upon substitution for an abstraction.\\

We treat substitution of breakpoints differently, as we can either toggle breakpoints on or off. Furthermore, we do not replace the subtree upon substitution for breakpoints. Instead, we must modify the bindings with paths starting with $p$, to either include or remove a $\texttt{one}$. Additionally, we change the type in the binding at the current path $p$ to indicate whether it has a breakpoint. Note that we toggle off the breakpoint if the type is of the form $\breakpoint{\tau}$, and toggle it on otherwise. Thus, the type indicates the structure of the tree.
%
%
\begin{table}
    \begin{flalign*}
        %
        \runa{t-sub-var} &\; \condinfrule{\Gamma_e(p)=(\Gamma_a,\;\tau) \quad \Gamma_a(x) = \tau' \quad \consistent{\tau}{\tau'} \quad p,\;{\Gamma_e}_2 \vdash E : ok}{p,\; \Gamma_e \vdash \replace{\texttt{var}\; x}.E : ok}{\text{where}\; \Gamma_e = p\; ({\Gamma_e}_1\; \circ\; {\Gamma_e}_2)} \\
        %
        \runa{t-sub-const} &\; \condinfrule{\Gamma_e(p)=(\Gamma_a,\;b) \quad p,\;{\Gamma_e}_2 \vdash E : ok}{p,\; \Gamma_e \vdash \replace{\texttt{const}\; c}.E : ok}{\text{where}\; \Gamma_e = p\; ({\Gamma_e}_1\; \circ\; {\Gamma_e}_2)}\\
        %
        \runa{t-sub-app} &\; \condinfrule{\Gamma_e(p)=(\Gamma_a,\; \tau_2') \quad \consistent{\tau_2}{\tau_2'} \quad p,\; \Gamma_e' \vdash E : ok}{p,\; \Gamma_e \vdash \replace{\texttt{app} : \tau_1 \rightarrow \tau_2,\; \tau_1}.E : ok}{\begin{align*}
            &\text{where}\; \Gamma_e = p\; ({\Gamma_e}_1\; \circ\; {\Gamma_e}_2)\;\\
            &\text{and}\; \Gamma_e' = {\Gamma_e}_2,\; p\; \texttt{one} : (\Gamma_a,\; \tau_1 \rightarrow \tau_2),\; p\; \texttt{two} : (\Gamma_a,\; \tau_1)
        \end{align*}}\\
        %
        \runa{t-sub-abs} &\; \condinfrule{\Gamma_e(p)=(\Gamma_a,\; \tau_1' \rightarrow \tau_2') \quad \consistent{\tau_1 \rightarrow \tau_2}{\tau_1' \rightarrow \tau_2'} \quad p,\; \Gamma_e' \vdash E : ok}{p,\; \Gamma_e \vdash \replace{\texttt{lambda}\; x : \tau_1 \rightarrow \tau_2}.E : ok}{\begin{align*}
        &\text{where}\;\Gamma_e = p\; ({\Gamma_e}_1\; \circ\; {\Gamma_e}_2)\\
        &\text{and}\;\Gamma_e' = {\Gamma_e}_2, p\; \texttt{one} : ((\Gamma_a,\; x : \tau_1),\; \tau_2)\end{align*}} \\
        %
        %\runa{t-sub} &\; \infrule{match(D,\; \Gamma_a,\; \tau) = tt \quad p,\;\Gamma_e' \vdash %E : ok}{p,\;\Gamma_e \vdash \replace{D}.E : ok} \\
        %&\text{if}\; D \neq \texttt{break}\\
        %&\text{and}\; \Gamma_e(p)=(\Gamma_a,\;\tau) \\
        %&\text{and}\; \Gamma_e = p\; ({\Gamma_e}_1\; \circ\; {\Gamma_e}_2)\\
        %&\text{and}\; \Gamma_e' = {\Gamma_e}_2 + context(D,\; \Gamma_a)\\
        %
        \runa{t-sub-break-1} &\; \infrule{\Gamma_e(p)=(\Gamma_a,\; \breakpoint{\tau}) \quad p,\; \Gamma_e' \vdash E : ok}{p,\; \Gamma_e \vdash \replace{\texttt{break}} : ok} \\
        &\text{where}\; \Gamma_e = p\; ({\Gamma_e}_1\; \circ\; {\Gamma_e}_2)\\
        &\text{and}\; {\Gamma_e}_1 = \emptyset,\; p\; \texttt{one}\; T_1..T_{n_1} : ({\Gamma_a}_1,\; \tau_1),..,p\; \texttt{one}\; T_1..T_{n_m} : ({\Gamma_a}_m,\; \tau_m)\\
        &\text{and}\; {\Gamma_e}_1' =\emptyset,\; p\; T_1..T_{n_1} : ({\Gamma_a}_1,\; \tau_1),..,p\; T_1..T_{n_m} : ({\Gamma_a}_m,\; \tau_m)\\
        &\text{and}\; \Gamma_e' = \left({\Gamma_e}_2 + {\Gamma_e}_1'\right),\; p : (\Gamma_a,\; \tau)\\
        %
        \runa{t-sub-break-2} &\; \infrule{\Gamma_e(p)=(\Gamma_a,\;\tau)\quad  p,\; \Gamma_e' \vdash E : ok}{p,\; \Gamma_e \vdash \replace{\texttt{break}} : ok} \\
        &\text{where}\; \Gamma_e = p\; ({\Gamma_e}_1\; \circ\; {\Gamma_e}_2)\\
        &\text{and}\; {\Gamma_e}_1 =\emptyset,\; p\; T_1..T_{n_1} : ({\Gamma_a}_1,\; \tau_1),..,p\; T_1..T_{n_m} : ({\Gamma_a}_m,\; \tau_m)\\
        &\text{and}\; {\Gamma_e}_1' = \emptyset,\; p\; \texttt{one}\; T_1..T_{n_1} : ({\Gamma_a}_1,\; \tau_1),..,p\; \texttt{one}\; T_1..T_{n_m} : ({\Gamma_a}_m,\; \tau_m)\\
        &\text{and}\; \Gamma_e' = \left({\Gamma_e}_2 + {\Gamma_e}_1'\right),\; p : (\Gamma_a,\; \breakpoint{\tau})\\
        %
        \runa{t-sub-hole} &\; \condinfrule{\Gamma_e(p)=(\Gamma_a,\;\tau') \quad \consistent{\tau}{\tau'} \quad p,\;{\Gamma_e}_2 \vdash E : ok}{p,\; \Gamma_e \vdash \replace{\texttt{hole} : \tau}.E : ok}{\text{where}\; \Gamma_e = p\; ({\Gamma_e}_1\; \circ\; {\Gamma_e}_2)}
        %
    \end{flalign*}
    \caption{Type rules for substitution.}
    \label{tab:typerulesv2sub}
\end{table}

%\begin{table*}[htp]
%    \centering
%    \begin{align*}
        %%
        %\runa{t-eval} &\; \infrule{p,\; \Gamma_e \vdash E : ok \dashv p',\; \Gamma_e'}{p,\; \Gamma_e \vdash \texttt{eval}.E : %ok \dashv p',\; \Gamma_e'}\\
        %%
        %\runa{t-sub} &\; \infrule{T=\tau \quad p,\;\Gamma_e'' \vdash E : ok \dashv p',\;\Gamma_e'}{p,\;\Gamma_e \vdash %\replace{D}.E : ok \dashv p',\;\Gamma_e'} \\
        %&\text{where}\; \Gamma_e(p)=(\Gamma_a,\;\tau) \\
        %&\text{and}\; T = type(D,\;\Gamma_a) \\
        %&\text{and}\; \Gamma_e = p\; ({\Gamma_e}_1\; \circ\; {\Gamma_e}_2)\\
        %&\text{and}\; \Gamma_e'' = {\Gamma_e}_1 + context(D,\; \Gamma_a)\\
        %%
        %\runa{t-child-1}&\; \infrule{\Gamma_e(p\; \texttt{one}) \neq undef \quad p,\; \texttt{one},\; \Gamma_e \vdash E : ok %\dashv p',\; \Gamma_e'}{p,\; \Gamma_e \vdash \left(\texttt{child}\; 1\right).E : ok \dashv p',\; \Gamma_e'}\\
        %%
        %\runa{t-child-2}&\; \infrule{\Gamma_e(p\; \texttt{two}) \neq undef \quad p,\; \texttt{one},\; \Gamma_e \vdash E : ok %\dashv p',\; \Gamma_e'}{p,\; \Gamma_e \vdash \left(\texttt{child}\; 2\right).E : ok \dashv p',\; \Gamma_e'}\\
        %%
        %\runa{t-parent}&\; \infrule{\Gamma_e(p) \neq undef \quad p,\; \Gamma_e \vdash E : ok \dashv p',\; \Gamma_e'}{p\; T,\; %\Gamma_e \vdash \texttt{parent}.E : ok \dashv p',\; \Gamma_e'}\\
        %%
        %\runa{t-rec} &\; \condinfrule{p,\; {\Gamma_e}_1 \vdash E : ok \dashv p,\; \Gamma_e'}{p,\; \Gamma_e \vdash \texttt{rec} %x.E : ok \dashv p,\; {\Gamma_e}_2}{\text{where}\; \Gamma_e = p\; ({\Gamma_e}_1\; \circ\; {\Gamma_e}_2)}\\
        %%
        %\runa{t-seq} &\; \infrule{p,\; \Gamma_e \vdash E_1 : ok \dashv p'',\; \Gamma_e'' \quad p'',\; \Gamma_e'' \vdash E_2 : %ok \dashv p',\; \Gamma_e'}{p,\; \Gamma_e \vdash E_1 \ggg E_2 : ok \dashv p',\; \Gamma_e'}\\
        %%
        %\runa{t-cond} &\; \infrule{p,\; {\Gamma_e}_1 + \delta \vdash E : ok \dashv p,\; \Gamma_e'}{p,\; \Gamma_e \vdash \phi %\Rightarrow E : ok \dashv p,\; {\Gamma_e}_2}\\
%        &\text{where}\; \Gamma_e = p\; ({\Gamma_e}_1\; \circ\; {\Gamma_e}_2)\\
%        &\text{and}\; \delta = \bigcap_{D \in limits(\phi)}follows(D)\\
%        %
%        \runa{t-ref} &\; \infrule{}{p,\;\Gamma_e \vdash x : ok \dashv p,\;\Gamma_e}\\
%        %
%        \runa{t-nil} &\; \infrule{}{p,\;\Gamma_e \vdash \mathbf{0} : ok \dashv p,\;\Gamma_e}\\
%    \end{align*}
%    \caption{Type rules for editor expressions.}
%    \label{tab:typerules}
%\end{table*}

\begin{theorem} (Subject reduction)
If $\Gamma_e, \;\Gamma_a \vdash \conf{E,\;a} : ok$ and $\conf{E, a} \xrightarrow{\alpha} \conf{E', a'}$ then $\Gamma_e, \;\Gamma_a \vdash \conf{E',\;a'} : ok$.
\end{theorem}

We define \textit{well-typedness} of a configuration $\conf{E,\;a}$ by the following rule: \\
$\condinfrule{\Gamma_a \vdash a : \tau \quad p,\; \Gamma_e \vdash E : ok}{\Gamma_e, \;\Gamma_a \vdash \conf{E,\;a} : ok}{\begin{align*}
        &\text{where}\;\\
        &\text{and}\;\end{align*}}$
        
        

\chapter{Type checking sized types for parallel complexity}\label{ch:typecheck}
As mentioned in Chapter \ref{ch:bgts}, Baillot and Ghyselen \cite{BaillotGhyselen2021} bound sizes of algebraic terms and synchronizations on channels using indices, leading to a partial order on indices. For instance, for a process of the form $\inputch{a}{v}{}{\asyncoutputch{b}{v}{}} \mid P$ (assuming synchronizations induce a cost in time complexity of one) we must enforce that the bound on $a$ is strictly smaller than the bound on $b$. Thus, we must impose constraints on the interpretations of indices. Another concern in the typing of the process above is the parallel complexity. Granted separate bounds on the complexities of $\inputch{a}{v}{}{\asyncoutputch{b}{v}{}}$ and $P$, how do we establish a tight bound on their parallel composition? This turns out to be another major challenge, as bounds may be parametric, such that comparison of bounds is a partial order. Finally, for a process of the form $!\inputch{a}{v}{}{P} \mid \asyncoutputch{a}{e}{}$ we must \textit{instantiate} the parametric complexity of $!\inputch{a}{v}{}{P}$ based on the deducible size bounds of $e$, which quickly becomes difficult as indices become more complex.\\ % As the type system is otherwise fairly standard, for instance using input/output types for channels, the challenge in introducing type check is to ensure constraints on indices are not violated.\\
%
%Type inference for the type system introduced in Baillot and Ghyselen \cite{BaillotGhyselen2021} is complicated by similar challenges to that of type checking, such as constraint satisfaction. Another concern with respect to sized types is that we must infer indices. Here, it is relevant to consider existing work on sized type inference, such as Hughes et al. \cite{HughesEtAl1996} and Avanzini and Dal Lago \cite{AvanziniLago2017}. The set of function symbols used to form indices should be be more strictly defined, to make inference tractable. We also must be careful with respect to recursion, predominantly with how primitive recursion can be identified. In this chapter, we address some of these challenges.
%
% The type system for parallel complexity of message-passing processes introduced in Baillot and Ghyselen \cite{BaillotGhyselen2021} uses sized types to express parametric complexity of invoking replicated inputs, and thereby achieve precise bounds on primitively recursive processes. This requires a notion of polymorphism in the message types of replicated inputs. Baillot and Ghyselen introduce size polymorphism by bounding sizes of algebraic terms and synchronizations on channels with algebraic expressions referred to as indices that may contain index variables representing unknown sizes. We may interpret an index with an index valuation that maps its index variables to naturals, such that the index may be evaluated.\\ %
%
% The bounds on sizes and synchronizations lead to a partial order on indices. For instance, for a process of the form $\inputch{a}{v}{}{\asyncoutputch{b}{v}{}} \mid P$ (assuming synchronizations induce a cost in time complexity of one) we must enforce that the bound on $a$ is strictly smaller than the bound on $b$. Thus, we must induce constraints on the interpretations of indices. As the type system is otherwise fairly standard, for instance using input/output types for channels, the challenge in introducing type check is to ensure constraints on indices are not violated.

The purpose of this section is to present a version of the type system by Baillot and Ghyselen that is algorithmic in the sense that its type rules can be easily implemented in a programming language, and so we must address the challenges described above. For the type checker, we assume we are given a set of constraints $\Phi$ on index variables in $\varphi$ and a type environment $\Gamma$. We first present the types of the type system as well as subtyping. Afterwards, we present auxiliary functions and type rules. For the type rules we also present the concept of combined complexities that we use to bound parallel complexities by deferring comparisons of indices when these are not defined. We then prove the soundness of the type checker and show how it can be extended accompanied by examples. Finally, we show how we can verify the constraint judgements that show up in the type rules.

\section{Auxiliary functions}
We first present two functions that will be used in the type rules. As the continuation of a replicated input may be invoked an arbitrary number of times at different time steps, we need to ensure that names used within the continuation are of types that are invariant to time as defined in Definition \ref{def:timeinvariance}, i.e. they may be used at any time step. In Definition \ref{def:readyfunc}, we define a function $\text{ready}(\varphi,\Phi,T)$ that discards use-capabilities from a type to obtain time invariance. For a server type $\forall_I\widetilde{i}.\texttt{serv}^\sigma_K(\widetilde{T)}$, outputs are well-typed whenever $\varphi;\Phi\vDash I \leq 0$, and so for names of such types, we only discard input capabilities, whenever we can guarantee the constraint judgement $\varphi;\Phi\vDash I \leq 0$. We return to how to guarantee constraint judgements in section \ref{sec:verifyinglinearjudgements}.
%
\begin{defi}\label{def:readyfunc}
We inductively define a function \textit{ready} that transforms a type into one that is time invariant.
\begin{align*}
    %\text{ready}(\varphi,\Phi,\epsilon) =&\; \epsilon\\
    %
    \text{ready}(\varphi,\Phi,\natinterval{I}{J}) =&\; \natinterval{I}{J}\\
    %
    \text{ready}(\varphi,\Phi,\forall_I\widetilde{i}.\texttt{serv}^{\sigma}_K(\widetilde{T})) =&\; \left\{ \begin{matrix}
        \forall_I\widetilde{i}.\texttt{serv}^{\sigma \cap \{\texttt{out}\}}_K(\widetilde{T}) & \text{if}\; \varphi;\Phi\vDash I \leq 0\\
        \forall_0\widetilde{i}.\texttt{serv}^{\emptyset}_K(\widetilde{T}) & \text{if}\; \varphi;\Phi\nvDash I \leq 0
    \end{matrix} \right.\\
    %
    %\text{ready}(\varphi,\Phi,\Gamma,a:\oservS{I}{\widetilde{i}}{K}{\widetilde{T}}) =&\; \left\{ \begin{matrix}
    %    \text{ready}(\varphi,\Phi,\Gamma), a:\oservS{I}{\widetilde{i}}{K}{\widetilde{T}} & \text{if}\; %\varphi;\Phi\vDash I \leq 0\\
    %    \text{ready}(\varphi,\Phi,\Gamma) & \text{if}\; \varphi;\Phi\nvDash I \leq 0
    %\end{matrix} \right.\\
    %%
    %\text{ready}(\varphi,\Phi,\Gamma,a:\iservS{I}{\widetilde{i}}{K}{\widetilde{T}}) =&\; %\text{ready}(\varphi,\Phi,\Gamma)\\
    %
    \text{ready}(\varphi,\Phi,\texttt{ch}^\sigma_I(\widetilde{T})) =&\;\texttt{ch}^\emptyset_0(\widetilde{T})%\\
    %
    %\text{ready}(\varphi,\Phi,\Gamma,a:\outchanneltypeS{I}{\widetilde{T}}) =&\; \text{ready}(\varphi,\Phi,\Gamma)\\
    %
    %\text{ready}(\varphi,\Phi,\Gamma,a:\inchanneltypeS{I}{\widetilde{T}}) =&\; \text{ready}(\varphi,\Phi,\Gamma)
\end{align*}
We extend \textit{ready} to type contexts such that for $v\in\texttt{dom}(\Gamma)$ we have that $\text{ready}(\varphi,\Phi,\Gamma)(v)=\text{ready}(\varphi,\Phi,\Gamma(v))$.
\end{defi}

% In Definition \ref{def:joinbase}, we introduce a binary function on base types $\uplus_{\varphi;\Phi}$ that computes a base type that is a subtype of both argument types, if such a base type exists. We do this by selecting the least lower bound and the greatest upper bound amongst the two argument base types as the new size bounds. This function will be useful for typing list expressions, as the elements of a list may be typed with different size bounds that we will need a common subtype of.

% \begin{defi}[Joining base types]\label{def:joinbase}

% \begin{align*}
%     \texttt{Nat}[I,J] \uplus_{\varphi;\Phi} \texttt{Nat}[I',J'] =&\; \left\{
%     \begin{matrix}
%         \texttt{Nat}[I,J] & \varphi;\Phi\vDash I \leq I'\;\text{and};\varphi;\Phi\vDash J' \leq J\\
%         \texttt{Nat}[I',J] & \varphi;\Phi\vDash I' \leq I\;\text{and};\varphi;\Phi\vDash J' \leq J\\
%         \texttt{Nat}[I,J'] & \varphi;\Phi\vDash I \leq I'\;\text{and};\varphi;\Phi\vDash J \leq J'\\
%         \texttt{Nat}[I',J'] & \varphi;\Phi\vDash I' \leq I\;\text{and};\varphi;\Phi\vDash J \leq J'
%     \end{matrix}
%     \right.\\
    
%     \texttt{List}[I,J](\mathcal{B}) \uplus_{\varphi;\Phi} \texttt{List}[I',J'](\mathcal{B}') =&\; \left\{
%     \begin{matrix}
%         \texttt{List}[I,J](\mathcal{B} \uplus_{\varphi;\Phi} \mathcal{B}') & \varphi;\Phi\vDash I \leq I'\;\text{and};\varphi;\Phi\vDash J' \leq J\\
%         \texttt{List}[I',J](\mathcal{B} \uplus_{\varphi;\Phi} \mathcal{B}') & \varphi;\Phi\vDash I' \leq I\;\text{and};\varphi;\Phi\vDash J' \leq J\\
%         \texttt{List}[I,J'](\mathcal{B} \uplus_{\varphi;\Phi} \mathcal{B}') & \varphi;\Phi\vDash I \leq I'\;\text{and};\varphi;\Phi\vDash J \leq J'\\
%         \texttt{List}[I',J'](\mathcal{B} \uplus_{\varphi;\Phi} \mathcal{B}') & \varphi;\Phi\vDash I' \leq I\;\text{and};\varphi;\Phi\vDash J \leq J'
%     \end{matrix}
%     \right.
% \end{align*}
% \end{defi}

% \begin{defi}[Removing servers]
%     Given a type context $\Gamma$, the function \textit{noserv} removes all server types from the context.
%     \begin{align*}
%         \text{noserv}(\emptyset) &= \emptyset\\
%         \text{noserv}(\Gamma, \natinterval{I}{J}) &= \text{noserv}(\Gamma),\natinterval{I}{J}\\
%         \text{noserv}(\Gamma, \chant{I}{\sigma}{\widetilde{T}}) &= \text{noserv}(\Gamma),\chant{I}{\sigma}{\widetilde{T}})\\
%         \text{noserv}(\Gamma, \servt{I}{\widetilde{i}}{\sigma}{K}{\widetilde{T}}) &= \text{noserv}(\Gamma)\\
%     \end{align*}
% \end{defi}


In Definition \ref{def:instantiatef}, we introduce a function $\text{instantiate}(\widetilde{i},\widetilde{T})$ that assigns the index variables in sequence $\widetilde{i}$ to indices in types of the sequence $\widetilde{T}$, by means of a substitution of indices for index variables. Note that the function is only defined for sequences such that the number of index variables equals the number of indices in the types. This function will be useful for outputs on servers, where the parametric types $\widetilde{S}$ of a server type $\forall_I\widetilde{i}.\texttt{serv}^{\{\texttt{out}\}}_K(\widetilde{S)}$ must match the types of expressions $\widetilde{T}$ to be output. More specifically, there must exist a substitution $\{\widetilde{J}/\widetilde{i}\}$ such that $\widetilde{T} \sqsubseteq \widetilde{S}\{\widetilde{J}/\widetilde{i}\}$. We return to this in Section \ref{section:typeruless}.

\begin{defi}[Server invocation]\label{def:instantiatef}
We inductively define a function \textit{instantiate} that constructs a substitution of indices for index variables, provided a sequence of index variables and a sequence of types%. The function is only defined for sequences such that the number of index variables equals the number of indices in the types.
\begin{align*}
    \text{instantiate}(\epsilon,\epsilon) =&\; \{\}\\
    \text{instantiate}((\widetilde{i},\widetilde{j}),(T,\widetilde{S})) =&\; \text{instantiate}(\widetilde{i},T),\text{instantiate}(\widetilde{j},\widetilde{S})\\
    \text{instantiate}((i,j),\texttt{Nat}[I,J]) =&\; \{I/i,J/j\}\\
    %\text{instantiate}((i,j,\widetilde{k}),\texttt{List}[I,J](\mathcal{B})) =&\; \{I/i,J/j\}, \text{instantiate}(\widetilde{k},\mathcal{B})\\
    \text{instantiate}((i,\widetilde{j}),\texttt{ch}_I^\sigma(\widetilde{T})) =&\; \{I/i\},\text{instantiate}(\widetilde{j},\widetilde{T})\\
    \text{instantiate}((i,j,\widetilde{k}),\forall_I\widetilde{l}.\texttt{serv}^\sigma_K(\widetilde{T})) =&\; \{I/i,K/j\},\text{instantiate}(\widetilde{k},\widetilde{T})
\end{align*}
\end{defi}
\section{Algorithmic type rules}\label{section:typeruless}
%We are now ready to introduce type rules for a type checker of the type system in Baillot and Ghyselen \cite{BaillotGhyselen2021}. %TODO 
%
%
%A piecewise complexity $\kappa$ is a set of pairs $(\Phi_i, K_i)$ where $K_i$ is an index describing a complexity that is valid within the feasible region described by the set of constraints $\Phi_i$. As such, a piecewise complexity $\kappa = \{(\Phi_1, K_1), \cdots, (\Phi_n, K_n)\}$ describes a complexity bound within the feasible region $\mathcal{M}_\varphi(\Phi_1) \cup \cdots \cup \mathcal{M}_\varphi(\Phi_n)$ for some $\varphi$ such that $\Phi_1, \cdots, \Phi_n$ use index variables in $\varphi$. In the case where $m$ feasible regions $\mathcal{M}_\varphi(\Phi_{i_1})$, $\mathcal{M}_\varphi(\Phi_{i_{m-1}})$ and $\mathcal{M}_\varphi(\Phi_{i_m})$ intersect, we choose the maximal complexity of the corresponding complexities for any valuation $\rho$ in the intersecting region.\\

When typing a process, we often need to find an index that is an upper bound on two other indices, for which there may be many options. To allow for the type checker to be as precise as possible, we want to find the minimum complexity that is a bound of two other complexities, which will depend on the representation of complexity, and as such, instead of representing complexity bounds using indices, we opt to use sets of indices which we refer to as \textit{combined complexities}. Intuitively, given any point in the space spanned by some index variables, the combined complexity at that point is the maximum of the complexities making up the combined complexity at that point. This is illustrated in Figure \ref{fig:combined_complexity} which shows a combined complexity consisting of three indices. The red dashed line represents the bound on the combined complexity. Representing complexities as sets of indices has the effect of \textit{externalizing} the process of finding bounds of complexities by deferring this until a later time. We will later define the algorithm \textit{basis} that removes superfluous indices of a combined complexity. In Figure \ref{fig:combined_complexity} the index $K$ is superfluous as it never contributes to the bound of the combined complexity.

\begin{figure}
    \centering
    \begin{tikzpicture}
\begin{axis}[
    axis lines = left,
    xlabel = \(i\),
    ylabel = {},
    domain = 0:2.5,
    xtick={\empty},ytick={\empty},
    ymin=0,
    ymax=5.2,
    xmax=2.7,
    restrict y to domain=0:5,
]
    \addplot[thick, color=orange]{x^2} node[above,pos=1] {I};
    \addplot[thick, color=blue]{x+1} node[above,pos=1] {J};
    \addplot[thick, color=green]{ln(x+1)*2} node[above,pos=1] {K};
    \draw [ultra thick, dashed, draw=red] (axis cs:0,1) -- (axis cs:1.62,2.62);
    \addplot[ultra thick, color=green, dashed, color=red, domain=1.62:2.5]{x^2};
\end{axis}
\end{tikzpicture}
    \caption{Combined complexities illustrated. The combined complexity consists of the three indices I, J, K of the single index variable $i$. The dashed red line shows the bound of the combined complexity. $K$ is a superfluous index in the combined complexity as it never contributes to the bound of the combined complexity.}
    \label{fig:combined_complexity}
\end{figure}


\begin{defi}[Combined complexity]\label{def:combinedcomp} 
    We refer to a set $\kappa$ of complexities as a \textit{combined complexity}. We extend constraint judgements to include combined complexities such that
    \begin{enumerate}
        \item $\varphi;\Phi\vDash \kappa \leq \kappa'$ if for all $K \in \kappa$ there exists $K'\in \kappa'$ such that $\varphi;\Phi\vDash K \leq K'$.
        % 
        \item $\varphi;\Phi\vDash \kappa = \kappa'$ if $\varphi;\Phi\vDash \kappa \leq \kappa'$ and $\varphi;\Phi\vDash \kappa' \leq \kappa$.
        \item $\kappa + I = \{K + I \mid K \in \kappa\}$.
        %
        \item $\kappa\{J/i\} = \{ K\{J/i\} \mid K\in\kappa \}$.
    \end{enumerate}
    In the above, we may substitute an index for a combined complexity. In such judgements, the index represents a singleton set. For instance, $\varphi;\Phi\vDash \kappa \leq K$ represents $\varphi;\Phi\vDash \kappa \leq \{K\}$.
    %$\varphi;\Phi \vDash \kappa \bowtie \kappa' \quad\text{ if }\quad \forall K \in \kappa. (\exists K' \in \kappa'. \varphi;\Phi \vDash K \bowtie K')$.
\end{defi}

More specifically, when considering a combined complexity $\kappa$, we are interested in the maximal complexity given some valuation $\rho$, which we find by simply comparing the different values for the complexities within $\kappa$ given $\rho$. Note that the complexity $K \in \kappa$ that is maximal may be different for different valuations. In Definition \ref{def:combinedcomp} we extend the binary relations in $\bowtie$ on indices to combined complexities, such that we can compare two combined complexities such as $\varphi;\Phi \vDash \kappa \bowtie \kappa'$ and a combined complexity and complexity such as $\varphi;\Phi \vDash \kappa \bowtie K$. Definition \ref{def:combinedcompbasis} defines the function \textit{basis} that discards any $K \in \kappa$ that can never be the maximal complexity given some set of constraints $\Phi$ (i.e. the complexities that are bounded by other complexities in the set), such that we can always keep the number of complexities in a combined complexity to a minimum. %Finally, we may also be interested in adding an index onto a combined complexity, and so we define the addition of indices onto combined complexities in Definition \ref{def:combinedcompadd}.
%
\begin{defi}\label{def:combinedcompbasis}
    We define the function \textit{basis} that takes a set of index variables $\varphi$, a set of constraints $\Phi$ and a combined complexity $\kappa$, and returns a new combined complexity without superfluous complexities (The \textit{basis} of $\kappa$)
    \begin{align*}
        \text{basis}(\varphi,\Phi,\kappa) = \bigcap\left\{ \kappa' \subseteq \kappa \mid \forall K\in\kappa.\exists K'\in\kappa'.\varphi;\Phi\vDash K \leq K' \right\}
    \end{align*}
    Moreover, the algorithm below computes the basis
    % \begin{align*}
    %     \text{basis}(\varphi, \kappa) = \{(\Phi, K) \in \kappa \mid \varphi;\Phi \not \vDash K < K' \text{ for all } (\Phi', K') \in \kappa\}
    % \end{align*}
    \begin{align*}
        &\text{basis}(\varphi, \Phi, \kappa) = \text{do}\\[-0.5em]
        &\quad \kappa' \leftarrow \kappa\\[-0.5em]
        &\quad \text{for } K \in \kappa \text{ do}\\[-0.5em]
        &\quad\quad \text{ if } \exists K' \in \kappa' \text{ with } K \not = K' \text{ and } \varphi;\Phi \vDash K \leq K' \text{ then}\\[-0.5em]
        &\quad\quad\quad \kappa' \leftarrow \kappa' \setminus \{K\}\\[-0.5em]
        &\quad \text{return } \kappa'
    \end{align*}
\end{defi}
%
% \begin{defi}[]\label{def:combinedcompadd}
%     We define the the addition of a combined complexity and index as
%     \begin{align*}
%         \kappa + I = \{K + I \mid K \in \kappa\}
%     \end{align*}

% \end{defi}
%
For typing expressions, we use the rules presented in Table \ref{tab:sizedtypedexpressiontypes}, excluding the rule $\runa{BG-sub}$. In Table \ref{tab:sizedprocesstypingrules} we show the type rules for processes. Type judgements are of the form $\varphi;\Phi;\Gamma \vdash P \triangleleft \kappa$ where $\kappa$ denotes the complexity of process $P$. The rule $\runa{S-tick}$ types a \texttt{tick} prefix and incurs a cost of one in time complexity. We advance the time of all types in the context accordingly when typing the continuation. Rule $\runa{S-annot}$ is similar but may incur a cost of $n$. Matches on naturals are typed with rule $\runa{S-nmatch}$. Most notably, we extend the set of known constraints when typing the two continuations. That is, we can deduce constraints on the lower and upper bounds on the size of the expression we match on. For instance, for the zero pattern we can deduce that the lower bound $I$ must be equal to $0$ (or equivalently $I \leq 0$), and for the successor pattern, we can guarantee that the upper bound $J$ must be greater than or equal to $1$. For the complexity of pattern matches and parallel composition, we take advantage of the fact that we represent complexities using combined complexities. As such, we include complexities in both $P$ and $Q$ in the result. To remove redundancy from the set $\kappa \cup \kappa'$, we use the basis function.\\

%
% \begin{table*}[!ht]
%     \begin{framed}\vspace{-1em}\begin{align*}
%         &\kern15em\\[-2em] % Stretch frame
%         &\kern0em\runa{S-nil}\infrule{}{\varphi;\Phi;\Gamma \vdash \withcomplex{\nil}{0}} \kern1em\runa{S-tick}\;\infrule{\varphi;\Phi;\susumesim{\Gamma}{1}\vdash P \triangleleft K}{\varphi;\Phi;\Gamma\vdash \tick P \triangleleft K + 1} \kern3em\runa{S-nu}\;\infrule{\varphi;\Phi;\Gamma,\withtype{a}{T} \vdash \withcomplex{P}{K}}{\varphi;\Phi;\Gamma \vdash \newvar{a: T}{\withcomplex{P}{K}}}\\[-1em]
%         %
%         &\kern-0em\runa{S-nmatch}\;\condinfrule{
%         \begin{matrix}
%             \varphi;\Phi;\Gamma \vdash \withtype{e}{\natinterval{I}{J}}\quad \varphi;\Phi, I \leq 0;\Gamma \vdash \withcomplex{P}{K} \\
%             \varphi;\Phi, J \geq 1;\Gamma, \withtype{x}{\natinterval{I-1}{J-1}} \vdash \withcomplex{Q}{K'}
%         \end{matrix}}{\varphi;\Phi;\Gamma \vdash \withcomplex{\match{e}{P}{x}{Q}}{L}}{\text{where}\quad L = \left\{
% \begin{matrix}
%     K & \text{if}\; \varphi;\Phi\vDash K' \leq K   \\
%     K' & \text{if}\; \varphi;\Phi\vDash K \leq K'  \\
%     K+K' & \text{otherwise}
% \end{matrix}
% \right.}\\[-1em]
%         %
%         %&\kern-0em\runa{S-nmatch-2}\;\infrule{
%         %\begin{matrix}
%         %    \varphi;\Phi;\Gamma \vdash \withtype{e}{\natinterval{I}{J}} \quad \varphi;\Phi\vDash K \leq K' \\
%         %    \varphi;\Phi, I \leq 0;\Gamma \vdash \withcomplex{P}{K} \quad \varphi;\Phi, J \geq 1;\Gamma, \withtype{x}{\natinterval{I-1}{J-1}} \vdash \withcomplex{Q}{K'}
%         %\end{matrix}}{\varphi;\Phi;\Gamma \vdash \withcomplex{\match{e}{P}{x}{Q}}{K'}}\\[-1em]
%         %
%         &\kern-0em\runa{S-lmatch}\;\condinfrule{
%         \begin{matrix}
%             \varphi;\Phi;\Gamma \vdash \withtype{e}{\texttt{List}[I,J](\mathcal{B})} \quad \varphi;\Phi, I \leq 0;\Gamma \vdash \withcomplex{P}{K} \\
%             \varphi;\Phi, J \geq 1;\Gamma, \withtype{x}{\mathcal{B}},y : \texttt{List}[I-1,J-1](\mathcal{B}) \vdash \withcomplex{Q}{K'}
%         \end{matrix}}{\varphi;\Phi;\Gamma \vdash \withcomplex{\texttt{match}\;e\;\{ [] \mapsto P;\; x :: y \mapsto Q \}}{L}}{\text{where}\quad L = \left\{
% \begin{matrix}
%     K & \text{if}\; \varphi;\Phi\vDash K' \leq K   \\
%     K' & \text{if}\; \varphi;\Phi\vDash K \leq K'  \\
%     K+K' & \text{otherwise}
% \end{matrix}
% \right.}\\[-1em]
%         %
%         %&\kern-0em\runa{S-lmatch-2}\;\infrule{
%         %\begin{matrix}
%         %    \varphi;\Phi;\Gamma \vdash \withtype{e}{\texttt{List}[I,J](\mathcal{B})} \quad \varphi;\Phi\vDash K \leq K' \\
%         %    \varphi;\Phi, I \leq 0;\Gamma \vdash \withcomplex{P}{K} \quad \varphi;\Phi, J \geq 1;\Gamma, \withtype{x}{\mathcal{B}},y : \texttt{List}[I-1,J-1](\mathcal{B}) \vdash \withcomplex{Q}{K'}
%       % \end{matrix}}{\varphi;\Phi;\Gamma \vdash \withcomplex{\texttt{match}\;e\;\{ [] \mapsto P;\; x :: y \mapsto Q \}}{K'}}\\[-1em]
%         %
%         &\kern4em\runa{S-par}\;\condinfrule{\varphi;\Phi;\Gamma\vdash P \triangleleft K\quad \varphi;\Phi;\Gamma\vdash Q \triangleleft K'}{\varphi;\Phi;\Gamma\vdash \parcomp{P}{Q} \triangleleft L}{\text{where}\quad L = \left\{
% \begin{matrix}
%     K & \text{if}\; \varphi;\Phi\vDash K' \leq K   \\
%     K' & \text{if}\; \varphi;\Phi\vDash K \leq K'  %\\
%     %K+K' & \text{otherwise}
% \end{matrix}
% \right.}\\[-1em]
%         %
%         %&\kern4em\runa{S-par-2}\;\infrule{\varphi;\Phi;\Gamma\vdash P \triangleleft K\quad \varphi;\Phi;\Gamma\vdash Q \triangleleft K'\quad \varphi;\Phi\vDash K \leq K'}{\varphi;\Phi;\Gamma\vdash \parcomp{P}{Q} \triangleleft K'}\\[-1em]
%         %
%         &\kern-0em\runa{S-iserv}\;\infrule{\texttt{in}\in\sigma\quad \varphi,\widetilde{i};\Phi;\text{ready}(\varphi,\Phi,\susumesim{\Gamma}{I}),a:\forall_0\widetilde{i}.\texttt{serv}^{\sigma\cap\{\texttt{out}\}}_K(\widetilde{T}),\widetilde{v} : \widetilde{T}\vdash P \triangleleft K'\quad \varphi,\widetilde{i};\Phi\vDash K' \leq K}{\varphi;\Phi;\Gamma,a:\forall_I\widetilde{i}.\texttt{serv}^\sigma_K(\widetilde{T})\vdash\; \bang\inputch{a}{\widetilde{v}}{}{P}\triangleleft I}\\[-1em]
%         %
%         &\kern-0em\runa{S-ich}\;\infrule{\texttt{in}\in\sigma\quad \varphi;\Phi;\susumesim{\Gamma}{I},a:\texttt{ch}_0^\sigma(\widetilde{T}),\widetilde{v} : \widetilde{T}\vdash P \triangleleft K}{\varphi;\Phi;\Gamma,a:\texttt{ch}_I^\sigma(\widetilde{T})\vdash \inputch{a}{\widetilde{v}}{}{P}\triangleleft K + I}
%         %
%         \kern8.5em \runa{S-och}\;\infrule{\texttt{out}\in \sigma\quad \varphi;\Phi;\susumesim{\Gamma}{I}\vdash \widetilde{e} : \widetilde{T}\quad \varphi;\Phi\vdash\widetilde{T}\sqsubseteq\widetilde{S}}{\varphi;\Phi;\Gamma,a:\texttt{ch}^{\sigma}_I(\widetilde{S})\vdash \asyncoutputch{a}{\widetilde{e}}{} \triangleleft I}\\[-1em]
%         %
%         &\kern0em\runa{S-oserv}\;\infrule{\texttt{out} \in \sigma \quad \varphi;\Phi;\susumesim{\Gamma}{I}\vdash \widetilde{e} : \widetilde{T}\quad \text{instantiate}(\widetilde{i},\widetilde{T})=\{\widetilde{J}/\widetilde{i}\}\quad  \varphi;\Phi\vdash\widetilde{T}\sqsubseteq\widetilde{S}\{\widetilde{J}/\widetilde{i}\}}{\varphi;\Phi;\Gamma,a:\forall_I\widetilde{i}.\texttt{serv}_K^\sigma(\widetilde{S})\vdash \asyncoutputch{a}{\widetilde{e}}{} \triangleleft K\!\substi{\widetilde{J}}{\widetilde{i}} + I}
%         %
%     \end{align*}\vspace{-1em}\end{framed}
%     \smallskip
%     \caption{Sized typing rules for parallel complexity of processes.}
%     \label{tab:sizedprocesstypingrules}
% \end{table*}

\begin{table*}[!ht]
    \begin{framed}\vspace{-1em}\begin{align*}
        %
        % S-nil
        &\runa{S-nu}\infrule{\varphi;\Phi;\Gamma, a:T \vdash P \triangleleft \kappa}{\varphi;\Phi;\Gamma \vdash \newvar{a:T}{P} \triangleleft \kappa}
        % S-par
        \kern1em\runa{S-par}\infrule{\varphi;\Phi;\Gamma \vdash P \triangleleft \kappa \quad \varphi;\Phi;\Gamma \vdash Q \triangleleft \kappa'}{\varphi;\Phi;\Gamma \vdash P \mid Q \triangleleft \text{basis}(\varphi, \Phi,\kappa \cup \kappa')}\\[-1em]
        %
        &\runa{S-tick}\infrule{\varphi;\Phi;\tforwardsim{\Gamma}{1} \vdash P \triangleleft \kappa}{\varphi;\Phi;\Gamma \vdash \tick P \triangleleft \kappa + 1}\kern2em
        %
        \runa{S-annot}\infrule{\varphi;\Phi;\tforwardsim{\Gamma}{n}\vdash P \triangleleft \kappa}{\varphi;\Phi;\Gamma\vdash n:P \triangleleft \kappa + n}\\[-1em]
        % S-match
        &\runa{S-match}\infrule{
        \begin{matrix}
            \varphi;\Phi;\Gamma \vdash e:\natinterval{I}{J} \quad \varphi;\Phi, I \leq 0;\Gamma \vdash P \triangleleft \kappa\\
            \varphi;\Phi, J \geq 1;\Gamma, x:\natinterval{I-1}{J-1} \vdash Q \triangleleft \kappa'
        \end{matrix}}{\varphi;\Phi;\Gamma \vdash \match{e}{P}{x}{Q} \triangleleft \text{basis}(\varphi, \Phi, \kappa \cup \kappa')}\\[-1em]
        % S-iserv
        &\runa{S-iserv}\infrule{\begin{matrix}
            \texttt{in} \in \sigma\quad \varphi;\Phi;\Gamma\vdash a:\servt{I}{i}{\sigma}{K}{\widetilde{T}}\\
            \varphi, \widetilde{i}; \Phi; \text{ready}(\varphi,\Phi,\tforwardsim{\Gamma}{I}), \widetilde{v} : \widetilde{T} \vdash P \triangleleft \kappa \quad \varphi,\widetilde{i};\Phi\vDash\kappa \leq K
        \end{matrix}}
        {\varphi;\Phi;\Gamma \vdash \;\bang\inputch{a}{\widetilde{v}}{}{P}\triangleleft \{I\}}
        %
        \kern14em\runa{S-nil}\kern-1em\infrule{}{\varphi;\Phi;\Gamma \vdash \nil \triangleleft \{0\}}\kern-3em\text{ }\\[-1em]
        % S-oserv
        &\runa{S-oserv}\infrule{\begin{matrix}
            \texttt{out} \in \sigma\quad \varphi;\Phi;\Gamma\vdash a:\servt{I}{i}{\sigma}{K}{\widetilde{T}}\\
            \varphi; \Phi;\tforwardsim{\Gamma}{I} \vdash \widetilde{e}:\widetilde{S} \quad \text{instantiate}(\widetilde{i}, \widetilde{S}) = \{\widetilde{J}/\widetilde{i}\} \quad \varphi;\Phi \vDash \widetilde{S} \sqsubseteq \widetilde{T}
        \end{matrix}}
        {\varphi;\Phi;\Gamma \vdash \asyncoutputch{a}{\widetilde{e}}{}\triangleleft \{K\{\widetilde{J}/\widetilde{i}\} + I\}}\\[-1em]
        % S-annot
        &\runa{S-ich}\infrule{\begin{matrix}
            \texttt{in} \in \sigma\quad \varphi;\Phi;\Gamma \vdash a:\chant{\sigma}{I}{\widetilde{T}}\\
            \varphi; \Phi; \tforwardsim{\Gamma}{I}, \widetilde{v}:\widetilde{T} \vdash P \triangleleft \kappa
        \end{matrix}}
        {\varphi;\Phi;\Gamma \vdash \inputch{a}{\widetilde{v}}{}{P} \triangleleft \kappa + I}\kern3em
        %
        \runa{S-och}\infrule{\begin{matrix}
            \texttt{out} \in \sigma\quad \varphi;\Phi;\Gamma \vdash a:\chant{\sigma}{I}{\widetilde{T}}\\
            \varphi; \Phi; \tforwardsim{\Gamma}{I} \vdash \widetilde{e}:\widetilde{S} \quad \varphi;\Phi \vDash \widetilde{S} \sqsubseteq \widetilde{T}
        \end{matrix}}
        {\varphi;\Phi;\Gamma \vdash \asyncoutputch{a}{\widetilde{e}}{} \triangleleft \{I\}}\\[-1em]
    \end{align*}\vspace{-1em}\end{framed}
    \smallskip
    \caption{Sized typing rules for parallel complexity of processes.}
    \label{tab:sizedprocesstypingrules}
\end{table*}

%
Rule $\runa{S-iserv}$ types a replicated input on a name $a$, and so $a$ must be bound to a server type with input capability. As the index $I$ in the server type denotes the time steps remaining before the server is ready to synchronize, we advance the time by $I$ units of time complexity when typing the continuation $P$. To ensure that bounds on synchronizations in $\downarrow^{\varphi;\Phi}_I\!\Gamma$ are not violated, we type $P$ under the time invariant part of $\downarrow^{\varphi;\Phi}_I\!\Gamma$, i.e. $\text{ready}(\varphi,\Phi,\downarrow_I\!\Gamma)$. Note that the bound on the span of the replicated input is the bound on the time remaining before the server is ready to synchronize. As the replicated input may be invoked many times, the cost of invoking the server is accounted for in rule $\runa{S-oserv}$ using the complexity bound $K$ in the server type. Therefore, we enforce that $K$ is in fact an upper bound on the span of the continuation $P$.\\

The rule $\runa{S-oserv}$ types outputs on names bound to server types. Here, as stated above, we must account for the cost of invoking a server, and as a replicated input on a server is parametric, we must \textit{instantiate} it based on the types of the expressions we are to output. Recall that in the type rule for outputs on servers from Chapter \ref{ch:bgts}, this is to be done by finding a substitution that satisfies the premise $\widetilde{T} \sqsubseteq \widetilde{S}\{\widetilde{J}/\widetilde{i}\}$. However, this turns out to be a difficult problem, and we can in fact prove it NP-complete for types of polynomial indices even if we disregard subtyping. However, note that it might not be necessary to use the full expressive power of polynomial indices, and so this may not necessarily affect type checking. Nevertheless, we over-approximate finding such a substitution, by using the function $\textit{instantiate}$. That is, we \textit{zip} together the index variables $\widetilde{i}$ with indices in types $\widetilde{T}$. Remark that Baillot and Ghyselen \cite{BaillotGhyselen2021} propose types for inference in their technical report, where the problem is simplified substantially, by forcing naturals to have lower bounds of $0$ and upper bounds with exactly one index variable and a constant. Our approach admits more expressive lower bounds and multiplications, while imposing no direct restrictions on the number of index variables in an index, and is thus more suitable for a type-checker.\\

We now prove the NP-completeness of the smaller problem of checking whether there exists a substitution $\{\widetilde{J}/\widetilde{i}\}$ that satisfies $T = S\{\widetilde{J}/\widetilde{i}\}$ where $T$ and $S$ are types with polynomial indices. The main idea is a reduction proof from the NP-complete 3-SAT problem, i.e. the satisfiability problem of a boolean formula in conjunctive normal form with exactly three literals in each clause \cite{Karp1972}. We first define a translation from a 3-SAT formula to a polynomial index in Definition \ref{def:3satredu}. This is a polynomial time computable reduction, as we simply replace each logical-and with a multiplication, each logical-or with an addition and each negation with a subtraction from 1. In Lemma \ref{lemma:soundtranslation}, we prove that the reduction is faithful with respect to satisfiability of a boolean formula. Finally, in Lemma \ref{lemma:npcompletesubst}, we prove that it is an NP-complete decision problem to verify the existence of a substitution that satisfies $T = S\{\widetilde{J}/\widetilde{i}\}$ for types $T$ and $S$.
%
\begin{defi}[3-SAT reduction]\label{def:3satredu}
We assume a one-to-one mapping $f$ from unknowns to index variables. Let $\phi$ be a 3-SAT formula
\begin{align*}
    \phi = \bigwedge_{i=1}^n \left(\ell_{i1} \lor \ell_{i2} \lor \ell_{i3}\right)% \land \cdots \land (A_n \lor B_n \lor C_n)
\end{align*}
where $\ell_{i1}$, $\ell_{i2}$ and $\ell_{i3}$ are of the forms $x$ or $\neg x$ for some variable $x$. We define a translation of $\phi$ to a polynomial index %$[\![\phi]\!]_{\text{3-SAT}}$
\begin{align*}
    [\![\phi]\!]_{\text{3-SAT}} = \prod_{i=1}^n \left([\![\ell_{i1}]\!]_{\text{3-SAT}} + [\![\ell_{i2}]\!]_{\text{3-SAT}} + [\![\ell_{i3}]\!]_{\text{3-SAT}}\right) %\cdots ([\![A_n]\!]_{\text{3-SAT}} + [\![B_n]\!]_{\text{3-SAT}} + [\![C_n]\!]_{\text{3-SAT}})
\end{align*}
where $[\![x]\!]_{\text{3-SAT}} = f(x)$ and $[\![\neg x]\!]_{\text{3-SAT}} = (1 - f(x))$.
\end{defi}


\begin{lemma}\label{lemma:soundtranslation}
Let $\phi$ be a 3-SAT formula. Then $\phi$ is satisfiable if and only if there exists a substitution $\{\widetilde{n}/\widetilde{i}\}$ such that $1\leq [\![\phi]\!]_{\text{3-SAT}}\{\widetilde{n}/\widetilde{i}\}$.
\begin{proof}
We consider the implications separately
\begin{enumerate}
    \item Assume that $\phi$ is satisfiable. Then there exists a truth assignment $\tau$ such that each clause of $\phi$ is true. Correspondingly, as $[\![\phi]\!]_{\text{3-SAT}}$ is a product of non-negative factors, we for some substitution $\{\widetilde{n}/\widetilde{i}\}$ have that $1 \leq [\![\phi]\!]_{\text{3-SAT}}\{\widetilde{n}/\widetilde{i}\}$ if and only if each factor in the product is positive. We compare the conditions for a clause to be true in $\phi$ to those for a corresponding factor in $[\![\phi]\!]_{\text{3-SAT}}$ to be positive, and show that a substitution $\{\widetilde{n}/\widetilde{i}\}$ exists such that $1 \leq [\![\phi]\!]_{\text{3-SAT}}\{\widetilde{n}/\widetilde{i}\}$. A clause in $\phi$ is a disjunction of three literals of either the form $x$ or $\neg x$ for some unknown $x$. Thus, for a clause to be true, we must have at least one literal $\tau(x) = tt$ or $\neg \tau(x) = tt$ with $\tau(x) = f\!f$. The corresponding factor in $[\![\phi]\!]_{\text{3-SAT}}$ is a sum of three terms of the forms $f(x)$ or $(1 - f(x))$ for some unknown $x$, where $f$ is a one-to-one mapping from unknowns to index variables. Here, we utilize that in the type system by Baillot and Ghyselen \cite{BaillotGhyselen2021}, we have $(1 - i\{\widetilde{n}/\widetilde{i}\}) = 0$ when $i\{\widetilde{n}/\widetilde{i}\} \geq 1$ and $(1 - i\{\widetilde{n}/\widetilde{i}\}) = 1$ when $i\{\widetilde{n}/\widetilde{i}\} = 0$. Thus, for a factor to be positive, it suffices that one term is positive, and so we can construct a substitution that guarantees this from the interpretation of $\phi$. That is, if $\tau(x) = tt$, we substitute $1$ for $f(x)$, and if $\tau(x) = f\!f$, we substitute 0 for $f(x)$. Then, whenever a literal is true in $\phi$, the corresponding term in $[\![\phi]\!]_{\text{3-SAT}}$ is positive, and so if $\phi$ is satisfiable then there exists a substitution $\{\widetilde{n}/\widetilde{i}\}$ such that $1 \leq [\![\phi]\!]_{\text{3-SAT}}\{\widetilde{n}/\widetilde{i}\}$.
     
    \item Assume that there exists a substitution $\{\widetilde{n}/\widetilde{i}\}$ such that $1 \leq [\![\phi]\!]_{\text{3-SAT}}\{\widetilde{n}/\widetilde{i}\}$. Then, as $[\![\phi]\!]_{\text{3-SAT}}$ is a product of non-negative factors, each factor must be positive. Correspondingly, if $\Phi$ is satisfiable, then there exists a truth assignment such that each clause of $\phi$ is true. We compare the conditions for a factor in $[\![\phi]\!]_{\text{3-SAT}}\{\widetilde{n}/\widetilde{i}\}$ to be positive to those for a corresponding clause in $\phi$ to be true, and show that $\phi$ is satisfiable. A factor in $[\![\phi]\!]_{\text{3-SAT}}$ is a sum of at most three terms of the forms $f(x)\{\widetilde{n}/\widetilde{i}\}$ or $(1 - f(x)\{\widetilde{n}/\widetilde{i}\})$. Here we again utilize that in the type system by Baillot and Ghyselen \cite{BaillotGhyselen2021}, we have $(1 - f(x)\{\widetilde{n}/\widetilde{i}\}) = 0$ when $f(x)\{\widetilde{n}/\widetilde{i}\} \geq 1$ and $(1 - f(x)\{\widetilde{n}/\widetilde{i}\}) = 1$ when $f(x)\{\widetilde{n}/\widetilde{i}\} = 0$, and so it must be that in the factor, we have at least one term $f(x)\{\widetilde{n}/\widetilde{i}\} \geq 1$ or $(1 - f(x)\{\widetilde{n}/\widetilde{i}\}) \geq 1$. Correspondingly, for the clause in $\phi$ to be true, at least one literal must be true. We show that there exists a truth assignment $\tau$ such that if a term in $[\![\phi]\!]_{\text{3-SAT}}$ is positive, then the corresponding literal in $\phi$ is true. If $f(x)\{\widetilde{n}/\widetilde{i}\}\geq 1$ then we set $\tau(x) = tt$, and if $f(x)\{\widetilde{n}/\widetilde{i}\} = 0$ we set $\tau(x) = f\!f$, as $[\![x]\!]_{\text{3-SAT}}\{\widetilde{n}/\widetilde{i}\} \geq 1$ when $f(x)\{\widetilde{n}/\widetilde{i}\} \geq 1$ and $[\![\neg x]\!]_{\text{3-SAT}} \geq 1$ when $f(x)\{\widetilde{n}/\widetilde{i}\}=0$. Then, whenever a term is positive in $[\![\phi]\!]_{\text{3-SAT}}\{\widetilde{n}/\widetilde{i}\}$, the corresponding literal in $\phi$ is true, and so if there exists a substitution $\{\widetilde{n}/\widetilde{i}\}$ such that $1 \leq [\![\phi]\!]_{\text{3-SAT}}\{\widetilde{n}/\widetilde{i}\}$, then $\phi$ is satisfiable.
    
\end{enumerate}
\end{proof}
\end{lemma}


\begin{lemma}\label{lemma:npcompletesubst}
Let $T$ and $S$ be types with polynomial indices. Then checking whether there exists a substitution $\{\widetilde{J}/\widetilde{i}\}$ such that $T = S\{\widetilde{J}/\widetilde{i}\}$ is an NP-complete problem.
\begin{proof}
By reduction from the 3-SAT problem. Assume that we have some algorithm that can verify the existence of a substitution $\{\widetilde{J}/\widetilde{i}\}$ such that $T = S\{\widetilde{J}/\widetilde{i}\}$, and let $\phi$ be a 3-SAT formula. Then using the algorithm, we can check whether $\phi$ is satisfiable by verifying whether there exists $\{\widetilde{J}/\widetilde{i}\}$ such that the following holds
\begin{align*}
    \texttt{Nat}[0,1] = \texttt{Nat}[0,(1 - (1 - [\![\phi]\!]_{\text{3-SAT}}))]\{\widetilde{J}/\widetilde{i}\}
\end{align*}
That is, $1 = (1 - (1 - [\![\phi]\!]_{\text{3-SAT}}\{\widetilde{J}/\widetilde{i}\}))$ implies $1 \leq [\![\phi]\!]_{\text{3-SAT}}\{\widetilde{J}/\widetilde{i}\}$, as $(1 - [\![\phi]\!]_{\text{3-SAT}}\{\widetilde{J}/\widetilde{i}\}) = 0$ when $[\![\phi]\!]_{\text{3-SAT}}\{\widetilde{J}/\widetilde{i}\} \geq 1$ and $(1 - [\![\phi]\!]_{\text{3-SAT}}\{\widetilde{J}/\widetilde{i}\}) = 1$ when $[\![\phi]\!]_{\text{3-SAT}}\{\widetilde{J}/\widetilde{i}\} = 0$. Furthermore, for $1 \leq [\![\phi]\!]_{\text{3-SAT}}\{\widetilde{J}/\widetilde{i}\}$ to hold, the indices in the sequence $\widetilde{J}$ cannot contain index variables, and so there must exist an equivalent substitution of naturals for index variables $\{\widetilde{n}/\widetilde{i}\}$. Then, by Lemma \ref{lemma:soundtranslation} we have that $\phi$ is satisfiable if and only if there exists a substitution $\{\widetilde{n}/\widetilde{i}\}$ such that $1\leq [\![\phi]\!]_{\text{3-SAT}}\{\widetilde{n}/\widetilde{i}\}$. Thus, as 3-SAT is an NP-complete problem, the reduction from 3-SAT is computable in polynomial time and as polynomial reduction is a transitive relation, i.e. any NP-problem is polynomial time reducible to verifying the existence of a substitution $\{\widetilde{J}/\widetilde{i}\}$ that satisfies the equation $T = S\{\widetilde{J}/\widetilde{i}\}$, it follows that the problem is NP-hard. To show that it is an NP-complete problem, we show that a \textit{certificate} can be verified in polynomial time. That is, given some substitution $\{\widetilde{J}/\widetilde{i}\}$, we can in linear time check whether $T=S\{\widetilde{J}/\widetilde{i}\}$ by substituting indices $\widetilde{J}$ for indices $\widetilde{i}$ in type $S$ and by then comparing the two types.\\
%
%
%Utilizing that $n - m = 0$ for $m\geq n$ in the type system of Baillot and Ghyselen \cite{BaillotGhyselen2021}, we can simulate any boolean formula using a polynomial index. By denoting $J = 0$ false and $I > 0$ true, we have the translation
% \begin{align*}
%     [\![a \land b]\!]_\phi =&\; [\![a]\!]_\phi [\![b]\!]_\phi\\
%     [\![a \lor b]\!]_\phi =&\; [\![a]\!]_\phi + [\![b]\!]_\phi\\
%     [\![\neg a]\!]_\phi =&\; (1 - [\![a]\!]_\phi)\\
%     [\![x]\!]_\phi =&\; i
% \end{align*}
% Then assuming some algorithm that checks whether there exists a substitution $\{\widetilde{J}/\widetilde{i}\}$ such that $T \sqsubseteq S\{\widetilde{J}/\widetilde{i}\}$, we can solve the boolean satisfiability problem. Let $\phi_0$ be any boolean formula and let $\widetilde{i}$ be the index variables in $[\![\phi_0]\!]_\phi$, and assume that there exists a substitution $\{\widetilde{J}/\widetilde{i}\}$ that satisfies the judgement
% \begin{align*}
%     \emptyset;\emptyset\vDash\texttt{Nat}[0,1] \sqsubseteq \texttt{Nat}[0,[\![\phi_0]\!]_\phi]\{\widetilde{J}/\widetilde{i}\}
% \end{align*}
% Then by rule $\runa{SS-nweak}$ we have that $\emptyset;\emptyset\vDash 1 \leq [\![\phi_0]\!]_\phi\{\widetilde{J}/\widetilde{i}\}$, and as $\varphi = \emptyset$, the indices $\widetilde{J}$ must be constants. Thus, $\emptyset;\emptyset\vDash 1 \leq [\![\phi_0]\!]_\phi\{\widetilde{J}/\widetilde{i}\}$ is equivalent to $1 \leq [\![\phi_0]\!]_\phi\{\widetilde{J}/\widetilde{i}\}$, and so $\phi_0$ must have a solution. If instead no such substitution exists, then for any $\{\widetilde{J}/\widetilde{i}\}$, it must be that $[\![\phi_0]\!]_\phi\{\widetilde{J}/\widetilde{i}\} = 0$ implying that $\phi_0$ is a contradiction. Therefore, as the boolean satisfiability problem is NP-complete, the algorithm we assumed must be NP-complete as well.
\end{proof}
\end{lemma}

In Example \ref{example:addition}, we show how a process implementing addition of naturals can be typed using our type rules, yielding a precise bound on the parallel complexity.
%

\begin{examp}\label{example:addition}
As an example of a process that is typable using our type rules, we show how the addition operator for naturals can be written as a process and subsequently be typed. We use a server to encode the addition operator
\begin{align*}
    !\inputch{\text{add}}{x,y,r}{}{\match{x}{\asyncoutputch{r}{y}{}}{z}{\tick{\asyncoutputch{\text{add}}{z,\succc y,r}{}}}}
\end{align*}
such that channel $r$ is used to output the addition of naturals $x$ and $y$. To type the process, we use the following contexts and set of index variables
\begin{align*}
    \Gamma\defeq&\; \text{add} : \forall_0 i,j,k,l,m,n,o.\texttt{serv}^{\{\texttt{in},\texttt{out}\}}_j(\texttt{Nat}[0,j],\texttt{Nat}[0,l],\texttt{ch}^{\{\texttt{out}\}}_j(\texttt{Nat}[0,j+l])) \\
    \Delta\defeq&\; \text{ready}(\cdot,\cdot,\Gamma), x : \texttt{Nat}[0,j], y: \texttt{Nat}[0,l], r:\texttt{ch}^{\{\texttt{out}\}}_j(\texttt{Nat}[0,j+l])\\
    \varphi \defeq&\; \{i,j,k,l,m,n,o\}
\end{align*}
%
We now derive a type for the encoding of the addition operator, yielding a precise bound of $j$, corresponding to an upper bound on the size of $x$, as we pattern match at most $j$ times on natural $x$. Notably we have that $\text{instantiate}((i,j,k,l,m,n,o),\texttt{Nat}[0,j\monus 1],\texttt{Nat}[1,l+1],\texttt{ch}^{\{\texttt{out}\}}_j(\texttt{Nat}[0,j+l]))=\{0/i,j\monus 1/j,0/k,l+1/l,j/m,0/n,j+l/o\}$.
%
{\small
\begin{align*}
    \begin{prooftree}
        %
        \infer0{\varphi;\cdot,0\leq 0;\Delta\vdash \asyncoutputch{r}{y}{} \triangleleft \{j\}}
        %
        % \infer0{\texttt{Nat}[0,j\monus 1] \sqsubseteq \texttt{Nat}[0,j]\{j\monus 1/j\}}
        % %
        % \infer0{\texttt{Nat}[0,l+1] \sqsubseteq \texttt{Nat}[0,l]\{l+1/l\}}
        % %
        % \infer0{\texttt{ch}^{\{\texttt{out}\}}_{j\monus 1}(\texttt{Nat}[0,j+l] \sqsubseteq \texttt{ch}^{\{\texttt{out}\}}_{j\monus 1}(\texttt{Nat}[0,j+l)\{j\monus 1/j,l+1/l\}}
        %
        \infer0{
        \begin{matrix}
        \varphi;\cdot,1\leq j\vdash\texttt{Nat}[0,j\monus 1] \sqsubseteq \texttt{Nat}[0,j]\{j\monus 1/j\}\\
        \varphi;\cdot,1\leq j\vdash\texttt{Nat}[1,l+1] \sqsubseteq \texttt{Nat}[0,l]\{l+1/l\}\\
        \varphi;\cdot,1\leq j\vdash\texttt{ch}^{\{\texttt{out}\}}_{j\monus 1}(\texttt{Nat}[0,j+l] \sqsubseteq \texttt{ch}^{\{\texttt{out}\}}_{j\monus 1}(\texttt{Nat}[0,j+l)\{j\monus 1/j,l+1/l\}
        \end{matrix}
        }
        %
        \infer1{\varphi;\cdot,1\leq j;\susumesim{\Delta}{1},z : \texttt{Nat}[0,j\monus 1]\vdash \asyncoutputch{\text{add}}{z,\succc y, r}{} \triangleleft \{j\monus 1\}}
        %
        \infer1{\varphi;\cdot,1\leq j;\Delta,z : \texttt{Nat}[0,j\monus 1]\vdash \tick{\asyncoutputch{\text{add}}{z,\succc y, r}{}} \triangleleft \{j\}}
        %
        \infer2{\varphi;\cdot;\Delta\vdash \match{x}{\asyncoutputch{r}{y}{}}{z}{\tick{\asyncoutputch{\text{add}}{z,\succc y,r}{}}} \triangleleft \{j\}}
        %
        \infer1{\cdot;\cdot;\Gamma\vdash\; !\inputch{\text{add}}{x,y,r}{}{\match{x}{\asyncoutputch{r}{y}{}}{z}{\tick{\asyncoutputch{\text{add}}{z,\succc y,r}{}}}}\triangleleft \{0\}}
    \end{prooftree}
\end{align*}}
%
\end{examp}

% \subsection{Undecidability of judgements}
% Verifying whether a polynomial constraint with integer coefficients imposes further restrictions onto the model set of index valuations of natural codomain of some set of known constraints can be reduced to Hilbert's tenth problem \cite{Davis1973}. That is, the problem of verifying whether a diophantine equation has an integer solution.\\

% We first assume some algorithm that can verify a judgement of the form $\varphi;\Phi\vDash C$ where $\varphi$ is a set of index variables and $C$ and $C'\in\Phi$ are binary constraints on polynomials of integer coefficients over relations from any subset of $\{\neq,\leq, <\}$. Recall that such a judgement holds exactly when for each index valuation $\rho : \varphi \longrightarrow \mathbb{N}$ over $\varphi$ for which $\rho \vDash C'$ for $C'\in\Phi$ we also have $\rho\vDash C$, i.e. $C$ does not impose further restrictions on interpretations of indices.\\

% We can then verify whether any diophantine equation has an integer solution. Let $p$ be an arbitrary polynomial of integer coefficients such that $p = 0$ is a diophantine equation. As only non-negative integers substitute for index variables, we first transform $p = 0$ to a new diophantine equation $p' = 0$ that has a non-negative integer solution exactly when $p = 0$ has an integer solution. To do this, we simply replace each index variable $i$ in $p$ with two new index variables $i_1 - i_2$. Then the judgement $\varphi;\emptyset\vDash p' \neq 0$ holds exactly when $p=0$ has no integer solution. That is, if $p=0$ has an integer solution, then there must exist a valuation $\rho_0$ such that $\rho_0\vDash \emptyset$ with $[\![p']\!]_{\rho_0} = 0$ and so $\rho_0\nvDash p' \neq 0$. Moreover, we need not rely on the relation $\neq$, as the judgements below are equivalent
% \begin{align*}
%     \varphi;\{p' \leq 0\} \vDash p' < 0\\
%     \varphi;\{p' \leq 0\} \vDash p' \leq 1
% \end{align*}
\section{Soundness}
\section{Verification of constraint judgements}\label{sec:verifyinglinearjudgements}
Until now we have not considered how we can verify constraint judgements in the type rules. The expressiveness of implementations of the type system by Baillot and Ghyselen \cite{BaillotGhyselen2021} depends on both the expressiveness of indices and whether judgements on the corresponding constraints are decidable. Naturally, we are interested in both of these properties, and so in this section, we show how judgements on linear constraints can be verified using algorithms. Later, we show how this can be extended to certain groups of polynomial constraints. We first make some needed changes to how the type checker uses subtraction.
%
\subsection{Subtraction of naturals}
The constraint judgements rely on a special minus operator ($\monus$) for indices such that $n \monus m=0$ when $m \geq n$, which we refer to as the \textit{monus} operator. This is apparent in the pattern match constructor type rule from Chapter \ref{ch:bgts}. Without this behavior, we may encounter problems when checking subtype premises in match processes. This has the consequence that equations such as $2\monus 3+3=3$ hold, such that indices form a semiring rather than a ring, as we are no longer guaranteed an additive inverse. In general, semirings lack many properties of rings that are desirable. For example, given two seemingly equivalent constraints $i \leq 5$ and $i \monus 5 \leq 0$, we see that by adding any constant to their left-hand sides, the constraints are no longer equivalent. Adding the constant 2 to their left-hand sides, we obtain $i + 2 \leq 5$ and $i \monus 5 + 2 \leq 0$, however, we see that the first constraint is satisfied given the valuation $i = 3$ but the second is not. In general the associative property of $+$ is lost.\\

Unfortunately, this is not an easy problem to solve implementation-wise, as indices are not actually evaluated but rather represent whole feasible regions. Thus, instead of trying to implement this operator exactly, we limit the number of processes typable by the type system. Removing the operator entirely is not an option as it us used by the type rules themselves. Instead, we ensure that one cannot \textit{exploit} the special behavior of monus by introducing additional conditions to the type rules of the type system. More precisely, any time the type system uses the monus operator such as $I \monus J$, we require the premise $\varphi;\Phi \vDash I \geq J$, in which case the monus operator is safe to treat as a regular minus. This, however, puts severe restrictions on the number of processes typable, and so we relax the restriction a bit by also checking the judgement $\varphi;\Phi \vDash I \leq J$, in which case we can conclude that the result is definitely $0$. If neither $\varphi;\Phi\vDash I \geq J$ nor $\varphi;\Phi\vDash I \leq 0$ hold, which is possible as $\leq$ and $\geq$ do not form a total order on indices, the result is undefined. We refer to this variant of monus as the \textit{partial} monus operator, as formalized in Definition \ref{def:partialmonus}. Note that this definition of monus allows us to obtain identical behavior to minus on a constraint $I \bowtie J$ by moving terms between the LHS and RHS, i.e. $I - K \bowtie J \Rightarrow I \bowtie J + K$, and so we can assume we have a standard minus operator when verifying judgements on constraints. For the remainder of this section, we assume this definition is used in the type rules instead of the usual monus. We may omit $\varphi;\Phi$ if it is clear from the context.%\\
%
%Definition \ref{def:partialmonus} defines the \textit{partial} monus operator that is undefined if we cannot determine if the result is either always positive or always zero. For the remainder of this thesis, we assume this definition is used in the type rules instead of the usual monus. We may omit $\varphi;\Phi$ if it is clear from the context.
%
\begin{defi}[Partial monus]\label{def:partialmonus}
Let $\Phi$ be a set of constraints in index variables $\varphi$. The partial monus operator is defined for two indices $I$ and $J$ as
\begin{equation*}
    I \monusE J = \begin{cases}
    I - J &\text{if $\varphi;\Phi \vDash J \leq I$}\\
    0 &\text{if $\varphi;\Phi \vDash I \leq J$}\\
    \textit{undefined} & \textit{otherwise}
    \end{cases}
\end{equation*}
\end{defi}

To ensure soundness of the algorithmic type rules after switching to the partial monus operator, we must make some changes to advancement of time. Consider the typing
\begin{align*}
    (\cdot,i);(\cdot,i\leq 3);\Gamma\vdash\; !\inputch{a}{}{}{\nil}  \mid 5 : \asyncoutputch{a}{}{} \triangleleft \{5\}
\end{align*}
where $\Gamma = \cdot,a : \forall_{3-i}\epsilon.\texttt{serv}^{\{\texttt{in},\texttt{out}\}}_0()$. Upon typing the time annotation, we advance the time of the server type by $5$ yielding the type $\forall_{3-i-5}\epsilon.\texttt{serv}^{\{\texttt{out}\}}_0()$ as $(\cdot,i);(\cdot,i\leq 3)\nvDash 3-i \geq 5$, which is defined as $(\cdot,i);(\cdot,i\leq 3)\vDash 3-i \leq 5$. However, if we apply the congruence rule $\runa{SC-sum}$ from right to left we obtain
\begin{align*}
    !\inputch{a}{}{}{\nil}  \mid 2 : 3 : \asyncoutputch{a}{}{}\equiv\;!\inputch{a}{}{}{\nil}  \mid 5 : \asyncoutputch{a}{}{}
\end{align*}
Then, we get a problem upon typing the first annotation. That is, as $(\cdot,i);(\cdot,i\leq 3)\nvDash 3-i \leq 2$ (i.e. when for some valuation $\rho$ we have $\rho(i) = 0$) the operation $(3-i) \monusE[(\cdot,i);(\cdot,i\leq 3)] 2$ is undefined. Thus, the type system loses its subject congruence property, and subsequently its subject reduction property. There are, however, several ways to address this. One option is to modify the type rules to perform a single advancement of time for a sequence of annotations. A more contained option is to remove monus from the definition of advancement of time, by enriching the formation rules of types with the constructor $\forall_{I}\widetilde{i}.\texttt{serv}^\sigma_K(\widetilde{T})^{-J}$ and by augmenting the definition of advancement as so
\begin{align*}
    \downarrow_I^{\varphi;\Phi}\!\!(\forall_J\widetilde{i}.\texttt{serv}^\sigma_K(\widetilde{T})) =&\; \left\{
\begin{matrix}
\forall_{J-I}\widetilde{i}.\texttt{serv}^\sigma_K(\widetilde{T}) & \text{ if } \varphi;\Phi\vDash I \leq J \\
\forall_0\widetilde{i}.\texttt{serv}^{\sigma\cap\{\texttt{out}\}}_K(\widetilde{T}) & \text{ if } \varphi;\Phi\vDash J \leq I \\
\forall_{J}\widetilde{i}.\texttt{serv}^{\sigma\cap\{\texttt{out}\}}_K(\widetilde{T})^{-I} & \text{ if } \varphi;\Phi\nvDash I \leq J \text{ and } \varphi;\Phi\nvDash J \leq I
\end{matrix}
\right.\\
%
\downarrow_I^{\varphi;\Phi}\!\!(\forall_{J}\widetilde{i}.\texttt{serv}^\sigma_K(\widetilde{T})^{-L}) =&\; \left\{
\begin{matrix}
\forall_{0}\widetilde{i}.\texttt{serv}^{\sigma\cap\{\texttt{out}\}}_K(\widetilde{T}) & \text{ if } \varphi;\Phi\vDash J \leq L+I \\
\forall_{J}\widetilde{i}.\texttt{serv}^{\sigma\cap\{\texttt{out}\}}_K(\widetilde{T})^{-(L+I)} & \text{ if } \varphi;\Phi\nvDash J \leq L+I
\end{matrix}
\right.
%
\end{align*}
This in essence introduces a form of \textit{lazy} time advancement, where time is not advanced until partial monus allows us to do so. Then, as the type rules for servers require a server type of the form $\forall_J\widetilde{i}.\texttt{serv}^\sigma_K(\widetilde{T})$, the summed advancement of time must always be less than or equal, or always greater than or equal to the time of the server, and so typing is invariant to the use of congruence rule $\runa{SC-sum}$. Revisiting the above example, we have that $\susume{\forall_{3-i}\epsilon.\texttt{serv}^{\{\texttt{in},\texttt{out}\}}_0()}{(\cdot,i)}{(\cdot,i\leq 3)}{5} =\; \susume{\susume{\forall_{3-i}\epsilon.\texttt{serv}^{\{\texttt{in},\texttt{out}\}}_0()}{(\cdot,i)}{(\cdot,i\leq 3)}{2}}{(\cdot,i)}{(\cdot,i\leq 3)}{3}$, and so we obtain the original typing
\begin{align*}
    (\cdot,i);(\cdot,i\leq 3);\Gamma\vdash\; !\inputch{a}{}{}{\nil}  \mid 2 : 3 : \asyncoutputch{a}{}{} \triangleleft \{5\}
\end{align*}



% \begin{remark}

%     Baillot and Ghyselen \cite{BaillotGhyselen2021} assume that the minus operator ($-$) for indices is defined such that $n-m=0$ when $m \geq n$. This has the consequence that expressions such as $2-3+3=3$ apply, such that indices form a semiring instead of a ring as we no longer have an additive inverse. In this work we lift this assumption by arguing that any index $I$ using a ring-centric definition for $-$ such that $I \leq 0$, can be simulated using another index $J$ using a semiring-centric definition for $-$ such that $J \leq 0$. For $I$, the order of summation of terms does not matter, and so we can freely change this. By moving any terms with a negative coefficient to the end of the summation, we obtain an expression of the form $c_1 i_1 + \cdots + c_n i_n - c_{n+1} i_{n+1} - \cdots - c_m i_m$ where $c_j$ are positive numbers and $i_j$ are index variables for $j = 0\dots m$. When evaluating this expression from left to right, the result will be increasing until $c_{n + 1} i_{n+1}$, as both the coefficients and index variables are positive, after which it will be decreasing. This results in an expression that is indifferent to the two definitions of $-$ when considering constraints of the form $I \leq 0$. Thus, a normalized constraint using a ring-centric definition of $-$ can be simulated using a normalized constraint using a semiring-centric definition of $-$.

% \end{remark}

\subsection{Undecidability of polynomial constraint judgements}
As we have seen, verifying that a constraint imposes no further restrictions onto index valuations amounts to checking whether all possible index valuations that satisfy a set of known constraints are also contained in the model space of our new constraint. It also amounts to checking whether the feasible region of the constraint contains the feasible region of a known system of inequality constraints, or checking whether the feasible region of the inverse constraint does not intersect the feasible region of a known system of inequality constraints. This turns out to be a difficult problem, and we can in fact prove it undecidable for diophantine constraints, i.e. multivariate polynomial inequalities with integer coefficients, when index variables must have natural (or integer) interpretations. The main idea is to reduce Hilbert's tenth problem \cite{Hilbert1902} to that of verification of judgements on constraints, as this problem has been proven undecidable \cite{Davis1973}. That is, we show that assuming some complete algorithm that verifies judgements on constraints, we can verify whether an arbitrary diophantine equation has a solution with all unknowns taking integer values. We show this result in Lemma \ref{lemma:judgementUndecidable}.
%
\begin{lemma}\label{lemma:judgementUndecidable}
Let $C$ and $C'\in \Phi$ be diophantine inequalities with unknowns in $\varphi$ and coefficients in $\mathbb{N}$. Then the judgement $\varphi;\Phi\vDash C$ is undecidable.
\begin{proof}
By reduction from Hilbert's tenth problem. Let $p=0$ be an arbitrary diophantine equation. We show that assuming some algorithm that can verify a judgement of the form $\varphi;\Phi\vDash C$, we can determine whether $p=0$ has an integer solution. We must pay special attention to the non-standard definition of subtraction in the type system by Baillot and Ghyselen \cite{BaillotGhyselen2021} and to the fact that only non-negative integers substitute for index variables. We first replace each integer variable $x$ in $p$ with two non-negative variables $i_x - j_x$, referring to the modified polynomial as $p'$. We can quickly verify that $p'=0$ has a non-negative integer solution if and only if $p=0$ has an integer solution
\begin{enumerate}
    \item Assume that $p'=0$ has a non-negative integer solution. Then for each variable $x$ in $p$ we assign $x = i_x - j_x$ reaching an integer solution to $p$.
    
    \item Assume that $p=0$ has an integer solution. Then for each pair $i_x$ and $j_x$ in $p'$ we assign $i_x = x$ and $j_x = 0$ when $x \geq 0$ and $i_x = 0$ and $j_x = |x|$ when $x < 0$ reaching a non-negative integer solution to $p'$.
\end{enumerate}
%
Then, by the distributive property of integer multiplication and the associative property of integer addition, we can utilize that $p'$ has an equivalent expanded form 
\begin{align*}
p' = n_1 t_1 + \cdots + n_k t_k + n_{k+1} t_{k+1} + \cdots + n_{k+l} t_{k+l}    
\end{align*}
such that $n_1,\dots,n_k\in\mathbb{N}$, $n_{k+1},\dots,n_{k+l} \in \mathbb{Z}^{\leq 0}$ and $t_1,\dots,t_k,t_{k+1},\dots,t_{k+l}$ are power products over the set of all index variables in $p'$ denoted $\varphi_{p'}$. We can then factor the negative coefficients
\begin{align*}
    p' \;&= n_1 t_1 + \cdots + n_k t_k + n_{k+1} t_{k+1} + \cdots + n_{k+l} t_{k+l}\\ 
    \;&= (n_1 t_1 + \cdots + n_k t_k) + (-1)(|n_{k+1}| t_{k+1} + \cdots + |n_{k+l}| t_{k+l})\\
    \;&= (n_1 t_1 + \cdots + n_k t_k) - (|n_{k+1}| t_{k+1} + \cdots + |n_{k+l}| t_{k+l})
\end{align*}
We use this to show that $p'=0$ has a non-negative integer solution if and only if the following judgement does not hold 
{\small
\begin{align*}
    \varphi_{p'};\{|n_{k+1}| t_{k+1} + \cdots + |n_{k+l}| t_{k+l} \leq n_1 t_1 + \cdots + n_k t_k\}\vDash 1 \leq (n_1 t_1 + \cdots + n_k t_k) - (|n_{k+1}| t_{k+1} + \cdots + |n_{k+l}| t_{k+l}) 
\end{align*}}
We consider the implications separately
\begin{enumerate}
    \item Assume that $p'=0$ has a non-negative integer solution. Then we have that $n_1 t_1 + \cdots + n_k t_k = |n_{k+1}| t_{k+1} + \cdots + |n_{k+l}| t_{k+l}$, and so there must exist a valuation $\rho : \varphi_{p'} \longrightarrow \mathbb{N}$ such that $[\![n_1 t_1 + \cdots + n_k t_k]\!]_\rho = [\![|n_{k+1}| t_{k+1} + \cdots + |n_{k+l}| t_{k+l}]\!]_\rho$. We trivially have that $\rho$ satisfies $[\![|n_{k+1}| t_{k+1} + \cdots + |n_{k+l}| t_{k+l}]\!]_\rho \leq [\![n_1 t_1 + \cdots + n_k t_k]\!]_\rho$. But $\rho$ is not in the model space of the constraint $1 \leq (n_1 t_1 + \cdots + n_k t_k) - (|n_{k+1}| t_{k+1} + \cdots + |n_{k+l}| t_{k+l})$, and so the judgement does not hold.
    
    \item Assume that the judgement does not hold. Then there must exist a valuation $\rho : \varphi_{p'} \longrightarrow \mathbb{N}$ that satisfies $[\![|n_{k+1}| t_{k+1} + \cdots + |n_{k+l}| t_{k+l}]\!]_\rho \leq [\![n_1 t_1 + \cdots + n_k t_k]\!]_\rho$, but that is not in the model space of the constraint $1 \leq (n_1 t_1 + \cdots + n_k t_k) - (|n_{k+1}| t_{k+1} + \cdots + |n_{k+l}| t_{k+l})$. This implies that $[\![n_1 t_1 + \cdots + n_k t_k]\!]_\rho = [\![|n_{k+1}| t_{k+1} + \cdots + |n_{k+l}| t_{k+l}]\!]_\rho$, and so $p'$ has a non-negative integer solution.
\end{enumerate}
% Then the subtraction operator in Baillot and Ghyselen $\cite{BaillotGhyselen2021}$ only has non-standard behavior when $[\![n_{k+1} t_{k+1} + \cdots + n_{k+l} t_{k+l}]\!]_\rho > [\![n_1 t_1 + \cdots + n_k t_k]\!]_\rho$ for some interpretation $\rho : \varphi_{p'} \longrightarrow \mathbb{N}$ where $\varphi_{p'}$ is the set of all index variables in $p'$. Thus, we have that the judgement
% \begin{align*}
%     \varphi_{p'};\{n_{k+1} t_{k+1} + \cdots + n_{k+l} t_{k+l} \leq n_1 t_1 + \cdots + n_k t_k\}\vDash 1 \leq (n_1 t_1 + \cdots + n_k t_k) - (n_{k+1} t_{k+1} + \cdots + n_{k+l} t_{k+l}) 
% \end{align*}
% holds exactly when there exists no index valuation $\rho$ over $\varphi_{p'}$ that simultaneously satisfies $[\![n_{k+1} t_{k+1} + \cdots + n_{k+l} t_{k+l}]\!]_\rho \leq [\![n_1 t_1 + \cdots + n_k t_k]\!]_\rho$ and $[\![p']\!]_\rho = 0$. 
As such, we can verify that the above judgement does not hold if and only if $p'$ has a non-negative integer solution, and by extension if and only if $p$ has an integer solution. Thus, we would have a solution to Hilbert's tenth problem, which is undecidable.
\end{proof}
\end{lemma}

As an unfortunate consequence of Lemma \ref{lemma:judgementUndecidable}, we are forced into considering approximate algorithms for verification of judgements over polynomial constraints (in general). However, this result does not imply that type checking is undecidable. It may well be that problematic judgements are not required to type check any process, as computational complexity has certain properties, such as monotonicity. Note that the freedom of type checking, i.e. we can specify an arbitrary type context as well as type annotations, enables us to select indices that lead to undecidable judgements. To prove that type checking is undedidable, however, a more reasonable result would be that there exists a process that is typable if and only if an undecidable judgement is satisfied. This is out of the scope of this thesis, and so we leave it as future work. % Remark that Baillot and Ghyselen \cite{BaillotGhyselen2021} introduce a notion of type inference in their technical report, where the set of constraints $\Phi$ is empty for any judgement on constraints, and so they are able to bypass some of the problems associated with checking such judgements. However, this comes at the price of expressiveness, as natural types are forced to have lower bounds of $0$ and upper bounds with exactly one index variable and constant. Such indices are arguably sufficient for describing the sizes of simple terms when all operations on these terms in a program can be correspondingly described with a single index variable and constant. However, this quickly becomes too restrictive, as we are unable to type servers that implement simple arithmetic operations such as addition and subtraction.

\subsection{Normalization of linear indices}

To make checking of judgements on constraints tractable, we reduce the set of function symbols on which indices are defined, such that indices may only contain integers and index variables, as well as addition, subtraction and scalar multiplication operators, such that we restrict ourselves to linear functions.
\begin{align*}
        I,J ::= n \mid i \mid I + J \mid I - J \mid n I
    \end{align*}
% \begin{defi}[Indices]
%     \begin{align*}
%         I,J ::= n \mid i \mid I + J \mid I - J \mid I \cdot J
%     \end{align*}
% \end{defi}


Such indices can be written in a \textit{normal} form, presented in Definition \ref{def:normlinindex}.

\begin{defi}[Normalized linear index]\label{def:normlinindex}
    Let $I$ be an index in index variables $\varphi = i_1,\dots,i_n$. We say that $I$ is a \textit{normalized} index when it is a linear combination of index variables $i_1, ..., i_n$. Let $m$ be an integer constant and $I_\alpha\in\mathbb{Z}$ the coefficient of variable $i_\alpha$, we then define normalized indices as
    %
    \begin{align*}
        I = \normlinearindex{m}{I}
    \end{align*}
    
    
    We use the notation $\mathcal{B}(I)$ and $\mathcal{E}(I)$ to refer to the constant and unique identifiers of index variables of $I$, respectively.
\end{defi}

Any index can be transformed to an equivalent normalized index (i.e. it is a normal form) through expansion with the distributive law, reordering by the commutative and associative laws and then by regrouping terms that share variables. Therefore, the set of normalized indices in index variables $i_1,\dots,i_n$ and with coefficients in $\mathbb{Z}$, denoted $\mathbb{Z}[i_1,\dots,i_n]$, is a free module with the variables as basis, as the variables are linearly independent. In Definition \ref{def:operationsmodule}, we show how scalar multiplication, addition and multiplication of normalized indices (i.e. linear combinations of monomials) can be defined. Definition \ref{def:normalizationindex} shows how an equivalent normalized index can be computed from an arbitrary linear index using these operations.
%
\begin{defi}[Operations in $\freemodule$]\label{def:operationsmodule}
Let $I = \normlinearindex[\varphi_1]{n}{I}$ and $J = \normlinearindex[\varphi_2]{m}{J}$ be normalized indices in index variables $i_1,\dots,i_n$. We define addition and scalar addition of such indices. Given a scalar $n\in\mathbb{Z}$, the scalar multiplication $n I$ is
%
\begin{align*}
    n I = \normlinearindex[\mathcal{E}(I)]{n \cdot m}{n I}
\end{align*}
When $d$ is a common divisor of all coefficients in $I$, i.e. $I_\alpha / n \in \mathbb{Z}$ for all $\alpha\in\varphi$, the inverse operation is defined
\begin{align*}
    \frac{I}{d} = \frac{n}{d} + \sum_{\alpha\in \mathcal{E}(I)} \frac{I_\alpha}{d} i_\alpha\quad\text{if}\;\frac{I_\alpha}{d} \in \mathbb{Z}\;\text{for all}\;\alpha\in\mathcal{E}(I)
\end{align*}

The addition of $I$ and $J$ is the sum of constants plus the sum of scaled variables where coefficients $I_\alpha$ and $J_\alpha$ are summed when $\alpha\in\varphi_1 \cap \varphi_2$
\begin{align*}
    I + J = n + m + \sum_{\alpha \in \mathcal{E}(I) \cup \mathcal{E}(J)}(I_\alpha + J_\alpha)i_\alpha
\end{align*}

where for any $\alpha\in \varphi_1 \cup \varphi_2$ such that $I_\alpha + J_\alpha = 0$ we omit the corresponding zero term. The inverse of addition is always defined for elements of a polynomial ring
%
\begin{align*}
    I - J = n - m + \sum_{\alpha \in \mathcal{E}(I) \cup \mathcal{E}(J)}(I_\alpha - J_\alpha)i^\alpha
\end{align*}
\end{defi}
%
%We now formalize the transformation of an index $I$ to an equivalent normalized index in Definition \ref{def:normalizationindex}. An integer constant $n$ corresponds to scaling the monomial identified by the exponent vector of all zeroes by $n$. An index variable $i$ represents the monomial consisting of exactly one $i$ scaled by $1$. For addition, subtraction and multiplication we simply normalize the two subindices and and use the corresponding operators for normalized indices.
\begin{defi}[Index normalization]\label{def:normalizationindex}
The normalization of some index $I$ in index variables $i_1,\dots,i_n$ into an equivalent normalized index $\mathcal{N}(I)\in \mathbb{Z}[i_1,\dots,i_n]$ is a homomorphism defined inductively
    \begin{align*}
        \mathcal{N}(n) =&\; n i_1^0\cdots i_n^0\\
        \mathcal{N}(i_j) =&\; 1 i_1^0 \cdots i_j^1 \cdots i_n^0\\
        \mathcal{N}(I + J) =&\; \mathcal{N}(I) + \mathcal{N}(J)\\
        \mathcal{N}(I - J) =&\; \mathcal{N}(I) - \mathcal{N}(J)\\
        \mathcal{N}(n I) =&\; n \mathcal{N}(I)
    \end{align*}
\end{defi}

% \subsubsection{Normalization of constraints}
% A constraint may provide stronger or weaker restrictions on index variables compared to another constraint, or it may provide entirely different restrictions that are neither stronger nor weaker. For example, assuming some index $J$, if we have the constraint $3 \cdot i \leq J$, the constraint $2 \cdot i \leq J$ is redundant as index variables can only be assigned natural numbers, and thus $3 \cdot i \leq J$ implies $n \cdot i \leq J$ for any $n \leq 3$. Similarly, $I \leq n \cdot j$ implies $I \leq m \cdot j$ for any $n \leq m$. We thus define the subconstraint relation $\sqsubseteq$, and by extension the subindex relation $\sqsubseteq_\text{Index}$, in Definition \ref{def:subconstraint}. If $C_1 \sqsubseteq C_2$ we say that $C_2$ is a subconstraint of $C_1$.


% \begin{defi}[Subindices and subconstraints] \label{def:subconstraint}
%     We define the subindex relation $\sqsubseteq_\text{Index}$ by the following rule
%     \begin{align*}
%         &I \sqsubseteq_\text{Index} J \quad \text{ if} \\
%         &\quad (\mathcal{B}(I) \leq \mathcal{B}(J)) \land\\
%         &\quad (\forall \alpha \in \mathcal{E}(I) \cap \mathcal{E}(J) : I_\alpha \leq J_\alpha)\land\\
%         &\quad (\forall \alpha \in \mathcal{E}(J) \setminus \mathcal{E}(I) : J_\alpha \geq 0)\land\\
%         &\quad (\forall \alpha \in \mathcal{E}(I) \setminus \mathcal{E}(J) : I_\alpha \leq 0)
%     \end{align*}
%     % \begin{align*}
%     %     &(\varphi, F) \sqsubseteq_\text{Index} (\varphi', F') \text{ if} \\
%     %     &\quad (\forall V \in \varphi \cap \varphi' : F(V) \leq F'(V)) \land\\
%     %     &\quad (\forall V \in \varphi' \setminus \varphi : F'(V) \geq 0) \land\\
%     %     &\quad (\forall V \in \varphi \setminus \varphi' : F'(V) \leq 0)
%     % \end{align*}
    
%     We define the subconstraint relation $\sqsubseteq$ by the following rule
%     \begin{align*}
%       &\infrule{I' \sqsubseteq_\text{Index} I \quad J \sqsubseteq_\text{Index} J'}{I \leq J \sqsubseteq I' \leq J'}
%       %
%       %
%       %&\infrule{I \leq J \sqsubseteq I' \leq J' \quad I' \leq J' \sqsubseteq I'' \leq J''}{I \leq J \sqsubseteq I'' \leq J''}
%     \end{align*}
% \end{defi}

We extend normalization to constraints. We first note that an equality constraint $I = J$ is satisfied if and only if $I \leq J$ and $J \leq I$ are both satisfied. Thus, it suffices to only consider inequality constraints. A normalized constraint is of the form $I \leq 0$ for some normalized index $I$, as formalized in Definition \ref{def:normconst}.
%
\begin{defi}[Normalized constraints]\label{def:normconst}
    Let $C = I \leq J$ be an inequality constraint such that $I$ and $J$ are normalized indices. We say that $I-J \leq 0$ is the normalization of $C$ denoted $\mathcal{N}(C)$, and we refer to constraints in this form as \textit{normalized} constraints.
    %We represent normalized constraints $C$ using a single normalized constraint $I$, such that $C$ is of the form
    %\begin{align*}
    %    C = I \leq 0
    %\end{align*}
%
\end{defi}
%
% We now show how any constraint $J \bowtie K$ can be represented using a set of normalized constraints of the form $I \leq 0$ where $I$ is a normalized index. To do this, we first represent the constraint $J \bowtie K$ using a set of constraints of the form $J \leq K$ using the function $\mathcal{N_R}$. We then finalize the normalization using the function $\mathcal{N}$ by first moving all indices to the left-hand side of the constraint.
%
% \begin{defi}
%     Given a constraint $I \bowtie J$ $(\bowtie\; \in \{\leq, \geq, =\})$, the function $\mathcal{N_R}$ converts $I \bowtie J$ to a set of constraints of the form $I \leq J$
%     %
%     \begin{align*}
%         \mathcal{N_R}(I \leq J) &= \{I \leq J\}\\
%         \mathcal{N_R}(I \geq J) &= \{J \leq I\}\\
%         %\mathcal{N_R}(I < J) &= \{I+1 \leq J\}\\
%         %\mathcal{N_R}(I > J) &= \{J+1 \leq I\}\\
%         \mathcal{N_R}(I = J) &= \{I \leq J, J \leq I\}
%     \end{align*}
% \end{defi}
%
% \begin{defi}
%     Given a constraint $C$, the function $\mathcal{N}$ converts $C$ into a set of normalized constraints of the form $I \leq 0$
%     %
%     \begin{align*}
%         \mathcal{N}(C) &= \left\{I-J \leq 0 \mid (I \leq J) \in \mathcal{N}_R(C)\right\}
%     \end{align*}
% \end{defi}
%
%Normalized constraints have the key property that, given any two constraints $I \leq 0$ and $J \leq 0$, we can combine these to obtain a new constraint $J + I \leq 0$. This is possible as we know that both $I$ and $J$ are both non-positive, and so their sum must also be non-positive. In general, given $n$ normalized constraints $I_1 \leq 0, ..., I_n \leq 0$, we can infer any linear combination $a_1 \cdot I_1 \leq 0 + ... + a_n \cdot I_n \leq 0$ where $a_i \geq 0$ for $i = 1..n$ as new constraints that can be inferred based on the constraints $I_1 \leq 0, ..., I_n \leq 0$. Linear combinations where all coefficients are non-negative are also called \textit{conical combinations}.
Normalizing constraints has a number of benefits. First of all, it ensures that equivalent constraints are always expressed the same way. Secondly, having all constraints in a common form where variables only appear once means we can easily reason about individual variables of a constraint, which will be useful later when we verify constraint judgements.
%
\subsection{Checking for emptiness of model space}
As explained in Section \ref{sec:cjalternativeform}, we can verify a constraint judgement $\varphi;\Phi \vDash C_0$ by letting $C_0'$ be the inverse of $C_0$ and checking if $\mathcal{M}_\varphi(\Phi \cup \{C_0'\}) = \emptyset$ holds. Being able to check for non-emptiness of a model space is therefore paramount for verifying constraint judgements. For convenience, given a finite ordered set of index variables $\varphi = \{i_1, i_2, \dots, i_n\}$, we represent a normalized constraint $I \leq 0$ as a vector $\left( \mathcal{B}(I), I_1\; I_2\; \cdots\; I_n \right)_{\varphi}$. As such, the constraint $-5i + -2j + -4k \leq 0$ can be represented by the vector $\cvect[\varphi_1]{0 {-5} {-2} {-4}}$ where $\varphi_1=\left\{i, j, k\right\}$. Another way to represent that same constraint is with the vector $\cvect[\varphi_2]{0 {-5} {-2} 0 {-4}}$ where $\varphi_2 = \left\{i,j,l,k\right\}$. We denote the vector representation of a constraint $C$ over a finite ordered set of index variables $\varphi$ by $\mathbf{C}_{\varphi}$. We extend this notation to sets of constraints, such that $\Phi_{\varphi}$ denotes the set of vector representations over $\varphi$ of normalized constraints in $\Phi$\\

Recall that the model space of any set of constraints $\Phi$ is the set of all valuations satisfying all constraints in $\Phi$. Thus, to show that $\mathcal{M}_\varphi(\Phi)$ is empty, we must show that no valuation $\rho$ exists satisfying all constraints in $\Phi$. This is a linear constraint satisfaction problem (CSP) with an infinite domain. One method for solving such is by optimization using the simplex algorithm. If the linear program of the CSP has a feasible solution, the model space is non-empty and if it does not have a feasible solution, the model space is empty.\\

As is usual for linear constraints, our linear constraints can be thought of as hyper-planes dividing some n-dimensional space in two, with one side constituting the feasible region and the other side the non-feasible region. By extension, for a set of constraints their shared feasible region is the intersection of all of their individual feasible regions. Since the feasible region of a set of constraints is defined by a set of hyper-planes, the feasible region consists of a convex polytope. This fact is used by the simplex algorithm when performing optimization.\\

The simplex algorithm has some requirements to the form of the linear program it is presented, i.e. that it must be in \textit{standard} form. The standard form is a linear program expressed as 
\begin{align*}
    \text{minimize}&\quad \mathbf{c}^T\mathbf{a}\\
    \text{subject to}&\quad M\mathbf{a} = \mathbf{b}\\
    &\quad\mathbf{a} \geq \mathbf{0}
\end{align*}
where $M$ is a matrix representing constraints, $\mathbf{a}$ is a vector of scalars, and $\mathbf{b}$ is a vector of constants. As such, we first need all our constraints to be of the form $a_0 \cdot i_0 + ... + a_n \cdot i_n \leq b$, after which we must convert them into equality constraints by introducing \textit{slack} variables that allow the equality to also take on lower values. Since all of our constraints are normalized and of the form $I \leq 0$, all of our slack variables will have negative coefficients. In our specific case, we let row $i$ of $M$ consist of $(\mathbf{C}^i_\varphi)_{-1}$, where $(\cdot)_{-1}$ removes the first element of the vector (the constant term here). We must also include our slack variables, and so we augment row $i$ of $M$ with the n-vector with all zeroes except at position $i$ where it is $-1$. We let $\mathbf{a}$ be a column vector containing our variables in $\varphi$ as well as our slack variables, and finally we let $\mathbf{b}_i = -(\mathbf{C}^i_\varphi)_1$. $\mathbf{c}$ may be an arbitrary vector.\\

Checking feasibility of the above linear program can itself be formulated as a linear program that is guaranteed to be feasible, enabling us to use efficient polynomial time linear programming algorithms, such as interior point methods, to check whether constraints are covered. Let $\mathbf{s}$ be a new vector, then we have the linear program
%
\begin{align*}
    \text{minimize}&\quad \mathbf{1}^T\mathbf{s}\\
    \text{subject to}&\quad M\mathbf{a} + \mathbf{s} = \mathbf{b}\\
    &\quad\mathbf{a},\mathbf{s} \geq \mathbf{0}
\end{align*}
where $\mathbf{1}$ is the vector of all ones. We can verify the feasibility of this problem with the certificate $(\mathbf{a},\mathbf{s})=(\mathbf{0},\mathbf{b})$. Then the original linear program is feasible if and only if the augmented problem has an optimal solution $(\mathbf{x}^*,\mathbf{s}^*)$ such that $\mathbf{s}^* = \mathbf{0}$.\\

Given a constraint judgement $\varphi;\Phi \vDash C_0$, it should be noted that while the simplex algorithm can be used to check if a solution exists to the constraints $\Phi \cup \{C_0'\}$, there is no guarantee that the solution is an integer solution nor that an integer solution exists at all. Thus, in the case that a non-integer solution exists but no integer solution, this method will over-approximate. An example of such is the two constraints $3i - 1 \leq 0$ and $-2i + 1 \leq 0$ yielding the feasible region where $\frac{1}{3} \leq i \leq \frac{1}{2}$, containing no integers. For an exact solution, we may use integer programming.

% \subsubsection{Conical combinations of constraints}
% We now show how constraints can be conically combined. For convenience, given a finite ordered set of index variables $\varphi = \{i_1, i_2, \dots, i_n\}$, we represent a normalized constraint $I \leq 0$ as a vector $\left( \mathcal{B}(I), I_1\; I_2\; \cdots\; I_n \right)_{\varphi}$. As such, the constraint $-5i + -2j + -4k \leq 0$ can be represented by the vector $\cvect[\varphi_1]{0 {-5} {-2} {-4}}$ where $\varphi_1=\left\{i, j, k\right\}$. Another way to represent that same constraint is with the vector $\cvect[\varphi_2]{0 {-5} {-2} 0 {-4}}$ where $\varphi_2 = \left\{i,j,l,k\right\}$. We denote the vector representation of a constraint $C$ over a finite ordered set of index variables $\varphi$ by $\mathbf{C}_{\varphi}$. We extend this notation to sets of constraints, such that $\Phi_{\varphi}$ denotes the set of vector representations over $\varphi$ of normalized constraints in $\Phi$. Then for a finite ordered set of exponent vectors $\varphi$ and a set of normalized constraints $\Phi$, we can infer any constraint $C$ represented by a vector $\mathbf{C}_\varphi\in \text{coni}(\Phi_\varphi)$ where $\text{coni}(\Phi_\varphi)$ is the \textit{conical hull} of $\Phi_\varphi$. That is, $\text{coni}(\Phi_\varphi)$ is the set of conical combinations with non-negative integer coefficients of vectors in $\Phi_\varphi$
% %
% \begin{align*}
%   \text{coni}(\Phi_\varphi) = \left\{\sum^k_{i=1} a_i {\mathbf{C}^i_\varphi} : {\mathbf{C}^i_\varphi} \in \Phi_\varphi,\; a_i,k \in \mathbb{N}\right\}  
% \end{align*}
% %
% Then, to check if a constraint $C^{new}$ is covered by the set of normalized constraints $\Phi = \{C_1,C_2,\dots, C_n\}$, we can test if $\mathbf{C}^{new}_\varphi$ is a member of the conical hull $\text{coni}(\Phi_\varphi)$. However, by itself, this does not take into account subconstraints of constraints in $\Phi$, as these may not necessarily be written as conical combinations of $\Phi_\varphi$. To account for these, we can include $m=|\varphi|$ vectors of size $m$ of the form $\cvect{-1 0 $\cdots$ 0}, \cvect{0 {-1} 0 $\cdots$ 0}, \dots, \cvect{0 $\cdots$ 0 {-1})}$ in $\Phi_\varphi$. As the conical hull $\text{coni}(\Phi_\varphi)$ is infinite when there exists $\mathbf{C}_\varphi \in \Phi_\varphi$ such that $\mathbf{C}_\varphi \neq \mathbf{0}$ where $\mathbf{0}$ is the vector of all zeroes, when checking for the existence of a conical combination of vectors in $\Phi_\varphi$ equal to $\mathbf{C}^\textit{new}_\varphi$, we can instead solve the following system of linear equations
% %
% \begin{align*}
%     a_1 {\mathbf{C}^1_\varphi}_1 + a_2 {\mathbf{C}^2_\varphi}_1 + \cdots + a_n {\mathbf{C}^n_\varphi}_1 =&\; {\mathbf{C}^{new}_\varphi}_1\\
%     a_1 {\mathbf{C}^1_\varphi}_2 + a_2 {\mathbf{C}^2_\varphi}_2 + \cdots + a_n {\mathbf{C}^n_\varphi}_2 =&\; {\mathbf{C}^{new}_\varphi}_2\\
%     &\!\!\!\vdots\\
%     a_1 {\mathbf{C}^1_\varphi}_m + a_2 {\mathbf{C}^2_\varphi}_m + \cdots + a_n {\mathbf{C}^n_\varphi}_m =&\; {\mathbf{C}^{new}_\varphi}_m
% \end{align*}
% %
% where $a_1,a_2,\dots,a_m\in\mathbb{Z}_{\geq 0}$ are non-negative integer numbers. However, this is an integer programming problem, and so it is NP-hard. We can relax the requirement for $a_1,a_2,\dots,a_m$ to be integers, as the equality relation is preserved under multiplication by any positive real number. We can then view the above system as a linear program, with additional constraints $a_i \geq 0$ for $1 \geq i \geq n$. That is, let $M = \vect{$\mathbf{C}^1_\varphi$ $\mathbf{C}^2_\varphi$ $\cdots$ $\mathbf{C}^n_\varphi$}$ be a matrix with column vectors representing constraints and $\mathbf{a} = \vect{$a_1$ $a_2$ $\cdots$ $a_n$}$ be a row vector of scalars, then checking whether $\mathbf{C}^{new}_\varphi\in\text{coni}(\Phi_\varphi)$ amounts to determining if the following linear program is feasible
% %
% \begin{align*}
%     \text{minimize}&\quad \mathbf{c}^T\mathbf{a}\\
%     \text{subject to}&\quad M\mathbf{a} = \mathbf{C}^{new}_\varphi\\
%     &\quad\mathbf{a} \geq \mathbf{0}
% \end{align*}
% %
% where $\mathbf{c}$ is an arbitrary vector of length $n$ and $\mathbf{0}$ is the vector of all zeroes of length $n$. Checking feasibility of the above linear program can itself be formulated as a linear program that is guaranteed to be feasible, enabling us to use efficient polynomial time linear programming algorithms, such as interior point methods, to check whether constraints are covered. Let $\mathbf{s}$ be a new vector of length $m$, then we have the linear program
% %
% \begin{align*}
%     \text{minimize}&\quad \mathbf{1}^T\mathbf{s}\\
%     \text{subject to}&\quad M\mathbf{a} + \mathbf{s} = \mathbf{C}^{new}_\varphi\\
%     &\quad\mathbf{a},\mathbf{s} \geq \mathbf{0}
% \end{align*}
% where $\mathbf{1}$ is the vector of all ones of length $m$. We can verify the feasibility of this problem with the certificate $(\mathbf{a},\mathbf{s})=(\mathbf{0},\mathbf{C}^{new}_\varphi)$. Then the original linear program is feasible if and only if the augmented problem has an optimal solution $(\mathbf{x}^*,\mathbf{s}^*)$ such that $\mathbf{s}^* = \mathbf{0}$.
% %

\begin{examp}
    Given the constraints
    \begin{align*}
        C^1 &= 3i - 3 \leq 0\\
        C^2 &= j + 2k - 2 \leq 0\\
        C^3 &= -k \leq 0\\
        C^{new} &= i + j - 3 \leq 0
    \end{align*}
    
    we want to check if the constraint judgement $\{i, j, k\};\{C^1, C^2, C^3\} \vDash C^{new}$ is satisfied\\
    
    
    We first let $C^{newinv}$ be the inversion of constraint $C^{new}$.
    \begin{align*}
        C^{newinv} &= 1i + 1j - 2 \geq 0
    \end{align*}
    
    We now want to check if the feasible region $\mathcal{M}_\varphi(\{C^1, C^2, C^3, C^{newinv}\})$ is nonempty. To do so, we construct a linear program with the four constraints. To convert all inequality constraints into equality constraints, we add the slack variables $s_1, s_2, s_3, s_4$ 
    
    \begin{align*}
        \text{minimize}&\quad i + j + k\\
        \text{subject to}&\quad 3i + 0j + 0k + s_1 = 3\\
        &\quad 0i + 1j + 2k + s_2 = 2\\
        &\quad 0i + 0j - 1k + s_3 = 0\\
        &\quad 1i + 1j + 0k - s_4 = 2\\
        &\quad i, j, k, s_1, s_2, s_3, s_4 \geq 0
    \end{align*}
    
    Using an algorithm such as the simplex algorithm, we see that there is no feasible solution, and so we conclude that the constraint judgement $\{i, j, k\};\{C^1, C^2, C^3\} \vDash C^{new}$ is satisfied.
    
    
    %%%%%%%%%%%%%%%%%%%%%%%
    
    % We first represent the four constraints as vectors in terms of some ordered set $\varphi$ of index variables and some ordered set $\varphi$ of exponent vectors.\\
    
    % Let $\varphi = \{i, j, k\}$ and $\varphi = \{\evect{1 0 0}, \evect{0 1 0}, \evect{0 0 1}, \evect{0 0 0}\}$. The constraints $C^1, C^2, C^3, C^{new}$ can now be written as the following vectors
    % %
    % \begin{align*}
    %     \mathbf{C}^1_\varphi &= \cvect{1 0 0 -3}\\
    %     \mathbf{C}^2_\varphi &= \cvect{0 1 1 -2}\\
    %     \mathbf{C}^3_\varphi &= \cvect{0 0 -1 0}\\
    %     \mathbf{C}^{new}_\varphi &= \cvect{2 3 2 -15}
    % \end{align*}
    
    % With the constraints now represented as vectors, we can prepare the equation $M\mathbf{a} = \mathbf{b}$ representing the conical combination, for which we wish to check if a solution exists given given the requirement that $\mathbf{a} \geq \mathbf{0}$. We first prepare the matrix $M$, where we represent the constraint vectors as column vectors
    % %
    % \begin{align*}
    %     &M = \vect{$\mathbf{C}^1_\varphi$ $\mathbf{C}^2_\varphi$ $\mathbf{C}^3_\varphi$ $\bm{\beta}_1$ $\bm{\beta}_2$ $\bm{\beta}_3$ $\bm{\beta}_4$}\\
    %     &\quad \text{where } \bm{\beta}_1 = \cvect{-1 0 0 0}, \bm{\beta}_2 = \cvect{0 {-1} 0 0}, \bm{\beta}_3 = \cvect{0 0 {-1} 0}, \bm{\beta}_4 = \cvect{0 0 0 {-1}}
    % \end{align*}
    % %
    % We include vectors $\bm{\beta}_i, i \in \{1, 2, 3, 4\}$ to ensure we can also use subconstraints of $\mathbf{C}^i, i \in \{1, 2, 3\}$ when checking if we can construct $\mathbf{C}^{new}_\varphi$. To check if a solution exists to the aforementioned equation, we solve the following linear program to check if $\mathbf{s} = \mathbf{0}$
    % \begin{align*}
    %     \text{minimize}&\quad \mathbf{1}^T\mathbf{s}\\
    %     \text{subject to}&\quad M\mathbf{a} + \mathbf{s} = \mathbf{C}^{new}_\varphi\\
    %     &\quad\mathbf{a},\mathbf{s} \geq \mathbf{0}
    % \end{align*}
    
    % This is possible given $\mathbf{a} = \vect{2 3 1 0 0 0 3}$, and so a solution exists to the canonical combination. Notice that we had to use the additional $\bm{\beta}$ vectors when constructing the conical combination. This shows the importance of subconstraints when checking type judgements.
\end{examp}
%
% \section{Soundness}
% %


% \begin{theorem}[Subject reduction]\label{theorem:srbg}
% If $\varphi;\Phi;\Gamma\vdash P \triangleleft K$ and $P \leadsto Q$ then $\varphi;\Phi;\Gamma\vdash Q \triangleleft K'$ with $\varphi;\Phi\vDash k' \leq K$.
% \begin{proof} by induction on the rules defining $\leadsto$.
%     \begin{description}
%     \item[$\runa{R-rep}$]
%     %
%     \item[$\runa{R-comm}$]
%     %
%     \item[$\runa{R-zero}$]
%     %
%     \item[$\runa{R-par}$]
%     %
%     \item[$\runa{R-succ}$]
%     %
%     \item[$\runa{R-empty}$]
%     %
%     \item[$\runa{R-res}$]
%     %
%     \item[$\runa{R-cons}$]
%     %
%     \item[$\runa{R-struct}$]
%     %
%     %\item[$\runa{R-tick}$] We have that $P = \tick{P'}$ and $Q=P'$. Then by $\runa{S-tick}$ we have $\varphi;\Phi;\Gamma\vdash $
%     \end{description}
% \end{proof}
% \end{theorem}

% \begin{lemma}\label{lemma:timeredtype}
% If $\varphi;\Phi;\Gamma\vdash P \triangleleft K$ with $P\!\not\!\leadsto$ and $P \Longrightarrow^{-1} Q$ then $\varphi;\Phi;\downarrow_1\!\Gamma\vdash Q \triangleleft K'$ with $\varphi;\Phi\vDash K' \leq K + 1$.
% \begin{proof}
    
% \end{proof}
% \end{lemma}

% \begin{theorem}\label{theorem:ubbg}
% If $\varphi;\Phi;\Gamma\vdash P \triangleleft K$ and $P \hookrightarrow^n Q$ then $\varphi;\Phi\vDash n \leq K$.
% \begin{proof} by induction on the number of time reductions $n$ in the sequence $P \hookrightarrow^n Q$.
    
% \end{proof}
% \end{theorem}


% % \begin{description}
% %     \item[$\runa{S-nil}$]
% %     %
% %     \item[$\runa{S-tick}$]
% %     %
% %     \item[$\runa{S-nu}$]
% %     %
% %     \item[$\runa{S-nmatch}$]
% %     %
% %     \item[$\runa{S-lmatch}$]
% %     %
% %     \item[$\runa{S-par}$]
% %     %
% %     \item[$\runa{S-iserv}$]
% %     %
% %     \item[$\runa{S-ich}$]
% %     %
% %     \item[$\runa{S-och}$]
% %     %
% %     \item[$\runa{S-oserv}$]
% %     \end{description}


% %
% It is worth noting that the Simplex algorithm does not guarantee an integer solution, and so we may get indices in constraints where the coefficients may be non-integer values. However, we can use the fact that any feasible linear programming problem with rational coefficients also has an (optimal) solution with rational values \cite{keller2016applied}. We use this fact and Lemma \ref{lemma:constraintcommonden} and \ref{lemma:constraintscaling} to show that we need not to worry about the solution given by the Simplex algorithm, given a rational linear programming problem. Definition \ref{def:constraintequivalence} defines what it means for constraints to be equivalent.

% \begin{defi}[Conditional constraint equivalence]\label{def:constraintequivalence}
%     Let $C_1$, $C_2$ and $C\in\Phi$ be linear constraints with integer coefficients and unknowns in $\varphi$. We say that $C_1$ and $C_2$ are equivalent with respect to $\varphi$ and $\Phi$, denoted $C_1 =_{\varphi;\Phi} C_2$, if we have that
%     \begin{equation*}
%     \mathcal{M}_\varphi(\{C_1\} \cup \Phi) = \mathcal{M}_\varphi(\{C_2\} \cup \Phi) %\mathcal{M}_\varphi(\{C_0\})
% \end{equation*}
% where $\mathcal{M}_{\varphi'}(\Phi')=\{\rho : \varphi' \rightarrow \mathbb{N} \mid \rho \vDash C\;\text{for}\; C \in \Phi'\}$ is the model space of a set of constraints $\Phi'$ over a set of index variables $\varphi'$.
%     %
%     %
%     %$\varphi;\Phi\vDash C_1$ if and only if $\varphi;\Phi\vDash C_2$.
%     %Two normalized constraints $C_1$ and $C_2$ are said to be \textit{equivalent} if for any index valuation $\rho$, we have that $\rho \vDash C_1$ if and only if $\rho \vDash C_2$.
% \end{defi}

% \begin{lemma}\label{lemma:constraintscaling}
% Let $I \leq 0$ be a linear constraint with unknowns in $\varphi$. Then $I \leq 0 =_{\varphi;\Phi} n I \leq 0$ for any $n>0$ and set of constraints $\Phi$.
% \begin{proof}
%     This follows from the fact that if $I \leq 0$ is satisfied, then the sign of $I$ must be non-positive, and so the sign of $n I$ must also be non-positive as $n > 0$. Conversely, if $I \leq 0$ is not satisfied, then the sign of $I$, must be positive and so the sign of $n I$ must also be positive.
% \end{proof}
% \end{lemma}

% \begin{lemma}\label{lemma:constraintcommonden}
% Let $I \leq 0$ be a normalized linear constraint with rational coefficients and unknowns in $\varphi$. Then there exists a normalized linear constraint $I' \leq 0$ with integer coefficients and unknowns in $\varphi$ such that $I \leq 0 =_{\varphi;\Phi} I' \leq 0$ for any set of constraints $\Phi$.% there exists an equivalent constraint $I' = \normlinearindex{n'}{I'}$ where $n', I'_{\alpha_1}, \dots,I'_{\alpha_{m}}$ are integers.
% \begin{proof}
%     It is well known that any set of rationals has a common denominator, whose multiplication with any rational in the set yields an integer. One is found by multiplying the denominators of all rationals in the set. As the coefficients of $I$ are non-negative, this common denominator must be positive. By Lemma \ref{lemma:constraintscaling}, we have that $I\leq 0 =_{\varphi;\Phi} n I \leq 0$ where $n$ is a positive number and $\Phi$ is any set of constraints.% the constraint $I \leq 0$ is equivalent to $d I \leq 0$.
% \end{proof}
% \end{lemma}

% %%
% % \begin{lemma}
% % Let $I \leq J$ and $C\in\Phi$ be a linear constraints with integer coefficients and unknowns in $\varphi$. Then $I \leq J =_{\varphi;\Phi} \mathcal{N}(I\leq J)$ if for any subtraction $K - L$ in $I$ or $J$, we have $\varphi;\Phi\vDash L \leq K$. 
% % \begin{proof}
    
% % \end{proof}
% % \end{lemma}
% % %

% % % \begin{lemma}
% % % Let $C$ and $C'\in\Phi$ be normalized linear constraints with integer coefficients and unknowns in $\varphi$. Then $\varphi;\Phi\nvDash C$ if there does not exist $\mathbf{C}^{new}_\varphi\in\text{coni}(\Phi_\varphi \cup \{\mathbf{0}\})$ with $\mathbf{C}_\varphi\leq \mathbf{C}^{new}_\varphi$.
% % % \begin{proof}
    
% % % \end{proof}
% % % \end{lemma}


% % %
% % \begin{theorem}
% % Let $C$ and $C'\in\Phi$ be normalized linear constraints with integer coefficients and unknowns in $\varphi$. Then $\varphi;\Phi\vDash_{\mathbb{R}^{\geq 0}} C$ if and only if there exists $\textbf{C}^{new}_\varphi\in\text{coni}(\Phi_\varphi \cup \{\mathbf{0}\})$ with $\textbf{C}_\varphi\leq \textbf{C}^{new}_\varphi$.
% % \begin{proof}
% %     We consider the implications separately
% %     \begin{enumerate}
% %         \item Assume that $\varphi;\Phi\vDash_{\mathbb{R}^{\geq 0}} C$. Then for all valuations $\rho : \varphi \longrightarrow \mathbb{R}^{\geq 0}$ such that $\rho\vDash \Phi$ we also have $\rho\vDash C$, or equivalently $([\![I_1]\!]_\rho \leq 0) \land \cdots \land ([\![I_n]\!]_\rho \leq 0) \implies [\![I]\!]_\rho \leq 0$, where $C = I_0 \leq 0$ and $C_i = I_i \leq 0$ for $C_i\in \Phi$. We show by contradiction that this implies there exists $\textbf{C}^{new}_\varphi\in\text{coni}(\Phi_\varphi \cup \{\mathbf{0}\})$ with $\textbf{C}_\varphi\leq \textbf{C}^{new}_\varphi$. Assume that such a conical combination does not exist. Then for all $\mathbf{C}'_\varphi\in\text{coni}(\Phi_\varphi \cup \{\mathbf{0}\})$ there is at least one coefficient ${\mathbf{C}'_\varphi}_k$ for some $0\leq k \leq |\varphi|$ such that ${\mathbf{C}'_\varphi}_k < {\mathbf{C}_\varphi}_k$. We show that this implies there exists $\rho\in\mathcal{M}_\varphi(\Phi)$ such that $\rho\nvDash C$.\\ 
        
        
% %         and so there must exist $\rho : \varphi \longrightarrow \mathbb{R}^{\geq 0}$ such that $\rho\vDash C'$ and $\rho\nvDash C$. However, as $\varphi;\Phi\vDash_{\mathbb{R}^{\geq 0}} C$ holds there must be some constraint $C''\in\Phi$ such that $\rho\nvDash C''$.\\
        
        
% %         Assume that there does not exist a conical combination $\textbf{C}^{new}_\varphi\in\text{coni}(\Phi_\varphi \cup \{\mathbf{0}\})$ with $\textbf{C}_\varphi\leq \textbf{C}^{new}_\varphi$. \\
        
        
        
% %         We have that $C = n + \sum_{\alpha\in\mathcal{E}(I)} I_\alpha i_\alpha$, and so for $\rho\vDash n + \sum_{\alpha\in\mathcal{E}(I)} I_\alpha i_\alpha$ to hold, it must be that $[\![\sum_{\alpha\in\mathcal{E}(I)} I_\alpha i_\alpha]\!]_\rho \leq -n$. This implies that $\Phi$ contains constraints that collectively bound the sizes of index variables that appear in $\sum_{\alpha\in\mathcal{E}(I)} I_\alpha i_\alpha$, such that $\sum_{\alpha\in\mathcal{E}(I)} I_\alpha i_\alpha$ cannot exceed $-n$. We now show by contradiction that there must then exist $\textbf{C}^{new}_\varphi\in\text{coni}(\Phi_\varphi \cup \{\mathbf{0}\})$ such that $\textbf{C}_\varphi\leq \textbf{C}^{new}_\varphi$. Assume that such a conical combination does not exist. Then it must be that for any $C'_\varphi\in\text{coni}(\Phi \cup \{\mathbf{0}\})$, at least one coefficient in $C'_\varphi$ is smaller than the corresponding coefficient in $C_\varphi$, implying that $C$ imposes a new restriction on valuations. Thus, there must exist a valuation $\rho$ such that $\rho\vDash\Phi$ but $\rho\nvDash C$, but then we have that $\varphi;\Phi\nvDash C$, and so we have a contradiction.
        
% %         \item Assume that there exists $\mathbf{C}^{new}_\varphi\in\text{coni}(\Phi_\varphi \cup \{\mathbf{0}\})$ with $\mathbf{C}_\varphi \leq \mathbf{C}^{new}_\varphi$. Then by Lemma \ref{TODO}, we have that $\varphi;\phi\vDash C^{new}$ and by Lemma \ref{TODO} it follows from $\mathbf{C}_\varphi \leq \mathbf{C}^{new}_\varphi$ that also $\varphi;\Phi\vDash C$.
% %     \end{enumerate}
% % \end{proof}
% % \end{theorem}
\subsection{Reducing polynomial constraints to linear constraints}\label{sec:verifyingpolynomial}

Many programs do not run in linear time, and so we cannot type them if we are constrained to just verifying linear constraint judgements. In this section we show how we can reduce certain polynomial constraints to linear constraints, enabling us to use the techniques described above. We first extend our definition of indices such that they can be used to express multivariate polynomials. We assume a normal form for polynomial indices akin to that of linear indices. Terms are now monomials with integer coefficients.
%
\begin{align*}
        I,J ::= n \mid i \mid I + J \mid I - J \mid I J
\end{align*}
%
When reducing normalized constraints with polynomial indices to normalized constraints with linear indices, we wish to construct new linear constraints that are only satisfied if the original polynomial constraint is satisfied. For example, given the constraint ${-i^2 + 10 \leq 0}$, one can see that this polynomial constraint can be simulated using the constraint $-i + \sqrt{10} \leq 0$. We notice that the reason this is possible is that when ${-i^2 + 10 \leq 0}$ holds, i.e. when $i \geq \sqrt{10}$, the value of $i$ can always be increased without violating the constraint. Similarly, when ${-i^2 + 10 \geq 0}$ holds, the value of $i$ can always be decreased until reaching its minimum value of 0 without violating the constraint. We can thus introduce a new simpler constraint with the same properties, i.e. $-i + \sqrt{10} \leq 0$. More specifically, the polynomials of the left-hand side of the two constraints share the same positive real-valued roots as well as the same sign for any value of $i$. In general, limiting ourselves to univariate polynomials, for any constraint whose left-hand side polynomial only has a single positive root, we can simulate such a constraint using a constraint of the form $a \cdot i + c \leq 0$. For describing complexities of programs, we expect to mostly encounter monotonic polynomials with at most a single positive real-valued root.\\

Note that the above has the consequence that we may end up with irrational coefficients for indices of constraints. While such constraints are not usually allowed, they can still represent valid bounds for index variables. As such, we can allow them as constraints in this context. Furthermore, any irrational coefficient can be approximated to an arbitrary precision using a rational coefficient.\\

For finding the roots of a specific polynomial we can use either analytical or numerical methods. Using analytical methods has the advantage of being able to determine all roots with exact values, however, we are limited to polynomials of degree at most four as stated by the Abel-Ruffini theorem \cite{abelruffinitheorem}. With numerical methods, we are not limited to polynomials of a specific degree, however, numerical methods often require a given interval to search for a root and do not guarantee to find all roots. Introducing constraints with false restrictions may lead to an under-approximating type system, and so we must be careful not to introduce such. We must therefore ensure we find all roots to avoid constraints with false restrictions. We can use Descartes rule of signs to get an upper bound on the number of positive real roots of a polynomial. Descarte's rule of signs states that the number of roots in a polynomial is at most the number of sign-changes in its sequence of coefficients.\\

For our application, we decide to only consider constraints whose left-hand side polynomial is univariate and monotonic with a single root. In some cases we may remove safely positive monomials in a normalized constraint to obtain such constraints. We limit ourselves to these constraints both to keep complexity down, as well as because we expect to mainly encounter such polynomials when considering complexity analysis of programs. We use Laguerre's method as a numerical method to find the root of the polynomial, which has the advantage that it does not require any specified interval when performing root-finding. Assuming we can find a root $r$, we add an additional constraint $\pm (i - r) \leq 0$ where the sign depends on whether the original polynomial is increasing or decreasing.\\

Additionally, given a non-linear monomial, we may also treat this as a single unit and construct linear combinations of this by treating the monomial as a single fresh variable. For example, given normalized constraints $C_1 = i^2 + 4 \leq 0$, $C_2 = i - 2 \leq 0$, and $C_3 = 2ij \leq 0$, we may view these as the linear constraints $C_1' = k + 4 \leq 0$, $C_2' = i - 2 \leq 0$, and $C_3' = 2l \leq 0$ where $k = i^2$ and $l = ij$. This may make the feasible region larger than it actually is, meaning we over-approximate when verifying judgements on polynomial constraints.

\begin{examp}
    We want to check if the following judgement holds
    $$\{i, j\}; \{-2i \leq 0, -1i^2 + 1j + 1 \leq 0\} \vDash -2i + 2 \leq 0$$
    
    %We notice that $-2i + 2$ cannot be written as a conical combination of the polynomials $-2i$ and $-1i^2 + 1j + 1$.\\
    
    We first try to generate new constraints of the form $-a \cdot i + r \leq 0$ for some index variable $i$ and some constants $a$ and $r$ based on our two existing constraints using the root-finding method. The first constraint is already of such form, so we can only consider the second. For the second, we first use the subconstraint relation to remove the term $1j$ obtaining $-1i^2 + 1 \leq 0$. Next, we note that $-1i^2 + 1$ is a monotonically decreasing polynomial as every coefficient excluding the constant term is negative. We then find the root $r = 1$ of the polynomial and add a new constraint $-1i + 1 \leq 0$ to our set of constraints. Finally, we can invert the constraint $-2i + 2 \leq 0$ obtaining the constraint $-2i + 1 \geq 0$. We now need to check if the feasible region $\mathcal{M}_{\{i, j, i^2\}}(\{-2i \leq 0, -1i^2 + 1j + 1 \leq 0, -1i + 1 \leq 0, -2i + 1 \geq 0\})$ is empty. Treating the monomial $i^2$ as its own separate variable and solving this as a linear program using an algorithm such as the simplex algorithm, we see that there is indeed no solution, and so the constraint is satisfied. 
\end{examp}

\subsection{Trivial judgements}
We now show how some judgements may be verified without neither transforming constraints into linear constraints nor solving any integer programs. To do so, we consider an example provided by Baillot and Ghyselen \cite{BaillotGhyselen2021}, where we exploit the fact that all coefficients in the normalized constraints are non-positive. Judgements with such constraints can be answered in linear time with respect to the number of monomials in the normalized equivalent of constraint $C$. That is if all coefficients in the normalized constraint are non-positive, we can guarantee that the constraint is always satisfied, recalling that only naturals substitute for index variables. Similarly, if there are no negative coefficients and at least one positive coefficient, we can guarantee that the constraint is never satisfied.\\

In practice, it turns out that we can type check many processes by simply over-approximating constraint judgements using pair-wise coefficient inequality constraints. In Example \ref{example:baillotghyssimple}, we show how all constraint judgements in the typings of both a linear and a polynomial time replicated input can be verified using this approach. 
%
\begin{examp}\label{example:baillotghyssimple}
Baillot and Ghyselen \cite{BaillotGhyselen2021} provide an example of how their type system for parallel complexity of message-passing processes can be used to bound the time complexity of a linear, a polynomial and an exponential time replicated input process. We show that we can verify all judgements on constraints in the typings of the first two processes using normalized constraints. We first define the processes $P_1$ and $P_2$
\begin{align*}
    P_i \defeq\; !\inputch{a}{n,r}{}{\tick\match{n}{\asyncoutputch{r}{}{}}{m}{\newvar{r'}{\newvar{r''}{Q_i}}}}
\end{align*}
for the corresponding definitions of $Q_1$ and $Q_2$
\begin{align*}
    Q_1 \defeq&\; \asyncoutputch{a}{m,r'}{} \mid \asyncoutputch{a}{m,r''}{} \mid \inputch{r'}{}{}{\inputch{r''}{}{}{\asyncoutputch{r}{}{}}}\\
    Q_2 \defeq&\; \asyncoutputch{a}{m,r'}{} \mid \inputch{r'}{}{}{(\asyncoutputch{a}{m,r''}{} \mid \asyncoutputch{r}{}{})} \mid \asyncinputch{r''}{}{}
\end{align*}
We type $Q_1$ and $Q_2$ under the respective contexts $\Gamma_1$ and $\Gamma_2$
\begin{align*}
    \Gamma_1 \defeq&\; a : \forall_0 i.\texttt{oserv}^{i+1}(\texttt{Nat}[0,i],\texttt{ch}_{i+1}()), n : \texttt{Nat}[0,i], m : \texttt{Nat}[0,i-1],\\ &\; r : \texttt{ch}_{i}(),
     r' : \texttt{ch}_{i}(), r'' : \texttt{ch}_i()\\
    %
    \Gamma_2 \defeq&\; a : \forall_0 i.\texttt{oserv}^{i^2+3i+2}(\texttt{Nat}[0,i],\texttt{ch}_{i+1}()), n : \texttt{Nat}[0,i], m : \texttt{Nat}[0,i-1],\\ &\; r : \texttt{ch}_{i}(),
     r' : \texttt{ch}_i(), r'' : \texttt{ch}_{2i-1}()
\end{align*}
Note that in the original work, the bound on the complexity of server $a$ in context $\Gamma_2$ is $(i^2+3i+2)/2$. However, we are forced to use a less precise bound, as the multiplicative inverse is not always defined for our view of indices. Upon typing process $P_1$, we amass the judgements on the left-hand side, with corresponding judgements with normalized constraints on the right-hand side
%
\begin{align*}
    &\{i\};\emptyset\vDash i + 1 \geq 1\kern7.5em\Longleftrightarrow &  \{i\};\emptyset\vDash -i \leq 0\\
   % &\{i\};\{i \geq 1\} \vDash i-1 \leq i \kern6em\Longleftrightarrow & \{i\};\{1-i \leq 0\} \vDash -1 \leq 0\\
    &\{i\};\{i \geq 1\} \vDash i \leq i + 1\kern5em\Longleftrightarrow &  \{i\};\{1-i \leq 0\} \vDash -1 \leq 0\\
    %
    &\{i\};\{i \geq 1\} \vDash i \geq i\kern6.7em\Longleftrightarrow &  \{i\};\{1-i \leq 0\} \vDash 0 \leq 0\\
    &\{i\};\{i \geq 1\} \vDash 0 \geq 0\kern6.45em\Longleftrightarrow &  \{i\};\{1-i \leq 0\} \vDash 0 \leq 0
\end{align*}
%
As all coefficients in the normalized constraints are non-positive, each judgement is trivially satisfied, and we can verify the bound $i + 1$ on server $a$. For process $P_2$ we correspondingly have the trivially satisfied judgements
\begin{align*}
    &\{i\};\emptyset\vDash i + 1 \geq 1\kern10.3em\Longleftrightarrow &  \{i\};\emptyset\vDash -i \leq 0\\
    &\{i\};\{0 \leq 0\}\vDash i \leq i^2 + 3i + 2\kern5.2em\Longleftrightarrow &  \{i\};\{0\leq 0\}\vDash -i^2-2i-2 \leq 0\\
    %
    &\{i\};\{i \geq 1\} \vDash i \leq i + 1\kern7.9em\Longleftrightarrow &  \{i\};\{1-i \leq 0\} \vDash -1 \leq 0\\
    %
    &\{i\};\{i \geq 1\} \vDash 2i-1 \leq i^2 + 3i + 2\kern3.2em\Longleftrightarrow &  \{i\};\{1-i \leq 0\} \vDash -i^2-i-3 \leq 0\\
    &\{i\};\{i \geq 1\} \vDash i \geq i\kern9.7em\Longleftrightarrow &  \{i\};\{1-i \leq 0\} \vDash 0 \leq 0\\
    %
    &\{i\};\{i \geq 1\} \vDash 0 \geq i^2+i\kern7.5em\Longleftrightarrow &  \{i\};\{1-i \leq 0\} \vDash -i^2-i \leq 0\\
    %
    &\{i\};\{i \geq 1\} \vDash i^2+2i \geq i^2+3i+2\kern2.9em\Longleftrightarrow &  \{i\};\{1-i \leq 0\} \vDash -i-2 \leq 0
\end{align*}

\end{examp}

% In the general sense, however, verifying whether judgements on polynomial constraints are satisfied is a difficult problem, as it amounts to verifying that a constraint is satisfied under all interpretations that satisfy our set of known constraints. In Example \ref{example:needconic}, we show how a constraint that is not satisfied for all interpretations can be shown to be covered by a set of two constraints, by utilizing the transitive, multiplicative and additive properties of inequalities to combine the two constraints. More specifically, we can exploit the fact that we can generate new constraints from any set of normalized constraints by taking a \textit{conical} combination of their left-hand side indices, as we shall formalize in the following section.
% %
% \begin{examp}\label{example:needconic}
%     Given the judgement
%     \begin{align*}
%         \{i\};\{i \leq 3, 5 \leq i^2\} \vDash 5i \leq 3i^2
%     \end{align*}
%     we want to verify that constraint $5i \leq 3i^2$ is covered by the set of constraints $\{i\leq 3, 5 \leq i^2\}$. This constraint is not satisfied by all interpretations, as substituting $1$ for $i$ yields $5 \not\leq 3$. However, we can rearrange and scale the constraints $i\leq 3$ and $5\leq i^2$ as follows
%     \begin{align*}
%         i \leq 3 \iff i-3\leq 0 \implies 5i - 15 \leq 0\\
%         %
%         5\leq i^2 \iff 0 \leq i^2-5 \implies 0 \leq 3i^2-15
%     \end{align*}
%     Then it follows that
%     \begin{align*}
%         5i-15 \leq 3i^2-15 \iff 5i \leq 3i^2
%     \end{align*}
%     %More generally, we can use the transitive, multiplicative and additive properties of the inequality relation to construct new constraints from a set of known constraints, thereby verifying that some constraint does not impose new restrictions on interpretations. 
% \end{examp}
% %
% \subsection{An alternative method for verifying univariate polynomial constraints}
% In this section we restrict ourselves to constraints whose left-hand side normalized indices are monotonic univariate polynomials. These constraints have a number of convenient properties we can take advantage of to greatly simplify the process of verifying whether a constraint judgement holds. Namely, these constraints perfectly divide the index variable $i$ of the polynomial into two intervals $[-\infty,n[$ and $[n, \infty]$ where for any value of $i$ in either the first or the second interval, the constraint is satisfied and for any $i$ in the other interval, it is not. The only exception to this is for constraints where its left-hand side index is constant, in which case $n = \pm\infty$. As such, we can describe the behavior of a monotonic univariate polynomial constraint using a single value as well as a sign denoting whether the polynomial is increasing or decreasing.\\

% The point $n$ that divides the range that satisfies a constraint from the range that does not satisfy the constraint, corresponds to the root of the left-hand side of the normalized constraint. This is similar to the method described in Section \ref{sec:verifyingpolynomial}, except we now only store the range that satisfies the constraint. Here we take advantage of the fact that the polynomial is monotonic, to ensure that there is only a single root. Verifying whether a constraint $I \leq 0$ covers another constraint $J \leq 0$ then amounts to comparing the roots of $I$ and $J$. This method can be extended to non-monotonic polynomials by simply considering sequences of intervals satisfying constraints and comparing sequences of intervals satisfying a constraint.

% \begin{examp}
%     Given the three constraints
%     %
%     \begin{align*}
%         C_1&: -i^2 + 10 \leq 0\\
%         C_2&: -i + 2 \leq 0\\
%         C_3&: -5 i^3 + 80 i^2 - 427 i + 758 \leq 0
%     \end{align*}
    
%     we want to check if the judgement $\{i\};\{C_1, C_2, C_3\} \vDash -i + 4 \leq 0$ holds. We first find the non-negative roots $r_1$, $r_2$ and $r_3$ of $C_1$, $C_2$ and $C_3$. This can be done either numerically or analytically as all polynomials are of degree $\leq 4$. The roots are $r_1 = \sqrt{10} \approx 3.16$, $r_2 = 2$ and $r_3 \approx 4.59$. The root of $-i + 4$ is $4$, and so we must check if any of the roots $r_1$, $r_2$ and $r_3$ are greater than or equal $4$. In this case, $r_3 \geq 4$, and so the judgement $\{i\};\{C_1, C_2, C_3\} \vDash -i + 4 \leq 0$ holds.
% \end{examp}

%If our indices are univariate polynomials, we can express the feasible region of a constraint as a sequence of disjoint intervals. Then for a constraint $I \leq 0$ such that $I$ is in index variable $i$, the interpretation $I\{n/i\}$ with $n\in\mathbb{N}$ is satisfied when $n$ is within one of the intervals representing the feasible region of $I\leq 0$. We can utilize this to determine whether the feasible region of one constraint contains the feasible region of another by computing their intersection. This can be generalized to judgements on constraints, such that we can verify whether one constraint imposes new restrictions on possible interpretations. Then the question remains \textit{how do we find a sequence of disjoint intervals that corresponds to the feasible region of constraint $I \leq 0$?}\\ 

%We can find such a sequence of intervals for a normalized constraint, by computing the roots of the corresponding index. For polynomials of degree $4$ or less, there exists exact analytical methods to compute the roots, and we can approximate them in the general case using numerical methods.

% \begin{lstlisting}[escapeinside={(*}{*)}]
% intersectIntervals(is1, is2):
%     if is1 or is2 is empty:
%         return empty list
        
%     i1 (*$\longleftarrow$*) head(is1)
%     i2 (*$\longleftarrow$*) head(is2)
%     ires (*$\longleftarrow$*) i1 (*$\cap$*) i2
    
%     if max(i1) > max(i2):
%         is2' (*$\longleftarrow$*) tail(is2)
%         intersectedIntervals (*$\longleftarrow$*) intersectIntervals(is1, is2')
%     else:
%         is1' (*$\longleftarrow$*) tail(is1)
%         intersectedIntervals (*$\longleftarrow$*) intersectIntervals(is1', is2)
    
%     if ires is not the empty interval:
%         add ires to intersectedIntervals as the head
    
%     return intersectedIntervals
% \end{lstlisting}


% \begin{lstlisting}[escapeinside={(*}{*)}]
% containsIntervals(is1, is2):
%     if is2 is empty:
%         return true
%     else if is1 is empty:
%         return false
    
%     i1 (*$\longleftarrow$*) head(is1)
%     i2 (*$\longleftarrow$*) head(is2)
%     ires (*$\longleftarrow$*) i1 (*$\cap$*) i2
    
%     if ires = i2:
%         is2' (*$\longleftarrow$*) tail(is2)
%         return containsIntervals(is1, is2')
%     else:
%         is1' (*$\longleftarrow$*) tail(is1)
%         return containsIntervals(is1', is2)
% \end{lstlisting}


% \begin{lstlisting}[escapeinside={(*}{*)}]
% findIntervals((*$\{i\}$*), (*$I \leq 0$*)):
%     roots (*$\longleftarrow$*) sorted list of all positive roots of (*$I$*)
    
%     if roots is empty:
%         if (*$I\{0/i\}$*) (*$\leq$*) 0:
%         return singleton list of [0, (*$\infty$*)]
%     else:
%         return empty list
    
%     low (*$\longleftarrow$*) 0
%     satisfiedIntervals (*$\longleftarrow$*) empty list

%     if head(roots) = 0:
%         roots (*$\longleftarrow$*) tail(roots)

%     while roots is not empty:
%         high (*$\longleftarrow$*) head(roots)
%         mid (*$\longleftarrow$*) (*$(\texttt{low} + \texttt{high})/2$*)
        
%         if (*$I\{\texttt{mid}/i\}$*) (*$\leq$*) 0:
%             append [low,high] to satisfiedIntervals
            
%         low (*$\longleftarrow$*) high
%         roots (*$\longleftarrow$*) tail(roots)
    
%     if (*$I\{(\texttt{low}+1)/i\}$*) (*$\leq$*) 0:
%         append [low, (*$\infty$*)] to satisfiedIntervals
    
%     return satisfiedIntervals
% \end{lstlisting}


% \begin{lstlisting}[escapeinside={(*}{*)}]
% checkJudgement((*$\{i\}$*), (*$\Phi$*), (*$I\leq 0$*)):
%     satisfiedIntervals (*$\longleftarrow$*) singleton list of [0, (*$\infty$*)]
    
%     for (*$(J \leq 0) \in \Phi$*):
%         isJ (*$\longleftarrow$*) findIntervals((*$\{i\}$*), (*$J\leq 0$*))
%         satisfiedIntervals (*$\longleftarrow$*) intersectIntervals(satisfiedIntervals, isJ)
        
%     isI (*$\longleftarrow$*) findIntervals((*$\{i\}$*), (*$I \leq 0$*))
        
%     return containsInterval(isI, satisfiedIntervals)
% \end{lstlisting}
\section{Examples of invalid configurations}
The following examples are written in the format $\conf{E, a}$, where $E$ is an editor expression and $a$ is the AST on which we apply the editor expression. \\

In equation \ref{condsubproblem} we show how conditioned substitution can cause problems.
\begin{equation}
    \conf{\left(@\texttt{break} \Rightarrow \replace{\texttt{break}}\right) \ggg \texttt{child}\; 1,\; \lambda x.\hole\; \cursor{\breakpoint{c}}} \label{condsubproblem}
\end{equation}
 In the example we check if the cursor is at a breakpoint, and since the check is true we \textit{toggle} the breakpoint thereby making the following \texttt{child} 1 command problematic. The constant c cannot have a child which means this configuration would cause a run-time error. \\
 
In equation \ref{parentproblem} we show how using the \texttt{parent} command can cause problem when the root is unknown.
\begin{equation}
    \conf{\left(\lozenge\texttt{hole} \Rightarrow \texttt{parent}\right) \ggg \texttt{parent},\; \cursor{\lambda x.\hole}\; c} \label{parentproblem}
\end{equation}
In the example we first check if there is a hole in some subtree of the current cursor. This condition holds and we therefore evaluate the \texttt{parent} command resulting in the AST $\cursor{\lambda x.\hole\; c}$. When the next \texttt{parent} command is evaluated we have a run-time error since we are already situated at the root.\\

In equation \ref{astproblem} we show how an editor expression can result in an AST that would cause a run-time error when evaluated.
\begin{equation}
    \conf{\left(\neg\Box(\texttt{lambda}\; x) \Rightarrow \texttt{child}\; 1\right) \ggg \replace{\texttt{var}\; x}.\texttt{eval},\; \cursor{\lambda x.\hole}\; c} \label{astproblem}
\end{equation}
In the example we first check if it is \textbf{not} necessary that the subtree of the cursor contains a lambda expression. This condition does not hold since it is necessary. Since the condition does not hold we do not evaluate the \texttt{child} 1 command, which means the following substitution of \texttt{var} x is problematic. The substitution results in the AST $\cursor{\texttt{var}\; x}\; c$, which causes a run-time error when the command \texttt{eval} is evaluated, since the left child of the function application is no longer a function.
%
\section{Over-approximations}
As we cannot determine statically whether a condition holds, we establish over-approximations to ensure run-time errors cannot occur in well-typed configurations. As equation \ref{parentproblem} shows, conditioned expressions can result in loss of information about the cursor location. As such, we enforce the cursor \textit{depth} in the tree to be the same before and after a conditioned expression. Furthermore, the first cursor movement in a conditioned expression must be a \texttt{child} prefix. As equation \ref{condsubproblem} shows, conditioned substitution also results in loss of information. Thus, we can no longer guarantee that subsequent substitution at a deeper level is well-typed. Similarly, we no longer know of the structure of the subtree, such that we must condition \texttt{child} prefixes.\\

The above discussion leads to the following list of over-approximations:
\begin{itemize}
    \item In conditioned and recursive expressions, the cursor depth must be the same before and after.
    \item In conditioned and recursive expressions, only the subtree encapsulated by the cursor is accessible.
    \item After conditioned substitution, subsequent substitution at a deeper level is no longer valid, and the \texttt{child} prefix command must be conditioned.
\end{itemize}
%
\section{AST type rules}
\begin{table*}[htp]
    \centering
    \begin{align*}
        \runa{t-var} &\; \infrule{\Gamma_a\left(x\right)=\tau}{\Gamma_a \vdash x : \tau}\\
        %
        \runa{t-const} &\; \infrule{}{\Gamma_a \vdash c : b}\\
        %
        \runa{t-app} &\; \infrule{\Gamma_a \vdash a_1 : \tau_1 \rightarrow \tau_2 \quad \Gamma_a \vdash a_2 : \tau_1}{\Gamma_a \vdash a_1\; a_2 : \tau_2}\\
        %
        \runa{t-lambda} &\; \infrule{\Gamma_a\left[x \mapsto \tau_1\right] \vdash a : \tau_2}{\Gamma_a \vdash \lambda x:\tau_1.a : \tau_1 \rightarrow \tau_2} \\
        %
        \runa{t-break} &\; \infrule{\Gamma_a \vdash a : \tau}{\Gamma_a \vdash \breakpoint{a} : \tau} \\
        %
        \runa{t-hole} &\; \infrule{}{\Gamma_a \vdash \left(\hole : \tau\right) : \tau}
        %
    \end{align*}
    \caption{Type rules for abstract syntax trees.}
    \label{tab:typerules}
\end{table*}

%\section{Type context format}
%Here, we propose a format for type contexts of editor expressions. The context of an editor expression could be a triple $\Psi = (\Gamma_a, \tau, \Gamma)$, where $\Gamma_a$ is the type context for the subtree encapsulated by the cursor, $\tau$ is the type of the subtree and $\Gamma$ is a function or map from prefix command types to editor expression contexts. That is, contexts for editor expressions are recursive. Say we have context $(\Gamma_a, \tau, \Gamma)$. Upon a $\texttt{child}\; 1$ prefix, we \textit{look up} $\texttt{one}$ in $\Gamma$. If $\Gamma(\texttt{one}) = undef$, the expression is not well-typed. Otherwise, we evaluate the prefixed expression in the new context $\Gamma(\texttt{one})$.\\

%We construct the initial context based on the AST in the configuration $\conf{E,\; a}$. Upon a substitution prefix, we modify the context, upon a child or parent prefix, we \textit{move} in the context, and upon a conditioned or recursive expression, we set some of the bindings to $undef$: $\Gamma(T)=undef$.\\

%$\Gamma = T_1 : \Psi_1,...,T_n : \Psi_n$ \\
%$\Psi = (\Gamma_a, \tau, \Gamma)$
%Γ = T1 : Ψ1,..,Tn : Ψn
%Ψ = (Γa, τ, Γ)

\section{Experimental type system}

In this section, we introduce a type system for our editor-calculus. For the type system, we introduce the syntactic categories $\tau \in \mathbf{ATyp}$ to denote types of AST nodes, $T \in \mathbf{CTyp}$ to denote \textit{child} types, and p $\in \mathbf{Pth}$ to denote AST paths.
%
\begin{align*}
    \tau ::=&\; b \mid \tau_1 \rightarrow \tau_2 \mid \breakpoint{\tau} \mid \texttt{indet}\\
    T ::=&\; \texttt{one} \mid \texttt{two}\\
    p ::=&\; p\; T \mid \epsilon
\end{align*}

In addition to the basic and arrow types in $\mathbf{ATyp}$, we include a type for breakpoints, $\breakpoint{\tau}$, and a type to denote indeterminate types, \texttt{indet}. We use $\mathbf{Pth}$ to denote paths in an AST by storing a sequence of \textbb{one} and \textbb{two} which denote if the path goes through the first or second child.\\

We define two sets for contexts in our type system. The first context, $\mathbf{ACtx}$, stores type bindings for variables in the AST. The second context, $\mathbf{ECtx}$, stores, for all available paths so far, a pair of an AST context and the type of the node at the end of the path. We use $\Gamma_a \in \mathbf{ACtx}$ and $\Gamma_e \in \mathbf{ECtx}$ as metavariables for the two contexts. To check if a path $p$ is available in a context $\Gamma_e$, we use the notion $\Gamma_e(p) \neq \text{undef}$. $\mathbf{ACtx}$ and $\mathbf{ECtx}$ are thus defined as the following.
%
\begin{align*}
\mathbf{ACtx} &= \mathbf{Var} \rightharpoonup \mathbf{ATyp}\\
\mathbf{ECtx} &= \mathbf{Pth} \rightharpoonup \left(\mathbf{ACtx} \times \mathbf{ATyp}\right)
\end{align*}

To support our type system, we modify the syntax for AST node modifications by including type annotations for application, abstraction and holes. The new syntax thus becomes the following.
%
\begin{align*}
  D ::= \; & \texttt{var}\;x \mid \texttt{const}\;c \mid \texttt{app} : \tau_1 \rightarrow \tau_2, \tau_1 \mid \texttt{lambda}\; x : \tau_1 \rightarrow \tau_2 \mid \texttt{break} \mid \texttt{hole} : \tau
\end{align*}

To support breakpoint types, we introduce the notion of type consistency into our typesystem. The purpose of consistency in our type system is to ensure breakpoints types are consistent with their respective type, as defined below.
%
\begin{definition}{(Type consistency)}
    We define two types $\tau_1, \tau_2$ to be \textit{consistent}, denoted $\tau_1 \sim \tau_2$, by the following rules.
    \begin{align*}
        \runa{cons-1} \hspace{-1cm}
        \infrule{}{\tau \sim \tau} \hspace{-1cm}
        \runa{cons-2} \hspace{-1cm}
        \infrule{}{\breakpoint{\tau} \sim \tau} \hspace{-1cm}
        \runa{cons-3} \hspace{-1cm}
        \infrule{}{\tau \sim \breakpoint{\tau}} \hspace{-1cm}
        \runa{cons-4}
        \infrule{\tau_1 \sim \tau_1' \quad \tau_2 \sim \tau_2'}{(\tau_1 \rightarrow \tau_2) \sim (\tau_1' \rightarrow \tau_2')}
    \end{align*}
\end{definition}


\begin{table*}[htp]
    \centering
    \begin{align*}
        \runa{ctx-split-1}&\; \infrule{}{\emptyset = p \left(\emptyset\; \circ\; \emptyset\right)}\\
        \runa{ctx-split-2}&\; \infrule{\Gamma_e = p \left({\Gamma_e}_1\; \circ\; {\Gamma_e}_2\right)}{\Gamma_e,\; p\; T_1..T_n: (\Gamma_a,\; \tau) = p \left(\left({\Gamma_e}_1,\; p\; T_1..T_n: (\Gamma_a,\; \tau)\right)\; \circ\; {\Gamma_e}_2\right)}\\
        \runa{ctx-split-3}&\; \infrule{p_1 \neq p_2 \quad \Gamma_e = p_2 \left({\Gamma_e}_1\; \circ\; {\Gamma_e}_2\right)}{\Gamma_e,\; p_1\; T_1..T_n: (\Gamma_a,\; \tau) = p_2 \left({\Gamma_e}_1\; \circ\; \left({\Gamma_e}_2,\; p_1\; T_1..T_n: (\Gamma_a,\; \tau)\right)\right)}\\
        %
        \runa{ctx-update-1}&\; \infrule{}{\Gamma_e = \Gamma_e + \emptyset}\\
        \runa{ctx-update-2}&\; \infrule{\Gamma_e = \left({\Gamma_e}_1,\; p: ({\Gamma_a}_2,\; \tau_2)\right) + {\Gamma_e}_2}{\Gamma_e,\; p: ({\Gamma_a}_1,\; \tau_1) = \left({\Gamma_e}_1,\; p: ({\Gamma_a}_2,\; \tau_2)\right) + {\Gamma_e}_2}\\
        \runa{ctx-update-3}&\; \infrule{\Gamma_e = {\Gamma_e}_1 + {\Gamma_e}_2}{\Gamma_e,\; p: (\Gamma_a,\; \tau) = {\Gamma_e}_1 + \left({\Gamma_e}_2,\; p: (\Gamma_a,\; \tau)\right)}
    \end{align*}
    \caption{Context split and context update for editor contexts.}
    \label{tab:context}
\end{table*}
% We define \textit{type contexts}, $\Gamma_e$ in Table \ref{tab:context} as a mapping from a path $p$ to a pair consisting of an AST context $\Gamma_a$ and AST type $\tau$. We denote the $\Gamma_e, p : (\Gamma_a, \tau)$ as the type context equal to the paths not in the domain of map $\Gamma_e$ except for $p$, where $\Gamma_e(p) = (\Gamma_a, \tau)$. For type contexts we introduce the concept of \textit{context splitting} on a path in terms of $\Gamma_e$ maintained through two sub-contexts $\Gamma_{e1}$ and $\Gamma_{e2}$. For this we require a split-operation $\circ$, defined for two sub-contexts on a path as $\Gamma_e = p(\Gamma_{e1}\; \circ \; \Gamma_{e2})$. Notice the empty context is defined with the symbol $\emptyset$ as in \runa{ctx-split-1}. In rule \runa{ctx-split-2} we have that $p$ is in $\Gamma_{e1}$, but not in $\Gamma_{e2}$. Thus, $p$ is not in $\Gamma = \Gamma_{e1}\; \circ \; \Gamma_{e2}$, which is similarly done for the \runa{ctx-split-3} in terms of $\Gamma_{e1}$.\\

Next we introduce the notion of \textit{context updates} to update bindings in a context with new types for the associated path $p$. We use the addition operator $+$, to denote sum-context $\Gamma$ of two compatible type contexts $\Gamma_{e1}$ and $\Gamma_{e2}$. The rules require linear paths to not have bindings exist in another context. Thus, we can only update a context $\Gamma_{e2}$ iff no bindings for a given path is in context $\Gamma_{e1}$. In rule \runa{ctx-update-2} we have bindings in $\Gamma_{e1}$, which means we cannot add bindings to $\Gamma_{e2}$. However, in rule \runa{ctx-update-3} we allow path bindings in $\Gamma_{e2}$ since no such bindings are in context $\Gamma_{e1}$.

% \begin{equation}
%     depth(e) = \left\{
%         \begin{array}{ll}
%             depth(E) + 1            & \quad if e = (\texttt{child}\; n).E \\
%             depth(E) - 1            & \quad if e = \texttt{parent}.E\\
%             depth(E_1) + depth(E_2) & \quad if e = E_1 \ggg E_2\\
%             depth(E)                & \quad if e = \texttt{rec}\; x.E\\
%             depth(E)                & \quad if e = \pi.E\\
%             0                       & \quad otherwise
%         \end{array}
%     \right.
% \end{equation}

\begin{definition}{(Relative cursor depth)}
    We define the function $depth : \mathbf{Edt} \rightarrow \mathbb{Z}$, from the set of atomic editor expression to the set of integers.
    \begin{align*}
    depth((\texttt{child}\; n).E) &= depth(E) + 1 \\
    depth(\texttt{parent}.E) &= depth(E) - 1 \\
    depth(E_1 \ggg E_2) &= depth(E_1) + depth(E_2) \\
    depth(\texttt{rec}\; x.E) &= depth(E) \\
    depth(\pi.E) &= depth(E) \\
    depth(E) &= 0 
\end{align*}
\end{definition}
The $depth$ function statically analyses the structure of an editor expression to determine the relative depth of the cursor after evaluation of the expression. This function is used to make sure the position of the cursor before and after evaluation of an expression is the same. As the function performs a static analysis, we do not consider conditioned subexpressions. Later, in the type rules, we will see why we can safely ignore conditioned subexpressions. \\


% Next we define the function $match : \mathbf{Aam} \times \mathbf{ACtx} \times \mathbf{ATyp} \rightarrow \{tt, f\!\!f\}$. This function returns true if the type of the given AST modification $D$, is equal to the given AST type $\tau$.  
% \begin{align*}
%     match(\texttt{var}\; x,\;\Gamma_a,\;\tau) &= \left\{\begin{matrix}
%  tt & \text{if}\; \Gamma_a(x) = \tau\\ 
%  f\!\!f & \text{otherwise}
% \end{matrix}\right.\\
%     match(\texttt{const}\; c,\;\Gamma_a,\; b) &= tt\\
%     match(\texttt{app} : \tau_1 \rightarrow \tau_2,\; \tau_1,\;\Gamma_a,\; \tau_2) &= tt\\
%     match(\texttt{lambda}\; x : \tau_1 \rightarrow \tau_2,\;\Gamma_a,\; \tau_1 \rightarrow \tau_2) &= tt\\
%     match(\texttt{break},\;\Gamma_a,\; \tau) &= tt\\
%     match(\texttt{hole} : \tau,\;\Gamma_a,\; \tau) &= tt\\
%     match(D,\; \Gamma_a,\; \tau) &= f\!\!f
% \end{align*}

%\begin{equation*}
%    %context : \left(\mathbf{Aam} \times \mathbf{ACtx}\right) \rightharpoonup %\left(\left(\mathbf{Pth} \rightarrow \left(\left(\mathbf{Var} \rightharpoonup %\mathbf{ATyp}\right) \times \mathbf{ATyp}\right)\right) \cup \{error\}\right)
    %context : \left(\mathbf{Aam} \times \mathbf{ACtx} \times \mathbf{Pth} \right) %\rightharpoonup \mathbf{ECtx}
%\end{equation*}
%\begin{align*}
% context(\texttt{const}\; c,\; \Gamma_a,\; p) =&\; \emptyset\\
%  context(\texttt{hole} : \tau,\; \Gamma_a,\; p) =&\; \emptyset\\
%context(\texttt{var}\; x,\; \Gamma_a,\; p) =&\; \emptyset\\
 %context((\texttt{app} : \tau_1 \rightarrow \tau2,\; \tau_1),\; \Gamma_a,\; p) =&\; %\emptyset,\; p\; \texttt{one} : (\Gamma_a,\; \tau_1 \rightarrow \tau_2),\; p\; \texttt{two} : %(\Gamma_a,\; \tau_1)\\
 %context(\texttt{lambda}\; x : \tau_1 \rightarrow \tau_2,\; \Gamma_a,\; p) =&\; \emptyset,\; %p\; \texttt{one} : ((\Gamma_a,\; x : \tau_1),\; \tau_2)
%\end{align*}
%
%

We define functions \textit{limits} and \textit{follows} to analyze which cursor movement is safe given a condition holds. \textit{limits} finds the set of possible AST node modifiers, on which the cursor may sit, given the condition holds. \textit{follows} gives a set of editor type context bindings guaranteed to be safe, given the cursor sits on AST node modifier $D$. Note that the AST type context is empty and that the node type is $\texttt{indet}$, as we cannot determine such information based on a condition. Thus, besides toggling of breakpoints, substitution is not well-typed at path $p$ if $\Gamma_e(p)=(\emptyset,\; \texttt{indet})$. We can combine functions \textit{limits} and \textit{follows} to provide additional bindings to the editor type context of a conditioned expression $\phi \Rightarrow E$. The intersection of \textit{follows} applied to each AST node modifier $D$ in the set $limits(\phi)$ is the set of bindings guaranteed to be safe, given $\phi$ holds.

\theoremstyle{definition}
\begin{definition}{(Condition constraints)}
We define a function $limits: \mathbf{Eed} \rightarrow \mathcal{P}(\mathbf{Aam})$ from the set of conditions to the power set of the set of AST node modifiers. We assume conditions are in conjunctive normal form.
\begin{align*}
    limits(@D)=&\;\{D\}\\
    limits(\neg @D)=&\;\mathbf{Aam}\setminus \{D\}\\
    limits(\lozenge D)=&\;\{D\} \cup \{\texttt{app},\; \texttt{lambda}\; x,\; \texttt{break}\}\\
    limits(\neg \lozenge D)=&\;\mathbf{Aam}\setminus \{D\}\\
    limits(\Box D)=&\;\{D\} \cup \{\texttt{app},\; \texttt{lambda}\; x,\; \texttt{break}\}\\
    limits(\neg \Box D)=&\;\mathbf{Aam}\setminus \{D\}\\
    limits(\phi_1 \land \phi_2)=&\;limits(\phi_1) \cap limits(\phi_2)\\
    limits(\phi_1 \lor \phi_2)=&\;limits(\phi_1) \cup limits(\phi_2)
\end{align*}
\end{definition}


\theoremstyle{definition}
\begin{definition}{(Safe movement)}
We define a function $follows: \mathbf{Aam} \times \mathbf{Pth} \rightarrow \mathcal{P}\left(\mathbf{Pth} \times \left(\mathbf{ACtx} \times \mathbf{ATyp}\right)\right)$ from the set of pairs of AST node modifiers and paths to the power set of editor context bindings.
\begin{align*}
    \textit{follows}(\texttt{var}\; x,\; p)=&\; \emptyset\\
    \textit{follows}(\texttt{const}\; c,\; p)=&\; \emptyset\\
    \textit{follows}(\texttt{app},\; p)=&\; \{p\; \texttt{one} : (\emptyset,\; \texttt{indet}),\; p\; \texttt{two} : (\emptyset,\; \texttt{indet})\}\\
    \textit{follows}(\texttt{lambda}\; x,\; p)=&\; \{p\; \texttt{one} : (\emptyset,\; \texttt{indet})\}\\
    \textit{follows}(\texttt{break},\; p)=&\; \{p\; \texttt{one} : (\emptyset,\; \texttt{indet})\}\\
    \textit{follows}(\texttt{hole},\; p)=&\; \emptyset
\end{align*}
\end{definition}

%
%
We now introduce the type rules for editor expressions. Type rules for substitution are shown in table \ref{tab:typerulesv2sub} and the remaining rules are shown in table \ref{tab:typerulesv2}. The \texttt{child} n prefix is handled by \runa{t-child-1} and \runa{t-child-2}. Here we check that the cursor movement is viable by looking up the new path in $\Gamma_e$. Notice that the remaining editor expression $E$, is evaluated using the new path. The \texttt{parent} prefix is handled similarly in \runa{t-parent} with the exception being that we deconstruct the path instead of building it. When using recursion we require that the depth of the cursor is unchanged after evaluating the expression. We ensure this in \runa{t-rec} with the side condition $depth(E) = 0$. Similarly, \runa{t-cond} utilizes the same side condition to ensure that the cursor is unaffected by whether the condition holds or not. Notice here that evaluation of the conditioned expression is limited by what can follow the condition if it holds, denoted by $\delta$. Sequential composition is handled by the type rule \runa{t-seq}. Here we split the type context into $\Gamma_{e1}$, which contains information about the current subtree, and $\Gamma{e2}$, which contains information about the rest of the tree. This split ensures that the potentially hazardous evaluation of $E_1$ is kept separate from the evaluation of $E_2$.\\

\begin{table*}[htp]
    \centering
    \begin{align*}
        %
        \runa{t-eval} &\; \infrule{p,\; \Gamma_e \vdash E : ok}{p,\; \Gamma_e \vdash \texttt{eval}.E : ok}\\
        %
        \runa{t-child-1}&\; \infrule{\Gamma_e(p\; \texttt{one}) \neq \text{undef} \quad p\; \texttt{one},\; \Gamma_e \vdash E : ok}{p,\; \Gamma_e \vdash \left(\texttt{child}\; 1\right).E : ok}\\
        %
        \runa{t-child-2}&\; \infrule{\Gamma_e(p\; \texttt{two}) \neq \text{undef} \quad p\; \texttt{one},\; \Gamma_e \vdash E : ok}{p,\; \Gamma_e \vdash \left(\texttt{child}\; 2\right).E : ok}\\
        %
        \runa{t-parent}&\; \infrule{\Gamma_e(p) \neq \text{undef} \quad p,\; \Gamma_e \vdash E : ok}{p\; T,\; \Gamma_e \vdash \texttt{parent}.E : ok}\\
        %
        \runa{t-rec} &\; \condinfrule{p,\; \Gamma_e \vdash E : ok}{p,\; \Gamma_e \vdash \texttt{rec} x.E : ok}{\text{if}\; depth(E) = 0}\\
        %
        \runa{t-cond} &\; \condinfrule{p,\; \Gamma_e + \delta \vdash E : ok}{p,\; \Gamma_e \vdash \phi \Rightarrow E : ok}{\begin{align*}
            \text{if}\; &depth(E) = 0\;\\
            \text{and}\; &\delta = \bigcap_{D \in limits(\phi)}follows(D,\; p)\\
        \end{align*}}\\
        %
        \runa{t-seq} &\; \condinfrule{p,\; {\Gamma_e}_1 \vdash E_1 : ok \quad p,\; {\Gamma_e}_2 \vdash E_2 : ok}{p,\; \Gamma_e \vdash E_1 \ggg E_2 : ok}{\text{where}\; \Gamma_e = p\; ({\Gamma_e}_1\; \circ\; {\Gamma_e}_2)}\\
        %
        \runa{t-ref} &\; \infrule{}{p,\;\Gamma_e \vdash x : ok}\\
        %
        \runa{t-nil} &\; \infrule{}{p,\;\Gamma_e \vdash \mathbf{0} : ok}
    \end{align*}
    \caption{Type rules for editor expressions.}
    \label{tab:typerulesv2}
\end{table*}
%
%
Table \ref{tab:typerulesv2sub} shows the type rules for substitution. For substitution to be well-typed, the AST node type $\tau$ in the type context binding associated with the current path $p$ must be consistent with the type of the AST node modifier to be inserted. In \runa{t-sub-var}, we handle the special case where we insert a variable reference $x$. For this to be well-typed, a binding $\Gamma_a(x)=\tau'$ must exist, such that $\consistent{\tau}{\tau'}$. Note that substitution replaces a subtree of the AST. Thus, the bindings in the editor type context with paths starting with $p$ are no longer valid. Therefore, we split the type context on path $p$, such that $\Gamma_e = p\left({\Gamma_e}_1\;\circ\;{\Gamma_e}_2\right)$, and evaluate the prefixed expression $E$ in the type context ${\Gamma_e}_2$. That is, the type context containing all bindings of $\Gamma_e$ not starting with $p$. Note that the binding with path exactly $p$ is in both ${\Gamma_e}_1$ and ${\Gamma_e}_2$, however. We add bindings to ${\Gamma_e}_2$ in rules $\runa{t-sub-app}$ and $\runa{t-sub-abs}$. Particularly, we expand the AST type context upon substitution for an abstraction.\\

We treat substitution of breakpoints differently, as we can either toggle breakpoints on or off. Furthermore, we do not replace the subtree upon substitution for breakpoints. Instead, we must modify the bindings with paths starting with $p$, to either include or remove a $\texttt{one}$. Additionally, we change the type in the binding at the current path $p$ to indicate whether it has a breakpoint. Note that we toggle off the breakpoint if the type is of the form $\breakpoint{\tau}$, and toggle it on otherwise. Thus, the type indicates the structure of the tree.
%
%
\begin{table}
    \begin{flalign*}
        %
        \runa{t-sub-var} &\; \condinfrule{\Gamma_e(p)=(\Gamma_a,\;\tau) \quad \Gamma_a(x) = \tau' \quad \consistent{\tau}{\tau'} \quad p,\;{\Gamma_e}_2 \vdash E : ok}{p,\; \Gamma_e \vdash \replace{\texttt{var}\; x}.E : ok}{\text{where}\; \Gamma_e = p\; ({\Gamma_e}_1\; \circ\; {\Gamma_e}_2)} \\
        %
        \runa{t-sub-const} &\; \condinfrule{\Gamma_e(p)=(\Gamma_a,\;b) \quad p,\;{\Gamma_e}_2 \vdash E : ok}{p,\; \Gamma_e \vdash \replace{\texttt{const}\; c}.E : ok}{\text{where}\; \Gamma_e = p\; ({\Gamma_e}_1\; \circ\; {\Gamma_e}_2)}\\
        %
        \runa{t-sub-app} &\; \condinfrule{\Gamma_e(p)=(\Gamma_a,\; \tau_2') \quad \consistent{\tau_2}{\tau_2'} \quad p,\; \Gamma_e' \vdash E : ok}{p,\; \Gamma_e \vdash \replace{\texttt{app} : \tau_1 \rightarrow \tau_2,\; \tau_1}.E : ok}{\begin{align*}
            &\text{where}\; \Gamma_e = p\; ({\Gamma_e}_1\; \circ\; {\Gamma_e}_2)\;\\
            &\text{and}\; \Gamma_e' = {\Gamma_e}_2,\; p\; \texttt{one} : (\Gamma_a,\; \tau_1 \rightarrow \tau_2),\; p\; \texttt{two} : (\Gamma_a,\; \tau_1)
        \end{align*}}\\
        %
        \runa{t-sub-abs} &\; \condinfrule{\Gamma_e(p)=(\Gamma_a,\; \tau_1' \rightarrow \tau_2') \quad \consistent{\tau_1 \rightarrow \tau_2}{\tau_1' \rightarrow \tau_2'} \quad p,\; \Gamma_e' \vdash E : ok}{p,\; \Gamma_e \vdash \replace{\texttt{lambda}\; x : \tau_1 \rightarrow \tau_2}.E : ok}{\begin{align*}
        &\text{where}\;\Gamma_e = p\; ({\Gamma_e}_1\; \circ\; {\Gamma_e}_2)\\
        &\text{and}\;\Gamma_e' = {\Gamma_e}_2, p\; \texttt{one} : ((\Gamma_a,\; x : \tau_1),\; \tau_2)\end{align*}} \\
        %
        %\runa{t-sub} &\; \infrule{match(D,\; \Gamma_a,\; \tau) = tt \quad p,\;\Gamma_e' \vdash %E : ok}{p,\;\Gamma_e \vdash \replace{D}.E : ok} \\
        %&\text{if}\; D \neq \texttt{break}\\
        %&\text{and}\; \Gamma_e(p)=(\Gamma_a,\;\tau) \\
        %&\text{and}\; \Gamma_e = p\; ({\Gamma_e}_1\; \circ\; {\Gamma_e}_2)\\
        %&\text{and}\; \Gamma_e' = {\Gamma_e}_2 + context(D,\; \Gamma_a)\\
        %
        \runa{t-sub-break-1} &\; \infrule{\Gamma_e(p)=(\Gamma_a,\; \breakpoint{\tau}) \quad p,\; \Gamma_e' \vdash E : ok}{p,\; \Gamma_e \vdash \replace{\texttt{break}} : ok} \\
        &\text{where}\; \Gamma_e = p\; ({\Gamma_e}_1\; \circ\; {\Gamma_e}_2)\\
        &\text{and}\; {\Gamma_e}_1 = \emptyset,\; p\; \texttt{one}\; T_1..T_{n_1} : ({\Gamma_a}_1,\; \tau_1),..,p\; \texttt{one}\; T_1..T_{n_m} : ({\Gamma_a}_m,\; \tau_m)\\
        &\text{and}\; {\Gamma_e}_1' =\emptyset,\; p\; T_1..T_{n_1} : ({\Gamma_a}_1,\; \tau_1),..,p\; T_1..T_{n_m} : ({\Gamma_a}_m,\; \tau_m)\\
        &\text{and}\; \Gamma_e' = \left({\Gamma_e}_2 + {\Gamma_e}_1'\right),\; p : (\Gamma_a,\; \tau)\\
        %
        \runa{t-sub-break-2} &\; \infrule{\Gamma_e(p)=(\Gamma_a,\;\tau)\quad  p,\; \Gamma_e' \vdash E : ok}{p,\; \Gamma_e \vdash \replace{\texttt{break}} : ok} \\
        &\text{where}\; \Gamma_e = p\; ({\Gamma_e}_1\; \circ\; {\Gamma_e}_2)\\
        &\text{and}\; {\Gamma_e}_1 =\emptyset,\; p\; T_1..T_{n_1} : ({\Gamma_a}_1,\; \tau_1),..,p\; T_1..T_{n_m} : ({\Gamma_a}_m,\; \tau_m)\\
        &\text{and}\; {\Gamma_e}_1' = \emptyset,\; p\; \texttt{one}\; T_1..T_{n_1} : ({\Gamma_a}_1,\; \tau_1),..,p\; \texttt{one}\; T_1..T_{n_m} : ({\Gamma_a}_m,\; \tau_m)\\
        &\text{and}\; \Gamma_e' = \left({\Gamma_e}_2 + {\Gamma_e}_1'\right),\; p : (\Gamma_a,\; \breakpoint{\tau})\\
        %
        \runa{t-sub-hole} &\; \condinfrule{\Gamma_e(p)=(\Gamma_a,\;\tau') \quad \consistent{\tau}{\tau'} \quad p,\;{\Gamma_e}_2 \vdash E : ok}{p,\; \Gamma_e \vdash \replace{\texttt{hole} : \tau}.E : ok}{\text{where}\; \Gamma_e = p\; ({\Gamma_e}_1\; \circ\; {\Gamma_e}_2)}
        %
    \end{flalign*}
    \caption{Type rules for substitution.}
    \label{tab:typerulesv2sub}
\end{table}

%\begin{table*}[htp]
%    \centering
%    \begin{align*}
        %%
        %\runa{t-eval} &\; \infrule{p,\; \Gamma_e \vdash E : ok \dashv p',\; \Gamma_e'}{p,\; \Gamma_e \vdash \texttt{eval}.E : %ok \dashv p',\; \Gamma_e'}\\
        %%
        %\runa{t-sub} &\; \infrule{T=\tau \quad p,\;\Gamma_e'' \vdash E : ok \dashv p',\;\Gamma_e'}{p,\;\Gamma_e \vdash %\replace{D}.E : ok \dashv p',\;\Gamma_e'} \\
        %&\text{where}\; \Gamma_e(p)=(\Gamma_a,\;\tau) \\
        %&\text{and}\; T = type(D,\;\Gamma_a) \\
        %&\text{and}\; \Gamma_e = p\; ({\Gamma_e}_1\; \circ\; {\Gamma_e}_2)\\
        %&\text{and}\; \Gamma_e'' = {\Gamma_e}_1 + context(D,\; \Gamma_a)\\
        %%
        %\runa{t-child-1}&\; \infrule{\Gamma_e(p\; \texttt{one}) \neq undef \quad p,\; \texttt{one},\; \Gamma_e \vdash E : ok %\dashv p',\; \Gamma_e'}{p,\; \Gamma_e \vdash \left(\texttt{child}\; 1\right).E : ok \dashv p',\; \Gamma_e'}\\
        %%
        %\runa{t-child-2}&\; \infrule{\Gamma_e(p\; \texttt{two}) \neq undef \quad p,\; \texttt{one},\; \Gamma_e \vdash E : ok %\dashv p',\; \Gamma_e'}{p,\; \Gamma_e \vdash \left(\texttt{child}\; 2\right).E : ok \dashv p',\; \Gamma_e'}\\
        %%
        %\runa{t-parent}&\; \infrule{\Gamma_e(p) \neq undef \quad p,\; \Gamma_e \vdash E : ok \dashv p',\; \Gamma_e'}{p\; T,\; %\Gamma_e \vdash \texttt{parent}.E : ok \dashv p',\; \Gamma_e'}\\
        %%
        %\runa{t-rec} &\; \condinfrule{p,\; {\Gamma_e}_1 \vdash E : ok \dashv p,\; \Gamma_e'}{p,\; \Gamma_e \vdash \texttt{rec} %x.E : ok \dashv p,\; {\Gamma_e}_2}{\text{where}\; \Gamma_e = p\; ({\Gamma_e}_1\; \circ\; {\Gamma_e}_2)}\\
        %%
        %\runa{t-seq} &\; \infrule{p,\; \Gamma_e \vdash E_1 : ok \dashv p'',\; \Gamma_e'' \quad p'',\; \Gamma_e'' \vdash E_2 : %ok \dashv p',\; \Gamma_e'}{p,\; \Gamma_e \vdash E_1 \ggg E_2 : ok \dashv p',\; \Gamma_e'}\\
        %%
        %\runa{t-cond} &\; \infrule{p,\; {\Gamma_e}_1 + \delta \vdash E : ok \dashv p,\; \Gamma_e'}{p,\; \Gamma_e \vdash \phi %\Rightarrow E : ok \dashv p,\; {\Gamma_e}_2}\\
%        &\text{where}\; \Gamma_e = p\; ({\Gamma_e}_1\; \circ\; {\Gamma_e}_2)\\
%        &\text{and}\; \delta = \bigcap_{D \in limits(\phi)}follows(D)\\
%        %
%        \runa{t-ref} &\; \infrule{}{p,\;\Gamma_e \vdash x : ok \dashv p,\;\Gamma_e}\\
%        %
%        \runa{t-nil} &\; \infrule{}{p,\;\Gamma_e \vdash \mathbf{0} : ok \dashv p,\;\Gamma_e}\\
%    \end{align*}
%    \caption{Type rules for editor expressions.}
%    \label{tab:typerules}
%\end{table*}

\begin{theorem} (Subject reduction)
If $\Gamma_e, \;\Gamma_a \vdash \conf{E,\;a} : ok$ and $\conf{E, a} \xrightarrow{\alpha} \conf{E', a'}$ then $\Gamma_e, \;\Gamma_a \vdash \conf{E',\;a'} : ok$.
\end{theorem}

We define \textit{well-typedness} of a configuration $\conf{E,\;a}$ by the following rule: \\
$\condinfrule{\Gamma_a \vdash a : \tau \quad p,\; \Gamma_e \vdash E : ok}{\Gamma_e, \;\Gamma_a \vdash \conf{E,\;a} : ok}{\begin{align*}
        &\text{where}\;\\
        &\text{and}\;\end{align*}}$
        
        

\chapter{Type checking sized types for parallel complexity}\label{ch:typecheck}
As mentioned in Chapter \ref{ch:bgts}, Baillot and Ghyselen \cite{BaillotGhyselen2021} bound sizes of algebraic terms and synchronizations on channels using indices, leading to a partial order on indices. For instance, for a process of the form $\inputch{a}{v}{}{\asyncoutputch{b}{v}{}} \mid P$ (assuming synchronizations induce a cost in time complexity of one) we must enforce that the bound on $a$ is strictly smaller than the bound on $b$. Thus, we must impose constraints on the interpretations of indices. Another concern in the typing of the process above is the parallel complexity. Granted separate bounds on the complexities of $\inputch{a}{v}{}{\asyncoutputch{b}{v}{}}$ and $P$, how do we establish a tight bound on their parallel composition? This turns out to be another major challenge, as bounds may be parametric, such that comparison of bounds is a partial order. Finally, for a process of the form $!\inputch{a}{v}{}{P} \mid \asyncoutputch{a}{e}{}$ we must \textit{instantiate} the parametric complexity of $!\inputch{a}{v}{}{P}$ based on the deducible size bounds of $e$, which quickly becomes difficult as indices become more complex.\\ % As the type system is otherwise fairly standard, for instance using input/output types for channels, the challenge in introducing type check is to ensure constraints on indices are not violated.\\
%
%Type inference for the type system introduced in Baillot and Ghyselen \cite{BaillotGhyselen2021} is complicated by similar challenges to that of type checking, such as constraint satisfaction. Another concern with respect to sized types is that we must infer indices. Here, it is relevant to consider existing work on sized type inference, such as Hughes et al. \cite{HughesEtAl1996} and Avanzini and Dal Lago \cite{AvanziniLago2017}. The set of function symbols used to form indices should be be more strictly defined, to make inference tractable. We also must be careful with respect to recursion, predominantly with how primitive recursion can be identified. In this chapter, we address some of these challenges.
%
% The type system for parallel complexity of message-passing processes introduced in Baillot and Ghyselen \cite{BaillotGhyselen2021} uses sized types to express parametric complexity of invoking replicated inputs, and thereby achieve precise bounds on primitively recursive processes. This requires a notion of polymorphism in the message types of replicated inputs. Baillot and Ghyselen introduce size polymorphism by bounding sizes of algebraic terms and synchronizations on channels with algebraic expressions referred to as indices that may contain index variables representing unknown sizes. We may interpret an index with an index valuation that maps its index variables to naturals, such that the index may be evaluated.\\ %
%
% The bounds on sizes and synchronizations lead to a partial order on indices. For instance, for a process of the form $\inputch{a}{v}{}{\asyncoutputch{b}{v}{}} \mid P$ (assuming synchronizations induce a cost in time complexity of one) we must enforce that the bound on $a$ is strictly smaller than the bound on $b$. Thus, we must induce constraints on the interpretations of indices. As the type system is otherwise fairly standard, for instance using input/output types for channels, the challenge in introducing type check is to ensure constraints on indices are not violated.

The purpose of this section is to present a version of the type system by Baillot and Ghyselen that is algorithmic in the sense that its type rules can be easily implemented in a programming language, and so we must address the challenges described above. For the type checker, we assume we are given a set of constraints $\Phi$ on index variables in $\varphi$ and a type environment $\Gamma$. We first present the types of the type system as well as subtyping. Afterwards, we present auxiliary functions and type rules. For the type rules we also present the concept of combined complexities that we use to bound parallel complexities by deferring comparisons of indices when these are not defined. We then prove the soundness of the type checker and show how it can be extended accompanied by examples. Finally, we show how we can verify the constraint judgements that show up in the type rules.

\section{Auxiliary functions}
We first present two functions that will be used in the type rules. As the continuation of a replicated input may be invoked an arbitrary number of times at different time steps, we need to ensure that names used within the continuation are of types that are invariant to time as defined in Definition \ref{def:timeinvariance}, i.e. they may be used at any time step. In Definition \ref{def:readyfunc}, we define a function $\text{ready}(\varphi,\Phi,T)$ that discards use-capabilities from a type to obtain time invariance. For a server type $\forall_I\widetilde{i}.\texttt{serv}^\sigma_K(\widetilde{T)}$, outputs are well-typed whenever $\varphi;\Phi\vDash I \leq 0$, and so for names of such types, we only discard input capabilities, whenever we can guarantee the constraint judgement $\varphi;\Phi\vDash I \leq 0$. We return to how to guarantee constraint judgements in section \ref{sec:verifyinglinearjudgements}.
%
\begin{defi}\label{def:readyfunc}
We inductively define a function \textit{ready} that transforms a type into one that is time invariant.
\begin{align*}
    %\text{ready}(\varphi,\Phi,\epsilon) =&\; \epsilon\\
    %
    \text{ready}(\varphi,\Phi,\natinterval{I}{J}) =&\; \natinterval{I}{J}\\
    %
    \text{ready}(\varphi,\Phi,\forall_I\widetilde{i}.\texttt{serv}^{\sigma}_K(\widetilde{T})) =&\; \left\{ \begin{matrix}
        \forall_I\widetilde{i}.\texttt{serv}^{\sigma \cap \{\texttt{out}\}}_K(\widetilde{T}) & \text{if}\; \varphi;\Phi\vDash I \leq 0\\
        \forall_0\widetilde{i}.\texttt{serv}^{\emptyset}_K(\widetilde{T}) & \text{if}\; \varphi;\Phi\nvDash I \leq 0
    \end{matrix} \right.\\
    %
    %\text{ready}(\varphi,\Phi,\Gamma,a:\oservS{I}{\widetilde{i}}{K}{\widetilde{T}}) =&\; \left\{ \begin{matrix}
    %    \text{ready}(\varphi,\Phi,\Gamma), a:\oservS{I}{\widetilde{i}}{K}{\widetilde{T}} & \text{if}\; %\varphi;\Phi\vDash I \leq 0\\
    %    \text{ready}(\varphi,\Phi,\Gamma) & \text{if}\; \varphi;\Phi\nvDash I \leq 0
    %\end{matrix} \right.\\
    %%
    %\text{ready}(\varphi,\Phi,\Gamma,a:\iservS{I}{\widetilde{i}}{K}{\widetilde{T}}) =&\; %\text{ready}(\varphi,\Phi,\Gamma)\\
    %
    \text{ready}(\varphi,\Phi,\texttt{ch}^\sigma_I(\widetilde{T})) =&\;\texttt{ch}^\emptyset_0(\widetilde{T})%\\
    %
    %\text{ready}(\varphi,\Phi,\Gamma,a:\outchanneltypeS{I}{\widetilde{T}}) =&\; \text{ready}(\varphi,\Phi,\Gamma)\\
    %
    %\text{ready}(\varphi,\Phi,\Gamma,a:\inchanneltypeS{I}{\widetilde{T}}) =&\; \text{ready}(\varphi,\Phi,\Gamma)
\end{align*}
We extend \textit{ready} to type contexts such that for $v\in\texttt{dom}(\Gamma)$ we have that $\text{ready}(\varphi,\Phi,\Gamma)(v)=\text{ready}(\varphi,\Phi,\Gamma(v))$.
\end{defi}

% In Definition \ref{def:joinbase}, we introduce a binary function on base types $\uplus_{\varphi;\Phi}$ that computes a base type that is a subtype of both argument types, if such a base type exists. We do this by selecting the least lower bound and the greatest upper bound amongst the two argument base types as the new size bounds. This function will be useful for typing list expressions, as the elements of a list may be typed with different size bounds that we will need a common subtype of.

% \begin{defi}[Joining base types]\label{def:joinbase}

% \begin{align*}
%     \texttt{Nat}[I,J] \uplus_{\varphi;\Phi} \texttt{Nat}[I',J'] =&\; \left\{
%     \begin{matrix}
%         \texttt{Nat}[I,J] & \varphi;\Phi\vDash I \leq I'\;\text{and};\varphi;\Phi\vDash J' \leq J\\
%         \texttt{Nat}[I',J] & \varphi;\Phi\vDash I' \leq I\;\text{and};\varphi;\Phi\vDash J' \leq J\\
%         \texttt{Nat}[I,J'] & \varphi;\Phi\vDash I \leq I'\;\text{and};\varphi;\Phi\vDash J \leq J'\\
%         \texttt{Nat}[I',J'] & \varphi;\Phi\vDash I' \leq I\;\text{and};\varphi;\Phi\vDash J \leq J'
%     \end{matrix}
%     \right.\\
    
%     \texttt{List}[I,J](\mathcal{B}) \uplus_{\varphi;\Phi} \texttt{List}[I',J'](\mathcal{B}') =&\; \left\{
%     \begin{matrix}
%         \texttt{List}[I,J](\mathcal{B} \uplus_{\varphi;\Phi} \mathcal{B}') & \varphi;\Phi\vDash I \leq I'\;\text{and};\varphi;\Phi\vDash J' \leq J\\
%         \texttt{List}[I',J](\mathcal{B} \uplus_{\varphi;\Phi} \mathcal{B}') & \varphi;\Phi\vDash I' \leq I\;\text{and};\varphi;\Phi\vDash J' \leq J\\
%         \texttt{List}[I,J'](\mathcal{B} \uplus_{\varphi;\Phi} \mathcal{B}') & \varphi;\Phi\vDash I \leq I'\;\text{and};\varphi;\Phi\vDash J \leq J'\\
%         \texttt{List}[I',J'](\mathcal{B} \uplus_{\varphi;\Phi} \mathcal{B}') & \varphi;\Phi\vDash I' \leq I\;\text{and};\varphi;\Phi\vDash J \leq J'
%     \end{matrix}
%     \right.
% \end{align*}
% \end{defi}

% \begin{defi}[Removing servers]
%     Given a type context $\Gamma$, the function \textit{noserv} removes all server types from the context.
%     \begin{align*}
%         \text{noserv}(\emptyset) &= \emptyset\\
%         \text{noserv}(\Gamma, \natinterval{I}{J}) &= \text{noserv}(\Gamma),\natinterval{I}{J}\\
%         \text{noserv}(\Gamma, \chant{I}{\sigma}{\widetilde{T}}) &= \text{noserv}(\Gamma),\chant{I}{\sigma}{\widetilde{T}})\\
%         \text{noserv}(\Gamma, \servt{I}{\widetilde{i}}{\sigma}{K}{\widetilde{T}}) &= \text{noserv}(\Gamma)\\
%     \end{align*}
% \end{defi}


In Definition \ref{def:instantiatef}, we introduce a function $\text{instantiate}(\widetilde{i},\widetilde{T})$ that assigns the index variables in sequence $\widetilde{i}$ to indices in types of the sequence $\widetilde{T}$, by means of a substitution of indices for index variables. Note that the function is only defined for sequences such that the number of index variables equals the number of indices in the types. This function will be useful for outputs on servers, where the parametric types $\widetilde{S}$ of a server type $\forall_I\widetilde{i}.\texttt{serv}^{\{\texttt{out}\}}_K(\widetilde{S)}$ must match the types of expressions $\widetilde{T}$ to be output. More specifically, there must exist a substitution $\{\widetilde{J}/\widetilde{i}\}$ such that $\widetilde{T} \sqsubseteq \widetilde{S}\{\widetilde{J}/\widetilde{i}\}$. We return to this in Section \ref{section:typeruless}.

\begin{defi}[Server invocation]\label{def:instantiatef}
We inductively define a function \textit{instantiate} that constructs a substitution of indices for index variables, provided a sequence of index variables and a sequence of types%. The function is only defined for sequences such that the number of index variables equals the number of indices in the types.
\begin{align*}
    \text{instantiate}(\epsilon,\epsilon) =&\; \{\}\\
    \text{instantiate}((\widetilde{i},\widetilde{j}),(T,\widetilde{S})) =&\; \text{instantiate}(\widetilde{i},T),\text{instantiate}(\widetilde{j},\widetilde{S})\\
    \text{instantiate}((i,j),\texttt{Nat}[I,J]) =&\; \{I/i,J/j\}\\
    %\text{instantiate}((i,j,\widetilde{k}),\texttt{List}[I,J](\mathcal{B})) =&\; \{I/i,J/j\}, \text{instantiate}(\widetilde{k},\mathcal{B})\\
    \text{instantiate}((i,\widetilde{j}),\texttt{ch}_I^\sigma(\widetilde{T})) =&\; \{I/i\},\text{instantiate}(\widetilde{j},\widetilde{T})\\
    \text{instantiate}((i,j,\widetilde{k}),\forall_I\widetilde{l}.\texttt{serv}^\sigma_K(\widetilde{T})) =&\; \{I/i,K/j\},\text{instantiate}(\widetilde{k},\widetilde{T})
\end{align*}
\end{defi}
\section{Algorithmic type rules}\label{section:typeruless}
%We are now ready to introduce type rules for a type checker of the type system in Baillot and Ghyselen \cite{BaillotGhyselen2021}. %TODO 
%
%
%A piecewise complexity $\kappa$ is a set of pairs $(\Phi_i, K_i)$ where $K_i$ is an index describing a complexity that is valid within the feasible region described by the set of constraints $\Phi_i$. As such, a piecewise complexity $\kappa = \{(\Phi_1, K_1), \cdots, (\Phi_n, K_n)\}$ describes a complexity bound within the feasible region $\mathcal{M}_\varphi(\Phi_1) \cup \cdots \cup \mathcal{M}_\varphi(\Phi_n)$ for some $\varphi$ such that $\Phi_1, \cdots, \Phi_n$ use index variables in $\varphi$. In the case where $m$ feasible regions $\mathcal{M}_\varphi(\Phi_{i_1})$, $\mathcal{M}_\varphi(\Phi_{i_{m-1}})$ and $\mathcal{M}_\varphi(\Phi_{i_m})$ intersect, we choose the maximal complexity of the corresponding complexities for any valuation $\rho$ in the intersecting region.\\

When typing a process, we often need to find an index that is an upper bound on two other indices, for which there may be many options. To allow for the type checker to be as precise as possible, we want to find the minimum complexity that is a bound of two other complexities, which will depend on the representation of complexity, and as such, instead of representing complexity bounds using indices, we opt to use sets of indices which we refer to as \textit{combined complexities}. Intuitively, given any point in the space spanned by some index variables, the combined complexity at that point is the maximum of the complexities making up the combined complexity at that point. This is illustrated in Figure \ref{fig:combined_complexity} which shows a combined complexity consisting of three indices. The red dashed line represents the bound on the combined complexity. Representing complexities as sets of indices has the effect of \textit{externalizing} the process of finding bounds of complexities by deferring this until a later time. We will later define the algorithm \textit{basis} that removes superfluous indices of a combined complexity. In Figure \ref{fig:combined_complexity} the index $K$ is superfluous as it never contributes to the bound of the combined complexity.

\begin{figure}
    \centering
    \begin{tikzpicture}
\begin{axis}[
    axis lines = left,
    xlabel = \(i\),
    ylabel = {},
    domain = 0:2.5,
    xtick={\empty},ytick={\empty},
    ymin=0,
    ymax=5.2,
    xmax=2.7,
    restrict y to domain=0:5,
]
    \addplot[thick, color=orange]{x^2} node[above,pos=1] {I};
    \addplot[thick, color=blue]{x+1} node[above,pos=1] {J};
    \addplot[thick, color=green]{ln(x+1)*2} node[above,pos=1] {K};
    \draw [ultra thick, dashed, draw=red] (axis cs:0,1) -- (axis cs:1.62,2.62);
    \addplot[ultra thick, color=green, dashed, color=red, domain=1.62:2.5]{x^2};
\end{axis}
\end{tikzpicture}
    \caption{Combined complexities illustrated. The combined complexity consists of the three indices I, J, K of the single index variable $i$. The dashed red line shows the bound of the combined complexity. $K$ is a superfluous index in the combined complexity as it never contributes to the bound of the combined complexity.}
    \label{fig:combined_complexity}
\end{figure}


\begin{defi}[Combined complexity]\label{def:combinedcomp} 
    We refer to a set $\kappa$ of complexities as a \textit{combined complexity}. We extend constraint judgements to include combined complexities such that
    \begin{enumerate}
        \item $\varphi;\Phi\vDash \kappa \leq \kappa'$ if for all $K \in \kappa$ there exists $K'\in \kappa'$ such that $\varphi;\Phi\vDash K \leq K'$.
        % 
        \item $\varphi;\Phi\vDash \kappa = \kappa'$ if $\varphi;\Phi\vDash \kappa \leq \kappa'$ and $\varphi;\Phi\vDash \kappa' \leq \kappa$.
        \item $\kappa + I = \{K + I \mid K \in \kappa\}$.
        %
        \item $\kappa\{J/i\} = \{ K\{J/i\} \mid K\in\kappa \}$.
    \end{enumerate}
    In the above, we may substitute an index for a combined complexity. In such judgements, the index represents a singleton set. For instance, $\varphi;\Phi\vDash \kappa \leq K$ represents $\varphi;\Phi\vDash \kappa \leq \{K\}$.
    %$\varphi;\Phi \vDash \kappa \bowtie \kappa' \quad\text{ if }\quad \forall K \in \kappa. (\exists K' \in \kappa'. \varphi;\Phi \vDash K \bowtie K')$.
\end{defi}

More specifically, when considering a combined complexity $\kappa$, we are interested in the maximal complexity given some valuation $\rho$, which we find by simply comparing the different values for the complexities within $\kappa$ given $\rho$. Note that the complexity $K \in \kappa$ that is maximal may be different for different valuations. In Definition \ref{def:combinedcomp} we extend the binary relations in $\bowtie$ on indices to combined complexities, such that we can compare two combined complexities such as $\varphi;\Phi \vDash \kappa \bowtie \kappa'$ and a combined complexity and complexity such as $\varphi;\Phi \vDash \kappa \bowtie K$. Definition \ref{def:combinedcompbasis} defines the function \textit{basis} that discards any $K \in \kappa$ that can never be the maximal complexity given some set of constraints $\Phi$ (i.e. the complexities that are bounded by other complexities in the set), such that we can always keep the number of complexities in a combined complexity to a minimum. %Finally, we may also be interested in adding an index onto a combined complexity, and so we define the addition of indices onto combined complexities in Definition \ref{def:combinedcompadd}.
%
\begin{defi}\label{def:combinedcompbasis}
    We define the function \textit{basis} that takes a set of index variables $\varphi$, a set of constraints $\Phi$ and a combined complexity $\kappa$, and returns a new combined complexity without superfluous complexities (The \textit{basis} of $\kappa$)
    \begin{align*}
        \text{basis}(\varphi,\Phi,\kappa) = \bigcap\left\{ \kappa' \subseteq \kappa \mid \forall K\in\kappa.\exists K'\in\kappa'.\varphi;\Phi\vDash K \leq K' \right\}
    \end{align*}
    Moreover, the algorithm below computes the basis
    % \begin{align*}
    %     \text{basis}(\varphi, \kappa) = \{(\Phi, K) \in \kappa \mid \varphi;\Phi \not \vDash K < K' \text{ for all } (\Phi', K') \in \kappa\}
    % \end{align*}
    \begin{align*}
        &\text{basis}(\varphi, \Phi, \kappa) = \text{do}\\[-0.5em]
        &\quad \kappa' \leftarrow \kappa\\[-0.5em]
        &\quad \text{for } K \in \kappa \text{ do}\\[-0.5em]
        &\quad\quad \text{ if } \exists K' \in \kappa' \text{ with } K \not = K' \text{ and } \varphi;\Phi \vDash K \leq K' \text{ then}\\[-0.5em]
        &\quad\quad\quad \kappa' \leftarrow \kappa' \setminus \{K\}\\[-0.5em]
        &\quad \text{return } \kappa'
    \end{align*}
\end{defi}
%
% \begin{defi}[]\label{def:combinedcompadd}
%     We define the the addition of a combined complexity and index as
%     \begin{align*}
%         \kappa + I = \{K + I \mid K \in \kappa\}
%     \end{align*}

% \end{defi}
%
For typing expressions, we use the rules presented in Table \ref{tab:sizedtypedexpressiontypes}, excluding the rule $\runa{BG-sub}$. In Table \ref{tab:sizedprocesstypingrules} we show the type rules for processes. Type judgements are of the form $\varphi;\Phi;\Gamma \vdash P \triangleleft \kappa$ where $\kappa$ denotes the complexity of process $P$. The rule $\runa{S-tick}$ types a \texttt{tick} prefix and incurs a cost of one in time complexity. We advance the time of all types in the context accordingly when typing the continuation. Rule $\runa{S-annot}$ is similar but may incur a cost of $n$. Matches on naturals are typed with rule $\runa{S-nmatch}$. Most notably, we extend the set of known constraints when typing the two continuations. That is, we can deduce constraints on the lower and upper bounds on the size of the expression we match on. For instance, for the zero pattern we can deduce that the lower bound $I$ must be equal to $0$ (or equivalently $I \leq 0$), and for the successor pattern, we can guarantee that the upper bound $J$ must be greater than or equal to $1$. For the complexity of pattern matches and parallel composition, we take advantage of the fact that we represent complexities using combined complexities. As such, we include complexities in both $P$ and $Q$ in the result. To remove redundancy from the set $\kappa \cup \kappa'$, we use the basis function.\\

%
% \begin{table*}[!ht]
%     \begin{framed}\vspace{-1em}\begin{align*}
%         &\kern15em\\[-2em] % Stretch frame
%         &\kern0em\runa{S-nil}\infrule{}{\varphi;\Phi;\Gamma \vdash \withcomplex{\nil}{0}} \kern1em\runa{S-tick}\;\infrule{\varphi;\Phi;\susumesim{\Gamma}{1}\vdash P \triangleleft K}{\varphi;\Phi;\Gamma\vdash \tick P \triangleleft K + 1} \kern3em\runa{S-nu}\;\infrule{\varphi;\Phi;\Gamma,\withtype{a}{T} \vdash \withcomplex{P}{K}}{\varphi;\Phi;\Gamma \vdash \newvar{a: T}{\withcomplex{P}{K}}}\\[-1em]
%         %
%         &\kern-0em\runa{S-nmatch}\;\condinfrule{
%         \begin{matrix}
%             \varphi;\Phi;\Gamma \vdash \withtype{e}{\natinterval{I}{J}}\quad \varphi;\Phi, I \leq 0;\Gamma \vdash \withcomplex{P}{K} \\
%             \varphi;\Phi, J \geq 1;\Gamma, \withtype{x}{\natinterval{I-1}{J-1}} \vdash \withcomplex{Q}{K'}
%         \end{matrix}}{\varphi;\Phi;\Gamma \vdash \withcomplex{\match{e}{P}{x}{Q}}{L}}{\text{where}\quad L = \left\{
% \begin{matrix}
%     K & \text{if}\; \varphi;\Phi\vDash K' \leq K   \\
%     K' & \text{if}\; \varphi;\Phi\vDash K \leq K'  \\
%     K+K' & \text{otherwise}
% \end{matrix}
% \right.}\\[-1em]
%         %
%         %&\kern-0em\runa{S-nmatch-2}\;\infrule{
%         %\begin{matrix}
%         %    \varphi;\Phi;\Gamma \vdash \withtype{e}{\natinterval{I}{J}} \quad \varphi;\Phi\vDash K \leq K' \\
%         %    \varphi;\Phi, I \leq 0;\Gamma \vdash \withcomplex{P}{K} \quad \varphi;\Phi, J \geq 1;\Gamma, \withtype{x}{\natinterval{I-1}{J-1}} \vdash \withcomplex{Q}{K'}
%         %\end{matrix}}{\varphi;\Phi;\Gamma \vdash \withcomplex{\match{e}{P}{x}{Q}}{K'}}\\[-1em]
%         %
%         &\kern-0em\runa{S-lmatch}\;\condinfrule{
%         \begin{matrix}
%             \varphi;\Phi;\Gamma \vdash \withtype{e}{\texttt{List}[I,J](\mathcal{B})} \quad \varphi;\Phi, I \leq 0;\Gamma \vdash \withcomplex{P}{K} \\
%             \varphi;\Phi, J \geq 1;\Gamma, \withtype{x}{\mathcal{B}},y : \texttt{List}[I-1,J-1](\mathcal{B}) \vdash \withcomplex{Q}{K'}
%         \end{matrix}}{\varphi;\Phi;\Gamma \vdash \withcomplex{\texttt{match}\;e\;\{ [] \mapsto P;\; x :: y \mapsto Q \}}{L}}{\text{where}\quad L = \left\{
% \begin{matrix}
%     K & \text{if}\; \varphi;\Phi\vDash K' \leq K   \\
%     K' & \text{if}\; \varphi;\Phi\vDash K \leq K'  \\
%     K+K' & \text{otherwise}
% \end{matrix}
% \right.}\\[-1em]
%         %
%         %&\kern-0em\runa{S-lmatch-2}\;\infrule{
%         %\begin{matrix}
%         %    \varphi;\Phi;\Gamma \vdash \withtype{e}{\texttt{List}[I,J](\mathcal{B})} \quad \varphi;\Phi\vDash K \leq K' \\
%         %    \varphi;\Phi, I \leq 0;\Gamma \vdash \withcomplex{P}{K} \quad \varphi;\Phi, J \geq 1;\Gamma, \withtype{x}{\mathcal{B}},y : \texttt{List}[I-1,J-1](\mathcal{B}) \vdash \withcomplex{Q}{K'}
%       % \end{matrix}}{\varphi;\Phi;\Gamma \vdash \withcomplex{\texttt{match}\;e\;\{ [] \mapsto P;\; x :: y \mapsto Q \}}{K'}}\\[-1em]
%         %
%         &\kern4em\runa{S-par}\;\condinfrule{\varphi;\Phi;\Gamma\vdash P \triangleleft K\quad \varphi;\Phi;\Gamma\vdash Q \triangleleft K'}{\varphi;\Phi;\Gamma\vdash \parcomp{P}{Q} \triangleleft L}{\text{where}\quad L = \left\{
% \begin{matrix}
%     K & \text{if}\; \varphi;\Phi\vDash K' \leq K   \\
%     K' & \text{if}\; \varphi;\Phi\vDash K \leq K'  %\\
%     %K+K' & \text{otherwise}
% \end{matrix}
% \right.}\\[-1em]
%         %
%         %&\kern4em\runa{S-par-2}\;\infrule{\varphi;\Phi;\Gamma\vdash P \triangleleft K\quad \varphi;\Phi;\Gamma\vdash Q \triangleleft K'\quad \varphi;\Phi\vDash K \leq K'}{\varphi;\Phi;\Gamma\vdash \parcomp{P}{Q} \triangleleft K'}\\[-1em]
%         %
%         &\kern-0em\runa{S-iserv}\;\infrule{\texttt{in}\in\sigma\quad \varphi,\widetilde{i};\Phi;\text{ready}(\varphi,\Phi,\susumesim{\Gamma}{I}),a:\forall_0\widetilde{i}.\texttt{serv}^{\sigma\cap\{\texttt{out}\}}_K(\widetilde{T}),\widetilde{v} : \widetilde{T}\vdash P \triangleleft K'\quad \varphi,\widetilde{i};\Phi\vDash K' \leq K}{\varphi;\Phi;\Gamma,a:\forall_I\widetilde{i}.\texttt{serv}^\sigma_K(\widetilde{T})\vdash\; \bang\inputch{a}{\widetilde{v}}{}{P}\triangleleft I}\\[-1em]
%         %
%         &\kern-0em\runa{S-ich}\;\infrule{\texttt{in}\in\sigma\quad \varphi;\Phi;\susumesim{\Gamma}{I},a:\texttt{ch}_0^\sigma(\widetilde{T}),\widetilde{v} : \widetilde{T}\vdash P \triangleleft K}{\varphi;\Phi;\Gamma,a:\texttt{ch}_I^\sigma(\widetilde{T})\vdash \inputch{a}{\widetilde{v}}{}{P}\triangleleft K + I}
%         %
%         \kern8.5em \runa{S-och}\;\infrule{\texttt{out}\in \sigma\quad \varphi;\Phi;\susumesim{\Gamma}{I}\vdash \widetilde{e} : \widetilde{T}\quad \varphi;\Phi\vdash\widetilde{T}\sqsubseteq\widetilde{S}}{\varphi;\Phi;\Gamma,a:\texttt{ch}^{\sigma}_I(\widetilde{S})\vdash \asyncoutputch{a}{\widetilde{e}}{} \triangleleft I}\\[-1em]
%         %
%         &\kern0em\runa{S-oserv}\;\infrule{\texttt{out} \in \sigma \quad \varphi;\Phi;\susumesim{\Gamma}{I}\vdash \widetilde{e} : \widetilde{T}\quad \text{instantiate}(\widetilde{i},\widetilde{T})=\{\widetilde{J}/\widetilde{i}\}\quad  \varphi;\Phi\vdash\widetilde{T}\sqsubseteq\widetilde{S}\{\widetilde{J}/\widetilde{i}\}}{\varphi;\Phi;\Gamma,a:\forall_I\widetilde{i}.\texttt{serv}_K^\sigma(\widetilde{S})\vdash \asyncoutputch{a}{\widetilde{e}}{} \triangleleft K\!\substi{\widetilde{J}}{\widetilde{i}} + I}
%         %
%     \end{align*}\vspace{-1em}\end{framed}
%     \smallskip
%     \caption{Sized typing rules for parallel complexity of processes.}
%     \label{tab:sizedprocesstypingrules}
% \end{table*}

\begin{table*}[!ht]
    \begin{framed}\vspace{-1em}\begin{align*}
        %
        % S-nil
        &\runa{S-nu}\infrule{\varphi;\Phi;\Gamma, a:T \vdash P \triangleleft \kappa}{\varphi;\Phi;\Gamma \vdash \newvar{a:T}{P} \triangleleft \kappa}
        % S-par
        \kern1em\runa{S-par}\infrule{\varphi;\Phi;\Gamma \vdash P \triangleleft \kappa \quad \varphi;\Phi;\Gamma \vdash Q \triangleleft \kappa'}{\varphi;\Phi;\Gamma \vdash P \mid Q \triangleleft \text{basis}(\varphi, \Phi,\kappa \cup \kappa')}\\[-1em]
        %
        &\runa{S-tick}\infrule{\varphi;\Phi;\tforwardsim{\Gamma}{1} \vdash P \triangleleft \kappa}{\varphi;\Phi;\Gamma \vdash \tick P \triangleleft \kappa + 1}\kern2em
        %
        \runa{S-annot}\infrule{\varphi;\Phi;\tforwardsim{\Gamma}{n}\vdash P \triangleleft \kappa}{\varphi;\Phi;\Gamma\vdash n:P \triangleleft \kappa + n}\\[-1em]
        % S-match
        &\runa{S-match}\infrule{
        \begin{matrix}
            \varphi;\Phi;\Gamma \vdash e:\natinterval{I}{J} \quad \varphi;\Phi, I \leq 0;\Gamma \vdash P \triangleleft \kappa\\
            \varphi;\Phi, J \geq 1;\Gamma, x:\natinterval{I-1}{J-1} \vdash Q \triangleleft \kappa'
        \end{matrix}}{\varphi;\Phi;\Gamma \vdash \match{e}{P}{x}{Q} \triangleleft \text{basis}(\varphi, \Phi, \kappa \cup \kappa')}\\[-1em]
        % S-iserv
        &\runa{S-iserv}\infrule{\begin{matrix}
            \texttt{in} \in \sigma\quad \varphi;\Phi;\Gamma\vdash a:\servt{I}{i}{\sigma}{K}{\widetilde{T}}\\
            \varphi, \widetilde{i}; \Phi; \text{ready}(\varphi,\Phi,\tforwardsim{\Gamma}{I}), \widetilde{v} : \widetilde{T} \vdash P \triangleleft \kappa \quad \varphi,\widetilde{i};\Phi\vDash\kappa \leq K
        \end{matrix}}
        {\varphi;\Phi;\Gamma \vdash \;\bang\inputch{a}{\widetilde{v}}{}{P}\triangleleft \{I\}}
        %
        \kern14em\runa{S-nil}\kern-1em\infrule{}{\varphi;\Phi;\Gamma \vdash \nil \triangleleft \{0\}}\kern-3em\text{ }\\[-1em]
        % S-oserv
        &\runa{S-oserv}\infrule{\begin{matrix}
            \texttt{out} \in \sigma\quad \varphi;\Phi;\Gamma\vdash a:\servt{I}{i}{\sigma}{K}{\widetilde{T}}\\
            \varphi; \Phi;\tforwardsim{\Gamma}{I} \vdash \widetilde{e}:\widetilde{S} \quad \text{instantiate}(\widetilde{i}, \widetilde{S}) = \{\widetilde{J}/\widetilde{i}\} \quad \varphi;\Phi \vDash \widetilde{S} \sqsubseteq \widetilde{T}
        \end{matrix}}
        {\varphi;\Phi;\Gamma \vdash \asyncoutputch{a}{\widetilde{e}}{}\triangleleft \{K\{\widetilde{J}/\widetilde{i}\} + I\}}\\[-1em]
        % S-annot
        &\runa{S-ich}\infrule{\begin{matrix}
            \texttt{in} \in \sigma\quad \varphi;\Phi;\Gamma \vdash a:\chant{\sigma}{I}{\widetilde{T}}\\
            \varphi; \Phi; \tforwardsim{\Gamma}{I}, \widetilde{v}:\widetilde{T} \vdash P \triangleleft \kappa
        \end{matrix}}
        {\varphi;\Phi;\Gamma \vdash \inputch{a}{\widetilde{v}}{}{P} \triangleleft \kappa + I}\kern3em
        %
        \runa{S-och}\infrule{\begin{matrix}
            \texttt{out} \in \sigma\quad \varphi;\Phi;\Gamma \vdash a:\chant{\sigma}{I}{\widetilde{T}}\\
            \varphi; \Phi; \tforwardsim{\Gamma}{I} \vdash \widetilde{e}:\widetilde{S} \quad \varphi;\Phi \vDash \widetilde{S} \sqsubseteq \widetilde{T}
        \end{matrix}}
        {\varphi;\Phi;\Gamma \vdash \asyncoutputch{a}{\widetilde{e}}{} \triangleleft \{I\}}\\[-1em]
    \end{align*}\vspace{-1em}\end{framed}
    \smallskip
    \caption{Sized typing rules for parallel complexity of processes.}
    \label{tab:sizedprocesstypingrules}
\end{table*}

%
Rule $\runa{S-iserv}$ types a replicated input on a name $a$, and so $a$ must be bound to a server type with input capability. As the index $I$ in the server type denotes the time steps remaining before the server is ready to synchronize, we advance the time by $I$ units of time complexity when typing the continuation $P$. To ensure that bounds on synchronizations in $\downarrow^{\varphi;\Phi}_I\!\Gamma$ are not violated, we type $P$ under the time invariant part of $\downarrow^{\varphi;\Phi}_I\!\Gamma$, i.e. $\text{ready}(\varphi,\Phi,\downarrow_I\!\Gamma)$. Note that the bound on the span of the replicated input is the bound on the time remaining before the server is ready to synchronize. As the replicated input may be invoked many times, the cost of invoking the server is accounted for in rule $\runa{S-oserv}$ using the complexity bound $K$ in the server type. Therefore, we enforce that $K$ is in fact an upper bound on the span of the continuation $P$.\\

The rule $\runa{S-oserv}$ types outputs on names bound to server types. Here, as stated above, we must account for the cost of invoking a server, and as a replicated input on a server is parametric, we must \textit{instantiate} it based on the types of the expressions we are to output. Recall that in the type rule for outputs on servers from Chapter \ref{ch:bgts}, this is to be done by finding a substitution that satisfies the premise $\widetilde{T} \sqsubseteq \widetilde{S}\{\widetilde{J}/\widetilde{i}\}$. However, this turns out to be a difficult problem, and we can in fact prove it NP-complete for types of polynomial indices even if we disregard subtyping. However, note that it might not be necessary to use the full expressive power of polynomial indices, and so this may not necessarily affect type checking. Nevertheless, we over-approximate finding such a substitution, by using the function $\textit{instantiate}$. That is, we \textit{zip} together the index variables $\widetilde{i}$ with indices in types $\widetilde{T}$. Remark that Baillot and Ghyselen \cite{BaillotGhyselen2021} propose types for inference in their technical report, where the problem is simplified substantially, by forcing naturals to have lower bounds of $0$ and upper bounds with exactly one index variable and a constant. Our approach admits more expressive lower bounds and multiplications, while imposing no direct restrictions on the number of index variables in an index, and is thus more suitable for a type-checker.\\

We now prove the NP-completeness of the smaller problem of checking whether there exists a substitution $\{\widetilde{J}/\widetilde{i}\}$ that satisfies $T = S\{\widetilde{J}/\widetilde{i}\}$ where $T$ and $S$ are types with polynomial indices. The main idea is a reduction proof from the NP-complete 3-SAT problem, i.e. the satisfiability problem of a boolean formula in conjunctive normal form with exactly three literals in each clause \cite{Karp1972}. We first define a translation from a 3-SAT formula to a polynomial index in Definition \ref{def:3satredu}. This is a polynomial time computable reduction, as we simply replace each logical-and with a multiplication, each logical-or with an addition and each negation with a subtraction from 1. In Lemma \ref{lemma:soundtranslation}, we prove that the reduction is faithful with respect to satisfiability of a boolean formula. Finally, in Lemma \ref{lemma:npcompletesubst}, we prove that it is an NP-complete decision problem to verify the existence of a substitution that satisfies $T = S\{\widetilde{J}/\widetilde{i}\}$ for types $T$ and $S$.
%
\begin{defi}[3-SAT reduction]\label{def:3satredu}
We assume a one-to-one mapping $f$ from unknowns to index variables. Let $\phi$ be a 3-SAT formula
\begin{align*}
    \phi = \bigwedge_{i=1}^n \left(\ell_{i1} \lor \ell_{i2} \lor \ell_{i3}\right)% \land \cdots \land (A_n \lor B_n \lor C_n)
\end{align*}
where $\ell_{i1}$, $\ell_{i2}$ and $\ell_{i3}$ are of the forms $x$ or $\neg x$ for some variable $x$. We define a translation of $\phi$ to a polynomial index %$[\![\phi]\!]_{\text{3-SAT}}$
\begin{align*}
    [\![\phi]\!]_{\text{3-SAT}} = \prod_{i=1}^n \left([\![\ell_{i1}]\!]_{\text{3-SAT}} + [\![\ell_{i2}]\!]_{\text{3-SAT}} + [\![\ell_{i3}]\!]_{\text{3-SAT}}\right) %\cdots ([\![A_n]\!]_{\text{3-SAT}} + [\![B_n]\!]_{\text{3-SAT}} + [\![C_n]\!]_{\text{3-SAT}})
\end{align*}
where $[\![x]\!]_{\text{3-SAT}} = f(x)$ and $[\![\neg x]\!]_{\text{3-SAT}} = (1 - f(x))$.
\end{defi}


\begin{lemma}\label{lemma:soundtranslation}
Let $\phi$ be a 3-SAT formula. Then $\phi$ is satisfiable if and only if there exists a substitution $\{\widetilde{n}/\widetilde{i}\}$ such that $1\leq [\![\phi]\!]_{\text{3-SAT}}\{\widetilde{n}/\widetilde{i}\}$.
\begin{proof}
We consider the implications separately
\begin{enumerate}
    \item Assume that $\phi$ is satisfiable. Then there exists a truth assignment $\tau$ such that each clause of $\phi$ is true. Correspondingly, as $[\![\phi]\!]_{\text{3-SAT}}$ is a product of non-negative factors, we for some substitution $\{\widetilde{n}/\widetilde{i}\}$ have that $1 \leq [\![\phi]\!]_{\text{3-SAT}}\{\widetilde{n}/\widetilde{i}\}$ if and only if each factor in the product is positive. We compare the conditions for a clause to be true in $\phi$ to those for a corresponding factor in $[\![\phi]\!]_{\text{3-SAT}}$ to be positive, and show that a substitution $\{\widetilde{n}/\widetilde{i}\}$ exists such that $1 \leq [\![\phi]\!]_{\text{3-SAT}}\{\widetilde{n}/\widetilde{i}\}$. A clause in $\phi$ is a disjunction of three literals of either the form $x$ or $\neg x$ for some unknown $x$. Thus, for a clause to be true, we must have at least one literal $\tau(x) = tt$ or $\neg \tau(x) = tt$ with $\tau(x) = f\!f$. The corresponding factor in $[\![\phi]\!]_{\text{3-SAT}}$ is a sum of three terms of the forms $f(x)$ or $(1 - f(x))$ for some unknown $x$, where $f$ is a one-to-one mapping from unknowns to index variables. Here, we utilize that in the type system by Baillot and Ghyselen \cite{BaillotGhyselen2021}, we have $(1 - i\{\widetilde{n}/\widetilde{i}\}) = 0$ when $i\{\widetilde{n}/\widetilde{i}\} \geq 1$ and $(1 - i\{\widetilde{n}/\widetilde{i}\}) = 1$ when $i\{\widetilde{n}/\widetilde{i}\} = 0$. Thus, for a factor to be positive, it suffices that one term is positive, and so we can construct a substitution that guarantees this from the interpretation of $\phi$. That is, if $\tau(x) = tt$, we substitute $1$ for $f(x)$, and if $\tau(x) = f\!f$, we substitute 0 for $f(x)$. Then, whenever a literal is true in $\phi$, the corresponding term in $[\![\phi]\!]_{\text{3-SAT}}$ is positive, and so if $\phi$ is satisfiable then there exists a substitution $\{\widetilde{n}/\widetilde{i}\}$ such that $1 \leq [\![\phi]\!]_{\text{3-SAT}}\{\widetilde{n}/\widetilde{i}\}$.
     
    \item Assume that there exists a substitution $\{\widetilde{n}/\widetilde{i}\}$ such that $1 \leq [\![\phi]\!]_{\text{3-SAT}}\{\widetilde{n}/\widetilde{i}\}$. Then, as $[\![\phi]\!]_{\text{3-SAT}}$ is a product of non-negative factors, each factor must be positive. Correspondingly, if $\Phi$ is satisfiable, then there exists a truth assignment such that each clause of $\phi$ is true. We compare the conditions for a factor in $[\![\phi]\!]_{\text{3-SAT}}\{\widetilde{n}/\widetilde{i}\}$ to be positive to those for a corresponding clause in $\phi$ to be true, and show that $\phi$ is satisfiable. A factor in $[\![\phi]\!]_{\text{3-SAT}}$ is a sum of at most three terms of the forms $f(x)\{\widetilde{n}/\widetilde{i}\}$ or $(1 - f(x)\{\widetilde{n}/\widetilde{i}\})$. Here we again utilize that in the type system by Baillot and Ghyselen \cite{BaillotGhyselen2021}, we have $(1 - f(x)\{\widetilde{n}/\widetilde{i}\}) = 0$ when $f(x)\{\widetilde{n}/\widetilde{i}\} \geq 1$ and $(1 - f(x)\{\widetilde{n}/\widetilde{i}\}) = 1$ when $f(x)\{\widetilde{n}/\widetilde{i}\} = 0$, and so it must be that in the factor, we have at least one term $f(x)\{\widetilde{n}/\widetilde{i}\} \geq 1$ or $(1 - f(x)\{\widetilde{n}/\widetilde{i}\}) \geq 1$. Correspondingly, for the clause in $\phi$ to be true, at least one literal must be true. We show that there exists a truth assignment $\tau$ such that if a term in $[\![\phi]\!]_{\text{3-SAT}}$ is positive, then the corresponding literal in $\phi$ is true. If $f(x)\{\widetilde{n}/\widetilde{i}\}\geq 1$ then we set $\tau(x) = tt$, and if $f(x)\{\widetilde{n}/\widetilde{i}\} = 0$ we set $\tau(x) = f\!f$, as $[\![x]\!]_{\text{3-SAT}}\{\widetilde{n}/\widetilde{i}\} \geq 1$ when $f(x)\{\widetilde{n}/\widetilde{i}\} \geq 1$ and $[\![\neg x]\!]_{\text{3-SAT}} \geq 1$ when $f(x)\{\widetilde{n}/\widetilde{i}\}=0$. Then, whenever a term is positive in $[\![\phi]\!]_{\text{3-SAT}}\{\widetilde{n}/\widetilde{i}\}$, the corresponding literal in $\phi$ is true, and so if there exists a substitution $\{\widetilde{n}/\widetilde{i}\}$ such that $1 \leq [\![\phi]\!]_{\text{3-SAT}}\{\widetilde{n}/\widetilde{i}\}$, then $\phi$ is satisfiable.
    
\end{enumerate}
\end{proof}
\end{lemma}


\begin{lemma}\label{lemma:npcompletesubst}
Let $T$ and $S$ be types with polynomial indices. Then checking whether there exists a substitution $\{\widetilde{J}/\widetilde{i}\}$ such that $T = S\{\widetilde{J}/\widetilde{i}\}$ is an NP-complete problem.
\begin{proof}
By reduction from the 3-SAT problem. Assume that we have some algorithm that can verify the existence of a substitution $\{\widetilde{J}/\widetilde{i}\}$ such that $T = S\{\widetilde{J}/\widetilde{i}\}$, and let $\phi$ be a 3-SAT formula. Then using the algorithm, we can check whether $\phi$ is satisfiable by verifying whether there exists $\{\widetilde{J}/\widetilde{i}\}$ such that the following holds
\begin{align*}
    \texttt{Nat}[0,1] = \texttt{Nat}[0,(1 - (1 - [\![\phi]\!]_{\text{3-SAT}}))]\{\widetilde{J}/\widetilde{i}\}
\end{align*}
That is, $1 = (1 - (1 - [\![\phi]\!]_{\text{3-SAT}}\{\widetilde{J}/\widetilde{i}\}))$ implies $1 \leq [\![\phi]\!]_{\text{3-SAT}}\{\widetilde{J}/\widetilde{i}\}$, as $(1 - [\![\phi]\!]_{\text{3-SAT}}\{\widetilde{J}/\widetilde{i}\}) = 0$ when $[\![\phi]\!]_{\text{3-SAT}}\{\widetilde{J}/\widetilde{i}\} \geq 1$ and $(1 - [\![\phi]\!]_{\text{3-SAT}}\{\widetilde{J}/\widetilde{i}\}) = 1$ when $[\![\phi]\!]_{\text{3-SAT}}\{\widetilde{J}/\widetilde{i}\} = 0$. Furthermore, for $1 \leq [\![\phi]\!]_{\text{3-SAT}}\{\widetilde{J}/\widetilde{i}\}$ to hold, the indices in the sequence $\widetilde{J}$ cannot contain index variables, and so there must exist an equivalent substitution of naturals for index variables $\{\widetilde{n}/\widetilde{i}\}$. Then, by Lemma \ref{lemma:soundtranslation} we have that $\phi$ is satisfiable if and only if there exists a substitution $\{\widetilde{n}/\widetilde{i}\}$ such that $1\leq [\![\phi]\!]_{\text{3-SAT}}\{\widetilde{n}/\widetilde{i}\}$. Thus, as 3-SAT is an NP-complete problem, the reduction from 3-SAT is computable in polynomial time and as polynomial reduction is a transitive relation, i.e. any NP-problem is polynomial time reducible to verifying the existence of a substitution $\{\widetilde{J}/\widetilde{i}\}$ that satisfies the equation $T = S\{\widetilde{J}/\widetilde{i}\}$, it follows that the problem is NP-hard. To show that it is an NP-complete problem, we show that a \textit{certificate} can be verified in polynomial time. That is, given some substitution $\{\widetilde{J}/\widetilde{i}\}$, we can in linear time check whether $T=S\{\widetilde{J}/\widetilde{i}\}$ by substituting indices $\widetilde{J}$ for indices $\widetilde{i}$ in type $S$ and by then comparing the two types.\\
%
%
%Utilizing that $n - m = 0$ for $m\geq n$ in the type system of Baillot and Ghyselen \cite{BaillotGhyselen2021}, we can simulate any boolean formula using a polynomial index. By denoting $J = 0$ false and $I > 0$ true, we have the translation
% \begin{align*}
%     [\![a \land b]\!]_\phi =&\; [\![a]\!]_\phi [\![b]\!]_\phi\\
%     [\![a \lor b]\!]_\phi =&\; [\![a]\!]_\phi + [\![b]\!]_\phi\\
%     [\![\neg a]\!]_\phi =&\; (1 - [\![a]\!]_\phi)\\
%     [\![x]\!]_\phi =&\; i
% \end{align*}
% Then assuming some algorithm that checks whether there exists a substitution $\{\widetilde{J}/\widetilde{i}\}$ such that $T \sqsubseteq S\{\widetilde{J}/\widetilde{i}\}$, we can solve the boolean satisfiability problem. Let $\phi_0$ be any boolean formula and let $\widetilde{i}$ be the index variables in $[\![\phi_0]\!]_\phi$, and assume that there exists a substitution $\{\widetilde{J}/\widetilde{i}\}$ that satisfies the judgement
% \begin{align*}
%     \emptyset;\emptyset\vDash\texttt{Nat}[0,1] \sqsubseteq \texttt{Nat}[0,[\![\phi_0]\!]_\phi]\{\widetilde{J}/\widetilde{i}\}
% \end{align*}
% Then by rule $\runa{SS-nweak}$ we have that $\emptyset;\emptyset\vDash 1 \leq [\![\phi_0]\!]_\phi\{\widetilde{J}/\widetilde{i}\}$, and as $\varphi = \emptyset$, the indices $\widetilde{J}$ must be constants. Thus, $\emptyset;\emptyset\vDash 1 \leq [\![\phi_0]\!]_\phi\{\widetilde{J}/\widetilde{i}\}$ is equivalent to $1 \leq [\![\phi_0]\!]_\phi\{\widetilde{J}/\widetilde{i}\}$, and so $\phi_0$ must have a solution. If instead no such substitution exists, then for any $\{\widetilde{J}/\widetilde{i}\}$, it must be that $[\![\phi_0]\!]_\phi\{\widetilde{J}/\widetilde{i}\} = 0$ implying that $\phi_0$ is a contradiction. Therefore, as the boolean satisfiability problem is NP-complete, the algorithm we assumed must be NP-complete as well.
\end{proof}
\end{lemma}

In Example \ref{example:addition}, we show how a process implementing addition of naturals can be typed using our type rules, yielding a precise bound on the parallel complexity.
%

\begin{examp}\label{example:addition}
As an example of a process that is typable using our type rules, we show how the addition operator for naturals can be written as a process and subsequently be typed. We use a server to encode the addition operator
\begin{align*}
    !\inputch{\text{add}}{x,y,r}{}{\match{x}{\asyncoutputch{r}{y}{}}{z}{\tick{\asyncoutputch{\text{add}}{z,\succc y,r}{}}}}
\end{align*}
such that channel $r$ is used to output the addition of naturals $x$ and $y$. To type the process, we use the following contexts and set of index variables
\begin{align*}
    \Gamma\defeq&\; \text{add} : \forall_0 i,j,k,l,m,n,o.\texttt{serv}^{\{\texttt{in},\texttt{out}\}}_j(\texttt{Nat}[0,j],\texttt{Nat}[0,l],\texttt{ch}^{\{\texttt{out}\}}_j(\texttt{Nat}[0,j+l])) \\
    \Delta\defeq&\; \text{ready}(\cdot,\cdot,\Gamma), x : \texttt{Nat}[0,j], y: \texttt{Nat}[0,l], r:\texttt{ch}^{\{\texttt{out}\}}_j(\texttt{Nat}[0,j+l])\\
    \varphi \defeq&\; \{i,j,k,l,m,n,o\}
\end{align*}
%
We now derive a type for the encoding of the addition operator, yielding a precise bound of $j$, corresponding to an upper bound on the size of $x$, as we pattern match at most $j$ times on natural $x$. Notably we have that $\text{instantiate}((i,j,k,l,m,n,o),\texttt{Nat}[0,j\monus 1],\texttt{Nat}[1,l+1],\texttt{ch}^{\{\texttt{out}\}}_j(\texttt{Nat}[0,j+l]))=\{0/i,j\monus 1/j,0/k,l+1/l,j/m,0/n,j+l/o\}$.
%
{\small
\begin{align*}
    \begin{prooftree}
        %
        \infer0{\varphi;\cdot,0\leq 0;\Delta\vdash \asyncoutputch{r}{y}{} \triangleleft \{j\}}
        %
        % \infer0{\texttt{Nat}[0,j\monus 1] \sqsubseteq \texttt{Nat}[0,j]\{j\monus 1/j\}}
        % %
        % \infer0{\texttt{Nat}[0,l+1] \sqsubseteq \texttt{Nat}[0,l]\{l+1/l\}}
        % %
        % \infer0{\texttt{ch}^{\{\texttt{out}\}}_{j\monus 1}(\texttt{Nat}[0,j+l] \sqsubseteq \texttt{ch}^{\{\texttt{out}\}}_{j\monus 1}(\texttt{Nat}[0,j+l)\{j\monus 1/j,l+1/l\}}
        %
        \infer0{
        \begin{matrix}
        \varphi;\cdot,1\leq j\vdash\texttt{Nat}[0,j\monus 1] \sqsubseteq \texttt{Nat}[0,j]\{j\monus 1/j\}\\
        \varphi;\cdot,1\leq j\vdash\texttt{Nat}[1,l+1] \sqsubseteq \texttt{Nat}[0,l]\{l+1/l\}\\
        \varphi;\cdot,1\leq j\vdash\texttt{ch}^{\{\texttt{out}\}}_{j\monus 1}(\texttt{Nat}[0,j+l] \sqsubseteq \texttt{ch}^{\{\texttt{out}\}}_{j\monus 1}(\texttt{Nat}[0,j+l)\{j\monus 1/j,l+1/l\}
        \end{matrix}
        }
        %
        \infer1{\varphi;\cdot,1\leq j;\susumesim{\Delta}{1},z : \texttt{Nat}[0,j\monus 1]\vdash \asyncoutputch{\text{add}}{z,\succc y, r}{} \triangleleft \{j\monus 1\}}
        %
        \infer1{\varphi;\cdot,1\leq j;\Delta,z : \texttt{Nat}[0,j\monus 1]\vdash \tick{\asyncoutputch{\text{add}}{z,\succc y, r}{}} \triangleleft \{j\}}
        %
        \infer2{\varphi;\cdot;\Delta\vdash \match{x}{\asyncoutputch{r}{y}{}}{z}{\tick{\asyncoutputch{\text{add}}{z,\succc y,r}{}}} \triangleleft \{j\}}
        %
        \infer1{\cdot;\cdot;\Gamma\vdash\; !\inputch{\text{add}}{x,y,r}{}{\match{x}{\asyncoutputch{r}{y}{}}{z}{\tick{\asyncoutputch{\text{add}}{z,\succc y,r}{}}}}\triangleleft \{0\}}
    \end{prooftree}
\end{align*}}
%
\end{examp}

% \subsection{Undecidability of judgements}
% Verifying whether a polynomial constraint with integer coefficients imposes further restrictions onto the model set of index valuations of natural codomain of some set of known constraints can be reduced to Hilbert's tenth problem \cite{Davis1973}. That is, the problem of verifying whether a diophantine equation has an integer solution.\\

% We first assume some algorithm that can verify a judgement of the form $\varphi;\Phi\vDash C$ where $\varphi$ is a set of index variables and $C$ and $C'\in\Phi$ are binary constraints on polynomials of integer coefficients over relations from any subset of $\{\neq,\leq, <\}$. Recall that such a judgement holds exactly when for each index valuation $\rho : \varphi \longrightarrow \mathbb{N}$ over $\varphi$ for which $\rho \vDash C'$ for $C'\in\Phi$ we also have $\rho\vDash C$, i.e. $C$ does not impose further restrictions on interpretations of indices.\\

% We can then verify whether any diophantine equation has an integer solution. Let $p$ be an arbitrary polynomial of integer coefficients such that $p = 0$ is a diophantine equation. As only non-negative integers substitute for index variables, we first transform $p = 0$ to a new diophantine equation $p' = 0$ that has a non-negative integer solution exactly when $p = 0$ has an integer solution. To do this, we simply replace each index variable $i$ in $p$ with two new index variables $i_1 - i_2$. Then the judgement $\varphi;\emptyset\vDash p' \neq 0$ holds exactly when $p=0$ has no integer solution. That is, if $p=0$ has an integer solution, then there must exist a valuation $\rho_0$ such that $\rho_0\vDash \emptyset$ with $[\![p']\!]_{\rho_0} = 0$ and so $\rho_0\nvDash p' \neq 0$. Moreover, we need not rely on the relation $\neq$, as the judgements below are equivalent
% \begin{align*}
%     \varphi;\{p' \leq 0\} \vDash p' < 0\\
%     \varphi;\{p' \leq 0\} \vDash p' \leq 1
% \end{align*}
\section{Soundness}
\section{Verification of constraint judgements}\label{sec:verifyinglinearjudgements}
Until now we have not considered how we can verify constraint judgements in the type rules. The expressiveness of implementations of the type system by Baillot and Ghyselen \cite{BaillotGhyselen2021} depends on both the expressiveness of indices and whether judgements on the corresponding constraints are decidable. Naturally, we are interested in both of these properties, and so in this section, we show how judgements on linear constraints can be verified using algorithms. Later, we show how this can be extended to certain groups of polynomial constraints. We first make some needed changes to how the type checker uses subtraction.
%
\subsection{Subtraction of naturals}
The constraint judgements rely on a special minus operator ($\monus$) for indices such that $n \monus m=0$ when $m \geq n$, which we refer to as the \textit{monus} operator. This is apparent in the pattern match constructor type rule from Chapter \ref{ch:bgts}. Without this behavior, we may encounter problems when checking subtype premises in match processes. This has the consequence that equations such as $2\monus 3+3=3$ hold, such that indices form a semiring rather than a ring, as we are no longer guaranteed an additive inverse. In general, semirings lack many properties of rings that are desirable. For example, given two seemingly equivalent constraints $i \leq 5$ and $i \monus 5 \leq 0$, we see that by adding any constant to their left-hand sides, the constraints are no longer equivalent. Adding the constant 2 to their left-hand sides, we obtain $i + 2 \leq 5$ and $i \monus 5 + 2 \leq 0$, however, we see that the first constraint is satisfied given the valuation $i = 3$ but the second is not. In general the associative property of $+$ is lost.\\

Unfortunately, this is not an easy problem to solve implementation-wise, as indices are not actually evaluated but rather represent whole feasible regions. Thus, instead of trying to implement this operator exactly, we limit the number of processes typable by the type system. Removing the operator entirely is not an option as it us used by the type rules themselves. Instead, we ensure that one cannot \textit{exploit} the special behavior of monus by introducing additional conditions to the type rules of the type system. More precisely, any time the type system uses the monus operator such as $I \monus J$, we require the premise $\varphi;\Phi \vDash I \geq J$, in which case the monus operator is safe to treat as a regular minus. This, however, puts severe restrictions on the number of processes typable, and so we relax the restriction a bit by also checking the judgement $\varphi;\Phi \vDash I \leq J$, in which case we can conclude that the result is definitely $0$. If neither $\varphi;\Phi\vDash I \geq J$ nor $\varphi;\Phi\vDash I \leq 0$ hold, which is possible as $\leq$ and $\geq$ do not form a total order on indices, the result is undefined. We refer to this variant of monus as the \textit{partial} monus operator, as formalized in Definition \ref{def:partialmonus}. Note that this definition of monus allows us to obtain identical behavior to minus on a constraint $I \bowtie J$ by moving terms between the LHS and RHS, i.e. $I - K \bowtie J \Rightarrow I \bowtie J + K$, and so we can assume we have a standard minus operator when verifying judgements on constraints. For the remainder of this section, we assume this definition is used in the type rules instead of the usual monus. We may omit $\varphi;\Phi$ if it is clear from the context.%\\
%
%Definition \ref{def:partialmonus} defines the \textit{partial} monus operator that is undefined if we cannot determine if the result is either always positive or always zero. For the remainder of this thesis, we assume this definition is used in the type rules instead of the usual monus. We may omit $\varphi;\Phi$ if it is clear from the context.
%
\begin{defi}[Partial monus]\label{def:partialmonus}
Let $\Phi$ be a set of constraints in index variables $\varphi$. The partial monus operator is defined for two indices $I$ and $J$ as
\begin{equation*}
    I \monusE J = \begin{cases}
    I - J &\text{if $\varphi;\Phi \vDash J \leq I$}\\
    0 &\text{if $\varphi;\Phi \vDash I \leq J$}\\
    \textit{undefined} & \textit{otherwise}
    \end{cases}
\end{equation*}
\end{defi}

To ensure soundness of the algorithmic type rules after switching to the partial monus operator, we must make some changes to advancement of time. Consider the typing
\begin{align*}
    (\cdot,i);(\cdot,i\leq 3);\Gamma\vdash\; !\inputch{a}{}{}{\nil}  \mid 5 : \asyncoutputch{a}{}{} \triangleleft \{5\}
\end{align*}
where $\Gamma = \cdot,a : \forall_{3-i}\epsilon.\texttt{serv}^{\{\texttt{in},\texttt{out}\}}_0()$. Upon typing the time annotation, we advance the time of the server type by $5$ yielding the type $\forall_{3-i-5}\epsilon.\texttt{serv}^{\{\texttt{out}\}}_0()$ as $(\cdot,i);(\cdot,i\leq 3)\nvDash 3-i \geq 5$, which is defined as $(\cdot,i);(\cdot,i\leq 3)\vDash 3-i \leq 5$. However, if we apply the congruence rule $\runa{SC-sum}$ from right to left we obtain
\begin{align*}
    !\inputch{a}{}{}{\nil}  \mid 2 : 3 : \asyncoutputch{a}{}{}\equiv\;!\inputch{a}{}{}{\nil}  \mid 5 : \asyncoutputch{a}{}{}
\end{align*}
Then, we get a problem upon typing the first annotation. That is, as $(\cdot,i);(\cdot,i\leq 3)\nvDash 3-i \leq 2$ (i.e. when for some valuation $\rho$ we have $\rho(i) = 0$) the operation $(3-i) \monusE[(\cdot,i);(\cdot,i\leq 3)] 2$ is undefined. Thus, the type system loses its subject congruence property, and subsequently its subject reduction property. There are, however, several ways to address this. One option is to modify the type rules to perform a single advancement of time for a sequence of annotations. A more contained option is to remove monus from the definition of advancement of time, by enriching the formation rules of types with the constructor $\forall_{I}\widetilde{i}.\texttt{serv}^\sigma_K(\widetilde{T})^{-J}$ and by augmenting the definition of advancement as so
\begin{align*}
    \downarrow_I^{\varphi;\Phi}\!\!(\forall_J\widetilde{i}.\texttt{serv}^\sigma_K(\widetilde{T})) =&\; \left\{
\begin{matrix}
\forall_{J-I}\widetilde{i}.\texttt{serv}^\sigma_K(\widetilde{T}) & \text{ if } \varphi;\Phi\vDash I \leq J \\
\forall_0\widetilde{i}.\texttt{serv}^{\sigma\cap\{\texttt{out}\}}_K(\widetilde{T}) & \text{ if } \varphi;\Phi\vDash J \leq I \\
\forall_{J}\widetilde{i}.\texttt{serv}^{\sigma\cap\{\texttt{out}\}}_K(\widetilde{T})^{-I} & \text{ if } \varphi;\Phi\nvDash I \leq J \text{ and } \varphi;\Phi\nvDash J \leq I
\end{matrix}
\right.\\
%
\downarrow_I^{\varphi;\Phi}\!\!(\forall_{J}\widetilde{i}.\texttt{serv}^\sigma_K(\widetilde{T})^{-L}) =&\; \left\{
\begin{matrix}
\forall_{0}\widetilde{i}.\texttt{serv}^{\sigma\cap\{\texttt{out}\}}_K(\widetilde{T}) & \text{ if } \varphi;\Phi\vDash J \leq L+I \\
\forall_{J}\widetilde{i}.\texttt{serv}^{\sigma\cap\{\texttt{out}\}}_K(\widetilde{T})^{-(L+I)} & \text{ if } \varphi;\Phi\nvDash J \leq L+I
\end{matrix}
\right.
%
\end{align*}
This in essence introduces a form of \textit{lazy} time advancement, where time is not advanced until partial monus allows us to do so. Then, as the type rules for servers require a server type of the form $\forall_J\widetilde{i}.\texttt{serv}^\sigma_K(\widetilde{T})$, the summed advancement of time must always be less than or equal, or always greater than or equal to the time of the server, and so typing is invariant to the use of congruence rule $\runa{SC-sum}$. Revisiting the above example, we have that $\susume{\forall_{3-i}\epsilon.\texttt{serv}^{\{\texttt{in},\texttt{out}\}}_0()}{(\cdot,i)}{(\cdot,i\leq 3)}{5} =\; \susume{\susume{\forall_{3-i}\epsilon.\texttt{serv}^{\{\texttt{in},\texttt{out}\}}_0()}{(\cdot,i)}{(\cdot,i\leq 3)}{2}}{(\cdot,i)}{(\cdot,i\leq 3)}{3}$, and so we obtain the original typing
\begin{align*}
    (\cdot,i);(\cdot,i\leq 3);\Gamma\vdash\; !\inputch{a}{}{}{\nil}  \mid 2 : 3 : \asyncoutputch{a}{}{} \triangleleft \{5\}
\end{align*}



% \begin{remark}

%     Baillot and Ghyselen \cite{BaillotGhyselen2021} assume that the minus operator ($-$) for indices is defined such that $n-m=0$ when $m \geq n$. This has the consequence that expressions such as $2-3+3=3$ apply, such that indices form a semiring instead of a ring as we no longer have an additive inverse. In this work we lift this assumption by arguing that any index $I$ using a ring-centric definition for $-$ such that $I \leq 0$, can be simulated using another index $J$ using a semiring-centric definition for $-$ such that $J \leq 0$. For $I$, the order of summation of terms does not matter, and so we can freely change this. By moving any terms with a negative coefficient to the end of the summation, we obtain an expression of the form $c_1 i_1 + \cdots + c_n i_n - c_{n+1} i_{n+1} - \cdots - c_m i_m$ where $c_j$ are positive numbers and $i_j$ are index variables for $j = 0\dots m$. When evaluating this expression from left to right, the result will be increasing until $c_{n + 1} i_{n+1}$, as both the coefficients and index variables are positive, after which it will be decreasing. This results in an expression that is indifferent to the two definitions of $-$ when considering constraints of the form $I \leq 0$. Thus, a normalized constraint using a ring-centric definition of $-$ can be simulated using a normalized constraint using a semiring-centric definition of $-$.

% \end{remark}

\subsection{Undecidability of polynomial constraint judgements}
As we have seen, verifying that a constraint imposes no further restrictions onto index valuations amounts to checking whether all possible index valuations that satisfy a set of known constraints are also contained in the model space of our new constraint. It also amounts to checking whether the feasible region of the constraint contains the feasible region of a known system of inequality constraints, or checking whether the feasible region of the inverse constraint does not intersect the feasible region of a known system of inequality constraints. This turns out to be a difficult problem, and we can in fact prove it undecidable for diophantine constraints, i.e. multivariate polynomial inequalities with integer coefficients, when index variables must have natural (or integer) interpretations. The main idea is to reduce Hilbert's tenth problem \cite{Hilbert1902} to that of verification of judgements on constraints, as this problem has been proven undecidable \cite{Davis1973}. That is, we show that assuming some complete algorithm that verifies judgements on constraints, we can verify whether an arbitrary diophantine equation has a solution with all unknowns taking integer values. We show this result in Lemma \ref{lemma:judgementUndecidable}.
%
\begin{lemma}\label{lemma:judgementUndecidable}
Let $C$ and $C'\in \Phi$ be diophantine inequalities with unknowns in $\varphi$ and coefficients in $\mathbb{N}$. Then the judgement $\varphi;\Phi\vDash C$ is undecidable.
\begin{proof}
By reduction from Hilbert's tenth problem. Let $p=0$ be an arbitrary diophantine equation. We show that assuming some algorithm that can verify a judgement of the form $\varphi;\Phi\vDash C$, we can determine whether $p=0$ has an integer solution. We must pay special attention to the non-standard definition of subtraction in the type system by Baillot and Ghyselen \cite{BaillotGhyselen2021} and to the fact that only non-negative integers substitute for index variables. We first replace each integer variable $x$ in $p$ with two non-negative variables $i_x - j_x$, referring to the modified polynomial as $p'$. We can quickly verify that $p'=0$ has a non-negative integer solution if and only if $p=0$ has an integer solution
\begin{enumerate}
    \item Assume that $p'=0$ has a non-negative integer solution. Then for each variable $x$ in $p$ we assign $x = i_x - j_x$ reaching an integer solution to $p$.
    
    \item Assume that $p=0$ has an integer solution. Then for each pair $i_x$ and $j_x$ in $p'$ we assign $i_x = x$ and $j_x = 0$ when $x \geq 0$ and $i_x = 0$ and $j_x = |x|$ when $x < 0$ reaching a non-negative integer solution to $p'$.
\end{enumerate}
%
Then, by the distributive property of integer multiplication and the associative property of integer addition, we can utilize that $p'$ has an equivalent expanded form 
\begin{align*}
p' = n_1 t_1 + \cdots + n_k t_k + n_{k+1} t_{k+1} + \cdots + n_{k+l} t_{k+l}    
\end{align*}
such that $n_1,\dots,n_k\in\mathbb{N}$, $n_{k+1},\dots,n_{k+l} \in \mathbb{Z}^{\leq 0}$ and $t_1,\dots,t_k,t_{k+1},\dots,t_{k+l}$ are power products over the set of all index variables in $p'$ denoted $\varphi_{p'}$. We can then factor the negative coefficients
\begin{align*}
    p' \;&= n_1 t_1 + \cdots + n_k t_k + n_{k+1} t_{k+1} + \cdots + n_{k+l} t_{k+l}\\ 
    \;&= (n_1 t_1 + \cdots + n_k t_k) + (-1)(|n_{k+1}| t_{k+1} + \cdots + |n_{k+l}| t_{k+l})\\
    \;&= (n_1 t_1 + \cdots + n_k t_k) - (|n_{k+1}| t_{k+1} + \cdots + |n_{k+l}| t_{k+l})
\end{align*}
We use this to show that $p'=0$ has a non-negative integer solution if and only if the following judgement does not hold 
{\small
\begin{align*}
    \varphi_{p'};\{|n_{k+1}| t_{k+1} + \cdots + |n_{k+l}| t_{k+l} \leq n_1 t_1 + \cdots + n_k t_k\}\vDash 1 \leq (n_1 t_1 + \cdots + n_k t_k) - (|n_{k+1}| t_{k+1} + \cdots + |n_{k+l}| t_{k+l}) 
\end{align*}}
We consider the implications separately
\begin{enumerate}
    \item Assume that $p'=0$ has a non-negative integer solution. Then we have that $n_1 t_1 + \cdots + n_k t_k = |n_{k+1}| t_{k+1} + \cdots + |n_{k+l}| t_{k+l}$, and so there must exist a valuation $\rho : \varphi_{p'} \longrightarrow \mathbb{N}$ such that $[\![n_1 t_1 + \cdots + n_k t_k]\!]_\rho = [\![|n_{k+1}| t_{k+1} + \cdots + |n_{k+l}| t_{k+l}]\!]_\rho$. We trivially have that $\rho$ satisfies $[\![|n_{k+1}| t_{k+1} + \cdots + |n_{k+l}| t_{k+l}]\!]_\rho \leq [\![n_1 t_1 + \cdots + n_k t_k]\!]_\rho$. But $\rho$ is not in the model space of the constraint $1 \leq (n_1 t_1 + \cdots + n_k t_k) - (|n_{k+1}| t_{k+1} + \cdots + |n_{k+l}| t_{k+l})$, and so the judgement does not hold.
    
    \item Assume that the judgement does not hold. Then there must exist a valuation $\rho : \varphi_{p'} \longrightarrow \mathbb{N}$ that satisfies $[\![|n_{k+1}| t_{k+1} + \cdots + |n_{k+l}| t_{k+l}]\!]_\rho \leq [\![n_1 t_1 + \cdots + n_k t_k]\!]_\rho$, but that is not in the model space of the constraint $1 \leq (n_1 t_1 + \cdots + n_k t_k) - (|n_{k+1}| t_{k+1} + \cdots + |n_{k+l}| t_{k+l})$. This implies that $[\![n_1 t_1 + \cdots + n_k t_k]\!]_\rho = [\![|n_{k+1}| t_{k+1} + \cdots + |n_{k+l}| t_{k+l}]\!]_\rho$, and so $p'$ has a non-negative integer solution.
\end{enumerate}
% Then the subtraction operator in Baillot and Ghyselen $\cite{BaillotGhyselen2021}$ only has non-standard behavior when $[\![n_{k+1} t_{k+1} + \cdots + n_{k+l} t_{k+l}]\!]_\rho > [\![n_1 t_1 + \cdots + n_k t_k]\!]_\rho$ for some interpretation $\rho : \varphi_{p'} \longrightarrow \mathbb{N}$ where $\varphi_{p'}$ is the set of all index variables in $p'$. Thus, we have that the judgement
% \begin{align*}
%     \varphi_{p'};\{n_{k+1} t_{k+1} + \cdots + n_{k+l} t_{k+l} \leq n_1 t_1 + \cdots + n_k t_k\}\vDash 1 \leq (n_1 t_1 + \cdots + n_k t_k) - (n_{k+1} t_{k+1} + \cdots + n_{k+l} t_{k+l}) 
% \end{align*}
% holds exactly when there exists no index valuation $\rho$ over $\varphi_{p'}$ that simultaneously satisfies $[\![n_{k+1} t_{k+1} + \cdots + n_{k+l} t_{k+l}]\!]_\rho \leq [\![n_1 t_1 + \cdots + n_k t_k]\!]_\rho$ and $[\![p']\!]_\rho = 0$. 
As such, we can verify that the above judgement does not hold if and only if $p'$ has a non-negative integer solution, and by extension if and only if $p$ has an integer solution. Thus, we would have a solution to Hilbert's tenth problem, which is undecidable.
\end{proof}
\end{lemma}

As an unfortunate consequence of Lemma \ref{lemma:judgementUndecidable}, we are forced into considering approximate algorithms for verification of judgements over polynomial constraints (in general). However, this result does not imply that type checking is undecidable. It may well be that problematic judgements are not required to type check any process, as computational complexity has certain properties, such as monotonicity. Note that the freedom of type checking, i.e. we can specify an arbitrary type context as well as type annotations, enables us to select indices that lead to undecidable judgements. To prove that type checking is undedidable, however, a more reasonable result would be that there exists a process that is typable if and only if an undecidable judgement is satisfied. This is out of the scope of this thesis, and so we leave it as future work. % Remark that Baillot and Ghyselen \cite{BaillotGhyselen2021} introduce a notion of type inference in their technical report, where the set of constraints $\Phi$ is empty for any judgement on constraints, and so they are able to bypass some of the problems associated with checking such judgements. However, this comes at the price of expressiveness, as natural types are forced to have lower bounds of $0$ and upper bounds with exactly one index variable and constant. Such indices are arguably sufficient for describing the sizes of simple terms when all operations on these terms in a program can be correspondingly described with a single index variable and constant. However, this quickly becomes too restrictive, as we are unable to type servers that implement simple arithmetic operations such as addition and subtraction.

\subsection{Normalization of linear indices}

To make checking of judgements on constraints tractable, we reduce the set of function symbols on which indices are defined, such that indices may only contain integers and index variables, as well as addition, subtraction and scalar multiplication operators, such that we restrict ourselves to linear functions.
\begin{align*}
        I,J ::= n \mid i \mid I + J \mid I - J \mid n I
    \end{align*}
% \begin{defi}[Indices]
%     \begin{align*}
%         I,J ::= n \mid i \mid I + J \mid I - J \mid I \cdot J
%     \end{align*}
% \end{defi}


Such indices can be written in a \textit{normal} form, presented in Definition \ref{def:normlinindex}.

\begin{defi}[Normalized linear index]\label{def:normlinindex}
    Let $I$ be an index in index variables $\varphi = i_1,\dots,i_n$. We say that $I$ is a \textit{normalized} index when it is a linear combination of index variables $i_1, ..., i_n$. Let $m$ be an integer constant and $I_\alpha\in\mathbb{Z}$ the coefficient of variable $i_\alpha$, we then define normalized indices as
    %
    \begin{align*}
        I = \normlinearindex{m}{I}
    \end{align*}
    
    
    We use the notation $\mathcal{B}(I)$ and $\mathcal{E}(I)$ to refer to the constant and unique identifiers of index variables of $I$, respectively.
\end{defi}

Any index can be transformed to an equivalent normalized index (i.e. it is a normal form) through expansion with the distributive law, reordering by the commutative and associative laws and then by regrouping terms that share variables. Therefore, the set of normalized indices in index variables $i_1,\dots,i_n$ and with coefficients in $\mathbb{Z}$, denoted $\mathbb{Z}[i_1,\dots,i_n]$, is a free module with the variables as basis, as the variables are linearly independent. In Definition \ref{def:operationsmodule}, we show how scalar multiplication, addition and multiplication of normalized indices (i.e. linear combinations of monomials) can be defined. Definition \ref{def:normalizationindex} shows how an equivalent normalized index can be computed from an arbitrary linear index using these operations.
%
\begin{defi}[Operations in $\freemodule$]\label{def:operationsmodule}
Let $I = \normlinearindex[\varphi_1]{n}{I}$ and $J = \normlinearindex[\varphi_2]{m}{J}$ be normalized indices in index variables $i_1,\dots,i_n$. We define addition and scalar addition of such indices. Given a scalar $n\in\mathbb{Z}$, the scalar multiplication $n I$ is
%
\begin{align*}
    n I = \normlinearindex[\mathcal{E}(I)]{n \cdot m}{n I}
\end{align*}
When $d$ is a common divisor of all coefficients in $I$, i.e. $I_\alpha / n \in \mathbb{Z}$ for all $\alpha\in\varphi$, the inverse operation is defined
\begin{align*}
    \frac{I}{d} = \frac{n}{d} + \sum_{\alpha\in \mathcal{E}(I)} \frac{I_\alpha}{d} i_\alpha\quad\text{if}\;\frac{I_\alpha}{d} \in \mathbb{Z}\;\text{for all}\;\alpha\in\mathcal{E}(I)
\end{align*}

The addition of $I$ and $J$ is the sum of constants plus the sum of scaled variables where coefficients $I_\alpha$ and $J_\alpha$ are summed when $\alpha\in\varphi_1 \cap \varphi_2$
\begin{align*}
    I + J = n + m + \sum_{\alpha \in \mathcal{E}(I) \cup \mathcal{E}(J)}(I_\alpha + J_\alpha)i_\alpha
\end{align*}

where for any $\alpha\in \varphi_1 \cup \varphi_2$ such that $I_\alpha + J_\alpha = 0$ we omit the corresponding zero term. The inverse of addition is always defined for elements of a polynomial ring
%
\begin{align*}
    I - J = n - m + \sum_{\alpha \in \mathcal{E}(I) \cup \mathcal{E}(J)}(I_\alpha - J_\alpha)i^\alpha
\end{align*}
\end{defi}
%
%We now formalize the transformation of an index $I$ to an equivalent normalized index in Definition \ref{def:normalizationindex}. An integer constant $n$ corresponds to scaling the monomial identified by the exponent vector of all zeroes by $n$. An index variable $i$ represents the monomial consisting of exactly one $i$ scaled by $1$. For addition, subtraction and multiplication we simply normalize the two subindices and and use the corresponding operators for normalized indices.
\begin{defi}[Index normalization]\label{def:normalizationindex}
The normalization of some index $I$ in index variables $i_1,\dots,i_n$ into an equivalent normalized index $\mathcal{N}(I)\in \mathbb{Z}[i_1,\dots,i_n]$ is a homomorphism defined inductively
    \begin{align*}
        \mathcal{N}(n) =&\; n i_1^0\cdots i_n^0\\
        \mathcal{N}(i_j) =&\; 1 i_1^0 \cdots i_j^1 \cdots i_n^0\\
        \mathcal{N}(I + J) =&\; \mathcal{N}(I) + \mathcal{N}(J)\\
        \mathcal{N}(I - J) =&\; \mathcal{N}(I) - \mathcal{N}(J)\\
        \mathcal{N}(n I) =&\; n \mathcal{N}(I)
    \end{align*}
\end{defi}

% \subsubsection{Normalization of constraints}
% A constraint may provide stronger or weaker restrictions on index variables compared to another constraint, or it may provide entirely different restrictions that are neither stronger nor weaker. For example, assuming some index $J$, if we have the constraint $3 \cdot i \leq J$, the constraint $2 \cdot i \leq J$ is redundant as index variables can only be assigned natural numbers, and thus $3 \cdot i \leq J$ implies $n \cdot i \leq J$ for any $n \leq 3$. Similarly, $I \leq n \cdot j$ implies $I \leq m \cdot j$ for any $n \leq m$. We thus define the subconstraint relation $\sqsubseteq$, and by extension the subindex relation $\sqsubseteq_\text{Index}$, in Definition \ref{def:subconstraint}. If $C_1 \sqsubseteq C_2$ we say that $C_2$ is a subconstraint of $C_1$.


% \begin{defi}[Subindices and subconstraints] \label{def:subconstraint}
%     We define the subindex relation $\sqsubseteq_\text{Index}$ by the following rule
%     \begin{align*}
%         &I \sqsubseteq_\text{Index} J \quad \text{ if} \\
%         &\quad (\mathcal{B}(I) \leq \mathcal{B}(J)) \land\\
%         &\quad (\forall \alpha \in \mathcal{E}(I) \cap \mathcal{E}(J) : I_\alpha \leq J_\alpha)\land\\
%         &\quad (\forall \alpha \in \mathcal{E}(J) \setminus \mathcal{E}(I) : J_\alpha \geq 0)\land\\
%         &\quad (\forall \alpha \in \mathcal{E}(I) \setminus \mathcal{E}(J) : I_\alpha \leq 0)
%     \end{align*}
%     % \begin{align*}
%     %     &(\varphi, F) \sqsubseteq_\text{Index} (\varphi', F') \text{ if} \\
%     %     &\quad (\forall V \in \varphi \cap \varphi' : F(V) \leq F'(V)) \land\\
%     %     &\quad (\forall V \in \varphi' \setminus \varphi : F'(V) \geq 0) \land\\
%     %     &\quad (\forall V \in \varphi \setminus \varphi' : F'(V) \leq 0)
%     % \end{align*}
    
%     We define the subconstraint relation $\sqsubseteq$ by the following rule
%     \begin{align*}
%       &\infrule{I' \sqsubseteq_\text{Index} I \quad J \sqsubseteq_\text{Index} J'}{I \leq J \sqsubseteq I' \leq J'}
%       %
%       %
%       %&\infrule{I \leq J \sqsubseteq I' \leq J' \quad I' \leq J' \sqsubseteq I'' \leq J''}{I \leq J \sqsubseteq I'' \leq J''}
%     \end{align*}
% \end{defi}

We extend normalization to constraints. We first note that an equality constraint $I = J$ is satisfied if and only if $I \leq J$ and $J \leq I$ are both satisfied. Thus, it suffices to only consider inequality constraints. A normalized constraint is of the form $I \leq 0$ for some normalized index $I$, as formalized in Definition \ref{def:normconst}.
%
\begin{defi}[Normalized constraints]\label{def:normconst}
    Let $C = I \leq J$ be an inequality constraint such that $I$ and $J$ are normalized indices. We say that $I-J \leq 0$ is the normalization of $C$ denoted $\mathcal{N}(C)$, and we refer to constraints in this form as \textit{normalized} constraints.
    %We represent normalized constraints $C$ using a single normalized constraint $I$, such that $C$ is of the form
    %\begin{align*}
    %    C = I \leq 0
    %\end{align*}
%
\end{defi}
%
% We now show how any constraint $J \bowtie K$ can be represented using a set of normalized constraints of the form $I \leq 0$ where $I$ is a normalized index. To do this, we first represent the constraint $J \bowtie K$ using a set of constraints of the form $J \leq K$ using the function $\mathcal{N_R}$. We then finalize the normalization using the function $\mathcal{N}$ by first moving all indices to the left-hand side of the constraint.
%
% \begin{defi}
%     Given a constraint $I \bowtie J$ $(\bowtie\; \in \{\leq, \geq, =\})$, the function $\mathcal{N_R}$ converts $I \bowtie J$ to a set of constraints of the form $I \leq J$
%     %
%     \begin{align*}
%         \mathcal{N_R}(I \leq J) &= \{I \leq J\}\\
%         \mathcal{N_R}(I \geq J) &= \{J \leq I\}\\
%         %\mathcal{N_R}(I < J) &= \{I+1 \leq J\}\\
%         %\mathcal{N_R}(I > J) &= \{J+1 \leq I\}\\
%         \mathcal{N_R}(I = J) &= \{I \leq J, J \leq I\}
%     \end{align*}
% \end{defi}
%
% \begin{defi}
%     Given a constraint $C$, the function $\mathcal{N}$ converts $C$ into a set of normalized constraints of the form $I \leq 0$
%     %
%     \begin{align*}
%         \mathcal{N}(C) &= \left\{I-J \leq 0 \mid (I \leq J) \in \mathcal{N}_R(C)\right\}
%     \end{align*}
% \end{defi}
%
%Normalized constraints have the key property that, given any two constraints $I \leq 0$ and $J \leq 0$, we can combine these to obtain a new constraint $J + I \leq 0$. This is possible as we know that both $I$ and $J$ are both non-positive, and so their sum must also be non-positive. In general, given $n$ normalized constraints $I_1 \leq 0, ..., I_n \leq 0$, we can infer any linear combination $a_1 \cdot I_1 \leq 0 + ... + a_n \cdot I_n \leq 0$ where $a_i \geq 0$ for $i = 1..n$ as new constraints that can be inferred based on the constraints $I_1 \leq 0, ..., I_n \leq 0$. Linear combinations where all coefficients are non-negative are also called \textit{conical combinations}.
Normalizing constraints has a number of benefits. First of all, it ensures that equivalent constraints are always expressed the same way. Secondly, having all constraints in a common form where variables only appear once means we can easily reason about individual variables of a constraint, which will be useful later when we verify constraint judgements.
%
\subsection{Checking for emptiness of model space}
As explained in Section \ref{sec:cjalternativeform}, we can verify a constraint judgement $\varphi;\Phi \vDash C_0$ by letting $C_0'$ be the inverse of $C_0$ and checking if $\mathcal{M}_\varphi(\Phi \cup \{C_0'\}) = \emptyset$ holds. Being able to check for non-emptiness of a model space is therefore paramount for verifying constraint judgements. For convenience, given a finite ordered set of index variables $\varphi = \{i_1, i_2, \dots, i_n\}$, we represent a normalized constraint $I \leq 0$ as a vector $\left( \mathcal{B}(I), I_1\; I_2\; \cdots\; I_n \right)_{\varphi}$. As such, the constraint $-5i + -2j + -4k \leq 0$ can be represented by the vector $\cvect[\varphi_1]{0 {-5} {-2} {-4}}$ where $\varphi_1=\left\{i, j, k\right\}$. Another way to represent that same constraint is with the vector $\cvect[\varphi_2]{0 {-5} {-2} 0 {-4}}$ where $\varphi_2 = \left\{i,j,l,k\right\}$. We denote the vector representation of a constraint $C$ over a finite ordered set of index variables $\varphi$ by $\mathbf{C}_{\varphi}$. We extend this notation to sets of constraints, such that $\Phi_{\varphi}$ denotes the set of vector representations over $\varphi$ of normalized constraints in $\Phi$\\

Recall that the model space of any set of constraints $\Phi$ is the set of all valuations satisfying all constraints in $\Phi$. Thus, to show that $\mathcal{M}_\varphi(\Phi)$ is empty, we must show that no valuation $\rho$ exists satisfying all constraints in $\Phi$. This is a linear constraint satisfaction problem (CSP) with an infinite domain. One method for solving such is by optimization using the simplex algorithm. If the linear program of the CSP has a feasible solution, the model space is non-empty and if it does not have a feasible solution, the model space is empty.\\

As is usual for linear constraints, our linear constraints can be thought of as hyper-planes dividing some n-dimensional space in two, with one side constituting the feasible region and the other side the non-feasible region. By extension, for a set of constraints their shared feasible region is the intersection of all of their individual feasible regions. Since the feasible region of a set of constraints is defined by a set of hyper-planes, the feasible region consists of a convex polytope. This fact is used by the simplex algorithm when performing optimization.\\

The simplex algorithm has some requirements to the form of the linear program it is presented, i.e. that it must be in \textit{standard} form. The standard form is a linear program expressed as 
\begin{align*}
    \text{minimize}&\quad \mathbf{c}^T\mathbf{a}\\
    \text{subject to}&\quad M\mathbf{a} = \mathbf{b}\\
    &\quad\mathbf{a} \geq \mathbf{0}
\end{align*}
where $M$ is a matrix representing constraints, $\mathbf{a}$ is a vector of scalars, and $\mathbf{b}$ is a vector of constants. As such, we first need all our constraints to be of the form $a_0 \cdot i_0 + ... + a_n \cdot i_n \leq b$, after which we must convert them into equality constraints by introducing \textit{slack} variables that allow the equality to also take on lower values. Since all of our constraints are normalized and of the form $I \leq 0$, all of our slack variables will have negative coefficients. In our specific case, we let row $i$ of $M$ consist of $(\mathbf{C}^i_\varphi)_{-1}$, where $(\cdot)_{-1}$ removes the first element of the vector (the constant term here). We must also include our slack variables, and so we augment row $i$ of $M$ with the n-vector with all zeroes except at position $i$ where it is $-1$. We let $\mathbf{a}$ be a column vector containing our variables in $\varphi$ as well as our slack variables, and finally we let $\mathbf{b}_i = -(\mathbf{C}^i_\varphi)_1$. $\mathbf{c}$ may be an arbitrary vector.\\

Checking feasibility of the above linear program can itself be formulated as a linear program that is guaranteed to be feasible, enabling us to use efficient polynomial time linear programming algorithms, such as interior point methods, to check whether constraints are covered. Let $\mathbf{s}$ be a new vector, then we have the linear program
%
\begin{align*}
    \text{minimize}&\quad \mathbf{1}^T\mathbf{s}\\
    \text{subject to}&\quad M\mathbf{a} + \mathbf{s} = \mathbf{b}\\
    &\quad\mathbf{a},\mathbf{s} \geq \mathbf{0}
\end{align*}
where $\mathbf{1}$ is the vector of all ones. We can verify the feasibility of this problem with the certificate $(\mathbf{a},\mathbf{s})=(\mathbf{0},\mathbf{b})$. Then the original linear program is feasible if and only if the augmented problem has an optimal solution $(\mathbf{x}^*,\mathbf{s}^*)$ such that $\mathbf{s}^* = \mathbf{0}$.\\

Given a constraint judgement $\varphi;\Phi \vDash C_0$, it should be noted that while the simplex algorithm can be used to check if a solution exists to the constraints $\Phi \cup \{C_0'\}$, there is no guarantee that the solution is an integer solution nor that an integer solution exists at all. Thus, in the case that a non-integer solution exists but no integer solution, this method will over-approximate. An example of such is the two constraints $3i - 1 \leq 0$ and $-2i + 1 \leq 0$ yielding the feasible region where $\frac{1}{3} \leq i \leq \frac{1}{2}$, containing no integers. For an exact solution, we may use integer programming.

% \subsubsection{Conical combinations of constraints}
% We now show how constraints can be conically combined. For convenience, given a finite ordered set of index variables $\varphi = \{i_1, i_2, \dots, i_n\}$, we represent a normalized constraint $I \leq 0$ as a vector $\left( \mathcal{B}(I), I_1\; I_2\; \cdots\; I_n \right)_{\varphi}$. As such, the constraint $-5i + -2j + -4k \leq 0$ can be represented by the vector $\cvect[\varphi_1]{0 {-5} {-2} {-4}}$ where $\varphi_1=\left\{i, j, k\right\}$. Another way to represent that same constraint is with the vector $\cvect[\varphi_2]{0 {-5} {-2} 0 {-4}}$ where $\varphi_2 = \left\{i,j,l,k\right\}$. We denote the vector representation of a constraint $C$ over a finite ordered set of index variables $\varphi$ by $\mathbf{C}_{\varphi}$. We extend this notation to sets of constraints, such that $\Phi_{\varphi}$ denotes the set of vector representations over $\varphi$ of normalized constraints in $\Phi$. Then for a finite ordered set of exponent vectors $\varphi$ and a set of normalized constraints $\Phi$, we can infer any constraint $C$ represented by a vector $\mathbf{C}_\varphi\in \text{coni}(\Phi_\varphi)$ where $\text{coni}(\Phi_\varphi)$ is the \textit{conical hull} of $\Phi_\varphi$. That is, $\text{coni}(\Phi_\varphi)$ is the set of conical combinations with non-negative integer coefficients of vectors in $\Phi_\varphi$
% %
% \begin{align*}
%   \text{coni}(\Phi_\varphi) = \left\{\sum^k_{i=1} a_i {\mathbf{C}^i_\varphi} : {\mathbf{C}^i_\varphi} \in \Phi_\varphi,\; a_i,k \in \mathbb{N}\right\}  
% \end{align*}
% %
% Then, to check if a constraint $C^{new}$ is covered by the set of normalized constraints $\Phi = \{C_1,C_2,\dots, C_n\}$, we can test if $\mathbf{C}^{new}_\varphi$ is a member of the conical hull $\text{coni}(\Phi_\varphi)$. However, by itself, this does not take into account subconstraints of constraints in $\Phi$, as these may not necessarily be written as conical combinations of $\Phi_\varphi$. To account for these, we can include $m=|\varphi|$ vectors of size $m$ of the form $\cvect{-1 0 $\cdots$ 0}, \cvect{0 {-1} 0 $\cdots$ 0}, \dots, \cvect{0 $\cdots$ 0 {-1})}$ in $\Phi_\varphi$. As the conical hull $\text{coni}(\Phi_\varphi)$ is infinite when there exists $\mathbf{C}_\varphi \in \Phi_\varphi$ such that $\mathbf{C}_\varphi \neq \mathbf{0}$ where $\mathbf{0}$ is the vector of all zeroes, when checking for the existence of a conical combination of vectors in $\Phi_\varphi$ equal to $\mathbf{C}^\textit{new}_\varphi$, we can instead solve the following system of linear equations
% %
% \begin{align*}
%     a_1 {\mathbf{C}^1_\varphi}_1 + a_2 {\mathbf{C}^2_\varphi}_1 + \cdots + a_n {\mathbf{C}^n_\varphi}_1 =&\; {\mathbf{C}^{new}_\varphi}_1\\
%     a_1 {\mathbf{C}^1_\varphi}_2 + a_2 {\mathbf{C}^2_\varphi}_2 + \cdots + a_n {\mathbf{C}^n_\varphi}_2 =&\; {\mathbf{C}^{new}_\varphi}_2\\
%     &\!\!\!\vdots\\
%     a_1 {\mathbf{C}^1_\varphi}_m + a_2 {\mathbf{C}^2_\varphi}_m + \cdots + a_n {\mathbf{C}^n_\varphi}_m =&\; {\mathbf{C}^{new}_\varphi}_m
% \end{align*}
% %
% where $a_1,a_2,\dots,a_m\in\mathbb{Z}_{\geq 0}$ are non-negative integer numbers. However, this is an integer programming problem, and so it is NP-hard. We can relax the requirement for $a_1,a_2,\dots,a_m$ to be integers, as the equality relation is preserved under multiplication by any positive real number. We can then view the above system as a linear program, with additional constraints $a_i \geq 0$ for $1 \geq i \geq n$. That is, let $M = \vect{$\mathbf{C}^1_\varphi$ $\mathbf{C}^2_\varphi$ $\cdots$ $\mathbf{C}^n_\varphi$}$ be a matrix with column vectors representing constraints and $\mathbf{a} = \vect{$a_1$ $a_2$ $\cdots$ $a_n$}$ be a row vector of scalars, then checking whether $\mathbf{C}^{new}_\varphi\in\text{coni}(\Phi_\varphi)$ amounts to determining if the following linear program is feasible
% %
% \begin{align*}
%     \text{minimize}&\quad \mathbf{c}^T\mathbf{a}\\
%     \text{subject to}&\quad M\mathbf{a} = \mathbf{C}^{new}_\varphi\\
%     &\quad\mathbf{a} \geq \mathbf{0}
% \end{align*}
% %
% where $\mathbf{c}$ is an arbitrary vector of length $n$ and $\mathbf{0}$ is the vector of all zeroes of length $n$. Checking feasibility of the above linear program can itself be formulated as a linear program that is guaranteed to be feasible, enabling us to use efficient polynomial time linear programming algorithms, such as interior point methods, to check whether constraints are covered. Let $\mathbf{s}$ be a new vector of length $m$, then we have the linear program
% %
% \begin{align*}
%     \text{minimize}&\quad \mathbf{1}^T\mathbf{s}\\
%     \text{subject to}&\quad M\mathbf{a} + \mathbf{s} = \mathbf{C}^{new}_\varphi\\
%     &\quad\mathbf{a},\mathbf{s} \geq \mathbf{0}
% \end{align*}
% where $\mathbf{1}$ is the vector of all ones of length $m$. We can verify the feasibility of this problem with the certificate $(\mathbf{a},\mathbf{s})=(\mathbf{0},\mathbf{C}^{new}_\varphi)$. Then the original linear program is feasible if and only if the augmented problem has an optimal solution $(\mathbf{x}^*,\mathbf{s}^*)$ such that $\mathbf{s}^* = \mathbf{0}$.
% %

\begin{examp}
    Given the constraints
    \begin{align*}
        C^1 &= 3i - 3 \leq 0\\
        C^2 &= j + 2k - 2 \leq 0\\
        C^3 &= -k \leq 0\\
        C^{new} &= i + j - 3 \leq 0
    \end{align*}
    
    we want to check if the constraint judgement $\{i, j, k\};\{C^1, C^2, C^3\} \vDash C^{new}$ is satisfied\\
    
    
    We first let $C^{newinv}$ be the inversion of constraint $C^{new}$.
    \begin{align*}
        C^{newinv} &= 1i + 1j - 2 \geq 0
    \end{align*}
    
    We now want to check if the feasible region $\mathcal{M}_\varphi(\{C^1, C^2, C^3, C^{newinv}\})$ is nonempty. To do so, we construct a linear program with the four constraints. To convert all inequality constraints into equality constraints, we add the slack variables $s_1, s_2, s_3, s_4$ 
    
    \begin{align*}
        \text{minimize}&\quad i + j + k\\
        \text{subject to}&\quad 3i + 0j + 0k + s_1 = 3\\
        &\quad 0i + 1j + 2k + s_2 = 2\\
        &\quad 0i + 0j - 1k + s_3 = 0\\
        &\quad 1i + 1j + 0k - s_4 = 2\\
        &\quad i, j, k, s_1, s_2, s_3, s_4 \geq 0
    \end{align*}
    
    Using an algorithm such as the simplex algorithm, we see that there is no feasible solution, and so we conclude that the constraint judgement $\{i, j, k\};\{C^1, C^2, C^3\} \vDash C^{new}$ is satisfied.
    
    
    %%%%%%%%%%%%%%%%%%%%%%%
    
    % We first represent the four constraints as vectors in terms of some ordered set $\varphi$ of index variables and some ordered set $\varphi$ of exponent vectors.\\
    
    % Let $\varphi = \{i, j, k\}$ and $\varphi = \{\evect{1 0 0}, \evect{0 1 0}, \evect{0 0 1}, \evect{0 0 0}\}$. The constraints $C^1, C^2, C^3, C^{new}$ can now be written as the following vectors
    % %
    % \begin{align*}
    %     \mathbf{C}^1_\varphi &= \cvect{1 0 0 -3}\\
    %     \mathbf{C}^2_\varphi &= \cvect{0 1 1 -2}\\
    %     \mathbf{C}^3_\varphi &= \cvect{0 0 -1 0}\\
    %     \mathbf{C}^{new}_\varphi &= \cvect{2 3 2 -15}
    % \end{align*}
    
    % With the constraints now represented as vectors, we can prepare the equation $M\mathbf{a} = \mathbf{b}$ representing the conical combination, for which we wish to check if a solution exists given given the requirement that $\mathbf{a} \geq \mathbf{0}$. We first prepare the matrix $M$, where we represent the constraint vectors as column vectors
    % %
    % \begin{align*}
    %     &M = \vect{$\mathbf{C}^1_\varphi$ $\mathbf{C}^2_\varphi$ $\mathbf{C}^3_\varphi$ $\bm{\beta}_1$ $\bm{\beta}_2$ $\bm{\beta}_3$ $\bm{\beta}_4$}\\
    %     &\quad \text{where } \bm{\beta}_1 = \cvect{-1 0 0 0}, \bm{\beta}_2 = \cvect{0 {-1} 0 0}, \bm{\beta}_3 = \cvect{0 0 {-1} 0}, \bm{\beta}_4 = \cvect{0 0 0 {-1}}
    % \end{align*}
    % %
    % We include vectors $\bm{\beta}_i, i \in \{1, 2, 3, 4\}$ to ensure we can also use subconstraints of $\mathbf{C}^i, i \in \{1, 2, 3\}$ when checking if we can construct $\mathbf{C}^{new}_\varphi$. To check if a solution exists to the aforementioned equation, we solve the following linear program to check if $\mathbf{s} = \mathbf{0}$
    % \begin{align*}
    %     \text{minimize}&\quad \mathbf{1}^T\mathbf{s}\\
    %     \text{subject to}&\quad M\mathbf{a} + \mathbf{s} = \mathbf{C}^{new}_\varphi\\
    %     &\quad\mathbf{a},\mathbf{s} \geq \mathbf{0}
    % \end{align*}
    
    % This is possible given $\mathbf{a} = \vect{2 3 1 0 0 0 3}$, and so a solution exists to the canonical combination. Notice that we had to use the additional $\bm{\beta}$ vectors when constructing the conical combination. This shows the importance of subconstraints when checking type judgements.
\end{examp}
%
% \section{Soundness}
% %


% \begin{theorem}[Subject reduction]\label{theorem:srbg}
% If $\varphi;\Phi;\Gamma\vdash P \triangleleft K$ and $P \leadsto Q$ then $\varphi;\Phi;\Gamma\vdash Q \triangleleft K'$ with $\varphi;\Phi\vDash k' \leq K$.
% \begin{proof} by induction on the rules defining $\leadsto$.
%     \begin{description}
%     \item[$\runa{R-rep}$]
%     %
%     \item[$\runa{R-comm}$]
%     %
%     \item[$\runa{R-zero}$]
%     %
%     \item[$\runa{R-par}$]
%     %
%     \item[$\runa{R-succ}$]
%     %
%     \item[$\runa{R-empty}$]
%     %
%     \item[$\runa{R-res}$]
%     %
%     \item[$\runa{R-cons}$]
%     %
%     \item[$\runa{R-struct}$]
%     %
%     %\item[$\runa{R-tick}$] We have that $P = \tick{P'}$ and $Q=P'$. Then by $\runa{S-tick}$ we have $\varphi;\Phi;\Gamma\vdash $
%     \end{description}
% \end{proof}
% \end{theorem}

% \begin{lemma}\label{lemma:timeredtype}
% If $\varphi;\Phi;\Gamma\vdash P \triangleleft K$ with $P\!\not\!\leadsto$ and $P \Longrightarrow^{-1} Q$ then $\varphi;\Phi;\downarrow_1\!\Gamma\vdash Q \triangleleft K'$ with $\varphi;\Phi\vDash K' \leq K + 1$.
% \begin{proof}
    
% \end{proof}
% \end{lemma}

% \begin{theorem}\label{theorem:ubbg}
% If $\varphi;\Phi;\Gamma\vdash P \triangleleft K$ and $P \hookrightarrow^n Q$ then $\varphi;\Phi\vDash n \leq K$.
% \begin{proof} by induction on the number of time reductions $n$ in the sequence $P \hookrightarrow^n Q$.
    
% \end{proof}
% \end{theorem}


% % \begin{description}
% %     \item[$\runa{S-nil}$]
% %     %
% %     \item[$\runa{S-tick}$]
% %     %
% %     \item[$\runa{S-nu}$]
% %     %
% %     \item[$\runa{S-nmatch}$]
% %     %
% %     \item[$\runa{S-lmatch}$]
% %     %
% %     \item[$\runa{S-par}$]
% %     %
% %     \item[$\runa{S-iserv}$]
% %     %
% %     \item[$\runa{S-ich}$]
% %     %
% %     \item[$\runa{S-och}$]
% %     %
% %     \item[$\runa{S-oserv}$]
% %     \end{description}


% %
% It is worth noting that the Simplex algorithm does not guarantee an integer solution, and so we may get indices in constraints where the coefficients may be non-integer values. However, we can use the fact that any feasible linear programming problem with rational coefficients also has an (optimal) solution with rational values \cite{keller2016applied}. We use this fact and Lemma \ref{lemma:constraintcommonden} and \ref{lemma:constraintscaling} to show that we need not to worry about the solution given by the Simplex algorithm, given a rational linear programming problem. Definition \ref{def:constraintequivalence} defines what it means for constraints to be equivalent.

% \begin{defi}[Conditional constraint equivalence]\label{def:constraintequivalence}
%     Let $C_1$, $C_2$ and $C\in\Phi$ be linear constraints with integer coefficients and unknowns in $\varphi$. We say that $C_1$ and $C_2$ are equivalent with respect to $\varphi$ and $\Phi$, denoted $C_1 =_{\varphi;\Phi} C_2$, if we have that
%     \begin{equation*}
%     \mathcal{M}_\varphi(\{C_1\} \cup \Phi) = \mathcal{M}_\varphi(\{C_2\} \cup \Phi) %\mathcal{M}_\varphi(\{C_0\})
% \end{equation*}
% where $\mathcal{M}_{\varphi'}(\Phi')=\{\rho : \varphi' \rightarrow \mathbb{N} \mid \rho \vDash C\;\text{for}\; C \in \Phi'\}$ is the model space of a set of constraints $\Phi'$ over a set of index variables $\varphi'$.
%     %
%     %
%     %$\varphi;\Phi\vDash C_1$ if and only if $\varphi;\Phi\vDash C_2$.
%     %Two normalized constraints $C_1$ and $C_2$ are said to be \textit{equivalent} if for any index valuation $\rho$, we have that $\rho \vDash C_1$ if and only if $\rho \vDash C_2$.
% \end{defi}

% \begin{lemma}\label{lemma:constraintscaling}
% Let $I \leq 0$ be a linear constraint with unknowns in $\varphi$. Then $I \leq 0 =_{\varphi;\Phi} n I \leq 0$ for any $n>0$ and set of constraints $\Phi$.
% \begin{proof}
%     This follows from the fact that if $I \leq 0$ is satisfied, then the sign of $I$ must be non-positive, and so the sign of $n I$ must also be non-positive as $n > 0$. Conversely, if $I \leq 0$ is not satisfied, then the sign of $I$, must be positive and so the sign of $n I$ must also be positive.
% \end{proof}
% \end{lemma}

% \begin{lemma}\label{lemma:constraintcommonden}
% Let $I \leq 0$ be a normalized linear constraint with rational coefficients and unknowns in $\varphi$. Then there exists a normalized linear constraint $I' \leq 0$ with integer coefficients and unknowns in $\varphi$ such that $I \leq 0 =_{\varphi;\Phi} I' \leq 0$ for any set of constraints $\Phi$.% there exists an equivalent constraint $I' = \normlinearindex{n'}{I'}$ where $n', I'_{\alpha_1}, \dots,I'_{\alpha_{m}}$ are integers.
% \begin{proof}
%     It is well known that any set of rationals has a common denominator, whose multiplication with any rational in the set yields an integer. One is found by multiplying the denominators of all rationals in the set. As the coefficients of $I$ are non-negative, this common denominator must be positive. By Lemma \ref{lemma:constraintscaling}, we have that $I\leq 0 =_{\varphi;\Phi} n I \leq 0$ where $n$ is a positive number and $\Phi$ is any set of constraints.% the constraint $I \leq 0$ is equivalent to $d I \leq 0$.
% \end{proof}
% \end{lemma}

% %%
% % \begin{lemma}
% % Let $I \leq J$ and $C\in\Phi$ be a linear constraints with integer coefficients and unknowns in $\varphi$. Then $I \leq J =_{\varphi;\Phi} \mathcal{N}(I\leq J)$ if for any subtraction $K - L$ in $I$ or $J$, we have $\varphi;\Phi\vDash L \leq K$. 
% % \begin{proof}
    
% % \end{proof}
% % \end{lemma}
% % %

% % % \begin{lemma}
% % % Let $C$ and $C'\in\Phi$ be normalized linear constraints with integer coefficients and unknowns in $\varphi$. Then $\varphi;\Phi\nvDash C$ if there does not exist $\mathbf{C}^{new}_\varphi\in\text{coni}(\Phi_\varphi \cup \{\mathbf{0}\})$ with $\mathbf{C}_\varphi\leq \mathbf{C}^{new}_\varphi$.
% % % \begin{proof}
    
% % % \end{proof}
% % % \end{lemma}


% % %
% % \begin{theorem}
% % Let $C$ and $C'\in\Phi$ be normalized linear constraints with integer coefficients and unknowns in $\varphi$. Then $\varphi;\Phi\vDash_{\mathbb{R}^{\geq 0}} C$ if and only if there exists $\textbf{C}^{new}_\varphi\in\text{coni}(\Phi_\varphi \cup \{\mathbf{0}\})$ with $\textbf{C}_\varphi\leq \textbf{C}^{new}_\varphi$.
% % \begin{proof}
% %     We consider the implications separately
% %     \begin{enumerate}
% %         \item Assume that $\varphi;\Phi\vDash_{\mathbb{R}^{\geq 0}} C$. Then for all valuations $\rho : \varphi \longrightarrow \mathbb{R}^{\geq 0}$ such that $\rho\vDash \Phi$ we also have $\rho\vDash C$, or equivalently $([\![I_1]\!]_\rho \leq 0) \land \cdots \land ([\![I_n]\!]_\rho \leq 0) \implies [\![I]\!]_\rho \leq 0$, where $C = I_0 \leq 0$ and $C_i = I_i \leq 0$ for $C_i\in \Phi$. We show by contradiction that this implies there exists $\textbf{C}^{new}_\varphi\in\text{coni}(\Phi_\varphi \cup \{\mathbf{0}\})$ with $\textbf{C}_\varphi\leq \textbf{C}^{new}_\varphi$. Assume that such a conical combination does not exist. Then for all $\mathbf{C}'_\varphi\in\text{coni}(\Phi_\varphi \cup \{\mathbf{0}\})$ there is at least one coefficient ${\mathbf{C}'_\varphi}_k$ for some $0\leq k \leq |\varphi|$ such that ${\mathbf{C}'_\varphi}_k < {\mathbf{C}_\varphi}_k$. We show that this implies there exists $\rho\in\mathcal{M}_\varphi(\Phi)$ such that $\rho\nvDash C$.\\ 
        
        
% %         and so there must exist $\rho : \varphi \longrightarrow \mathbb{R}^{\geq 0}$ such that $\rho\vDash C'$ and $\rho\nvDash C$. However, as $\varphi;\Phi\vDash_{\mathbb{R}^{\geq 0}} C$ holds there must be some constraint $C''\in\Phi$ such that $\rho\nvDash C''$.\\
        
        
% %         Assume that there does not exist a conical combination $\textbf{C}^{new}_\varphi\in\text{coni}(\Phi_\varphi \cup \{\mathbf{0}\})$ with $\textbf{C}_\varphi\leq \textbf{C}^{new}_\varphi$. \\
        
        
        
% %         We have that $C = n + \sum_{\alpha\in\mathcal{E}(I)} I_\alpha i_\alpha$, and so for $\rho\vDash n + \sum_{\alpha\in\mathcal{E}(I)} I_\alpha i_\alpha$ to hold, it must be that $[\![\sum_{\alpha\in\mathcal{E}(I)} I_\alpha i_\alpha]\!]_\rho \leq -n$. This implies that $\Phi$ contains constraints that collectively bound the sizes of index variables that appear in $\sum_{\alpha\in\mathcal{E}(I)} I_\alpha i_\alpha$, such that $\sum_{\alpha\in\mathcal{E}(I)} I_\alpha i_\alpha$ cannot exceed $-n$. We now show by contradiction that there must then exist $\textbf{C}^{new}_\varphi\in\text{coni}(\Phi_\varphi \cup \{\mathbf{0}\})$ such that $\textbf{C}_\varphi\leq \textbf{C}^{new}_\varphi$. Assume that such a conical combination does not exist. Then it must be that for any $C'_\varphi\in\text{coni}(\Phi \cup \{\mathbf{0}\})$, at least one coefficient in $C'_\varphi$ is smaller than the corresponding coefficient in $C_\varphi$, implying that $C$ imposes a new restriction on valuations. Thus, there must exist a valuation $\rho$ such that $\rho\vDash\Phi$ but $\rho\nvDash C$, but then we have that $\varphi;\Phi\nvDash C$, and so we have a contradiction.
        
% %         \item Assume that there exists $\mathbf{C}^{new}_\varphi\in\text{coni}(\Phi_\varphi \cup \{\mathbf{0}\})$ with $\mathbf{C}_\varphi \leq \mathbf{C}^{new}_\varphi$. Then by Lemma \ref{TODO}, we have that $\varphi;\phi\vDash C^{new}$ and by Lemma \ref{TODO} it follows from $\mathbf{C}_\varphi \leq \mathbf{C}^{new}_\varphi$ that also $\varphi;\Phi\vDash C$.
% %     \end{enumerate}
% % \end{proof}
% % \end{theorem}
\subsection{Reducing polynomial constraints to linear constraints}\label{sec:verifyingpolynomial}

Many programs do not run in linear time, and so we cannot type them if we are constrained to just verifying linear constraint judgements. In this section we show how we can reduce certain polynomial constraints to linear constraints, enabling us to use the techniques described above. We first extend our definition of indices such that they can be used to express multivariate polynomials. We assume a normal form for polynomial indices akin to that of linear indices. Terms are now monomials with integer coefficients.
%
\begin{align*}
        I,J ::= n \mid i \mid I + J \mid I - J \mid I J
\end{align*}
%
When reducing normalized constraints with polynomial indices to normalized constraints with linear indices, we wish to construct new linear constraints that are only satisfied if the original polynomial constraint is satisfied. For example, given the constraint ${-i^2 + 10 \leq 0}$, one can see that this polynomial constraint can be simulated using the constraint $-i + \sqrt{10} \leq 0$. We notice that the reason this is possible is that when ${-i^2 + 10 \leq 0}$ holds, i.e. when $i \geq \sqrt{10}$, the value of $i$ can always be increased without violating the constraint. Similarly, when ${-i^2 + 10 \geq 0}$ holds, the value of $i$ can always be decreased until reaching its minimum value of 0 without violating the constraint. We can thus introduce a new simpler constraint with the same properties, i.e. $-i + \sqrt{10} \leq 0$. More specifically, the polynomials of the left-hand side of the two constraints share the same positive real-valued roots as well as the same sign for any value of $i$. In general, limiting ourselves to univariate polynomials, for any constraint whose left-hand side polynomial only has a single positive root, we can simulate such a constraint using a constraint of the form $a \cdot i + c \leq 0$. For describing complexities of programs, we expect to mostly encounter monotonic polynomials with at most a single positive real-valued root.\\

Note that the above has the consequence that we may end up with irrational coefficients for indices of constraints. While such constraints are not usually allowed, they can still represent valid bounds for index variables. As such, we can allow them as constraints in this context. Furthermore, any irrational coefficient can be approximated to an arbitrary precision using a rational coefficient.\\

For finding the roots of a specific polynomial we can use either analytical or numerical methods. Using analytical methods has the advantage of being able to determine all roots with exact values, however, we are limited to polynomials of degree at most four as stated by the Abel-Ruffini theorem \cite{abelruffinitheorem}. With numerical methods, we are not limited to polynomials of a specific degree, however, numerical methods often require a given interval to search for a root and do not guarantee to find all roots. Introducing constraints with false restrictions may lead to an under-approximating type system, and so we must be careful not to introduce such. We must therefore ensure we find all roots to avoid constraints with false restrictions. We can use Descartes rule of signs to get an upper bound on the number of positive real roots of a polynomial. Descarte's rule of signs states that the number of roots in a polynomial is at most the number of sign-changes in its sequence of coefficients.\\

For our application, we decide to only consider constraints whose left-hand side polynomial is univariate and monotonic with a single root. In some cases we may remove safely positive monomials in a normalized constraint to obtain such constraints. We limit ourselves to these constraints both to keep complexity down, as well as because we expect to mainly encounter such polynomials when considering complexity analysis of programs. We use Laguerre's method as a numerical method to find the root of the polynomial, which has the advantage that it does not require any specified interval when performing root-finding. Assuming we can find a root $r$, we add an additional constraint $\pm (i - r) \leq 0$ where the sign depends on whether the original polynomial is increasing or decreasing.\\

Additionally, given a non-linear monomial, we may also treat this as a single unit and construct linear combinations of this by treating the monomial as a single fresh variable. For example, given normalized constraints $C_1 = i^2 + 4 \leq 0$, $C_2 = i - 2 \leq 0$, and $C_3 = 2ij \leq 0$, we may view these as the linear constraints $C_1' = k + 4 \leq 0$, $C_2' = i - 2 \leq 0$, and $C_3' = 2l \leq 0$ where $k = i^2$ and $l = ij$. This may make the feasible region larger than it actually is, meaning we over-approximate when verifying judgements on polynomial constraints.

\begin{examp}
    We want to check if the following judgement holds
    $$\{i, j\}; \{-2i \leq 0, -1i^2 + 1j + 1 \leq 0\} \vDash -2i + 2 \leq 0$$
    
    %We notice that $-2i + 2$ cannot be written as a conical combination of the polynomials $-2i$ and $-1i^2 + 1j + 1$.\\
    
    We first try to generate new constraints of the form $-a \cdot i + r \leq 0$ for some index variable $i$ and some constants $a$ and $r$ based on our two existing constraints using the root-finding method. The first constraint is already of such form, so we can only consider the second. For the second, we first use the subconstraint relation to remove the term $1j$ obtaining $-1i^2 + 1 \leq 0$. Next, we note that $-1i^2 + 1$ is a monotonically decreasing polynomial as every coefficient excluding the constant term is negative. We then find the root $r = 1$ of the polynomial and add a new constraint $-1i + 1 \leq 0$ to our set of constraints. Finally, we can invert the constraint $-2i + 2 \leq 0$ obtaining the constraint $-2i + 1 \geq 0$. We now need to check if the feasible region $\mathcal{M}_{\{i, j, i^2\}}(\{-2i \leq 0, -1i^2 + 1j + 1 \leq 0, -1i + 1 \leq 0, -2i + 1 \geq 0\})$ is empty. Treating the monomial $i^2$ as its own separate variable and solving this as a linear program using an algorithm such as the simplex algorithm, we see that there is indeed no solution, and so the constraint is satisfied. 
\end{examp}

\subsection{Trivial judgements}
We now show how some judgements may be verified without neither transforming constraints into linear constraints nor solving any integer programs. To do so, we consider an example provided by Baillot and Ghyselen \cite{BaillotGhyselen2021}, where we exploit the fact that all coefficients in the normalized constraints are non-positive. Judgements with such constraints can be answered in linear time with respect to the number of monomials in the normalized equivalent of constraint $C$. That is if all coefficients in the normalized constraint are non-positive, we can guarantee that the constraint is always satisfied, recalling that only naturals substitute for index variables. Similarly, if there are no negative coefficients and at least one positive coefficient, we can guarantee that the constraint is never satisfied.\\

In practice, it turns out that we can type check many processes by simply over-approximating constraint judgements using pair-wise coefficient inequality constraints. In Example \ref{example:baillotghyssimple}, we show how all constraint judgements in the typings of both a linear and a polynomial time replicated input can be verified using this approach. 
%
\begin{examp}\label{example:baillotghyssimple}
Baillot and Ghyselen \cite{BaillotGhyselen2021} provide an example of how their type system for parallel complexity of message-passing processes can be used to bound the time complexity of a linear, a polynomial and an exponential time replicated input process. We show that we can verify all judgements on constraints in the typings of the first two processes using normalized constraints. We first define the processes $P_1$ and $P_2$
\begin{align*}
    P_i \defeq\; !\inputch{a}{n,r}{}{\tick\match{n}{\asyncoutputch{r}{}{}}{m}{\newvar{r'}{\newvar{r''}{Q_i}}}}
\end{align*}
for the corresponding definitions of $Q_1$ and $Q_2$
\begin{align*}
    Q_1 \defeq&\; \asyncoutputch{a}{m,r'}{} \mid \asyncoutputch{a}{m,r''}{} \mid \inputch{r'}{}{}{\inputch{r''}{}{}{\asyncoutputch{r}{}{}}}\\
    Q_2 \defeq&\; \asyncoutputch{a}{m,r'}{} \mid \inputch{r'}{}{}{(\asyncoutputch{a}{m,r''}{} \mid \asyncoutputch{r}{}{})} \mid \asyncinputch{r''}{}{}
\end{align*}
We type $Q_1$ and $Q_2$ under the respective contexts $\Gamma_1$ and $\Gamma_2$
\begin{align*}
    \Gamma_1 \defeq&\; a : \forall_0 i.\texttt{oserv}^{i+1}(\texttt{Nat}[0,i],\texttt{ch}_{i+1}()), n : \texttt{Nat}[0,i], m : \texttt{Nat}[0,i-1],\\ &\; r : \texttt{ch}_{i}(),
     r' : \texttt{ch}_{i}(), r'' : \texttt{ch}_i()\\
    %
    \Gamma_2 \defeq&\; a : \forall_0 i.\texttt{oserv}^{i^2+3i+2}(\texttt{Nat}[0,i],\texttt{ch}_{i+1}()), n : \texttt{Nat}[0,i], m : \texttt{Nat}[0,i-1],\\ &\; r : \texttt{ch}_{i}(),
     r' : \texttt{ch}_i(), r'' : \texttt{ch}_{2i-1}()
\end{align*}
Note that in the original work, the bound on the complexity of server $a$ in context $\Gamma_2$ is $(i^2+3i+2)/2$. However, we are forced to use a less precise bound, as the multiplicative inverse is not always defined for our view of indices. Upon typing process $P_1$, we amass the judgements on the left-hand side, with corresponding judgements with normalized constraints on the right-hand side
%
\begin{align*}
    &\{i\};\emptyset\vDash i + 1 \geq 1\kern7.5em\Longleftrightarrow &  \{i\};\emptyset\vDash -i \leq 0\\
   % &\{i\};\{i \geq 1\} \vDash i-1 \leq i \kern6em\Longleftrightarrow & \{i\};\{1-i \leq 0\} \vDash -1 \leq 0\\
    &\{i\};\{i \geq 1\} \vDash i \leq i + 1\kern5em\Longleftrightarrow &  \{i\};\{1-i \leq 0\} \vDash -1 \leq 0\\
    %
    &\{i\};\{i \geq 1\} \vDash i \geq i\kern6.7em\Longleftrightarrow &  \{i\};\{1-i \leq 0\} \vDash 0 \leq 0\\
    &\{i\};\{i \geq 1\} \vDash 0 \geq 0\kern6.45em\Longleftrightarrow &  \{i\};\{1-i \leq 0\} \vDash 0 \leq 0
\end{align*}
%
As all coefficients in the normalized constraints are non-positive, each judgement is trivially satisfied, and we can verify the bound $i + 1$ on server $a$. For process $P_2$ we correspondingly have the trivially satisfied judgements
\begin{align*}
    &\{i\};\emptyset\vDash i + 1 \geq 1\kern10.3em\Longleftrightarrow &  \{i\};\emptyset\vDash -i \leq 0\\
    &\{i\};\{0 \leq 0\}\vDash i \leq i^2 + 3i + 2\kern5.2em\Longleftrightarrow &  \{i\};\{0\leq 0\}\vDash -i^2-2i-2 \leq 0\\
    %
    &\{i\};\{i \geq 1\} \vDash i \leq i + 1\kern7.9em\Longleftrightarrow &  \{i\};\{1-i \leq 0\} \vDash -1 \leq 0\\
    %
    &\{i\};\{i \geq 1\} \vDash 2i-1 \leq i^2 + 3i + 2\kern3.2em\Longleftrightarrow &  \{i\};\{1-i \leq 0\} \vDash -i^2-i-3 \leq 0\\
    &\{i\};\{i \geq 1\} \vDash i \geq i\kern9.7em\Longleftrightarrow &  \{i\};\{1-i \leq 0\} \vDash 0 \leq 0\\
    %
    &\{i\};\{i \geq 1\} \vDash 0 \geq i^2+i\kern7.5em\Longleftrightarrow &  \{i\};\{1-i \leq 0\} \vDash -i^2-i \leq 0\\
    %
    &\{i\};\{i \geq 1\} \vDash i^2+2i \geq i^2+3i+2\kern2.9em\Longleftrightarrow &  \{i\};\{1-i \leq 0\} \vDash -i-2 \leq 0
\end{align*}

\end{examp}

% In the general sense, however, verifying whether judgements on polynomial constraints are satisfied is a difficult problem, as it amounts to verifying that a constraint is satisfied under all interpretations that satisfy our set of known constraints. In Example \ref{example:needconic}, we show how a constraint that is not satisfied for all interpretations can be shown to be covered by a set of two constraints, by utilizing the transitive, multiplicative and additive properties of inequalities to combine the two constraints. More specifically, we can exploit the fact that we can generate new constraints from any set of normalized constraints by taking a \textit{conical} combination of their left-hand side indices, as we shall formalize in the following section.
% %
% \begin{examp}\label{example:needconic}
%     Given the judgement
%     \begin{align*}
%         \{i\};\{i \leq 3, 5 \leq i^2\} \vDash 5i \leq 3i^2
%     \end{align*}
%     we want to verify that constraint $5i \leq 3i^2$ is covered by the set of constraints $\{i\leq 3, 5 \leq i^2\}$. This constraint is not satisfied by all interpretations, as substituting $1$ for $i$ yields $5 \not\leq 3$. However, we can rearrange and scale the constraints $i\leq 3$ and $5\leq i^2$ as follows
%     \begin{align*}
%         i \leq 3 \iff i-3\leq 0 \implies 5i - 15 \leq 0\\
%         %
%         5\leq i^2 \iff 0 \leq i^2-5 \implies 0 \leq 3i^2-15
%     \end{align*}
%     Then it follows that
%     \begin{align*}
%         5i-15 \leq 3i^2-15 \iff 5i \leq 3i^2
%     \end{align*}
%     %More generally, we can use the transitive, multiplicative and additive properties of the inequality relation to construct new constraints from a set of known constraints, thereby verifying that some constraint does not impose new restrictions on interpretations. 
% \end{examp}
% %
% \subsection{An alternative method for verifying univariate polynomial constraints}
% In this section we restrict ourselves to constraints whose left-hand side normalized indices are monotonic univariate polynomials. These constraints have a number of convenient properties we can take advantage of to greatly simplify the process of verifying whether a constraint judgement holds. Namely, these constraints perfectly divide the index variable $i$ of the polynomial into two intervals $[-\infty,n[$ and $[n, \infty]$ where for any value of $i$ in either the first or the second interval, the constraint is satisfied and for any $i$ in the other interval, it is not. The only exception to this is for constraints where its left-hand side index is constant, in which case $n = \pm\infty$. As such, we can describe the behavior of a monotonic univariate polynomial constraint using a single value as well as a sign denoting whether the polynomial is increasing or decreasing.\\

% The point $n$ that divides the range that satisfies a constraint from the range that does not satisfy the constraint, corresponds to the root of the left-hand side of the normalized constraint. This is similar to the method described in Section \ref{sec:verifyingpolynomial}, except we now only store the range that satisfies the constraint. Here we take advantage of the fact that the polynomial is monotonic, to ensure that there is only a single root. Verifying whether a constraint $I \leq 0$ covers another constraint $J \leq 0$ then amounts to comparing the roots of $I$ and $J$. This method can be extended to non-monotonic polynomials by simply considering sequences of intervals satisfying constraints and comparing sequences of intervals satisfying a constraint.

% \begin{examp}
%     Given the three constraints
%     %
%     \begin{align*}
%         C_1&: -i^2 + 10 \leq 0\\
%         C_2&: -i + 2 \leq 0\\
%         C_3&: -5 i^3 + 80 i^2 - 427 i + 758 \leq 0
%     \end{align*}
    
%     we want to check if the judgement $\{i\};\{C_1, C_2, C_3\} \vDash -i + 4 \leq 0$ holds. We first find the non-negative roots $r_1$, $r_2$ and $r_3$ of $C_1$, $C_2$ and $C_3$. This can be done either numerically or analytically as all polynomials are of degree $\leq 4$. The roots are $r_1 = \sqrt{10} \approx 3.16$, $r_2 = 2$ and $r_3 \approx 4.59$. The root of $-i + 4$ is $4$, and so we must check if any of the roots $r_1$, $r_2$ and $r_3$ are greater than or equal $4$. In this case, $r_3 \geq 4$, and so the judgement $\{i\};\{C_1, C_2, C_3\} \vDash -i + 4 \leq 0$ holds.
% \end{examp}

%If our indices are univariate polynomials, we can express the feasible region of a constraint as a sequence of disjoint intervals. Then for a constraint $I \leq 0$ such that $I$ is in index variable $i$, the interpretation $I\{n/i\}$ with $n\in\mathbb{N}$ is satisfied when $n$ is within one of the intervals representing the feasible region of $I\leq 0$. We can utilize this to determine whether the feasible region of one constraint contains the feasible region of another by computing their intersection. This can be generalized to judgements on constraints, such that we can verify whether one constraint imposes new restrictions on possible interpretations. Then the question remains \textit{how do we find a sequence of disjoint intervals that corresponds to the feasible region of constraint $I \leq 0$?}\\ 

%We can find such a sequence of intervals for a normalized constraint, by computing the roots of the corresponding index. For polynomials of degree $4$ or less, there exists exact analytical methods to compute the roots, and we can approximate them in the general case using numerical methods.

% \begin{lstlisting}[escapeinside={(*}{*)}]
% intersectIntervals(is1, is2):
%     if is1 or is2 is empty:
%         return empty list
        
%     i1 (*$\longleftarrow$*) head(is1)
%     i2 (*$\longleftarrow$*) head(is2)
%     ires (*$\longleftarrow$*) i1 (*$\cap$*) i2
    
%     if max(i1) > max(i2):
%         is2' (*$\longleftarrow$*) tail(is2)
%         intersectedIntervals (*$\longleftarrow$*) intersectIntervals(is1, is2')
%     else:
%         is1' (*$\longleftarrow$*) tail(is1)
%         intersectedIntervals (*$\longleftarrow$*) intersectIntervals(is1', is2)
    
%     if ires is not the empty interval:
%         add ires to intersectedIntervals as the head
    
%     return intersectedIntervals
% \end{lstlisting}


% \begin{lstlisting}[escapeinside={(*}{*)}]
% containsIntervals(is1, is2):
%     if is2 is empty:
%         return true
%     else if is1 is empty:
%         return false
    
%     i1 (*$\longleftarrow$*) head(is1)
%     i2 (*$\longleftarrow$*) head(is2)
%     ires (*$\longleftarrow$*) i1 (*$\cap$*) i2
    
%     if ires = i2:
%         is2' (*$\longleftarrow$*) tail(is2)
%         return containsIntervals(is1, is2')
%     else:
%         is1' (*$\longleftarrow$*) tail(is1)
%         return containsIntervals(is1', is2)
% \end{lstlisting}


% \begin{lstlisting}[escapeinside={(*}{*)}]
% findIntervals((*$\{i\}$*), (*$I \leq 0$*)):
%     roots (*$\longleftarrow$*) sorted list of all positive roots of (*$I$*)
    
%     if roots is empty:
%         if (*$I\{0/i\}$*) (*$\leq$*) 0:
%         return singleton list of [0, (*$\infty$*)]
%     else:
%         return empty list
    
%     low (*$\longleftarrow$*) 0
%     satisfiedIntervals (*$\longleftarrow$*) empty list

%     if head(roots) = 0:
%         roots (*$\longleftarrow$*) tail(roots)

%     while roots is not empty:
%         high (*$\longleftarrow$*) head(roots)
%         mid (*$\longleftarrow$*) (*$(\texttt{low} + \texttt{high})/2$*)
        
%         if (*$I\{\texttt{mid}/i\}$*) (*$\leq$*) 0:
%             append [low,high] to satisfiedIntervals
            
%         low (*$\longleftarrow$*) high
%         roots (*$\longleftarrow$*) tail(roots)
    
%     if (*$I\{(\texttt{low}+1)/i\}$*) (*$\leq$*) 0:
%         append [low, (*$\infty$*)] to satisfiedIntervals
    
%     return satisfiedIntervals
% \end{lstlisting}


% \begin{lstlisting}[escapeinside={(*}{*)}]
% checkJudgement((*$\{i\}$*), (*$\Phi$*), (*$I\leq 0$*)):
%     satisfiedIntervals (*$\longleftarrow$*) singleton list of [0, (*$\infty$*)]
    
%     for (*$(J \leq 0) \in \Phi$*):
%         isJ (*$\longleftarrow$*) findIntervals((*$\{i\}$*), (*$J\leq 0$*))
%         satisfiedIntervals (*$\longleftarrow$*) intersectIntervals(satisfiedIntervals, isJ)
        
%     isI (*$\longleftarrow$*) findIntervals((*$\{i\}$*), (*$I \leq 0$*))
        
%     return containsInterval(isI, satisfiedIntervals)
% \end{lstlisting}
\section{Examples of invalid configurations}
The following examples are written in the format $\conf{E, a}$, where $E$ is an editor expression and $a$ is the AST on which we apply the editor expression. \\

In equation \ref{condsubproblem} we show how conditioned substitution can cause problems.
\begin{equation}
    \conf{\left(@\texttt{break} \Rightarrow \replace{\texttt{break}}\right) \ggg \texttt{child}\; 1,\; \lambda x.\hole\; \cursor{\breakpoint{c}}} \label{condsubproblem}
\end{equation}
 In the example we check if the cursor is at a breakpoint, and since the check is true we \textit{toggle} the breakpoint thereby making the following \texttt{child} 1 command problematic. The constant c cannot have a child which means this configuration would cause a run-time error. \\
 
In equation \ref{parentproblem} we show how using the \texttt{parent} command can cause problem when the root is unknown.
\begin{equation}
    \conf{\left(\lozenge\texttt{hole} \Rightarrow \texttt{parent}\right) \ggg \texttt{parent},\; \cursor{\lambda x.\hole}\; c} \label{parentproblem}
\end{equation}
In the example we first check if there is a hole in some subtree of the current cursor. This condition holds and we therefore evaluate the \texttt{parent} command resulting in the AST $\cursor{\lambda x.\hole\; c}$. When the next \texttt{parent} command is evaluated we have a run-time error since we are already situated at the root.\\

In equation \ref{astproblem} we show how an editor expression can result in an AST that would cause a run-time error when evaluated.
\begin{equation}
    \conf{\left(\neg\Box(\texttt{lambda}\; x) \Rightarrow \texttt{child}\; 1\right) \ggg \replace{\texttt{var}\; x}.\texttt{eval},\; \cursor{\lambda x.\hole}\; c} \label{astproblem}
\end{equation}
In the example we first check if it is \textbf{not} necessary that the subtree of the cursor contains a lambda expression. This condition does not hold since it is necessary. Since the condition does not hold we do not evaluate the \texttt{child} 1 command, which means the following substitution of \texttt{var} x is problematic. The substitution results in the AST $\cursor{\texttt{var}\; x}\; c$, which causes a run-time error when the command \texttt{eval} is evaluated, since the left child of the function application is no longer a function.
%
\section{Over-approximations}
As we cannot determine statically whether a condition holds, we establish over-approximations to ensure run-time errors cannot occur in well-typed configurations. As equation \ref{parentproblem} shows, conditioned expressions can result in loss of information about the cursor location. As such, we enforce the cursor \textit{depth} in the tree to be the same before and after a conditioned expression. Furthermore, the first cursor movement in a conditioned expression must be a \texttt{child} prefix. As equation \ref{condsubproblem} shows, conditioned substitution also results in loss of information. Thus, we can no longer guarantee that subsequent substitution at a deeper level is well-typed. Similarly, we no longer know of the structure of the subtree, such that we must condition \texttt{child} prefixes.\\

The above discussion leads to the following list of over-approximations:
\begin{itemize}
    \item In conditioned and recursive expressions, the cursor depth must be the same before and after.
    \item In conditioned and recursive expressions, only the subtree encapsulated by the cursor is accessible.
    \item After conditioned substitution, subsequent substitution at a deeper level is no longer valid, and the \texttt{child} prefix command must be conditioned.
\end{itemize}
%
\section{AST type rules}
\begin{table*}[htp]
    \centering
    \begin{align*}
        \runa{t-var} &\; \infrule{\Gamma_a\left(x\right)=\tau}{\Gamma_a \vdash x : \tau}\\
        %
        \runa{t-const} &\; \infrule{}{\Gamma_a \vdash c : b}\\
        %
        \runa{t-app} &\; \infrule{\Gamma_a \vdash a_1 : \tau_1 \rightarrow \tau_2 \quad \Gamma_a \vdash a_2 : \tau_1}{\Gamma_a \vdash a_1\; a_2 : \tau_2}\\
        %
        \runa{t-lambda} &\; \infrule{\Gamma_a\left[x \mapsto \tau_1\right] \vdash a : \tau_2}{\Gamma_a \vdash \lambda x:\tau_1.a : \tau_1 \rightarrow \tau_2} \\
        %
        \runa{t-break} &\; \infrule{\Gamma_a \vdash a : \tau}{\Gamma_a \vdash \breakpoint{a} : \tau} \\
        %
        \runa{t-hole} &\; \infrule{}{\Gamma_a \vdash \left(\hole : \tau\right) : \tau}
        %
    \end{align*}
    \caption{Type rules for abstract syntax trees.}
    \label{tab:typerules}
\end{table*}

%\section{Type context format}
%Here, we propose a format for type contexts of editor expressions. The context of an editor expression could be a triple $\Psi = (\Gamma_a, \tau, \Gamma)$, where $\Gamma_a$ is the type context for the subtree encapsulated by the cursor, $\tau$ is the type of the subtree and $\Gamma$ is a function or map from prefix command types to editor expression contexts. That is, contexts for editor expressions are recursive. Say we have context $(\Gamma_a, \tau, \Gamma)$. Upon a $\texttt{child}\; 1$ prefix, we \textit{look up} $\texttt{one}$ in $\Gamma$. If $\Gamma(\texttt{one}) = undef$, the expression is not well-typed. Otherwise, we evaluate the prefixed expression in the new context $\Gamma(\texttt{one})$.\\

%We construct the initial context based on the AST in the configuration $\conf{E,\; a}$. Upon a substitution prefix, we modify the context, upon a child or parent prefix, we \textit{move} in the context, and upon a conditioned or recursive expression, we set some of the bindings to $undef$: $\Gamma(T)=undef$.\\

%$\Gamma = T_1 : \Psi_1,...,T_n : \Psi_n$ \\
%$\Psi = (\Gamma_a, \tau, \Gamma)$
%Γ = T1 : Ψ1,..,Tn : Ψn
%Ψ = (Γa, τ, Γ)

\section{Experimental type system}

In this section, we introduce a type system for our editor-calculus. For the type system, we introduce the syntactic categories $\tau \in \mathbf{ATyp}$ to denote types of AST nodes, $T \in \mathbf{CTyp}$ to denote \textit{child} types, and p $\in \mathbf{Pth}$ to denote AST paths.
%
\begin{align*}
    \tau ::=&\; b \mid \tau_1 \rightarrow \tau_2 \mid \breakpoint{\tau} \mid \texttt{indet}\\
    T ::=&\; \texttt{one} \mid \texttt{two}\\
    p ::=&\; p\; T \mid \epsilon
\end{align*}

In addition to the basic and arrow types in $\mathbf{ATyp}$, we include a type for breakpoints, $\breakpoint{\tau}$, and a type to denote indeterminate types, \texttt{indet}. We use $\mathbf{Pth}$ to denote paths in an AST by storing a sequence of \textbb{one} and \textbb{two} which denote if the path goes through the first or second child.\\

We define two sets for contexts in our type system. The first context, $\mathbf{ACtx}$, stores type bindings for variables in the AST. The second context, $\mathbf{ECtx}$, stores, for all available paths so far, a pair of an AST context and the type of the node at the end of the path. We use $\Gamma_a \in \mathbf{ACtx}$ and $\Gamma_e \in \mathbf{ECtx}$ as metavariables for the two contexts. To check if a path $p$ is available in a context $\Gamma_e$, we use the notion $\Gamma_e(p) \neq \text{undef}$. $\mathbf{ACtx}$ and $\mathbf{ECtx}$ are thus defined as the following.
%
\begin{align*}
\mathbf{ACtx} &= \mathbf{Var} \rightharpoonup \mathbf{ATyp}\\
\mathbf{ECtx} &= \mathbf{Pth} \rightharpoonup \left(\mathbf{ACtx} \times \mathbf{ATyp}\right)
\end{align*}

To support our type system, we modify the syntax for AST node modifications by including type annotations for application, abstraction and holes. The new syntax thus becomes the following.
%
\begin{align*}
  D ::= \; & \texttt{var}\;x \mid \texttt{const}\;c \mid \texttt{app} : \tau_1 \rightarrow \tau_2, \tau_1 \mid \texttt{lambda}\; x : \tau_1 \rightarrow \tau_2 \mid \texttt{break} \mid \texttt{hole} : \tau
\end{align*}

To support breakpoint types, we introduce the notion of type consistency into our typesystem. The purpose of consistency in our type system is to ensure breakpoints types are consistent with their respective type, as defined below.
%
\begin{definition}{(Type consistency)}
    We define two types $\tau_1, \tau_2$ to be \textit{consistent}, denoted $\tau_1 \sim \tau_2$, by the following rules.
    \begin{align*}
        \runa{cons-1} \hspace{-1cm}
        \infrule{}{\tau \sim \tau} \hspace{-1cm}
        \runa{cons-2} \hspace{-1cm}
        \infrule{}{\breakpoint{\tau} \sim \tau} \hspace{-1cm}
        \runa{cons-3} \hspace{-1cm}
        \infrule{}{\tau \sim \breakpoint{\tau}} \hspace{-1cm}
        \runa{cons-4}
        \infrule{\tau_1 \sim \tau_1' \quad \tau_2 \sim \tau_2'}{(\tau_1 \rightarrow \tau_2) \sim (\tau_1' \rightarrow \tau_2')}
    \end{align*}
\end{definition}


\begin{table*}[htp]
    \centering
    \begin{align*}
        \runa{ctx-split-1}&\; \infrule{}{\emptyset = p \left(\emptyset\; \circ\; \emptyset\right)}\\
        \runa{ctx-split-2}&\; \infrule{\Gamma_e = p \left({\Gamma_e}_1\; \circ\; {\Gamma_e}_2\right)}{\Gamma_e,\; p\; T_1..T_n: (\Gamma_a,\; \tau) = p \left(\left({\Gamma_e}_1,\; p\; T_1..T_n: (\Gamma_a,\; \tau)\right)\; \circ\; {\Gamma_e}_2\right)}\\
        \runa{ctx-split-3}&\; \infrule{p_1 \neq p_2 \quad \Gamma_e = p_2 \left({\Gamma_e}_1\; \circ\; {\Gamma_e}_2\right)}{\Gamma_e,\; p_1\; T_1..T_n: (\Gamma_a,\; \tau) = p_2 \left({\Gamma_e}_1\; \circ\; \left({\Gamma_e}_2,\; p_1\; T_1..T_n: (\Gamma_a,\; \tau)\right)\right)}\\
        %
        \runa{ctx-update-1}&\; \infrule{}{\Gamma_e = \Gamma_e + \emptyset}\\
        \runa{ctx-update-2}&\; \infrule{\Gamma_e = \left({\Gamma_e}_1,\; p: ({\Gamma_a}_2,\; \tau_2)\right) + {\Gamma_e}_2}{\Gamma_e,\; p: ({\Gamma_a}_1,\; \tau_1) = \left({\Gamma_e}_1,\; p: ({\Gamma_a}_2,\; \tau_2)\right) + {\Gamma_e}_2}\\
        \runa{ctx-update-3}&\; \infrule{\Gamma_e = {\Gamma_e}_1 + {\Gamma_e}_2}{\Gamma_e,\; p: (\Gamma_a,\; \tau) = {\Gamma_e}_1 + \left({\Gamma_e}_2,\; p: (\Gamma_a,\; \tau)\right)}
    \end{align*}
    \caption{Context split and context update for editor contexts.}
    \label{tab:context}
\end{table*}
% We define \textit{type contexts}, $\Gamma_e$ in Table \ref{tab:context} as a mapping from a path $p$ to a pair consisting of an AST context $\Gamma_a$ and AST type $\tau$. We denote the $\Gamma_e, p : (\Gamma_a, \tau)$ as the type context equal to the paths not in the domain of map $\Gamma_e$ except for $p$, where $\Gamma_e(p) = (\Gamma_a, \tau)$. For type contexts we introduce the concept of \textit{context splitting} on a path in terms of $\Gamma_e$ maintained through two sub-contexts $\Gamma_{e1}$ and $\Gamma_{e2}$. For this we require a split-operation $\circ$, defined for two sub-contexts on a path as $\Gamma_e = p(\Gamma_{e1}\; \circ \; \Gamma_{e2})$. Notice the empty context is defined with the symbol $\emptyset$ as in \runa{ctx-split-1}. In rule \runa{ctx-split-2} we have that $p$ is in $\Gamma_{e1}$, but not in $\Gamma_{e2}$. Thus, $p$ is not in $\Gamma = \Gamma_{e1}\; \circ \; \Gamma_{e2}$, which is similarly done for the \runa{ctx-split-3} in terms of $\Gamma_{e1}$.\\

Next we introduce the notion of \textit{context updates} to update bindings in a context with new types for the associated path $p$. We use the addition operator $+$, to denote sum-context $\Gamma$ of two compatible type contexts $\Gamma_{e1}$ and $\Gamma_{e2}$. The rules require linear paths to not have bindings exist in another context. Thus, we can only update a context $\Gamma_{e2}$ iff no bindings for a given path is in context $\Gamma_{e1}$. In rule \runa{ctx-update-2} we have bindings in $\Gamma_{e1}$, which means we cannot add bindings to $\Gamma_{e2}$. However, in rule \runa{ctx-update-3} we allow path bindings in $\Gamma_{e2}$ since no such bindings are in context $\Gamma_{e1}$.

% \begin{equation}
%     depth(e) = \left\{
%         \begin{array}{ll}
%             depth(E) + 1            & \quad if e = (\texttt{child}\; n).E \\
%             depth(E) - 1            & \quad if e = \texttt{parent}.E\\
%             depth(E_1) + depth(E_2) & \quad if e = E_1 \ggg E_2\\
%             depth(E)                & \quad if e = \texttt{rec}\; x.E\\
%             depth(E)                & \quad if e = \pi.E\\
%             0                       & \quad otherwise
%         \end{array}
%     \right.
% \end{equation}

\begin{definition}{(Relative cursor depth)}
    We define the function $depth : \mathbf{Edt} \rightarrow \mathbb{Z}$, from the set of atomic editor expression to the set of integers.
    \begin{align*}
    depth((\texttt{child}\; n).E) &= depth(E) + 1 \\
    depth(\texttt{parent}.E) &= depth(E) - 1 \\
    depth(E_1 \ggg E_2) &= depth(E_1) + depth(E_2) \\
    depth(\texttt{rec}\; x.E) &= depth(E) \\
    depth(\pi.E) &= depth(E) \\
    depth(E) &= 0 
\end{align*}
\end{definition}
The $depth$ function statically analyses the structure of an editor expression to determine the relative depth of the cursor after evaluation of the expression. This function is used to make sure the position of the cursor before and after evaluation of an expression is the same. As the function performs a static analysis, we do not consider conditioned subexpressions. Later, in the type rules, we will see why we can safely ignore conditioned subexpressions. \\


% Next we define the function $match : \mathbf{Aam} \times \mathbf{ACtx} \times \mathbf{ATyp} \rightarrow \{tt, f\!\!f\}$. This function returns true if the type of the given AST modification $D$, is equal to the given AST type $\tau$.  
% \begin{align*}
%     match(\texttt{var}\; x,\;\Gamma_a,\;\tau) &= \left\{\begin{matrix}
%  tt & \text{if}\; \Gamma_a(x) = \tau\\ 
%  f\!\!f & \text{otherwise}
% \end{matrix}\right.\\
%     match(\texttt{const}\; c,\;\Gamma_a,\; b) &= tt\\
%     match(\texttt{app} : \tau_1 \rightarrow \tau_2,\; \tau_1,\;\Gamma_a,\; \tau_2) &= tt\\
%     match(\texttt{lambda}\; x : \tau_1 \rightarrow \tau_2,\;\Gamma_a,\; \tau_1 \rightarrow \tau_2) &= tt\\
%     match(\texttt{break},\;\Gamma_a,\; \tau) &= tt\\
%     match(\texttt{hole} : \tau,\;\Gamma_a,\; \tau) &= tt\\
%     match(D,\; \Gamma_a,\; \tau) &= f\!\!f
% \end{align*}

%\begin{equation*}
%    %context : \left(\mathbf{Aam} \times \mathbf{ACtx}\right) \rightharpoonup %\left(\left(\mathbf{Pth} \rightarrow \left(\left(\mathbf{Var} \rightharpoonup %\mathbf{ATyp}\right) \times \mathbf{ATyp}\right)\right) \cup \{error\}\right)
    %context : \left(\mathbf{Aam} \times \mathbf{ACtx} \times \mathbf{Pth} \right) %\rightharpoonup \mathbf{ECtx}
%\end{equation*}
%\begin{align*}
% context(\texttt{const}\; c,\; \Gamma_a,\; p) =&\; \emptyset\\
%  context(\texttt{hole} : \tau,\; \Gamma_a,\; p) =&\; \emptyset\\
%context(\texttt{var}\; x,\; \Gamma_a,\; p) =&\; \emptyset\\
 %context((\texttt{app} : \tau_1 \rightarrow \tau2,\; \tau_1),\; \Gamma_a,\; p) =&\; %\emptyset,\; p\; \texttt{one} : (\Gamma_a,\; \tau_1 \rightarrow \tau_2),\; p\; \texttt{two} : %(\Gamma_a,\; \tau_1)\\
 %context(\texttt{lambda}\; x : \tau_1 \rightarrow \tau_2,\; \Gamma_a,\; p) =&\; \emptyset,\; %p\; \texttt{one} : ((\Gamma_a,\; x : \tau_1),\; \tau_2)
%\end{align*}
%
%

We define functions \textit{limits} and \textit{follows} to analyze which cursor movement is safe given a condition holds. \textit{limits} finds the set of possible AST node modifiers, on which the cursor may sit, given the condition holds. \textit{follows} gives a set of editor type context bindings guaranteed to be safe, given the cursor sits on AST node modifier $D$. Note that the AST type context is empty and that the node type is $\texttt{indet}$, as we cannot determine such information based on a condition. Thus, besides toggling of breakpoints, substitution is not well-typed at path $p$ if $\Gamma_e(p)=(\emptyset,\; \texttt{indet})$. We can combine functions \textit{limits} and \textit{follows} to provide additional bindings to the editor type context of a conditioned expression $\phi \Rightarrow E$. The intersection of \textit{follows} applied to each AST node modifier $D$ in the set $limits(\phi)$ is the set of bindings guaranteed to be safe, given $\phi$ holds.

\theoremstyle{definition}
\begin{definition}{(Condition constraints)}
We define a function $limits: \mathbf{Eed} \rightarrow \mathcal{P}(\mathbf{Aam})$ from the set of conditions to the power set of the set of AST node modifiers. We assume conditions are in conjunctive normal form.
\begin{align*}
    limits(@D)=&\;\{D\}\\
    limits(\neg @D)=&\;\mathbf{Aam}\setminus \{D\}\\
    limits(\lozenge D)=&\;\{D\} \cup \{\texttt{app},\; \texttt{lambda}\; x,\; \texttt{break}\}\\
    limits(\neg \lozenge D)=&\;\mathbf{Aam}\setminus \{D\}\\
    limits(\Box D)=&\;\{D\} \cup \{\texttt{app},\; \texttt{lambda}\; x,\; \texttt{break}\}\\
    limits(\neg \Box D)=&\;\mathbf{Aam}\setminus \{D\}\\
    limits(\phi_1 \land \phi_2)=&\;limits(\phi_1) \cap limits(\phi_2)\\
    limits(\phi_1 \lor \phi_2)=&\;limits(\phi_1) \cup limits(\phi_2)
\end{align*}
\end{definition}


\theoremstyle{definition}
\begin{definition}{(Safe movement)}
We define a function $follows: \mathbf{Aam} \times \mathbf{Pth} \rightarrow \mathcal{P}\left(\mathbf{Pth} \times \left(\mathbf{ACtx} \times \mathbf{ATyp}\right)\right)$ from the set of pairs of AST node modifiers and paths to the power set of editor context bindings.
\begin{align*}
    \textit{follows}(\texttt{var}\; x,\; p)=&\; \emptyset\\
    \textit{follows}(\texttt{const}\; c,\; p)=&\; \emptyset\\
    \textit{follows}(\texttt{app},\; p)=&\; \{p\; \texttt{one} : (\emptyset,\; \texttt{indet}),\; p\; \texttt{two} : (\emptyset,\; \texttt{indet})\}\\
    \textit{follows}(\texttt{lambda}\; x,\; p)=&\; \{p\; \texttt{one} : (\emptyset,\; \texttt{indet})\}\\
    \textit{follows}(\texttt{break},\; p)=&\; \{p\; \texttt{one} : (\emptyset,\; \texttt{indet})\}\\
    \textit{follows}(\texttt{hole},\; p)=&\; \emptyset
\end{align*}
\end{definition}

%
%
We now introduce the type rules for editor expressions. Type rules for substitution are shown in table \ref{tab:typerulesv2sub} and the remaining rules are shown in table \ref{tab:typerulesv2}. The \texttt{child} n prefix is handled by \runa{t-child-1} and \runa{t-child-2}. Here we check that the cursor movement is viable by looking up the new path in $\Gamma_e$. Notice that the remaining editor expression $E$, is evaluated using the new path. The \texttt{parent} prefix is handled similarly in \runa{t-parent} with the exception being that we deconstruct the path instead of building it. When using recursion we require that the depth of the cursor is unchanged after evaluating the expression. We ensure this in \runa{t-rec} with the side condition $depth(E) = 0$. Similarly, \runa{t-cond} utilizes the same side condition to ensure that the cursor is unaffected by whether the condition holds or not. Notice here that evaluation of the conditioned expression is limited by what can follow the condition if it holds, denoted by $\delta$. Sequential composition is handled by the type rule \runa{t-seq}. Here we split the type context into $\Gamma_{e1}$, which contains information about the current subtree, and $\Gamma{e2}$, which contains information about the rest of the tree. This split ensures that the potentially hazardous evaluation of $E_1$ is kept separate from the evaluation of $E_2$.\\

\begin{table*}[htp]
    \centering
    \begin{align*}
        %
        \runa{t-eval} &\; \infrule{p,\; \Gamma_e \vdash E : ok}{p,\; \Gamma_e \vdash \texttt{eval}.E : ok}\\
        %
        \runa{t-child-1}&\; \infrule{\Gamma_e(p\; \texttt{one}) \neq \text{undef} \quad p\; \texttt{one},\; \Gamma_e \vdash E : ok}{p,\; \Gamma_e \vdash \left(\texttt{child}\; 1\right).E : ok}\\
        %
        \runa{t-child-2}&\; \infrule{\Gamma_e(p\; \texttt{two}) \neq \text{undef} \quad p\; \texttt{one},\; \Gamma_e \vdash E : ok}{p,\; \Gamma_e \vdash \left(\texttt{child}\; 2\right).E : ok}\\
        %
        \runa{t-parent}&\; \infrule{\Gamma_e(p) \neq \text{undef} \quad p,\; \Gamma_e \vdash E : ok}{p\; T,\; \Gamma_e \vdash \texttt{parent}.E : ok}\\
        %
        \runa{t-rec} &\; \condinfrule{p,\; \Gamma_e \vdash E : ok}{p,\; \Gamma_e \vdash \texttt{rec} x.E : ok}{\text{if}\; depth(E) = 0}\\
        %
        \runa{t-cond} &\; \condinfrule{p,\; \Gamma_e + \delta \vdash E : ok}{p,\; \Gamma_e \vdash \phi \Rightarrow E : ok}{\begin{align*}
            \text{if}\; &depth(E) = 0\;\\
            \text{and}\; &\delta = \bigcap_{D \in limits(\phi)}follows(D,\; p)\\
        \end{align*}}\\
        %
        \runa{t-seq} &\; \condinfrule{p,\; {\Gamma_e}_1 \vdash E_1 : ok \quad p,\; {\Gamma_e}_2 \vdash E_2 : ok}{p,\; \Gamma_e \vdash E_1 \ggg E_2 : ok}{\text{where}\; \Gamma_e = p\; ({\Gamma_e}_1\; \circ\; {\Gamma_e}_2)}\\
        %
        \runa{t-ref} &\; \infrule{}{p,\;\Gamma_e \vdash x : ok}\\
        %
        \runa{t-nil} &\; \infrule{}{p,\;\Gamma_e \vdash \mathbf{0} : ok}
    \end{align*}
    \caption{Type rules for editor expressions.}
    \label{tab:typerulesv2}
\end{table*}
%
%
Table \ref{tab:typerulesv2sub} shows the type rules for substitution. For substitution to be well-typed, the AST node type $\tau$ in the type context binding associated with the current path $p$ must be consistent with the type of the AST node modifier to be inserted. In \runa{t-sub-var}, we handle the special case where we insert a variable reference $x$. For this to be well-typed, a binding $\Gamma_a(x)=\tau'$ must exist, such that $\consistent{\tau}{\tau'}$. Note that substitution replaces a subtree of the AST. Thus, the bindings in the editor type context with paths starting with $p$ are no longer valid. Therefore, we split the type context on path $p$, such that $\Gamma_e = p\left({\Gamma_e}_1\;\circ\;{\Gamma_e}_2\right)$, and evaluate the prefixed expression $E$ in the type context ${\Gamma_e}_2$. That is, the type context containing all bindings of $\Gamma_e$ not starting with $p$. Note that the binding with path exactly $p$ is in both ${\Gamma_e}_1$ and ${\Gamma_e}_2$, however. We add bindings to ${\Gamma_e}_2$ in rules $\runa{t-sub-app}$ and $\runa{t-sub-abs}$. Particularly, we expand the AST type context upon substitution for an abstraction.\\

We treat substitution of breakpoints differently, as we can either toggle breakpoints on or off. Furthermore, we do not replace the subtree upon substitution for breakpoints. Instead, we must modify the bindings with paths starting with $p$, to either include or remove a $\texttt{one}$. Additionally, we change the type in the binding at the current path $p$ to indicate whether it has a breakpoint. Note that we toggle off the breakpoint if the type is of the form $\breakpoint{\tau}$, and toggle it on otherwise. Thus, the type indicates the structure of the tree.
%
%
\begin{table}
    \begin{flalign*}
        %
        \runa{t-sub-var} &\; \condinfrule{\Gamma_e(p)=(\Gamma_a,\;\tau) \quad \Gamma_a(x) = \tau' \quad \consistent{\tau}{\tau'} \quad p,\;{\Gamma_e}_2 \vdash E : ok}{p,\; \Gamma_e \vdash \replace{\texttt{var}\; x}.E : ok}{\text{where}\; \Gamma_e = p\; ({\Gamma_e}_1\; \circ\; {\Gamma_e}_2)} \\
        %
        \runa{t-sub-const} &\; \condinfrule{\Gamma_e(p)=(\Gamma_a,\;b) \quad p,\;{\Gamma_e}_2 \vdash E : ok}{p,\; \Gamma_e \vdash \replace{\texttt{const}\; c}.E : ok}{\text{where}\; \Gamma_e = p\; ({\Gamma_e}_1\; \circ\; {\Gamma_e}_2)}\\
        %
        \runa{t-sub-app} &\; \condinfrule{\Gamma_e(p)=(\Gamma_a,\; \tau_2') \quad \consistent{\tau_2}{\tau_2'} \quad p,\; \Gamma_e' \vdash E : ok}{p,\; \Gamma_e \vdash \replace{\texttt{app} : \tau_1 \rightarrow \tau_2,\; \tau_1}.E : ok}{\begin{align*}
            &\text{where}\; \Gamma_e = p\; ({\Gamma_e}_1\; \circ\; {\Gamma_e}_2)\;\\
            &\text{and}\; \Gamma_e' = {\Gamma_e}_2,\; p\; \texttt{one} : (\Gamma_a,\; \tau_1 \rightarrow \tau_2),\; p\; \texttt{two} : (\Gamma_a,\; \tau_1)
        \end{align*}}\\
        %
        \runa{t-sub-abs} &\; \condinfrule{\Gamma_e(p)=(\Gamma_a,\; \tau_1' \rightarrow \tau_2') \quad \consistent{\tau_1 \rightarrow \tau_2}{\tau_1' \rightarrow \tau_2'} \quad p,\; \Gamma_e' \vdash E : ok}{p,\; \Gamma_e \vdash \replace{\texttt{lambda}\; x : \tau_1 \rightarrow \tau_2}.E : ok}{\begin{align*}
        &\text{where}\;\Gamma_e = p\; ({\Gamma_e}_1\; \circ\; {\Gamma_e}_2)\\
        &\text{and}\;\Gamma_e' = {\Gamma_e}_2, p\; \texttt{one} : ((\Gamma_a,\; x : \tau_1),\; \tau_2)\end{align*}} \\
        %
        %\runa{t-sub} &\; \infrule{match(D,\; \Gamma_a,\; \tau) = tt \quad p,\;\Gamma_e' \vdash %E : ok}{p,\;\Gamma_e \vdash \replace{D}.E : ok} \\
        %&\text{if}\; D \neq \texttt{break}\\
        %&\text{and}\; \Gamma_e(p)=(\Gamma_a,\;\tau) \\
        %&\text{and}\; \Gamma_e = p\; ({\Gamma_e}_1\; \circ\; {\Gamma_e}_2)\\
        %&\text{and}\; \Gamma_e' = {\Gamma_e}_2 + context(D,\; \Gamma_a)\\
        %
        \runa{t-sub-break-1} &\; \infrule{\Gamma_e(p)=(\Gamma_a,\; \breakpoint{\tau}) \quad p,\; \Gamma_e' \vdash E : ok}{p,\; \Gamma_e \vdash \replace{\texttt{break}} : ok} \\
        &\text{where}\; \Gamma_e = p\; ({\Gamma_e}_1\; \circ\; {\Gamma_e}_2)\\
        &\text{and}\; {\Gamma_e}_1 = \emptyset,\; p\; \texttt{one}\; T_1..T_{n_1} : ({\Gamma_a}_1,\; \tau_1),..,p\; \texttt{one}\; T_1..T_{n_m} : ({\Gamma_a}_m,\; \tau_m)\\
        &\text{and}\; {\Gamma_e}_1' =\emptyset,\; p\; T_1..T_{n_1} : ({\Gamma_a}_1,\; \tau_1),..,p\; T_1..T_{n_m} : ({\Gamma_a}_m,\; \tau_m)\\
        &\text{and}\; \Gamma_e' = \left({\Gamma_e}_2 + {\Gamma_e}_1'\right),\; p : (\Gamma_a,\; \tau)\\
        %
        \runa{t-sub-break-2} &\; \infrule{\Gamma_e(p)=(\Gamma_a,\;\tau)\quad  p,\; \Gamma_e' \vdash E : ok}{p,\; \Gamma_e \vdash \replace{\texttt{break}} : ok} \\
        &\text{where}\; \Gamma_e = p\; ({\Gamma_e}_1\; \circ\; {\Gamma_e}_2)\\
        &\text{and}\; {\Gamma_e}_1 =\emptyset,\; p\; T_1..T_{n_1} : ({\Gamma_a}_1,\; \tau_1),..,p\; T_1..T_{n_m} : ({\Gamma_a}_m,\; \tau_m)\\
        &\text{and}\; {\Gamma_e}_1' = \emptyset,\; p\; \texttt{one}\; T_1..T_{n_1} : ({\Gamma_a}_1,\; \tau_1),..,p\; \texttt{one}\; T_1..T_{n_m} : ({\Gamma_a}_m,\; \tau_m)\\
        &\text{and}\; \Gamma_e' = \left({\Gamma_e}_2 + {\Gamma_e}_1'\right),\; p : (\Gamma_a,\; \breakpoint{\tau})\\
        %
        \runa{t-sub-hole} &\; \condinfrule{\Gamma_e(p)=(\Gamma_a,\;\tau') \quad \consistent{\tau}{\tau'} \quad p,\;{\Gamma_e}_2 \vdash E : ok}{p,\; \Gamma_e \vdash \replace{\texttt{hole} : \tau}.E : ok}{\text{where}\; \Gamma_e = p\; ({\Gamma_e}_1\; \circ\; {\Gamma_e}_2)}
        %
    \end{flalign*}
    \caption{Type rules for substitution.}
    \label{tab:typerulesv2sub}
\end{table}

%\begin{table*}[htp]
%    \centering
%    \begin{align*}
        %%
        %\runa{t-eval} &\; \infrule{p,\; \Gamma_e \vdash E : ok \dashv p',\; \Gamma_e'}{p,\; \Gamma_e \vdash \texttt{eval}.E : %ok \dashv p',\; \Gamma_e'}\\
        %%
        %\runa{t-sub} &\; \infrule{T=\tau \quad p,\;\Gamma_e'' \vdash E : ok \dashv p',\;\Gamma_e'}{p,\;\Gamma_e \vdash %\replace{D}.E : ok \dashv p',\;\Gamma_e'} \\
        %&\text{where}\; \Gamma_e(p)=(\Gamma_a,\;\tau) \\
        %&\text{and}\; T = type(D,\;\Gamma_a) \\
        %&\text{and}\; \Gamma_e = p\; ({\Gamma_e}_1\; \circ\; {\Gamma_e}_2)\\
        %&\text{and}\; \Gamma_e'' = {\Gamma_e}_1 + context(D,\; \Gamma_a)\\
        %%
        %\runa{t-child-1}&\; \infrule{\Gamma_e(p\; \texttt{one}) \neq undef \quad p,\; \texttt{one},\; \Gamma_e \vdash E : ok %\dashv p',\; \Gamma_e'}{p,\; \Gamma_e \vdash \left(\texttt{child}\; 1\right).E : ok \dashv p',\; \Gamma_e'}\\
        %%
        %\runa{t-child-2}&\; \infrule{\Gamma_e(p\; \texttt{two}) \neq undef \quad p,\; \texttt{one},\; \Gamma_e \vdash E : ok %\dashv p',\; \Gamma_e'}{p,\; \Gamma_e \vdash \left(\texttt{child}\; 2\right).E : ok \dashv p',\; \Gamma_e'}\\
        %%
        %\runa{t-parent}&\; \infrule{\Gamma_e(p) \neq undef \quad p,\; \Gamma_e \vdash E : ok \dashv p',\; \Gamma_e'}{p\; T,\; %\Gamma_e \vdash \texttt{parent}.E : ok \dashv p',\; \Gamma_e'}\\
        %%
        %\runa{t-rec} &\; \condinfrule{p,\; {\Gamma_e}_1 \vdash E : ok \dashv p,\; \Gamma_e'}{p,\; \Gamma_e \vdash \texttt{rec} %x.E : ok \dashv p,\; {\Gamma_e}_2}{\text{where}\; \Gamma_e = p\; ({\Gamma_e}_1\; \circ\; {\Gamma_e}_2)}\\
        %%
        %\runa{t-seq} &\; \infrule{p,\; \Gamma_e \vdash E_1 : ok \dashv p'',\; \Gamma_e'' \quad p'',\; \Gamma_e'' \vdash E_2 : %ok \dashv p',\; \Gamma_e'}{p,\; \Gamma_e \vdash E_1 \ggg E_2 : ok \dashv p',\; \Gamma_e'}\\
        %%
        %\runa{t-cond} &\; \infrule{p,\; {\Gamma_e}_1 + \delta \vdash E : ok \dashv p,\; \Gamma_e'}{p,\; \Gamma_e \vdash \phi %\Rightarrow E : ok \dashv p,\; {\Gamma_e}_2}\\
%        &\text{where}\; \Gamma_e = p\; ({\Gamma_e}_1\; \circ\; {\Gamma_e}_2)\\
%        &\text{and}\; \delta = \bigcap_{D \in limits(\phi)}follows(D)\\
%        %
%        \runa{t-ref} &\; \infrule{}{p,\;\Gamma_e \vdash x : ok \dashv p,\;\Gamma_e}\\
%        %
%        \runa{t-nil} &\; \infrule{}{p,\;\Gamma_e \vdash \mathbf{0} : ok \dashv p,\;\Gamma_e}\\
%    \end{align*}
%    \caption{Type rules for editor expressions.}
%    \label{tab:typerules}
%\end{table*}

\begin{theorem} (Subject reduction)
If $\Gamma_e, \;\Gamma_a \vdash \conf{E,\;a} : ok$ and $\conf{E, a} \xrightarrow{\alpha} \conf{E', a'}$ then $\Gamma_e, \;\Gamma_a \vdash \conf{E',\;a'} : ok$.
\end{theorem}

We define \textit{well-typedness} of a configuration $\conf{E,\;a}$ by the following rule: \\
$\condinfrule{\Gamma_a \vdash a : \tau \quad p,\; \Gamma_e \vdash E : ok}{\Gamma_e, \;\Gamma_a \vdash \conf{E,\;a} : ok}{\begin{align*}
        &\text{where}\;\\
        &\text{and}\;\end{align*}}$
        
        

\chapter{Type checking sized types for parallel complexity}\label{ch:typecheck}
As mentioned in Chapter \ref{ch:bgts}, Baillot and Ghyselen \cite{BaillotGhyselen2021} bound sizes of algebraic terms and synchronizations on channels using indices, leading to a partial order on indices. For instance, for a process of the form $\inputch{a}{v}{}{\asyncoutputch{b}{v}{}} \mid P$ (assuming synchronizations induce a cost in time complexity of one) we must enforce that the bound on $a$ is strictly smaller than the bound on $b$. Thus, we must impose constraints on the interpretations of indices. Another concern in the typing of the process above is the parallel complexity. Granted separate bounds on the complexities of $\inputch{a}{v}{}{\asyncoutputch{b}{v}{}}$ and $P$, how do we establish a tight bound on their parallel composition? This turns out to be another major challenge, as bounds may be parametric, such that comparison of bounds is a partial order. Finally, for a process of the form $!\inputch{a}{v}{}{P} \mid \asyncoutputch{a}{e}{}$ we must \textit{instantiate} the parametric complexity of $!\inputch{a}{v}{}{P}$ based on the deducible size bounds of $e$, which quickly becomes difficult as indices become more complex.\\ % As the type system is otherwise fairly standard, for instance using input/output types for channels, the challenge in introducing type check is to ensure constraints on indices are not violated.\\
%
%Type inference for the type system introduced in Baillot and Ghyselen \cite{BaillotGhyselen2021} is complicated by similar challenges to that of type checking, such as constraint satisfaction. Another concern with respect to sized types is that we must infer indices. Here, it is relevant to consider existing work on sized type inference, such as Hughes et al. \cite{HughesEtAl1996} and Avanzini and Dal Lago \cite{AvanziniLago2017}. The set of function symbols used to form indices should be be more strictly defined, to make inference tractable. We also must be careful with respect to recursion, predominantly with how primitive recursion can be identified. In this chapter, we address some of these challenges.
%
% The type system for parallel complexity of message-passing processes introduced in Baillot and Ghyselen \cite{BaillotGhyselen2021} uses sized types to express parametric complexity of invoking replicated inputs, and thereby achieve precise bounds on primitively recursive processes. This requires a notion of polymorphism in the message types of replicated inputs. Baillot and Ghyselen introduce size polymorphism by bounding sizes of algebraic terms and synchronizations on channels with algebraic expressions referred to as indices that may contain index variables representing unknown sizes. We may interpret an index with an index valuation that maps its index variables to naturals, such that the index may be evaluated.\\ %
%
% The bounds on sizes and synchronizations lead to a partial order on indices. For instance, for a process of the form $\inputch{a}{v}{}{\asyncoutputch{b}{v}{}} \mid P$ (assuming synchronizations induce a cost in time complexity of one) we must enforce that the bound on $a$ is strictly smaller than the bound on $b$. Thus, we must induce constraints on the interpretations of indices. As the type system is otherwise fairly standard, for instance using input/output types for channels, the challenge in introducing type check is to ensure constraints on indices are not violated.

The purpose of this section is to present a version of the type system by Baillot and Ghyselen that is algorithmic in the sense that its type rules can be easily implemented in a programming language, and so we must address the challenges described above. For the type checker, we assume we are given a set of constraints $\Phi$ on index variables in $\varphi$ and a type environment $\Gamma$. We first present the types of the type system as well as subtyping. Afterwards, we present auxiliary functions and type rules. For the type rules we also present the concept of combined complexities that we use to bound parallel complexities by deferring comparisons of indices when these are not defined. We then prove the soundness of the type checker and show how it can be extended accompanied by examples. Finally, we show how we can verify the constraint judgements that show up in the type rules.

\section{Auxiliary functions}
We first present two functions that will be used in the type rules. As the continuation of a replicated input may be invoked an arbitrary number of times at different time steps, we need to ensure that names used within the continuation are of types that are invariant to time as defined in Definition \ref{def:timeinvariance}, i.e. they may be used at any time step. In Definition \ref{def:readyfunc}, we define a function $\text{ready}(\varphi,\Phi,T)$ that discards use-capabilities from a type to obtain time invariance. For a server type $\forall_I\widetilde{i}.\texttt{serv}^\sigma_K(\widetilde{T)}$, outputs are well-typed whenever $\varphi;\Phi\vDash I \leq 0$, and so for names of such types, we only discard input capabilities, whenever we can guarantee the constraint judgement $\varphi;\Phi\vDash I \leq 0$. We return to how to guarantee constraint judgements in section \ref{sec:verifyinglinearjudgements}.
%
\begin{defi}\label{def:readyfunc}
We inductively define a function \textit{ready} that transforms a type into one that is time invariant.
\begin{align*}
    %\text{ready}(\varphi,\Phi,\epsilon) =&\; \epsilon\\
    %
    \text{ready}(\varphi,\Phi,\natinterval{I}{J}) =&\; \natinterval{I}{J}\\
    %
    \text{ready}(\varphi,\Phi,\forall_I\widetilde{i}.\texttt{serv}^{\sigma}_K(\widetilde{T})) =&\; \left\{ \begin{matrix}
        \forall_I\widetilde{i}.\texttt{serv}^{\sigma \cap \{\texttt{out}\}}_K(\widetilde{T}) & \text{if}\; \varphi;\Phi\vDash I \leq 0\\
        \forall_0\widetilde{i}.\texttt{serv}^{\emptyset}_K(\widetilde{T}) & \text{if}\; \varphi;\Phi\nvDash I \leq 0
    \end{matrix} \right.\\
    %
    %\text{ready}(\varphi,\Phi,\Gamma,a:\oservS{I}{\widetilde{i}}{K}{\widetilde{T}}) =&\; \left\{ \begin{matrix}
    %    \text{ready}(\varphi,\Phi,\Gamma), a:\oservS{I}{\widetilde{i}}{K}{\widetilde{T}} & \text{if}\; %\varphi;\Phi\vDash I \leq 0\\
    %    \text{ready}(\varphi,\Phi,\Gamma) & \text{if}\; \varphi;\Phi\nvDash I \leq 0
    %\end{matrix} \right.\\
    %%
    %\text{ready}(\varphi,\Phi,\Gamma,a:\iservS{I}{\widetilde{i}}{K}{\widetilde{T}}) =&\; %\text{ready}(\varphi,\Phi,\Gamma)\\
    %
    \text{ready}(\varphi,\Phi,\texttt{ch}^\sigma_I(\widetilde{T})) =&\;\texttt{ch}^\emptyset_0(\widetilde{T})%\\
    %
    %\text{ready}(\varphi,\Phi,\Gamma,a:\outchanneltypeS{I}{\widetilde{T}}) =&\; \text{ready}(\varphi,\Phi,\Gamma)\\
    %
    %\text{ready}(\varphi,\Phi,\Gamma,a:\inchanneltypeS{I}{\widetilde{T}}) =&\; \text{ready}(\varphi,\Phi,\Gamma)
\end{align*}
We extend \textit{ready} to type contexts such that for $v\in\texttt{dom}(\Gamma)$ we have that $\text{ready}(\varphi,\Phi,\Gamma)(v)=\text{ready}(\varphi,\Phi,\Gamma(v))$.
\end{defi}

% In Definition \ref{def:joinbase}, we introduce a binary function on base types $\uplus_{\varphi;\Phi}$ that computes a base type that is a subtype of both argument types, if such a base type exists. We do this by selecting the least lower bound and the greatest upper bound amongst the two argument base types as the new size bounds. This function will be useful for typing list expressions, as the elements of a list may be typed with different size bounds that we will need a common subtype of.

% \begin{defi}[Joining base types]\label{def:joinbase}

% \begin{align*}
%     \texttt{Nat}[I,J] \uplus_{\varphi;\Phi} \texttt{Nat}[I',J'] =&\; \left\{
%     \begin{matrix}
%         \texttt{Nat}[I,J] & \varphi;\Phi\vDash I \leq I'\;\text{and};\varphi;\Phi\vDash J' \leq J\\
%         \texttt{Nat}[I',J] & \varphi;\Phi\vDash I' \leq I\;\text{and};\varphi;\Phi\vDash J' \leq J\\
%         \texttt{Nat}[I,J'] & \varphi;\Phi\vDash I \leq I'\;\text{and};\varphi;\Phi\vDash J \leq J'\\
%         \texttt{Nat}[I',J'] & \varphi;\Phi\vDash I' \leq I\;\text{and};\varphi;\Phi\vDash J \leq J'
%     \end{matrix}
%     \right.\\
    
%     \texttt{List}[I,J](\mathcal{B}) \uplus_{\varphi;\Phi} \texttt{List}[I',J'](\mathcal{B}') =&\; \left\{
%     \begin{matrix}
%         \texttt{List}[I,J](\mathcal{B} \uplus_{\varphi;\Phi} \mathcal{B}') & \varphi;\Phi\vDash I \leq I'\;\text{and};\varphi;\Phi\vDash J' \leq J\\
%         \texttt{List}[I',J](\mathcal{B} \uplus_{\varphi;\Phi} \mathcal{B}') & \varphi;\Phi\vDash I' \leq I\;\text{and};\varphi;\Phi\vDash J' \leq J\\
%         \texttt{List}[I,J'](\mathcal{B} \uplus_{\varphi;\Phi} \mathcal{B}') & \varphi;\Phi\vDash I \leq I'\;\text{and};\varphi;\Phi\vDash J \leq J'\\
%         \texttt{List}[I',J'](\mathcal{B} \uplus_{\varphi;\Phi} \mathcal{B}') & \varphi;\Phi\vDash I' \leq I\;\text{and};\varphi;\Phi\vDash J \leq J'
%     \end{matrix}
%     \right.
% \end{align*}
% \end{defi}

% \begin{defi}[Removing servers]
%     Given a type context $\Gamma$, the function \textit{noserv} removes all server types from the context.
%     \begin{align*}
%         \text{noserv}(\emptyset) &= \emptyset\\
%         \text{noserv}(\Gamma, \natinterval{I}{J}) &= \text{noserv}(\Gamma),\natinterval{I}{J}\\
%         \text{noserv}(\Gamma, \chant{I}{\sigma}{\widetilde{T}}) &= \text{noserv}(\Gamma),\chant{I}{\sigma}{\widetilde{T}})\\
%         \text{noserv}(\Gamma, \servt{I}{\widetilde{i}}{\sigma}{K}{\widetilde{T}}) &= \text{noserv}(\Gamma)\\
%     \end{align*}
% \end{defi}


In Definition \ref{def:instantiatef}, we introduce a function $\text{instantiate}(\widetilde{i},\widetilde{T})$ that assigns the index variables in sequence $\widetilde{i}$ to indices in types of the sequence $\widetilde{T}$, by means of a substitution of indices for index variables. Note that the function is only defined for sequences such that the number of index variables equals the number of indices in the types. This function will be useful for outputs on servers, where the parametric types $\widetilde{S}$ of a server type $\forall_I\widetilde{i}.\texttt{serv}^{\{\texttt{out}\}}_K(\widetilde{S)}$ must match the types of expressions $\widetilde{T}$ to be output. More specifically, there must exist a substitution $\{\widetilde{J}/\widetilde{i}\}$ such that $\widetilde{T} \sqsubseteq \widetilde{S}\{\widetilde{J}/\widetilde{i}\}$. We return to this in Section \ref{section:typeruless}.

\begin{defi}[Server invocation]\label{def:instantiatef}
We inductively define a function \textit{instantiate} that constructs a substitution of indices for index variables, provided a sequence of index variables and a sequence of types%. The function is only defined for sequences such that the number of index variables equals the number of indices in the types.
\begin{align*}
    \text{instantiate}(\epsilon,\epsilon) =&\; \{\}\\
    \text{instantiate}((\widetilde{i},\widetilde{j}),(T,\widetilde{S})) =&\; \text{instantiate}(\widetilde{i},T),\text{instantiate}(\widetilde{j},\widetilde{S})\\
    \text{instantiate}((i,j),\texttt{Nat}[I,J]) =&\; \{I/i,J/j\}\\
    %\text{instantiate}((i,j,\widetilde{k}),\texttt{List}[I,J](\mathcal{B})) =&\; \{I/i,J/j\}, \text{instantiate}(\widetilde{k},\mathcal{B})\\
    \text{instantiate}((i,\widetilde{j}),\texttt{ch}_I^\sigma(\widetilde{T})) =&\; \{I/i\},\text{instantiate}(\widetilde{j},\widetilde{T})\\
    \text{instantiate}((i,j,\widetilde{k}),\forall_I\widetilde{l}.\texttt{serv}^\sigma_K(\widetilde{T})) =&\; \{I/i,K/j\},\text{instantiate}(\widetilde{k},\widetilde{T})
\end{align*}
\end{defi}
\section{Algorithmic type rules}\label{section:typeruless}
%We are now ready to introduce type rules for a type checker of the type system in Baillot and Ghyselen \cite{BaillotGhyselen2021}. %TODO 
%
%
%A piecewise complexity $\kappa$ is a set of pairs $(\Phi_i, K_i)$ where $K_i$ is an index describing a complexity that is valid within the feasible region described by the set of constraints $\Phi_i$. As such, a piecewise complexity $\kappa = \{(\Phi_1, K_1), \cdots, (\Phi_n, K_n)\}$ describes a complexity bound within the feasible region $\mathcal{M}_\varphi(\Phi_1) \cup \cdots \cup \mathcal{M}_\varphi(\Phi_n)$ for some $\varphi$ such that $\Phi_1, \cdots, \Phi_n$ use index variables in $\varphi$. In the case where $m$ feasible regions $\mathcal{M}_\varphi(\Phi_{i_1})$, $\mathcal{M}_\varphi(\Phi_{i_{m-1}})$ and $\mathcal{M}_\varphi(\Phi_{i_m})$ intersect, we choose the maximal complexity of the corresponding complexities for any valuation $\rho$ in the intersecting region.\\

When typing a process, we often need to find an index that is an upper bound on two other indices, for which there may be many options. To allow for the type checker to be as precise as possible, we want to find the minimum complexity that is a bound of two other complexities, which will depend on the representation of complexity, and as such, instead of representing complexity bounds using indices, we opt to use sets of indices which we refer to as \textit{combined complexities}. Intuitively, given any point in the space spanned by some index variables, the combined complexity at that point is the maximum of the complexities making up the combined complexity at that point. This is illustrated in Figure \ref{fig:combined_complexity} which shows a combined complexity consisting of three indices. The red dashed line represents the bound on the combined complexity. Representing complexities as sets of indices has the effect of \textit{externalizing} the process of finding bounds of complexities by deferring this until a later time. We will later define the algorithm \textit{basis} that removes superfluous indices of a combined complexity. In Figure \ref{fig:combined_complexity} the index $K$ is superfluous as it never contributes to the bound of the combined complexity.

\begin{figure}
    \centering
    \begin{tikzpicture}
\begin{axis}[
    axis lines = left,
    xlabel = \(i\),
    ylabel = {},
    domain = 0:2.5,
    xtick={\empty},ytick={\empty},
    ymin=0,
    ymax=5.2,
    xmax=2.7,
    restrict y to domain=0:5,
]
    \addplot[thick, color=orange]{x^2} node[above,pos=1] {I};
    \addplot[thick, color=blue]{x+1} node[above,pos=1] {J};
    \addplot[thick, color=green]{ln(x+1)*2} node[above,pos=1] {K};
    \draw [ultra thick, dashed, draw=red] (axis cs:0,1) -- (axis cs:1.62,2.62);
    \addplot[ultra thick, color=green, dashed, color=red, domain=1.62:2.5]{x^2};
\end{axis}
\end{tikzpicture}
    \caption{Combined complexities illustrated. The combined complexity consists of the three indices I, J, K of the single index variable $i$. The dashed red line shows the bound of the combined complexity. $K$ is a superfluous index in the combined complexity as it never contributes to the bound of the combined complexity.}
    \label{fig:combined_complexity}
\end{figure}


\begin{defi}[Combined complexity]\label{def:combinedcomp} 
    We refer to a set $\kappa$ of complexities as a \textit{combined complexity}. We extend constraint judgements to include combined complexities such that
    \begin{enumerate}
        \item $\varphi;\Phi\vDash \kappa \leq \kappa'$ if for all $K \in \kappa$ there exists $K'\in \kappa'$ such that $\varphi;\Phi\vDash K \leq K'$.
        % 
        \item $\varphi;\Phi\vDash \kappa = \kappa'$ if $\varphi;\Phi\vDash \kappa \leq \kappa'$ and $\varphi;\Phi\vDash \kappa' \leq \kappa$.
        \item $\kappa + I = \{K + I \mid K \in \kappa\}$.
        %
        \item $\kappa\{J/i\} = \{ K\{J/i\} \mid K\in\kappa \}$.
    \end{enumerate}
    In the above, we may substitute an index for a combined complexity. In such judgements, the index represents a singleton set. For instance, $\varphi;\Phi\vDash \kappa \leq K$ represents $\varphi;\Phi\vDash \kappa \leq \{K\}$.
    %$\varphi;\Phi \vDash \kappa \bowtie \kappa' \quad\text{ if }\quad \forall K \in \kappa. (\exists K' \in \kappa'. \varphi;\Phi \vDash K \bowtie K')$.
\end{defi}

More specifically, when considering a combined complexity $\kappa$, we are interested in the maximal complexity given some valuation $\rho$, which we find by simply comparing the different values for the complexities within $\kappa$ given $\rho$. Note that the complexity $K \in \kappa$ that is maximal may be different for different valuations. In Definition \ref{def:combinedcomp} we extend the binary relations in $\bowtie$ on indices to combined complexities, such that we can compare two combined complexities such as $\varphi;\Phi \vDash \kappa \bowtie \kappa'$ and a combined complexity and complexity such as $\varphi;\Phi \vDash \kappa \bowtie K$. Definition \ref{def:combinedcompbasis} defines the function \textit{basis} that discards any $K \in \kappa$ that can never be the maximal complexity given some set of constraints $\Phi$ (i.e. the complexities that are bounded by other complexities in the set), such that we can always keep the number of complexities in a combined complexity to a minimum. %Finally, we may also be interested in adding an index onto a combined complexity, and so we define the addition of indices onto combined complexities in Definition \ref{def:combinedcompadd}.
%
\begin{defi}\label{def:combinedcompbasis}
    We define the function \textit{basis} that takes a set of index variables $\varphi$, a set of constraints $\Phi$ and a combined complexity $\kappa$, and returns a new combined complexity without superfluous complexities (The \textit{basis} of $\kappa$)
    \begin{align*}
        \text{basis}(\varphi,\Phi,\kappa) = \bigcap\left\{ \kappa' \subseteq \kappa \mid \forall K\in\kappa.\exists K'\in\kappa'.\varphi;\Phi\vDash K \leq K' \right\}
    \end{align*}
    Moreover, the algorithm below computes the basis
    % \begin{align*}
    %     \text{basis}(\varphi, \kappa) = \{(\Phi, K) \in \kappa \mid \varphi;\Phi \not \vDash K < K' \text{ for all } (\Phi', K') \in \kappa\}
    % \end{align*}
    \begin{align*}
        &\text{basis}(\varphi, \Phi, \kappa) = \text{do}\\[-0.5em]
        &\quad \kappa' \leftarrow \kappa\\[-0.5em]
        &\quad \text{for } K \in \kappa \text{ do}\\[-0.5em]
        &\quad\quad \text{ if } \exists K' \in \kappa' \text{ with } K \not = K' \text{ and } \varphi;\Phi \vDash K \leq K' \text{ then}\\[-0.5em]
        &\quad\quad\quad \kappa' \leftarrow \kappa' \setminus \{K\}\\[-0.5em]
        &\quad \text{return } \kappa'
    \end{align*}
\end{defi}
%
% \begin{defi}[]\label{def:combinedcompadd}
%     We define the the addition of a combined complexity and index as
%     \begin{align*}
%         \kappa + I = \{K + I \mid K \in \kappa\}
%     \end{align*}

% \end{defi}
%
For typing expressions, we use the rules presented in Table \ref{tab:sizedtypedexpressiontypes}, excluding the rule $\runa{BG-sub}$. In Table \ref{tab:sizedprocesstypingrules} we show the type rules for processes. Type judgements are of the form $\varphi;\Phi;\Gamma \vdash P \triangleleft \kappa$ where $\kappa$ denotes the complexity of process $P$. The rule $\runa{S-tick}$ types a \texttt{tick} prefix and incurs a cost of one in time complexity. We advance the time of all types in the context accordingly when typing the continuation. Rule $\runa{S-annot}$ is similar but may incur a cost of $n$. Matches on naturals are typed with rule $\runa{S-nmatch}$. Most notably, we extend the set of known constraints when typing the two continuations. That is, we can deduce constraints on the lower and upper bounds on the size of the expression we match on. For instance, for the zero pattern we can deduce that the lower bound $I$ must be equal to $0$ (or equivalently $I \leq 0$), and for the successor pattern, we can guarantee that the upper bound $J$ must be greater than or equal to $1$. For the complexity of pattern matches and parallel composition, we take advantage of the fact that we represent complexities using combined complexities. As such, we include complexities in both $P$ and $Q$ in the result. To remove redundancy from the set $\kappa \cup \kappa'$, we use the basis function.\\

%
% \begin{table*}[!ht]
%     \begin{framed}\vspace{-1em}\begin{align*}
%         &\kern15em\\[-2em] % Stretch frame
%         &\kern0em\runa{S-nil}\infrule{}{\varphi;\Phi;\Gamma \vdash \withcomplex{\nil}{0}} \kern1em\runa{S-tick}\;\infrule{\varphi;\Phi;\susumesim{\Gamma}{1}\vdash P \triangleleft K}{\varphi;\Phi;\Gamma\vdash \tick P \triangleleft K + 1} \kern3em\runa{S-nu}\;\infrule{\varphi;\Phi;\Gamma,\withtype{a}{T} \vdash \withcomplex{P}{K}}{\varphi;\Phi;\Gamma \vdash \newvar{a: T}{\withcomplex{P}{K}}}\\[-1em]
%         %
%         &\kern-0em\runa{S-nmatch}\;\condinfrule{
%         \begin{matrix}
%             \varphi;\Phi;\Gamma \vdash \withtype{e}{\natinterval{I}{J}}\quad \varphi;\Phi, I \leq 0;\Gamma \vdash \withcomplex{P}{K} \\
%             \varphi;\Phi, J \geq 1;\Gamma, \withtype{x}{\natinterval{I-1}{J-1}} \vdash \withcomplex{Q}{K'}
%         \end{matrix}}{\varphi;\Phi;\Gamma \vdash \withcomplex{\match{e}{P}{x}{Q}}{L}}{\text{where}\quad L = \left\{
% \begin{matrix}
%     K & \text{if}\; \varphi;\Phi\vDash K' \leq K   \\
%     K' & \text{if}\; \varphi;\Phi\vDash K \leq K'  \\
%     K+K' & \text{otherwise}
% \end{matrix}
% \right.}\\[-1em]
%         %
%         %&\kern-0em\runa{S-nmatch-2}\;\infrule{
%         %\begin{matrix}
%         %    \varphi;\Phi;\Gamma \vdash \withtype{e}{\natinterval{I}{J}} \quad \varphi;\Phi\vDash K \leq K' \\
%         %    \varphi;\Phi, I \leq 0;\Gamma \vdash \withcomplex{P}{K} \quad \varphi;\Phi, J \geq 1;\Gamma, \withtype{x}{\natinterval{I-1}{J-1}} \vdash \withcomplex{Q}{K'}
%         %\end{matrix}}{\varphi;\Phi;\Gamma \vdash \withcomplex{\match{e}{P}{x}{Q}}{K'}}\\[-1em]
%         %
%         &\kern-0em\runa{S-lmatch}\;\condinfrule{
%         \begin{matrix}
%             \varphi;\Phi;\Gamma \vdash \withtype{e}{\texttt{List}[I,J](\mathcal{B})} \quad \varphi;\Phi, I \leq 0;\Gamma \vdash \withcomplex{P}{K} \\
%             \varphi;\Phi, J \geq 1;\Gamma, \withtype{x}{\mathcal{B}},y : \texttt{List}[I-1,J-1](\mathcal{B}) \vdash \withcomplex{Q}{K'}
%         \end{matrix}}{\varphi;\Phi;\Gamma \vdash \withcomplex{\texttt{match}\;e\;\{ [] \mapsto P;\; x :: y \mapsto Q \}}{L}}{\text{where}\quad L = \left\{
% \begin{matrix}
%     K & \text{if}\; \varphi;\Phi\vDash K' \leq K   \\
%     K' & \text{if}\; \varphi;\Phi\vDash K \leq K'  \\
%     K+K' & \text{otherwise}
% \end{matrix}
% \right.}\\[-1em]
%         %
%         %&\kern-0em\runa{S-lmatch-2}\;\infrule{
%         %\begin{matrix}
%         %    \varphi;\Phi;\Gamma \vdash \withtype{e}{\texttt{List}[I,J](\mathcal{B})} \quad \varphi;\Phi\vDash K \leq K' \\
%         %    \varphi;\Phi, I \leq 0;\Gamma \vdash \withcomplex{P}{K} \quad \varphi;\Phi, J \geq 1;\Gamma, \withtype{x}{\mathcal{B}},y : \texttt{List}[I-1,J-1](\mathcal{B}) \vdash \withcomplex{Q}{K'}
%       % \end{matrix}}{\varphi;\Phi;\Gamma \vdash \withcomplex{\texttt{match}\;e\;\{ [] \mapsto P;\; x :: y \mapsto Q \}}{K'}}\\[-1em]
%         %
%         &\kern4em\runa{S-par}\;\condinfrule{\varphi;\Phi;\Gamma\vdash P \triangleleft K\quad \varphi;\Phi;\Gamma\vdash Q \triangleleft K'}{\varphi;\Phi;\Gamma\vdash \parcomp{P}{Q} \triangleleft L}{\text{where}\quad L = \left\{
% \begin{matrix}
%     K & \text{if}\; \varphi;\Phi\vDash K' \leq K   \\
%     K' & \text{if}\; \varphi;\Phi\vDash K \leq K'  %\\
%     %K+K' & \text{otherwise}
% \end{matrix}
% \right.}\\[-1em]
%         %
%         %&\kern4em\runa{S-par-2}\;\infrule{\varphi;\Phi;\Gamma\vdash P \triangleleft K\quad \varphi;\Phi;\Gamma\vdash Q \triangleleft K'\quad \varphi;\Phi\vDash K \leq K'}{\varphi;\Phi;\Gamma\vdash \parcomp{P}{Q} \triangleleft K'}\\[-1em]
%         %
%         &\kern-0em\runa{S-iserv}\;\infrule{\texttt{in}\in\sigma\quad \varphi,\widetilde{i};\Phi;\text{ready}(\varphi,\Phi,\susumesim{\Gamma}{I}),a:\forall_0\widetilde{i}.\texttt{serv}^{\sigma\cap\{\texttt{out}\}}_K(\widetilde{T}),\widetilde{v} : \widetilde{T}\vdash P \triangleleft K'\quad \varphi,\widetilde{i};\Phi\vDash K' \leq K}{\varphi;\Phi;\Gamma,a:\forall_I\widetilde{i}.\texttt{serv}^\sigma_K(\widetilde{T})\vdash\; \bang\inputch{a}{\widetilde{v}}{}{P}\triangleleft I}\\[-1em]
%         %
%         &\kern-0em\runa{S-ich}\;\infrule{\texttt{in}\in\sigma\quad \varphi;\Phi;\susumesim{\Gamma}{I},a:\texttt{ch}_0^\sigma(\widetilde{T}),\widetilde{v} : \widetilde{T}\vdash P \triangleleft K}{\varphi;\Phi;\Gamma,a:\texttt{ch}_I^\sigma(\widetilde{T})\vdash \inputch{a}{\widetilde{v}}{}{P}\triangleleft K + I}
%         %
%         \kern8.5em \runa{S-och}\;\infrule{\texttt{out}\in \sigma\quad \varphi;\Phi;\susumesim{\Gamma}{I}\vdash \widetilde{e} : \widetilde{T}\quad \varphi;\Phi\vdash\widetilde{T}\sqsubseteq\widetilde{S}}{\varphi;\Phi;\Gamma,a:\texttt{ch}^{\sigma}_I(\widetilde{S})\vdash \asyncoutputch{a}{\widetilde{e}}{} \triangleleft I}\\[-1em]
%         %
%         &\kern0em\runa{S-oserv}\;\infrule{\texttt{out} \in \sigma \quad \varphi;\Phi;\susumesim{\Gamma}{I}\vdash \widetilde{e} : \widetilde{T}\quad \text{instantiate}(\widetilde{i},\widetilde{T})=\{\widetilde{J}/\widetilde{i}\}\quad  \varphi;\Phi\vdash\widetilde{T}\sqsubseteq\widetilde{S}\{\widetilde{J}/\widetilde{i}\}}{\varphi;\Phi;\Gamma,a:\forall_I\widetilde{i}.\texttt{serv}_K^\sigma(\widetilde{S})\vdash \asyncoutputch{a}{\widetilde{e}}{} \triangleleft K\!\substi{\widetilde{J}}{\widetilde{i}} + I}
%         %
%     \end{align*}\vspace{-1em}\end{framed}
%     \smallskip
%     \caption{Sized typing rules for parallel complexity of processes.}
%     \label{tab:sizedprocesstypingrules}
% \end{table*}

\begin{table*}[!ht]
    \begin{framed}\vspace{-1em}\begin{align*}
        %
        % S-nil
        &\runa{S-nu}\infrule{\varphi;\Phi;\Gamma, a:T \vdash P \triangleleft \kappa}{\varphi;\Phi;\Gamma \vdash \newvar{a:T}{P} \triangleleft \kappa}
        % S-par
        \kern1em\runa{S-par}\infrule{\varphi;\Phi;\Gamma \vdash P \triangleleft \kappa \quad \varphi;\Phi;\Gamma \vdash Q \triangleleft \kappa'}{\varphi;\Phi;\Gamma \vdash P \mid Q \triangleleft \text{basis}(\varphi, \Phi,\kappa \cup \kappa')}\\[-1em]
        %
        &\runa{S-tick}\infrule{\varphi;\Phi;\tforwardsim{\Gamma}{1} \vdash P \triangleleft \kappa}{\varphi;\Phi;\Gamma \vdash \tick P \triangleleft \kappa + 1}\kern2em
        %
        \runa{S-annot}\infrule{\varphi;\Phi;\tforwardsim{\Gamma}{n}\vdash P \triangleleft \kappa}{\varphi;\Phi;\Gamma\vdash n:P \triangleleft \kappa + n}\\[-1em]
        % S-match
        &\runa{S-match}\infrule{
        \begin{matrix}
            \varphi;\Phi;\Gamma \vdash e:\natinterval{I}{J} \quad \varphi;\Phi, I \leq 0;\Gamma \vdash P \triangleleft \kappa\\
            \varphi;\Phi, J \geq 1;\Gamma, x:\natinterval{I-1}{J-1} \vdash Q \triangleleft \kappa'
        \end{matrix}}{\varphi;\Phi;\Gamma \vdash \match{e}{P}{x}{Q} \triangleleft \text{basis}(\varphi, \Phi, \kappa \cup \kappa')}\\[-1em]
        % S-iserv
        &\runa{S-iserv}\infrule{\begin{matrix}
            \texttt{in} \in \sigma\quad \varphi;\Phi;\Gamma\vdash a:\servt{I}{i}{\sigma}{K}{\widetilde{T}}\\
            \varphi, \widetilde{i}; \Phi; \text{ready}(\varphi,\Phi,\tforwardsim{\Gamma}{I}), \widetilde{v} : \widetilde{T} \vdash P \triangleleft \kappa \quad \varphi,\widetilde{i};\Phi\vDash\kappa \leq K
        \end{matrix}}
        {\varphi;\Phi;\Gamma \vdash \;\bang\inputch{a}{\widetilde{v}}{}{P}\triangleleft \{I\}}
        %
        \kern14em\runa{S-nil}\kern-1em\infrule{}{\varphi;\Phi;\Gamma \vdash \nil \triangleleft \{0\}}\kern-3em\text{ }\\[-1em]
        % S-oserv
        &\runa{S-oserv}\infrule{\begin{matrix}
            \texttt{out} \in \sigma\quad \varphi;\Phi;\Gamma\vdash a:\servt{I}{i}{\sigma}{K}{\widetilde{T}}\\
            \varphi; \Phi;\tforwardsim{\Gamma}{I} \vdash \widetilde{e}:\widetilde{S} \quad \text{instantiate}(\widetilde{i}, \widetilde{S}) = \{\widetilde{J}/\widetilde{i}\} \quad \varphi;\Phi \vDash \widetilde{S} \sqsubseteq \widetilde{T}
        \end{matrix}}
        {\varphi;\Phi;\Gamma \vdash \asyncoutputch{a}{\widetilde{e}}{}\triangleleft \{K\{\widetilde{J}/\widetilde{i}\} + I\}}\\[-1em]
        % S-annot
        &\runa{S-ich}\infrule{\begin{matrix}
            \texttt{in} \in \sigma\quad \varphi;\Phi;\Gamma \vdash a:\chant{\sigma}{I}{\widetilde{T}}\\
            \varphi; \Phi; \tforwardsim{\Gamma}{I}, \widetilde{v}:\widetilde{T} \vdash P \triangleleft \kappa
        \end{matrix}}
        {\varphi;\Phi;\Gamma \vdash \inputch{a}{\widetilde{v}}{}{P} \triangleleft \kappa + I}\kern3em
        %
        \runa{S-och}\infrule{\begin{matrix}
            \texttt{out} \in \sigma\quad \varphi;\Phi;\Gamma \vdash a:\chant{\sigma}{I}{\widetilde{T}}\\
            \varphi; \Phi; \tforwardsim{\Gamma}{I} \vdash \widetilde{e}:\widetilde{S} \quad \varphi;\Phi \vDash \widetilde{S} \sqsubseteq \widetilde{T}
        \end{matrix}}
        {\varphi;\Phi;\Gamma \vdash \asyncoutputch{a}{\widetilde{e}}{} \triangleleft \{I\}}\\[-1em]
    \end{align*}\vspace{-1em}\end{framed}
    \smallskip
    \caption{Sized typing rules for parallel complexity of processes.}
    \label{tab:sizedprocesstypingrules}
\end{table*}

%
Rule $\runa{S-iserv}$ types a replicated input on a name $a$, and so $a$ must be bound to a server type with input capability. As the index $I$ in the server type denotes the time steps remaining before the server is ready to synchronize, we advance the time by $I$ units of time complexity when typing the continuation $P$. To ensure that bounds on synchronizations in $\downarrow^{\varphi;\Phi}_I\!\Gamma$ are not violated, we type $P$ under the time invariant part of $\downarrow^{\varphi;\Phi}_I\!\Gamma$, i.e. $\text{ready}(\varphi,\Phi,\downarrow_I\!\Gamma)$. Note that the bound on the span of the replicated input is the bound on the time remaining before the server is ready to synchronize. As the replicated input may be invoked many times, the cost of invoking the server is accounted for in rule $\runa{S-oserv}$ using the complexity bound $K$ in the server type. Therefore, we enforce that $K$ is in fact an upper bound on the span of the continuation $P$.\\

The rule $\runa{S-oserv}$ types outputs on names bound to server types. Here, as stated above, we must account for the cost of invoking a server, and as a replicated input on a server is parametric, we must \textit{instantiate} it based on the types of the expressions we are to output. Recall that in the type rule for outputs on servers from Chapter \ref{ch:bgts}, this is to be done by finding a substitution that satisfies the premise $\widetilde{T} \sqsubseteq \widetilde{S}\{\widetilde{J}/\widetilde{i}\}$. However, this turns out to be a difficult problem, and we can in fact prove it NP-complete for types of polynomial indices even if we disregard subtyping. However, note that it might not be necessary to use the full expressive power of polynomial indices, and so this may not necessarily affect type checking. Nevertheless, we over-approximate finding such a substitution, by using the function $\textit{instantiate}$. That is, we \textit{zip} together the index variables $\widetilde{i}$ with indices in types $\widetilde{T}$. Remark that Baillot and Ghyselen \cite{BaillotGhyselen2021} propose types for inference in their technical report, where the problem is simplified substantially, by forcing naturals to have lower bounds of $0$ and upper bounds with exactly one index variable and a constant. Our approach admits more expressive lower bounds and multiplications, while imposing no direct restrictions on the number of index variables in an index, and is thus more suitable for a type-checker.\\

We now prove the NP-completeness of the smaller problem of checking whether there exists a substitution $\{\widetilde{J}/\widetilde{i}\}$ that satisfies $T = S\{\widetilde{J}/\widetilde{i}\}$ where $T$ and $S$ are types with polynomial indices. The main idea is a reduction proof from the NP-complete 3-SAT problem, i.e. the satisfiability problem of a boolean formula in conjunctive normal form with exactly three literals in each clause \cite{Karp1972}. We first define a translation from a 3-SAT formula to a polynomial index in Definition \ref{def:3satredu}. This is a polynomial time computable reduction, as we simply replace each logical-and with a multiplication, each logical-or with an addition and each negation with a subtraction from 1. In Lemma \ref{lemma:soundtranslation}, we prove that the reduction is faithful with respect to satisfiability of a boolean formula. Finally, in Lemma \ref{lemma:npcompletesubst}, we prove that it is an NP-complete decision problem to verify the existence of a substitution that satisfies $T = S\{\widetilde{J}/\widetilde{i}\}$ for types $T$ and $S$.
%
\begin{defi}[3-SAT reduction]\label{def:3satredu}
We assume a one-to-one mapping $f$ from unknowns to index variables. Let $\phi$ be a 3-SAT formula
\begin{align*}
    \phi = \bigwedge_{i=1}^n \left(\ell_{i1} \lor \ell_{i2} \lor \ell_{i3}\right)% \land \cdots \land (A_n \lor B_n \lor C_n)
\end{align*}
where $\ell_{i1}$, $\ell_{i2}$ and $\ell_{i3}$ are of the forms $x$ or $\neg x$ for some variable $x$. We define a translation of $\phi$ to a polynomial index %$[\![\phi]\!]_{\text{3-SAT}}$
\begin{align*}
    [\![\phi]\!]_{\text{3-SAT}} = \prod_{i=1}^n \left([\![\ell_{i1}]\!]_{\text{3-SAT}} + [\![\ell_{i2}]\!]_{\text{3-SAT}} + [\![\ell_{i3}]\!]_{\text{3-SAT}}\right) %\cdots ([\![A_n]\!]_{\text{3-SAT}} + [\![B_n]\!]_{\text{3-SAT}} + [\![C_n]\!]_{\text{3-SAT}})
\end{align*}
where $[\![x]\!]_{\text{3-SAT}} = f(x)$ and $[\![\neg x]\!]_{\text{3-SAT}} = (1 - f(x))$.
\end{defi}


\begin{lemma}\label{lemma:soundtranslation}
Let $\phi$ be a 3-SAT formula. Then $\phi$ is satisfiable if and only if there exists a substitution $\{\widetilde{n}/\widetilde{i}\}$ such that $1\leq [\![\phi]\!]_{\text{3-SAT}}\{\widetilde{n}/\widetilde{i}\}$.
\begin{proof}
We consider the implications separately
\begin{enumerate}
    \item Assume that $\phi$ is satisfiable. Then there exists a truth assignment $\tau$ such that each clause of $\phi$ is true. Correspondingly, as $[\![\phi]\!]_{\text{3-SAT}}$ is a product of non-negative factors, we for some substitution $\{\widetilde{n}/\widetilde{i}\}$ have that $1 \leq [\![\phi]\!]_{\text{3-SAT}}\{\widetilde{n}/\widetilde{i}\}$ if and only if each factor in the product is positive. We compare the conditions for a clause to be true in $\phi$ to those for a corresponding factor in $[\![\phi]\!]_{\text{3-SAT}}$ to be positive, and show that a substitution $\{\widetilde{n}/\widetilde{i}\}$ exists such that $1 \leq [\![\phi]\!]_{\text{3-SAT}}\{\widetilde{n}/\widetilde{i}\}$. A clause in $\phi$ is a disjunction of three literals of either the form $x$ or $\neg x$ for some unknown $x$. Thus, for a clause to be true, we must have at least one literal $\tau(x) = tt$ or $\neg \tau(x) = tt$ with $\tau(x) = f\!f$. The corresponding factor in $[\![\phi]\!]_{\text{3-SAT}}$ is a sum of three terms of the forms $f(x)$ or $(1 - f(x))$ for some unknown $x$, where $f$ is a one-to-one mapping from unknowns to index variables. Here, we utilize that in the type system by Baillot and Ghyselen \cite{BaillotGhyselen2021}, we have $(1 - i\{\widetilde{n}/\widetilde{i}\}) = 0$ when $i\{\widetilde{n}/\widetilde{i}\} \geq 1$ and $(1 - i\{\widetilde{n}/\widetilde{i}\}) = 1$ when $i\{\widetilde{n}/\widetilde{i}\} = 0$. Thus, for a factor to be positive, it suffices that one term is positive, and so we can construct a substitution that guarantees this from the interpretation of $\phi$. That is, if $\tau(x) = tt$, we substitute $1$ for $f(x)$, and if $\tau(x) = f\!f$, we substitute 0 for $f(x)$. Then, whenever a literal is true in $\phi$, the corresponding term in $[\![\phi]\!]_{\text{3-SAT}}$ is positive, and so if $\phi$ is satisfiable then there exists a substitution $\{\widetilde{n}/\widetilde{i}\}$ such that $1 \leq [\![\phi]\!]_{\text{3-SAT}}\{\widetilde{n}/\widetilde{i}\}$.
     
    \item Assume that there exists a substitution $\{\widetilde{n}/\widetilde{i}\}$ such that $1 \leq [\![\phi]\!]_{\text{3-SAT}}\{\widetilde{n}/\widetilde{i}\}$. Then, as $[\![\phi]\!]_{\text{3-SAT}}$ is a product of non-negative factors, each factor must be positive. Correspondingly, if $\Phi$ is satisfiable, then there exists a truth assignment such that each clause of $\phi$ is true. We compare the conditions for a factor in $[\![\phi]\!]_{\text{3-SAT}}\{\widetilde{n}/\widetilde{i}\}$ to be positive to those for a corresponding clause in $\phi$ to be true, and show that $\phi$ is satisfiable. A factor in $[\![\phi]\!]_{\text{3-SAT}}$ is a sum of at most three terms of the forms $f(x)\{\widetilde{n}/\widetilde{i}\}$ or $(1 - f(x)\{\widetilde{n}/\widetilde{i}\})$. Here we again utilize that in the type system by Baillot and Ghyselen \cite{BaillotGhyselen2021}, we have $(1 - f(x)\{\widetilde{n}/\widetilde{i}\}) = 0$ when $f(x)\{\widetilde{n}/\widetilde{i}\} \geq 1$ and $(1 - f(x)\{\widetilde{n}/\widetilde{i}\}) = 1$ when $f(x)\{\widetilde{n}/\widetilde{i}\} = 0$, and so it must be that in the factor, we have at least one term $f(x)\{\widetilde{n}/\widetilde{i}\} \geq 1$ or $(1 - f(x)\{\widetilde{n}/\widetilde{i}\}) \geq 1$. Correspondingly, for the clause in $\phi$ to be true, at least one literal must be true. We show that there exists a truth assignment $\tau$ such that if a term in $[\![\phi]\!]_{\text{3-SAT}}$ is positive, then the corresponding literal in $\phi$ is true. If $f(x)\{\widetilde{n}/\widetilde{i}\}\geq 1$ then we set $\tau(x) = tt$, and if $f(x)\{\widetilde{n}/\widetilde{i}\} = 0$ we set $\tau(x) = f\!f$, as $[\![x]\!]_{\text{3-SAT}}\{\widetilde{n}/\widetilde{i}\} \geq 1$ when $f(x)\{\widetilde{n}/\widetilde{i}\} \geq 1$ and $[\![\neg x]\!]_{\text{3-SAT}} \geq 1$ when $f(x)\{\widetilde{n}/\widetilde{i}\}=0$. Then, whenever a term is positive in $[\![\phi]\!]_{\text{3-SAT}}\{\widetilde{n}/\widetilde{i}\}$, the corresponding literal in $\phi$ is true, and so if there exists a substitution $\{\widetilde{n}/\widetilde{i}\}$ such that $1 \leq [\![\phi]\!]_{\text{3-SAT}}\{\widetilde{n}/\widetilde{i}\}$, then $\phi$ is satisfiable.
    
\end{enumerate}
\end{proof}
\end{lemma}


\begin{lemma}\label{lemma:npcompletesubst}
Let $T$ and $S$ be types with polynomial indices. Then checking whether there exists a substitution $\{\widetilde{J}/\widetilde{i}\}$ such that $T = S\{\widetilde{J}/\widetilde{i}\}$ is an NP-complete problem.
\begin{proof}
By reduction from the 3-SAT problem. Assume that we have some algorithm that can verify the existence of a substitution $\{\widetilde{J}/\widetilde{i}\}$ such that $T = S\{\widetilde{J}/\widetilde{i}\}$, and let $\phi$ be a 3-SAT formula. Then using the algorithm, we can check whether $\phi$ is satisfiable by verifying whether there exists $\{\widetilde{J}/\widetilde{i}\}$ such that the following holds
\begin{align*}
    \texttt{Nat}[0,1] = \texttt{Nat}[0,(1 - (1 - [\![\phi]\!]_{\text{3-SAT}}))]\{\widetilde{J}/\widetilde{i}\}
\end{align*}
That is, $1 = (1 - (1 - [\![\phi]\!]_{\text{3-SAT}}\{\widetilde{J}/\widetilde{i}\}))$ implies $1 \leq [\![\phi]\!]_{\text{3-SAT}}\{\widetilde{J}/\widetilde{i}\}$, as $(1 - [\![\phi]\!]_{\text{3-SAT}}\{\widetilde{J}/\widetilde{i}\}) = 0$ when $[\![\phi]\!]_{\text{3-SAT}}\{\widetilde{J}/\widetilde{i}\} \geq 1$ and $(1 - [\![\phi]\!]_{\text{3-SAT}}\{\widetilde{J}/\widetilde{i}\}) = 1$ when $[\![\phi]\!]_{\text{3-SAT}}\{\widetilde{J}/\widetilde{i}\} = 0$. Furthermore, for $1 \leq [\![\phi]\!]_{\text{3-SAT}}\{\widetilde{J}/\widetilde{i}\}$ to hold, the indices in the sequence $\widetilde{J}$ cannot contain index variables, and so there must exist an equivalent substitution of naturals for index variables $\{\widetilde{n}/\widetilde{i}\}$. Then, by Lemma \ref{lemma:soundtranslation} we have that $\phi$ is satisfiable if and only if there exists a substitution $\{\widetilde{n}/\widetilde{i}\}$ such that $1\leq [\![\phi]\!]_{\text{3-SAT}}\{\widetilde{n}/\widetilde{i}\}$. Thus, as 3-SAT is an NP-complete problem, the reduction from 3-SAT is computable in polynomial time and as polynomial reduction is a transitive relation, i.e. any NP-problem is polynomial time reducible to verifying the existence of a substitution $\{\widetilde{J}/\widetilde{i}\}$ that satisfies the equation $T = S\{\widetilde{J}/\widetilde{i}\}$, it follows that the problem is NP-hard. To show that it is an NP-complete problem, we show that a \textit{certificate} can be verified in polynomial time. That is, given some substitution $\{\widetilde{J}/\widetilde{i}\}$, we can in linear time check whether $T=S\{\widetilde{J}/\widetilde{i}\}$ by substituting indices $\widetilde{J}$ for indices $\widetilde{i}$ in type $S$ and by then comparing the two types.\\
%
%
%Utilizing that $n - m = 0$ for $m\geq n$ in the type system of Baillot and Ghyselen \cite{BaillotGhyselen2021}, we can simulate any boolean formula using a polynomial index. By denoting $J = 0$ false and $I > 0$ true, we have the translation
% \begin{align*}
%     [\![a \land b]\!]_\phi =&\; [\![a]\!]_\phi [\![b]\!]_\phi\\
%     [\![a \lor b]\!]_\phi =&\; [\![a]\!]_\phi + [\![b]\!]_\phi\\
%     [\![\neg a]\!]_\phi =&\; (1 - [\![a]\!]_\phi)\\
%     [\![x]\!]_\phi =&\; i
% \end{align*}
% Then assuming some algorithm that checks whether there exists a substitution $\{\widetilde{J}/\widetilde{i}\}$ such that $T \sqsubseteq S\{\widetilde{J}/\widetilde{i}\}$, we can solve the boolean satisfiability problem. Let $\phi_0$ be any boolean formula and let $\widetilde{i}$ be the index variables in $[\![\phi_0]\!]_\phi$, and assume that there exists a substitution $\{\widetilde{J}/\widetilde{i}\}$ that satisfies the judgement
% \begin{align*}
%     \emptyset;\emptyset\vDash\texttt{Nat}[0,1] \sqsubseteq \texttt{Nat}[0,[\![\phi_0]\!]_\phi]\{\widetilde{J}/\widetilde{i}\}
% \end{align*}
% Then by rule $\runa{SS-nweak}$ we have that $\emptyset;\emptyset\vDash 1 \leq [\![\phi_0]\!]_\phi\{\widetilde{J}/\widetilde{i}\}$, and as $\varphi = \emptyset$, the indices $\widetilde{J}$ must be constants. Thus, $\emptyset;\emptyset\vDash 1 \leq [\![\phi_0]\!]_\phi\{\widetilde{J}/\widetilde{i}\}$ is equivalent to $1 \leq [\![\phi_0]\!]_\phi\{\widetilde{J}/\widetilde{i}\}$, and so $\phi_0$ must have a solution. If instead no such substitution exists, then for any $\{\widetilde{J}/\widetilde{i}\}$, it must be that $[\![\phi_0]\!]_\phi\{\widetilde{J}/\widetilde{i}\} = 0$ implying that $\phi_0$ is a contradiction. Therefore, as the boolean satisfiability problem is NP-complete, the algorithm we assumed must be NP-complete as well.
\end{proof}
\end{lemma}

In Example \ref{example:addition}, we show how a process implementing addition of naturals can be typed using our type rules, yielding a precise bound on the parallel complexity.
%

\begin{examp}\label{example:addition}
As an example of a process that is typable using our type rules, we show how the addition operator for naturals can be written as a process and subsequently be typed. We use a server to encode the addition operator
\begin{align*}
    !\inputch{\text{add}}{x,y,r}{}{\match{x}{\asyncoutputch{r}{y}{}}{z}{\tick{\asyncoutputch{\text{add}}{z,\succc y,r}{}}}}
\end{align*}
such that channel $r$ is used to output the addition of naturals $x$ and $y$. To type the process, we use the following contexts and set of index variables
\begin{align*}
    \Gamma\defeq&\; \text{add} : \forall_0 i,j,k,l,m,n,o.\texttt{serv}^{\{\texttt{in},\texttt{out}\}}_j(\texttt{Nat}[0,j],\texttt{Nat}[0,l],\texttt{ch}^{\{\texttt{out}\}}_j(\texttt{Nat}[0,j+l])) \\
    \Delta\defeq&\; \text{ready}(\cdot,\cdot,\Gamma), x : \texttt{Nat}[0,j], y: \texttt{Nat}[0,l], r:\texttt{ch}^{\{\texttt{out}\}}_j(\texttt{Nat}[0,j+l])\\
    \varphi \defeq&\; \{i,j,k,l,m,n,o\}
\end{align*}
%
We now derive a type for the encoding of the addition operator, yielding a precise bound of $j$, corresponding to an upper bound on the size of $x$, as we pattern match at most $j$ times on natural $x$. Notably we have that $\text{instantiate}((i,j,k,l,m,n,o),\texttt{Nat}[0,j\monus 1],\texttt{Nat}[1,l+1],\texttt{ch}^{\{\texttt{out}\}}_j(\texttt{Nat}[0,j+l]))=\{0/i,j\monus 1/j,0/k,l+1/l,j/m,0/n,j+l/o\}$.
%
{\small
\begin{align*}
    \begin{prooftree}
        %
        \infer0{\varphi;\cdot,0\leq 0;\Delta\vdash \asyncoutputch{r}{y}{} \triangleleft \{j\}}
        %
        % \infer0{\texttt{Nat}[0,j\monus 1] \sqsubseteq \texttt{Nat}[0,j]\{j\monus 1/j\}}
        % %
        % \infer0{\texttt{Nat}[0,l+1] \sqsubseteq \texttt{Nat}[0,l]\{l+1/l\}}
        % %
        % \infer0{\texttt{ch}^{\{\texttt{out}\}}_{j\monus 1}(\texttt{Nat}[0,j+l] \sqsubseteq \texttt{ch}^{\{\texttt{out}\}}_{j\monus 1}(\texttt{Nat}[0,j+l)\{j\monus 1/j,l+1/l\}}
        %
        \infer0{
        \begin{matrix}
        \varphi;\cdot,1\leq j\vdash\texttt{Nat}[0,j\monus 1] \sqsubseteq \texttt{Nat}[0,j]\{j\monus 1/j\}\\
        \varphi;\cdot,1\leq j\vdash\texttt{Nat}[1,l+1] \sqsubseteq \texttt{Nat}[0,l]\{l+1/l\}\\
        \varphi;\cdot,1\leq j\vdash\texttt{ch}^{\{\texttt{out}\}}_{j\monus 1}(\texttt{Nat}[0,j+l] \sqsubseteq \texttt{ch}^{\{\texttt{out}\}}_{j\monus 1}(\texttt{Nat}[0,j+l)\{j\monus 1/j,l+1/l\}
        \end{matrix}
        }
        %
        \infer1{\varphi;\cdot,1\leq j;\susumesim{\Delta}{1},z : \texttt{Nat}[0,j\monus 1]\vdash \asyncoutputch{\text{add}}{z,\succc y, r}{} \triangleleft \{j\monus 1\}}
        %
        \infer1{\varphi;\cdot,1\leq j;\Delta,z : \texttt{Nat}[0,j\monus 1]\vdash \tick{\asyncoutputch{\text{add}}{z,\succc y, r}{}} \triangleleft \{j\}}
        %
        \infer2{\varphi;\cdot;\Delta\vdash \match{x}{\asyncoutputch{r}{y}{}}{z}{\tick{\asyncoutputch{\text{add}}{z,\succc y,r}{}}} \triangleleft \{j\}}
        %
        \infer1{\cdot;\cdot;\Gamma\vdash\; !\inputch{\text{add}}{x,y,r}{}{\match{x}{\asyncoutputch{r}{y}{}}{z}{\tick{\asyncoutputch{\text{add}}{z,\succc y,r}{}}}}\triangleleft \{0\}}
    \end{prooftree}
\end{align*}}
%
\end{examp}

% \subsection{Undecidability of judgements}
% Verifying whether a polynomial constraint with integer coefficients imposes further restrictions onto the model set of index valuations of natural codomain of some set of known constraints can be reduced to Hilbert's tenth problem \cite{Davis1973}. That is, the problem of verifying whether a diophantine equation has an integer solution.\\

% We first assume some algorithm that can verify a judgement of the form $\varphi;\Phi\vDash C$ where $\varphi$ is a set of index variables and $C$ and $C'\in\Phi$ are binary constraints on polynomials of integer coefficients over relations from any subset of $\{\neq,\leq, <\}$. Recall that such a judgement holds exactly when for each index valuation $\rho : \varphi \longrightarrow \mathbb{N}$ over $\varphi$ for which $\rho \vDash C'$ for $C'\in\Phi$ we also have $\rho\vDash C$, i.e. $C$ does not impose further restrictions on interpretations of indices.\\

% We can then verify whether any diophantine equation has an integer solution. Let $p$ be an arbitrary polynomial of integer coefficients such that $p = 0$ is a diophantine equation. As only non-negative integers substitute for index variables, we first transform $p = 0$ to a new diophantine equation $p' = 0$ that has a non-negative integer solution exactly when $p = 0$ has an integer solution. To do this, we simply replace each index variable $i$ in $p$ with two new index variables $i_1 - i_2$. Then the judgement $\varphi;\emptyset\vDash p' \neq 0$ holds exactly when $p=0$ has no integer solution. That is, if $p=0$ has an integer solution, then there must exist a valuation $\rho_0$ such that $\rho_0\vDash \emptyset$ with $[\![p']\!]_{\rho_0} = 0$ and so $\rho_0\nvDash p' \neq 0$. Moreover, we need not rely on the relation $\neq$, as the judgements below are equivalent
% \begin{align*}
%     \varphi;\{p' \leq 0\} \vDash p' < 0\\
%     \varphi;\{p' \leq 0\} \vDash p' \leq 1
% \end{align*}
\section{Soundness}
\section{Verification of constraint judgements}\label{sec:verifyinglinearjudgements}
Until now we have not considered how we can verify constraint judgements in the type rules. The expressiveness of implementations of the type system by Baillot and Ghyselen \cite{BaillotGhyselen2021} depends on both the expressiveness of indices and whether judgements on the corresponding constraints are decidable. Naturally, we are interested in both of these properties, and so in this section, we show how judgements on linear constraints can be verified using algorithms. Later, we show how this can be extended to certain groups of polynomial constraints. We first make some needed changes to how the type checker uses subtraction.
%
\subsection{Subtraction of naturals}
The constraint judgements rely on a special minus operator ($\monus$) for indices such that $n \monus m=0$ when $m \geq n$, which we refer to as the \textit{monus} operator. This is apparent in the pattern match constructor type rule from Chapter \ref{ch:bgts}. Without this behavior, we may encounter problems when checking subtype premises in match processes. This has the consequence that equations such as $2\monus 3+3=3$ hold, such that indices form a semiring rather than a ring, as we are no longer guaranteed an additive inverse. In general, semirings lack many properties of rings that are desirable. For example, given two seemingly equivalent constraints $i \leq 5$ and $i \monus 5 \leq 0$, we see that by adding any constant to their left-hand sides, the constraints are no longer equivalent. Adding the constant 2 to their left-hand sides, we obtain $i + 2 \leq 5$ and $i \monus 5 + 2 \leq 0$, however, we see that the first constraint is satisfied given the valuation $i = 3$ but the second is not. In general the associative property of $+$ is lost.\\

Unfortunately, this is not an easy problem to solve implementation-wise, as indices are not actually evaluated but rather represent whole feasible regions. Thus, instead of trying to implement this operator exactly, we limit the number of processes typable by the type system. Removing the operator entirely is not an option as it us used by the type rules themselves. Instead, we ensure that one cannot \textit{exploit} the special behavior of monus by introducing additional conditions to the type rules of the type system. More precisely, any time the type system uses the monus operator such as $I \monus J$, we require the premise $\varphi;\Phi \vDash I \geq J$, in which case the monus operator is safe to treat as a regular minus. This, however, puts severe restrictions on the number of processes typable, and so we relax the restriction a bit by also checking the judgement $\varphi;\Phi \vDash I \leq J$, in which case we can conclude that the result is definitely $0$. If neither $\varphi;\Phi\vDash I \geq J$ nor $\varphi;\Phi\vDash I \leq 0$ hold, which is possible as $\leq$ and $\geq$ do not form a total order on indices, the result is undefined. We refer to this variant of monus as the \textit{partial} monus operator, as formalized in Definition \ref{def:partialmonus}. Note that this definition of monus allows us to obtain identical behavior to minus on a constraint $I \bowtie J$ by moving terms between the LHS and RHS, i.e. $I - K \bowtie J \Rightarrow I \bowtie J + K$, and so we can assume we have a standard minus operator when verifying judgements on constraints. For the remainder of this section, we assume this definition is used in the type rules instead of the usual monus. We may omit $\varphi;\Phi$ if it is clear from the context.%\\
%
%Definition \ref{def:partialmonus} defines the \textit{partial} monus operator that is undefined if we cannot determine if the result is either always positive or always zero. For the remainder of this thesis, we assume this definition is used in the type rules instead of the usual monus. We may omit $\varphi;\Phi$ if it is clear from the context.
%
\begin{defi}[Partial monus]\label{def:partialmonus}
Let $\Phi$ be a set of constraints in index variables $\varphi$. The partial monus operator is defined for two indices $I$ and $J$ as
\begin{equation*}
    I \monusE J = \begin{cases}
    I - J &\text{if $\varphi;\Phi \vDash J \leq I$}\\
    0 &\text{if $\varphi;\Phi \vDash I \leq J$}\\
    \textit{undefined} & \textit{otherwise}
    \end{cases}
\end{equation*}
\end{defi}

To ensure soundness of the algorithmic type rules after switching to the partial monus operator, we must make some changes to advancement of time. Consider the typing
\begin{align*}
    (\cdot,i);(\cdot,i\leq 3);\Gamma\vdash\; !\inputch{a}{}{}{\nil}  \mid 5 : \asyncoutputch{a}{}{} \triangleleft \{5\}
\end{align*}
where $\Gamma = \cdot,a : \forall_{3-i}\epsilon.\texttt{serv}^{\{\texttt{in},\texttt{out}\}}_0()$. Upon typing the time annotation, we advance the time of the server type by $5$ yielding the type $\forall_{3-i-5}\epsilon.\texttt{serv}^{\{\texttt{out}\}}_0()$ as $(\cdot,i);(\cdot,i\leq 3)\nvDash 3-i \geq 5$, which is defined as $(\cdot,i);(\cdot,i\leq 3)\vDash 3-i \leq 5$. However, if we apply the congruence rule $\runa{SC-sum}$ from right to left we obtain
\begin{align*}
    !\inputch{a}{}{}{\nil}  \mid 2 : 3 : \asyncoutputch{a}{}{}\equiv\;!\inputch{a}{}{}{\nil}  \mid 5 : \asyncoutputch{a}{}{}
\end{align*}
Then, we get a problem upon typing the first annotation. That is, as $(\cdot,i);(\cdot,i\leq 3)\nvDash 3-i \leq 2$ (i.e. when for some valuation $\rho$ we have $\rho(i) = 0$) the operation $(3-i) \monusE[(\cdot,i);(\cdot,i\leq 3)] 2$ is undefined. Thus, the type system loses its subject congruence property, and subsequently its subject reduction property. There are, however, several ways to address this. One option is to modify the type rules to perform a single advancement of time for a sequence of annotations. A more contained option is to remove monus from the definition of advancement of time, by enriching the formation rules of types with the constructor $\forall_{I}\widetilde{i}.\texttt{serv}^\sigma_K(\widetilde{T})^{-J}$ and by augmenting the definition of advancement as so
\begin{align*}
    \downarrow_I^{\varphi;\Phi}\!\!(\forall_J\widetilde{i}.\texttt{serv}^\sigma_K(\widetilde{T})) =&\; \left\{
\begin{matrix}
\forall_{J-I}\widetilde{i}.\texttt{serv}^\sigma_K(\widetilde{T}) & \text{ if } \varphi;\Phi\vDash I \leq J \\
\forall_0\widetilde{i}.\texttt{serv}^{\sigma\cap\{\texttt{out}\}}_K(\widetilde{T}) & \text{ if } \varphi;\Phi\vDash J \leq I \\
\forall_{J}\widetilde{i}.\texttt{serv}^{\sigma\cap\{\texttt{out}\}}_K(\widetilde{T})^{-I} & \text{ if } \varphi;\Phi\nvDash I \leq J \text{ and } \varphi;\Phi\nvDash J \leq I
\end{matrix}
\right.\\
%
\downarrow_I^{\varphi;\Phi}\!\!(\forall_{J}\widetilde{i}.\texttt{serv}^\sigma_K(\widetilde{T})^{-L}) =&\; \left\{
\begin{matrix}
\forall_{0}\widetilde{i}.\texttt{serv}^{\sigma\cap\{\texttt{out}\}}_K(\widetilde{T}) & \text{ if } \varphi;\Phi\vDash J \leq L+I \\
\forall_{J}\widetilde{i}.\texttt{serv}^{\sigma\cap\{\texttt{out}\}}_K(\widetilde{T})^{-(L+I)} & \text{ if } \varphi;\Phi\nvDash J \leq L+I
\end{matrix}
\right.
%
\end{align*}
This in essence introduces a form of \textit{lazy} time advancement, where time is not advanced until partial monus allows us to do so. Then, as the type rules for servers require a server type of the form $\forall_J\widetilde{i}.\texttt{serv}^\sigma_K(\widetilde{T})$, the summed advancement of time must always be less than or equal, or always greater than or equal to the time of the server, and so typing is invariant to the use of congruence rule $\runa{SC-sum}$. Revisiting the above example, we have that $\susume{\forall_{3-i}\epsilon.\texttt{serv}^{\{\texttt{in},\texttt{out}\}}_0()}{(\cdot,i)}{(\cdot,i\leq 3)}{5} =\; \susume{\susume{\forall_{3-i}\epsilon.\texttt{serv}^{\{\texttt{in},\texttt{out}\}}_0()}{(\cdot,i)}{(\cdot,i\leq 3)}{2}}{(\cdot,i)}{(\cdot,i\leq 3)}{3}$, and so we obtain the original typing
\begin{align*}
    (\cdot,i);(\cdot,i\leq 3);\Gamma\vdash\; !\inputch{a}{}{}{\nil}  \mid 2 : 3 : \asyncoutputch{a}{}{} \triangleleft \{5\}
\end{align*}



% \begin{remark}

%     Baillot and Ghyselen \cite{BaillotGhyselen2021} assume that the minus operator ($-$) for indices is defined such that $n-m=0$ when $m \geq n$. This has the consequence that expressions such as $2-3+3=3$ apply, such that indices form a semiring instead of a ring as we no longer have an additive inverse. In this work we lift this assumption by arguing that any index $I$ using a ring-centric definition for $-$ such that $I \leq 0$, can be simulated using another index $J$ using a semiring-centric definition for $-$ such that $J \leq 0$. For $I$, the order of summation of terms does not matter, and so we can freely change this. By moving any terms with a negative coefficient to the end of the summation, we obtain an expression of the form $c_1 i_1 + \cdots + c_n i_n - c_{n+1} i_{n+1} - \cdots - c_m i_m$ where $c_j$ are positive numbers and $i_j$ are index variables for $j = 0\dots m$. When evaluating this expression from left to right, the result will be increasing until $c_{n + 1} i_{n+1}$, as both the coefficients and index variables are positive, after which it will be decreasing. This results in an expression that is indifferent to the two definitions of $-$ when considering constraints of the form $I \leq 0$. Thus, a normalized constraint using a ring-centric definition of $-$ can be simulated using a normalized constraint using a semiring-centric definition of $-$.

% \end{remark}

\subsection{Undecidability of polynomial constraint judgements}
As we have seen, verifying that a constraint imposes no further restrictions onto index valuations amounts to checking whether all possible index valuations that satisfy a set of known constraints are also contained in the model space of our new constraint. It also amounts to checking whether the feasible region of the constraint contains the feasible region of a known system of inequality constraints, or checking whether the feasible region of the inverse constraint does not intersect the feasible region of a known system of inequality constraints. This turns out to be a difficult problem, and we can in fact prove it undecidable for diophantine constraints, i.e. multivariate polynomial inequalities with integer coefficients, when index variables must have natural (or integer) interpretations. The main idea is to reduce Hilbert's tenth problem \cite{Hilbert1902} to that of verification of judgements on constraints, as this problem has been proven undecidable \cite{Davis1973}. That is, we show that assuming some complete algorithm that verifies judgements on constraints, we can verify whether an arbitrary diophantine equation has a solution with all unknowns taking integer values. We show this result in Lemma \ref{lemma:judgementUndecidable}.
%
\begin{lemma}\label{lemma:judgementUndecidable}
Let $C$ and $C'\in \Phi$ be diophantine inequalities with unknowns in $\varphi$ and coefficients in $\mathbb{N}$. Then the judgement $\varphi;\Phi\vDash C$ is undecidable.
\begin{proof}
By reduction from Hilbert's tenth problem. Let $p=0$ be an arbitrary diophantine equation. We show that assuming some algorithm that can verify a judgement of the form $\varphi;\Phi\vDash C$, we can determine whether $p=0$ has an integer solution. We must pay special attention to the non-standard definition of subtraction in the type system by Baillot and Ghyselen \cite{BaillotGhyselen2021} and to the fact that only non-negative integers substitute for index variables. We first replace each integer variable $x$ in $p$ with two non-negative variables $i_x - j_x$, referring to the modified polynomial as $p'$. We can quickly verify that $p'=0$ has a non-negative integer solution if and only if $p=0$ has an integer solution
\begin{enumerate}
    \item Assume that $p'=0$ has a non-negative integer solution. Then for each variable $x$ in $p$ we assign $x = i_x - j_x$ reaching an integer solution to $p$.
    
    \item Assume that $p=0$ has an integer solution. Then for each pair $i_x$ and $j_x$ in $p'$ we assign $i_x = x$ and $j_x = 0$ when $x \geq 0$ and $i_x = 0$ and $j_x = |x|$ when $x < 0$ reaching a non-negative integer solution to $p'$.
\end{enumerate}
%
Then, by the distributive property of integer multiplication and the associative property of integer addition, we can utilize that $p'$ has an equivalent expanded form 
\begin{align*}
p' = n_1 t_1 + \cdots + n_k t_k + n_{k+1} t_{k+1} + \cdots + n_{k+l} t_{k+l}    
\end{align*}
such that $n_1,\dots,n_k\in\mathbb{N}$, $n_{k+1},\dots,n_{k+l} \in \mathbb{Z}^{\leq 0}$ and $t_1,\dots,t_k,t_{k+1},\dots,t_{k+l}$ are power products over the set of all index variables in $p'$ denoted $\varphi_{p'}$. We can then factor the negative coefficients
\begin{align*}
    p' \;&= n_1 t_1 + \cdots + n_k t_k + n_{k+1} t_{k+1} + \cdots + n_{k+l} t_{k+l}\\ 
    \;&= (n_1 t_1 + \cdots + n_k t_k) + (-1)(|n_{k+1}| t_{k+1} + \cdots + |n_{k+l}| t_{k+l})\\
    \;&= (n_1 t_1 + \cdots + n_k t_k) - (|n_{k+1}| t_{k+1} + \cdots + |n_{k+l}| t_{k+l})
\end{align*}
We use this to show that $p'=0$ has a non-negative integer solution if and only if the following judgement does not hold 
{\small
\begin{align*}
    \varphi_{p'};\{|n_{k+1}| t_{k+1} + \cdots + |n_{k+l}| t_{k+l} \leq n_1 t_1 + \cdots + n_k t_k\}\vDash 1 \leq (n_1 t_1 + \cdots + n_k t_k) - (|n_{k+1}| t_{k+1} + \cdots + |n_{k+l}| t_{k+l}) 
\end{align*}}
We consider the implications separately
\begin{enumerate}
    \item Assume that $p'=0$ has a non-negative integer solution. Then we have that $n_1 t_1 + \cdots + n_k t_k = |n_{k+1}| t_{k+1} + \cdots + |n_{k+l}| t_{k+l}$, and so there must exist a valuation $\rho : \varphi_{p'} \longrightarrow \mathbb{N}$ such that $[\![n_1 t_1 + \cdots + n_k t_k]\!]_\rho = [\![|n_{k+1}| t_{k+1} + \cdots + |n_{k+l}| t_{k+l}]\!]_\rho$. We trivially have that $\rho$ satisfies $[\![|n_{k+1}| t_{k+1} + \cdots + |n_{k+l}| t_{k+l}]\!]_\rho \leq [\![n_1 t_1 + \cdots + n_k t_k]\!]_\rho$. But $\rho$ is not in the model space of the constraint $1 \leq (n_1 t_1 + \cdots + n_k t_k) - (|n_{k+1}| t_{k+1} + \cdots + |n_{k+l}| t_{k+l})$, and so the judgement does not hold.
    
    \item Assume that the judgement does not hold. Then there must exist a valuation $\rho : \varphi_{p'} \longrightarrow \mathbb{N}$ that satisfies $[\![|n_{k+1}| t_{k+1} + \cdots + |n_{k+l}| t_{k+l}]\!]_\rho \leq [\![n_1 t_1 + \cdots + n_k t_k]\!]_\rho$, but that is not in the model space of the constraint $1 \leq (n_1 t_1 + \cdots + n_k t_k) - (|n_{k+1}| t_{k+1} + \cdots + |n_{k+l}| t_{k+l})$. This implies that $[\![n_1 t_1 + \cdots + n_k t_k]\!]_\rho = [\![|n_{k+1}| t_{k+1} + \cdots + |n_{k+l}| t_{k+l}]\!]_\rho$, and so $p'$ has a non-negative integer solution.
\end{enumerate}
% Then the subtraction operator in Baillot and Ghyselen $\cite{BaillotGhyselen2021}$ only has non-standard behavior when $[\![n_{k+1} t_{k+1} + \cdots + n_{k+l} t_{k+l}]\!]_\rho > [\![n_1 t_1 + \cdots + n_k t_k]\!]_\rho$ for some interpretation $\rho : \varphi_{p'} \longrightarrow \mathbb{N}$ where $\varphi_{p'}$ is the set of all index variables in $p'$. Thus, we have that the judgement
% \begin{align*}
%     \varphi_{p'};\{n_{k+1} t_{k+1} + \cdots + n_{k+l} t_{k+l} \leq n_1 t_1 + \cdots + n_k t_k\}\vDash 1 \leq (n_1 t_1 + \cdots + n_k t_k) - (n_{k+1} t_{k+1} + \cdots + n_{k+l} t_{k+l}) 
% \end{align*}
% holds exactly when there exists no index valuation $\rho$ over $\varphi_{p'}$ that simultaneously satisfies $[\![n_{k+1} t_{k+1} + \cdots + n_{k+l} t_{k+l}]\!]_\rho \leq [\![n_1 t_1 + \cdots + n_k t_k]\!]_\rho$ and $[\![p']\!]_\rho = 0$. 
As such, we can verify that the above judgement does not hold if and only if $p'$ has a non-negative integer solution, and by extension if and only if $p$ has an integer solution. Thus, we would have a solution to Hilbert's tenth problem, which is undecidable.
\end{proof}
\end{lemma}

As an unfortunate consequence of Lemma \ref{lemma:judgementUndecidable}, we are forced into considering approximate algorithms for verification of judgements over polynomial constraints (in general). However, this result does not imply that type checking is undecidable. It may well be that problematic judgements are not required to type check any process, as computational complexity has certain properties, such as monotonicity. Note that the freedom of type checking, i.e. we can specify an arbitrary type context as well as type annotations, enables us to select indices that lead to undecidable judgements. To prove that type checking is undedidable, however, a more reasonable result would be that there exists a process that is typable if and only if an undecidable judgement is satisfied. This is out of the scope of this thesis, and so we leave it as future work. % Remark that Baillot and Ghyselen \cite{BaillotGhyselen2021} introduce a notion of type inference in their technical report, where the set of constraints $\Phi$ is empty for any judgement on constraints, and so they are able to bypass some of the problems associated with checking such judgements. However, this comes at the price of expressiveness, as natural types are forced to have lower bounds of $0$ and upper bounds with exactly one index variable and constant. Such indices are arguably sufficient for describing the sizes of simple terms when all operations on these terms in a program can be correspondingly described with a single index variable and constant. However, this quickly becomes too restrictive, as we are unable to type servers that implement simple arithmetic operations such as addition and subtraction.

\subsection{Normalization of linear indices}

To make checking of judgements on constraints tractable, we reduce the set of function symbols on which indices are defined, such that indices may only contain integers and index variables, as well as addition, subtraction and scalar multiplication operators, such that we restrict ourselves to linear functions.
\begin{align*}
        I,J ::= n \mid i \mid I + J \mid I - J \mid n I
    \end{align*}
% \begin{defi}[Indices]
%     \begin{align*}
%         I,J ::= n \mid i \mid I + J \mid I - J \mid I \cdot J
%     \end{align*}
% \end{defi}


Such indices can be written in a \textit{normal} form, presented in Definition \ref{def:normlinindex}.

\begin{defi}[Normalized linear index]\label{def:normlinindex}
    Let $I$ be an index in index variables $\varphi = i_1,\dots,i_n$. We say that $I$ is a \textit{normalized} index when it is a linear combination of index variables $i_1, ..., i_n$. Let $m$ be an integer constant and $I_\alpha\in\mathbb{Z}$ the coefficient of variable $i_\alpha$, we then define normalized indices as
    %
    \begin{align*}
        I = \normlinearindex{m}{I}
    \end{align*}
    
    
    We use the notation $\mathcal{B}(I)$ and $\mathcal{E}(I)$ to refer to the constant and unique identifiers of index variables of $I$, respectively.
\end{defi}

Any index can be transformed to an equivalent normalized index (i.e. it is a normal form) through expansion with the distributive law, reordering by the commutative and associative laws and then by regrouping terms that share variables. Therefore, the set of normalized indices in index variables $i_1,\dots,i_n$ and with coefficients in $\mathbb{Z}$, denoted $\mathbb{Z}[i_1,\dots,i_n]$, is a free module with the variables as basis, as the variables are linearly independent. In Definition \ref{def:operationsmodule}, we show how scalar multiplication, addition and multiplication of normalized indices (i.e. linear combinations of monomials) can be defined. Definition \ref{def:normalizationindex} shows how an equivalent normalized index can be computed from an arbitrary linear index using these operations.
%
\begin{defi}[Operations in $\freemodule$]\label{def:operationsmodule}
Let $I = \normlinearindex[\varphi_1]{n}{I}$ and $J = \normlinearindex[\varphi_2]{m}{J}$ be normalized indices in index variables $i_1,\dots,i_n$. We define addition and scalar addition of such indices. Given a scalar $n\in\mathbb{Z}$, the scalar multiplication $n I$ is
%
\begin{align*}
    n I = \normlinearindex[\mathcal{E}(I)]{n \cdot m}{n I}
\end{align*}
When $d$ is a common divisor of all coefficients in $I$, i.e. $I_\alpha / n \in \mathbb{Z}$ for all $\alpha\in\varphi$, the inverse operation is defined
\begin{align*}
    \frac{I}{d} = \frac{n}{d} + \sum_{\alpha\in \mathcal{E}(I)} \frac{I_\alpha}{d} i_\alpha\quad\text{if}\;\frac{I_\alpha}{d} \in \mathbb{Z}\;\text{for all}\;\alpha\in\mathcal{E}(I)
\end{align*}

The addition of $I$ and $J$ is the sum of constants plus the sum of scaled variables where coefficients $I_\alpha$ and $J_\alpha$ are summed when $\alpha\in\varphi_1 \cap \varphi_2$
\begin{align*}
    I + J = n + m + \sum_{\alpha \in \mathcal{E}(I) \cup \mathcal{E}(J)}(I_\alpha + J_\alpha)i_\alpha
\end{align*}

where for any $\alpha\in \varphi_1 \cup \varphi_2$ such that $I_\alpha + J_\alpha = 0$ we omit the corresponding zero term. The inverse of addition is always defined for elements of a polynomial ring
%
\begin{align*}
    I - J = n - m + \sum_{\alpha \in \mathcal{E}(I) \cup \mathcal{E}(J)}(I_\alpha - J_\alpha)i^\alpha
\end{align*}
\end{defi}
%
%We now formalize the transformation of an index $I$ to an equivalent normalized index in Definition \ref{def:normalizationindex}. An integer constant $n$ corresponds to scaling the monomial identified by the exponent vector of all zeroes by $n$. An index variable $i$ represents the monomial consisting of exactly one $i$ scaled by $1$. For addition, subtraction and multiplication we simply normalize the two subindices and and use the corresponding operators for normalized indices.
\begin{defi}[Index normalization]\label{def:normalizationindex}
The normalization of some index $I$ in index variables $i_1,\dots,i_n$ into an equivalent normalized index $\mathcal{N}(I)\in \mathbb{Z}[i_1,\dots,i_n]$ is a homomorphism defined inductively
    \begin{align*}
        \mathcal{N}(n) =&\; n i_1^0\cdots i_n^0\\
        \mathcal{N}(i_j) =&\; 1 i_1^0 \cdots i_j^1 \cdots i_n^0\\
        \mathcal{N}(I + J) =&\; \mathcal{N}(I) + \mathcal{N}(J)\\
        \mathcal{N}(I - J) =&\; \mathcal{N}(I) - \mathcal{N}(J)\\
        \mathcal{N}(n I) =&\; n \mathcal{N}(I)
    \end{align*}
\end{defi}

% \subsubsection{Normalization of constraints}
% A constraint may provide stronger or weaker restrictions on index variables compared to another constraint, or it may provide entirely different restrictions that are neither stronger nor weaker. For example, assuming some index $J$, if we have the constraint $3 \cdot i \leq J$, the constraint $2 \cdot i \leq J$ is redundant as index variables can only be assigned natural numbers, and thus $3 \cdot i \leq J$ implies $n \cdot i \leq J$ for any $n \leq 3$. Similarly, $I \leq n \cdot j$ implies $I \leq m \cdot j$ for any $n \leq m$. We thus define the subconstraint relation $\sqsubseteq$, and by extension the subindex relation $\sqsubseteq_\text{Index}$, in Definition \ref{def:subconstraint}. If $C_1 \sqsubseteq C_2$ we say that $C_2$ is a subconstraint of $C_1$.


% \begin{defi}[Subindices and subconstraints] \label{def:subconstraint}
%     We define the subindex relation $\sqsubseteq_\text{Index}$ by the following rule
%     \begin{align*}
%         &I \sqsubseteq_\text{Index} J \quad \text{ if} \\
%         &\quad (\mathcal{B}(I) \leq \mathcal{B}(J)) \land\\
%         &\quad (\forall \alpha \in \mathcal{E}(I) \cap \mathcal{E}(J) : I_\alpha \leq J_\alpha)\land\\
%         &\quad (\forall \alpha \in \mathcal{E}(J) \setminus \mathcal{E}(I) : J_\alpha \geq 0)\land\\
%         &\quad (\forall \alpha \in \mathcal{E}(I) \setminus \mathcal{E}(J) : I_\alpha \leq 0)
%     \end{align*}
%     % \begin{align*}
%     %     &(\varphi, F) \sqsubseteq_\text{Index} (\varphi', F') \text{ if} \\
%     %     &\quad (\forall V \in \varphi \cap \varphi' : F(V) \leq F'(V)) \land\\
%     %     &\quad (\forall V \in \varphi' \setminus \varphi : F'(V) \geq 0) \land\\
%     %     &\quad (\forall V \in \varphi \setminus \varphi' : F'(V) \leq 0)
%     % \end{align*}
    
%     We define the subconstraint relation $\sqsubseteq$ by the following rule
%     \begin{align*}
%       &\infrule{I' \sqsubseteq_\text{Index} I \quad J \sqsubseteq_\text{Index} J'}{I \leq J \sqsubseteq I' \leq J'}
%       %
%       %
%       %&\infrule{I \leq J \sqsubseteq I' \leq J' \quad I' \leq J' \sqsubseteq I'' \leq J''}{I \leq J \sqsubseteq I'' \leq J''}
%     \end{align*}
% \end{defi}

We extend normalization to constraints. We first note that an equality constraint $I = J$ is satisfied if and only if $I \leq J$ and $J \leq I$ are both satisfied. Thus, it suffices to only consider inequality constraints. A normalized constraint is of the form $I \leq 0$ for some normalized index $I$, as formalized in Definition \ref{def:normconst}.
%
\begin{defi}[Normalized constraints]\label{def:normconst}
    Let $C = I \leq J$ be an inequality constraint such that $I$ and $J$ are normalized indices. We say that $I-J \leq 0$ is the normalization of $C$ denoted $\mathcal{N}(C)$, and we refer to constraints in this form as \textit{normalized} constraints.
    %We represent normalized constraints $C$ using a single normalized constraint $I$, such that $C$ is of the form
    %\begin{align*}
    %    C = I \leq 0
    %\end{align*}
%
\end{defi}
%
% We now show how any constraint $J \bowtie K$ can be represented using a set of normalized constraints of the form $I \leq 0$ where $I$ is a normalized index. To do this, we first represent the constraint $J \bowtie K$ using a set of constraints of the form $J \leq K$ using the function $\mathcal{N_R}$. We then finalize the normalization using the function $\mathcal{N}$ by first moving all indices to the left-hand side of the constraint.
%
% \begin{defi}
%     Given a constraint $I \bowtie J$ $(\bowtie\; \in \{\leq, \geq, =\})$, the function $\mathcal{N_R}$ converts $I \bowtie J$ to a set of constraints of the form $I \leq J$
%     %
%     \begin{align*}
%         \mathcal{N_R}(I \leq J) &= \{I \leq J\}\\
%         \mathcal{N_R}(I \geq J) &= \{J \leq I\}\\
%         %\mathcal{N_R}(I < J) &= \{I+1 \leq J\}\\
%         %\mathcal{N_R}(I > J) &= \{J+1 \leq I\}\\
%         \mathcal{N_R}(I = J) &= \{I \leq J, J \leq I\}
%     \end{align*}
% \end{defi}
%
% \begin{defi}
%     Given a constraint $C$, the function $\mathcal{N}$ converts $C$ into a set of normalized constraints of the form $I \leq 0$
%     %
%     \begin{align*}
%         \mathcal{N}(C) &= \left\{I-J \leq 0 \mid (I \leq J) \in \mathcal{N}_R(C)\right\}
%     \end{align*}
% \end{defi}
%
%Normalized constraints have the key property that, given any two constraints $I \leq 0$ and $J \leq 0$, we can combine these to obtain a new constraint $J + I \leq 0$. This is possible as we know that both $I$ and $J$ are both non-positive, and so their sum must also be non-positive. In general, given $n$ normalized constraints $I_1 \leq 0, ..., I_n \leq 0$, we can infer any linear combination $a_1 \cdot I_1 \leq 0 + ... + a_n \cdot I_n \leq 0$ where $a_i \geq 0$ for $i = 1..n$ as new constraints that can be inferred based on the constraints $I_1 \leq 0, ..., I_n \leq 0$. Linear combinations where all coefficients are non-negative are also called \textit{conical combinations}.
Normalizing constraints has a number of benefits. First of all, it ensures that equivalent constraints are always expressed the same way. Secondly, having all constraints in a common form where variables only appear once means we can easily reason about individual variables of a constraint, which will be useful later when we verify constraint judgements.
%
\subsection{Checking for emptiness of model space}
As explained in Section \ref{sec:cjalternativeform}, we can verify a constraint judgement $\varphi;\Phi \vDash C_0$ by letting $C_0'$ be the inverse of $C_0$ and checking if $\mathcal{M}_\varphi(\Phi \cup \{C_0'\}) = \emptyset$ holds. Being able to check for non-emptiness of a model space is therefore paramount for verifying constraint judgements. For convenience, given a finite ordered set of index variables $\varphi = \{i_1, i_2, \dots, i_n\}$, we represent a normalized constraint $I \leq 0$ as a vector $\left( \mathcal{B}(I), I_1\; I_2\; \cdots\; I_n \right)_{\varphi}$. As such, the constraint $-5i + -2j + -4k \leq 0$ can be represented by the vector $\cvect[\varphi_1]{0 {-5} {-2} {-4}}$ where $\varphi_1=\left\{i, j, k\right\}$. Another way to represent that same constraint is with the vector $\cvect[\varphi_2]{0 {-5} {-2} 0 {-4}}$ where $\varphi_2 = \left\{i,j,l,k\right\}$. We denote the vector representation of a constraint $C$ over a finite ordered set of index variables $\varphi$ by $\mathbf{C}_{\varphi}$. We extend this notation to sets of constraints, such that $\Phi_{\varphi}$ denotes the set of vector representations over $\varphi$ of normalized constraints in $\Phi$\\

Recall that the model space of any set of constraints $\Phi$ is the set of all valuations satisfying all constraints in $\Phi$. Thus, to show that $\mathcal{M}_\varphi(\Phi)$ is empty, we must show that no valuation $\rho$ exists satisfying all constraints in $\Phi$. This is a linear constraint satisfaction problem (CSP) with an infinite domain. One method for solving such is by optimization using the simplex algorithm. If the linear program of the CSP has a feasible solution, the model space is non-empty and if it does not have a feasible solution, the model space is empty.\\

As is usual for linear constraints, our linear constraints can be thought of as hyper-planes dividing some n-dimensional space in two, with one side constituting the feasible region and the other side the non-feasible region. By extension, for a set of constraints their shared feasible region is the intersection of all of their individual feasible regions. Since the feasible region of a set of constraints is defined by a set of hyper-planes, the feasible region consists of a convex polytope. This fact is used by the simplex algorithm when performing optimization.\\

The simplex algorithm has some requirements to the form of the linear program it is presented, i.e. that it must be in \textit{standard} form. The standard form is a linear program expressed as 
\begin{align*}
    \text{minimize}&\quad \mathbf{c}^T\mathbf{a}\\
    \text{subject to}&\quad M\mathbf{a} = \mathbf{b}\\
    &\quad\mathbf{a} \geq \mathbf{0}
\end{align*}
where $M$ is a matrix representing constraints, $\mathbf{a}$ is a vector of scalars, and $\mathbf{b}$ is a vector of constants. As such, we first need all our constraints to be of the form $a_0 \cdot i_0 + ... + a_n \cdot i_n \leq b$, after which we must convert them into equality constraints by introducing \textit{slack} variables that allow the equality to also take on lower values. Since all of our constraints are normalized and of the form $I \leq 0$, all of our slack variables will have negative coefficients. In our specific case, we let row $i$ of $M$ consist of $(\mathbf{C}^i_\varphi)_{-1}$, where $(\cdot)_{-1}$ removes the first element of the vector (the constant term here). We must also include our slack variables, and so we augment row $i$ of $M$ with the n-vector with all zeroes except at position $i$ where it is $-1$. We let $\mathbf{a}$ be a column vector containing our variables in $\varphi$ as well as our slack variables, and finally we let $\mathbf{b}_i = -(\mathbf{C}^i_\varphi)_1$. $\mathbf{c}$ may be an arbitrary vector.\\

Checking feasibility of the above linear program can itself be formulated as a linear program that is guaranteed to be feasible, enabling us to use efficient polynomial time linear programming algorithms, such as interior point methods, to check whether constraints are covered. Let $\mathbf{s}$ be a new vector, then we have the linear program
%
\begin{align*}
    \text{minimize}&\quad \mathbf{1}^T\mathbf{s}\\
    \text{subject to}&\quad M\mathbf{a} + \mathbf{s} = \mathbf{b}\\
    &\quad\mathbf{a},\mathbf{s} \geq \mathbf{0}
\end{align*}
where $\mathbf{1}$ is the vector of all ones. We can verify the feasibility of this problem with the certificate $(\mathbf{a},\mathbf{s})=(\mathbf{0},\mathbf{b})$. Then the original linear program is feasible if and only if the augmented problem has an optimal solution $(\mathbf{x}^*,\mathbf{s}^*)$ such that $\mathbf{s}^* = \mathbf{0}$.\\

Given a constraint judgement $\varphi;\Phi \vDash C_0$, it should be noted that while the simplex algorithm can be used to check if a solution exists to the constraints $\Phi \cup \{C_0'\}$, there is no guarantee that the solution is an integer solution nor that an integer solution exists at all. Thus, in the case that a non-integer solution exists but no integer solution, this method will over-approximate. An example of such is the two constraints $3i - 1 \leq 0$ and $-2i + 1 \leq 0$ yielding the feasible region where $\frac{1}{3} \leq i \leq \frac{1}{2}$, containing no integers. For an exact solution, we may use integer programming.

% \subsubsection{Conical combinations of constraints}
% We now show how constraints can be conically combined. For convenience, given a finite ordered set of index variables $\varphi = \{i_1, i_2, \dots, i_n\}$, we represent a normalized constraint $I \leq 0$ as a vector $\left( \mathcal{B}(I), I_1\; I_2\; \cdots\; I_n \right)_{\varphi}$. As such, the constraint $-5i + -2j + -4k \leq 0$ can be represented by the vector $\cvect[\varphi_1]{0 {-5} {-2} {-4}}$ where $\varphi_1=\left\{i, j, k\right\}$. Another way to represent that same constraint is with the vector $\cvect[\varphi_2]{0 {-5} {-2} 0 {-4}}$ where $\varphi_2 = \left\{i,j,l,k\right\}$. We denote the vector representation of a constraint $C$ over a finite ordered set of index variables $\varphi$ by $\mathbf{C}_{\varphi}$. We extend this notation to sets of constraints, such that $\Phi_{\varphi}$ denotes the set of vector representations over $\varphi$ of normalized constraints in $\Phi$. Then for a finite ordered set of exponent vectors $\varphi$ and a set of normalized constraints $\Phi$, we can infer any constraint $C$ represented by a vector $\mathbf{C}_\varphi\in \text{coni}(\Phi_\varphi)$ where $\text{coni}(\Phi_\varphi)$ is the \textit{conical hull} of $\Phi_\varphi$. That is, $\text{coni}(\Phi_\varphi)$ is the set of conical combinations with non-negative integer coefficients of vectors in $\Phi_\varphi$
% %
% \begin{align*}
%   \text{coni}(\Phi_\varphi) = \left\{\sum^k_{i=1} a_i {\mathbf{C}^i_\varphi} : {\mathbf{C}^i_\varphi} \in \Phi_\varphi,\; a_i,k \in \mathbb{N}\right\}  
% \end{align*}
% %
% Then, to check if a constraint $C^{new}$ is covered by the set of normalized constraints $\Phi = \{C_1,C_2,\dots, C_n\}$, we can test if $\mathbf{C}^{new}_\varphi$ is a member of the conical hull $\text{coni}(\Phi_\varphi)$. However, by itself, this does not take into account subconstraints of constraints in $\Phi$, as these may not necessarily be written as conical combinations of $\Phi_\varphi$. To account for these, we can include $m=|\varphi|$ vectors of size $m$ of the form $\cvect{-1 0 $\cdots$ 0}, \cvect{0 {-1} 0 $\cdots$ 0}, \dots, \cvect{0 $\cdots$ 0 {-1})}$ in $\Phi_\varphi$. As the conical hull $\text{coni}(\Phi_\varphi)$ is infinite when there exists $\mathbf{C}_\varphi \in \Phi_\varphi$ such that $\mathbf{C}_\varphi \neq \mathbf{0}$ where $\mathbf{0}$ is the vector of all zeroes, when checking for the existence of a conical combination of vectors in $\Phi_\varphi$ equal to $\mathbf{C}^\textit{new}_\varphi$, we can instead solve the following system of linear equations
% %
% \begin{align*}
%     a_1 {\mathbf{C}^1_\varphi}_1 + a_2 {\mathbf{C}^2_\varphi}_1 + \cdots + a_n {\mathbf{C}^n_\varphi}_1 =&\; {\mathbf{C}^{new}_\varphi}_1\\
%     a_1 {\mathbf{C}^1_\varphi}_2 + a_2 {\mathbf{C}^2_\varphi}_2 + \cdots + a_n {\mathbf{C}^n_\varphi}_2 =&\; {\mathbf{C}^{new}_\varphi}_2\\
%     &\!\!\!\vdots\\
%     a_1 {\mathbf{C}^1_\varphi}_m + a_2 {\mathbf{C}^2_\varphi}_m + \cdots + a_n {\mathbf{C}^n_\varphi}_m =&\; {\mathbf{C}^{new}_\varphi}_m
% \end{align*}
% %
% where $a_1,a_2,\dots,a_m\in\mathbb{Z}_{\geq 0}$ are non-negative integer numbers. However, this is an integer programming problem, and so it is NP-hard. We can relax the requirement for $a_1,a_2,\dots,a_m$ to be integers, as the equality relation is preserved under multiplication by any positive real number. We can then view the above system as a linear program, with additional constraints $a_i \geq 0$ for $1 \geq i \geq n$. That is, let $M = \vect{$\mathbf{C}^1_\varphi$ $\mathbf{C}^2_\varphi$ $\cdots$ $\mathbf{C}^n_\varphi$}$ be a matrix with column vectors representing constraints and $\mathbf{a} = \vect{$a_1$ $a_2$ $\cdots$ $a_n$}$ be a row vector of scalars, then checking whether $\mathbf{C}^{new}_\varphi\in\text{coni}(\Phi_\varphi)$ amounts to determining if the following linear program is feasible
% %
% \begin{align*}
%     \text{minimize}&\quad \mathbf{c}^T\mathbf{a}\\
%     \text{subject to}&\quad M\mathbf{a} = \mathbf{C}^{new}_\varphi\\
%     &\quad\mathbf{a} \geq \mathbf{0}
% \end{align*}
% %
% where $\mathbf{c}$ is an arbitrary vector of length $n$ and $\mathbf{0}$ is the vector of all zeroes of length $n$. Checking feasibility of the above linear program can itself be formulated as a linear program that is guaranteed to be feasible, enabling us to use efficient polynomial time linear programming algorithms, such as interior point methods, to check whether constraints are covered. Let $\mathbf{s}$ be a new vector of length $m$, then we have the linear program
% %
% \begin{align*}
%     \text{minimize}&\quad \mathbf{1}^T\mathbf{s}\\
%     \text{subject to}&\quad M\mathbf{a} + \mathbf{s} = \mathbf{C}^{new}_\varphi\\
%     &\quad\mathbf{a},\mathbf{s} \geq \mathbf{0}
% \end{align*}
% where $\mathbf{1}$ is the vector of all ones of length $m$. We can verify the feasibility of this problem with the certificate $(\mathbf{a},\mathbf{s})=(\mathbf{0},\mathbf{C}^{new}_\varphi)$. Then the original linear program is feasible if and only if the augmented problem has an optimal solution $(\mathbf{x}^*,\mathbf{s}^*)$ such that $\mathbf{s}^* = \mathbf{0}$.
% %

\begin{examp}
    Given the constraints
    \begin{align*}
        C^1 &= 3i - 3 \leq 0\\
        C^2 &= j + 2k - 2 \leq 0\\
        C^3 &= -k \leq 0\\
        C^{new} &= i + j - 3 \leq 0
    \end{align*}
    
    we want to check if the constraint judgement $\{i, j, k\};\{C^1, C^2, C^3\} \vDash C^{new}$ is satisfied\\
    
    
    We first let $C^{newinv}$ be the inversion of constraint $C^{new}$.
    \begin{align*}
        C^{newinv} &= 1i + 1j - 2 \geq 0
    \end{align*}
    
    We now want to check if the feasible region $\mathcal{M}_\varphi(\{C^1, C^2, C^3, C^{newinv}\})$ is nonempty. To do so, we construct a linear program with the four constraints. To convert all inequality constraints into equality constraints, we add the slack variables $s_1, s_2, s_3, s_4$ 
    
    \begin{align*}
        \text{minimize}&\quad i + j + k\\
        \text{subject to}&\quad 3i + 0j + 0k + s_1 = 3\\
        &\quad 0i + 1j + 2k + s_2 = 2\\
        &\quad 0i + 0j - 1k + s_3 = 0\\
        &\quad 1i + 1j + 0k - s_4 = 2\\
        &\quad i, j, k, s_1, s_2, s_3, s_4 \geq 0
    \end{align*}
    
    Using an algorithm such as the simplex algorithm, we see that there is no feasible solution, and so we conclude that the constraint judgement $\{i, j, k\};\{C^1, C^2, C^3\} \vDash C^{new}$ is satisfied.
    
    
    %%%%%%%%%%%%%%%%%%%%%%%
    
    % We first represent the four constraints as vectors in terms of some ordered set $\varphi$ of index variables and some ordered set $\varphi$ of exponent vectors.\\
    
    % Let $\varphi = \{i, j, k\}$ and $\varphi = \{\evect{1 0 0}, \evect{0 1 0}, \evect{0 0 1}, \evect{0 0 0}\}$. The constraints $C^1, C^2, C^3, C^{new}$ can now be written as the following vectors
    % %
    % \begin{align*}
    %     \mathbf{C}^1_\varphi &= \cvect{1 0 0 -3}\\
    %     \mathbf{C}^2_\varphi &= \cvect{0 1 1 -2}\\
    %     \mathbf{C}^3_\varphi &= \cvect{0 0 -1 0}\\
    %     \mathbf{C}^{new}_\varphi &= \cvect{2 3 2 -15}
    % \end{align*}
    
    % With the constraints now represented as vectors, we can prepare the equation $M\mathbf{a} = \mathbf{b}$ representing the conical combination, for which we wish to check if a solution exists given given the requirement that $\mathbf{a} \geq \mathbf{0}$. We first prepare the matrix $M$, where we represent the constraint vectors as column vectors
    % %
    % \begin{align*}
    %     &M = \vect{$\mathbf{C}^1_\varphi$ $\mathbf{C}^2_\varphi$ $\mathbf{C}^3_\varphi$ $\bm{\beta}_1$ $\bm{\beta}_2$ $\bm{\beta}_3$ $\bm{\beta}_4$}\\
    %     &\quad \text{where } \bm{\beta}_1 = \cvect{-1 0 0 0}, \bm{\beta}_2 = \cvect{0 {-1} 0 0}, \bm{\beta}_3 = \cvect{0 0 {-1} 0}, \bm{\beta}_4 = \cvect{0 0 0 {-1}}
    % \end{align*}
    % %
    % We include vectors $\bm{\beta}_i, i \in \{1, 2, 3, 4\}$ to ensure we can also use subconstraints of $\mathbf{C}^i, i \in \{1, 2, 3\}$ when checking if we can construct $\mathbf{C}^{new}_\varphi$. To check if a solution exists to the aforementioned equation, we solve the following linear program to check if $\mathbf{s} = \mathbf{0}$
    % \begin{align*}
    %     \text{minimize}&\quad \mathbf{1}^T\mathbf{s}\\
    %     \text{subject to}&\quad M\mathbf{a} + \mathbf{s} = \mathbf{C}^{new}_\varphi\\
    %     &\quad\mathbf{a},\mathbf{s} \geq \mathbf{0}
    % \end{align*}
    
    % This is possible given $\mathbf{a} = \vect{2 3 1 0 0 0 3}$, and so a solution exists to the canonical combination. Notice that we had to use the additional $\bm{\beta}$ vectors when constructing the conical combination. This shows the importance of subconstraints when checking type judgements.
\end{examp}
%
% \section{Soundness}
% %


% \begin{theorem}[Subject reduction]\label{theorem:srbg}
% If $\varphi;\Phi;\Gamma\vdash P \triangleleft K$ and $P \leadsto Q$ then $\varphi;\Phi;\Gamma\vdash Q \triangleleft K'$ with $\varphi;\Phi\vDash k' \leq K$.
% \begin{proof} by induction on the rules defining $\leadsto$.
%     \begin{description}
%     \item[$\runa{R-rep}$]
%     %
%     \item[$\runa{R-comm}$]
%     %
%     \item[$\runa{R-zero}$]
%     %
%     \item[$\runa{R-par}$]
%     %
%     \item[$\runa{R-succ}$]
%     %
%     \item[$\runa{R-empty}$]
%     %
%     \item[$\runa{R-res}$]
%     %
%     \item[$\runa{R-cons}$]
%     %
%     \item[$\runa{R-struct}$]
%     %
%     %\item[$\runa{R-tick}$] We have that $P = \tick{P'}$ and $Q=P'$. Then by $\runa{S-tick}$ we have $\varphi;\Phi;\Gamma\vdash $
%     \end{description}
% \end{proof}
% \end{theorem}

% \begin{lemma}\label{lemma:timeredtype}
% If $\varphi;\Phi;\Gamma\vdash P \triangleleft K$ with $P\!\not\!\leadsto$ and $P \Longrightarrow^{-1} Q$ then $\varphi;\Phi;\downarrow_1\!\Gamma\vdash Q \triangleleft K'$ with $\varphi;\Phi\vDash K' \leq K + 1$.
% \begin{proof}
    
% \end{proof}
% \end{lemma}

% \begin{theorem}\label{theorem:ubbg}
% If $\varphi;\Phi;\Gamma\vdash P \triangleleft K$ and $P \hookrightarrow^n Q$ then $\varphi;\Phi\vDash n \leq K$.
% \begin{proof} by induction on the number of time reductions $n$ in the sequence $P \hookrightarrow^n Q$.
    
% \end{proof}
% \end{theorem}


% % \begin{description}
% %     \item[$\runa{S-nil}$]
% %     %
% %     \item[$\runa{S-tick}$]
% %     %
% %     \item[$\runa{S-nu}$]
% %     %
% %     \item[$\runa{S-nmatch}$]
% %     %
% %     \item[$\runa{S-lmatch}$]
% %     %
% %     \item[$\runa{S-par}$]
% %     %
% %     \item[$\runa{S-iserv}$]
% %     %
% %     \item[$\runa{S-ich}$]
% %     %
% %     \item[$\runa{S-och}$]
% %     %
% %     \item[$\runa{S-oserv}$]
% %     \end{description}


% %
% It is worth noting that the Simplex algorithm does not guarantee an integer solution, and so we may get indices in constraints where the coefficients may be non-integer values. However, we can use the fact that any feasible linear programming problem with rational coefficients also has an (optimal) solution with rational values \cite{keller2016applied}. We use this fact and Lemma \ref{lemma:constraintcommonden} and \ref{lemma:constraintscaling} to show that we need not to worry about the solution given by the Simplex algorithm, given a rational linear programming problem. Definition \ref{def:constraintequivalence} defines what it means for constraints to be equivalent.

% \begin{defi}[Conditional constraint equivalence]\label{def:constraintequivalence}
%     Let $C_1$, $C_2$ and $C\in\Phi$ be linear constraints with integer coefficients and unknowns in $\varphi$. We say that $C_1$ and $C_2$ are equivalent with respect to $\varphi$ and $\Phi$, denoted $C_1 =_{\varphi;\Phi} C_2$, if we have that
%     \begin{equation*}
%     \mathcal{M}_\varphi(\{C_1\} \cup \Phi) = \mathcal{M}_\varphi(\{C_2\} \cup \Phi) %\mathcal{M}_\varphi(\{C_0\})
% \end{equation*}
% where $\mathcal{M}_{\varphi'}(\Phi')=\{\rho : \varphi' \rightarrow \mathbb{N} \mid \rho \vDash C\;\text{for}\; C \in \Phi'\}$ is the model space of a set of constraints $\Phi'$ over a set of index variables $\varphi'$.
%     %
%     %
%     %$\varphi;\Phi\vDash C_1$ if and only if $\varphi;\Phi\vDash C_2$.
%     %Two normalized constraints $C_1$ and $C_2$ are said to be \textit{equivalent} if for any index valuation $\rho$, we have that $\rho \vDash C_1$ if and only if $\rho \vDash C_2$.
% \end{defi}

% \begin{lemma}\label{lemma:constraintscaling}
% Let $I \leq 0$ be a linear constraint with unknowns in $\varphi$. Then $I \leq 0 =_{\varphi;\Phi} n I \leq 0$ for any $n>0$ and set of constraints $\Phi$.
% \begin{proof}
%     This follows from the fact that if $I \leq 0$ is satisfied, then the sign of $I$ must be non-positive, and so the sign of $n I$ must also be non-positive as $n > 0$. Conversely, if $I \leq 0$ is not satisfied, then the sign of $I$, must be positive and so the sign of $n I$ must also be positive.
% \end{proof}
% \end{lemma}

% \begin{lemma}\label{lemma:constraintcommonden}
% Let $I \leq 0$ be a normalized linear constraint with rational coefficients and unknowns in $\varphi$. Then there exists a normalized linear constraint $I' \leq 0$ with integer coefficients and unknowns in $\varphi$ such that $I \leq 0 =_{\varphi;\Phi} I' \leq 0$ for any set of constraints $\Phi$.% there exists an equivalent constraint $I' = \normlinearindex{n'}{I'}$ where $n', I'_{\alpha_1}, \dots,I'_{\alpha_{m}}$ are integers.
% \begin{proof}
%     It is well known that any set of rationals has a common denominator, whose multiplication with any rational in the set yields an integer. One is found by multiplying the denominators of all rationals in the set. As the coefficients of $I$ are non-negative, this common denominator must be positive. By Lemma \ref{lemma:constraintscaling}, we have that $I\leq 0 =_{\varphi;\Phi} n I \leq 0$ where $n$ is a positive number and $\Phi$ is any set of constraints.% the constraint $I \leq 0$ is equivalent to $d I \leq 0$.
% \end{proof}
% \end{lemma}

% %%
% % \begin{lemma}
% % Let $I \leq J$ and $C\in\Phi$ be a linear constraints with integer coefficients and unknowns in $\varphi$. Then $I \leq J =_{\varphi;\Phi} \mathcal{N}(I\leq J)$ if for any subtraction $K - L$ in $I$ or $J$, we have $\varphi;\Phi\vDash L \leq K$. 
% % \begin{proof}
    
% % \end{proof}
% % \end{lemma}
% % %

% % % \begin{lemma}
% % % Let $C$ and $C'\in\Phi$ be normalized linear constraints with integer coefficients and unknowns in $\varphi$. Then $\varphi;\Phi\nvDash C$ if there does not exist $\mathbf{C}^{new}_\varphi\in\text{coni}(\Phi_\varphi \cup \{\mathbf{0}\})$ with $\mathbf{C}_\varphi\leq \mathbf{C}^{new}_\varphi$.
% % % \begin{proof}
    
% % % \end{proof}
% % % \end{lemma}


% % %
% % \begin{theorem}
% % Let $C$ and $C'\in\Phi$ be normalized linear constraints with integer coefficients and unknowns in $\varphi$. Then $\varphi;\Phi\vDash_{\mathbb{R}^{\geq 0}} C$ if and only if there exists $\textbf{C}^{new}_\varphi\in\text{coni}(\Phi_\varphi \cup \{\mathbf{0}\})$ with $\textbf{C}_\varphi\leq \textbf{C}^{new}_\varphi$.
% % \begin{proof}
% %     We consider the implications separately
% %     \begin{enumerate}
% %         \item Assume that $\varphi;\Phi\vDash_{\mathbb{R}^{\geq 0}} C$. Then for all valuations $\rho : \varphi \longrightarrow \mathbb{R}^{\geq 0}$ such that $\rho\vDash \Phi$ we also have $\rho\vDash C$, or equivalently $([\![I_1]\!]_\rho \leq 0) \land \cdots \land ([\![I_n]\!]_\rho \leq 0) \implies [\![I]\!]_\rho \leq 0$, where $C = I_0 \leq 0$ and $C_i = I_i \leq 0$ for $C_i\in \Phi$. We show by contradiction that this implies there exists $\textbf{C}^{new}_\varphi\in\text{coni}(\Phi_\varphi \cup \{\mathbf{0}\})$ with $\textbf{C}_\varphi\leq \textbf{C}^{new}_\varphi$. Assume that such a conical combination does not exist. Then for all $\mathbf{C}'_\varphi\in\text{coni}(\Phi_\varphi \cup \{\mathbf{0}\})$ there is at least one coefficient ${\mathbf{C}'_\varphi}_k$ for some $0\leq k \leq |\varphi|$ such that ${\mathbf{C}'_\varphi}_k < {\mathbf{C}_\varphi}_k$. We show that this implies there exists $\rho\in\mathcal{M}_\varphi(\Phi)$ such that $\rho\nvDash C$.\\ 
        
        
% %         and so there must exist $\rho : \varphi \longrightarrow \mathbb{R}^{\geq 0}$ such that $\rho\vDash C'$ and $\rho\nvDash C$. However, as $\varphi;\Phi\vDash_{\mathbb{R}^{\geq 0}} C$ holds there must be some constraint $C''\in\Phi$ such that $\rho\nvDash C''$.\\
        
        
% %         Assume that there does not exist a conical combination $\textbf{C}^{new}_\varphi\in\text{coni}(\Phi_\varphi \cup \{\mathbf{0}\})$ with $\textbf{C}_\varphi\leq \textbf{C}^{new}_\varphi$. \\
        
        
        
% %         We have that $C = n + \sum_{\alpha\in\mathcal{E}(I)} I_\alpha i_\alpha$, and so for $\rho\vDash n + \sum_{\alpha\in\mathcal{E}(I)} I_\alpha i_\alpha$ to hold, it must be that $[\![\sum_{\alpha\in\mathcal{E}(I)} I_\alpha i_\alpha]\!]_\rho \leq -n$. This implies that $\Phi$ contains constraints that collectively bound the sizes of index variables that appear in $\sum_{\alpha\in\mathcal{E}(I)} I_\alpha i_\alpha$, such that $\sum_{\alpha\in\mathcal{E}(I)} I_\alpha i_\alpha$ cannot exceed $-n$. We now show by contradiction that there must then exist $\textbf{C}^{new}_\varphi\in\text{coni}(\Phi_\varphi \cup \{\mathbf{0}\})$ such that $\textbf{C}_\varphi\leq \textbf{C}^{new}_\varphi$. Assume that such a conical combination does not exist. Then it must be that for any $C'_\varphi\in\text{coni}(\Phi \cup \{\mathbf{0}\})$, at least one coefficient in $C'_\varphi$ is smaller than the corresponding coefficient in $C_\varphi$, implying that $C$ imposes a new restriction on valuations. Thus, there must exist a valuation $\rho$ such that $\rho\vDash\Phi$ but $\rho\nvDash C$, but then we have that $\varphi;\Phi\nvDash C$, and so we have a contradiction.
        
% %         \item Assume that there exists $\mathbf{C}^{new}_\varphi\in\text{coni}(\Phi_\varphi \cup \{\mathbf{0}\})$ with $\mathbf{C}_\varphi \leq \mathbf{C}^{new}_\varphi$. Then by Lemma \ref{TODO}, we have that $\varphi;\phi\vDash C^{new}$ and by Lemma \ref{TODO} it follows from $\mathbf{C}_\varphi \leq \mathbf{C}^{new}_\varphi$ that also $\varphi;\Phi\vDash C$.
% %     \end{enumerate}
% % \end{proof}
% % \end{theorem}
\subsection{Reducing polynomial constraints to linear constraints}\label{sec:verifyingpolynomial}

Many programs do not run in linear time, and so we cannot type them if we are constrained to just verifying linear constraint judgements. In this section we show how we can reduce certain polynomial constraints to linear constraints, enabling us to use the techniques described above. We first extend our definition of indices such that they can be used to express multivariate polynomials. We assume a normal form for polynomial indices akin to that of linear indices. Terms are now monomials with integer coefficients.
%
\begin{align*}
        I,J ::= n \mid i \mid I + J \mid I - J \mid I J
\end{align*}
%
When reducing normalized constraints with polynomial indices to normalized constraints with linear indices, we wish to construct new linear constraints that are only satisfied if the original polynomial constraint is satisfied. For example, given the constraint ${-i^2 + 10 \leq 0}$, one can see that this polynomial constraint can be simulated using the constraint $-i + \sqrt{10} \leq 0$. We notice that the reason this is possible is that when ${-i^2 + 10 \leq 0}$ holds, i.e. when $i \geq \sqrt{10}$, the value of $i$ can always be increased without violating the constraint. Similarly, when ${-i^2 + 10 \geq 0}$ holds, the value of $i$ can always be decreased until reaching its minimum value of 0 without violating the constraint. We can thus introduce a new simpler constraint with the same properties, i.e. $-i + \sqrt{10} \leq 0$. More specifically, the polynomials of the left-hand side of the two constraints share the same positive real-valued roots as well as the same sign for any value of $i$. In general, limiting ourselves to univariate polynomials, for any constraint whose left-hand side polynomial only has a single positive root, we can simulate such a constraint using a constraint of the form $a \cdot i + c \leq 0$. For describing complexities of programs, we expect to mostly encounter monotonic polynomials with at most a single positive real-valued root.\\

Note that the above has the consequence that we may end up with irrational coefficients for indices of constraints. While such constraints are not usually allowed, they can still represent valid bounds for index variables. As such, we can allow them as constraints in this context. Furthermore, any irrational coefficient can be approximated to an arbitrary precision using a rational coefficient.\\

For finding the roots of a specific polynomial we can use either analytical or numerical methods. Using analytical methods has the advantage of being able to determine all roots with exact values, however, we are limited to polynomials of degree at most four as stated by the Abel-Ruffini theorem \cite{abelruffinitheorem}. With numerical methods, we are not limited to polynomials of a specific degree, however, numerical methods often require a given interval to search for a root and do not guarantee to find all roots. Introducing constraints with false restrictions may lead to an under-approximating type system, and so we must be careful not to introduce such. We must therefore ensure we find all roots to avoid constraints with false restrictions. We can use Descartes rule of signs to get an upper bound on the number of positive real roots of a polynomial. Descarte's rule of signs states that the number of roots in a polynomial is at most the number of sign-changes in its sequence of coefficients.\\

For our application, we decide to only consider constraints whose left-hand side polynomial is univariate and monotonic with a single root. In some cases we may remove safely positive monomials in a normalized constraint to obtain such constraints. We limit ourselves to these constraints both to keep complexity down, as well as because we expect to mainly encounter such polynomials when considering complexity analysis of programs. We use Laguerre's method as a numerical method to find the root of the polynomial, which has the advantage that it does not require any specified interval when performing root-finding. Assuming we can find a root $r$, we add an additional constraint $\pm (i - r) \leq 0$ where the sign depends on whether the original polynomial is increasing or decreasing.\\

Additionally, given a non-linear monomial, we may also treat this as a single unit and construct linear combinations of this by treating the monomial as a single fresh variable. For example, given normalized constraints $C_1 = i^2 + 4 \leq 0$, $C_2 = i - 2 \leq 0$, and $C_3 = 2ij \leq 0$, we may view these as the linear constraints $C_1' = k + 4 \leq 0$, $C_2' = i - 2 \leq 0$, and $C_3' = 2l \leq 0$ where $k = i^2$ and $l = ij$. This may make the feasible region larger than it actually is, meaning we over-approximate when verifying judgements on polynomial constraints.

\begin{examp}
    We want to check if the following judgement holds
    $$\{i, j\}; \{-2i \leq 0, -1i^2 + 1j + 1 \leq 0\} \vDash -2i + 2 \leq 0$$
    
    %We notice that $-2i + 2$ cannot be written as a conical combination of the polynomials $-2i$ and $-1i^2 + 1j + 1$.\\
    
    We first try to generate new constraints of the form $-a \cdot i + r \leq 0$ for some index variable $i$ and some constants $a$ and $r$ based on our two existing constraints using the root-finding method. The first constraint is already of such form, so we can only consider the second. For the second, we first use the subconstraint relation to remove the term $1j$ obtaining $-1i^2 + 1 \leq 0$. Next, we note that $-1i^2 + 1$ is a monotonically decreasing polynomial as every coefficient excluding the constant term is negative. We then find the root $r = 1$ of the polynomial and add a new constraint $-1i + 1 \leq 0$ to our set of constraints. Finally, we can invert the constraint $-2i + 2 \leq 0$ obtaining the constraint $-2i + 1 \geq 0$. We now need to check if the feasible region $\mathcal{M}_{\{i, j, i^2\}}(\{-2i \leq 0, -1i^2 + 1j + 1 \leq 0, -1i + 1 \leq 0, -2i + 1 \geq 0\})$ is empty. Treating the monomial $i^2$ as its own separate variable and solving this as a linear program using an algorithm such as the simplex algorithm, we see that there is indeed no solution, and so the constraint is satisfied. 
\end{examp}

\subsection{Trivial judgements}
We now show how some judgements may be verified without neither transforming constraints into linear constraints nor solving any integer programs. To do so, we consider an example provided by Baillot and Ghyselen \cite{BaillotGhyselen2021}, where we exploit the fact that all coefficients in the normalized constraints are non-positive. Judgements with such constraints can be answered in linear time with respect to the number of monomials in the normalized equivalent of constraint $C$. That is if all coefficients in the normalized constraint are non-positive, we can guarantee that the constraint is always satisfied, recalling that only naturals substitute for index variables. Similarly, if there are no negative coefficients and at least one positive coefficient, we can guarantee that the constraint is never satisfied.\\

In practice, it turns out that we can type check many processes by simply over-approximating constraint judgements using pair-wise coefficient inequality constraints. In Example \ref{example:baillotghyssimple}, we show how all constraint judgements in the typings of both a linear and a polynomial time replicated input can be verified using this approach. 
%
\begin{examp}\label{example:baillotghyssimple}
Baillot and Ghyselen \cite{BaillotGhyselen2021} provide an example of how their type system for parallel complexity of message-passing processes can be used to bound the time complexity of a linear, a polynomial and an exponential time replicated input process. We show that we can verify all judgements on constraints in the typings of the first two processes using normalized constraints. We first define the processes $P_1$ and $P_2$
\begin{align*}
    P_i \defeq\; !\inputch{a}{n,r}{}{\tick\match{n}{\asyncoutputch{r}{}{}}{m}{\newvar{r'}{\newvar{r''}{Q_i}}}}
\end{align*}
for the corresponding definitions of $Q_1$ and $Q_2$
\begin{align*}
    Q_1 \defeq&\; \asyncoutputch{a}{m,r'}{} \mid \asyncoutputch{a}{m,r''}{} \mid \inputch{r'}{}{}{\inputch{r''}{}{}{\asyncoutputch{r}{}{}}}\\
    Q_2 \defeq&\; \asyncoutputch{a}{m,r'}{} \mid \inputch{r'}{}{}{(\asyncoutputch{a}{m,r''}{} \mid \asyncoutputch{r}{}{})} \mid \asyncinputch{r''}{}{}
\end{align*}
We type $Q_1$ and $Q_2$ under the respective contexts $\Gamma_1$ and $\Gamma_2$
\begin{align*}
    \Gamma_1 \defeq&\; a : \forall_0 i.\texttt{oserv}^{i+1}(\texttt{Nat}[0,i],\texttt{ch}_{i+1}()), n : \texttt{Nat}[0,i], m : \texttt{Nat}[0,i-1],\\ &\; r : \texttt{ch}_{i}(),
     r' : \texttt{ch}_{i}(), r'' : \texttt{ch}_i()\\
    %
    \Gamma_2 \defeq&\; a : \forall_0 i.\texttt{oserv}^{i^2+3i+2}(\texttt{Nat}[0,i],\texttt{ch}_{i+1}()), n : \texttt{Nat}[0,i], m : \texttt{Nat}[0,i-1],\\ &\; r : \texttt{ch}_{i}(),
     r' : \texttt{ch}_i(), r'' : \texttt{ch}_{2i-1}()
\end{align*}
Note that in the original work, the bound on the complexity of server $a$ in context $\Gamma_2$ is $(i^2+3i+2)/2$. However, we are forced to use a less precise bound, as the multiplicative inverse is not always defined for our view of indices. Upon typing process $P_1$, we amass the judgements on the left-hand side, with corresponding judgements with normalized constraints on the right-hand side
%
\begin{align*}
    &\{i\};\emptyset\vDash i + 1 \geq 1\kern7.5em\Longleftrightarrow &  \{i\};\emptyset\vDash -i \leq 0\\
   % &\{i\};\{i \geq 1\} \vDash i-1 \leq i \kern6em\Longleftrightarrow & \{i\};\{1-i \leq 0\} \vDash -1 \leq 0\\
    &\{i\};\{i \geq 1\} \vDash i \leq i + 1\kern5em\Longleftrightarrow &  \{i\};\{1-i \leq 0\} \vDash -1 \leq 0\\
    %
    &\{i\};\{i \geq 1\} \vDash i \geq i\kern6.7em\Longleftrightarrow &  \{i\};\{1-i \leq 0\} \vDash 0 \leq 0\\
    &\{i\};\{i \geq 1\} \vDash 0 \geq 0\kern6.45em\Longleftrightarrow &  \{i\};\{1-i \leq 0\} \vDash 0 \leq 0
\end{align*}
%
As all coefficients in the normalized constraints are non-positive, each judgement is trivially satisfied, and we can verify the bound $i + 1$ on server $a$. For process $P_2$ we correspondingly have the trivially satisfied judgements
\begin{align*}
    &\{i\};\emptyset\vDash i + 1 \geq 1\kern10.3em\Longleftrightarrow &  \{i\};\emptyset\vDash -i \leq 0\\
    &\{i\};\{0 \leq 0\}\vDash i \leq i^2 + 3i + 2\kern5.2em\Longleftrightarrow &  \{i\};\{0\leq 0\}\vDash -i^2-2i-2 \leq 0\\
    %
    &\{i\};\{i \geq 1\} \vDash i \leq i + 1\kern7.9em\Longleftrightarrow &  \{i\};\{1-i \leq 0\} \vDash -1 \leq 0\\
    %
    &\{i\};\{i \geq 1\} \vDash 2i-1 \leq i^2 + 3i + 2\kern3.2em\Longleftrightarrow &  \{i\};\{1-i \leq 0\} \vDash -i^2-i-3 \leq 0\\
    &\{i\};\{i \geq 1\} \vDash i \geq i\kern9.7em\Longleftrightarrow &  \{i\};\{1-i \leq 0\} \vDash 0 \leq 0\\
    %
    &\{i\};\{i \geq 1\} \vDash 0 \geq i^2+i\kern7.5em\Longleftrightarrow &  \{i\};\{1-i \leq 0\} \vDash -i^2-i \leq 0\\
    %
    &\{i\};\{i \geq 1\} \vDash i^2+2i \geq i^2+3i+2\kern2.9em\Longleftrightarrow &  \{i\};\{1-i \leq 0\} \vDash -i-2 \leq 0
\end{align*}

\end{examp}

% In the general sense, however, verifying whether judgements on polynomial constraints are satisfied is a difficult problem, as it amounts to verifying that a constraint is satisfied under all interpretations that satisfy our set of known constraints. In Example \ref{example:needconic}, we show how a constraint that is not satisfied for all interpretations can be shown to be covered by a set of two constraints, by utilizing the transitive, multiplicative and additive properties of inequalities to combine the two constraints. More specifically, we can exploit the fact that we can generate new constraints from any set of normalized constraints by taking a \textit{conical} combination of their left-hand side indices, as we shall formalize in the following section.
% %
% \begin{examp}\label{example:needconic}
%     Given the judgement
%     \begin{align*}
%         \{i\};\{i \leq 3, 5 \leq i^2\} \vDash 5i \leq 3i^2
%     \end{align*}
%     we want to verify that constraint $5i \leq 3i^2$ is covered by the set of constraints $\{i\leq 3, 5 \leq i^2\}$. This constraint is not satisfied by all interpretations, as substituting $1$ for $i$ yields $5 \not\leq 3$. However, we can rearrange and scale the constraints $i\leq 3$ and $5\leq i^2$ as follows
%     \begin{align*}
%         i \leq 3 \iff i-3\leq 0 \implies 5i - 15 \leq 0\\
%         %
%         5\leq i^2 \iff 0 \leq i^2-5 \implies 0 \leq 3i^2-15
%     \end{align*}
%     Then it follows that
%     \begin{align*}
%         5i-15 \leq 3i^2-15 \iff 5i \leq 3i^2
%     \end{align*}
%     %More generally, we can use the transitive, multiplicative and additive properties of the inequality relation to construct new constraints from a set of known constraints, thereby verifying that some constraint does not impose new restrictions on interpretations. 
% \end{examp}
% %
% \subsection{An alternative method for verifying univariate polynomial constraints}
% In this section we restrict ourselves to constraints whose left-hand side normalized indices are monotonic univariate polynomials. These constraints have a number of convenient properties we can take advantage of to greatly simplify the process of verifying whether a constraint judgement holds. Namely, these constraints perfectly divide the index variable $i$ of the polynomial into two intervals $[-\infty,n[$ and $[n, \infty]$ where for any value of $i$ in either the first or the second interval, the constraint is satisfied and for any $i$ in the other interval, it is not. The only exception to this is for constraints where its left-hand side index is constant, in which case $n = \pm\infty$. As such, we can describe the behavior of a monotonic univariate polynomial constraint using a single value as well as a sign denoting whether the polynomial is increasing or decreasing.\\

% The point $n$ that divides the range that satisfies a constraint from the range that does not satisfy the constraint, corresponds to the root of the left-hand side of the normalized constraint. This is similar to the method described in Section \ref{sec:verifyingpolynomial}, except we now only store the range that satisfies the constraint. Here we take advantage of the fact that the polynomial is monotonic, to ensure that there is only a single root. Verifying whether a constraint $I \leq 0$ covers another constraint $J \leq 0$ then amounts to comparing the roots of $I$ and $J$. This method can be extended to non-monotonic polynomials by simply considering sequences of intervals satisfying constraints and comparing sequences of intervals satisfying a constraint.

% \begin{examp}
%     Given the three constraints
%     %
%     \begin{align*}
%         C_1&: -i^2 + 10 \leq 0\\
%         C_2&: -i + 2 \leq 0\\
%         C_3&: -5 i^3 + 80 i^2 - 427 i + 758 \leq 0
%     \end{align*}
    
%     we want to check if the judgement $\{i\};\{C_1, C_2, C_3\} \vDash -i + 4 \leq 0$ holds. We first find the non-negative roots $r_1$, $r_2$ and $r_3$ of $C_1$, $C_2$ and $C_3$. This can be done either numerically or analytically as all polynomials are of degree $\leq 4$. The roots are $r_1 = \sqrt{10} \approx 3.16$, $r_2 = 2$ and $r_3 \approx 4.59$. The root of $-i + 4$ is $4$, and so we must check if any of the roots $r_1$, $r_2$ and $r_3$ are greater than or equal $4$. In this case, $r_3 \geq 4$, and so the judgement $\{i\};\{C_1, C_2, C_3\} \vDash -i + 4 \leq 0$ holds.
% \end{examp}

%If our indices are univariate polynomials, we can express the feasible region of a constraint as a sequence of disjoint intervals. Then for a constraint $I \leq 0$ such that $I$ is in index variable $i$, the interpretation $I\{n/i\}$ with $n\in\mathbb{N}$ is satisfied when $n$ is within one of the intervals representing the feasible region of $I\leq 0$. We can utilize this to determine whether the feasible region of one constraint contains the feasible region of another by computing their intersection. This can be generalized to judgements on constraints, such that we can verify whether one constraint imposes new restrictions on possible interpretations. Then the question remains \textit{how do we find a sequence of disjoint intervals that corresponds to the feasible region of constraint $I \leq 0$?}\\ 

%We can find such a sequence of intervals for a normalized constraint, by computing the roots of the corresponding index. For polynomials of degree $4$ or less, there exists exact analytical methods to compute the roots, and we can approximate them in the general case using numerical methods.

% \begin{lstlisting}[escapeinside={(*}{*)}]
% intersectIntervals(is1, is2):
%     if is1 or is2 is empty:
%         return empty list
        
%     i1 (*$\longleftarrow$*) head(is1)
%     i2 (*$\longleftarrow$*) head(is2)
%     ires (*$\longleftarrow$*) i1 (*$\cap$*) i2
    
%     if max(i1) > max(i2):
%         is2' (*$\longleftarrow$*) tail(is2)
%         intersectedIntervals (*$\longleftarrow$*) intersectIntervals(is1, is2')
%     else:
%         is1' (*$\longleftarrow$*) tail(is1)
%         intersectedIntervals (*$\longleftarrow$*) intersectIntervals(is1', is2)
    
%     if ires is not the empty interval:
%         add ires to intersectedIntervals as the head
    
%     return intersectedIntervals
% \end{lstlisting}


% \begin{lstlisting}[escapeinside={(*}{*)}]
% containsIntervals(is1, is2):
%     if is2 is empty:
%         return true
%     else if is1 is empty:
%         return false
    
%     i1 (*$\longleftarrow$*) head(is1)
%     i2 (*$\longleftarrow$*) head(is2)
%     ires (*$\longleftarrow$*) i1 (*$\cap$*) i2
    
%     if ires = i2:
%         is2' (*$\longleftarrow$*) tail(is2)
%         return containsIntervals(is1, is2')
%     else:
%         is1' (*$\longleftarrow$*) tail(is1)
%         return containsIntervals(is1', is2)
% \end{lstlisting}


% \begin{lstlisting}[escapeinside={(*}{*)}]
% findIntervals((*$\{i\}$*), (*$I \leq 0$*)):
%     roots (*$\longleftarrow$*) sorted list of all positive roots of (*$I$*)
    
%     if roots is empty:
%         if (*$I\{0/i\}$*) (*$\leq$*) 0:
%         return singleton list of [0, (*$\infty$*)]
%     else:
%         return empty list
    
%     low (*$\longleftarrow$*) 0
%     satisfiedIntervals (*$\longleftarrow$*) empty list

%     if head(roots) = 0:
%         roots (*$\longleftarrow$*) tail(roots)

%     while roots is not empty:
%         high (*$\longleftarrow$*) head(roots)
%         mid (*$\longleftarrow$*) (*$(\texttt{low} + \texttt{high})/2$*)
        
%         if (*$I\{\texttt{mid}/i\}$*) (*$\leq$*) 0:
%             append [low,high] to satisfiedIntervals
            
%         low (*$\longleftarrow$*) high
%         roots (*$\longleftarrow$*) tail(roots)
    
%     if (*$I\{(\texttt{low}+1)/i\}$*) (*$\leq$*) 0:
%         append [low, (*$\infty$*)] to satisfiedIntervals
    
%     return satisfiedIntervals
% \end{lstlisting}


% \begin{lstlisting}[escapeinside={(*}{*)}]
% checkJudgement((*$\{i\}$*), (*$\Phi$*), (*$I\leq 0$*)):
%     satisfiedIntervals (*$\longleftarrow$*) singleton list of [0, (*$\infty$*)]
    
%     for (*$(J \leq 0) \in \Phi$*):
%         isJ (*$\longleftarrow$*) findIntervals((*$\{i\}$*), (*$J\leq 0$*))
%         satisfiedIntervals (*$\longleftarrow$*) intersectIntervals(satisfiedIntervals, isJ)
        
%     isI (*$\longleftarrow$*) findIntervals((*$\{i\}$*), (*$I \leq 0$*))
        
%     return containsInterval(isI, satisfiedIntervals)
% \end{lstlisting}
\section{Examples of invalid configurations}
The following examples are written in the format $\conf{E, a}$, where $E$ is an editor expression and $a$ is the AST on which we apply the editor expression. \\

In equation \ref{condsubproblem} we show how conditioned substitution can cause problems.
\begin{equation}
    \conf{\left(@\texttt{break} \Rightarrow \replace{\texttt{break}}\right) \ggg \texttt{child}\; 1,\; \lambda x.\hole\; \cursor{\breakpoint{c}}} \label{condsubproblem}
\end{equation}
 In the example we check if the cursor is at a breakpoint, and since the check is true we \textit{toggle} the breakpoint thereby making the following \texttt{child} 1 command problematic. The constant c cannot have a child which means this configuration would cause a run-time error. \\
 
In equation \ref{parentproblem} we show how using the \texttt{parent} command can cause problem when the root is unknown.
\begin{equation}
    \conf{\left(\lozenge\texttt{hole} \Rightarrow \texttt{parent}\right) \ggg \texttt{parent},\; \cursor{\lambda x.\hole}\; c} \label{parentproblem}
\end{equation}
In the example we first check if there is a hole in some subtree of the current cursor. This condition holds and we therefore evaluate the \texttt{parent} command resulting in the AST $\cursor{\lambda x.\hole\; c}$. When the next \texttt{parent} command is evaluated we have a run-time error since we are already situated at the root.\\

In equation \ref{astproblem} we show how an editor expression can result in an AST that would cause a run-time error when evaluated.
\begin{equation}
    \conf{\left(\neg\Box(\texttt{lambda}\; x) \Rightarrow \texttt{child}\; 1\right) \ggg \replace{\texttt{var}\; x}.\texttt{eval},\; \cursor{\lambda x.\hole}\; c} \label{astproblem}
\end{equation}
In the example we first check if it is \textbf{not} necessary that the subtree of the cursor contains a lambda expression. This condition does not hold since it is necessary. Since the condition does not hold we do not evaluate the \texttt{child} 1 command, which means the following substitution of \texttt{var} x is problematic. The substitution results in the AST $\cursor{\texttt{var}\; x}\; c$, which causes a run-time error when the command \texttt{eval} is evaluated, since the left child of the function application is no longer a function.
%
\section{Over-approximations}
As we cannot determine statically whether a condition holds, we establish over-approximations to ensure run-time errors cannot occur in well-typed configurations. As equation \ref{parentproblem} shows, conditioned expressions can result in loss of information about the cursor location. As such, we enforce the cursor \textit{depth} in the tree to be the same before and after a conditioned expression. Furthermore, the first cursor movement in a conditioned expression must be a \texttt{child} prefix. As equation \ref{condsubproblem} shows, conditioned substitution also results in loss of information. Thus, we can no longer guarantee that subsequent substitution at a deeper level is well-typed. Similarly, we no longer know of the structure of the subtree, such that we must condition \texttt{child} prefixes.\\

The above discussion leads to the following list of over-approximations:
\begin{itemize}
    \item In conditioned and recursive expressions, the cursor depth must be the same before and after.
    \item In conditioned and recursive expressions, only the subtree encapsulated by the cursor is accessible.
    \item After conditioned substitution, subsequent substitution at a deeper level is no longer valid, and the \texttt{child} prefix command must be conditioned.
\end{itemize}
%
\section{AST type rules}
\begin{table*}[htp]
    \centering
    \begin{align*}
        \runa{t-var} &\; \infrule{\Gamma_a\left(x\right)=\tau}{\Gamma_a \vdash x : \tau}\\
        %
        \runa{t-const} &\; \infrule{}{\Gamma_a \vdash c : b}\\
        %
        \runa{t-app} &\; \infrule{\Gamma_a \vdash a_1 : \tau_1 \rightarrow \tau_2 \quad \Gamma_a \vdash a_2 : \tau_1}{\Gamma_a \vdash a_1\; a_2 : \tau_2}\\
        %
        \runa{t-lambda} &\; \infrule{\Gamma_a\left[x \mapsto \tau_1\right] \vdash a : \tau_2}{\Gamma_a \vdash \lambda x:\tau_1.a : \tau_1 \rightarrow \tau_2} \\
        %
        \runa{t-break} &\; \infrule{\Gamma_a \vdash a : \tau}{\Gamma_a \vdash \breakpoint{a} : \tau} \\
        %
        \runa{t-hole} &\; \infrule{}{\Gamma_a \vdash \left(\hole : \tau\right) : \tau}
        %
    \end{align*}
    \caption{Type rules for abstract syntax trees.}
    \label{tab:typerules}
\end{table*}

%\section{Type context format}
%Here, we propose a format for type contexts of editor expressions. The context of an editor expression could be a triple $\Psi = (\Gamma_a, \tau, \Gamma)$, where $\Gamma_a$ is the type context for the subtree encapsulated by the cursor, $\tau$ is the type of the subtree and $\Gamma$ is a function or map from prefix command types to editor expression contexts. That is, contexts for editor expressions are recursive. Say we have context $(\Gamma_a, \tau, \Gamma)$. Upon a $\texttt{child}\; 1$ prefix, we \textit{look up} $\texttt{one}$ in $\Gamma$. If $\Gamma(\texttt{one}) = undef$, the expression is not well-typed. Otherwise, we evaluate the prefixed expression in the new context $\Gamma(\texttt{one})$.\\

%We construct the initial context based on the AST in the configuration $\conf{E,\; a}$. Upon a substitution prefix, we modify the context, upon a child or parent prefix, we \textit{move} in the context, and upon a conditioned or recursive expression, we set some of the bindings to $undef$: $\Gamma(T)=undef$.\\

%$\Gamma = T_1 : \Psi_1,...,T_n : \Psi_n$ \\
%$\Psi = (\Gamma_a, \tau, \Gamma)$
%Γ = T1 : Ψ1,..,Tn : Ψn
%Ψ = (Γa, τ, Γ)

\section{Experimental type system}

In this section, we introduce a type system for our editor-calculus. For the type system, we introduce the syntactic categories $\tau \in \mathbf{ATyp}$ to denote types of AST nodes, $T \in \mathbf{CTyp}$ to denote \textit{child} types, and p $\in \mathbf{Pth}$ to denote AST paths.
%
\begin{align*}
    \tau ::=&\; b \mid \tau_1 \rightarrow \tau_2 \mid \breakpoint{\tau} \mid \texttt{indet}\\
    T ::=&\; \texttt{one} \mid \texttt{two}\\
    p ::=&\; p\; T \mid \epsilon
\end{align*}

In addition to the basic and arrow types in $\mathbf{ATyp}$, we include a type for breakpoints, $\breakpoint{\tau}$, and a type to denote indeterminate types, \texttt{indet}. We use $\mathbf{Pth}$ to denote paths in an AST by storing a sequence of \textbb{one} and \textbb{two} which denote if the path goes through the first or second child.\\

We define two sets for contexts in our type system. The first context, $\mathbf{ACtx}$, stores type bindings for variables in the AST. The second context, $\mathbf{ECtx}$, stores, for all available paths so far, a pair of an AST context and the type of the node at the end of the path. We use $\Gamma_a \in \mathbf{ACtx}$ and $\Gamma_e \in \mathbf{ECtx}$ as metavariables for the two contexts. To check if a path $p$ is available in a context $\Gamma_e$, we use the notion $\Gamma_e(p) \neq \text{undef}$. $\mathbf{ACtx}$ and $\mathbf{ECtx}$ are thus defined as the following.
%
\begin{align*}
\mathbf{ACtx} &= \mathbf{Var} \rightharpoonup \mathbf{ATyp}\\
\mathbf{ECtx} &= \mathbf{Pth} \rightharpoonup \left(\mathbf{ACtx} \times \mathbf{ATyp}\right)
\end{align*}

To support our type system, we modify the syntax for AST node modifications by including type annotations for application, abstraction and holes. The new syntax thus becomes the following.
%
\begin{align*}
  D ::= \; & \texttt{var}\;x \mid \texttt{const}\;c \mid \texttt{app} : \tau_1 \rightarrow \tau_2, \tau_1 \mid \texttt{lambda}\; x : \tau_1 \rightarrow \tau_2 \mid \texttt{break} \mid \texttt{hole} : \tau
\end{align*}

To support breakpoint types, we introduce the notion of type consistency into our typesystem. The purpose of consistency in our type system is to ensure breakpoints types are consistent with their respective type, as defined below.
%
\begin{definition}{(Type consistency)}
    We define two types $\tau_1, \tau_2$ to be \textit{consistent}, denoted $\tau_1 \sim \tau_2$, by the following rules.
    \begin{align*}
        \runa{cons-1} \hspace{-1cm}
        \infrule{}{\tau \sim \tau} \hspace{-1cm}
        \runa{cons-2} \hspace{-1cm}
        \infrule{}{\breakpoint{\tau} \sim \tau} \hspace{-1cm}
        \runa{cons-3} \hspace{-1cm}
        \infrule{}{\tau \sim \breakpoint{\tau}} \hspace{-1cm}
        \runa{cons-4}
        \infrule{\tau_1 \sim \tau_1' \quad \tau_2 \sim \tau_2'}{(\tau_1 \rightarrow \tau_2) \sim (\tau_1' \rightarrow \tau_2')}
    \end{align*}
\end{definition}


\begin{table*}[htp]
    \centering
    \begin{align*}
        \runa{ctx-split-1}&\; \infrule{}{\emptyset = p \left(\emptyset\; \circ\; \emptyset\right)}\\
        \runa{ctx-split-2}&\; \infrule{\Gamma_e = p \left({\Gamma_e}_1\; \circ\; {\Gamma_e}_2\right)}{\Gamma_e,\; p\; T_1..T_n: (\Gamma_a,\; \tau) = p \left(\left({\Gamma_e}_1,\; p\; T_1..T_n: (\Gamma_a,\; \tau)\right)\; \circ\; {\Gamma_e}_2\right)}\\
        \runa{ctx-split-3}&\; \infrule{p_1 \neq p_2 \quad \Gamma_e = p_2 \left({\Gamma_e}_1\; \circ\; {\Gamma_e}_2\right)}{\Gamma_e,\; p_1\; T_1..T_n: (\Gamma_a,\; \tau) = p_2 \left({\Gamma_e}_1\; \circ\; \left({\Gamma_e}_2,\; p_1\; T_1..T_n: (\Gamma_a,\; \tau)\right)\right)}\\
        %
        \runa{ctx-update-1}&\; \infrule{}{\Gamma_e = \Gamma_e + \emptyset}\\
        \runa{ctx-update-2}&\; \infrule{\Gamma_e = \left({\Gamma_e}_1,\; p: ({\Gamma_a}_2,\; \tau_2)\right) + {\Gamma_e}_2}{\Gamma_e,\; p: ({\Gamma_a}_1,\; \tau_1) = \left({\Gamma_e}_1,\; p: ({\Gamma_a}_2,\; \tau_2)\right) + {\Gamma_e}_2}\\
        \runa{ctx-update-3}&\; \infrule{\Gamma_e = {\Gamma_e}_1 + {\Gamma_e}_2}{\Gamma_e,\; p: (\Gamma_a,\; \tau) = {\Gamma_e}_1 + \left({\Gamma_e}_2,\; p: (\Gamma_a,\; \tau)\right)}
    \end{align*}
    \caption{Context split and context update for editor contexts.}
    \label{tab:context}
\end{table*}
% We define \textit{type contexts}, $\Gamma_e$ in Table \ref{tab:context} as a mapping from a path $p$ to a pair consisting of an AST context $\Gamma_a$ and AST type $\tau$. We denote the $\Gamma_e, p : (\Gamma_a, \tau)$ as the type context equal to the paths not in the domain of map $\Gamma_e$ except for $p$, where $\Gamma_e(p) = (\Gamma_a, \tau)$. For type contexts we introduce the concept of \textit{context splitting} on a path in terms of $\Gamma_e$ maintained through two sub-contexts $\Gamma_{e1}$ and $\Gamma_{e2}$. For this we require a split-operation $\circ$, defined for two sub-contexts on a path as $\Gamma_e = p(\Gamma_{e1}\; \circ \; \Gamma_{e2})$. Notice the empty context is defined with the symbol $\emptyset$ as in \runa{ctx-split-1}. In rule \runa{ctx-split-2} we have that $p$ is in $\Gamma_{e1}$, but not in $\Gamma_{e2}$. Thus, $p$ is not in $\Gamma = \Gamma_{e1}\; \circ \; \Gamma_{e2}$, which is similarly done for the \runa{ctx-split-3} in terms of $\Gamma_{e1}$.\\

Next we introduce the notion of \textit{context updates} to update bindings in a context with new types for the associated path $p$. We use the addition operator $+$, to denote sum-context $\Gamma$ of two compatible type contexts $\Gamma_{e1}$ and $\Gamma_{e2}$. The rules require linear paths to not have bindings exist in another context. Thus, we can only update a context $\Gamma_{e2}$ iff no bindings for a given path is in context $\Gamma_{e1}$. In rule \runa{ctx-update-2} we have bindings in $\Gamma_{e1}$, which means we cannot add bindings to $\Gamma_{e2}$. However, in rule \runa{ctx-update-3} we allow path bindings in $\Gamma_{e2}$ since no such bindings are in context $\Gamma_{e1}$.

% \begin{equation}
%     depth(e) = \left\{
%         \begin{array}{ll}
%             depth(E) + 1            & \quad if e = (\texttt{child}\; n).E \\
%             depth(E) - 1            & \quad if e = \texttt{parent}.E\\
%             depth(E_1) + depth(E_2) & \quad if e = E_1 \ggg E_2\\
%             depth(E)                & \quad if e = \texttt{rec}\; x.E\\
%             depth(E)                & \quad if e = \pi.E\\
%             0                       & \quad otherwise
%         \end{array}
%     \right.
% \end{equation}

\begin{definition}{(Relative cursor depth)}
    We define the function $depth : \mathbf{Edt} \rightarrow \mathbb{Z}$, from the set of atomic editor expression to the set of integers.
    \begin{align*}
    depth((\texttt{child}\; n).E) &= depth(E) + 1 \\
    depth(\texttt{parent}.E) &= depth(E) - 1 \\
    depth(E_1 \ggg E_2) &= depth(E_1) + depth(E_2) \\
    depth(\texttt{rec}\; x.E) &= depth(E) \\
    depth(\pi.E) &= depth(E) \\
    depth(E) &= 0 
\end{align*}
\end{definition}
The $depth$ function statically analyses the structure of an editor expression to determine the relative depth of the cursor after evaluation of the expression. This function is used to make sure the position of the cursor before and after evaluation of an expression is the same. As the function performs a static analysis, we do not consider conditioned subexpressions. Later, in the type rules, we will see why we can safely ignore conditioned subexpressions. \\


% Next we define the function $match : \mathbf{Aam} \times \mathbf{ACtx} \times \mathbf{ATyp} \rightarrow \{tt, f\!\!f\}$. This function returns true if the type of the given AST modification $D$, is equal to the given AST type $\tau$.  
% \begin{align*}
%     match(\texttt{var}\; x,\;\Gamma_a,\;\tau) &= \left\{\begin{matrix}
%  tt & \text{if}\; \Gamma_a(x) = \tau\\ 
%  f\!\!f & \text{otherwise}
% \end{matrix}\right.\\
%     match(\texttt{const}\; c,\;\Gamma_a,\; b) &= tt\\
%     match(\texttt{app} : \tau_1 \rightarrow \tau_2,\; \tau_1,\;\Gamma_a,\; \tau_2) &= tt\\
%     match(\texttt{lambda}\; x : \tau_1 \rightarrow \tau_2,\;\Gamma_a,\; \tau_1 \rightarrow \tau_2) &= tt\\
%     match(\texttt{break},\;\Gamma_a,\; \tau) &= tt\\
%     match(\texttt{hole} : \tau,\;\Gamma_a,\; \tau) &= tt\\
%     match(D,\; \Gamma_a,\; \tau) &= f\!\!f
% \end{align*}

%\begin{equation*}
%    %context : \left(\mathbf{Aam} \times \mathbf{ACtx}\right) \rightharpoonup %\left(\left(\mathbf{Pth} \rightarrow \left(\left(\mathbf{Var} \rightharpoonup %\mathbf{ATyp}\right) \times \mathbf{ATyp}\right)\right) \cup \{error\}\right)
    %context : \left(\mathbf{Aam} \times \mathbf{ACtx} \times \mathbf{Pth} \right) %\rightharpoonup \mathbf{ECtx}
%\end{equation*}
%\begin{align*}
% context(\texttt{const}\; c,\; \Gamma_a,\; p) =&\; \emptyset\\
%  context(\texttt{hole} : \tau,\; \Gamma_a,\; p) =&\; \emptyset\\
%context(\texttt{var}\; x,\; \Gamma_a,\; p) =&\; \emptyset\\
 %context((\texttt{app} : \tau_1 \rightarrow \tau2,\; \tau_1),\; \Gamma_a,\; p) =&\; %\emptyset,\; p\; \texttt{one} : (\Gamma_a,\; \tau_1 \rightarrow \tau_2),\; p\; \texttt{two} : %(\Gamma_a,\; \tau_1)\\
 %context(\texttt{lambda}\; x : \tau_1 \rightarrow \tau_2,\; \Gamma_a,\; p) =&\; \emptyset,\; %p\; \texttt{one} : ((\Gamma_a,\; x : \tau_1),\; \tau_2)
%\end{align*}
%
%

We define functions \textit{limits} and \textit{follows} to analyze which cursor movement is safe given a condition holds. \textit{limits} finds the set of possible AST node modifiers, on which the cursor may sit, given the condition holds. \textit{follows} gives a set of editor type context bindings guaranteed to be safe, given the cursor sits on AST node modifier $D$. Note that the AST type context is empty and that the node type is $\texttt{indet}$, as we cannot determine such information based on a condition. Thus, besides toggling of breakpoints, substitution is not well-typed at path $p$ if $\Gamma_e(p)=(\emptyset,\; \texttt{indet})$. We can combine functions \textit{limits} and \textit{follows} to provide additional bindings to the editor type context of a conditioned expression $\phi \Rightarrow E$. The intersection of \textit{follows} applied to each AST node modifier $D$ in the set $limits(\phi)$ is the set of bindings guaranteed to be safe, given $\phi$ holds.

\theoremstyle{definition}
\begin{definition}{(Condition constraints)}
We define a function $limits: \mathbf{Eed} \rightarrow \mathcal{P}(\mathbf{Aam})$ from the set of conditions to the power set of the set of AST node modifiers. We assume conditions are in conjunctive normal form.
\begin{align*}
    limits(@D)=&\;\{D\}\\
    limits(\neg @D)=&\;\mathbf{Aam}\setminus \{D\}\\
    limits(\lozenge D)=&\;\{D\} \cup \{\texttt{app},\; \texttt{lambda}\; x,\; \texttt{break}\}\\
    limits(\neg \lozenge D)=&\;\mathbf{Aam}\setminus \{D\}\\
    limits(\Box D)=&\;\{D\} \cup \{\texttt{app},\; \texttt{lambda}\; x,\; \texttt{break}\}\\
    limits(\neg \Box D)=&\;\mathbf{Aam}\setminus \{D\}\\
    limits(\phi_1 \land \phi_2)=&\;limits(\phi_1) \cap limits(\phi_2)\\
    limits(\phi_1 \lor \phi_2)=&\;limits(\phi_1) \cup limits(\phi_2)
\end{align*}
\end{definition}


\theoremstyle{definition}
\begin{definition}{(Safe movement)}
We define a function $follows: \mathbf{Aam} \times \mathbf{Pth} \rightarrow \mathcal{P}\left(\mathbf{Pth} \times \left(\mathbf{ACtx} \times \mathbf{ATyp}\right)\right)$ from the set of pairs of AST node modifiers and paths to the power set of editor context bindings.
\begin{align*}
    \textit{follows}(\texttt{var}\; x,\; p)=&\; \emptyset\\
    \textit{follows}(\texttt{const}\; c,\; p)=&\; \emptyset\\
    \textit{follows}(\texttt{app},\; p)=&\; \{p\; \texttt{one} : (\emptyset,\; \texttt{indet}),\; p\; \texttt{two} : (\emptyset,\; \texttt{indet})\}\\
    \textit{follows}(\texttt{lambda}\; x,\; p)=&\; \{p\; \texttt{one} : (\emptyset,\; \texttt{indet})\}\\
    \textit{follows}(\texttt{break},\; p)=&\; \{p\; \texttt{one} : (\emptyset,\; \texttt{indet})\}\\
    \textit{follows}(\texttt{hole},\; p)=&\; \emptyset
\end{align*}
\end{definition}

%
%
We now introduce the type rules for editor expressions. Type rules for substitution are shown in table \ref{tab:typerulesv2sub} and the remaining rules are shown in table \ref{tab:typerulesv2}. The \texttt{child} n prefix is handled by \runa{t-child-1} and \runa{t-child-2}. Here we check that the cursor movement is viable by looking up the new path in $\Gamma_e$. Notice that the remaining editor expression $E$, is evaluated using the new path. The \texttt{parent} prefix is handled similarly in \runa{t-parent} with the exception being that we deconstruct the path instead of building it. When using recursion we require that the depth of the cursor is unchanged after evaluating the expression. We ensure this in \runa{t-rec} with the side condition $depth(E) = 0$. Similarly, \runa{t-cond} utilizes the same side condition to ensure that the cursor is unaffected by whether the condition holds or not. Notice here that evaluation of the conditioned expression is limited by what can follow the condition if it holds, denoted by $\delta$. Sequential composition is handled by the type rule \runa{t-seq}. Here we split the type context into $\Gamma_{e1}$, which contains information about the current subtree, and $\Gamma{e2}$, which contains information about the rest of the tree. This split ensures that the potentially hazardous evaluation of $E_1$ is kept separate from the evaluation of $E_2$.\\

\begin{table*}[htp]
    \centering
    \begin{align*}
        %
        \runa{t-eval} &\; \infrule{p,\; \Gamma_e \vdash E : ok}{p,\; \Gamma_e \vdash \texttt{eval}.E : ok}\\
        %
        \runa{t-child-1}&\; \infrule{\Gamma_e(p\; \texttt{one}) \neq \text{undef} \quad p\; \texttt{one},\; \Gamma_e \vdash E : ok}{p,\; \Gamma_e \vdash \left(\texttt{child}\; 1\right).E : ok}\\
        %
        \runa{t-child-2}&\; \infrule{\Gamma_e(p\; \texttt{two}) \neq \text{undef} \quad p\; \texttt{one},\; \Gamma_e \vdash E : ok}{p,\; \Gamma_e \vdash \left(\texttt{child}\; 2\right).E : ok}\\
        %
        \runa{t-parent}&\; \infrule{\Gamma_e(p) \neq \text{undef} \quad p,\; \Gamma_e \vdash E : ok}{p\; T,\; \Gamma_e \vdash \texttt{parent}.E : ok}\\
        %
        \runa{t-rec} &\; \condinfrule{p,\; \Gamma_e \vdash E : ok}{p,\; \Gamma_e \vdash \texttt{rec} x.E : ok}{\text{if}\; depth(E) = 0}\\
        %
        \runa{t-cond} &\; \condinfrule{p,\; \Gamma_e + \delta \vdash E : ok}{p,\; \Gamma_e \vdash \phi \Rightarrow E : ok}{\begin{align*}
            \text{if}\; &depth(E) = 0\;\\
            \text{and}\; &\delta = \bigcap_{D \in limits(\phi)}follows(D,\; p)\\
        \end{align*}}\\
        %
        \runa{t-seq} &\; \condinfrule{p,\; {\Gamma_e}_1 \vdash E_1 : ok \quad p,\; {\Gamma_e}_2 \vdash E_2 : ok}{p,\; \Gamma_e \vdash E_1 \ggg E_2 : ok}{\text{where}\; \Gamma_e = p\; ({\Gamma_e}_1\; \circ\; {\Gamma_e}_2)}\\
        %
        \runa{t-ref} &\; \infrule{}{p,\;\Gamma_e \vdash x : ok}\\
        %
        \runa{t-nil} &\; \infrule{}{p,\;\Gamma_e \vdash \mathbf{0} : ok}
    \end{align*}
    \caption{Type rules for editor expressions.}
    \label{tab:typerulesv2}
\end{table*}
%
%
Table \ref{tab:typerulesv2sub} shows the type rules for substitution. For substitution to be well-typed, the AST node type $\tau$ in the type context binding associated with the current path $p$ must be consistent with the type of the AST node modifier to be inserted. In \runa{t-sub-var}, we handle the special case where we insert a variable reference $x$. For this to be well-typed, a binding $\Gamma_a(x)=\tau'$ must exist, such that $\consistent{\tau}{\tau'}$. Note that substitution replaces a subtree of the AST. Thus, the bindings in the editor type context with paths starting with $p$ are no longer valid. Therefore, we split the type context on path $p$, such that $\Gamma_e = p\left({\Gamma_e}_1\;\circ\;{\Gamma_e}_2\right)$, and evaluate the prefixed expression $E$ in the type context ${\Gamma_e}_2$. That is, the type context containing all bindings of $\Gamma_e$ not starting with $p$. Note that the binding with path exactly $p$ is in both ${\Gamma_e}_1$ and ${\Gamma_e}_2$, however. We add bindings to ${\Gamma_e}_2$ in rules $\runa{t-sub-app}$ and $\runa{t-sub-abs}$. Particularly, we expand the AST type context upon substitution for an abstraction.\\

We treat substitution of breakpoints differently, as we can either toggle breakpoints on or off. Furthermore, we do not replace the subtree upon substitution for breakpoints. Instead, we must modify the bindings with paths starting with $p$, to either include or remove a $\texttt{one}$. Additionally, we change the type in the binding at the current path $p$ to indicate whether it has a breakpoint. Note that we toggle off the breakpoint if the type is of the form $\breakpoint{\tau}$, and toggle it on otherwise. Thus, the type indicates the structure of the tree.
%
%
\begin{table}
    \begin{flalign*}
        %
        \runa{t-sub-var} &\; \condinfrule{\Gamma_e(p)=(\Gamma_a,\;\tau) \quad \Gamma_a(x) = \tau' \quad \consistent{\tau}{\tau'} \quad p,\;{\Gamma_e}_2 \vdash E : ok}{p,\; \Gamma_e \vdash \replace{\texttt{var}\; x}.E : ok}{\text{where}\; \Gamma_e = p\; ({\Gamma_e}_1\; \circ\; {\Gamma_e}_2)} \\
        %
        \runa{t-sub-const} &\; \condinfrule{\Gamma_e(p)=(\Gamma_a,\;b) \quad p,\;{\Gamma_e}_2 \vdash E : ok}{p,\; \Gamma_e \vdash \replace{\texttt{const}\; c}.E : ok}{\text{where}\; \Gamma_e = p\; ({\Gamma_e}_1\; \circ\; {\Gamma_e}_2)}\\
        %
        \runa{t-sub-app} &\; \condinfrule{\Gamma_e(p)=(\Gamma_a,\; \tau_2') \quad \consistent{\tau_2}{\tau_2'} \quad p,\; \Gamma_e' \vdash E : ok}{p,\; \Gamma_e \vdash \replace{\texttt{app} : \tau_1 \rightarrow \tau_2,\; \tau_1}.E : ok}{\begin{align*}
            &\text{where}\; \Gamma_e = p\; ({\Gamma_e}_1\; \circ\; {\Gamma_e}_2)\;\\
            &\text{and}\; \Gamma_e' = {\Gamma_e}_2,\; p\; \texttt{one} : (\Gamma_a,\; \tau_1 \rightarrow \tau_2),\; p\; \texttt{two} : (\Gamma_a,\; \tau_1)
        \end{align*}}\\
        %
        \runa{t-sub-abs} &\; \condinfrule{\Gamma_e(p)=(\Gamma_a,\; \tau_1' \rightarrow \tau_2') \quad \consistent{\tau_1 \rightarrow \tau_2}{\tau_1' \rightarrow \tau_2'} \quad p,\; \Gamma_e' \vdash E : ok}{p,\; \Gamma_e \vdash \replace{\texttt{lambda}\; x : \tau_1 \rightarrow \tau_2}.E : ok}{\begin{align*}
        &\text{where}\;\Gamma_e = p\; ({\Gamma_e}_1\; \circ\; {\Gamma_e}_2)\\
        &\text{and}\;\Gamma_e' = {\Gamma_e}_2, p\; \texttt{one} : ((\Gamma_a,\; x : \tau_1),\; \tau_2)\end{align*}} \\
        %
        %\runa{t-sub} &\; \infrule{match(D,\; \Gamma_a,\; \tau) = tt \quad p,\;\Gamma_e' \vdash %E : ok}{p,\;\Gamma_e \vdash \replace{D}.E : ok} \\
        %&\text{if}\; D \neq \texttt{break}\\
        %&\text{and}\; \Gamma_e(p)=(\Gamma_a,\;\tau) \\
        %&\text{and}\; \Gamma_e = p\; ({\Gamma_e}_1\; \circ\; {\Gamma_e}_2)\\
        %&\text{and}\; \Gamma_e' = {\Gamma_e}_2 + context(D,\; \Gamma_a)\\
        %
        \runa{t-sub-break-1} &\; \infrule{\Gamma_e(p)=(\Gamma_a,\; \breakpoint{\tau}) \quad p,\; \Gamma_e' \vdash E : ok}{p,\; \Gamma_e \vdash \replace{\texttt{break}} : ok} \\
        &\text{where}\; \Gamma_e = p\; ({\Gamma_e}_1\; \circ\; {\Gamma_e}_2)\\
        &\text{and}\; {\Gamma_e}_1 = \emptyset,\; p\; \texttt{one}\; T_1..T_{n_1} : ({\Gamma_a}_1,\; \tau_1),..,p\; \texttt{one}\; T_1..T_{n_m} : ({\Gamma_a}_m,\; \tau_m)\\
        &\text{and}\; {\Gamma_e}_1' =\emptyset,\; p\; T_1..T_{n_1} : ({\Gamma_a}_1,\; \tau_1),..,p\; T_1..T_{n_m} : ({\Gamma_a}_m,\; \tau_m)\\
        &\text{and}\; \Gamma_e' = \left({\Gamma_e}_2 + {\Gamma_e}_1'\right),\; p : (\Gamma_a,\; \tau)\\
        %
        \runa{t-sub-break-2} &\; \infrule{\Gamma_e(p)=(\Gamma_a,\;\tau)\quad  p,\; \Gamma_e' \vdash E : ok}{p,\; \Gamma_e \vdash \replace{\texttt{break}} : ok} \\
        &\text{where}\; \Gamma_e = p\; ({\Gamma_e}_1\; \circ\; {\Gamma_e}_2)\\
        &\text{and}\; {\Gamma_e}_1 =\emptyset,\; p\; T_1..T_{n_1} : ({\Gamma_a}_1,\; \tau_1),..,p\; T_1..T_{n_m} : ({\Gamma_a}_m,\; \tau_m)\\
        &\text{and}\; {\Gamma_e}_1' = \emptyset,\; p\; \texttt{one}\; T_1..T_{n_1} : ({\Gamma_a}_1,\; \tau_1),..,p\; \texttt{one}\; T_1..T_{n_m} : ({\Gamma_a}_m,\; \tau_m)\\
        &\text{and}\; \Gamma_e' = \left({\Gamma_e}_2 + {\Gamma_e}_1'\right),\; p : (\Gamma_a,\; \breakpoint{\tau})\\
        %
        \runa{t-sub-hole} &\; \condinfrule{\Gamma_e(p)=(\Gamma_a,\;\tau') \quad \consistent{\tau}{\tau'} \quad p,\;{\Gamma_e}_2 \vdash E : ok}{p,\; \Gamma_e \vdash \replace{\texttt{hole} : \tau}.E : ok}{\text{where}\; \Gamma_e = p\; ({\Gamma_e}_1\; \circ\; {\Gamma_e}_2)}
        %
    \end{flalign*}
    \caption{Type rules for substitution.}
    \label{tab:typerulesv2sub}
\end{table}

%\begin{table*}[htp]
%    \centering
%    \begin{align*}
        %%
        %\runa{t-eval} &\; \infrule{p,\; \Gamma_e \vdash E : ok \dashv p',\; \Gamma_e'}{p,\; \Gamma_e \vdash \texttt{eval}.E : %ok \dashv p',\; \Gamma_e'}\\
        %%
        %\runa{t-sub} &\; \infrule{T=\tau \quad p,\;\Gamma_e'' \vdash E : ok \dashv p',\;\Gamma_e'}{p,\;\Gamma_e \vdash %\replace{D}.E : ok \dashv p',\;\Gamma_e'} \\
        %&\text{where}\; \Gamma_e(p)=(\Gamma_a,\;\tau) \\
        %&\text{and}\; T = type(D,\;\Gamma_a) \\
        %&\text{and}\; \Gamma_e = p\; ({\Gamma_e}_1\; \circ\; {\Gamma_e}_2)\\
        %&\text{and}\; \Gamma_e'' = {\Gamma_e}_1 + context(D,\; \Gamma_a)\\
        %%
        %\runa{t-child-1}&\; \infrule{\Gamma_e(p\; \texttt{one}) \neq undef \quad p,\; \texttt{one},\; \Gamma_e \vdash E : ok %\dashv p',\; \Gamma_e'}{p,\; \Gamma_e \vdash \left(\texttt{child}\; 1\right).E : ok \dashv p',\; \Gamma_e'}\\
        %%
        %\runa{t-child-2}&\; \infrule{\Gamma_e(p\; \texttt{two}) \neq undef \quad p,\; \texttt{one},\; \Gamma_e \vdash E : ok %\dashv p',\; \Gamma_e'}{p,\; \Gamma_e \vdash \left(\texttt{child}\; 2\right).E : ok \dashv p',\; \Gamma_e'}\\
        %%
        %\runa{t-parent}&\; \infrule{\Gamma_e(p) \neq undef \quad p,\; \Gamma_e \vdash E : ok \dashv p',\; \Gamma_e'}{p\; T,\; %\Gamma_e \vdash \texttt{parent}.E : ok \dashv p',\; \Gamma_e'}\\
        %%
        %\runa{t-rec} &\; \condinfrule{p,\; {\Gamma_e}_1 \vdash E : ok \dashv p,\; \Gamma_e'}{p,\; \Gamma_e \vdash \texttt{rec} %x.E : ok \dashv p,\; {\Gamma_e}_2}{\text{where}\; \Gamma_e = p\; ({\Gamma_e}_1\; \circ\; {\Gamma_e}_2)}\\
        %%
        %\runa{t-seq} &\; \infrule{p,\; \Gamma_e \vdash E_1 : ok \dashv p'',\; \Gamma_e'' \quad p'',\; \Gamma_e'' \vdash E_2 : %ok \dashv p',\; \Gamma_e'}{p,\; \Gamma_e \vdash E_1 \ggg E_2 : ok \dashv p',\; \Gamma_e'}\\
        %%
        %\runa{t-cond} &\; \infrule{p,\; {\Gamma_e}_1 + \delta \vdash E : ok \dashv p,\; \Gamma_e'}{p,\; \Gamma_e \vdash \phi %\Rightarrow E : ok \dashv p,\; {\Gamma_e}_2}\\
%        &\text{where}\; \Gamma_e = p\; ({\Gamma_e}_1\; \circ\; {\Gamma_e}_2)\\
%        &\text{and}\; \delta = \bigcap_{D \in limits(\phi)}follows(D)\\
%        %
%        \runa{t-ref} &\; \infrule{}{p,\;\Gamma_e \vdash x : ok \dashv p,\;\Gamma_e}\\
%        %
%        \runa{t-nil} &\; \infrule{}{p,\;\Gamma_e \vdash \mathbf{0} : ok \dashv p,\;\Gamma_e}\\
%    \end{align*}
%    \caption{Type rules for editor expressions.}
%    \label{tab:typerules}
%\end{table*}

\begin{theorem} (Subject reduction)
If $\Gamma_e, \;\Gamma_a \vdash \conf{E,\;a} : ok$ and $\conf{E, a} \xrightarrow{\alpha} \conf{E', a'}$ then $\Gamma_e, \;\Gamma_a \vdash \conf{E',\;a'} : ok$.
\end{theorem}

We define \textit{well-typedness} of a configuration $\conf{E,\;a}$ by the following rule: \\
$\condinfrule{\Gamma_a \vdash a : \tau \quad p,\; \Gamma_e \vdash E : ok}{\Gamma_e, \;\Gamma_a \vdash \conf{E,\;a} : ok}{\begin{align*}
        &\text{where}\;\\
        &\text{and}\;\end{align*}}$
        
        

\section{Conclusion}\label{ch:conclusion}
In this paper, we have explored the challenges of implementing both a type checker and type inference for the type system for parallel complexity of $\pi$-calculus processes by Baillot and Ghyselen \cite{BaillotGhyselen2021}. In this chapter, we first present and discuss our results, after which we discuss limitations of our implementations. Finally, we consider future work, and discuss how some of the limitations may be relaxed.
%
\subsection{Results}
Type checking and type inference are limited by our ability to respectively verify or satisfy constraint judgements. As such judgements are universally quantified over index variables, this becomes non-trivial. We can reduce verification of constraint judgements to linear programming, where a constraint judgement holds if there is no solution to the corresponding linear program. We have also shown that certain polynomial constraint judgements can be reduced to linear constraint judgements, enabling type checking of some processes with polynomial time behavior. We have introduced algorithmic type rules for type checking, and proved their soundness with respect to parallel complexity.\\ %To ensure we maintain the subject reduction property, we introduced combined complexities and the associated function \textit{basis}, such that we effectively defer checks of constraint judgements introduced by parallel composition until a later time. The type checker has been proved sound.\\

Our type inference algorithm performs multiple passes over a program to first infer simple types, which are then used to infer constraints on the variance of input/output types, sizes of naturals, bounds on channel synchronization and complexity bounds. We can reduce such constraints to a set of constraint judgements with existentially quantified variables representing coefficients that when solved provide a bound on the span. These constraints are significantly more difficult to solve than those that emerge for type checking, and so we over-approximate them using naive quantifier elimination. We provide a Haskell implementation based on the Z3 SMT solver.\\

%We have introduced constraint based type inference to the type system by Baillot and Ghyselen, which is inspired by the work of Kobayashi et al. \cite{KobayashiEtAl2000}. To infer types, we do multiple passes over the program to infer different kinds of constraints, whose solution corresponds to a valid typing of the program. To solve the constraints, we reduce them into simple constraints on coefficients that are solved by an off-the-shelf SMT solver. However, we found that by naively generating constraints, we get constraints consisting of nested existential and universal quantifiers that are very difficult to solve even for state-of-the-art SMT solvers. So, to solve these constraints we make a number of over-approximations during reduction of constraints. We implement type inference in Haskell using the Z3 SMT solver to solve constraints.\\

Overall, we find that we can type check many constant and linear time processes and some polynomial time processes. Similarly, we can infer precise bounds on many constant time and some linear time processes in reasonable time. However, both type checking and type inference are limited not only by the expressiveness of complexity bounds, but also by size bounds on naturals, as some encoded algorithms may return naturals with size bounds that exceed their complexity bounds. As our type inference algorithm infers polynomial constraints, we are also limited by the ability to find solutions, which we find to be difficult for many simple processes.

\subsection{Discussion}

During type checking, we limit ourselves to linear indices such that we can reduce constraint judgements to a linear program that can be solved by an algorithm such as the simplex algorithm. We have primarily made this choice for the sake of simplicity, however, tools such as \textit{Gröbner bases} exist that can be used to help solving systems of polynomial equations. Even more generally, existing provers such as the Z3 prover, which utilizes, among other things, Gröbner bases, could be used to solve both linear and some super-linear systems.\\

As for type inference, we are able to infer precise bounds on some constant and linear time servers. We notice that inferred constraint satisfaction problems for servers quickly grow in difficulty based on the number of index variables. Even simple processes with no ticks take significantly longer time to solve, when an extra index variable is introduced. We ascribe this to the polynomial constraints introduced upon instantiating servers, and so it may make sense to consider further over-approximations of substitutions.\\

An interesting observation is that our restriction to linear indices makes us unable to infer bounds on some linear time processes. One such example is that of the Fibonacci number encoding, where we are unable to express size bounds on the \textit{returned} Fibonacci number, which grows according to the Fibonacci sequence, yet we can express the complexity of the server. Although the complexity is not directly affected by the size of this natural, we are thus unable to infer a sized type for the server. As the complexity of a process does not necessarily depend on all size bounds, it may be possible to let some of them be unknown, and thereby relax the restriction.\\

We also notice the importance of antecedents in constraint judgements. When all antecedents are discarded, we are unable to infer bounds on any linear time server. Moreover, we observe that coefficient variables do not take negative valuations unless we simulate antecedents. We believe this is due to our over-approximations of index inequality constraints. That is, we implicitly infer the constraint $\varphi;\Phi\vDash 0 \leq I$ for all indices $I$, which we over-approximate using coefficient-wise inequality constraints, which means all coefficients must be non-negative. This seems to translate to all coefficient variables being non-negative as well. However, when antecedents are simulated, we may substitute a positive term into an index, providing more flexibility to coefficient variable valuations. We believe this is why our simulation of inequality antecedents greatly increases expressiveness of our type inference algorithm.

% type check
%   better ways to reduce (more) polynomial constraints to linear: Hoffmann and Hofmann
%   Gröbner basis for solving polynomial constraint judgements?
% type inference
%   what can we infer bounds on: Constant time processes, some linear time processes; Processes with different span and work!
%   what can we not infer bounds on: fib; not only a question of complexity but also size of outputs (non-linear); What can we do about it? - better use of index variables ..


%   The importance of antecedents
%   Consequences of our over-approximations:  0 <= I means all coefficients in I are non-negative!

\subsection{Future work}

We have introduced algorithmic type rules for the type system by Baillot and Ghyselen \cite{BaillotGhyselen2021}, enabling us to implement a type checker that given some type environment and process, can verify if process is well-typed under the environment, and thereby provide us a bound on the parallel complexity. As such, a natural next step is to implement the type checker in for instance Haskell.\\

We have primarily limited ourselves to linear indices, and thereby linear complexity bounds, as the corresponding constraint judgements are then simpler to verify or satisfy. It may be worthwhile to explore how this restriction may be relaxed. Hofmann and Hoffmann \cite{HofmannAndHoffmann2010} and Hoffmann et al. \cite{HoffmannEtAl2012}, show how linear constraints can be derived from polynomial bounds, by representing polynomials as sums of binomial coefficients. However, this is in the setting of first-order functional programs, and so it may be interesting to see if this method can be applied to message-passing processes.\\

Our type inference algorithm for parallel complexity of $\pi$-calculus processes provides correct bounds on the parallel complexity of the processes we have tried (when a bound can be inferred in reasonable time). However, we have yet to formally prove its correctness. In particular, we are interested in proving that our algorithm always infers principal typings. We expect this property to be quite straightforward to prove, as our inference rules are based on the type rules. One exception to this is our treatment of time invariance.\\

% Better support for lower-bounds: antecedents; I = 0, can be solved in the same way as I >= 1 with proper index variable use !!

In this thesis, we have limited ourselves to the type system for parallel complexity by Baillot and Ghyselen \cite{BaillotGhyselen2021}. However, the usage-based generalization of the type system provided by Baillot et al. \cite{BaillotEtAl2021} increases the expressiveness and precision. Usages allow for more precise description of the behavior of channels, for instance making processes with some forms of indeterministic communication typable. Usages are well-suited for inference, and as this type system shares many similarities with the type system by Baillot and Ghyselen, it may be interesting to see if our type inference algorithm can be extended to usages. Constraint-based inference of usage types have been studied previously, and judging from this work, we expect usage reliability to be the main challenge due to universal quantification over index variables \cite{KobayashiEtAl2000, Kobayashi2005}.

% implementation of type check
% extension of type inference to polynomial bounds alá Hoffmann and Hofmann and Hoffmann et al. (their representation of polynomials)
% prove soundness of type inference
% Extending type inference to usages
% Better support for lower-bounds: antecedents; I = 0, can be solved in the same way as I >= 1 with proper index variable use !!
%%% Local Variables:
%%% mode: latex
%%% TeX-master: "../esop2023"
%%% End:


%
%
\bibliographystyle{compj}
\bibliography{mybib}

\appendix
\chapter{Soundness of sized type implementation}\label{app:sizedtypesoundness}
\setcounter{theorem}{10}


\begin{lemma}[Additive advancement of time]
Let $\Phi$ be a set of constraints with unknowns in $\varphi$ and let $T$ be a type then $\susume{\susume{T}{\varphi}{\Phi}{J}}{\varphi}{\Phi}{I} =\; \susume{T}{\varphi}{\Phi}{I+J}$.
\begin{proof} On the structure of $T$.
    \begin{description}
    \item[$(\susume{\susume{\texttt{Nat}[K,L]}{\varphi}{\Phi}{J}}{\varphi}{\Phi}{I})$] obtained directly from  $\susume{\texttt{Nat}[K,L]}{\varphi}{\Phi}{J} = \texttt{Nat}[K,L]$ and $\susume{\texttt{Nat}[K,L]}{\varphi}{\Phi}{I} = \texttt{Nat}[K,L]$.
    %
    \item[$(\susume{\susume{\texttt{ch}^\sigma_L(\widetilde{T})}{\varphi}{\Phi}{J}}{\varphi}{\Phi}{I})$] We either have that
    \begin{enumerate}
        \item $\varphi;\Phi\vDash J \leq L$ and so we have that $\susume{\texttt{ch}^\sigma_L(\widetilde{T})}{\varphi}{\Phi}{J}=\texttt{ch}^\sigma_{L-J}(\widetilde{T})$. Then if $\varphi;\Phi\vDash I \leq L-J$ we also have $\varphi;\Phi\vDash I+J \leq L$ as $\varphi;\Phi\vDash J \leq L$, and so we obtain $\susume{\susume{\texttt{ch}^\sigma_L(\widetilde{T})}{\varphi}{\Phi}{J}}{\varphi}{\Phi}{I}=\susume{\texttt{ch}^\sigma_L(\widetilde{T})}{\varphi}{\Phi}{I+J}=\texttt{ch}^\sigma_{L-(I+J)}(\widetilde{T})$. Otherwise, we have that $\varphi;\Phi\nvDash I \leq L-J$, implying that $\varphi;\Phi\nvDash I+J \leq L$ as $\varphi;\Phi\vDash J \leq L$, and so we obtain $\susume{\susume{\texttt{ch}^\sigma_L(\widetilde{T})}{\varphi}{\Phi}{J}}{\varphi}{\Phi}{I}=\susume{\texttt{ch}^\sigma_L(\widetilde{T})}{\varphi}{\Phi}{I+J}=\texttt{ch}^\emptyset_{L-(I+J)}(\widetilde{T})$.
        %
        \item $\varphi;\Phi\nvDash J \leq L$ and so we have that $\susume{\texttt{ch}^\sigma_L(\widetilde{T})}{\varphi}{\Phi}{J}=\texttt{ch}^\emptyset_{L-J}(\widetilde{T})$ and $\susume{\texttt{ch}^\emptyset_{L-J}(\widetilde{T})}{\varphi}{\Phi}{I}=\texttt{ch}^\emptyset_{(L-J)-I}(\widetilde{T})$. It follows from the fact that $I$ is non-negative that also $\varphi;\Phi\nvDash I+J \leq L$ and so we obtain $\susume{\susume{\texttt{ch}^\sigma_L(\widetilde{T})}{\varphi}{\Phi}{J}}{\varphi}{\Phi}{I}=\texttt{ch}^\emptyset_{L-(J+I)}(\widetilde{T})=\texttt{ch}^\emptyset_{(L-J)-I}(\widetilde{T})$.
    \end{enumerate}
    %
    \item[$(\susume{\susume{\forall_L\widetilde{i}.\texttt{serv}^\sigma_K(\widetilde{T})}{\varphi}{\Phi}{J}}{\varphi}{\Phi}{I})$] We either have that
    \begin{enumerate}
        \item $\varphi;\Phi\vDash J \leq L$ and so we have that $\susume{\forall_L\widetilde{i}.\texttt{serv}^\sigma_K(\widetilde{T})}{\varphi}{\Phi}{J}=\forall_{L-J}\widetilde{i}.\texttt{serv}^\sigma_K(\widetilde{T})$. Then if $\varphi;\Phi\vDash I \leq L-J$ we also have $\varphi;\Phi\vDash I+J \leq L$ as $\varphi;\Phi\vDash J \leq L$, and so we obtain $\susume{\susume{\forall_L\widetilde{i}.\texttt{serv}^\sigma_K(\widetilde{T})}{\varphi}{\Phi}{J}}{\varphi}{\Phi}{I}=\susume{\forall_L\widetilde{i}.\texttt{serv}^\sigma_K(\widetilde{T})}{\varphi}{\Phi}{I+J}=\forall_{L-(I+J)}\widetilde{i}.\texttt{serv}^\sigma_K(\widetilde{T})$. Otherwise, we have that $\varphi;\Phi\nvDash I \leq L-J$, implying that $\varphi;\Phi\nvDash I+J \leq L$ as $\varphi;\Phi\vDash J \leq L$, and so we obtain $\susume{\susume{\forall_L\widetilde{i}.\texttt{serv}^\sigma_K(\widetilde{T})}{\varphi}{\Phi}{J}}{\varphi}{\Phi}{I}=\susume{\forall_L\widetilde{i}.\texttt{serv}^\sigma_K(\widetilde{T})}{\varphi}{\Phi}{I+J}=\forall_{L-(I+J)}\widetilde{i}.\texttt{serv}^{\sigma\cap\{\texttt{out}\}}_K(\widetilde{T})$.
        %
        \item $\varphi;\Phi\nvDash J \leq L$ and so we have that $\susume{\forall_L\widetilde{i}.\texttt{serv}^\sigma_K(\widetilde{T})}{\varphi}{\Phi}{J}=\forall_{L-J}\widetilde{i}.\texttt{serv}^{\sigma\cap\{\texttt{out}\}}_K(\widetilde{T})$ and $\susume{\forall_{L-J}\widetilde{i}.\texttt{serv}^{\sigma\cap\{\texttt{out}\}}_K(\widetilde{T})}{\varphi}{\Phi}{I}=\forall_{(L-J)-I}\widetilde{i}.\texttt{serv}^{\sigma\cap\{\texttt{out}\}}_K(\widetilde{T})$. It follows from the fact that $I$ is non-negative that also $\varphi;\Phi\nvDash I+J \leq L$ and so we obtain $\susume{\susume{\forall_L\widetilde{i}.\texttt{serv}^\sigma_K(\widetilde{T})}{\varphi}{\Phi}{J}}{\varphi}{\Phi}{I}=\forall_{L-(I+J)}\widetilde{i}.\texttt{serv}^{\sigma\cap\{\texttt{out}\}}_K(\widetilde{T})=\forall_{(L-J)-I}\widetilde{i}.\texttt{serv}^{\sigma\cap\{\texttt{out}\}}_K(\widetilde{T})$.
    \end{enumerate}
    \end{description}
\end{proof}
\end{lemma}
%

% \begin{lemma}
% If $\varphi;\Phi;\Gamma;\Delta\vdash e : T$ then $\varphi;\Phi;\Gamma;\cdot\vdash \circledcirc e : S$ with $\varphi;\Phi\vdash S \sqsubseteq T$.
% \begin{proof} By induction on $e$. We only show the interesting cases
%     \begin{description}
%     %
%     \item[$(e_\theta)$] We have two cases
%     \begin{enumerate}
%         \item $\theta = v$ By $\runa{S-avar}$ we have that $\varphi;\Phi;\Gamma;\Delta,v:T\vdash e_v : T$ and $\varphi;\Phi;\Gamma;\Delta,v:T\vdash e : S$ such that $\varphi;\Phi\vdash S \sqsubseteq T$. As $\circledcirc e_v = \circledcirc e$, we obtain by induction that $\varphi;\Phi;\Gamma;\cdot\vdash \circledcirc e : S'$ with $\varphi;\Phi\vdash S' \sqsubseteq S$ and by transitivity we have that $\varphi;\Phi\vdash S'\sqsubseteq T$.
%         %
%         \item $\theta = e'$ By $\runa{S-strength}$ we have that $\varphi;\Phi;\Gamma;\Delta\vdash e_{e'} : \texttt{Nat}[I-1,J-1]$,
%         $\varphi;\Phi;\Gamma;\Delta\vdash e : \texttt{Nat}[I',J']$ and $\varphi;\Phi;\Gamma;\Delta\vdash e' : \texttt{Nat}[I,J]$ such that $\varphi;\Phi\vdash \texttt{Nat}[I',J']\sqsubseteq\texttt{Nat}[I-1,J-1]$. By induction we obtain $\varphi;\Phi;\Gamma;\Delta\vdash \circledcirc e : \texttt{Nat}[I'',J'']$ with $\varphi;\Phi\vdash \texttt{Nat}[I'',J''] \sqsubseteq \texttt{Nat}[I',J']$. By the transitive property of $\leq$ it follows that $\varphi;\Phi\vdash \texttt{Nat}[I'',J''] \sqsubseteq \texttt{Nat}[I-1,J-1]$. 
%     \end{enumerate}
%     %
%     %\item[$(e :: e')$] By $\runa{S-cons}$ we have that $\varphi;\Phi;\Gamma\vdash_\Delta e :: e' : \texttt{List}[I+1,J+1](\mathcal{B}_1 \uplus_{\varphi;\Phi} \mathcal{B}_2)$, $\varphi;\Phi;\Gamma\vdash_\Delta e : \mathcal{B}_1$ and $\varphi;\Phi;\Gamma\vdash_\Delta e' : \texttt{List}[I,J](\mathcal{B}_2)$. As $\circledcirc (e :: e') = (\circledcirc e) :: (\circledcirc e')$, we have by induction and by $\runa{SS-lweak}$ that $\varphi;\Phi;\Gamma\vdash_\Delta \circledcirc e : \mathcal{B}_1'$ and $\varphi;\Phi;\Gamma\vdash_\emptyset \circledcirc e' : \texttt{List}[I',J'](\mathcal{B}_2')$ such that $\varphi;\Phi\vDash I \leq I'$, $\varphi;\Phi\vDash J' \leq J$, $\varphi;\Phi\vdash \mathcal{B}_1' \sqsubseteq \mathcal{B}_1$ and $\varphi;\Phi\vdash \mathcal{B}_2' \sqsubseteq \mathcal{B}_2$. By application of $\runa{S-cons}$ we obtain $\varphi;\Phi;\Gamma\vdash_\emptyset (\circledcirc e) :: (\circledcirc e') : \texttt{List}[I' + 1, J' + 1](\mathcal{B}_1' \uplus_{\varphi;\Phi} \mathcal{B}_2')$. It follows from $\varphi;\Phi\vDash I \leq I'$ and $\varphi;\Phi\vDash J' \leq J$ that also $\varphi;\Phi\vDash I+1 \leq I'+1$ and $\varphi;\Phi\vDash J'+1 \leq J+1$
%     %%and so by $\runa{SS-nweak}$ $\varphi;\Phi\vdash\texttt{Nat}[I'+1,J'+1] \sqsubseteq \texttt{Nat}[I+1,J+1]$.
%     %
%     \item[$(s(e))$] By $\runa{S-succ}$ we have that $\varphi;\Phi;\Gamma;\Delta\vdash s(e) : \texttt{Nat}[I+1,J+1]$ and $\varphi;\Phi;\Gamma;\Delta\vdash e : \texttt{Nat}[I,J]$. As $\circledcirc s(e) = s(\circledcirc e)$, we have by induction and by $\runa{SS-nweak}$ that $\varphi;\Phi;\Gamma\vdash_\emptyset \circledcirc e : \texttt{Nat}[I',J']$ such that $\varphi;\Phi\vDash I \leq I'$ and $\varphi;\Phi\vDash J' \leq J$. By application of $\runa{S-succ}$ we obtain $\varphi;\Phi;\Gamma;\cdot\vdash s(\circledcirc e) : \texttt{Nat}[I' + 1, J' + 1]$. It follows from $\varphi;\Phi\vDash I \leq I'$ and $\varphi;\Phi\vDash J' \leq J$ that also $\varphi;\Phi\vDash I+1 \leq I'+1$ and $\varphi;\Phi\vDash J'+1 \leq J+1$ and so by $\runa{SS-nweak}$ $\varphi;\Phi\vdash\texttt{Nat}[I'+1,J'+1] \sqsubseteq \texttt{Nat}[I+1,J+1]$.
%     %
%     \end{description}
% \end{proof}
% \end{lemma}
\setcounter{theorem}{16}
\begin{lemma}[Substitution]\text{ }
\begin{enumerate}
    \item If $\varphi;\Phi;\Gamma,v:T\vdash e' : S$ and $\varphi;\Phi;\Gamma\vdash e : T$ then $\varphi;\Phi;\Gamma\vdash e'[v\mapsto e] : S$.
    \item If $\varphi;\Phi;\Gamma,v:T\vdash P \triangleleft \kappa$ and $\varphi;\Phi;\Gamma\vdash e : T$ then $\varphi;\Phi;\Gamma\vdash P[v\mapsto e] \triangleleft \kappa$.
\end{enumerate}
\begin{proof} The first point is proved by induction on the type rules of expressions, and the second by induction on the type rules for processes. We consider them separately
\begin{enumerate}
    \item 
\begin{description}
%
\item[$\runa{S-zero}$] We have that $\varphi;\Phi;\Gamma,v:T\vdash 0 : \texttt{Nat}[0,0]$. We obtain $\varphi;\Phi;\Gamma\vdash 0[v\mapsto e] : \texttt{Nat}[0,0]$ directly from $0[v\mapsto e] = 0$ and $\varphi;\Phi;\Gamma\vdash 0 : \texttt{Nat}[0,0]$.
%
\item[$\runa{S-succ}$] We have that $\varphi;\Phi;\Gamma,v:T\vdash e' : \texttt{Nat}[I,J]$, $\varphi;\Phi;\Gamma,v:T\vdash s(e') : \texttt{Nat}[I+1,J+1]$ and $\varphi;\Phi;\Gamma\vdash e : T$. By induction we obtain $\varphi;\Phi;\Gamma\vdash e'[v\mapsto e] : \texttt{Nat}[I,J]$, and so by application of $\runa{S-succ}$ we derive $\varphi;\Phi;\Gamma\vdash s(e'[v\mapsto e]) : \texttt{Nat}[I+1,J+1]$.
%
\item[$\runa{S-var}$] We have two cases. Either we have that $\varphi;\Phi;\Gamma,v:T\vdash v : T$ and we substitute $e$ for $v$, or we have that $\varphi;\Phi;\Gamma,v:T,w:S\vdash v : T$. The first case is obtained directly from the assumption that $\varphi;\Phi;\Gamma\vdash e : T$. The second case is obtained directly from $v[w\mapsto e] = v$ when $v\neq w$ and $\varphi;\Phi;\Gamma,v:T\vdash v : T$ by $\runa{S-var}$.
%
\item[$\runa{S-subsumption}$] We have that $\varphi;\Phi;\Gamma,v:T\vdash e' : S'$ and $\varphi;\Phi\vdash S' \sqsubseteq S$ such that $\varphi;\Phi;\Gamma,v:T\vdash e' : S$. By the assumption we have that $\varphi;\Phi;\Gamma\vdash e : T$, and so by induction we obtain $\varphi;\Phi;\Gamma\vdash e'[v\mapsto e] : S'$, and so by application of $\runa{S-subsumption}$, we derive $\varphi;\Phi;\Gamma\vdash e'[v\mapsto e] : S$.
%
% \item[$\runa{S-avar}$] We have that $\varphi;\Phi;\Gamma,v:T;\Delta,w:S'\vdash e' : S$, $\varphi;\Phi\vdash S \sqsubseteq S'$, $\varphi;\Phi;\Gamma,v:T;\Delta\vdash {e'}_w : S'$ and $\varphi;\Phi;\Gamma;\Delta,w:S'\vdash e : T$. By induction we obtain $\varphi;\Phi;\Gamma;\Delta,w:S'\vdash e'[v \mapsto T] : S$, and by application of $\runa{S-avar}$ we derive $\varphi;\Phi;\Gamma,v:T;\Delta\vdash {e'}_w[v\mapsto T] : S'$. 
% %
% \item[$\runa{S-strength}$] We have that $\varphi;\Phi;\Gamma,v:T;\Delta\vdash e' : \texttt{Nat}[I',J']$, $\varphi;\Phi;\Gamma,v:T;\Delta\vdash e'' : \texttt{Nat}[I,J]$, $\varphi;\Phi\vdash \texttt{Nat}[I',J'] \sqsubseteq \texttt{Nat}[I-1,J-1]$, $\varphi;\Phi;\Gamma,v:T;\Delta\vdash {e'}_{e_''} : \texttt{Nat}[I-1,J-1]$ and $\varphi;\Phi;\Gamma;\Delta\vdash e : T$. By induction we obtain $\varphi;\Phi;\Gamma,v:T;\Delta\vdash e'[v\mapsto e] : \texttt{Nat}[I',J']$ and $\varphi;\Phi;\Gamma,v:T;\Delta\vdash e''[v\mapsto e] : \texttt{Nat}[I,J]$. By application of $\runa{S-strength}$ we then obtain $\varphi;\Phi;\Gamma;\Delta\vdash {e'}_{e_''}[v\mapsto e] : \texttt{Nat}[I-1,J-1]$.
%
\end{description}
    %
    \item 
\begin{description}
%
\item[$\runa{S-nil}$] We have that $\varphi;\Phi;\Gamma,v:T\vdash \nil \triangleleft \{0\}$. We obtain $\varphi;\Phi;\Gamma\vdash \nil[v\mapsto e] \triangleleft \{0\}$ directly from $\nil[v\mapsto e] = \nil$ and $\varphi;\Phi;\Gamma\vdash \nil \triangleleft \{0\}$.
%
\item[$\runa{S-tick}$] We have that $\varphi;\Phi;\downarrow_1\!\!(\Gamma,v:T)\vdash P \triangleleft \kappa$ and $\varphi;\Phi;\Gamma,v:T\vdash \tick{P} \triangleleft \kappa + 1$. By Lemma \ref{lemma:susumedefer}, we have that $\varphi;\Phi;\downarrow_1\!\!\Gamma\vdash e :\; \susume{T}{\varphi}{\Phi}{1}$, and so by induction we obtain $\varphi;\Phi;\downarrow_1\!\!\Gamma\vdash P[v\mapsto e] \triangleleft \kappa$. By application of $\runa{S-tick}$ we then derive $\varphi;\Phi;\Gamma\vdash \tick{P[v\mapsto e]} \triangleleft \kappa + 1$.
%
\item[$\runa{S-match}$] We have that $\varphi;\Phi;\Gamma,v:T\vdash e' : \texttt{Nat}[I,J]$, $\varphi;(\Phi,I\leq 0);\Gamma,v:T\vdash P \triangleleft \kappa$, $\varphi;(\Phi,J\geq 1);\Gamma,v:T,x:\texttt{Nat}[I-1,J-1]\vdash Q \triangleleft \kappa'$, $\varphi;\Phi;\Gamma,v:T\vdash \match{e}{P}{x}{Q} \triangleleft \text{basis}(\varphi,\Phi,\kappa\cup\kappa')$ and $\varphi;\Phi;\Gamma\vdash e : T$. From point 1 we obtain $\varphi;\Phi;\Gamma\vdash e'[v\mapsto e] : \texttt{Nat}[I,J]$ and by weakening (Lemma \ref{lemma:weakening}) and induction we derive $\varphi;(\Phi,I\leq 0);\Gamma\vdash P[v\mapsto e] \triangleleft \kappa$ and $\varphi;(\Phi,J\geq 1);\Gamma,x:\texttt{Nat}[I-1,J-1]\vdash Q[v\mapsto e] \triangleleft \kappa'$. Thus, by application of $\runa{S-match}$, we obtain $\varphi;\Phi;\Gamma\vdash \match{e}{P}{x}{Q}[v\mapsto e] \triangleleft \text{basis}(\varphi,\Phi,\kappa\cup\kappa')$. 
%
\item[$\runa{S-par}$] We have that $\varphi;\Phi;\Gamma,v:T\vdash P \triangleleft \kappa$, $\varphi;\Phi;\Gamma,v:T\vdash Q \triangleleft \kappa'$, $\varphi;\Phi;\Gamma,v:T\vdash P \mid Q \triangleleft \text{basis}(\varphi,\Phi,\kappa\cup\kappa')$ and $\varphi;\Phi;\Gamma\vdash e : T$. By induction we obtain $\varphi;\Phi;\Gamma\vdash P[v\mapsto e] \triangleleft \kappa$ and $\varphi;\Phi;\Gamma\vdash Q[v\mapsto e] \triangleleft \kappa'$. Thus, by application of $\runa{S-par}$, we derive $\varphi;\Phi;\Gamma\vdash (P \mid Q)[v\mapsto e] \triangleleft \text{basis}(\varphi,\Phi,\kappa\cup\kappa')$.
%
\item[$\runa{S-nu}$] We have that $\varphi;\Phi;\Gamma,v:T,a:S;\Delta\vdash P \triangleleft \kappa$, $\varphi;\Phi;\Gamma,v:T\vdash \newvar{a}{P} \triangleleft \kappa$ and $\varphi;\Phi;\Gamma\vdash e : T$. By weakening (Lemma \ref{lemma:weakening}) we obtain $\varphi;\Phi;\Gamma,a:S\vdash e : T$, and so by induction we have that $\varphi;\Phi;\Gamma,a:S\vdash P[v\mapsto e] \triangleleft \kappa$. Thus, by application of $\runa{S-nu}$ we derive $\varphi;\Phi;\Gamma\vdash (\newvar{a}{P})[v\mapsto e] \triangleleft \kappa$.
%
\item[$\runa{S-iserv}$] We have that $\varphi;\Phi;\Gamma,w:S\vdash a : \forall_0\widetilde{i}.\texttt{serv}^\sigma_K(\widetilde{T})$, $(\varphi,\widetilde{i});\Phi;\text{ready}(\varphi,\Phi,\downarrow_I\!\!(\Gamma,w:S)),\widetilde{v}:\widetilde{T}\vdash P \triangleleft \kappa$, $\varphi;\Phi;\Gamma,w:S\vdash\; !\inputch{a}{\widetilde{v}}{}{P} \triangleleft \{I\}$ and $\varphi;\Phi;\Gamma\vdash e : S$. By Lemma \ref{lemma:susumedefer} this implies $\varphi;\Phi;\text{ready}(\varphi,\Phi,\downarrow_I\!\!\Gamma)\vdash e : \text{ready}(\varphi,\Phi,\downarrow_I\!\!S)$, and from point $1$ we obtain $\varphi;\Phi;\Gamma\vdash a[w\mapsto e] : \forall_0\widetilde{i}.\texttt{serv}^\sigma_K(\widetilde{T})$. By weakening (Lemma \ref{lemma:weakening}) we then derive $\varphi;\Phi;\text{ready}(\varphi,\Phi,\downarrow_I\!\!\Gamma),\widetilde{v}:\widetilde{T}\vdash e : \text{ready}(\varphi,\Phi,\downarrow_I\!\!S)$, and so by induction we obtain $(\varphi,\widetilde{i});\Phi;\text{ready}(\varphi,\Phi,\downarrow_I\!\!\Gamma),\widetilde{v}:\widetilde{T}\vdash P[w\mapsto e] \triangleleft \kappa$. Finally, by application of $\runa{S-iserv}$, we derive $\varphi;\Phi;\Gamma\vdash\; !\inputch{a}{\widetilde{v}}{}{P}[w\mapsto e] \triangleleft \{I\}$.
%
\item[$\runa{S-ich}$] We have that $\varphi;\Phi;\Gamma,v:T\vdash a : \texttt{ch}^\sigma_I(\widetilde{S})$, $\varphi;\Phi;\downarrow_I\;\;(\Gamma,v:T),\widetilde{w}:\widetilde{S}\vdash P \triangleleft \kappa$, $\varphi;\Phi;\Gamma,v:T\vdash \inputch{a}{\widetilde{w}}{}{P} \triangleleft \kappa + I$ and $\varphi;\Phi;\Gamma\vdash e : T$. From point $1$ we obtain $\varphi;\Phi;\Gamma\vdash a[v\mapsto e] : \texttt{ch}^\sigma_I(\widetilde{S})$ (Note that it may be that $v=a$). By Lemma \ref{lemma:susumedefer}, we have that $\varphi;\Phi;\downarrow_I\!\!\Gamma\vdash e :\; \susume{T}{\varphi}{\Phi}{I}$, and so by weakening (Lemma \ref{lemma:weakening}) and induction we derive $\varphi;\Phi;\downarrow_I\;\;\Gamma,\widetilde{w}:\widetilde{S}\vdash P[v\mapsto e] \triangleleft \kappa$. Thus, by application of $\runa{S-ich}$ we obtain $\varphi;\Phi;\Gamma\vdash (\inputch{a}{\widetilde{w}}{}{P})[v\mapsto e] \triangleleft \kappa + I$. 
%
\item[$\runa{S-och}$] We have that $\varphi;\Phi;\Gamma,v:T\vdash a : \texttt{ch}^\sigma_I(\widetilde{S})$, $\varphi;\Phi;\downarrow_I\!\!(\Gamma,v:T)\vdash \widetilde{e}' : \widetilde{S}'$, $\varphi;\Phi;\Gamma,v:T\vdash \asyncoutputch{a}{\widetilde{e}'}{} \triangleleft \{I\}$ and $\varphi;\Phi;\Gamma\vdash e : T$. By Lemma \ref{lemma:susumedefer}, we have that $\varphi;\Phi;\downarrow_I\!\!\Gamma\vdash e :\; \susume{T}{\varphi}{\Phi}{I}$, and so from point $1$ we obtain $\varphi;\Phi;\downarrow_I\!\!\Gamma\vdash \widetilde{e}'[v\mapsto e] : \widetilde{S}'$ and $\varphi;\Phi;\Gamma\vdash a[v\mapsto e] : \texttt{ch}^\sigma_I(\widetilde{S})$. By application of $\runa{S-och}$ we thus obtain $\varphi;\Phi;\Gamma\vdash \asyncoutputch{a}{\widetilde{e}'}{}[v\mapsto e] \triangleleft \{I\}$.
%
\item[$\runa{S-annot}$] We have that $\varphi;\Phi;\downarrow_n\!\!(\Gamma,v:T)\vdash P \triangleleft \kappa$ and $\varphi;\Phi;\Gamma,v:T\vdash n : P \triangleleft \kappa + n$. By Lemma \ref{lemma:susumedefer}, we have that $\varphi;\Phi;\downarrow_n\!\!\Gamma\vdash e :\; \susume{T}{\varphi}{\Phi}{n}$, and so by induction we obtain $\varphi;\Phi;\downarrow_n\!\!\Gamma\vdash P[v\mapsto e] \triangleleft \kappa$. By application of $\runa{S-annot}$ we then derive $\varphi;\Phi;\Gamma\vdash n : P[v\mapsto e] \triangleleft \kappa + n$.
%
\item[$\runa{S-oserv}$] We have that $\varphi;\Phi;\Gamma,v:T\vdash a : \forall_0\widetilde{i}.\texttt{serv}^\sigma_K(\widetilde{S})$, $\varphi;\Phi;\downarrow_I\!\!(\Gamma,v:T)\vdash \widetilde{e}' : \widetilde{S}'$, $\varphi;\Phi;\Gamma,v:T\vdash \asyncoutputch{a}{\widetilde{e}}{} \triangleleft \{K\{\widetilde{J}/\widetilde{i}\}+I\}$ and $\varphi;\Phi;\Gamma\vdash e : T$, where $\text{instantiate}(\widetilde{i},\widetilde{S}')=\{\widetilde{J}/\widetilde{i}\}$. From point $1$ we obtain $\varphi;\Phi;\Gamma\vdash a[v\mapsto e] : \forall_0\widetilde{i}.\texttt{serv}^\sigma_K(\widetilde{S})$, and by Lemma \ref{lemma:basisdefer} we derive $\varphi;\Phi;\downarrow_I\!\!\Gamma\vdash e :\; \susume{T}{\varphi}{\Phi}{I}$. Thus, by induction we obtain $\varphi;\Phi;\downarrow_I\!\!\Gamma\vdash \widetilde{e}'[v\mapsto e] : \widetilde{S}'$. Finally, by application of $\runa{S-oserv}$, we obtain $\varphi;\Phi;\Gamma\vdash \asyncoutputch{a}{\widetilde{e}[v\mapsto e]}{} \triangleleft \{K\{\widetilde{J}/\widetilde{i}\}+I\}$.
%
%
\end{description}
\end{enumerate}
\end{proof}
\end{lemma}

\begin{lemma}[Subject congruence]
Let $P$ and $Q$ be processes such that $P\equiv Q$ then $\varphi;\Phi;\Gamma\vdash P \triangleleft \kappa$ if and only if $\varphi;\Phi;\Gamma\vdash Q \triangleleft \kappa'$ with $\varphi;\Phi\vDash \kappa = \kappa'$.
\begin{proof} By induction on the rules defining $\equiv$.
\begin{description}
\item[$\runa{SC-nil}$] We have that $P \mid \nil \equiv P$. We either have that $\varphi;\Phi;\Gamma\vdash P \mid \nil \triangleleft \kappa'$ or $\varphi;\Phi;\Gamma\vdash P \triangleleft \kappa$. In the former case, we must use type rule $\runa{S-par}$, and so we derive $\varphi;\Phi;\Gamma\vdash P \triangleleft \kappa$. Thus, it suffices to show that $\varphi;\Phi\vDash \kappa = \kappa'$. By $\runa{S-nil}$ we have that $\varphi;\Phi;\Gamma\vdash \nil \triangleleft \{0\}$. By $\runa{S-par}$ we have that $\kappa'=\text{basis}(\varphi,\Phi,\kappa \cup \{0\}) = \text{basis}(\varphi,\Phi,\kappa)$, as $\varphi;\Phi\vDash 0 \leq \kappa$. By Lemma \ref{lemma:basisdefer} we have that $\varphi;\Phi\vDash\text{basis}(\varphi,\Phi,\kappa)=\kappa$.
%
\item[$\runa{SC-commu}$] We have that $P\mid Q \equiv Q\mid P$. In either case we must use type rule $\runa{S-par}$ and so we have that $\varphi;\Phi;\Gamma\vdash P \triangleleft \kappa$ and $\varphi;\Phi;\Gamma\vdash Q \triangleleft \kappa'$. By the commutative law of set union, $\kappa\cup\kappa'=\kappa'\cup\kappa$ and so by extension, $\text{basis}(\varphi,\Phi,\kappa\cup\kappa')=\text{basis}(\varphi,\Phi,\kappa'\cup\kappa)$. Thus, by application of $\runa{S-par}$ we obtain $\varphi;\Phi;\Gamma\vdash Q \mid P \triangleleft \text{basis}(\varphi,\Phi,\kappa\cup\kappa')$ and $\varphi;\Phi;\Gamma\vdash P \mid Q \triangleleft \text{basis}(\varphi,\Phi,\kappa\cup\kappa')$.
%
\item[$\runa{SC-assoc}$] We have that $P\mid (Q \mid R) \equiv (P\mid Q) \mid R$. In either case we must use type rule $\runa{S-par}$ twice such that $\varphi;\Phi;\Gamma\vdash P \triangleleft \kappa$, $\varphi;\Phi;\Gamma\vdash Q \triangleleft \kappa'$ and
$\varphi;\Phi;\Gamma\vdash R \triangleleft \kappa''$. From this we obtain two derivation trees of the form in both cases
    \begin{align*}
        \begin{prooftree}
        \Infer0{\pi_P}
        \Infer1{\varphi;\Phi;\Gamma\vdash P \triangleleft \kappa}
        %
        \Infer0{\pi_Q}
        \Infer1{\varphi;\Phi;\Gamma\vdash Q \triangleleft \kappa'}
        %
        \Infer0{\pi_R}
        \Infer1{\varphi;\Phi;\Gamma\vdash R \triangleleft \kappa''}
        %
        \Infer2{\varphi;\Phi;\Gamma\vdash Q \mid R \triangleleft \text{basis}(\varphi,\Phi,\kappa'\cup\kappa'')}
        %
        \Infer2{\varphi;\Phi;\Gamma\vdash P \mid (Q \mid R) \triangleleft \text{basis}(\varphi,\Phi,\kappa\cup\text{basis}(\varphi,\Phi,\kappa'\cup\kappa''))}
        \end{prooftree}\\
        %
        \\
        %
        \begin{prooftree}
        \Infer0{\pi_P}
        \Infer1{\varphi;\Phi;\Gamma\vdash P \triangleleft \kappa}
        %
        \Infer0{\pi_Q}
        \Infer1{\varphi;\Phi;\Gamma\vdash Q \triangleleft \kappa'}
        %
        \Infer2{\varphi;\Phi;\Gamma\vdash P \mid Q \triangleleft \text{basis}(\varphi,\Phi,\kappa\cup\kappa')}
        %
        \Infer0{\pi_R}
        \Infer1{\varphi;\Phi;\Gamma\vdash R \triangleleft \kappa''}
        %
        \Infer2{\varphi;\Phi;\Gamma\vdash (P \mid Q) \mid R \triangleleft \text{basis}(\varphi,\Phi,\text{basis}(\varphi,\Phi,\kappa\cup\kappa')\cup\kappa'')}
        \end{prooftree}
    \end{align*}
Thus, it suffices to show that $\text{basis}(\varphi,\Phi,\kappa\cup\text{basis}(\varphi,\Phi,\kappa'\cup\kappa''))=\text{basis}(\varphi,\Phi,\text{basis}(\varphi,\Phi,\kappa\cup\kappa')\cup\kappa'')$. We obtain this directly from Lemma \ref{lemma:basisdefer}.
%
\item[$\runa{SC-scope}$] We have that $\newvar{a}{(P \mid Q)} \equiv \newvar{a}{P\mid Q}$ and that $a$ is not free in $Q$. We consider the implications separately
\begin{enumerate}
    \item We have that $\varphi;\Phi;\Gamma\vdash \newvar{a}{(P \mid Q)} \triangleleft \kappa''$. Thus, we must use type rule $\runa{S-nu}$ and $\runa{S-par}$ such that $\varphi;\Phi;\Gamma,a:T\vdash P \mid Q \triangleleft \kappa''$, $\varphi;\Phi;\Gamma,a:T\vdash P \triangleleft \kappa$ and $\varphi;\Phi;\Gamma,a:T\vdash Q \triangleleft \kappa'$. By strengthening (Lemma \ref{lemma:strengthening}) we obtain $\varphi;\Phi;\Gamma\vdash Q \triangleleft \kappa'$, and by application of $\runa{S-nu}$ we derive $\varphi;\Phi;\Gamma\vdash \newvar{a}{P} \triangleleft \kappa$. Thus, by application of $\runa{S-par}$ we obtain $\varphi;\Phi;\Gamma\vdash \newvar{a}{P} \mid Q \triangleleft \kappa''$.
    %
    \item We have that $\varphi;\Phi;\Gamma\vdash \newvar{a}{P} \mid Q \triangleleft \kappa''$. Thus, we must use type rule $\runa{S-par}$ and $\runa{S-nu}$ such that $\varphi;\Phi;\Gamma\vdash \newvar{a}{P} \triangleleft \kappa$, $\varphi;\Phi;\Gamma,a:T\vdash P \triangleleft \kappa$ and $\varphi;\Phi;\Gamma\vdash Q \triangleleft \kappa'$. By weakening (Lemma \ref{lemma:weakening}) we obtain $\varphi;\Phi;\Gamma,a:T\vdash Q \triangleleft \kappa'$ and so by application of $\runa{S-par}$ and $\runa{S-nu}$ we derive $\varphi;\Phi;\Gamma,a:T\vdash P \mid Q \triangleleft \kappa''$ and $\varphi;\Phi;\Gamma\vdash \newvar{a}{(P \mid Q)} \triangleleft \kappa''$.
\end{enumerate}
%
\item[$\runa{SC-par}$] We have that $P\mid Q \equiv P' \mid Q$ with $P\equiv P'$. We must use type rule $\runa{S-par}$ and so we either have that $\varphi;\Phi;\Gamma\vdash P \mid Q \triangleleft \kappa''$ or $\varphi;\Phi;\Gamma\vdash P' \mid Q \triangleleft \kappa''$ with $\varphi;\Phi;\Gamma\vdash Q \triangleleft \kappa'$. When $P$ is well-typed we obtain an equivalent typing for $P'$ and vice-versa by induction. Thus, we have that $\varphi;\Phi;\Gamma\vdash P \triangleleft \kappa$ and $\varphi;\Phi;\Gamma\vdash P' \triangleleft \kappa'$ with $\varphi;\Phi\vDash \kappa = \kappa'$, and so in either case, it suffices to apply $\runa{S-par}$.
%
\item[$\runa{SC-res}$] We have that $\newvar{a}{P} \equiv \newvar{a}{Q}$ with $P \equiv Q$. We must use type rule $\runa{S-nu}$ and so we either have that $\varphi;\Phi;\Gamma\vdash \newvar{a}{P} \triangleleft \kappa$ with $\varphi;\Phi;\Gamma,a:T\vdash P \triangleleft \kappa$ or $\varphi;\Phi;\Gamma\vdash \newvar{a}{Q} \triangleleft \kappa'$ with $\varphi;\Phi;\Gamma,a:T\vdash Q \triangleleft \kappa'$. In either case we use induction to obtain an equivalent typing for $Q$ when we have the same typing for $P$ and vice-versa, i.e. $\varphi;\Phi\vDash \kappa = \kappa'$. Thus in either case, it suffices to apply $\runa{S-nu}$.
%
\item[$\runa{SC-zero}$] This result is obtained directly from $\susume{\Gamma}{\varphi}{\Phi}{0}=\Gamma$.% We have that $P \equiv 0 : P$, and so we must use type rule $\runa{S-annot}$. 
%
\item[$\runa{SC-sum}$] We have that $n : m : P \equiv n+m : P$, and so we must use type rule $\runa{S-annot}$. In the first case we have that $\varphi;\Phi;\downarrow_m\!\!(\downarrow_n\!\!\Gamma)\vdash P \triangleleft \kappa$, $\varphi;\Phi;\downarrow_n\!\!\Gamma\vdash m : P \triangleleft \kappa + m$ and $\varphi;\Phi;\Gamma\vdash n : m : P \triangleleft \kappa + m + n$. In the second case we have that $\varphi;\Phi;\downarrow_{n+m}\!\!\Gamma\vdash P \triangleleft \kappa$ and $\varphi;\Phi;\Gamma\vdash (n+m) : P \triangleleft \kappa + m + n$. Thus, it suffices to show that $\susume{\Gamma}{\varphi}{\Phi}{n+m} = \susume{\susume{\Gamma}{\varphi}{\Phi}{n}}{\varphi}{\Phi}{m}$. We obtain this directly from Lemma \ref{lemma:addsusume}.
%
\item[$\runa{SC-dis}$] We have that $n : (P \mid Q) \equiv (n : P) \mid (n : Q)$, and so we must use type rule $\runa{S-par}$ and $\runa{S-annot}$. We have the two derivation trees 
{\small
\begin{align*}
    \begin{prooftree}
    \Infer0{\pi_P}
    \Infer1{\varphi;\Phi;\downarrow_n\!\!\Gamma\vdash P \triangleleft \kappa}
    %
    \Infer0{\pi_Q}
    \Infer1{\varphi;\Phi;\downarrow_n\!\!\Gamma\vdash Q \triangleleft \kappa'}
    %
    \Infer2{\varphi;\Phi;\downarrow_n\!\!\Gamma\vdash P \mid Q \triangleleft \text{basis}(\varphi,\Phi,\kappa\cup\kappa')}
    %
    \Infer1{\varphi;\Phi;\Gamma\vdash n : (P \mid Q) \triangleleft \text{basis}(\varphi,\Phi,\kappa\cup\kappa') + n}
    \end{prooftree}\quad
    %
    \begin{prooftree}
    \Infer0{\pi_P}
    \Infer1{\varphi;\Phi;\downarrow_n\!\!\Gamma\vdash P \triangleleft \kappa}
    \Infer1{\varphi;\Phi;\Gamma\vdash n : P \triangleleft \kappa + n}
    %
    \Infer0{\pi_Q}
    \Infer1{\varphi;\Phi;\downarrow_n\!\!\Gamma\vdash Q \triangleleft \kappa'}
    \Infer1{\varphi;\Phi;\Gamma\vdash n : Q \triangleleft \kappa' + n}
    %
    \Infer2{\varphi;\Phi;\Gamma\vdash (n : P) \mid (n:Q) \triangleleft \text{basis}(\varphi,\Phi,(\kappa+n)\cup(\kappa'+n))}
    \end{prooftree}
\end{align*}}
Thus, it suffices to show that $\varphi;\Phi\vDash \text{basis}(\varphi,\Phi,\kappa\cup\kappa') + n = \text{basis}(\varphi,\Phi,(\kappa+n)\cup(\kappa'+n))$. We obtain this directly from Lemma \ref{lemma:basisdefer}.
%
\item[$\runa{SC-ares}$] We have that $n : \newvar{a}{P} \equiv \newvar{a}{n : P}$, and so we must use type rule $\runa{S-annot}$ and $\runa{S-nu}$. We have the two derivation trees
\begin{align*}
    \begin{prooftree}
    \Infer0{\pi_P}
    \Infer1{\varphi;\Phi;\downarrow_n\!\!\Gamma,a:T\vdash P \triangleleft \kappa}
    \Infer1{\varphi;\Phi;\downarrow_n\!\!\Gamma\vdash \newvar{a}{P} \triangleleft \kappa}
    \Infer1{\varphi;\Phi;\Gamma\vdash n : \newvar{a}{P} \triangleleft \kappa + n}
    \end{prooftree}\quad
    %
    \begin{prooftree}
    \Infer0{\pi_P}
    \Infer1{\varphi;\Phi;\downarrow_n\!\!(\Gamma,a:\uparrow^n\!\!T)\vdash P \triangleleft \kappa'}
    \Infer1{\varphi;\Phi;\Gamma,a:\uparrow^n\!\!T\vdash n : P \triangleleft \kappa' + n}
    \Infer1{\varphi;\Phi;\Gamma\vdash \newvar{a}{n : P} \triangleleft \kappa' + n}
    \end{prooftree}
\end{align*}
From Lemma \ref{lemma:delayingg}, we have that $\susume{\uparrow^n\!\!T}{\varphi}{\Phi}{n}=T$, and so $\susume{\Gamma,a:\uparrow^n\!\!T^}{\varphi}{\Phi}{n}=\;\susume{\Gamma}{\varphi}{\Phi}{n},a:T$. This implies that $\kappa=\kappa'$, and so from either typing we can reach the other by application of type rule $\runa{S-nu}$ and $\runa{S-annot}$.
\end{description}
\end{proof}
\end{lemma}
\chapter{Type inference examples}\label{app:runningexm}

We first show the inferred and reduced constraint satisfaction problems for the process
\begin{align*}
    &\kern0em P_{\text{npar}} \defeq\\
    &(\nu \text{npar})(\\
    &\kern2em !\text{npar}(n,r).\texttt{match}\; n\; \{\\
    &\kern3em 0 \mapsto \asyncoutputch{r}{}{}\\
    &\kern3em s(x) \mapsto (\nu r' )(\nu r'' )(\\
    &\kern4em {\color{blue} \tick{{\color{black} \asyncoutputch{r'}{}{}}} } \mid
 \asyncoutputch{\text{npar}}{x,r''}{} \mid \inputch{r'}{}{}{\inputch{r''}{}{}{\asyncoutputch{r}{}{}}}) \} \\
    &\kern2em \mid \\
    &(\nu r)( \asyncoutputch{\text{npar}}{10,r}{} \mid \inputch{r}{}{}{\nil} ))
\end{align*}
We infer the following constraints

{
\tiny

\begin{align*}
    \texttt{Nat}[0, \alpha_{52}] \sim \texttt{Nat}[0, \alpha_{53}]\\ \texttt{Nat}[0, \alpha_{62}] \sim \texttt{Nat}[0, (1\alpha_{62}+0)]\\ \texttt{Nat}[0, \alpha_{24} + \alpha_{25}i_{0}] \sim \texttt{Nat}[0, (1\alpha_{24}+0) + 1\alpha_{25}i_{0}]\\ \texttt{ch}^{\gamma_{0}}_{\alpha_{2} + \alpha_{3}i_{0}}() \sim \texttt{ch}^{\gamma_{6}}_{((\alpha_{28}+\alpha_{32})+\alpha_{36}) + ((\alpha_{29}+\alpha_{33})+\alpha_{37})i_{0}}()\\ \texttt{ch}^{\gamma_{1}}_{(\alpha_{6}+1) + \alpha_{7}i_{0}}() \sim \texttt{ch}^{\gamma_{8}}_{\alpha_{36} + \alpha_{37}i_{0}}()\\ \texttt{ch}^{\gamma_{2}}_{(\alpha_{14}+\alpha_{16}) + (\alpha_{15}+\alpha_{17})i_{0}}() \sim \texttt{ch}^{\gamma_{7}}_{(\alpha_{32}+\alpha_{36}) + (\alpha_{33}+\alpha_{37})i_{0}}()\\ \texttt{ch}^{\gamma_{5}}_{\alpha_{26} + \alpha_{27}i_{0}}() \sim \texttt{ch}^{\gamma_{4}}_{(\alpha_{21}\alpha_{24}+\alpha_{20}) + \alpha_{21}\alpha_{25}i_{0}}()\\ \texttt{ch}^{\gamma_{13}}_{(\alpha_{55}+\alpha_{56})}() \sim \texttt{ch}^{\gamma_{17}}_{\alpha_{65}}()\\ \texttt{ch}^{\gamma_{16}}_{\alpha_{63}}() \sim \texttt{ch}^{\gamma_{15}}_{(\alpha_{60}\alpha_{62}+\alpha_{59})}()\\ \forall_{\alpha_{40}}{i_{0}}.\texttt{serv}^{\gamma_{9}}_{\alpha_{41} + \alpha_{42}i_{0}}(\texttt{Nat}[0, 0 + 1i_{0}], \texttt{ch}^{\gamma_{10}}_{\alpha_{43} + \alpha_{44}i_{0}}()) \sim \forall_{\alpha_{56}}{i_{0}}.\texttt{serv}^{\gamma_{14}}_{\alpha_{57} + \alpha_{58}i_{0}}(\texttt{Nat}[0, 0 + 1i_{0}], \texttt{ch}^{\gamma_{15}}_{\alpha_{59} + \alpha_{60}i_{0}}())\\ \{\};\{\} \vDash \texttt{inv}(\forall_{\alpha_{16} + \alpha_{17}i_{0}}{i_{0}}.\texttt{serv}^{\gamma_{3}}_{\alpha_{18} + \alpha_{19}i_{0}}(\texttt{Nat}[0, 0 + 1i_{0}], \texttt{ch}^{\gamma_{4}}_{\alpha_{20} + \alpha_{21}i_{0}}()))\\ () \implies (\{\};\{\}  \vdash \texttt{Nat}[0, \alpha_{54}] \sqsubseteq \texttt{Nat}[0, \alpha_{62}])\\ () \implies (\{\};\{\}  \vdash \texttt{Nat}[0, (\alpha_{53}+10)] \sqsubseteq \texttt{Nat}[0, \alpha_{54}])\\ () \implies (\{\};\{\}  \vdash \texttt{ch}^{\gamma_{13}}_{\alpha_{55}}() \sqsubseteq \texttt{ch}^{\gamma_{16}}_{\alpha_{63}}())\\ () \implies (\{\};\{\}  \vdash \forall_{\alpha_{47}}{i_{0}}.\texttt{serv}^{\gamma_{11}}_{\alpha_{48} + \alpha_{49}i_{0}}(\texttt{Nat}[0, 0 + 1i_{0}], \texttt{ch}^{\gamma_{12}}_{\alpha_{50} + \alpha_{51}i_{0}}()) \sqsubseteq \forall_{\alpha_{46}}{i_{0}}.\texttt{serv}^{\gamma_{3}}_{\alpha_{18} + \alpha_{19}i_{0}}(\texttt{Nat}[0, 0 + 1i_{0}], \texttt{ch}^{\gamma_{4}}_{\alpha_{20} + \alpha_{21}i_{0}}()))\\ () \implies (\{i_{0}\};\{\}  \vdash \texttt{Nat}[0, \alpha_{0} + \alpha_{1}i_{0}] \sqsubseteq \texttt{Nat}[0, (\alpha_{12}+1) + \alpha_{13}i_{0}])\\ () \implies (\{i_{0}\};\{\}  \vdash \texttt{Nat}[0, 0 + 1i_{0}] \sqsubseteq \texttt{Nat}[0, \alpha_{0} + \alpha_{1}i_{0}])\\ () \implies (\{i_{0}\};\{\}  \vdash \texttt{Nat}[0, 0 + 1i_{0}] \sqsubseteq \texttt{Nat}[0, 0 + 1i_{0}])\\ () \implies (\{i_{0}\};\{\}  \vdash \texttt{ch}^{\gamma_{4}}_{\alpha_{20} + \alpha_{21}i_{0}}() \sqsubseteq \texttt{ch}^{\gamma_{10}}_{\alpha_{43} + \alpha_{44}i_{0}}())\\ () \implies (\{i_{0}\};\{\}  \vdash \texttt{ch}^{\gamma_{10}}_{\alpha_{43} + \alpha_{44}i_{0}}() \sqsubseteq \texttt{ch}^{\gamma_{0}}_{\alpha_{2} + \alpha_{3}i_{0}}())\\ () \implies (\{i_{0}\};\{1 \leq \alpha_{0} + \alpha_{1}i_{0}\}  \vdash \texttt{Nat}[0, \alpha_{12} + \alpha_{13}i_{0}] \sqsubseteq \texttt{Nat}[0, \alpha_{24} + \alpha_{25}i_{0}])\\ () \implies (\{i_{0}\};\{1 \leq \alpha_{0} + \alpha_{1}i_{0}\}  \vdash \texttt{ch}^{\gamma_{2}}_{\alpha_{14} + \alpha_{15}i_{0}}() \sqsubseteq \texttt{ch}^{\gamma_{5}}_{\alpha_{26} + \alpha_{27}i_{0}}())\\ () \implies (\{\};\{\}  \vDash \alpha_{40} \leq \alpha_{45})\\ () \implies (\{\};\{\}  \vDash \alpha_{46} \leq \alpha_{40})\\ () \implies (\{\};\{\}  \vDash 0 \leq \alpha_{40}), () \implies (\{\};\{\}  \vDash 0 \leq \alpha_{45}), () \implies (\{\};\{\}  \vDash 0 \leq \alpha_{52})\\
    () \implies (\{\};\{\}  \vDash 0 \leq \alpha_{56}), () \implies (\{\};\{\}  \vDash 0 \leq \alpha_{64}), () \implies (\{\};\{\}  \vDash 0 \leq \alpha_{65}),
    () \implies (\{\};\{\}  \vDash 0 \leq \alpha_{57} + \alpha_{58}i_{0})\\ () \implies (\{\};\{\}  \vDash (\alpha_{56}+(\alpha_{58}\alpha_{62}+\alpha_{57})) \leq \alpha_{61})\\ () \implies (\{\};\{\}  \vDash (\alpha_{64}+\alpha_{65}) \leq \alpha_{66})\\ () \implies (\{i_{0}\};\{\}  \vDash 0 \leq \alpha_{0} + \alpha_{1}i_{0})\\ () \implies (\{i_{0}\};\{\}  \vDash \alpha_{41} + \alpha_{42}i_{0} \leq \alpha_{18} + \alpha_{19}i_{0})\\ () \implies (\{i_{0}\};\{0 \leq 0\}  \vDash 0 \leq \alpha_{2} + \alpha_{3}i_{0}), () \implies (\{i_{0}\};\{0 \leq 0\}  \vDash \alpha_{2} + \alpha_{3}i_{0} \leq \alpha_{4} + \alpha_{5}i_{0})\\ () \implies (\{i_{0}\};\{1 \leq \alpha_{0} + \alpha_{1}i_{0}\}  \vDash 0 \leq \alpha_{6} + \alpha_{7}i_{0}), () \implies (\{i_{0}\};\{1 \leq \alpha_{0} + \alpha_{1}i_{0}\}  \vDash 0 \leq \alpha_{12} + \alpha_{13}i_{0})\\ () \implies (\{i_{0}\};\{1 \leq \alpha_{0} + \alpha_{1}i_{0}\}  \vDash 0 \leq \alpha_{18} + \alpha_{19}i_{0}), () \implies (\{i_{0}\};\{1 \leq \alpha_{0} + \alpha_{1}i_{0}\}  \vDash 0 \leq \alpha_{28} + \alpha_{29}i_{0})\\ () \implies (\{i_{0}\};\{1 \leq \alpha_{0} + \alpha_{1}i_{0}\}  \vDash 0 \leq \alpha_{32} + \alpha_{33}i_{0}), () \implies (\{i_{0}\};\{1 \leq \alpha_{0} + \alpha_{1}i_{0}\}  \vDash 0 \leq \alpha_{36} + \alpha_{37}i_{0})\\ () \implies (\{i_{0}\};\{1 \leq \alpha_{0} + \alpha_{1}i_{0}\}  \vDash \alpha_{6} + \alpha_{7}i_{0} \leq \alpha_{8} + \alpha_{9}i_{0})\\ () \implies (\{i_{0}\};\{1 \leq \alpha_{0} + \alpha_{1}i_{0}\}  \vDash (\alpha_{8}+1) + \alpha_{9}i_{0} \leq \alpha_{10} + \alpha_{11}i_{0})\\ () \implies (\{i_{0}\};\{1 \leq \alpha_{0} + \alpha_{1}i_{0}\}  \vDash \alpha_{28} + \alpha_{29}i_{0} \leq \alpha_{30} + \alpha_{31}i_{0})\\ () \implies (\{i_{0}\};\{1 \leq \alpha_{0} + \alpha_{1}i_{0}\}  \vDash (\alpha_{16}+(\alpha_{19}\alpha_{24}+\alpha_{18})) + (\alpha_{17}+\alpha_{19}\alpha_{25})i_{0} \leq \alpha_{22} + \alpha_{23}i_{0})\\ () \implies (\{i_{0}\};\{1 \leq \alpha_{0} + \alpha_{1}i_{0}\}  \vDash (\alpha_{30}+\alpha_{32}) + (\alpha_{31}+\alpha_{33})i_{0} \leq \alpha_{34} + \alpha_{35}i_{0})\\ () \implies (\{i_{0}\};\{1 \leq \alpha_{0} + \alpha_{1}i_{0}\}  \vDash (\alpha_{34}+\alpha_{36}) + (\alpha_{35}+\alpha_{37})i_{0} \leq \alpha_{38} + \alpha_{39}i_{0})\\ 
    () \implies \gamma_{3} \subseteq \gamma_{9},() \implies \{\texttt{in}\} \subseteq \gamma_{7}, () \implies \{\texttt{in}\} \subseteq \gamma_{8}\\ () \implies \{\texttt{in}\} \subseteq \gamma_{9}, () \implies \{\texttt{in}\} \subseteq \gamma_{17}, () \implies \{\texttt{out}\} \subseteq \gamma_{0}\\ () \implies \{\texttt{out}\} \subseteq \gamma_{1}, () \implies \{\texttt{out}\} \subseteq \gamma_{3}, () \implies \{\texttt{out}\} \subseteq \gamma_{6}, () \implies \{\texttt{out}\} \subseteq \gamma_{14}\\ \alpha_{45} \sim \alpha_{61}\\ \alpha_{61} \sim \alpha_{66}\\ 0 \sim \alpha_{16} + \alpha_{17}i_{0}\\ \alpha_{4} + \alpha_{5}i_{0} \sim \alpha_{10} + \alpha_{11}i_{0}\\ \alpha_{4} + \alpha_{5}i_{0} \sim \alpha_{41} + \alpha_{42}i_{0}\\ \alpha_{10} + \alpha_{11}i_{0} \sim \alpha_{22} + \alpha_{23}i_{0}\\ \alpha_{10} + \alpha_{11}i_{0} \sim \alpha_{38} + \alpha_{39}i_{0}
\end{align*}

}

We reduce the constraints to

{
\tiny

\begin{align*}
    \alpha_{40} \sim \alpha_{56}, \alpha_{45} \sim \alpha_{61}, \alpha_{47} \sim \alpha_{46}, \alpha_{52} \sim \alpha_{53}, \alpha_{55} \sim \alpha_{63},, \alpha_{61} \sim \alpha_{66}\\ \alpha_{62} \sim (1\alpha_{62}+0), \alpha_{63} \sim (\alpha_{60}\alpha_{62}+\alpha_{59}), 0 \sim 0, 0 \sim \alpha_{16} + \alpha_{17}i_{0},
    (\alpha_{55}+\alpha_{56}) \sim \alpha_{65}\\
    %
    \alpha_{2} + \alpha_{3}i_{0} \sim ((\alpha_{28}+\alpha_{32})+\alpha_{36}) + ((\alpha_{29}+\alpha_{33})+\alpha_{37})i_{0}\\
    \alpha_{4} + \alpha_{5}i_{0} \sim \alpha_{10} + \alpha_{11}i_{0}, \alpha_{4} + \alpha_{5}i_{0} \sim \alpha_{41} + \alpha_{42}i_{0}\\
    (\alpha_{6}+1) + \alpha_{7}i_{0} \sim \alpha_{36} + \alpha_{37}i_{0}\\ \alpha_{10} + \alpha_{11}i_{0} \sim \alpha_{22} + \alpha_{23}i_{0}\\ \alpha_{10} + \alpha_{11}i_{0} \sim \alpha_{38} + \alpha_{39}i_{0}\\ \alpha_{14} + \alpha_{15}i_{0} \sim \alpha_{26} + \alpha_{27}i_{0}\\ \alpha_{20} + \alpha_{21}i_{0} \sim \alpha_{43} + \alpha_{44}i_{0}\\ \alpha_{20} + \alpha_{21}i_{0} \sim \alpha_{50} + \alpha_{51}i_{0}\\ \alpha_{24} + \alpha_{25}i_{0} \sim (1\alpha_{24}+0) + 1\alpha_{25}i_{0}\\ \alpha_{26} + \alpha_{27}i_{0} \sim (\alpha_{21}\alpha_{24}+\alpha_{20}) + \alpha_{21}\alpha_{25}i_{0}\\ \alpha_{41} + \alpha_{42}i_{0} \sim \alpha_{57} + \alpha_{58}i_{0}\\ \alpha_{43} + \alpha_{44}i_{0} \sim \alpha_{2} + \alpha_{3}i_{0}\\ \alpha_{43} + \alpha_{44}i_{0} \sim \alpha_{59} + \alpha_{60}i_{0}\\ \alpha_{50} + \alpha_{51}i_{0} \sim \alpha_{20} + \alpha_{21}i_{0}\\ 0 + 1i_{0} \sim 0 + 1i_{0}\\ (\alpha_{14}+\alpha_{16}) + (\alpha_{15}+\alpha_{17})i_{0} \sim (\alpha_{32}+\alpha_{36}) + (\alpha_{33}+\alpha_{37})i_{0}\\ \{\};\{\}  \vDash \alpha_{40} \leq \alpha_{45}\\ \{\};\{\}  \vDash \alpha_{46} \leq \alpha_{40}\\ \{\};\{\}  \vDash \alpha_{54} \leq \alpha_{62}\\ \{\};\{\}  \vDash 0 \leq \alpha_{40}\\ \{\};\{\}  \vDash 0 \leq \alpha_{45}\\ \{\};\{\}  \vDash 0 \leq \alpha_{52}\\ \{\};\{\}  \vDash 0 \leq \alpha_{56}\\ \{\};\{\}  \vDash 0 \leq \alpha_{64}\\ \{\};\{\}  \vDash 0 \leq \alpha_{65}\\ \{\};\{\}  \vDash 0 \leq 0\\ \{\};\{\}  \vDash 0 \leq \alpha_{57} + \alpha_{58}i_{0}\\ \{\};\{\}  \vDash (\alpha_{53}+10) \leq \alpha_{54}\\ \{\};\{\}  \vDash (\alpha_{56}+(\alpha_{58}\alpha_{62}+\alpha_{57})) \leq \alpha_{61}\\ \{\};\{\}  \vDash (\alpha_{64}+\alpha_{65}) \leq \alpha_{66}\\ \{i_{0}\};\{\}  \vDash 0 \leq 0\\ \{i_{0}\};\{\}  \vDash 0 \leq \alpha_{0} + \alpha_{1}i_{0}\\ \{i_{0}\};\{\}  \vDash \alpha_{0} + \alpha_{1}i_{0} \leq (\alpha_{12}+1) + \alpha_{13}i_{0}\\ \{i_{0}\};\{\}  \vDash \alpha_{18} + \alpha_{19}i_{0} \leq \alpha_{48} + \alpha_{49}i_{0}\\ \{i_{0}\};\{\}  \vDash \alpha_{41} + \alpha_{42}i_{0} \leq \alpha_{18} + \alpha_{19}i_{0}\\ \{i_{0}\};\{\}  \vDash 0 + 1i_{0} \leq \alpha_{0} + \alpha_{1}i_{0}\\ \{i_{0}\};\{\}  \vDash 0 + 1i_{0} \leq 0 + 1i_{0}\\ \{i_{0}\};\{0 \leq 0\}  \vDash 0 \leq \alpha_{2} + \alpha_{3}i_{0}\\ \{i_{0}\};\{0 \leq 0\}  \vDash \alpha_{2} + \alpha_{3}i_{0} \leq \alpha_{4} + \alpha_{5}i_{0}\\ \{i_{0}\};\{1 \leq \alpha_{0} + \alpha_{1}i_{0}\}  \vDash 0 \leq 0\\ \{i_{0}\};\{1 \leq \alpha_{0} + \alpha_{1}i_{0}\}  \vDash 0 \leq \alpha_{6} + \alpha_{7}i_{0}\\ \{i_{0}\};\{1 \leq \alpha_{0} + \alpha_{1}i_{0}\}  \vDash 0 \leq \alpha_{12} + \alpha_{13}i_{0}\\ \{i_{0}\};\{1 \leq \alpha_{0} + \alpha_{1}i_{0}\}  \vDash 0 \leq \alpha_{18} + \alpha_{19}i_{0}\\ \{i_{0}\};\{1 \leq \alpha_{0} + \alpha_{1}i_{0}\}  \vDash 0 \leq \alpha_{28} + \alpha_{29}i_{0}\\ \{i_{0}\};\{1 \leq \alpha_{0} + \alpha_{1}i_{0}\}  \vDash 0 \leq \alpha_{32} + \alpha_{33}i_{0}\\ \{i_{0}\};\{1 \leq \alpha_{0} + \alpha_{1}i_{0}\}  \vDash 0 \leq \alpha_{36} + \alpha_{37}i_{0}\\ \{i_{0}\};\{1 \leq \alpha_{0} + \alpha_{1}i_{0}\}  \vDash \alpha_{6} + \alpha_{7}i_{0} \leq \alpha_{8} + \alpha_{9}i_{0}\\ \{i_{0}\};\{1 \leq \alpha_{0} + \alpha_{1}i_{0}\}  \vDash (\alpha_{8}+1) + \alpha_{9}i_{0} \leq \alpha_{10} + \alpha_{11}i_{0}\\ \{i_{0}\};\{1 \leq \alpha_{0} + \alpha_{1}i_{0}\}  \vDash \alpha_{12} + \alpha_{13}i_{0} \leq \alpha_{24} + \alpha_{25}i_{0}\\ \{i_{0}\};\{1 \leq \alpha_{0} + \alpha_{1}i_{0}\}  \vDash \alpha_{28} + \alpha_{29}i_{0} \leq \alpha_{30} + \alpha_{31}i_{0}\\ \{i_{0}\};\{1 \leq \alpha_{0} + \alpha_{1}i_{0}\}  \vDash (\alpha_{16}+(\alpha_{19}\alpha_{24}+\alpha_{18})) + (\alpha_{17}+\alpha_{19}\alpha_{25})i_{0} \leq \alpha_{22} + \alpha_{23}i_{0}\\ \{i_{0}\};\{1 \leq \alpha_{0} + \alpha_{1}i_{0}\}  \vDash (\alpha_{30}+\alpha_{32}) + (\alpha_{31}+\alpha_{33})i_{0} \leq \alpha_{34} + \alpha_{35}i_{0}\\ \{i_{0}\};\{1 \leq \alpha_{0} + \alpha_{1}i_{0}\}  \vDash (\alpha_{34}+\alpha_{36}) + (\alpha_{35}+\alpha_{37})i_{0} \leq \alpha_{38} + \alpha_{39}i_{0}
\end{align*}

}

We were able to find the following solution to the reduced constraint satisfaction problem (written as pairs of coefficient variables and naturals)\\

$\splitatcommas{
(\alpha_{0},0),(\alpha_{1},1),(\alpha_{2},1),(\alpha_{3},0),(\alpha_{4},1),(\alpha_{5},0),(\alpha_{6},0),(\alpha_{7},0),(\alpha_{8},0),(\alpha_{9},0),(\alpha_{10},1),(\alpha_{11},0),(\alpha_{12},0),(\alpha_{13},1),(\alpha_{14},1),(\alpha_{15},0),(\alpha_{16},0),(\alpha_{17},0),(\alpha_{18},1),(\alpha_{19},0),(\alpha_{20},1),(\alpha_{21},0),(\alpha_{22},1),(\alpha_{23},0),(\alpha_{24},0),(\alpha_{25},1),(\alpha_{26},1),(\alpha_{27},0),(\alpha_{28},0),(\alpha_{29},0),(\alpha_{30},0),(\alpha_{31},0),(\alpha_{32},0),(\alpha_{33},0),(\alpha_{34},0),(\alpha_{35},0),(\alpha_{36},1),(\alpha_{37},0),(\alpha_{38},1),(\alpha_{39},0),(\alpha_{40},0),(\alpha_{41},1),(\alpha_{42},0),(\alpha_{43},1),(\alpha_{44},0),(\alpha_{45},1),(\alpha_{46},0),(\alpha_{47},0),(\alpha_{48},1),(\alpha_{49},0),(\alpha_{50},1),(\alpha_{51},0),(\alpha_{52},0),(\alpha_{53},0),(\alpha_{54},10),(\alpha_{55},1),(\alpha_{56},0),(\alpha_{57},1),(\alpha_{58},0),(\alpha_{59},1),(\alpha_{60},0),(\alpha_{61},1),(\alpha_{62},10),(\alpha_{63},1),(\alpha_{64},0),(\alpha_{65},1),(\alpha_{66},1)
}$\\

We next consider the process 
\begin{align*}
    &\kern0em P_{\text{npar}}' \defeq\\
    &(\nu \text{npar})(\\
    &\kern2em !\text{npar}(n,r).\texttt{match}\; n\; \{\\
    &\kern3em 0 \mapsto \asyncoutputch{r}{}{}\\
    &\kern3em s(x) \mapsto (\nu r' )(\nu r'' ) {\color{blue}\texttt{tick}}. (\\ &\kern4em {\asyncoutputch{r'}{}{}} \mid
 \asyncoutputch{\text{npar}}{x,r''}{} \mid \inputch{r'}{}{}{\inputch{r''}{}{}{\asyncoutputch{r}{}{}}}) \} \\
    &\kern2em \mid \\
    &(\nu r)( \asyncoutputch{\text{npar}}{m,r}{} \mid \inputch{r}{}{}{\nil} ))
\end{align*}
We infer the following constraints

{ \tiny

\begin{align*}
    \texttt{Nat}[0, \alpha_{52}] \sim \texttt{Nat}[0, \alpha_{53}]\\ \texttt{Nat}[0, \alpha_{62}] \sim \texttt{Nat}[0, (1\alpha_{62}+0)]\\ \texttt{Nat}[0, \alpha_{22} + \alpha_{23}i_{0}] \sim \texttt{Nat}[0, (1\alpha_{22}+0) + 1\alpha_{23}i_{0}]\\ \texttt{ch}^{\gamma_{0}}_{\alpha_{2} + \alpha_{3}i_{0}}() \sim \texttt{ch}^{\gamma_{6}}_{(((\alpha_{26}+\alpha_{30})+\alpha_{34})+1) + ((\alpha_{27}+\alpha_{31})+\alpha_{35})i_{0}}()\\ \texttt{ch}^{\gamma_{1}}_{\alpha_{6} + \alpha_{7}i_{0}}() \sim \texttt{ch}^{\gamma_{8}}_{\alpha_{34} + \alpha_{35}i_{0}}()\\ \texttt{ch}^{\gamma_{2}}_{(\alpha_{12}+\alpha_{14}) + (\alpha_{13}+\alpha_{15})i_{0}}() \sim \texttt{ch}^{\gamma_{7}}_{(\alpha_{30}+\alpha_{34}) + (\alpha_{31}+\alpha_{35})i_{0}}()\\ \texttt{ch}^{\gamma_{5}}_{\alpha_{24} + \alpha_{25}i_{0}}() \sim \texttt{ch}^{\gamma_{4}}_{(\alpha_{19}\alpha_{22}+\alpha_{18}) + \alpha_{19}\alpha_{23}i_{0}}()\\ \texttt{ch}^{\gamma_{13}}_{(\alpha_{55}+\alpha_{56})}() \sim \texttt{ch}^{\gamma_{17}}_{\alpha_{65}}()\\ \texttt{ch}^{\gamma_{16}}_{\alpha_{63}}() \sim \texttt{ch}^{\gamma_{15}}_{(\alpha_{60}\alpha_{62}+\alpha_{59})}()\\ \forall_{\alpha_{40}}{i_{0}}.\texttt{serv}^{\gamma_{9}}_{\alpha_{41} + \alpha_{42}i_{0}}(\texttt{Nat}[0, 0 + 1i_{0}], \texttt{ch}^{\gamma_{10}}_{\alpha_{43} + \alpha_{44}i_{0}}()) \sim \forall_{\alpha_{56}}{i_{0}}.\texttt{serv}^{\gamma_{14}}_{\alpha_{57} + \alpha_{58}i_{0}}(\texttt{Nat}[0, 0 + 1i_{0}], \texttt{ch}^{\gamma_{15}}_{\alpha_{59} + \alpha_{60}i_{0}}())\\ \{\};\{\} \vDash \texttt{inv}(\forall_{(\alpha_{14}+1) + \alpha_{15}i_{0}}{i_{0}}.\texttt{serv}^{\gamma_{3}}_{\alpha_{16} + \alpha_{17}i_{0}}(\texttt{Nat}[0, 0 + 1i_{0}], \texttt{ch}^{\gamma_{4}}_{\alpha_{18} + \alpha_{19}i_{0}}()))\\ () \implies (\{\};\{\}  \vdash \texttt{Nat}[0, \alpha_{54}] \sqsubseteq \texttt{Nat}[0, \alpha_{62}])\\ () \implies (\{\};\{\}  \vdash \texttt{Nat}[0, (\alpha_{53}+10)] \sqsubseteq \texttt{Nat}[0, \alpha_{54}])\\ () \implies (\{\};\{\}  \vdash \texttt{ch}^{\gamma_{13}}_{\alpha_{55}}() \sqsubseteq \texttt{ch}^{\gamma_{16}}_{\alpha_{63}}())\\ () \implies (\{\};\{\}  \vdash \forall_{\alpha_{47}}{i_{0}}.\texttt{serv}^{\gamma_{11}}_{\alpha_{48} + \alpha_{49}i_{0}}(\texttt{Nat}[0, 0 + 1i_{0}], \texttt{ch}^{\gamma_{12}}_{\alpha_{50} + \alpha_{51}i_{0}}()) \sqsubseteq \forall_{\alpha_{46}}{i_{0}}.\texttt{serv}^{\gamma_{3}}_{\alpha_{16} + \alpha_{17}i_{0}}(\texttt{Nat}[0, 0 + 1i_{0}], \texttt{ch}^{\gamma_{4}}_{\alpha_{18} + \alpha_{19}i_{0}}()))\\ () \implies (\{i_{0}\};\{\}  \vdash \texttt{Nat}[0, \alpha_{0} + \alpha_{1}i_{0}] \sqsubseteq \texttt{Nat}[0, (\alpha_{10}+1) + \alpha_{11}i_{0}])\\ () \implies (\{i_{0}\};\{\}  \vdash \texttt{Nat}[0, 0 + 1i_{0}] \sqsubseteq \texttt{Nat}[0, \alpha_{0} + \alpha_{1}i_{0}])\\ () \implies (\{i_{0}\};\{\}  \vdash \texttt{Nat}[0, 0 + 1i_{0}] \sqsubseteq \texttt{Nat}[0, 0 + 1i_{0}])\\ () \implies (\{i_{0}\};\{\}  \vdash \texttt{ch}^{\gamma_{4}}_{\alpha_{18} + \alpha_{19}i_{0}}() \sqsubseteq \texttt{ch}^{\gamma_{10}}_{\alpha_{43} + \alpha_{44}i_{0}}())\\ () \implies (\{i_{0}\};\{\}  \vdash \texttt{ch}^{\gamma_{10}}_{\alpha_{43} + \alpha_{44}i_{0}}() \sqsubseteq \texttt{ch}^{\gamma_{0}}_{\alpha_{2} + \alpha_{3}i_{0}}())\\ () \implies (\{i_{0}\};\{1 \leq \alpha_{0} + \alpha_{1}i_{0}\}  \vdash \texttt{Nat}[0, \alpha_{10} + \alpha_{11}i_{0}] \sqsubseteq \texttt{Nat}[0, \alpha_{22} + \alpha_{23}i_{0}])\\ () \implies (\{i_{0}\};\{1 \leq \alpha_{0} + \alpha_{1}i_{0}\}  \vdash \texttt{ch}^{\gamma_{2}}_{\alpha_{12} + \alpha_{13}i_{0}}() \sqsubseteq \texttt{ch}^{\gamma_{5}}_{\alpha_{24} + \alpha_{25}i_{0}}())\\ () \implies (\{\};\{\}  \vDash \alpha_{40} \leq \alpha_{45})\\ () \implies (\{\};\{\}  \vDash \alpha_{46} \leq \alpha_{40})\\ () \implies (\{\};\{\}  \vDash 0 \leq \alpha_{40}), () \implies (\{\};\{\}  \vDash 0 \leq \alpha_{45}), () \implies (\{\};\{\}  \vDash 0 \leq \alpha_{52})\\ () \implies (\{\};\{\}  \vDash 0 \leq \alpha_{56}), () \implies (\{\};\{\}  \vDash 0 \leq \alpha_{64}), () \implies (\{\};\{\}  \vDash 0 \leq \alpha_{65})\\ () \implies (\{\};\{\}  \vDash 0 \leq \alpha_{57} + \alpha_{58}i_{0})\\ () \implies (\{\};\{\}  \vDash (\alpha_{56}+(\alpha_{58}\alpha_{62}+\alpha_{57})) \leq \alpha_{61})\\ () \implies (\{\};\{\}  \vDash (\alpha_{64}+\alpha_{65}) \leq \alpha_{66})\\ () \implies (\{i_{0}\};\{\}  \vDash 0 \leq \alpha_{0} + \alpha_{1}i_{0})\\ () \implies (\{i_{0}\};\{\}  \vDash \alpha_{41} + \alpha_{42}i_{0} \leq \alpha_{16} + \alpha_{17}i_{0})\\ () \implies (\{i_{0}\};\{0 \leq 0\}  \vDash 0 \leq \alpha_{2} + \alpha_{3}i_{0})\\ () \implies (\{i_{0}\};\{0 \leq 0\}  \vDash \alpha_{2} + \alpha_{3}i_{0} \leq \alpha_{4} + \alpha_{5}i_{0})\\ () \implies (\{i_{0}\};\{1 \leq \alpha_{0} + \alpha_{1}i_{0}\}  \vDash 0 \leq \alpha_{6} + \alpha_{7}i_{0})\\ () \implies (\{i_{0}\};\{1 \leq \alpha_{0} + \alpha_{1}i_{0}\}  \vDash 0 \leq \alpha_{10} + \alpha_{11}i_{0})\\ () \implies (\{i_{0}\};\{1 \leq \alpha_{0} + \alpha_{1}i_{0}\}  \vDash 0 \leq \alpha_{16} + \alpha_{17}i_{0})\\ () \implies (\{i_{0}\};\{1 \leq \alpha_{0} + \alpha_{1}i_{0}\}  \vDash 0 \leq \alpha_{26} + \alpha_{27}i_{0})\\ () \implies (\{i_{0}\};\{1 \leq \alpha_{0} + \alpha_{1}i_{0}\}  \vDash 0 \leq \alpha_{30} + \alpha_{31}i_{0})\\ () \implies (\{i_{0}\};\{1 \leq \alpha_{0} + \alpha_{1}i_{0}\}  \vDash 0 \leq \alpha_{34} + \alpha_{35}i_{0})\\ () \implies (\{i_{0}\};\{1 \leq \alpha_{0} + \alpha_{1}i_{0}\}  \vDash \alpha_{6} + \alpha_{7}i_{0} \leq \alpha_{8} + \alpha_{9}i_{0})\\ () \implies (\{i_{0}\};\{1 \leq \alpha_{0} + \alpha_{1}i_{0}\}  \vDash (\alpha_{8}+1) + \alpha_{9}i_{0} \leq \alpha_{38} + \alpha_{39}i_{0})\\ () \implies (\{i_{0}\};\{1 \leq \alpha_{0} + \alpha_{1}i_{0}\}  \vDash \alpha_{26} + \alpha_{27}i_{0} \leq \alpha_{28} + \alpha_{29}i_{0})\\ () \implies (\{i_{0}\};\{1 \leq \alpha_{0} + \alpha_{1}i_{0}\}  \vDash (\alpha_{14}+(\alpha_{17}\alpha_{22}+\alpha_{16})) + (\alpha_{15}+\alpha_{17}\alpha_{23})i_{0} \leq \alpha_{20} + \alpha_{21}i_{0})\\ () \implies (\{i_{0}\};\{1 \leq \alpha_{0} + \alpha_{1}i_{0}\}  \vDash (\alpha_{28}+\alpha_{30}) + (\alpha_{29}+\alpha_{31})i_{0} \leq \alpha_{32} + \alpha_{33}i_{0})\\ () \implies (\{i_{0}\};\{1 \leq \alpha_{0} + \alpha_{1}i_{0}\}  \vDash (\alpha_{32}+\alpha_{34}) + (\alpha_{33}+\alpha_{35})i_{0} \leq \alpha_{36} + \alpha_{37}i_{0})\\ 
    () \implies \gamma_{3} \subseteq \gamma_{9}, () \implies \{\texttt{in}\} \subseteq \gamma_{7}, () \implies \{\texttt{in}\} \subseteq \gamma_{8}\\ 
    () \implies \{\texttt{in}\} \subseteq \gamma_{9}, () \implies \{\texttt{in}\} \subseteq \gamma_{17}, () \implies \{\texttt{out}\} \subseteq \gamma_{0}, () \implies \{\texttt{out}\} \subseteq \gamma_{1}\\ 
    () \implies \{\texttt{out}\} \subseteq \gamma_{3}, () \implies \{\texttt{out}\} \subseteq \gamma_{6}, () \implies \{\texttt{out}\} \subseteq \gamma_{14}\\ \alpha_{45} \sim \alpha_{61}\\ \alpha_{61} \sim \alpha_{66}\\ 0 \sim \alpha_{14} + \alpha_{15}i_{0}\\ \alpha_{4} + \alpha_{5}i_{0} \sim \alpha_{38} + \alpha_{39}i_{0}, \alpha_{4} + \alpha_{5}i_{0} \sim \alpha_{41} + \alpha_{42}i_{0}, \alpha_{8} + \alpha_{9}i_{0} \sim \alpha_{20} + \alpha_{21}i_{0}, \alpha_{8} + \alpha_{9}i_{0} \sim \alpha_{36} + \alpha_{37}i_{0}
\end{align*}

}

We reduce the constraints to

{\tiny
\begin{align*}
    \alpha_{40} \sim \alpha_{56}, \alpha_{45} \sim \alpha_{61}, \alpha_{47} \sim \alpha_{46}, \alpha_{52} \sim \alpha_{53}, \alpha_{55} \sim \alpha_{63}, \alpha_{61} \sim \alpha_{66}, \alpha_{62} \sim (1\alpha_{62}+0)\\ 
    \alpha_{63} \sim (\alpha_{60}\alpha_{62}+\alpha_{59}), 0 \sim 0, 0 \sim \alpha_{14} + \alpha_{15}i_{0}, (\alpha_{55}+\alpha_{56}) \sim \alpha_{65}\\ \alpha_{2} + \alpha_{3}i_{0} \sim (((\alpha_{26}+\alpha_{30})+\alpha_{34})+1) + ((\alpha_{27}+\alpha_{31})+\alpha_{35})i_{0}\\ \alpha_{4} + \alpha_{5}i_{0} \sim \alpha_{38} + \alpha_{39}i_{0}\\ \alpha_{4} + \alpha_{5}i_{0} \sim \alpha_{41} + \alpha_{42}i_{0}\\ \alpha_{6} + \alpha_{7}i_{0} \sim \alpha_{34} + \alpha_{35}i_{0}\\ \alpha_{8} + \alpha_{9}i_{0} \sim \alpha_{20} + \alpha_{21}i_{0}\\ \alpha_{8} + \alpha_{9}i_{0} \sim \alpha_{36} + \alpha_{37}i_{0}\\ \alpha_{12} + \alpha_{13}i_{0} \sim \alpha_{24} + \alpha_{25}i_{0}\\ \alpha_{18} + \alpha_{19}i_{0} \sim \alpha_{43} + \alpha_{44}i_{0}\\ \alpha_{18} + \alpha_{19}i_{0} \sim \alpha_{50} + \alpha_{51}i_{0}\\ \alpha_{22} + \alpha_{23}i_{0} \sim (1\alpha_{22}+0) + 1\alpha_{23}i_{0}\\ \alpha_{24} + \alpha_{25}i_{0} \sim (\alpha_{19}\alpha_{22}+\alpha_{18}) + \alpha_{19}\alpha_{23}i_{0}\\ \alpha_{41} + \alpha_{42}i_{0} \sim \alpha_{57} + \alpha_{58}i_{0}\\ \alpha_{43} + \alpha_{44}i_{0} \sim \alpha_{2} + \alpha_{3}i_{0}\\ \alpha_{43} + \alpha_{44}i_{0} \sim \alpha_{59} + \alpha_{60}i_{0}\\ \alpha_{50} + \alpha_{51}i_{0} \sim \alpha_{18} + \alpha_{19}i_{0}\\ 0 + 1i_{0} \sim 0 + 1i_{0}\\ (\alpha_{12}+\alpha_{14}) + (\alpha_{13}+\alpha_{15})i_{0} \sim (\alpha_{30}+\alpha_{34}) + (\alpha_{31}+\alpha_{35})i_{0}\\ \{\};\{\}  \vDash \alpha_{40} \leq \alpha_{45}\\ \{\};\{\}  \vDash \alpha_{46} \leq \alpha_{40}\\ \{\};\{\}  \vDash \alpha_{54} \leq \alpha_{62}\\ \{\};\{\}  \vDash 0 \leq \alpha_{40}, \{\};\{\}  \vDash 0 \leq \alpha_{45}, \{\};\{\}  \vDash 0 \leq \alpha_{52}\\ \{\};\{\}  \vDash 0 \leq \alpha_{56}, \{\};\{\}  \vDash 0 \leq \alpha_{64}, \{\};\{\}  \vDash 0 \leq \alpha_{65}\\ \{\};\{\}  \vDash 0 \leq 0\\ \{\};\{\}  \vDash 0 \leq \alpha_{57} + \alpha_{58}i_{0}\\ \{\};\{\}  \vDash (\alpha_{53}+10) \leq \alpha_{54}\\ \{\};\{\}  \vDash (\alpha_{56}+(\alpha_{58}\alpha_{62}+\alpha_{57})) \leq \alpha_{61}\\ \{\};\{\}  \vDash (\alpha_{64}+\alpha_{65}) \leq \alpha_{66}\\ \{i_{0}\};\{\}  \vDash 0 \leq 0\\ \{i_{0}\};\{\}  \vDash 0 \leq \alpha_{0} + \alpha_{1}i_{0}\\ \{i_{0}\};\{\}  \vDash \alpha_{0} + \alpha_{1}i_{0} \leq (\alpha_{10}+1) + \alpha_{11}i_{0}\\ \{i_{0}\};\{\}  \vDash \alpha_{16} + \alpha_{17}i_{0} \leq \alpha_{48} + \alpha_{49}i_{0}\\ \{i_{0}\};\{\}  \vDash \alpha_{41} + \alpha_{42}i_{0} \leq \alpha_{16} + \alpha_{17}i_{0}\\ \{i_{0}\};\{\}  \vDash 0 + 1i_{0} \leq \alpha_{0} + \alpha_{1}i_{0}\\ \{i_{0}\};\{\}  \vDash 0 + 1i_{0} \leq 0 + 1i_{0}\\ \{i_{0}\};\{0 \leq 0\}  \vDash 0 \leq \alpha_{2} + \alpha_{3}i_{0}\\ \{i_{0}\};\{0 \leq 0\}  \vDash \alpha_{2} + \alpha_{3}i_{0} \leq \alpha_{4} + \alpha_{5}i_{0}\\ \{i_{0}\};\{1 \leq \alpha_{0} + \alpha_{1}i_{0}\}  \vDash 0 \leq 0\\ \{i_{0}\};\{1 \leq \alpha_{0} + \alpha_{1}i_{0}\}  \vDash 0 \leq \alpha_{6} + \alpha_{7}i_{0}\\ \{i_{0}\};\{1 \leq \alpha_{0} + \alpha_{1}i_{0}\}  \vDash 0 \leq \alpha_{10} + \alpha_{11}i_{0}\\ \{i_{0}\};\{1 \leq \alpha_{0} + \alpha_{1}i_{0}\}  \vDash 0 \leq \alpha_{16} + \alpha_{17}i_{0}\\ \{i_{0}\};\{1 \leq \alpha_{0} + \alpha_{1}i_{0}\}  \vDash 0 \leq \alpha_{26} + \alpha_{27}i_{0}\\ \{i_{0}\};\{1 \leq \alpha_{0} + \alpha_{1}i_{0}\}  \vDash 0 \leq \alpha_{30} + \alpha_{31}i_{0}\\ \{i_{0}\};\{1 \leq \alpha_{0} + \alpha_{1}i_{0}\}  \vDash 0 \leq \alpha_{34} + \alpha_{35}i_{0}\\ \{i_{0}\};\{1 \leq \alpha_{0} + \alpha_{1}i_{0}\}  \vDash \alpha_{6} + \alpha_{7}i_{0} \leq \alpha_{8} + \alpha_{9}i_{0}\\ \{i_{0}\};\{1 \leq \alpha_{0} + \alpha_{1}i_{0}\}  \vDash (\alpha_{8}+1) + \alpha_{9}i_{0} \leq \alpha_{38} + \alpha_{39}i_{0}\\ \{i_{0}\};\{1 \leq \alpha_{0} + \alpha_{1}i_{0}\}  \vDash \alpha_{10} + \alpha_{11}i_{0} \leq \alpha_{22} + \alpha_{23}i_{0}\\ \{i_{0}\};\{1 \leq \alpha_{0} + \alpha_{1}i_{0}\}  \vDash \alpha_{26} + \alpha_{27}i_{0} \leq \alpha_{28} + \alpha_{29}i_{0}\\ \{i_{0}\};\{1 \leq \alpha_{0} + \alpha_{1}i_{0}\}  \vDash (\alpha_{14}+(\alpha_{17}\alpha_{22}+\alpha_{16})) + (\alpha_{15}+\alpha_{17}\alpha_{23})i_{0} \leq \alpha_{20} + \alpha_{21}i_{0}\\ \{i_{0}\};\{1 \leq \alpha_{0} + \alpha_{1}i_{0}\}  \vDash (\alpha_{28}+\alpha_{30}) + (\alpha_{29}+\alpha_{31})i_{0} \leq \alpha_{32} + \alpha_{33}i_{0}\\ \{i_{0}\};\{1 \leq \alpha_{0} + \alpha_{1}i_{0}\}  \vDash (\alpha_{32}+\alpha_{34}) + (\alpha_{33}+\alpha_{35})i_{0} \leq \alpha_{36} + \alpha_{37}i_{0}
\end{align*}
}

We find the following solution to the reduced constraint satisfaction problem (written as pairs of coefficient variables and natural numbers)\\

$\splitatcommas{(\alpha_{0},0),(\alpha_{1},1),(\alpha_{2},0),(\alpha_{3},1),(\alpha_{4},0),(\alpha_{5},1),(\alpha_{6},0),(\alpha_{7},0),(\alpha_{8},0),(\alpha_{9},0),(\alpha_{10},-1),(\alpha_{11},1),(\alpha_{12},-2),(\alpha_{13},2),(\alpha_{14},0),(\alpha_{15},0),(\alpha_{16},0),(\alpha_{17},2),(\alpha_{18},0),(\alpha_{19},1),(\alpha_{20},0),(\alpha_{21},0),(\alpha_{22},-2),(\alpha_{23},2),(\alpha_{24},-2),(\alpha_{25},2),(\alpha_{26},1),(\alpha_{27},-1),(\alpha_{28},0),(\alpha_{29},0),(\alpha_{30},-2),(\alpha_{31},2),(\alpha_{32},-1),(\alpha_{33},1),(\alpha_{34},0),(\alpha_{35},0),(\alpha_{36},0),(\alpha_{37},0),(\alpha_{38},0),(\alpha_{39},1),(\alpha_{40},0),(\alpha_{41},0),(\alpha_{42},1),(\alpha_{43},0),(\alpha_{44},1),(\alpha_{45},10),(\alpha_{46},0),(\alpha_{47},0),(\alpha_{48},0),(\alpha_{49},2),(\alpha_{50},0),(\alpha_{51},1),(\alpha_{52},0),(\alpha_{53},0),(\alpha_{54},10),(\alpha_{55},10),(\alpha_{56},0),(\alpha_{57},0),(\alpha_{58},1),(\alpha_{59},0),(\alpha_{60},1),(\alpha_{61},10),(\alpha_{62},10),(\alpha_{63},10),(\alpha_{64},0),(\alpha_{65},10),(\alpha_{66},10)}$

%\input{sections/turingcompleteness}
%\appendix
%\chapter{Semantic equivalence}\label{app:languageequiv}
\setcounter{theorem}{1}

\begin{lemma}
Let $P$ and $Q$ be processes such that $P \equiv Q$, then also $P \succeq_K Q$ and $Q \succeq_K P$.\\

where $\succeq_{K}$ is the structural preorder from Kobayashi \cite{Kobayashi2000}.
\begin{proof} By induction on the rules defining $\equiv$. We only consider the rules that are unique to $\equiv$.
\begin{description}
    \item[$\runa{SC-par}$] By rule $\runa{SC-par}$ we have that $P \equiv P'$ such that $P \mid Q \equiv P' \mid Q$. By induction $P \succeq_K P'$ and $P' \succeq_K P$. Then it follows from $\runa{SP-par}$ that $P \mid Q \succeq_K P' \mid Q$ and $P' \mid Q \succeq_k P \mid Q$.
    
    \item[$\runa{SC-res}$] By rule $\runa{SC-res}$ we have that $P \equiv Q$ such that $\newvar{x}{P} \equiv \newvar{x}{Q}$. By induction $P \succeq_K Q$ and $Q \succeq_K P$. Then it follows from $\runa{SP-res}$ that $\newvar{x}{P} \succeq_K \newvar{x}{Q}$ and $\newvar{x}{Q} \succeq_k \newvar{x}{P}$.
\end{description}

\end{proof}%\label{lemma:SCImpliesSP}
\end{lemma}

\begin{lemma}[Semantic equivalence]
Let $P$ and $P'$ be processes. We have that if
\begin{enumerate}
    \item $P \longrightarrow Q$ then $P \longrightarrow_{K} Q$
    \item $P \longrightarrow_{K} Q$ and $P' \succeq_K P$ then $P' \longrightarrow R$ and $R \succeq_{K} Q$
\end{enumerate}
where $\longrightarrow_{K}$ and $\succeq_{K}$ are the reduction relation and structural preorder from Kobayashi \cite{Kobayashi2000}, respectively.
\begin{proof} By induction on the rules defining $\longrightarrow$ and $\longrightarrow_{K}$. We consider the two cases separately, and we only show the clauses for non-trivial rules.
\begin{enumerate}
    \item 
    \begin{description}
        \item[\runa{R-rep}] Assume that $\bang\inputch{a}{\widetilde{v}}{}{P} \mid \outputch{a}{\widetilde{e}}{}{Q} \longrightarrow\; \bang\inputch{a}{\widetilde{v}}{}{P} \mid P[\widetilde{v}\mapsto\widetilde{e}] \mid Q$. We have that $\bang\inputch{a}{\widetilde{v}}{}{P} \mid \outputch{a}{\widetilde{e}}{}{Q}\succeq_K\; \bang\inputch{a}{\widetilde{v}}{}{P} \mid \inputch{a}{\widetilde{v}}{}{P} \mid \outputch{a}{\widetilde{e}}{}{Q}$, and so it follows from application of $\runa{K-struct}$ and $\runa{K-comm}$ that $\bang\inputch{a}{\widetilde{v}}{}{P} \mid \outputch{a}{\widetilde{e}}{}{Q} \longrightarrow_K\; \bang\inputch{a}{\widetilde{v}}{}{P} \mid P[\widetilde{v}\mapsto\widetilde{e}] \mid Q$.
        %
        %\item[\runa{R-comm}] Assume that $\inputch{a}{\widetilde{v}}{}{P} \mid \outputch{a}{\widetilde{e}}{}{Q} \longrightarrow P[\widetilde{v}\mapsto\widetilde{e}] \mid Q$. It trivially follows from application of $\runa{K-comm}$ that $\inputch{a}{\widetilde{v}}{}{P} \mid \outputch{a}{\widetilde{e}}{}{Q} \longrightarrow_K P[\widetilde{v}\mapsto\widetilde{e}] \mid Q$, as $\succeq_K$ is reflexive.
        %
        %\item[\runa{R-zero}] Assume that $\match{e}{P}{x}{Q} \longrightarrow P$
        %
        %\item[\runa{R-succ}]
        %
        %\item[\runa{R-par}]
        %
        %\item[\runa{R-res}]
        
        \item[\runa{R-struct}] Assume that $P \longrightarrow Q$ by rule $\runa{R-struct}$. Then we have that $P \equiv P_1$, $P_1 \longrightarrow P_1'$ and $P_1' \equiv Q$. By Lemma \ref{lemma:SCImpliesSP} it follows that $P \succeq_K P_1$ and $P_1' \succeq_K Q$. By induction $P' \longrightarrow_K Q'$, and so by rule $\runa{K-struct}$, $P \longrightarrow_K Q$.
            %\begin{description}
            %    \item[\runa{SC-par}]
            %    
            %    \item[\runa{SC-res}]
            %\end{description}
        
        
    \end{description} 
    
    \item
    \begin{description}
        %\item[\runa{K-comm}]
        %    
        %\item[\runa{K-zero}]
        %
        %\item[\runa{K-succ}]
        %
        %\item[\runa{K-par}]
        %
        %\item[\runa{K-res-1}]
        %
        %\item[\runa{K-res-2}]
        %
        \item[\runa{K-struct}] Assume that $P \longrightarrow_K Q$ by rule $\runa{K-struct}$ and $P' \succeq_K P$. Then we have that $P \succeq_K P_2$, $P_2 \longrightarrow_K P_2'$ and $P_2' \succeq_K Q$. We show that $P'$ can match this reduction by rule $\runa{R-struct}$ such that $P' \longrightarrow R$ and $R \succeq_K Q$. Rule $\runa{R-struct}$ implies $P' \equiv P_1$ such that $P_1 \longrightarrow P_1'$ and $P_1' \equiv R$. By Lemma \ref{lemma:SCImpliesSP}, $P' \equiv P_1$ implies $P' \succeq_K P_1$ and $P_1 \succeq_K P'$. Thus by the assumption, it must be that $P_1 \succeq_K P$, and by transitivity $P \succeq_K P_2$ implies $P_1 \succeq_K P_2$. By induction $P_1 \succeq_K P_2$ and $P_2 \longrightarrow_K P_2'$ implies $P_1 \longrightarrow P_1'$ and $P_1' \succeq_K P_2'$. Now, it suffices to show that $P_1' \equiv R$ and $P_2' \succeq_K Q$ implies $R \succeq_K Q$. By Lemma \ref{lemma:SCImpliesSP}, $P_1' \equiv R$ implies $P_1' \succeq_K R$ and $R \succeq_K P_1'$, and so it follows from $P_1' \succeq_K P_2'$ that $R \succeq_K P_2'$. By transitivity $P_2' \succeq_K Q$ implies $R \succeq_K Q$, concluding the proof.
        
        
        %By the definition of $\succeq_K$, we have that $P \succeq_K P'' %\mid P_1 \mid \dots \mid P_n$ such that $P' \succeq P''$ and %$P_1,\dots,P_n$ are copies of replicated inputs in $P'$.
        %    \begin{description}
        %        \item[\runa{SP-rep-1}]
        %        
        %        \item[\runa{SP-par}]
        %        
        %        \item[\runa{SP-res}]
        %        
        %        \item[\runa{SP-rep-2}]
        %    \end{description}
            
    \end{description}
\end{enumerate}
\end{proof}
\end{lemma}
%
%%
%
% \begin{lemma}
% Let $P$ and $Q$ be processes such that $P \succeq_{K} Q$. We have that if
% \begin{enumerate}
%     \item $P \longrightarrow R$ then $Q \longrightarrow C$ and $R \succeq_K C$
%     \item $Q \longrightarrow R$ then $P \longrightarrow C$ and $C \succeq_K R$
%     \item $P \longrightarrow_{K} R$ then $Q \longrightarrow_K C$ and $R \succeq_K C$ or $C \succeq_K R$
%     \item $Q \longrightarrow_{K} R$ then $P \longrightarrow_{K} R$
% \end{enumerate}
% where $\longrightarrow_{K}$ and $\succeq_{K}$ is the reduction relation and structural preorder from Kobayashi \cite{Kobayashi2000}, respectively.
% \begin{proof}
% Proof by induction on the rules defining $\longrightarrow$ and $\longrightarrow_{K}$. We consider each case separately.
% \begin{enumerate}
%     \item
%         \begin{description}
%             \item[\runa{R-rep}]
            
%             \item[\runa{R-comm}]
            
%             \item[\runa{R-zero}]
            
%             \item[\runa{R-succ}]
            
%             \item[\runa{R-par}]
            
%             \item[\runa{R-res}]
            
%             \item[\runa{R-struct}]
            
%         \end{description}
        
%     \item 
%         \begin{description}
%             \item[\runa{R-rep}]
            
%             \item[\runa{R-comm}]
            
%             \item[\runa{R-zero}]
            
%             \item[\runa{R-succ}]
            
%             \item[\runa{R-par}]
            
%             \item[\runa{R-res}]
            
%             \item[\runa{R-struct}]
            
%         \end{description}
%     \item 
%         \begin{description}
%             \item[\runa{K-comm}]
            
%             \item[\runa{K-zero}]
            
%             \item[\runa{K-succ}]
            
%             \item[\runa{K-par}]
            
%             \item[\runa{K-res-1}]
            
%             \item[\runa{K-res-2}]
            
%             \item[\runa{K-struct}]
            
%         \end{description}
        
%     \item As $P \succeq_K Q$, this trivially holds by applying rule $\runa{K-struct}$ with a corresponding reduction, such that $P \longrightarrow_K R$.
% \end{enumerate}
% \end{proof}
% \end{lemma}
%\chapter{Session type soundness}\label{app:dasetallsoundness}
\setcounter{theorem}{11}
%

\begin{lemma}
Let $P$ be an arbitrary process such that $b$ is not free in $P$. 
\begin{enumerate}
\item If $\Delta,a:A\vdash P :: c\!:\!C$ then $\Delta,b:A\vdash P[a\mapsto b] :: c\!:\!C$.

\item If $\Delta\vdash P :: a\!:\!A$ then $\Delta\vdash P[a\mapsto b] :: b\!:\!A$.
\end{enumerate}
\begin{proof}
By induction on the type rules
\begin{description}
\item[$\runa{TS-$\mathbf{1}$L}$] We have that $\Delta,a:\mathbf{1} \vdash P :: c\!:\!C$ because $\Delta \vdash P :: c\!:\!C$. We consider the cases separately
\begin{enumerate}
    \item Let $d\in\text{dom}(\Delta)$. Then there exists $\Delta'$ and $D$ such that $\Delta=\Delta',d:D$. By induction we then have that $\Delta',b:D\vdash P[d\mapsto b] :: c\!:\!C$, and by application of $\runa{TS-$\mathbf{1}$L}$ we obtain $\Delta',b:D,a:\mathbf{1}\vdash P[d\mapsto b] :: c\!:\!C$ as required. We obtain the special case $\Delta,b:\mathbf{1}\vdash P[a\mapsto b] :: c\!:\!C$ directly from application of \runa{TS-$\mathbf{1}$L}.
    
    \item By induction we have that $\Delta\vdash P[c\mapsto b] :: b\!:\!C$, then by application of $\runa{TS-$\mathbf{1}$L}$ we obtain $\Delta,a:\mathbf{1}\vdash P[c\mapsto b] :: b\!:\!C$, as required.
\end{enumerate}

% We consider (1) first. Then for $P[d\mapsto b]$ we either have $d=a$ or $\Delta = \Delta',d:D$. In the first case we obtain $\Delta,b:\mathbf{1}\vdash P[a\mapsto b] :: c\!:\!C$ directly from \runa{TS-$\mathbf{1}$L} as $P[a\mapsto b]=P$ and $P$ can consume a session of type $\mathbf{1}$ for any name. For the second case we have by induction that $\Delta',b:D\vdash P[d\mapsto b] :: c\!:\!C$, and from $\runa{TS-$\mathbf{1}$L}$ we obtain $\Delta',b:D,a:\mathbf{1}\vdash P[d\mapsto b] :: c\!:\!C$.\\

% We then consider (2). For $P[d\mapsto b]$ we have $d=c$. Then we have by induction that $\Delta\vdash P[c\mapsto b] :: b\!:\!C$, and it follows from $\runa{TS-$\mathbf{1}$L}$ that also $\Delta,a:\mathbf{1}\vdash P[c\mapsto b] :: b\!:\!C$.

\item[$\runa{TS-$\mathbf{1}$R}$] We have that $\cdot\vdash \mathbf{0} :: a\!:\!\mathbf{1}$ and $\mathbf{0}[a\mapsto b]=\mathbf{0}$. (1) does not apply, as $\cdot$ is the empty type context. For (2) we obtain $\cdot \vdash \mathbf{0}[a\mapsto b] :: b\!:\!\mathbf{1}$ directly by application of $\runa{TS-$\mathbf{1}$R}$, as $\mathbf{0}$ can provide a session on any name.

\item[$\runa{TS-$\otimes$L}$] We have that $\Delta,a:A\otimes B \vdash \inputch{a}{v}{}{P'} :: c\!:\!C$ because $\Delta,v:A,a:B\vdash P' :: c\!:\!C$. We consider the cases separately
\begin{enumerate}
    \item We either replace $a$ or some $d\in\text{dom}(\Delta)$. We consider them separately
    \begin{itemize}
        \item We have that $(\inputch{a}{v}{}{P'})[a\mapsto b]=\inputch{b}{v}{}{P'}[a\mapsto b]$. By induction $\Delta,v:A,b:B\vdash P[a\mapsto b] :: c\!:\!C$ and by application of $\runa{TS-$\otimes$L}$ we obtain $\Delta,b:A\otimes B\vdash \inputch{b}{v}{}{P[a\mapsto b]} :: c\!:\!C$ as required.
        
        \item There exists $\Delta'$ and $D$ such that $\Delta=\Delta',d:D$ and we have that $(\inputch{a}{v}{}{P'})[d\mapsto b]=\inputch{a}{v}{}{P'}[d\mapsto b]$. By induction $\Delta',b:D,v:a,a:B\vdash P[d\mapsto b] :: c\!:\!C$ and by application of $\runa{TS-$\otimes$L}$ we obtain $\Delta',b:D,a:A\otimes B\vdash \inputch{a}{v}{}{P[d\mapsto b]} :: c\!:\!C$ as required. 
    \end{itemize}
    
    \item We have that $(\inputch{a}{v}{}{P'})[c\mapsto b]=\inputch{a}{v}{}{P'[c\mapsto b]}$. By induction we have that $\Delta,v:A,a:B\vdash P'[c\mapsto b] :: b\!:\!C$, and by application of $\runa{TS-$\otimes$L}$ we obtain $\Delta,a:A\otimes B \vdash \inputch{a}{v}{}{P'[c\mapsto b]} :: b\!:\!C$ as required.
\end{enumerate}

\item[$\runa{TS-$\otimes$R}$] We have that $\Delta,v:A\vdash \outputch{a}{v}{}{P'} :: a\!:\!A\otimes B$ because $\Delta\vdash P' :: a\!:\!B$. We consider the cases separately
\begin{enumerate}
    \item We replace some $d\in\text{dom}(\Delta)$ or $v$. We consider them separately
    \begin{itemize}
        \item There exists $\Delta'$ and $D$ such that $\Delta=\Delta',d:D$ and we have that $(\outputch{a}{v}{}{P'})[d\mapsto b]=\outputch{a}{v}{}{P'}[d\mapsto b]$. By induction $\Delta',b:D\vdash P[d\mapsto b] :: a\!:\!B$ and by application of $\runa{TS-$\otimes$R}$ we obtain $\Delta',b:D,v:A\vdash \outputch{a}{v}{}{P[d\mapsto b]} :: a\!:\!A\otimes B$ as required. 
        
        \item We have that $(\outputch{a}{v}{}{P'})[v\mapsto b]=\outputch{a}{b}{}{P'}$. From $\Delta\vdash P' :: a\!:\!B$ we then obtain $\Delta,b:A\vdash \outputch{a}{b}{}{P'} :: a\!:\!A\otimes B$ directly by application of $\runa{TS-$\otimes$R}$.
    \end{itemize}
    
    \item We have that $(\outputch{a}{v}{}{P'})[a\mapsto b]=\outputch{b}{v}{}{P'[a\mapsto b]}$. By induction we have that $\Delta\vdash P'[a\mapsto b] :: b\!:\!B$, and by application of $\runa{TS-$\otimes$R}$ we obtain $\Delta,v:A \vdash \outputch{b}{v}{}{P'[a\mapsto b]} :: b\!:\!A\otimes B$ as required.
\end{enumerate}

%We have that $\Delta,v:A \vdash \outputch{a}{v}{}{P'} :: a\!:\!A\otimes B$ and $\Delta\vdash P' :: a\!:\!B$. The first part of the lemma applies to $(\outputch{a}{v}{}{P'})[v\mapsto b]=\outputch{a}{b}{}{P'[v\mapsto b]}$ and to $(\outputch{a}{v}{}{P'})[d\mapsto b]=\outputch{a}{v}{}{P'[d\mapsto b]}$ when $\Delta=\Delta',d:D$. For the first case we obtain $\Delta,b:A\vdash \outputch{a}{b}{}{P'[v\mapsto b]} :: a\!:\!A\otimes B$ directly from $\runa{TS-$\otimes$R}$, as it must be that $P'[v\mapsto b]=P'$ since $\Delta\vdash P' :: a\!:\!B$. In the second case we have by induction that $\Delta',b:D\vdash P'[d\mapsto b] :: a\!:\!B$ and from $\runa{TS-$\otimes$R}$ we obtain $\Delta',b:D,v:A \vdash \outputch{a}{v}{}{P'[d\mapsto b]} :: a\!:\!A\otimes B$. The second part of the lemma applies to $(\outputch{a}{v}{}{P'})[a\mapsto b]=\outputch{b}{v}{}{P'[a\mapsto b]}$. Then we have by induction that $\Delta\vdash P'[a\mapsto b] :: b\!:\!B$, and it follows from $\runa{TS-$\otimes$R}$ that also $\Delta,v:A \vdash \outputch{b}{v}{}{P'[a\mapsto b]} :: b\!:\!A\otimes B$.

\item[$\runa{TS-$\multimap$L}$] We have that $\Delta,a:A\multimap B,v:a \vdash \outputch{a}{v}{}{P'} :: c\!:\!C$ because $\Delta,a:B\vdash P' :: c\!:\!C$. We consider the cases separately
\begin{enumerate}
    \item We either replace $a$, some $d\in\text{dom}(\Delta)$ or $v$. We consider them separately
    \begin{itemize}
        \item We have that $(\outputch{a}{v}{}{P'})[a\mapsto b]=\outputch{b}{v}{}{P'}[a\mapsto b]$. By induction $\Delta,b:B\vdash P[a\mapsto b] :: c\!:\!C$ and by application of $\runa{TS-$\multimap$L}$ we obtain $\Delta,b:A\multimap B,v:A\vdash \outputch{b}{v}{}{P[a\mapsto b]} :: c\!:\!C$ as required.
        
        \item There exists $\Delta'$ and $D$ such that $\Delta=\Delta',d:D$ and we have that $(\outputch{a}{v}{}{P'})[d\mapsto b]=\outputch{a}{v}{}{P'}[d\mapsto b]$. By induction $\Delta',b:D,a:B\vdash P[d\mapsto b] :: c\!:\!C$ and by application of $\runa{TS-$\multimap$L}$ we obtain $\Delta',b:D,a:A\multimap B,v:A\vdash \outputch{a}{v}{}{P[d\mapsto b]} :: c\!:\!C$ as required. 
        
        \item We have that $(\outputch{a}{v}{}{P'})[v\mapsto b]=\outputch{a}{b}{}{P'}$. From $\Delta,a:B\vdash P' :: c\!:\!C$ we then obtain $\Delta,a:A\multimap B,b:A\vdash \outputch{a}{b}{}{P'} :: c\!:\!C$ directly by application of $\runa{TS-$\multimap$L}$.
    \end{itemize}
    
    \item We have that $(\outputch{a}{v}{}{P'})[c\mapsto b]=\outputch{a}{v}{}{P'[c\mapsto b]}$. By induction we have that $\Delta,a:B\vdash P'[c\mapsto b] :: b\!:\!C$, and by application of $\runa{TS-$\multimap$L}$ we obtain $\Delta,a:A\multimap B,v:A \vdash \outputch{a}{v}{}{P'[c\mapsto b]} :: b\!:\!C$ as required.
\end{enumerate}

%We have that $\Delta,a:A\multimap B,v:A \vdash \outputch{a}{v}{}{P'} :: c\!:\!C$ and $\Delta,a:B\vdash P' :: c\!:\!C$. The first part of the lemma applies to $(\outputch{a}{v}{}{P'})[v\mapsto b]=\outputch{a}{b}{}{P'[v\mapsto b]}$, $(\outputch{a}{v}{}{P'})[a\mapsto b]=\outputch{b}{v}{}{P'[a\mapsto b]}$ and to $(\outputch{a}{v}{}{P'})[d\mapsto b]=\outputch{a}{v}{}{P'[d\mapsto b]}$ when $\Delta=\Delta',d:D$. For the first case we obtain $\Delta,a:A\multimap B, b:A\vdash \outputch{a}{b}{}{P'[v\mapsto b]} :: c\!:\!C$ directly from $\runa{TS-$\multimap$L}$, as it must be that $P'[v\mapsto b]=P'$ since $\Delta,a:B\vdash P' :: c\!:\!C$. In the second case we have by induction that $\Delta,b:B\vdash P'[a\mapsto b] :: c\!:\!C$ and from $\runa{TS-$\multimap$L}$ we obtain $\Delta,b:A\multimap B,v:A \vdash \outputch{b}{v}{}{P'[a\mapsto b]} :: c\!:\!C$. For the third case we have by induction $\Delta',b:D,a:B\vdash P'[d\mapsto b] :: c\!:\!C$ and from $\runa{TS-$\multimap$L}$ we obtain $\Delta',b:D,a:A\multimap B,v:A \vdash \outputch{a}{v}{}{P'[d\mapsto b]} :: c\!:\!C$. The second part of the lemma applies to $(\outputch{a}{v}{}{P'})[c\mapsto b]=\outputch{a}{v}{}{P'[c\mapsto b]}$. Then we have by induction that $\Delta,a:B\vdash P'[c\mapsto b] :: b\!:\!C$, and it follows from $\runa{TS-$\multimap$L}$ that also $\Delta,a:A\otimes B,v:A \vdash \outputch{a}{v}{}{P'[c\mapsto b]} :: b\!:\!C$.

\item[$\runa{TS-$\multimap$R}$] We have that $\Delta\vdash \inputch{a}{v}{}{P'} :: a\!:\!A\multimap B$ because $\Delta,v:A\vdash P' :: a\!:\!B$. We consider the cases separately
\begin{enumerate}
    \item We replace some $d\in\text{dom}(\Delta)$. There exists $\Delta'$ and $D$ such that $\Delta=\Delta',d:D$ and we have that $(\inputch{a}{v}{}{P'})[d\mapsto b]=\inputch{a}{v}{}{P'}[d\mapsto b]$. By induction $\Delta',b:D,v:A\vdash P[d\mapsto b] :: a\!:\!B$ and by application of $\runa{TS-$\multimap$R}$ we obtain $\Delta',b:D\vdash \inputch{a}{v}{}{P[d\mapsto b]} :: a\!:\!A\otimes B$ as required. 
    
    \item We have that $(\inputch{a}{v}{}{P'})[a\mapsto b]=\inputch{b}{v}{}{P'[a\mapsto b]}$. By induction we have that $\Delta,v:A\vdash P'[a\mapsto b] :: b\!:\!B$, and by application of $\runa{TS-$\multimap$R}$ we obtain $\Delta\vdash \inputch{b}{v}{}{P'[a\mapsto b]} :: b\!:\!A\otimes B$ as required.
\end{enumerate}

%We have that $\Delta\vdash \inputch{a}{v}{}{P'} :: a\!:\!A\multimap B$ and $\Delta,v:A\vdash P' :: a\!:\!B$. The first part of the lemma applies to $(\inputch{a}{v}{}{P'})[d\mapsto b]=\inputch{a}{v}{}{P'[d\mapsto b]}$ when $\Delta=\Delta',d:D$. We have by induction that $\Delta',b:D,v:A\vdash P'[d\mapsto b] :: a\!:\!B$ and from $\runa{TS-$\multimap$R}$ we obtain $\Delta',b:D\vdash \inputch{a}{v}{}{P'[d\mapsto b]} :: a\!:\!A\multimap B$. The second part of the lemma applies to $(\inputch{a}{v}{}{P'})[a\mapsto b]=\inputch{b}{v}{}{P'[a\mapsto b]}$. Then we have by induction that $\Delta,v:A\vdash P'[a\mapsto b] :: b\!:\!B$, and it follows from $\runa{TS-$\multimap$R}$ that also $\Delta \vdash \inputch{b}{v}{}{P'[a\mapsto b]} :: b\!:\!A\otimes B$.

\item[$\runa{TS-cut}$] We have that $\Delta_1,\Delta_2\vdash \newvar{a}{(P' \mid P'') :: c\!:\!C}$ because $\Delta_1\vdash P' :: a\!:\!A$ and $\Delta_2,a:A\vdash P'' :: c\!:\!C$. Then $(\newvar{a}{(P' \mid P'')})[d\mapsto b]=\newvar{a}{(P'[d\mapsto b] \mid P''[d\mapsto b])}$ and we can assume that $d\neq a$, as $\Delta_1,\Delta_2\vdash \newvar{a}{(P' \mid P'') :: c\!:\!C}$ does not hold when $a\in \text{dom}(\Delta_1,\Delta_2)$ or $a=c$. We consider the cases separately
\begin{enumerate}
    \item We replace some $d\in\text{dom}(\Delta_1)$ or $d\in\text{dom}(\Delta_2)$, such that either $\Delta_1=\Delta_1',d:D$ or $\Delta_2=\Delta_2',d:D$, and so by induction we have either $\Delta_1',b:D\vdash P'[d\mapsto b] :: a\!:\!A$ or $\Delta_2',a:A,b:D\vdash P''[d\mapsto b] :: c\!:\!C$. Thus we obtain either $\Delta_1',b:D,\Delta_2\vdash \newvar{a}{(P'[d\mapsto b] \mid P''[d\mapsto b])} :: c\!:\!C$ or $\Delta_1,\Delta_2',b:D\vdash \newvar{a}{(P'[d\mapsto b] \mid P''[d\mapsto b])} :: c\!:\!C$ by application of $\runa{TS-cut}$ as required.
    
    \item We need only consider $P''$ as $d\neq a$, and so if $d=c$ we have by induction that $\Delta_2,a:A\vdash P''[d\mapsto b] :: b\!:\!C$. Then we obtain $\Delta_1,\Delta_2\vdash \newvar{a}{(P'[c\mapsto b] \mid P''[c\mapsto b])} :: b\!:\!C$ directly by application of $\runa{TS-cut}$ as required. 
\end{enumerate}

%The first part of the lemma applies when either $\Delta_1=\Delta_1',d:D$ or $\Delta_2=\Delta_2',d:D$, and so by induction we have either $\Delta_1',b:D\vdash P'[d\mapsto b] :: a\!:\!A$ or $\Delta_2',a:A,b:D\vdash P''[d\mapsto b] :: c\!:\!C$. Thus we obtain either $\Delta_1',b:D,\Delta_2\vdash \newvar{a}{(P'[d\mapsto b] \mid P''[d\mapsto b])} :: c\!:\!C$ or $\Delta_1,\Delta_2',b:D\vdash \newvar{a}{(P'[d\mapsto b] \mid P''[d\mapsto b])} :: c\!:\!C$ by $\runa{TS-cut}$.

%The second part of the lemma can only apply to $P''$ as $d\neq a$, and so if $d=c$ we have by induction that $\Delta_2,a:A\vdash P''[d\mapsto b] :: b\!:\!C$. Then we obtain $\Delta_1,\Delta_2\vdash \newvar{a}{(P'[c\mapsto b] \mid P''[c\mapsto b])} :: b\!:\!C$ directly from $\runa{TS-cut}$. 

\item[$\runa{TS-id}$] We have that $b:A\vdash a \leftarrow b :: a\!:\!A$, $(a \leftarrow b)[a\mapsto c]= c \leftarrow b$ and $(a\leftarrow b)[b\mapsto c]=a\leftarrow c$. We obtain $c:A\vdash a\leftarrow c :: a\!:\!A$ and $b:A\vdash c \leftarrow b :: c\!:\!A$ directly by application of $\runa{TS-id}$ as required.

\item[$\runa{TS-$\oplus$L}$] We have that $\Delta,a : \oplus\{l:A_l\}_{l\in L}\vdash a.\texttt{case}\{l\Rightarrow P_l\}_{l\in L} :: c\!:\!C$ because for $l \in L$ we also have $\Delta,a:A_l \vdash P_l :: c\!:\!C$. We consider the cases separately
\begin{enumerate}
    \item We either replace $a$ or some $d\in\text{dom}(\Delta)$. We consider them separately
    \begin{itemize}
        \item We have that $(a.\texttt{case}\{l\Rightarrow P_l\}_{l\in L})[a\mapsto b]=b.\texttt{case}\{l\Rightarrow P_l[a\mapsto b]\}_{l\in L}$. Then for $l\in L$ we have by induction that $\Delta,b:A_l\vdash P_l[a\mapsto b] :: c\!:\!C$, and by application of $\runa{TS-$\oplus$L}$ we obtain $\Delta,b:\oplus\{l:A_l\}_{l\in L}\vdash b.\texttt{case}\{l\Rightarrow P_l[a\mapsto b]\}_{l\in L} :: c\!:\!C$ as required.
        
        \item There exists $\Delta'$ and $D$ such that $\Delta=\Delta',d:D$ and we have that $(a.\texttt{case}\{l\Rightarrow P_l\}_{l\in L})[d\mapsto b]=a.\texttt{case}\{l\Rightarrow P_l[d\mapsto b]\}_{l\in L}$. Then for $l\in L$ we have by induction that $\Delta',b:D,a:A_l\vdash P_l[d\mapsto b] :: c\!:\!C$, and by application of $\runa{TS-$\oplus$L}$ we obtain $\Delta',b:D,a:\oplus\{l:A_l\}_{l\in L}\vdash a.\texttt{case}\{l\Rightarrow P_l[d\mapsto b]\}_{l\in L} :: c\!:\!C$ as required.
    \end{itemize}
    
    \item We have that $(a.\texttt{case}\{l\Rightarrow P_l\}_{l\in L})[c\mapsto b]=a.\texttt{case}\{l\Rightarrow P_l[c\mapsto b]\}_{l\in L}$. Then for $l\in L$ we have by induction that $\Delta',b:D,a:A_l\vdash P_l[d\mapsto b] :: c\!:\!C$, and by application of $\runa{TS-$\oplus$L}$ we obtain $\Delta',b:D,a:\oplus\{l:A_l\}_{l\in L}\vdash a.\texttt{case}\{l\Rightarrow P_l[d\mapsto b]\}_{l\in L} :: c\!:\!C$ as required.
\end{enumerate}

% The first part of the lemma applies to $(a.\texttt{case}\{l\Rightarrow P_l\}_{l\in L})[a\mapsto b]=b.\texttt{case}\{l\Rightarrow P_l[a\mapsto b]\}_{l\in L}$ and to $(a.\texttt{case}\{l\Rightarrow P_l\}_{l\in L})[d\mapsto b]=a.\texttt{case}\{l\Rightarrow P_l[d\mapsto b]\}_{l\in L}$ when $\Delta=\Delta',d:D$. In the first case we have by induction for $l\in L$ that $\Delta,b:A_l\vdash P_l[a\mapsto b] :: c\!:\!C$, and so by $\runa{TS-$\oplus$L}$ we obtain $\Delta,b:\oplus\{l:A_l\}_{l\in L}\vdash b.\texttt{case}\{l\Rightarrow P_l[a\mapsto b]\}_{l\in L} :: c\!:\!C$.
% For the second case we have by induction for $l\in L$ that $\Delta',b:D,a:A_l\vdash P_l[d\mapsto b] :: c\!:\!C$, and so by $\runa{TS-$\oplus$L}$ we obtain $\Delta',b:D,a:\oplus\{l:A_l\}_{l\in L}\vdash a.\texttt{case}\{l\Rightarrow P_l[d\mapsto b]\}_{l\in L} :: c\!:\!C$.
% The second part of the lemma applies to $(a.\texttt{case}\{l\Rightarrow P_l\}_{l\in L})[c\mapsto b]=a.\texttt{case}\{l\Rightarrow P_l[c\mapsto b]\}_{l\in L}$. We have by induction for $l\in L$ that $\Delta,a:A_l\vdash P_l[c\mapsto b] :: b\!:\!C$, and so by $\runa{TS-$\oplus$L}$ we obtain $\Delta,a:\oplus\{l:A_l\}_{l\in L}\vdash a.\texttt{case}\{l\Rightarrow P_l[c\mapsto b]\}_{l\in L} :: b\!:\!C$.

\item[$\runa{TS-$\oplus$R}$] We have that $\Delta\vdash a.k; P' :: a\!:\!\oplus\{l:A_l\}_{l\in L}$ because $k \in L$ and $\Delta \vdash P' :: a\!:\!A_k$. We consider the cases separately
\begin{enumerate}
    \item We replace some $d\in\text{dom}(\Delta)$. There exists $\Delta'$ and $D$ such that $\Delta=\Delta',d:D$ and we have that $(a.k; P')[d\mapsto b]=a.k; P'[d\mapsto b]$. Then we have by induction that $\Delta',b:D\vdash P'[d\mapsto b] :: a\!:\!a_k$, and by application of $\runa{TS-$\oplus$R}$ we obtain $\Delta',b:D\vdash a.k; P'[d\mapsto b] :: a\!:\!\oplus\{l:A_l\}_{l\in L}$ as required.
    
    \item We have that $(a.k; P')[a\mapsto b]=b.k; P'[a\mapsto b]$. Then we have by induction that $\Delta\vdash P'[a\mapsto b] :: b\!:\!A_k$, and by application of $\runa{TS-$\oplus$L}$ we obtain $\Delta\vdash b.k; P'[a\mapsto b] :: b\!:\!\oplus\{l:A_l\}_{l\in L}$ as required.
\end{enumerate}

%We have that $\Delta\vdash a.k; P' :: a\!:\!\oplus\{l:A_l\}_{l\in L}$ such that $k \in L$ and $\Delta \vdash P' :: a\!:\!A_k$. The first part of the lemma applies to $(a.k; P')[d\mapsto b]=a.k; P'[d\mapsto b]$ when $\Delta=\Delta',d:D$. We have by induction that $\Delta',b:D\vdash P'[d\mapsto b] :: a\!:\!a_k$, and so by $\runa{TS-$\oplus$R}$ we obtain $\Delta',b:D\vdash a.k; P'[d\mapsto b] :: a\!:\!\oplus\{l:A_l\}_{l\in L}$. The second part of the lemma applies to $(a.k; P')[a\mapsto b]=b.k; P'[a\mapsto b]$. We have by induction that $\Delta\vdash P'[a\mapsto b] :: b\!:\!A_k$, and so by $\runa{TS-$\oplus$L}$ we obtain $\Delta\vdash b.k; P'[a\mapsto b] :: b\!:\!\oplus\{l:A_l\}_{l\in L}$.

\item[$\runa{TS-$\&$L}$] We have that $\Delta,a : \&\{l:A_l\}_{l\in L}\vdash a.k; P' :: c\!:\!C$ because $k \in L$ and $\Delta,a:A_k \vdash P' :: c\!:\!C$. We consider the cases separately
\begin{enumerate}
    \item We either replace $a$ or some $d\in\text{dom}(\Delta)$. We consider them separately
    \begin{itemize}
        \item We have that $(a.k; P')[a\mapsto b]=b.k; P'[a\mapsto b]$. Then we have by induction that $\Delta,b:A_k\vdash P'[a\mapsto b] :: c\!:\!C$, and by application of $\runa{TS-$\&$L}$ we obtain $\Delta,b:\&\{l:A_l\}_{l\in L}\vdash b.k; P'[a\mapsto b] :: c\!:\!C$ as required.
        
        \item There exists $\Delta'$ and $D$ such that $\Delta=\Delta',d:D$ and we have that $(a.k; P')[d\mapsto b]=a.k; P'[d\mapsto b]$. Then we have by induction that $\Delta',b:D,a:A_k\vdash P'[d\mapsto b] :: c\!:\!C$, and by application of $\runa{TS-$\&$L}$ we obtain $\Delta',b:D,a:\&\{l:A_l\}_{l\in L}\vdash a.k; P'[d\mapsto b] :: c\!:\!C$ as required.
    \end{itemize}
    
    \item We have that $(a.k; P')[c\mapsto b]=a.k; P'[c\mapsto b]$. Then we have by induction that $\Delta,a:A_k\vdash P'[c\mapsto b] :: b\!:\!C$, and by application of $\runa{TS-$\&$L}$ we obtain $\Delta,a:\&\{l:A_l\}_{l\in L}\vdash a.k; P'[c\mapsto b] :: b\!:\!C$ as required.
\end{enumerate}

%We have that $\Delta,a : \&\{l:A_l\}_{l\in L}\vdash a.k; P' :: c\!:\!C$ such that $k \in L$ and $\Delta,a:A_k \vdash P' :: c\!:\!C$. The first part of the lemma applies to $(a.k; P')[a\mapsto b]=b.k; P'[a\mapsto b]$ and to $(a.k; P')[d\mapsto b]=a.k; P'[d\mapsto b]$ when $\Delta=\Delta',d:D$. In the first case we have by induction that $\Delta,b:A_k\vdash P'[a\mapsto b] :: c\!:\!C$, and so by $\runa{TS-$\&$L}$ we obtain $\Delta,b:\&\{l:A_l\}_{l\in L}\vdash b.k; P'[a\mapsto b] :: c\!:\!C$. For the second case we have by induction that $\Delta',b:D,a:A_k\vdash P'[d\mapsto b] :: c\!:\!C$, and so by $\runa{TS-$\&$L}$ we obtain $\Delta',b:D,a:\&\{l:A_l\}_{l\in L}\vdash a.k; P'[d\mapsto b] :: c\!:\!C$. The second part of the lemma applies to $(a.k; P')[c\mapsto b]=a.k; P'[c\mapsto b]$. We have by induction that $\Delta,a:A_k\vdash P'[c\mapsto b] :: b\!:\!C$, and so by $\runa{TS-$\&$L}$ we obtain $\Delta,a:\&\{l:A_l\}_{l\in L}\vdash a.k; P'[c\mapsto b] :: b\!:\!C$.

\item[$\runa{TS-$\&$R}$] We have that $\Delta\vdash a.\texttt{case}\{l\Rightarrow P_l\}_{l\in L} :: a\!:\!\&\{l:A_l\}_{l\in L}$ because for $l\in L$ we also have that $\Delta \vdash P_l :: a\!:\!A_l$. We consider the cases separately
\begin{enumerate}
    \item We replace some $d\in\text{dom}(\Delta)$. There exists $\Delta'$ and $D$ such that $\Delta=\Delta',d:D$ and we have that $(a.\texttt{case}\{l\Rightarrow P_l\}_{l\in L})[d\mapsto b]=a.\texttt{case}\{l\Rightarrow P_l[d\mapsto b]\}_{l\in L}$. Then for $l\in L$ we have by induction that $\Delta',b:D\vdash P_l[d\mapsto b] :: a\!:\!a_l$, and by application of $\runa{TS-$\&$R}$ we obtain $\Delta',b:D\vdash a.\texttt{case}\{l\Rightarrow P_l[d\mapsto b]\}_{l\in L} :: a\!:\!\&\{l:A_l\}_{l\in L}$ as required.
    
    \item We have that $(a.\texttt{case}\{l\Rightarrow P_l\}_{l\in L})[a\mapsto b]=b.\texttt{case}\{l\Rightarrow P_l[a\mapsto b]\}_{l\in L}$. Then for $l\in L$ we have by induction that $\Delta\vdash P_l[a\mapsto b] :: b\!:\!A_l$, and by application of $\runa{TS-$\&$L}$ we obtain $\Delta\vdash b.\texttt{case}\{l\Rightarrow P_l[a\mapsto b]\}_{l\in L} :: b\!:\!\&\{l:A_l\}_{l\in L}$ as required.
\end{enumerate}

%We have that $\Delta\vdash a.\texttt{case}\{l\Rightarrow P_l\}_{l\in L} :: a\!:\!\&\{l:A_l\}_{l\in L}$ such that for $l\in L$ we also have that $\Delta \vdash P_l :: a\!:\!A_l$. The first part of the lemma applies to $(a.\texttt{case}\{l\Rightarrow P_l\}_{l\in L})[d\mapsto b]=a.\texttt{case}\{l\Rightarrow P_l[d\mapsto b]\}_{l\in L}$ when $\Delta=\Delta',d:D$. We have by induction for $l\in L$ that $\Delta',b:D\vdash P_l[d\mapsto b] :: a\!:\!a_l$, and so by $\runa{TS-$\&$R}$ we obtain $\Delta',b:D\vdash a.\texttt{case}\{l\Rightarrow P_l[d\mapsto b]\}_{l\in L} :: a\!:\!\&\{l:A_l\}_{l\in L}$. The second part of the lemma applies to $(a.\texttt{case}\{l\Rightarrow P_l\}_{l\in L})[a\mapsto b]=b.\texttt{case}\{l\Rightarrow P_l[a\mapsto b]\}_{l\in L}$. We have by induction for $l\in L$ that $\Delta\vdash P_l[a\mapsto b] :: b\!:\!A_l$, and so by $\runa{TS-$\&$L}$ we obtain $\Delta\vdash b.\texttt{case}\{l\Rightarrow P_l[a\mapsto b]\}_{l\in L} :: b\!:\!\&\{l:A_l\}_{l\in L}$.

\item[$\runa{TS-def}$] We have that $\Delta,\widetilde{b}:\widetilde{B}\vdash \newvar{a}{(a \leftarrow f \leftarrow \widetilde{b} \mid P'')} :: c\!:\!C$ because $(\widetilde{v}:\widetilde{B}\vdash f = Q' :: g\!:\!A)\in\Sigma$ and $\Delta,a:A\vdash P'' :: c\!:\!C$. Then $(\newvar{a}{(a\leftarrow f \leftarrow\widetilde{b} \mid P'')})[d\mapsto h]=\newvar{a}{((a\leftarrow f \leftarrow\widetilde{b})[d\mapsto h] \mid P''[d\mapsto h])}$ and we can assume that $d\neq a$, as $\Delta,\widetilde{b}:\widetilde{B}\vdash \newvar{a}{(a\leftarrow f \leftarrow\widetilde{b} \mid P'') :: c\!:\!C}$ does not hold when $a\in \text{dom}(\Delta,\widetilde{b}:\widetilde{B})$ or $a=c$. We consider the cases separately
\begin{enumerate}
    \item We either have that $d\in\text{dom}(\Delta)$ or $d\in\text{dom}(\widetilde{b}:\widetilde{B})$. We consider them separately
    \begin{itemize}
        \item There exists $\Delta'$ and $D$ such that $\Delta=\Delta',d:D$ and by induction we have that $\Delta',h:D,a:A\vdash P''[d\mapsto h] :: c\!:C$. By application of $\runa{TS-def}$ we obtain $\Delta',h:D,\widetilde{b}:\widetilde{B}\vdash \newvar{a}{((a\leftarrow f \leftarrow\widetilde{b})[d\mapsto h] \mid P''[d\mapsto h])} :: c\!:\!C$ as required.
        
        \item There exists $\widetilde{b'}$, $\widetilde{B'}$ and $D$ such that $\widetilde{b}:\widetilde{B}=\widetilde{b'}:\widetilde{B'},d:D$ and by induction we have that $\Delta',h:D,a:A\vdash P''[d\mapsto h] :: c\!:C$. By application of $\runa{TS-def}$ we obtain $\Delta',h:D,\widetilde{b}:\widetilde{B}\vdash \newvar{a}{((a\leftarrow f \leftarrow\widetilde{b})[d\mapsto h] \mid P''[d\mapsto h])} :: c\!:\!C$ as required.
    \end{itemize}
    
    \item We need only consider $P''$ as $d\neq a$, and so $d=c$. By induction we have that $\Delta,a:A\vdash P''[c\mapsto b] :: b\!:\!C$, and by application of $\runa{TS-def}$ we obtain $\Delta',h:D,\widetilde{b}:\widetilde{B}\vdash \newvar{a}{((a\leftarrow f \leftarrow\widetilde{b})[d\mapsto h] \mid P''[d\mapsto h])} :: c\!:\!C$ as required.
\end{enumerate}

% The first part of the lemma applies when either
% \begin{enumerate}
%     \item $\widetilde{b}:\widetilde{B}=\widetilde{b'}:\widetilde{B'},d:D$ such that $(a\leftarrow f \leftarrow\widetilde{b})[d\mapsto h]=a\leftarrow f \leftarrow \widetilde{b'}:\widetilde{B'},h:D$, and so we obtain $\Delta,\widetilde{b'}:\widetilde{B'},h:D\vdash \newvar{a}{(a\leftarrow f \leftarrow \widetilde{b'}:\widetilde{B'},h:D \mid P''[d\mapsto h]) :: c\!:\!C}$ from $\runa{TS-def}$.
    
%     \item $\Delta=\Delta',d:D$, and so by induction we obtain $\Delta',h:D,a:A\vdash P''[d\mapsto h] :: c\!:C$. It follows from $\runa{TS-def}$ that $\Delta',h:D,\widetilde{b}:\widetilde{B}\vdash \newvar{a}{((a\leftarrow f \leftarrow\widetilde{b})[d\mapsto h] \mid P''[d\mapsto h])} :: c\!:\!C$.
% \end{enumerate}
% The second part of the lemma can only apply to $P''$ as $d\neq a$, and so if $d=c$ we have by induction that $\Delta,a:A\vdash P''[d\mapsto b] :: b\!:\!C$. Then we obtain $\Delta,\widetilde{b}:\widetilde{B}\vdash \newvar{a}{((a\leftarrow f \leftarrow\widetilde{b})[d\mapsto h] \mid P''[d\mapsto h])} :: b\!:\!C$ directly from $\runa{TS-def}$. 

\item[$\runa{TS-$\ocircle$LR'}$] We have that $\Delta\vdash \tick P' :: a\!:\!A$ because $[\Delta]^{-1}_L\vdash P' :: a\!:\![A]^{-1}_R$. We consider the cases separately
\begin{enumerate}
    \item We replace some $d\in\text{dom}(\Delta)$, and so there exists $\Delta'$ and $D$ such that $\Delta=\Delta',d:D$. Then we have that $(\tick P')[d\mapsto b]=\tick P'[d\mapsto b]$, and by induction we have that $[\Delta']^{-1}_L,b:[D]^{-1}_L\vdash P'[d\mapsto b] :: a\!:\![A]^{-1}_R$. By application of $\runa{TS-$\ocircle$LR'}$ we obtain $\Delta',b:D\vdash \tick P'[d\mapsto b] :: a\!:\!A$ as required.
    
    \item We have that $(\tick P')[a\mapsto b]=\tick P'[a\mapsto b]$, and so by induction we have that $[\Delta]^{-1}_L\vdash P'[a\mapsto b] :: b\!:\![A]^{-1}_R$. By application of $\runa{TS-$\ocircle$LR'}$ we obtain $\Delta\vdash \tick P'[a\mapsto b] :: b\!:\!A$ as required.
\end{enumerate}

%The first part of the lemma applies to $(\tick P')[d\mapsto b]=\tick P'[d\mapsto b]$ when $\Delta=\Delta',d:D$. We have by induction that $[\Delta']^{-1}_L,b:[D]^{-1}_L\vdash P'[d\mapsto b] :: a\!:\![A]^{-1}_R$, and so by $\runa{TS-$\ocircle$LR'}$ we obtain $\Delta',b:D\vdash \tick P'[d\mapsto b] :: a\!:\!A$. The second part of the lemma applies to $(\tick P')[a\mapsto b]=\tick P'[a\mapsto b]$. We have by induction that $[\Delta]^{-1}_L\vdash P'[a\mapsto b] :: b\!:\![A]^{-1}_R$, and so by $\runa{TS-$\ocircle$LR'}$ we obtain $\Delta\vdash \tick P'[a\mapsto b] :: b\!:\!A$.

\item[$\runa{TS-$\ocircle$LR}$] We have that $\Delta\vdash P :: a\!:\!A$ because $[\Delta]^{-1}_L\vdash P :: a\!:\![A]^{-1}_R$. We consider the cases separately
\begin{enumerate}
    \item We replace some $d\in\text{dom}(\Delta)$, and so there exists $\Delta'$ and $D$ such that $\Delta=\Delta',d:D$. Then by induction we have that $[\Delta']^{-1}_L,b:[D]^{-1}_L\vdash P[d\mapsto b] :: a\!:\![A]^{-1}_R$. By application of $\runa{TS-$\ocircle$LR}$ we obtain $\Delta',b:D\vdash P[d\mapsto b] :: a\!:\!A$ as required.
    
    \item We replace $a$, and so by induction we have that $[\Delta]^{-1}_L\vdash P[a\mapsto b] :: b\!:\![A]^{-1}_R$. By application of $\runa{TS-$\ocircle$LR}$ we obtain $\Delta\vdash P[a\mapsto b] :: b\!:\!A$ as required.
\end{enumerate}

%The first part of the lemma applies to $P[d\mapsto b]$ when $\Delta=\Delta',d:D$. We have by induction that $[\Delta']^{-1}_L,b:[D]^{-1}_L\vdash P[d\mapsto b] :: a\!:\![A]^{-1}_R$, and so by $\runa{TS-$\ocircle$LR}$ we obtain $\Delta',b:D\vdash P[d\mapsto b] :: a\!:\!A$. The second part of the lemma applies to $P[a\mapsto b]$. We have by induction that $[\Delta]^{-1}_L\vdash P[a\mapsto b] :: b\!:\![A]^{-1}_R$, and so by $\runa{TS-$\ocircle$LR}$ we obtain $\Delta\vdash P[a\mapsto b] :: b\!:\!A$.

\item[$\runa{TS-$\lozenge$L}$] We have that $\Delta,a:\lozenge A\vdash P :: c\!:\!C$ because $\Delta\;\texttt{delayed}^\Box$, $C\;\texttt{delayed}^\lozenge$ and $\Delta,a:A\vdash P :: c\!:\!C$. We consider the cases separately
\begin{enumerate}
    \item We either replace $a$ or some $d\in\text{dom}(\Delta)$. We consider them separately
    \begin{itemize}
        \item By induction we have that $\Delta,b:A\vdash P[a\mapsto b] :: c\!:\!C$, and by application of $\runa{TS-$\lozenge$L}$ we obtain $\Delta,b:\lozenge A\vdash P[a\mapsto b] :: c\!:\!C$ as required.
        
        \item There exists $\Delta'$ and $D$ such that $\Delta=\Delta',d:D$, and so by induction we have that $\Delta',b:D,a:A\vdash P[d\mapsto b] :: c\!:\!C$. As the types are unchanged it follows that also $\Delta',b:D\;\texttt{delayed}^\Box$. Then by application of $\runa{TS-$\lozenge$L}$ we obtain $\Delta',b:D,a:\lozenge A\vdash P[d\mapsto b] :: c\!:\!C$ as required.
    \end{itemize}
    
    \item We replace $c$, and so by induction we have that $\Delta, a:A\vdash P[c\mapsto b] :: b\!:\!C$. By application of $\runa{TS-$\lozenge$L}$ we obtain $\Delta,a:\lozenge A\vdash P[c\mapsto b] :: b\!:\!C$ as required.
\end{enumerate}

%The first part of the lemma applies to $P[d\mapsto b]$ when either $d=a$ or $\Delta = \Delta',d:D$. In the first case we have by induction that $\Delta,b:A\vdash P[a\mapsto b] :: c\!:\!C$, and so from $\runa{TS-$\lozenge$L}$ we obtain $\Delta,b:\lozenge A\vdash P[a\mapsto b] :: c\!:\!C$. For the second case we have by induction that $\Delta',b:D,a:A\vdash P[d\mapsto b] :: c\!:\!C$, and as the types are unchanged it follows that also $\Delta',b:D\;\texttt{delayed}^\Box$. Then from  and from $\runa{TS-$\lozenge$L}$ we obtain $\Delta',b:D,a:\lozenge A\vdash P[d\mapsto b] :: c\!:\!C$. The second part of the lemma applies to $P[d\mapsto b]$ when $d=c$. Then we have by induction that $\Delta, a:A\vdash P[c\mapsto b] :: b\!:\!C$, and it follows from $\runa{TS-$\lozenge$L}$ that also $\Delta,a:\lozenge A\vdash P[c\mapsto b] :: b\!:\!C$.

\item[$\runa{TS-$\lozenge$R}$] We have that $\Delta\vdash P :: c\!:\!\lozenge C$ because $\Delta\vdash P :: c\!:\!C$. We consider the cases separately
\begin{enumerate}
    \item We replace some $d\in\text{dom}(\Delta)$, and so there exists $\Delta'$ and $D$ such that $\Delta=\Delta',d:D$. By induction we have that $\Delta',b:D\vdash P[d\mapsto b] :: c\!:\!C$, and by application of $\runa{TS-$\lozenge$R}$ we obtain $\Delta',b:D\vdash P[d\mapsto b] :: c\!:\!C$ as required.
    
    \item We replace $c$, and so by induction we have that $\Delta\vdash P[c\mapsto b] :: b\!:\!C$. By application of $\runa{TS-$\lozenge$R}$ we obtain $\Delta\vdash P[c\mapsto b] :: b\!:\!\lozenge C$ as required.
\end{enumerate}

%The first part of the lemma applies to $P[d\mapsto b]$ when $\Delta = \Delta',d:D$. We have by induction that $\Delta',b:D\vdash P[d\mapsto b] :: c\!:\!C$, and from $\runa{TS-$\lozenge$R}$ we obtain $\Delta',b:D\vdash P[d\mapsto b] :: c\!:\!C$. The second part of the lemma applies to $P[d\mapsto b]$ when $d=c$. Then we have by induction that $\Delta\vdash P[c\mapsto b] :: b\!:\!C$, and it follows from $\runa{TS-$\lozenge$R}$ that also $\Delta\vdash P[c\mapsto b] :: b\!:\!\lozenge C$.

\item[$\runa{TS-$\Box$L}$] We have that $\Delta,a:\Box A\vdash P :: c\!:\!C$ because $\Delta,a:A\vdash P :: c\!:\!C$. We consider the cases separately
\begin{enumerate}
    \item We either replace $a$ or some $d\in\text{dom}(\Delta)$. We consider them separately
    \begin{itemize}
        \item By induction we have that $\Delta,b:A\vdash P[a\mapsto b] :: c\!:\!C$ and by application of $\runa{TS-$\Box$L}$ we obtain $\Delta,b:\Box A\vdash P[a\mapsto b] :: c\!:\!C$ as required.
        
        \item There exists $\Delta'$ and $D$ such that $\Delta=\Delta',d:D$, and so by induction we have that $\Delta',b:D,a:A\vdash P[d\mapsto b] :: c\!:\!C$. By application of $\runa{TS-$\Box$L}$ we obtain $\Delta',b:D,a:\Box A\vdash P[d\mapsto b] :: c\!:\!C$ as required.
    \end{itemize}
    
    \item We replace $c$, and so by induction we have that $\Delta, a:A\vdash P[c\mapsto b] :: b\!:\!C$, and by application of $\runa{TS-$\Box$L}$ we obtain $\Delta,a:\Box A\vdash P[c\mapsto b] :: b\!:\!C$ as required.
\end{enumerate}

%The first part of the lemma applies to $P[d\mapsto b]$ when either $d=a$ or $\Delta = \Delta',d:D$. In the first case we have by induction that $\Delta,b:A\vdash P[a\mapsto b] :: c\!:\!C$ and so we obtain from $\runa{TS-$\Box$L}$ that also $\Delta,b:\Box A\vdash P[a\mapsto b] :: c\!:\!C$. For the second case we have by induction that $\Delta',b:D,a:A\vdash P[d\mapsto b] :: c\!:\!C$, and from $\runa{TS-$\Box$L}$ we obtain $\Delta',b:D,a:\Box A\vdash P[d\mapsto b] :: c\!:\!C$. The second part of the lemma applies to $P[d\mapsto b]$ when $d=c$. Then we have by induction that $\Delta, a:A\vdash P[c\mapsto b] :: b\!:\!C$, and it follows from $\runa{TS-$\Box$L}$ that also $\Delta,a:\Box A\vdash P[c\mapsto b] :: b\!:\!C$.

\item[$\runa{TS-$\Box$R}$] We have that $\Delta\vdash P :: c\!:\!\Box C$ because $\Delta\;\texttt{delayed}^\Box$ and $\Delta\vdash P :: c\!:\!C$. We consider the cases separately
\begin{enumerate}
    \item We replace some $d\in\text{dom}(\Delta)$, and so there exists $\Delta'$ and $D$ such that $\Delta=\Delta',d:D$. By induction we have that $\Delta',b:D\vdash P[d\mapsto b] :: c\!:\!C$. Then, as the types are unchanged, it follows that also $\Delta',b:D\;\texttt{delayed}^\Box$. By application of $\runa{TS-$\Box$R}$ we then obtain $\Delta',b:D\vdash P[d\mapsto b] :: c\!:\!\Box C$ as required.
    
    \item We replace $c$, and so by induction we have that $\Delta\vdash P[c\mapsto b] :: b\!:\!C$, and by application of $\runa{TS-$\Box$R}$ we obtain $\Delta\vdash P[c\mapsto b] :: b\!:\!\Box C$ as required.
\end{enumerate}

%The first part of the lemma applies to $P[d\mapsto b]$ when $\Delta = \Delta',d:D$. We have by induction that $\Delta',b:D\vdash P[d\mapsto b] :: c\!:\!C$, and as the types are unchanged it follows that also $\Delta',b:D\;\texttt{delayed}^\Box$. Then from $\runa{TS-$\Box$R}$ we obtain $\Delta',b:D\vdash P[d\mapsto b] :: c\!:\!\Box C$. The second part of the lemma applies to $P[d\mapsto b]$ when $d=c$. Then we have by induction that $\Delta\vdash P[c\mapsto b] :: b\!:\!C$, and it follows from $\runa{TS-$\Box$R}$ that also $\Delta\vdash P[c\mapsto b] :: b\!:\!\Box C$.

\end{description}
\end{proof}
\end{lemma}


% \begin{lemma}\label{lemma:contextredex}
% If $P$ is a redex and $\Delta\vdash C[P] :: a\!:\!A$ such that $P\longrightarrow P'$ then $\Delta\vdash C[P'] :: a\!:\!A$.
% \begin{proof}
% By induction on the reduction rules 
% \end{proof}
% \end{lemma}




\begin{theorem}[Subject reduction]
If $\Delta \vdash P :: a\!:\!A$ and $P \longrightarrow Q$ then $\Delta\vdash Q :: a\!:\!A$.
\begin{proof}
By induction on the reduction rules. For a process to be well-typed and reduce, it must be typed with either $\runa{TS-$\ocircle$LR'}$, $\runa{TS-cut}$ or $\runa{TS-def}$, and so it suffices to consider $\runa{R-tick}$, $\runa{R-res}$, $\runa{R-id}$ and $\runa{R-struct}$. We omit $\runa{R-struct}$ as typability is closed under structural congruence. We consider the cases separately
\begin{description}
\item[$\runa{R-tick}$] We have that $P=\texttt{tick}.P'$ and $Q = P'$. Then by $\runa{TS-$\ocircle$LR'}$, we have that $[\Delta]^{-1}_L \vdash P' :: [a:A]^{-1}_R$ such that $\Delta \vdash \texttt{tick}.P' :: a\!:\!A$. It follows from type rule $\runa{TS-$\ocircle$LR}$ that also $\Delta \vdash P' :: a\!:\!A$.

\item[$\runa{R-res}$] We have that $P\equiv\newvar{a}{(P'' \mid P'')}$. Then for $P$ to be well-typed, we must use either $\runa{TS-cut}$ or $\runa{TS-def}$. We consider the cases separately
\begin{description}
    \item[$\runa{TS-cut}$] We have that $\Delta\vdash \newvar{a}{(P' \mid P'') :: c\!:\!C}$ with $\Delta=\Delta_1,\Delta_2$ such that $\Delta_1\vdash P' :: a\!:\!A$ and $\Delta_2,a:A\vdash P'' :: c\!:\!C$. Then either $P' \longrightarrow Q'$ or $P''\longrightarrow Q''$ by $\runa{R-par}$ or $P' \mid P'' \longrightarrow Q'\mid Q''$ such that $P'\neq Q'$ and $P''\neq Q''$. In the two first cases we obtain $\Delta_1\vdash Q' :: a\!:\!A$ by induction from $\Delta_1\vdash P' :: a\!:\!A$ and $\Delta_2,a:A\vdash Q'' :: c\!:\!C$ by induction from $\Delta_2,a:A\vdash P'' :: c\!:\!C$, respectively, from which we obtain $\Delta\vdash Q :: c\!:\!C$ by $\runa{TS-cut}$. In the third case as $P' \mid P''$ reduces, $P'$ and $P''$ must synchronize. We then have the canonical form $P\equiv\newvar{\widetilde{d}}{(R_1 \mid R_2 \mid \cdots \mid R_n)}$ such that $R_1$ and $R_2$ correspond to the prefixes that synchronize in $P'$ and $P''$, and so we have $R_1 \mid R_2 \longrightarrow R_1' \mid R_2'$. Then for $P$ to be well-typed, each parallel composition must be wrapped in a restriction and be typed with either $\runa{TS-cut}$ or $\runa{TS-def}$, such that $\widetilde{d}$ contains $n-1$ names. By premise of these rules, there must be some partitions $\Delta_1=\Delta_1',\Delta_1''$ and $\Delta_2=\Delta_2',\Delta_2''$ such that $\Delta_1'\vdash R_1 :: a\!:\!A$ and $\Delta_2',a:A\vdash R_2 :: b\!:\!B$. For $Q$ to also be well-typed under the same typing as $P$, it then suffices to show that there exists some new partition $\Delta_1',\Delta_2'=\Delta_3,\Delta_4$ and type $A'$ such that $\Delta_3\vdash R_1' :: a\!:\!A'$ and $\Delta_4,a:A'\vdash R_2' :: b\!:\!B$. We consider the cases of the reduction 
    \begin{description}
    \item[$\runa{R-comm}$]
    Assume we reduce by $\runa{R-comm}$ then $R_1 \mid R_2 \equiv \inputch{a}{v}{}{R_1'} \mid \outputch{a}{w}{}{R_2'}$ for some name $w$ and processes $R_1'$ and $R_2'$, such that $\inputch{a}{v}{}{R_1'} \mid \outputch{a}{b}{}{R_2'} \longrightarrow R_1'[v\mapsto b] \mid R_2'$. Then $R_1$ and $R_2$ have two possible typings
    \begin{enumerate}
    \item $A=A'\multimap A''$ and $\Delta_2'=\Delta_3',w:A'$ such that $\Delta_1' \vdash \inputch{a}{v}{}{R_1'} :: a\!:\!A' \multimap A''$ and $\Delta_3',a : A'\multimap A'', w : A' \vdash \outputch{a}{w}{}{R_2'} :: b\!:\!B$ by $\runa{TS-$\multimap$R}$ and $\runa{TS-$\multimap$L}$. By the premises to these rules we have that $\Delta_1',v : A' \vdash R_1' :: a\!:\!A''$ and $\Delta_3',a:A''\vdash R_2' :: b\!:\!B$. This implies $\Delta_1',w : A'\vdash R'[v\mapsto w] :: a\!:\!A''$ by Lemma \ref{lemma:substlem}, and $\Delta_1',\Delta_2'=\Delta_1',w:A',\Delta_3'$. %so by $\runa{TS-cut}$ it follows that $(\Delta',b : A'),\Delta_3\vdash \newvar{a}{(R'[v\mapsto b] \mid R'') :: c\!:\!C}$ and $\Delta = (\Delta',b : A'),\Delta_3$.
    
    %
    
    \item $A=A'\otimes A''$ and $\Delta_1'=\Delta_3',w:A$ such that $\Delta_3,w:A' \vdash \outputch{a}{w}{}{R_1'} :: a\!:\!A'\otimes A''$ and $\Delta_2',a : A'\otimes A''\vdash \inputch{a}{v}{}{R_2'} :: b\!:\!B$ by $\runa{TS-$\otimes$R}$ and $\runa{TS-$\otimes$L}$. By the premises to these rules we have that $\Delta_3'\vdash R_1' :: a\!:\!A''$ and $\Delta_2',a:A'',v:A'\vdash R_2' :: b\!:\!B$. This implies $\Delta_2',a:A'',w:A'\vdash R_2'[v\mapsto w] :: b\!:\!B$ by Lemma \ref{lemma:substlem}, and $\Delta_1',\Delta_2'=\Delta_3',\Delta_2',w:A'$. %so by $\runa{TS-cut}$ it follows that $\Delta_3,(\Delta'',b : A')\vdash \newvar{a}{(R'' \mid R'[v\mapsto b])} :: c\!:\!C$ and $\Delta = \Delta_3,(\Delta'',b : A')$.
\end{enumerate}
    
    %
    
    
    \item[$\runa{R-choice}$] Assume we reduce by $\runa{R-choice}$ then $R_1 \mid R_2 \equiv a.\texttt{case}\{ l \Rightarrow P_l \}_{l\in L} \mid a.k; R$ for some label $k$ and set of labels $L$, such that $k\in L$ and $a.\texttt{case}\{ l \Rightarrow P_l \}_{l\in L} \mid a.k; R \longrightarrow P_k \mid R$. Then $R_1$ and $R_2$ have two possible typings
\begin{enumerate}
    \item $A=\&\{l : A_l\}_{l\in L}$ and $\Delta_1'\vdash a.\texttt{case}\{l \Rightarrow P_l\}_{l\in L} :: a\!:\!\&\{l : A_l\}_{l\in L}$ and $\Delta_2', a : \&\{l : A_l\}_{l\in L}\vdash a.k; R :: b\!:\!B$ by $\runa{TS-$\&$R}$ and $\runa{TS-$\&$L}$. By the premises of these rules we have that $\Delta_1' \vdash P_k :: a\!:\!A_k$ and $\Delta_2',a : A_k\vdash R :: b\!:\!B$, such that $R_1' = P_k$ and $R_2'=R$ and so we obtain $\Delta_1',\Delta_2'=\Delta_1',\Delta_2'$ directly.
        
    %
    
    \item $A=\oplus\{l : A_l\}_{l\in L}$ and $\Delta_1'\vdash a.k; R :: a\!:\!\oplus\{l : A_l\}_{l\in L}$ and $\Delta_2',a : \oplus\{l : A_l\}_{l\in L}\vdash a.\texttt{case}\{l\Rightarrow P_l\}_{l\in L} :: b\!:\!B$ by $\runa{TS-$\oplus$R}$ and $\runa{TS-$\oplus$L}$. By the premises of these rules we have that $\Delta_1'\vdash R :: a\!:\!A_k$ and $\Delta_2',a : A_k\vdash P_k :: b\!:\!B$, such that $R_1'=R$ and $R_2'=P_k$ and so we obtain $\Delta_1',\Delta_2'=\Delta_1',\Delta_2'$ directly.
    
\end{enumerate}
    
    \end{description}
    
    
    %$R_1 \mid R_2$ must be a redex such that $R_1 \mid R_2 \longrightarrow R_1' \mid R_2'$, and so we obtain $\Delta\vdash\newvar{\widetilde{b}}{([R_1' \mid R_2'] \mid R_\text{rem})} :: c\!:\!C$ from $\Delta\vdash\newvar{\widetilde{b}}{([R_1 \mid R_2] \mid R_\text{rem})} :: c\!:\!C$ by Lemma \ref{lemma:contextredex}.
    
    \item[$\runa{TS-def}$] We have that $\Delta\vdash \newvar{a}{(a \leftarrow f \leftarrow \widetilde{b} \mid P'')} :: c\!:\!C$ with $\Delta=\Delta',\widetilde{b} : \widetilde{B}$ such that $(\widetilde{d} :\widetilde{B} \vdash f = Q' :: g\!:\!A) \in \Sigma$ and $\Delta',a:A\vdash P'' :: c\!:\!C$. Then either $a \leftarrow f \leftarrow \widetilde{b} \longrightarrow Q'[g\mapsto a, \widetilde{d}\mapsto\widetilde{b}]$ by $\runa{R-par}$ and $\runa{R-def}$ or $P''\longrightarrow Q''$ by $\runa{R-par}$. In the first case we obtain $\widetilde{b}:\widetilde{B}\vdash Q'[g\mapsto a, \widetilde{d}\mapsto\widetilde{b}] :: a\!:\!A$ from $\widetilde{d} :\widetilde{B} \vdash Q' :: g\!:\!A$ by Lemma \ref{lemma:substlem}. It follows from $\runa{TS-cut}$ that $\Delta\vdash \newvar{a}{(Q'[g\mapsto a, \widetilde{d}\mapsto\widetilde{b}] \mid P'')} :: c\!:\!C$. In the second case we obtain $\Delta',a:A\vdash Q'' :: c\!:\!C$ by induction from $\Delta',a:A\vdash P'' :: c\!:\!C$. It follows from $\runa{TS-def}$ that $\Delta\vdash \newvar{a}{(a \leftarrow f \leftarrow \widetilde{b} \mid Q'')} :: c\!:\!C$.
\end{description}

\item[$\runa{R-id}$] We have that $P \equiv \newvar{a}{\newvar{b}{(P' \mid a \leftarrow b)}}$ such that $Q \equiv \newvar{h}{(P'[a\mapsto h,b\mapsto h])}$ for some name $h$ not free in $P'$. Then, as restrictions are only typable by $\runa{TS-cut}$ and $\runa{TS-def}$, $P'$ must be of the form $R' \mid R''$ such that $P \equiv \newvar{a}{(R' \mid \newvar{b}{(R'' \mid a \leftarrow b)})}$ or $P \equiv \newvar{b}{(R' \mid \newvar{a}{(R'' \mid a \leftarrow b)})}$. We consider the cases separately
\begin{enumerate}
    \item $\Delta'',a:A \vdash R' :: c\!:\!C$ such that $\Delta'\vdash \newvar{b}{(R'' \mid a \leftarrow b)} :: a\!:\!A$ and $\Delta',\Delta''\vdash P :: c\!:\!C$ using $\runa{TS-cut}$. Then we can type $\newvar{b}{(R'' \mid a \leftarrow b)}$ with either $\runa{TS-cut}$ or $\runa{TS-def}$
    \begin{enumerate}
        \item $\Delta' \vdash R'' :: b\!:\!A$ such that $b:A\vdash a \leftarrow b :: a\!:\!A$ and $\Delta' \vdash \newvar{b}{(R'' \mid a \leftarrow b)} :: a\!:\!A$. Then it follows by Lemma \ref{lemma:substlem} that $\Delta''\vdash R''[a\mapsto h,b\mapsto h] :: h\!:\!A$ and $\Delta'',h:A \vdash R' :: c\!:\!C$ such that $\Delta',\Delta''\vdash\newvar{h}{(R'[a\mapsto h,b\mapsto h] \mid R''[a\mapsto h,b\mapsto h]) :: c\!:\!C}$.
        
        \item $R'' = b \leftarrow f \leftarrow \widetilde{d}$ and $(\widetilde{e} : \widetilde{B}\vdash f = R :: g\!:\!A) \in \Sigma$ such that $\Delta' = \widetilde{d}:\widetilde{B}$, $b:A\vdash a \leftarrow b :: a\!:\!A$ and $\Delta' \vdash \newvar{b}{(R'' \mid a \leftarrow b)} :: a\!:\!A$. Then it follows by Lemma \ref{lemma:substlem} that $\Delta'',h:A \vdash R' :: c\!:\!C$ such that $\Delta',\Delta''\vdash\newvar{h}{(R'[a\mapsto h,b\mapsto h] \mid h \leftarrow f \leftarrow \widetilde{d}) :: c\!:\!C}$.
    \end{enumerate}
    
    %
    
    \item Either $\Delta' \vdash R' :: b\!:\!A$ or $b \leftarrow f \leftarrow \widetilde{d}$, $\Delta' = \widetilde{d}:\widetilde{B}$ and $(\widetilde{e} : \widetilde{B}\vdash f = R :: g\!:\!A) \in \Sigma$ such that $\Delta'',b:A\vdash \newvar{a}{(R'' \mid a \leftarrow b)} :: c\!:\!C$ and $\Delta',\Delta''\vdash P :: c\!:\!C$ using $\runa{TS-cut}$ or $\runa{TS-def}$, respectively. In either case we must use $\runa{TS-cut}$ to get $\Delta'',b:A\vdash \newvar{a}{(R'' \mid a \leftarrow b)} :: c\!:\!C$, as we have that $b:A\vdash a\leftarrow b :: a\!:\!A$ and $\Delta'',a:A\vdash R'' :: c\!:\!C$. Then we reach $\Delta',\Delta''\vdash\newvar{h}{(R'[a\mapsto h,b\mapsto h] \mid R''[a\mapsto h,b\mapsto h])} :: c\!:\!C$ by either $\runa{TS-cut}$ or $\runa{TS-def}$. In either case we have that $\Delta'',h:A\vdash R''[a\mapsto h,b\mapsto h] :: c\!:\!C$. In the first case we have that $\Delta' \vdash R'[a\mapsto h,b\mapsto h] :: h\!:\!A$ and the latter case trivially follows by $R'[a\mapsto h,b\mapsto h] = h \leftarrow f \leftarrow \widetilde{d}$, concluding the proof.
\end{enumerate}
\end{description}

% OLD BEGIN !

% by induction on the extended reduction rules. The proof uses the fact that a well-typed process cannot \textit{consume} the session it provides on reduction, by type rules $\runa{TS-cut}$ and $\runa{TS-def}$. The proof is slightly tedious, as the type rules are not syntax directed.
% \begin{description}
% \item[$\runa{R-tick}$] Assume that $P$ reduces by $\runa{R-tick}$, such that $P$ is of the form $\texttt{tick}.P'$ and $Q = P'$. Then by $\runa{TS-$\ocircle$LR'}$, we have that $[\Delta]^{-1}_L \vdash P' :: [a:A]^{-1}_R$ such that $\Delta \vdash \texttt{tick}.P' :: a\!:\!A$. It follows from type rule $\runa{TS-$\ocircle$LR}$ that also $\Delta \vdash P' :: a\!:\!A$.

% %

% \item[$\runa{R-id}$] Assume that $P$ reduces by $\runa{R-id}$ then we have that $P \equiv \newvar{a}{\newvar{b}{(P' \mid a \leftarrow b)}}$ such that $Q \equiv \newvar{h}{(P'[a\mapsto h,b\mapsto h])}$ for some name $h \notin fv(P')$. Then, as restrictions are only typable by $\runa{TS-cut}$ and $\runa{TS-def}$, $P'$ must be of the form $R' \mid R''$ such that $P \equiv \newvar{a}{(R' \mid \newvar{b}{(R'' \mid a \leftarrow b)})}$ or $P \equiv \newvar{b}{(R' \mid \newvar{a}{(R'' \mid a \leftarrow b)})}$. We consider the cases separately
% \begin{enumerate}
%     \item $\Delta'',a:A \vdash R' :: c\!:\!C$ such that $\Delta'\vdash \newvar{b}{(R'' \mid a \leftarrow b)} :: a\!:\!A$ and $\Delta',\Delta''\vdash P :: c\!:\!C$ using $\runa{TS-cut}$. Then we can type $\newvar{b}{(R'' \mid a \leftarrow b)}$ with either $\runa{TS-cut}$ or $\runa{TS-def}$
%     \begin{enumerate}
%         \item $\Delta' \vdash R'' :: b\!:\!A$ such that $b:A\vdash a \leftarrow b :: a\!:\!A$ and $\Delta' \vdash \newvar{b}{(R'' \mid a \leftarrow b)} :: a\!:\!A$. Then it follows by Lemma \ref{lemma:substlem} that $\Delta''\vdash R''[a\mapsto h,b\mapsto h] :: h\!:\!A$ and $\Delta'',h:A \vdash R' :: c\!:\!C$ such that $\Delta',\Delta''\vdash\newvar{h}{(R'[a\mapsto h,b\mapsto h] \mid R''[a\mapsto h,b\mapsto h]) :: c\!:\!C}$.
        
%         \item $R'' = b \leftarrow f \leftarrow \widetilde{d}$ and $(\widetilde{e} : \widetilde{B}\vdash f = R :: g\!:\!A) \in \Sigma$ such that $\Delta' = \widetilde{d}:\widetilde{B}$, $b:A\vdash a \leftarrow b :: a\!:\!A$ and $\Delta' \vdash \newvar{b}{(R'' \mid a \leftarrow b)} :: a\!:\!A$. Then it follows by Lemma \ref{lemma:substlem} that $\Delta'',h:A \vdash R' :: c\!:\!C$ such that $\Delta',\Delta''\vdash\newvar{h}{(R'[a\mapsto h,b\mapsto h] \mid h \leftarrow f \leftarrow \widetilde{d}) :: c\!:\!C}$.
%     \end{enumerate}
    
%     %
    
%     \item Either $\Delta' \vdash R' :: b\!:\!A$ or $b \leftarrow f \leftarrow \widetilde{d}$, $\Delta' = \widetilde{d}:\widetilde{B}$ and $(\widetilde{e} : \widetilde{B}\vdash f = R :: g\!:\!A) \in \Sigma$ such that $\Delta'',b:A\vdash \newvar{a}{(R'' \mid a \leftarrow b)} :: c\!:\!C$ and $\Delta',\Delta''\vdash P :: c\!:\!C$ using $\runa{TS-cut}$ or $\runa{TS-def}$, respectively. In either case we must use $\runa{TS-cut}$ to get $\Delta'',b:A\vdash \newvar{a}{(R'' \mid a \leftarrow b)} :: c\!:\!C$, as we have that $b:A\vdash a\leftarrow b :: a\!:\!A$ and $\Delta'',a:A\vdash R'' :: c\!:\!C$. Then we reach $\Delta',\Delta''\vdash\newvar{h}{(R'[a\mapsto h,b\mapsto h] \mid R''[a\mapsto h,b\mapsto h])} :: c\!:\!C$ by either $\runa{TS-cut}$ or $\runa{TS-def}$. In either case we have that $\Delta'',h:A\vdash R''[a\mapsto h,b\mapsto h] :: c\!:\!C$. In the first case we have that $\Delta' \vdash R'[a\mapsto h,b\mapsto h] :: h\!:\!A$ and the latter case trivially follows by $R'[a\mapsto h,b\mapsto h] = h \leftarrow f \leftarrow \widetilde{d}$.
% \end{enumerate}

% %

% \item[$\runa{R-comm}$] Assume we reduce $P$ by $\runa{R-comm}$ then $P \equiv \inputch{a}{v}{}{R'} \mid \outputch{a}{b}{}{R''}$ for some name $b$ and processes $R'$ and $R''$, such that $\inputch{a}{v}{}{R'} \mid \outputch{a}{b}{}{R''} \longrightarrow R'[v\mapsto b] \mid R''$. For $P$ to be well-typed, it must be part of a larger process $\Delta',\Delta''\vdash\newvar{a}{P} :: c\!:\!C$ typed with $\runa{TS-cut}$ for which we have two cases
% \begin{enumerate}
%     \item $\Delta' \vdash \inputch{a}{v}{}{R'} :: a\!:\!A' \multimap A''$ and $\Delta_3,a : A'\multimap A'', b : A' \vdash \outputch{a}{b}{}{R''} :: c\!:\!C$ by $\runa{TS-$\multimap$R}$ and $\runa{TS-$\multimap$L}$ such that $\Delta'' = \Delta_3,b:A'$. By the premises to these rules we have that $\Delta',v : A' \vdash R' :: a\!:\!A''$ and $\Delta_3,a:A''\vdash R'' :: c\!:\!C$. This implies $\Delta',b : A'\vdash R'[v\mapsto b] :: a\!:\!A''$, and so by $\runa{TS-cut}$ it follows that $(\Delta',b : A'),\Delta_3\vdash \newvar{a}{(R'[v\mapsto b] \mid R'') :: c\!:\!C}$ and $\Delta = (\Delta',b : A'),\Delta_3$.
    
%     %
    
%     \item $\Delta_3,b:A' \vdash \outputch{a}{b}{}{R''} :: a\!:\!A'\otimes A''$ and $\Delta'',a : A'\otimes A''\vdash \inputch{a}{v}{}{R'} :: c\!:\!C$ by $\runa{TS-$\otimes$R}$ and $\runa{TS-$\otimes$L}$ such that $\Delta' = \Delta_3,b:A'$. By the premises to these rules we have that $\Delta_3\vdash R'' :: a\!:\!A''$ and $\Delta'',a:A'',v:A'\vdash R' :: c\!:\!C$. This implies $\Delta'',a:A'',b:A'\vdash R'[v\mapsto b] :: c\!:\!C$, and so by $\runa{TS-cut}$ it follows that $\Delta_3,(\Delta'',b : A')\vdash \newvar{a}{(R'' \mid R'[v\mapsto b])} :: c\!:\!C$ and $\Delta = \Delta_3,(\Delta'',b : A')$.
% \end{enumerate}

% %

% \item[$\runa{R-choice}$] Assume we reduce $P$ by $\runa{R-choice}$ then $P \equiv a.\texttt{case}\{ l \Rightarrow P_l \}_{l\in L} \mid a.k; R$ for some label $k$ and set of labels $L$, such that $k\in L$ and $a.\texttt{case}\{ l \Rightarrow P_l \}_{l\in L} \mid a.k; R \longrightarrow P_k \mid R$. For $P$ to be well-typed, it must be part of a larger process $\Delta',\Delta''\vdash \newvar{a}{P} :: c\!:\!C$ typed with $\runa{TS-cut}$ for which we have two cases
% \begin{enumerate}
%     \item $\Delta'\vdash a.\texttt{case}\{l \Rightarrow P_l\}_{l\in L} :: a\!:\!\&\{l : A_l\}_{l\in L}$ and $\Delta'', a : \&\{l : A_l\}_{l\in L}\vdash a.k; R :: c\!:\!C$ by $\runa{TS-$\&$R}$ and $\runa{TS-$\&$L}$. By the premises of these rules we have that $\Delta' \vdash P_k :: a\!:\!A_k$ and $\Delta'',a : A_k\vdash R :: c\!:\!C$, and so it follows by $\runa{TS-cut}$ that $\Delta',\Delta''\vdash \newvar{a}{(P_k \mid R) :: c\!:\!C}$.
        
%     %
    
%     \item $\Delta'\vdash a.k; R :: a\!:\!\oplus\{l : A_l\}_{l\in L}$ and $\Delta'',a : \oplus\{l : A_l\}_{l\in L}\vdash a.\texttt{case}\{l\Rightarrow P_l\}_{l\in L} :: c\!:\!C$ by $\runa{TS-$\oplus$R}$ and $\runa{TS-$\oplus$L}$. By the premises of these rules we have that $\Delta'\vdash R :: a\!:\!A_k$ and $\Delta'',a : A_k\vdash P_k :: c\!:\!C$, and so it follows by $\runa{TS-cut}$ that $\Delta',\Delta''\vdash \newvar{a}{(R \mid P_k)} :: c\!:\!C$.
    
% \end{enumerate}

% %

% \item[$\runa{R-def}$] Assume $P$ reduces by $\runa{R-def}$ then $P = b \leftarrow f \leftarrow \widetilde{d}$ and $(\widetilde{c}:\widetilde{B}\vdash f = P' :: a\!:\!A) \in \Sigma$, such that $Q = P'[a\mapsto b,\widetilde{c}\mapsto\widetilde{d}]$. For $P$ to be well-typed it must be part of a larger process $\widetilde{d}:\widetilde{B},\Delta'\vdash \newvar{b}{(P \mid R)} :: c\!:\!C$ typed with $\runa{TS-def}$ such that $\Delta',b:A\vdash R :: c\!:\!C$. By Lemma \ref{lemma:substlem} we have that $\widetilde{d}:\widetilde{B}\vdash P'[a\mapsto b,\widetilde{c}\mapsto\widetilde{d}] :: b\!:\!B$ and so by $\runa{TS-cut}$ we have that $\widetilde{d}:\widetilde{B},\Delta'\vdash \newvar{b}{(Q \mid R)} :: c\!:\!C$.

% %

% \item[$\runa{R-res}$] Assume that $P$ reduces by $\runa{R-res}$ then we have that $P \equiv \newvar{a}{P'}$ for some name $a$ such that $P' \longrightarrow Q'$. Then $P$ must be typed either with $\runa{TS-cut}$ or $\runa{TS-def}$ and so $P' \equiv R' \mid R''$ yielding two cases
% \begin{enumerate}
%     \item $\Delta'\vdash R' :: a\!:\!A$ such that $\Delta'',a:A\vdash R'' :: c\!:\!C$ and $\Delta',\Delta''\vdash \newvar{a}{P'}::c\!:\!C$. Either $R' \mid R''$ reduces by $\runa{R-par}$, $\runa{R-comm}$, $\runa{R-choice}$ or $\runa{R-struct}$. The first three cases are covered by the clauses for the corresponding rules, and the last case holds by induction as typability is closed under structural congruence.
    
%     \item $R' = a \leftarrow f \leftarrow \widetilde{b}$ and $(\widetilde{e} : \widetilde{B}\vdash f = R :: g\!:\!A) \in \Sigma$ such that $\Delta' = \widetilde{b}:\widetilde{B}$, $\Delta'',a:A\vdash R'' :: c\!:\!C$ and $\Delta',\Delta''\vdash \newvar{a}{P'}::c\!:\!C$. Then either $R' \mid R''$ reduces by $\runa{R-par}$ or $\runa{R-struct}$. The first case is covered by the clause for $\runa{R-par}$, and the last case holds by induction as typability is closed under structural congruence.
% \end{enumerate}

% %

% \item[$\runa{R-par}$] Assume that $P$ reduces by $\runa{R-par}$ then we have that $P \equiv P' \mid P''$ such that $P' \longrightarrow Q'$. For $P$ to be well-typed, it must be part of a larger well-typed process $\newvar{a}{(P'\mid P'')}$ typed with either $\runa{TS-cut}$ or $\runa{TS-def}$ such that either
% \begin{enumerate}
%     \item $\Delta'\vdash P' :: a\!:\!A$ such that $\Delta'',a:A\vdash P'' :: c\!:\!C$ and $\Delta',\Delta''\vdash \newvar{a}{(P'\mid P'')}::c\!:\!C$. Then by induction we have that $\Delta'\vdash Q' :: a\!:\!A$ and so it follows that $\Delta',\Delta''\vdash \newvar{a}{(Q' \mid P'')}::c\!:\!C$
    
%     \item $P' = a \leftarrow f \leftarrow \widetilde{b}$ and $(\widetilde{e} : \widetilde{B}\vdash f = R :: g\!:\!A) \in \Sigma$ such that $\Delta' = \widetilde{b}:\widetilde{B}$, $\Delta'',a:A\vdash P'' :: c\!:\!C$ and $\widetilde{b}:\widetilde{B},\Delta''\vdash \newvar{a}{P' \mid P''}::c\!:\!C$. Then it must be that $P'$ reduces to $Q'$ by $\runa{TS-def}$ such that $Q' = R[g\mapsto a,\widetilde{e}\mapsto\widetilde{b}]$. By renaming $\widetilde{e} : \widetilde{B}\vdash R :: g\!:\!A$ implies $\widetilde{b} : \widetilde{B}\vdash Q' :: a\!:\!A$ such that $\widetilde{b}:\widetilde{B},\Delta''\vdash \newvar{a}{(Q' \mid P''):: c\!:\!C}$ by $\runa{T-cut}$.
% \end{enumerate}

%%%
%%
%%
%%%

%when they contain no named processes, for $P$ to be well-typed, $P$ must be a subprocess of a larger well-typed process $R \equiv \newvar{a}{\newvar{b}{P}} \equiv \newvar{a}{(\outputch{a}{d}{}{P'} \mid \newvar{b}{(\inputch{b}{v}{}{P''} \mid b \leftarrow a}))}$ such that $\Delta',\Delta''\vdash R :: c\!:\!C$. Then from the premises of $\runa{TS-cut}$, we have that $\Delta'',a:A\vdash \outputch{a}{d}{}{P'} ::c\!:\!C$ and (by $\runa{TS-cut}$ again) $\Delta'\vdash \newvar{b}{(\inputch{b}{v}{}{P''} \mid b \leftarrow a}) :: a\!:\!A$ such that $\Delta' \vdash \inputch{b}{v}{}{P''} :: b\!:\!A$ by $\runa{TS-$\multimap$R}$ and $b : A\vdash b \leftarrow a :: a\!:\!A$ by $\runa{TS-id}$. The full reduced process is then $\newvar{a}{\newvar{b}{(P' \mid P''[v\mapsto d])}}$

%
%%%%%%%%%%
%

% \item[$\runa{R-res}$] Assume that $P$ reduces by $\runa{R-res}$. Then for $P$ to be well-typed, $P$ must be typed by either $\runa{TS-cut}$ or $\runa{TS-def}$. We consider the cases separately
% \begin{description}
% \item[$\runa{TS-cut}$] We have that $P$ is of the form $\newvar{a}{(P'\mid P'')}$ such that $\Delta' \vdash P' :: a\!:\!A$, $\Delta'', a : A\vdash P'' :: c\!:\!C$ and $\Delta',\Delta'' \vdash \newvar{a}{(P'\mid P'')} :: c\!:\!C$. By $\runa{R-res}$ we have that $P' \mid P''$ must reduce, for which several rules apply
% \begin{description}
% \item[$\runa{R-comm}$] If we reduce the parallel composition by $\runa{R-comm}$ then $P' \mid P'' \equiv \inputch{a}{v}{}{R'} \mid \outputch{a}{b}{}{R''}$ for some name $b$ and processes $R'$ and $R''$, such that $\inputch{a}{v}{}{R'} \mid \outputch{a}{b}{}{R''} \longrightarrow R'[v\mapsto b] \mid R''$. We have two cases
% \begin{enumerate}
%     \item $\Delta' \vdash \inputch{a}{v}{}{R'} :: a\!:\!A' \multimap A''$ and $\Delta_3,a : A'\multimap A'', b : A' \vdash \outputch{a}{b}{}{R''} :: c\!:\!C$ by $\runa{TS-$\multimap$R}$ and $\runa{TS-$\multimap$L}$ such that $\Delta'' = \Delta_3,b:A'$. By the premises to these rules we have that $\Delta',v : A' \vdash R' :: a\!:\!A''$ and $\Delta_3,a:A''\vdash R'' :: c\!:\!C$. This implies $\Delta',b : A'\vdash R'[v\mapsto b] :: a\!:\!A''$, and so by $\runa{TS-cut}$ it follows that $(\Delta',b : A'),\Delta_3\vdash \newvar{a}{(R'[v\mapsto b] \mid R'') :: c\!:\!C}$ and $\Delta = (\Delta',b : A'),\Delta_3$.
    
%     %
    
%     \item $\Delta_3,b:A' \vdash \outputch{a}{b}{}{R''} :: a\!:\!A'\otimes A''$ and $\Delta'',a : A'\otimes A''\vdash \inputch{a}{v}{}{R'} :: c\!:\!C$ by $\runa{TS-$\otimes$R}$ and $\runa{TS-$\otimes$L}$ such that $\Delta' = \Delta_3,b:A'$. By the premises to these rules we have that $\Delta_3\vdash R'' :: a\!:\!A''$ and $\Delta'',a:A'',v:A'\vdash R' :: c\!:\!C$. This implies $\Delta'',a:A'',b:A'\vdash R'[v\mapsto b] :: c\!:\!C$, and so by $\runa{TS-cut}$ it follows that $\Delta_3,(\Delta'',b : A')\vdash \newvar{a}{(R'' \mid R'[v\mapsto b])} :: c\!:\!C$ and $\Delta = \Delta_3,(\Delta'',b : A')$.
% \end{enumerate}

% \item[$\runa{R-choice}$] If we reduce the parallel composition by $\runa{R-choice}$ then $P' \mid P'' \equiv a.\texttt{case}\{ l \Rightarrow P_l \}_{l\in L} \mid a.k; R$ for some label and set of labels $k$ and $L$, such that $k\in L$ and $a.\texttt{case}\{ l \Rightarrow P_l \}_{l\in L} \mid a.k; R \longrightarrow P_k \mid R$. We have two cases
% \begin{enumerate}
%     \item $\Delta'\vdash a.\texttt{case}\{l \Rightarrow P_l\}_{l\in L} :: a\!:\!\&\{l : A_l\}_{l\in L}$ and $\Delta'', a : \&\{l : A_l\}_{l\in L}\vdash a.k; R :: c\!:\!C$ by $\runa{TS-$\&$R}$ and $\runa{TS-$\&$L}$. By the premises of these rules we have that $\Delta' \vdash P_k :: a\!:\!A_k$ and $\Delta'',a : A_k\vdash R :: c\!:\!C$, and so it follows by $\runa{TS-cut}$ that $\Delta',\Delta''\vdash \newvar{a}{(P_k \mid R) :: c\!:\!C}$.
        
%     %
    
%     \item $\Delta'\vdash a.k; R :: a\!:\!\oplus\{l : A_l\}_{l\in L}$ and $\Delta'',a : \oplus\{l : A_l\}_{l\in L}\vdash a.\texttt{case}\{l\Rightarrow P_l\}_{l\in L} :: c\!:\!C$ by $\runa{TS-$\oplus$R}$ and $\runa{TS-$\oplus$L}$. By the premises of these rules we have that $\Delta'\vdash R :: a\!:\!A_k$ and $\Delta'',a : A_k\vdash P_k :: c\!:\!C$, and so it follows by $\runa{TS-cut}$ that $\Delta',\Delta''\vdash \newvar{a}{(R \mid P_k)} :: c\!:\!C$.
    
% \end{enumerate}

% \item[$\runa{R-id-1}$]
% \item[$\runa{R-id-2}$]
% \item[$\runa{R-par}$] If we reduce the parallel composition by $\runa{R-par}$ then $P' \longrightarrow Q'$. Here we can apply induction, as $P'$ cannot be typed as $\Delta' \vdash P' :: a\!:\!A$ and reduce unless it is prefixed by a tick or is wrapped with a restriction (or is structurally congruent to such a process by $\runa{R-struct}$). And so, it follows that $\Delta' \vdash Q' :: a\!:\!A$, such that $\Delta',\Delta'' \vdash \newvar{a}{(Q'\mid P'')} :: c\!:\!C$.
% \item[$\runa{R-struct}$] todo: induction (with R-par after).
% \end{description}
% \item[$\runa{TS-def}$] We have that $P$ is of the form $\newvar{a}{(a\leftarrow f \leftarrow \widetilde{b} \mid P')}$ such that $(\widetilde{d} : \widetilde{B}\vdash f = P :: g\!:\!A) \in \Sigma$, $\Delta',a : A \vdash P' :: c\!:\!C$ and $\Delta',\widetilde{b} : \widetilde{B}\vdash \newvar{a}{(a\leftarrow f \leftarrow \widetilde{b} \mid P') :: c\!:\!C}$. By $\runa{R-res}$ we have that $a\leftarrow f \leftarrow \widetilde{b} \mid P'$ must reduce, for which $\runa{R-par}$ and $\runa{R-struct}$ apply. Note that the parallel composition cannot reduce by $\runa{R-par}$, as  does not  several rules apply.
% \begin{description}
% \item[$\runa{R-par}$] todo: R-def --> can type with R-cut after.
% \item[$\runa{R-struct}$] todo: induction (with R-par after). 
% \end{description}
% \end{description}


% \item[$\runa{R-struct}$] Assume that $P$ reduces by $\runa{R-struct}$. Then $P \equiv P'$, $P' \longrightarrow Q'$ and $Q' \equiv Q$. As typability is closed under structural congruence and $\Delta \vdash P :: c\!:\!C$ it follows that $\Delta \vdash P' :: c\!:\!C$. By induction this implies $\Delta \vdash Q' :: c\!:\!C$, and as $Q' \equiv Q$ we have that $\Delta\vdash Q :: c\!:\!C$.
% \end{description}
\end{proof}
\end{theorem}
%
%
% \begin{lemma}
% For any session type $A$ for which $[A]^{-1}_R$ is defined, $\text{time}(A) - 1 \geq \text{time}([A]^{-1}_R)$.
% \begin{proof}
% by case analysis on $[A]^{-1}_R$
% \begin{description}
% \item[$\dasfwr{\ocircle A'}$] We have that $\dasfwr{\ocircle A'} = A'$ and $\text{time}(\ocircle A') = 1 + \text{time}(A')$. It follows that $\text{time}(\ocircle A') - 1 \geq \text{time}(A')$.

% \item[$\dasfwr{\lozenge A'}$] We have that $\dasfwr{\lozenge A'} = \lozenge A'$ and $\text{time}(\lozenge A') = \infty$. As $\infty - 1 = \infty$ it follows that $\text{time}(\lozenge A') - 1 \geq \text{time}(\lozenge A')$.
% \end{description}
% \end{proof}
% \end{lemma}

\begin{lemma}
Let $\hat{A}[A]$ and $\hat{A}[[A]^{-1}_R]$ be session types then $\text{time}(\hat{A}[A])-1\geq\text{time}(\hat{A[[A]^{-1}_R]})$.
\begin{proof}
On the shape of $\hat{A}[\cdot]$. By definition $\hat{A}[\cdot]$ is a prefix of modalities. If the prefix contains an $\lozenge$ or $\Box$ modality, then $\text{time}(\hat{A}[A])=\text{time}(\hat{A}[[A]^{-1}_R])=\infty$ for any two session types $A$ and $[A]^{-1}_R$. As $\infty-1 = \infty$ we obtain $\text{time}(\hat{A}[A])-1\geq\text{time}(\hat{A}[[A]^{-1}_R])$. Otherwise, $\hat{A}[\cdot]$ only contains $\ocircle$ modalities, and so $\text{time}(\hat{A}[A])$ is equal to $\text{time}(A)$ plus the count of $\ocircle$ modalities in $\hat{A}[\cdot]$ which is constant. Then it remains to show that $\text{time}(A)-1\geq\text{time}([A]^{-1}_R)$. As $[A]^{-1}_R$ is defined we have that either
\begin{enumerate}
    \item $A=\dasfwr{\ocircle A'}$ with $\dasfwr{\ocircle A'} = A'$ and $\text{time}(\ocircle A') = 1 + \text{time}(A')$. It follows that $\text{time}(\ocircle A') - 1 \geq \text{time}(A')$.
    
    \item $A=\dasfwr{\lozenge A'}$ with $\dasfwr{\lozenge A'} = \lozenge A'$ and $\text{time}(\lozenge A') = \infty$. As $\infty - 1 = \infty$ it follows that $\text{time}(\lozenge A') - 1 \geq \text{time}(\lozenge A')$.
\end{enumerate}
% \item[$\dasfwr{\ocircle A'}$] We have that $\dasfwr{\ocircle A'} = A'$ and $\text{time}(\ocircle A') = 1 + \text{time}(A')$. It follows that $\text{time}(\ocircle A') - 1 \geq \text{time}(A')$.

% \item[$\dasfwr{\lozenge A'}$] We have that $\dasfwr{\lozenge A'} = \lozenge A'$ and $\text{time}(\lozenge A') = \infty$. As $\infty - 1 = \infty$ it follows that $\text{time}(\lozenge A') - 1 \geq \text{time}(\lozenge A')$.
% \end{description}
\end{proof}
\end{lemma}

\begin{lemma}
If $\hat{B}[B]\;\texttt{delayed}^\Box$ then also $\hat{B}[[B]^{-1}_L]\;\texttt{delayed}^\Box$ and if $\hat{A}[A]\;\texttt{delayed}^\lozenge$ then also $\hat{A}[[A]^{-1}_R]\;\texttt{delayed}^\lozenge$.
\begin{proof}
On the shapes of $\hat{B}[B]$, $\hat{B}[[B]^{-1}_L]$, $\hat{A}[A]$ and $\hat{A}[[A]^{-1}_R]$. We consider $\texttt{delayed}^\Box$ and $\texttt{delayed}^\lozenge$ separately
\begin{enumerate}
    \item If $\hat{B}[B]\;\texttt{delayed}^\Box$ then either $\hat{B}[\cdot]=\ocircle^*\Box$ or $\hat{B}[\cdot]=\ocircle^*$ and $B=\ocircle^*\Box$. The first case is obtained directly and the second case holds by the fact that $[\cdot]^{-1}_L$ preserves $\Box$ by definition.
    
    \item If $\hat{A}[A]\;\texttt{delayed}^\lozenge$ then either $\hat{A}[\cdot]=\ocircle^*\lozenge$ or $\hat{A}[\cdot]=\ocircle^*$ and $A=\ocircle^*\lozenge$. The first case is obtained directly and the second case holds by the fact that $[\cdot]^{-1}_R$ preserves $\lozenge$ by definition.
    
\end{enumerate}
\end{proof}
\end{lemma}


% \begin{lemma}
% Let $P$ be an arbitrary process.
% \begin{enumerate}
%     \item If $\Delta,a:\hat{A}[\lozenge A']\vdash P :: c\!:\!C$ then there exists $\hat{\Delta'}[\Delta']=\Delta$ and $\hat{C'}[C']$ such that $\Delta'\;\texttt{delayed}^\Box$ and $C'\;\texttt{delayed}^\lozenge$.
    
%     \item If $\Delta\vdash P :: a\!:\!\hat{A}[\Box A']$ then there exists $\hat{\Delta'}[\Delta']=\Delta$ such that $\Delta'\;\texttt{delayed}^\Box$.
% \end{enumerate}
% \begin{proof}
% By induction on the type rules. We only show the interesting cases
% \begin{description}
% \item[$\runa{TS-$\ocircle$LR'}$] Consider first (1).\\

% Consider then (2). If $P$ is well-typed with $\runa{TS-$\ocircle$LR'}$ then $P = \tick P'$ such that $\Delta\vdash \tick P' :: a\!:\!\hat{A}[\Box A']$ and $[\Delta]^{-1}_L\vdash P' :: a\!:\![\hat{A}[\Box A']^{-1}_R]$. By induction we have $\hat{\Delta'}[\Delta']=[\Delta]^{-1}_L$ such that $\Delta'\;\texttt{delayed}^\Box$. Then there also exists $\hat{\Delta''}[\Delta']=\Delta$.

% \item[$\runa{TS-$\ocircle$LR}$] Consider first (1).\\

% Consider then (2). If $P$ is well-typed with $\runa{TS-$\ocircle$LR}$ then $\Delta\vdash P :: a\!:\!\hat{A}[\Box A']$ and $[\Delta]^{-1}_L\vdash P :: a\!:\![\hat{A}[\Box A']^{-1}_R]$. By induction we have $\hat{\Delta'}[\Delta']=[\Delta]^{-1}_L$ such that $\Delta'\;\texttt{delayed}^\Box$. Then there also exists $\hat{\Delta''}[\Delta']=\Delta$.

% \item[$\runa{TS-$\lozenge$L}$] Consider first (1).\\

% Consider then (2). If $P$ is well-typed with $\runa{TS-$\lozenge$L}$ then $\Delta,b:\lozenge B \vdash P :: a\!:\!\hat{A}[\Box A']$ and $\Delta,b:B\vdash P :: a\!:\!\hat{A}[\Box A']$. By induction we have $\hat{\Delta'}[\Delta'],b:\hat{B'}[B']=\Delta,b:B$ such that $\Delta',b:B'\;\texttt{delayed}^\Box$. Then there also exists $\hat{\Delta'}[\Delta'],b:\lozenge\hat{B'}[B']=\Delta,b:\lozenge B$.

% \item[$\runa{TS-$\lozenge$R}$] Consider first (1).\\

% Consider then (2). If $P$ is well-typed with $\runa{TS-$\lozenge$R}$ then $\Delta \vdash P :: a\!:\!\lozenge\hat{A}[\Box A']$ and $\Delta \vdash P :: a\!:\!\hat{A}[\Box A']$. By induction we have $\hat{\Delta'}[\Delta']=\Delta$ such that $\Delta'\;\texttt{delayed}^\Box$. 

% \item[$\runa{TS-$\Box$L}$] Consider first (1).\\

% Consider then (2). If $P$ is well-typed with $\runa{TS-$\Box$L}$ then $\Delta,b:\Box B \vdash P :: a\!:\!\hat{A}[\Box A']$ and $\Delta,b:B \vdash P :: a\!:\!\hat{A}[\Box A']$. By induction we have $\hat{\Delta'}[\Delta'],b:\hat{B'}[B']=\Delta,b:B$ such that $\Delta',b:B'\;\texttt{delayed}^\Box$. Then there also exists $\hat{\Delta'}[\Delta'],b:\Box\hat{B'}[B']=\Delta,b:\Box B$. 

% \item[$\runa{TS-$\Box$R}$] Consider first (1).\\

% Consider then (2).

% \item[$\runa{TS-cut}$] Consider first (1).\\

% Consider then (2). 

% \end{description}
% \end{proof}
% \end{lemma}


% \begin{lemma}
% If $\Delta\vdash P :: a\!:\!A$ such that $P$ is not prefixed on $a$ and $A$ contains no $\lozenge$ with $P \Longrightarrow^{-1} Q$ such that $P \neq Q$ and $P\!\not\!\leadsto$ then there exists $\hat{\Delta'}[\Delta']=\Delta$ and $\hat{A'}[A']=A$ such that $\hat{\Delta'}[[\Delta']^{-1}_L]\vdash Q :: a\!:\!\hat{A'}[[A']^{-1}_R]$.
% \begin{proof}
% By induction on the type rules. We need not consider type rule $\runa{TS-def}$ as that would imply $P\!\!\leadsto$ by $\runa{R-res}$, $\runa{R-par}$ and $\runa{R-def}$. We also need not consider type rules for process prefixes except for $\runa{TS-$\ocircle$LR'}$, as $P\Longrightarrow^{-1} Q$ is productive iff at least one tick is not prefixed. We consider the cases
% \begin{description}
% \item[$\runa{TS-$\ocircle$LR'}$] We have that $\Delta\vdash \tick P' :: a\!:\!A$ such that $[\Delta]^{-1}_L\vdash P' :: a\!:\![A]^{-1}_R$. Then as $Q = P'$, we obtain $\hat{\Delta'}[[\Delta']^{-1}_L]\vdash Q :: a\!:\!\hat{A'}[[A']^{-1}_R]$ where $\hat{A'}=[\cdot]$, $\text{dom}(\Delta)=\text{dom}(\hat{\Delta'})$ and for $b\in\text{dom}(\hat{\Delta'})$ we have $\hat{\Delta'}(b)=[\cdot]$.

% \item[$\runa{TS-$\ocircle$LR}$] We have that $\Delta\vdash P :: a\!:\!A$ and $[\Delta]^{-1}_L\vdash P :: a\!:\![A]^{-1}_R$. By induction there exists $\hat{\Delta'}[\Delta']=[\Delta]^{-1}_L$ and $\hat{A'}[A']=[A]^{-1}_R$ such that $\hat{\Delta'}[[\Delta']^{-1}_L]\vdash Q :: a\!:\!\hat{A'}[[A']^{-1}_R]$. As $\hat{\Delta'}[\Delta']=[\Delta]^{-1}_L$ and $\hat{A'}[A']=[A]^{-1}_R$ it must be that $\hat{\Delta'}[\Delta']=[\hat{\Delta''}[\Delta'']]^{-1}_L$ and $\hat{A'}[A']=[\hat{A''}[A'']]^{-1}_R$ for some $\hat{\Delta''}[\Delta'']$ and $\hat{A''}[A'']$ such that also $[\hat{\Delta''}[[\Delta'']^{-1}_L]]^{-1}_L\vdash Q :: a\!:\![\hat{A''}[[A'']^{-1}_R]]^{-1}_R$. It follows from $\runa{TS-$\ocircle$LR}$ that also $\hat{\Delta''}[[\Delta'']^{-1}_L]\vdash Q :: a\!:\!\hat{A''}[[A'']^{-1}_R]$.

% \item[$\runa{TS-$\lozenge$L}$] We have that $\Delta,a:\lozenge A\vdash P :: b\!:\!B$, such that $\Delta\;\texttt{delayed}^\Box$, $B\;\texttt{delayed}^\lozenge$ and $\Delta,a:A\vdash P :: b\!:\!B$. By induction there exists $\hat{\Delta'}[\Delta'],a:\hat{A'}[A']=\Delta,a:A$ and $\hat{B'}[B']=B$ such that $\hat{\Delta'}[[\Delta']^{-1}_L],a:\hat{A'}[[A']^{-1}_L]\vdash Q :: b\!:\!\hat{B'}[[B']^{-1}_R]$ and by Lemma \ref{lemma:progdel} $\hat{\Delta'}[\Delta']\;\texttt{delayed}^\Box$ implies $\hat{\Delta'}[[\Delta']^{-1}_L]\;\texttt{delayed}^\Box$ and $\hat{B'}[B']\;\texttt{delayed}^\lozenge$ implies $\hat{B'}[[B']^{-1}_R]\;\texttt{delayed}^\lozenge$ and so by $\runa{TS-$\lozenge$L}$ we obtain $\hat{\Delta'}[[\Delta']^{-1}_L],a:\lozenge\hat{A'}[[A']^{-1}_L]\vdash Q :: b\!:\!\hat{B'}[[B']^{-1}_R]$.

% \item[$\runa{TS-$\lozenge$R}$] We have that $\Delta\vdash P :: a\!:\!\lozenge A$ and $\Delta\vdash P :: a\!:\!A$. By induction there exists $\hat{\Delta'}[\Delta']=\Delta$ and $\hat{A'}[A']=A$ such that $\hat{\Delta'}[[\Delta']^{-1}_L]\vdash Q :: a\!:\!\hat{A'}[[A']^{-1}_R]$. It follows directly from $\runa{TS-$\lozenge$R}$ that also $\hat{\Delta'}[[\Delta']^{-1}_L]\vdash Q :: a\!:\!\lozenge\hat{A'}[[A']^{-1}_R]$.

% \item[$\runa{TS-$\Box$L}$] We have that $\Delta,a:\Box A\vdash P :: b\!:\!B$ and $\Delta,a:A\vdash P :: b\!:\!B$. By induction there exists $\hat{\Delta'}[\Delta'],a:\hat{A'}[A']=\Delta,a:A$ and $\hat{B'}[B']=B$ such that $\hat{\Delta'}[[\Delta']^{-1}_L],a:\hat{A'}[[A']^{-1}_L]\vdash Q :: b\!:\!\hat{B'}[[B']^{-1}_R]$. It follows directly from $\runa{TS-$\Box$L}$ that also $\hat{\Delta'}[[\Delta']^{-1}_L],a:\Box\hat{A'}[[A']^{-1}_L]\vdash Q :: b\!:\!\hat{B'}[[B']^{-1}_R]$.

% \item[$\runa{TS-$\Box$R}$] We have that $\Delta\vdash P :: a\!:\!\Box A$, $\Delta\;\texttt{delayed}^\Box$ and $\Delta\vdash P :: a\!:\!A$. By induction there exists $\hat{\Delta'}[\Delta']=\Delta$ and $\hat{A'}[A']=A$ such that $\hat{\Delta'}[[\Delta']^{-1}_L]\vdash Q :: a\!:\!\hat{A'}[[A']^{-1}_R]$ and from Lemma \ref{lemma:progdel} $\hat{\Delta'}[[\Delta']^{-1}_L]\;\texttt{delayed}^\Box$ follows from $\hat{\Delta'}[\Delta']\;\texttt{delayed}^\Box$, and so by $\runa{TS-$\Box$R}$ we obtain $\hat{\Delta'}[[\Delta']^{-1}_L]\vdash Q :: a\!:\!\Box\hat{A'}[[A']^{-1}_R]$.

% \item[$\runa{TS-cut}$] We have that $\Delta_1,\Delta_2\vdash \newvar{a}{(P'\mid 
% P'')} :: c\!:\!C$ with $\Delta_1\vdash P' :: a\!:\!A$ and $\Delta_2,a:A\vdash P'' :: c\!:\!C$. We identify three cases where $P \Longrightarrow^{-1} Q$ is productive
% \begin{enumerate}
%     \item $\newvar{a}{(P'\mid P'')} \Longrightarrow^{-1} \newvar{a}{(Q' \mid P'')}$ with $P' \neq Q'$. For $P$ to not be prefixed on $c$, $P''$ also cannot be prefixed on $c$. Then as $P'' \Longrightarrow^{-1} P''$ and $P\!\not\!\leadsto$ the subprocess that provides a session on $A$ in $P'$ must be prefixed with a tick or be $\mathbf{0}$. Thus, $P'$ cannot be prefixed on $a$. Then by induction there exists $\hat{\Delta_1'}[\Delta_1']=\Delta_1$ and $\hat{A'}[A']=A$ such that $\hat{\Delta_1'}[[\Delta_1']^{-1}_L]\vdash Q' :: a\!:\!\hat{A'}[[A']^{-1}_R]$. As $\hat{A'}[[A']^{-1}_R]$ is defined it must be that either
%     \begin{itemize}
%         \item $\hat{A'}[A']=\hat{A'}[\lozenge A'']$ for some session type $A''$ and so $\hat{A'}[A']=\hat{A'}[[A']^{-1}_R]$. From this we obtain $\Delta_2,a:\hat{A'}[\lozenge A'']\vdash P'' :: c\!:\!C$. However, by Lemma \ref{lemma:deldiaimp} $C$ then contains an $\lozenge$ modality, contradicting our assumption. %there exists $\Delta_2=\hat{\Delta_2'}[\Delta_2']$ and $C=\hat{C'}[C']$ such that $\Delta_2'\;\texttt{delayed}^\Box$ and $C'\;\texttt{delayed}^\lozenge$. Then $C'=\ocircle^n\lozenge C''$ for some $n\geq 0$ and session type $C''$ and so we obtain $C=\hat{C'}[\ocircle^n[\lozenge C'']]=\hat{C'}[\ocircle^n[[\lozenge C'']^{-1}_R]]=\hat{C''}[[\lozenge C'']^{-1}_R]$ for some $\hat{C''}[\cdot]$, and for $b\in\text{dom}(\Delta_2')$ we have $\Delta_2'(b)=\ocircle^m\Box B'$ for some $m\geq 0$ and session type $B'$ and so $\Delta_2'(b)=\ocircle^m[\Box B']=\ocircle^m[[\Box B']^{-1}_L]$. Thus, there exists $\hat{\Delta_2''}[\Delta_2'']=\Delta_2$ such that $\hat{\Delta_2''}[[\Delta_2'']^{-1}_L],\hat{A'}[[A']^{-1}_R]\vdash P'' :: c\!:\!\hat{C''}[[\lozenge C'']^{-1}_R]$. From $\runa{TS-cut}$ we then obtain $\hat{\Delta_1'}[[\Delta_1']^{-1}_L],\hat{\Delta_2''}[[\Delta_2'']^{-1}_L]\vdash \newvar{a}{(Q'\mid P'') :: c\!:\!\hat{C''}[[\lozenge C'']^{-1}_R]}$.
        
%         \item $\hat{A'}[A']=\hat{A'}[\ocircle A'']$ for some session type $A''$ and so $\hat{A'}[[\ocircle A'']^{-1}_R] = \hat{A'}[A'']$. As $P''$ is not prefixed on $c$ and $\Delta_2,a:\hat{A'}[\ocircle A'']\vdash P'' :: c\!:\!C$ with $P''\Longrightarrow^{-1} P''$, $C$ must be prefixed with at least one modality typed with $\runa{TS-$\ocircle$LR}$ corresponding to the $\ocircle$ modality in the prefix of $A$. This implies $\Delta_2=\hat{\Delta_2'}[\Delta_2']$ and $C=\hat{C'}[C']$ such that $\hat{\Delta_2'}[[\Delta_2']^{-1}_L],a:\hat{A'}[A'']\vdash P'' :: c\!:\!\hat{C'}[[C']^{-1}_R]$, and so it follows from $\runa{TS-cut}$ that $\hat{\Delta_1'}[[\Delta_1']^{-1}_L],\hat{\Delta_2'}[[\Delta_2']^{-1}_L]\vdash \newvar{a}{(Q'\mid P'') :: c\!:\!\hat{C'}[[C']^{-1}_R]}$.
%     \end{itemize}
    
%     \item $\newvar{a}{(P'\mid P'')} \Longrightarrow^{-1} \newvar{a}{(P' \mid Q'')}$ with $P'' \neq Q''$. For $P$ to not be prefixed on $c$ $P''$ also cannot be prefixed on $c$, and $C$ contains no $\lozenge$ by assumption. Then by induction there exists $\hat{\Delta_2'}[\Delta_2'],a:\hat{A''}[A'']=\Delta_2,a:A$ and $\hat{C'}[C']=C$ such that $\hat{\Delta_2'}[[\Delta_2']^{-1}_L],a:\hat{A''}[[A'']^{-1}_L]\vdash P'' :: c\!:\!\hat{C'}[[C']^{-1}_R]$. As $\hat{A'}[[A']^{-1}_L]$ is defined it must be that either
%     \begin{itemize}
%         \item $\hat{A'}[A']=\hat{A'}[\Box A'']$ for some session type $A''$ and so $\hat{A'}[A']=\hat{A'}[[A']^{-1}_L]$. 
        
%         Then as $P' \Longrightarrow^{-1} P'$ and $P\!\not\!\leadsto$ the subprocess that consumes a session on $A$ in $P'$ must be prefixed with a tick or be $\mathbf{0}$. Thus, $P''$ cannot be prefixed on $a$.
        
%         From this we obtain $\Delta_1\vdash P' :: a\!:\!\hat{A'}[[A']^{-1}_L]$. Then by Lemma \ref{lemma:deldiaimp} there exists $\Delta_1=\hat{\Delta_1'}[\Delta_1']$ such that $\Delta_1'\;\texttt{delayed}^\Box$. Then for $b\in\text{dom}(\Delta_1')$ we have $\Delta_1'(b)=\ocircle^m\Box B'$ for some $m\geq 0$ and session type $B'$ and so $\Delta_1'(b)=\ocircle^m[\Box B']=\ocircle^m[[\Box B']^{-1}_L]$. Thus, there exists $\hat{\Delta_1''}[\Delta_1'']=\Delta_1$ such that $\hat{\Delta_1''}[[\Delta_1'']^{-1}_L]\vdash P' :: a\!:\!\hat{A'}[[A']^{-1}_L]$. From $\runa{TS-cut}$ we then obtain $\hat{\Delta_1''}[[\Delta_1'']^{-1}_L],\hat{\Delta_2'}[[\Delta_2']^{-1}_L]\vdash \newvar{a}{(P'\mid Q'') :: c\!:\!\hat{C''}[[\lozenge C'']^{-1}_R]}$.
        
%         \item $\hat{A'}[A']=\hat{A'}[\ocircle A'']$ for some session type $A''$ and so $\hat{A'}[[\ocircle A'']^{-1}_L] = \hat{A'}[A'']$. As $\Delta_1\vdash P' :: a:\hat{A'}[\ocircle A'']$ with $P' \Longrightarrow^{-1} P'$, we must use $\runa{TS-$\ocircle$LR}$ to consume the $\ocircle$ modality. This implies $\Delta_1=\hat{\Delta_1'}[\Delta_1']$ such that $\hat{\Delta_1'}[[\Delta_1']^{-1}_L] \vdash P' :: a:\hat{A'}[A'']$, and so it follows from $\runa{TS-cut}$ that $\hat{\Delta_1'}[[\Delta_1']^{-1}_L],\hat{\Delta_2'}[[\Delta_2']^{-1}_L]\vdash \newvar{a}{(P'\mid Q'') :: c\!:\!\hat{C'}[[C']^{-1}_R]}$.
        
%     \end{itemize}
    
%     \item $\newvar{a}{(P'\mid P'')} \Longrightarrow^{-1} \newvar{a}{(Q' \mid Q'')}$ with $P' \neq Q'$ and $P'' \neq Q''$. For $P$ to not be prefixed on $c$ $P''$ also cannot be prefixed on $c$. Then by induction there exists $\hat{\Delta_2'}[\Delta_2'],a:\hat{A_2}[A_2]=\Delta_2,a:A$ and $\hat{C'}[C']=C$ such that $\hat{\Delta_2'}[[\Delta_2']^{-1}_L],a:\hat{A_2}[[A_2]^{-1}_L]\vdash Q'' :: c\!:\!\hat{C'}[[C']^{-1}_R]$.
    
%     As $P\!\not\!\leadsto$ either $P'$ or $P''$ is not prefixed on $a$. We consider the cases
%     \begin{itemize}
%         \item $P'$ is not prefixed on $a$. Then by induction we have $\hat{\Delta_1'}[\Delta_1']=\Delta_1$ and $\hat{A_1}[A_1]=A$ such that $\hat{\Delta_1'}[[\Delta_1']^{-1}_L]\vdash Q' :: a\!:\!\hat{A_1}[[A_1]^{-1}_R]$. As $\hat{A_1}[[A_1]^{-1}_R]$ and $\hat{A_2}[[A_2]^{-1}_L]$ are defined it must be that either
        
%         \item
        
%         \item
%     \end{itemize}
    
%     \begin{itemize}
%         %\item $\hat{A_1}[A_1]=\hat{A_1}[\lozenge A_1']$ and $\hat{A_2}[A_2]=\hat{A_2}[\Box A_2']$ for some session types $A_1'$ and $A_2'$. We obtain $\hat{\Delta_1'}[[\Delta_1']^{-1}_L]\vdash Q' :: a\!:\!\hat{A_2}[[\Box A_2']^{-1}_L]$ directly from $\hat{A_1}[[\lozenge A_1']^{-1}_R]=\hat{A_1}[A_1]$ and $\hat{A_2}[[\Box A_2']^{-1}_L]=\hat{A_2}[A_2]$ as $\hat{A_1}[A_1]=\hat{A_2}[A_2]$.
        
%         \item $\hat{A_1}[A_1]=\hat{A_1}[\lozenge A_1']$ and so $\hat{A_1}[[\lozenge A_1']^{-1}_R]=\hat{A_1}[A_1]$. From this we obtain $\Delta_2,a:\hat{A_1}[[\lozenge A_1']^{-1}_R]\vdash P'' :: c\!:\!C$ and by Theorem \ref{theorem:sr} $\Delta_2,a:\hat{A_1}[[\lozenge A_1']^{-1}_R]\vdash Q'' :: c\!:\!C$. Then by Lemma \ref{lemma:deldiaimp} there exists $\hat{\Delta_2''}[\Delta_2'']=\Delta_2$ and $\hat{C''}[C'']=C$ such that $\Delta_2''\;\texttt{delayed}^\Box$ and $C''\;\texttt{delayed}^\lozenge$. Then $C''=\ocircle^n\lozenge C_3$ for some $n\geq 0$ and session type $C_3$ and so we obtain $C=\hat{C''}[\ocircle^n[\lozenge C_3]]=\hat{C''}[\ocircle^n[[\lozenge C_3]^{-1}_R]]=\hat{C_3}[[\lozenge C_3]^{-1}_R]$ for some $\hat{C_3}[\cdot]$, and for $b\in\text{dom}(\Delta_2'')$ we have $\Delta_2''(b)=\ocircle^m\Box B'$ for some $m\geq 0$ and session type $B'$ and so $\Delta_2''(b)=\ocircle^m[\Box B']=\ocircle^m[[\Box B']^{-1}_L]$. Thus, there exists $\hat{\Delta_3}[\Delta_3]=\Delta_2$ such that $\hat{\Delta_3}[[\Delta_3]^{-1}_L],\hat{A_1}[[A_1]^{-1}_R]\vdash Q'' :: c\!:\!\hat{C_3}[[\lozenge C_3]^{-1}_R]$. From $\runa{TS-cut}$ we then obtain $\hat{\Delta_1'}[[\Delta_1']^{-1}_L],\hat{\Delta_3}[[\Delta_3]^{-1}_L]\vdash \newvar{a}{(Q'\mid Q'') :: c\!:\!\hat{C_3}[[\lozenge C_3]^{-1}_R]}$.
        
%         %As $\hat{A_1}[A_1]=\hat{A_2}[A_2]=A$ for $\Delta_1\vdash P' :: a\!:\!A$ to hold, the $\ocircle$ modality removed from $\hat{A_2}[A_2']$ must be typed with $\runa{TS-$\ocircle$LR}$ for $P'$. By premise there must then exist $\hat{\Delta_1''}[\Delta_1'']=\Delta_1$ such that $\hat{\Delta_1''}[[\Delta_1'']^{-1}_L]\vdash P' :: a\!:\!\hat{A_2}[A_2']$, but then we could have used $\runa{TS-$\ocircle$LR'}$ here instead, from which we obtain $\hat{\Delta_1''}[[\Delta_1'']^{-1}_L]\vdash Q' :: a\!:\!\hat{A_2}[A_2']$.
        
%         \item $\hat{A_2}[A_2]=\hat{A_2}[\Box A_2']$ and so $\hat{A_2}[[\Box A_2']^{-1}_L]=\hat{A_2}[A_2]$. From this we obtain $\Delta_1\vdash P' :: a\!:\!\hat{A_2}[[\Box A_2']^{-1}_L]$ and by Theorem \ref{theorem:sr} $\Delta_1\vdash Q' :: a\!:\!\hat{A_2}[[\Box A_2']^{-1}_L]$. Then by Lemma \ref{lemma:deldiaimp} there exists $\hat{\Delta_1''}[\Delta_1'']=\Delta_1$ such that $\Delta_1''\;\texttt{delayed}^\Box$. Then for $b\in\text{dom}(\Delta_1'')$ we have $\Delta_1''(b)=\ocircle^m\Box B'$ for some $m\geq 0$ and session type $B'$ and so $\Delta_1''(b)=\ocircle^m[\Box B']=\ocircle^m[[\Box B']^{-1}_L]$. Thus, there exists $\hat{\Delta_3}[\Delta_3]=\Delta_1$ such that $\hat{\Delta_3}[[\Delta_3]^{-1}_L]\vdash Q' :: a\!:\!\hat{A_2}[[A_2]^{-1}_L]$. From $\runa{TS-cut}$ we then obtain $\hat{\Delta_3}[[\Delta_3]^{-1}_L],\hat{\Delta_2'}[[\Delta_2']^{-1}_L]\vdash \newvar{a}{(Q'\mid Q'') :: c\!:\!\hat{C''}[[\lozenge C'']^{-1}_R]}$.
        
        
        
%         %As $\hat{A_1}[A_1]=\hat{A_2}[A_2]=A$ for $\Delta_2,a:A\vdash P'' :: c\!:\!C$ to hold, the $\ocircle$ modality removed from $\hat{A_1}[A_1']$ must be typed with $\runa{TS-$\ocircle$LR}$ for $P''$. By premise there must then exist $\hat{\Delta_2''}[\Delta_2'']=\Delta_2$ and $\hat{C''}[C'']=C$ such that $\hat{\Delta_2''}[[\Delta_2'']^{-1}_L],a:\hat{A_1}[A_1']\vdash P'' :: a\!:\!\hat{C''}[[C'']^{-1}_R]$, but then we could have used $\runa{TS-$\ocircle$LR'}$ here instead, from which we obtain $\hat{\Delta_2''}[[\Delta_2'']^{-1}_L],a:\hat{A_1}[A_1']\vdash Q'' :: a\!:\!\hat{C''}[[C'']^{-1}_R]$.
        
%         \item $\hat{A_1}[A_1]=\hat{A_1}[\ocircle A_1']$ and $\hat{A_2}[A_2]=\hat{A_2}[\ocircle A_2']$ for some session types $A_1'$ and $A_2'$ and so $\hat{A_1}[[\ocircle A_1']^{-1}_R]=\hat{A_1}[A_1']$ and $\hat{A_2}[[\ocircle A_2']^{-1}_L]=\hat{A_2}[A_2']$. Either $\hat{A_1}[A_1']=\hat{A_2}[A_2']$ and we obtain $\hat{\Delta_2'}[[\Delta_2']^{-1}_L],a:\hat{A_1}[A_1']\vdash Q'':: c\!:\!\hat{C'}[[C']^{-1}_R]$ directly, or there is at least one $\Box$ or $\lozenge$ modality between the two $\ocircle$ modalities removed from $A_1$ and $A_2$, respectively. Then the remaining $\ocircle$ modality in $Q'$ and $Q''$ must be typed with either $\runa{TS-$\ocircle$LR'}$ or $\runa{TS-$\ocircle$LR}$. As $\Box$ and $\lozenge$ are not syntax directed, and as $\runa{TS-$\ocircle$LR'}$ and $\runa{TS-$\ocircle$LR}$ do not affect $\texttt{delayed}^\Box$ and $\texttt{delayed}^\lozenge$, by definition of $[\cdot]^{-1}_L$ and $[\cdot]^{-1}_R$, we can rearrange the prefixes of modalities in $\hat{A_1}[[\ocircle A_1']^{-1}_R]$ and $\hat{A_2}[[\ocircle A_2']^{-1}_L]$ without affecting typability such that we have $\hat{A_1'}[[\ocircle A_1'']^{-1}_R]=\hat{A_2'}[[\ocircle A_2'']^{-1}_L]$ with $\hat{\Delta_1'}[[\Delta_1']^{-1}_L]\vdash Q' :: a\!:\hat{A_1}[[\ocircle A_1']^{-1}_R]$, and $\hat{\Delta_2'}[[\Delta_2']^{-1}_L],a:\hat{A_2}[[\ocircle A_2']^{-1}_L]\vdash Q'' :: c\!:\!\hat{C'}[[C']^{-1}_R]$. Thus, by application of $\runa{TS-cut}$ we obtain $\hat{\Delta_1'}[[\Delta_1']^{-1}_L],\hat{\Delta_2'}[[\Delta_2']^{-1}_L]\vdash \newvar{a}{(Q' \mid Q'')} :: c\!:\!\hat{C'}[[C']^{-1}_R]$.
        
        
%         %there are only $\ocircle$ modalities between the two modalities removed, without affecting typability.  move the $\Box$ and $\lozenge$ modalities between the two $\ocircle$ modalities outward,  
        
        
%         %two distinct $\ocircle$ modalities were removed. Then as $\hat{A_1}[A_1]=\hat{A_2}[A_2]$ for $\Delta_2,a:A\vdash P'' :: c\!:\!C$ to hold, the $\ocircle$ modality must be typed with $\runa{TS-$\ocircle$LR}$. By premise there must then exist $\hat{\Delta_2''}[\Delta_2'']=\Delta_2$ and $\hat{C''}[C'']=C$ such that $\hat{\Delta_2''}[[\Delta_2'']^{-1}_L],a:\hat{A_1}[A_1']\vdash P'':: c\!:\!\hat{C''}[[C'']^{-1}_R]$, but then we could have used $\runa{TS-$\ocircle$LR'}$ here instead, from which we obtain $\hat{\Delta_2''}[[\Delta_2'']^{-1}_L],a:\hat{A_1}[A_1']\vdash Q'':: c\!:\!\hat{C''}[[C'']^{-1}_R]$.
        
%     \end{itemize}
    
% \end{enumerate}

% \end{description}
% %As $P \Longrightarrow Q$ we know that $P$ cannot be prefixed with anything except for a tick, and so it is sufficient to consider $\runa{TS-$\ocircle$LR'}$, $\runa{TS-cut}$, $\runa{TS-def}$, $\runa{TS-$\ocircle$LR}$, $\runa{TS-$\lozenge$L}$, $\runa{TS-$\lozenge$R}$, $\runa{TS-$\Box$L}$ and $\runa{TS-$\Box$R}$
% % \begin{description}
% % \item[$\runa{TS-$\ocircle$LR'}$] We have that $P=\tick P'$, $Q=P'$ and $[\Delta]^{-1}_L\vdash P' :: a\!:\![A]^{-1}_R$. By lemma \ref{lemma:timegeq} $\text{time}(A)-1\geq\text{time}([A]^{-1}_R)$, and by Lemma \ref{TODO}, $\texttt{delayed}^\Box$ and $\texttt{delayed}^\lozenge$ are invariant to $[\cdot]^{-1}_L$ and $[\cdot]^{-1}_R$, respectively.

% % \item[$\runa{TS-cut}$]

% % %

% % \item[$\runa{TS-def}$]

% % %

% % \item[$\runa{TS-$\ocircle$LR}$] If $\Delta \vdash P :: a\!:\!A$ by $\runa{TS-$\ocircle$LR}$ then $[\Delta]^{-1}_L\vdash P :: a\!:\![A]^{-1}_R$. By Lemma \ref{lemma:timegeq} $\text{time}(A)-1\geq\text{time}([A]^{-1}_R)$ and by Lemma \ref{TODO}, $\texttt{delayed}^\Box$ and $\texttt{delayed}^\lozenge$ are invariant to $[\cdot]^{-1}_L$ and $[\cdot]^{-1}_R$, respectively. By induction, $\Delta''\vdash Q :: a\!:\!A''$ such that $\text{time}([A]^{-1}_R)-1\geq\text{time}(A'')$. 

% % %

% % \item[$\runa{TS-$\lozenge$L}$]

% % %

% % \item[$\runa{TS-$\lozenge$R}$]

% % %

% % \item[$\runa{TS-$\Box$L}$]

% % %

% % \item[$\runa{TS-$\Box$R}$]


% % \end{description}
% % \begin{enumerate}
% %     \item $P = \tick P'$ and $\Delta'\vdash P :: a\!:\!A'$ by $\runa{TS-$\ocircle$LR'}$. Then $P$ is in canonical form and $Q = P'$. By $\runa{TS-$\ocircle$LR'}$ we obtain that $[\Delta']^{-1}_L\vdash P' :: a\!:\![A']^{-1}_R$. We show that this implies $[\Delta]^{-1}_L\vdash P' :: a\!:\![A]^{-1}_R$ by induction on the temporal type rules
% %     \begin{description}
% %     \item[hmm]
    
    
% %     \end{description}
    
% %     \item $P \equiv \newvar{a}{(Q \mid R)}$ and $\Delta P\vdash P :: a\!:\!A$ by $\runa{TS-cut}$ or $\runa{TS-def}$.
% % \end{enumerate}
% \end{proof}
% \end{lemma}


% %
% % \begin{theorem}[Subject Reduction]
% % If $\Delta \vdash P :: a\!:\!A$ and $P \longrightarrow Q$ then $\Delta\vdash Q :: a\!:\!A$.
% % \begin{proof}
% % Proof by induction on the extended reduction rules. The proof uses the fact that a well-typed process cannot \textit{consume} the session it provides on reduction, by type rules $\runa{TS-cut}$ and $\runa{TS-def}$. The proof is slightly tedious, as the type rules are not syntax directed.
% % \begin{description}
% % \item[$\runa{R-tick}$] Assume that $P$ reduces by $\runa{R-tick}$, such that $P$ is of the form $\texttt{tick}.P'$ and $Q = P'$. Then by $\runa{TS-$\ocircle$LR'}$, we have that $[\Delta]^{-1}_L \vdash P' :: [a:A]^{-1}_R$ such that $\Delta \vdash \texttt{tick}.P' :: a\!:\!A$. It follows from type rule $\runa{TS-$\ocircle$LR}$ that also $\Delta \vdash P' :: a\!:\!A$.

% % %

% % \item[$\runa{R-id}$] Assume that $P$ reduces by $\runa{R-id}$ then we have that $P \equiv \newvar{a}{\newvar{b}{(P' \mid a \leftarrow b)}}$ such that $Q \equiv \newvar{h}{(P'[a\mapsto h,b\mapsto h])}$ for some name $h \notin fv(P')$. Then, as restrictions are only typable by $\runa{TS-cut}$ and $\runa{TS-def}$, $P'$ must be of the form $R' \mid R''$ such that $P \equiv \newvar{a}{(R' \mid \newvar{b}{(R'' \mid a \leftarrow b)})}$ or $P \equiv \newvar{b}{(R' \mid \newvar{a}{(R'' \mid a \leftarrow b)})}$. We consider the cases separately
% % \begin{enumerate}
% %     \item $\Delta'',a:A \vdash R' :: c\!:\!C$ such that $\Delta'\vdash \newvar{b}{(R'' \mid a \leftarrow b)} :: a\!:\!A$ and $\Delta',\Delta''\vdash P :: c\!:\!C$ using $\runa{TS-cut}$. Then we can type $\newvar{b}{(R'' \mid a \leftarrow b)}$ with either $\runa{TS-cut}$ or $\runa{TS-def}$
% %     \begin{enumerate}
% %         \item $\Delta' \vdash R'' :: b\!:\!A$ such that $b:A\vdash a \leftarrow b :: a\!:\!A$ and $\Delta' \vdash \newvar{b}{(R'' \mid a \leftarrow b)} :: a\!:\!A$. Then it follows by renaming that $\Delta''\vdash R''[a\mapsto h,b\mapsto h] :: h\!:\!A$ and $\Delta'',h:A \vdash R' :: c\!:\!C$ such that $\Delta',\Delta''\vdash\newvar{h}{(R'[a\mapsto h,b\mapsto h] \mid R''[a\mapsto h,b\mapsto h]) :: c\!:\!C}$.
        
% %         \item $R'' = b \leftarrow f \leftarrow \widetilde{d}$ and $(\widetilde{e} : \widetilde{B}\vdash f = R :: g\!:\!A) \in \Sigma$ such that $\Delta' = \widetilde{d}:\widetilde{B}$, $b:A\vdash a \leftarrow b :: a\!:\!A$ and $\Delta' \vdash \newvar{b}{(R'' \mid a \leftarrow b)} :: a\!:\!A$. Then it follows by renaming that $\Delta'',h:A \vdash R' :: c\!:\!C$ such that $\Delta',\Delta''\vdash\newvar{h}{(R'[a\mapsto h,b\mapsto h] \mid h \leftarrow f \leftarrow \widetilde{d}) :: c\!:\!C}$.
% %     \end{enumerate}
    
% %     %
    
% %     \item Either $\Delta' \vdash R' :: b\!:\!A$ or $b \leftarrow f \leftarrow \widetilde{d}$, $\Delta' = \widetilde{d}:\widetilde{B}$ and $(\widetilde{e} : \widetilde{B}\vdash f = R :: g\!:\!A) \in \Sigma$ such that $\Delta'',b:A\vdash \newvar{a}{(R'' \mid a \leftarrow b)} :: c\!:\!C$ and $\Delta',\Delta''\vdash P :: c\!:\!C$ using $\runa{TS-cut}$ or $\runa{TS-def}$, respectively. In either case we must use $\runa{TS-cut}$ to get $\Delta'',b:A\vdash \newvar{a}{(R'' \mid a \leftarrow b)} :: c\!:\!C$, as we have that $b:A\vdash a\leftarrow b :: a\!:\!A$ and $\Delta'',a:A\vdash R'' :: c\!:\!C$. Then we reach $\Delta',\Delta''\vdash\newvar{h}{(R'[a\mapsto h,b\mapsto h] \mid R''[a\mapsto h,b\mapsto h])} :: c\!:\!C$ by either $\runa{TS-cut}$ or $\runa{TS-def}$. In either case we have that $\Delta'',h:A\vdash R''[a\mapsto h,b\mapsto h] :: c\!:\!C$. In the first case we have that $\Delta' \vdash R'[a\mapsto h,b\mapsto h] :: h\!:\!A$ and the latter case trivially follows by $R'[a\mapsto h,b\mapsto h] = h \leftarrow f \leftarrow \widetilde{d}$.
% % \end{enumerate}

% % %

% % \item[$\runa{R-comm}$] Assume we reduce $P$ by $\runa{R-comm}$ then $P \equiv \inputch{a}{v}{}{R'} \mid \outputch{a}{b}{}{R''}$ for some name $b$ and processes $R'$ and $R''$, such that $\inputch{a}{v}{}{R'} \mid \outputch{a}{b}{}{R''} \longrightarrow R'[v\mapsto b] \mid R''$. For $P$ to be well-typed, it must be part of a larger process $\Delta',\Delta''\vdash\newvar{a}{P} :: c\!:\!C$ typed with $\runa{TS-cut}$ for which we have two cases
% % \begin{enumerate}
% %     \item $\Delta' \vdash \inputch{a}{v}{}{R'} :: a\!:\!A' \multimap A''$ and $\Delta_3,a : A'\multimap A'', b : A' \vdash \outputch{a}{b}{}{R''} :: c\!:\!C$ by $\runa{TS-$\multimap$R}$ and $\runa{TS-$\multimap$L}$ such that $\Delta'' = \Delta_3,b:A'$. By the premises to these rules we have that $\Delta',v : A' \vdash R' :: a\!:\!A''$ and $\Delta_3,a:A''\vdash R'' :: c\!:\!C$. This implies $\Delta',b : A'\vdash R'[v\mapsto b] :: a\!:\!A''$, and so by $\runa{TS-cut}$ it follows that $(\Delta',b : A'),\Delta_3\vdash \newvar{a}{(R'[v\mapsto b] \mid R'') :: c\!:\!C}$ and $\Delta = (\Delta',b : A'),\Delta_3$.
    
% %     %
    
% %     \item $\Delta_3,b:A' \vdash \outputch{a}{b}{}{R''} :: a\!:\!A'\otimes A''$ and $\Delta'',a : A'\otimes A''\vdash \inputch{a}{v}{}{R'} :: c\!:\!C$ by $\runa{TS-$\otimes$R}$ and $\runa{TS-$\otimes$L}$ such that $\Delta' = \Delta_3,b:A'$. By the premises to these rules we have that $\Delta_3\vdash R'' :: a\!:\!A''$ and $\Delta'',a:A'',v:A'\vdash R' :: c\!:\!C$. This implies $\Delta'',a:A'',b:A'\vdash R'[v\mapsto b] :: c\!:\!C$, and so by $\runa{TS-cut}$ it follows that $\Delta_3,(\Delta'',b : A')\vdash \newvar{a}{(R'' \mid R'[v\mapsto b])} :: c\!:\!C$ and $\Delta = \Delta_3,(\Delta'',b : A')$.
% % \end{enumerate}

% % %

% % \item[$\runa{R-choice}$] Assume we reduce $P$ by $\runa{R-choice}$ then $P \equiv a.\texttt{case}\{ l \Rightarrow P_l \}_{l\in L} \mid a.k; R$ for some label $k$ and set of labels $L$, such that $k\in L$ and $a.\texttt{case}\{ l \Rightarrow P_l \}_{l\in L} \mid a.k; R \longrightarrow P_k \mid R$. For $P$ to be well-typed, it must be part of a larger process $\Delta',\Delta''\vdash \newvar{a}{P} :: c\!:\!C$ typed with $\runa{TS-cut}$ for which we have two cases
% % \begin{enumerate}
% %     \item $\Delta'\vdash a.\texttt{case}\{l \Rightarrow P_l\}_{l\in L} :: a\!:\!\&\{l : A_l\}_{l\in L}$ and $\Delta'', a : \&\{l : A_l\}_{l\in L}\vdash a.k; R :: c\!:\!C$ by $\runa{TS-$\&$R}$ and $\runa{TS-$\&$L}$. By the premises of these rules we have that $\Delta' \vdash P_k :: a\!:\!A_k$ and $\Delta'',a : A_k\vdash R :: c\!:\!C$, and so it follows by $\runa{TS-cut}$ that $\Delta',\Delta''\vdash \newvar{a}{(P_k \mid R) :: c\!:\!C}$.
        
% %     %
    
% %     \item $\Delta'\vdash a.k; R :: a\!:\!\oplus\{l : A_l\}_{l\in L}$ and $\Delta'',a : \oplus\{l : A_l\}_{l\in L}\vdash a.\texttt{case}\{l\Rightarrow P_l\}_{l\in L} :: c\!:\!C$ by $\runa{TS-$\oplus$R}$ and $\runa{TS-$\oplus$L}$. By the premises of these rules we have that $\Delta'\vdash R :: a\!:\!A_k$ and $\Delta'',a : A_k\vdash P_k :: c\!:\!C$, and so it follows by $\runa{TS-cut}$ that $\Delta',\Delta''\vdash \newvar{a}{(R \mid P_k)} :: c\!:\!C$.
    
% % \end{enumerate}

% % %

% % \item[$\runa{R-def}$] Assume $P$ reduces by $\runa{R-def}$ then $P = b \leftarrow f \leftarrow \widetilde{d}$ and $(\widetilde{c}:\widetilde{B}\vdash f = P' :: a\!:\!A) \in \Sigma$, such that $Q = P'[a\mapsto b,\widetilde{c}\mapsto\widetilde{d}]$. For $P$ to be well-typed it must be part of a larger process $\widetilde{d}:\widetilde{B},\Delta'\vdash \newvar{b}{(P \mid R)} :: c\!:\!C$ typed with $\runa{TS-def}$ such that $\Delta',b:A\vdash R :: c\!:\!C$. By renaming we have that $\widetilde{d}:\widetilde{B}\vdash P'[a\mapsto b,\widetilde{c}\mapsto\widetilde{d}] :: b\!:\!B$ and so by $\runa{TS-cut}$ we have that $\widetilde{d}:\widetilde{B},\Delta'\vdash \newvar{b}{(Q \mid R)} :: c\!:\!C$.

% % %

% % \item[$\runa{R-res}$] Assume that $P$ reduces by $\runa{R-res}$ then we have that $P \equiv \newvar{a}{P'}$ for some name $a$ such that $P' \longrightarrow Q'$. Then $P$ must be typed either with $\runa{TS-cut}$ or $\runa{TS-def}$ and so $P' \equiv R' \mid R''$ yielding two cases
% % \begin{enumerate}
% %     \item $\Delta'\vdash R' :: a\!:\!A$ such that $\Delta'',a:A\vdash R'' :: c\!:\!C$ and $\Delta',\Delta''\vdash \newvar{a}{P'}::c\!:\!C$. Either $R' \mid R''$ reduces by $\runa{R-par}$, $\runa{R-comm}$, $\runa{R-choice}$ or $\runa{R-struct}$. The first three cases are covered by the clauses for the corresponding rules, and the last case holds by induction as typability is closed under structural congruence.
    
% %     \item $R' = a \leftarrow f \leftarrow \widetilde{b}$ and $(\widetilde{e} : \widetilde{B}\vdash f = R :: g\!:\!A) \in \Sigma$ such that $\Delta' = \widetilde{b}:\widetilde{B}$, $\Delta'',a:A\vdash R'' :: c\!:\!C$ and $\Delta',\Delta''\vdash \newvar{a}{P'}::c\!:\!C$. Then either $R' \mid R''$ reduces by $\runa{R-par}$ or $\runa{R-struct}$. The first case is covered by the clause for $\runa{R-par}$, and the last case holds by induction as typability is closed under structural congruence.
% % \end{enumerate}

% % %

% % \item[$\runa{R-par}$] Assume that $P$ reduces by $\runa{R-par}$ then we have that $P \equiv P' \mid P''$ such that $P' \longrightarrow Q'$. For $P$ to be well-typed, it must be part of a larger well-typed process $\newvar{a}{(P'\mid P'')}$ typed with either $\runa{TS-cut}$ or $\runa{TS-def}$ such that either
% % \begin{enumerate}
% %     \item $\Delta'\vdash P' :: a\!:\!A$ such that $\Delta'',a:A\vdash P'' :: c\!:\!C$ and $\Delta',\Delta''\vdash \newvar{a}{(P'\mid P'')}::c\!:\!C$. Then by induction we have that $\Delta'\vdash Q' :: a\!:\!A$ and so it follows that $\Delta',\Delta''\vdash \newvar{a}{(Q' \mid P'')}::c\!:\!C$
    
% %     \item $P' = a \leftarrow f \leftarrow \widetilde{b}$ and $(\widetilde{e} : \widetilde{B}\vdash f = R :: g\!:\!A) \in \Sigma$ such that $\Delta' = \widetilde{b}:\widetilde{B}$, $\Delta'',a:A\vdash P'' :: c\!:\!C$ and $\widetilde{b}:\widetilde{B},\Delta''\vdash \newvar{a}{P' \mid P''}::c\!:\!C$. Then it must be that $P'$ reduces to $Q'$ by $\runa{TS-def}$ such that $Q' = R[g\mapsto a,\widetilde{e}\mapsto\widetilde{b}]$. By renaming $\widetilde{e} : \widetilde{B}\vdash R :: g\!:\!A$ implies $\widetilde{b} : \widetilde{B}\vdash Q' :: a\!:\!A$ such that $\widetilde{b}:\widetilde{B},\Delta''\vdash \newvar{a}{(Q' \mid P''):: c\!:\!C}$ by $\runa{T-cut}$.
% % \end{enumerate}

% % %%%
% % %%
% % %%
% % %%%

% % %when they contain no named processes, for $P$ to be well-typed, $P$ must be a subprocess of a larger well-typed process $R \equiv \newvar{a}{\newvar{b}{P}} \equiv \newvar{a}{(\outputch{a}{d}{}{P'} \mid \newvar{b}{(\inputch{b}{v}{}{P''} \mid b \leftarrow a}))}$ such that $\Delta',\Delta''\vdash R :: c\!:\!C$. Then from the premises of $\runa{TS-cut}$, we have that $\Delta'',a:A\vdash \outputch{a}{d}{}{P'} ::c\!:\!C$ and (by $\runa{TS-cut}$ again) $\Delta'\vdash \newvar{b}{(\inputch{b}{v}{}{P''} \mid b \leftarrow a}) :: a\!:\!A$ such that $\Delta' \vdash \inputch{b}{v}{}{P''} :: b\!:\!A$ by $\runa{TS-$\multimap$R}$ and $b : A\vdash b \leftarrow a :: a\!:\!A$ by $\runa{TS-id}$. The full reduced process is then $\newvar{a}{\newvar{b}{(P' \mid P''[v\mapsto d])}}$

% % %
% % %%%%%%%%%%
% % %

% % % \item[$\runa{R-res}$] Assume that $P$ reduces by $\runa{R-res}$. Then for $P$ to be well-typed, $P$ must be typed by either $\runa{TS-cut}$ or $\runa{TS-def}$. We consider the cases separately
% % % \begin{description}
% % % \item[$\runa{TS-cut}$] We have that $P$ is of the form $\newvar{a}{(P'\mid P'')}$ such that $\Delta' \vdash P' :: a\!:\!A$, $\Delta'', a : A\vdash P'' :: c\!:\!C$ and $\Delta',\Delta'' \vdash \newvar{a}{(P'\mid P'')} :: c\!:\!C$. By $\runa{R-res}$ we have that $P' \mid P''$ must reduce, for which several rules apply
% % % \begin{description}
% % % \item[$\runa{R-comm}$] If we reduce the parallel composition by $\runa{R-comm}$ then $P' \mid P'' \equiv \inputch{a}{v}{}{R'} \mid \outputch{a}{b}{}{R''}$ for some name $b$ and processes $R'$ and $R''$, such that $\inputch{a}{v}{}{R'} \mid \outputch{a}{b}{}{R''} \longrightarrow R'[v\mapsto b] \mid R''$. We have two cases
% % % \begin{enumerate}
% % %     \item $\Delta' \vdash \inputch{a}{v}{}{R'} :: a\!:\!A' \multimap A''$ and $\Delta_3,a : A'\multimap A'', b : A' \vdash \outputch{a}{b}{}{R''} :: c\!:\!C$ by $\runa{TS-$\multimap$R}$ and $\runa{TS-$\multimap$L}$ such that $\Delta'' = \Delta_3,b:A'$. By the premises to these rules we have that $\Delta',v : A' \vdash R' :: a\!:\!A''$ and $\Delta_3,a:A''\vdash R'' :: c\!:\!C$. This implies $\Delta',b : A'\vdash R'[v\mapsto b] :: a\!:\!A''$, and so by $\runa{TS-cut}$ it follows that $(\Delta',b : A'),\Delta_3\vdash \newvar{a}{(R'[v\mapsto b] \mid R'') :: c\!:\!C}$ and $\Delta = (\Delta',b : A'),\Delta_3$.
    
% % %     %
    
% % %     \item $\Delta_3,b:A' \vdash \outputch{a}{b}{}{R''} :: a\!:\!A'\otimes A''$ and $\Delta'',a : A'\otimes A''\vdash \inputch{a}{v}{}{R'} :: c\!:\!C$ by $\runa{TS-$\otimes$R}$ and $\runa{TS-$\otimes$L}$ such that $\Delta' = \Delta_3,b:A'$. By the premises to these rules we have that $\Delta_3\vdash R'' :: a\!:\!A''$ and $\Delta'',a:A'',v:A'\vdash R' :: c\!:\!C$. This implies $\Delta'',a:A'',b:A'\vdash R'[v\mapsto b] :: c\!:\!C$, and so by $\runa{TS-cut}$ it follows that $\Delta_3,(\Delta'',b : A')\vdash \newvar{a}{(R'' \mid R'[v\mapsto b])} :: c\!:\!C$ and $\Delta = \Delta_3,(\Delta'',b : A')$.
% % % \end{enumerate}

% % % \item[$\runa{R-choice}$] If we reduce the parallel composition by $\runa{R-choice}$ then $P' \mid P'' \equiv a.\texttt{case}\{ l \Rightarrow P_l \}_{l\in L} \mid a.k; R$ for some label and set of labels $k$ and $L$, such that $k\in L$ and $a.\texttt{case}\{ l \Rightarrow P_l \}_{l\in L} \mid a.k; R \longrightarrow P_k \mid R$. We have two cases
% % % \begin{enumerate}
% % %     \item $\Delta'\vdash a.\texttt{case}\{l \Rightarrow P_l\}_{l\in L} :: a\!:\!\&\{l : A_l\}_{l\in L}$ and $\Delta'', a : \&\{l : A_l\}_{l\in L}\vdash a.k; R :: c\!:\!C$ by $\runa{TS-$\&$R}$ and $\runa{TS-$\&$L}$. By the premises of these rules we have that $\Delta' \vdash P_k :: a\!:\!A_k$ and $\Delta'',a : A_k\vdash R :: c\!:\!C$, and so it follows by $\runa{TS-cut}$ that $\Delta',\Delta''\vdash \newvar{a}{(P_k \mid R) :: c\!:\!C}$.
        
% % %     %
    
% % %     \item $\Delta'\vdash a.k; R :: a\!:\!\oplus\{l : A_l\}_{l\in L}$ and $\Delta'',a : \oplus\{l : A_l\}_{l\in L}\vdash a.\texttt{case}\{l\Rightarrow P_l\}_{l\in L} :: c\!:\!C$ by $\runa{TS-$\oplus$R}$ and $\runa{TS-$\oplus$L}$. By the premises of these rules we have that $\Delta'\vdash R :: a\!:\!A_k$ and $\Delta'',a : A_k\vdash P_k :: c\!:\!C$, and so it follows by $\runa{TS-cut}$ that $\Delta',\Delta''\vdash \newvar{a}{(R \mid P_k)} :: c\!:\!C$.
    
% % % \end{enumerate}

% % % \item[$\runa{R-id-1}$]
% % % \item[$\runa{R-id-2}$]
% % % \item[$\runa{R-par}$] If we reduce the parallel composition by $\runa{R-par}$ then $P' \longrightarrow Q'$. Here we can apply induction, as $P'$ cannot be typed as $\Delta' \vdash P' :: a\!:\!A$ and reduce unless it is prefixed by a tick or is wrapped with a restriction (or is structurally congruent to such a process by $\runa{R-struct}$). And so, it follows that $\Delta' \vdash Q' :: a\!:\!A$, such that $\Delta',\Delta'' \vdash \newvar{a}{(Q'\mid P'')} :: c\!:\!C$.
% % % \item[$\runa{R-struct}$] todo: induction (with R-par after).
% % % \end{description}
% % % \item[$\runa{TS-def}$] We have that $P$ is of the form $\newvar{a}{(a\leftarrow f \leftarrow \widetilde{b} \mid P')}$ such that $(\widetilde{d} : \widetilde{B}\vdash f = P :: g\!:\!A) \in \Sigma$, $\Delta',a : A \vdash P' :: c\!:\!C$ and $\Delta',\widetilde{b} : \widetilde{B}\vdash \newvar{a}{(a\leftarrow f \leftarrow \widetilde{b} \mid P') :: c\!:\!C}$. By $\runa{R-res}$ we have that $a\leftarrow f \leftarrow \widetilde{b} \mid P'$ must reduce, for which $\runa{R-par}$ and $\runa{R-struct}$ apply. Note that the parallel composition cannot reduce by $\runa{R-par}$, as  does not  several rules apply.
% % % \begin{description}
% % % \item[$\runa{R-par}$] todo: R-def --> can type with R-cut after.
% % % \item[$\runa{R-struct}$] todo: induction (with R-par after). 
% % % \end{description}
% % % \end{description}


% % \item[$\runa{R-struct}$] Assume that $P$ reduces by $\runa{R-struct}$. Then $P \equiv P'$, $P' \longrightarrow Q'$ and $Q' \equiv Q$. As typability is closed under structural congruence and $\Delta \vdash P :: c\!:\!C$ it follows that $\Delta \vdash P' :: c\!:\!C$. By induction this implies $\Delta \vdash Q' :: c\!:\!C$, and as $Q' \equiv Q$ we have that $\Delta\vdash Q :: c\!:\!C$.
% % \end{description}
% % \end{proof}
% % \end{theorem}

% %

% \begin{theorem}
% If $\Delta\vdash P :: a\!:\!A$ and $P$ reduces to $Q$ by the tick-last strategy using $n$ productive time reductions then $\text{time}(A) \geq n$.
% \begin{proof} By induction on the size of n.
% \begin{description}
% \item[$n = 0$] For any process $P$, context $\Delta$ and session $a:A$ such that $\Delta\vdash P :: a\!:\!A$, we have that $\text{time}(A)\geq 0$, and so this is obtained trivially.

% \item[$n+1$] Assume that $\Delta\vdash P :: a\!:\!A$ and $P$ reduces to $Q$ by the tick-last strategy with $n+1$ productive time reductions. Then we have that $P \leadsto^* P'$ and $P' \Longrightarrow^{-1} Q'$ with $P' \neq Q'$ and $P'\!\not\!\leadsto$ such that $Q'$ reduces to $Q$ by the tick-last strategy with $n$ productive time reductions. By Theorem \ref{theorem:sr} $\Delta\vdash P' :: a\!:\!A$ follows from $\Delta\vdash P :: a\!:\!A$. By Lemma \ref{lemma:timered} $\Delta\vdash P' :: a\!:\!A$ and $P' \Longrightarrow^{-1} Q'$ with $P' \neq Q'$ implies there exists $\hat{\Delta'}[\Delta']=\Delta$ and $\hat{A'}[A']=A$ such that $\hat{\Delta'}[[\Delta']^{-1}_L]\vdash Q' :: a\!:\!\hat{A'}[[A']^{-1}_R]$ and by Lemma \ref{lemma:timegeq} $\text{time}(\hat{A'}[A']) - 1 \geq \text{time}(\hat{A'}[[A']^{-1}_R])$. Then by the induction hypothesis, $\text{time}(\hat{A'}[[A']^{-1}_R]) \geq n$. It follows from $\hat{A'}[A']=A$ that $\text{time}(A) \geq n + 1$.

% \end{description}

% %We do not consider left rules, i.e. those that consume rather than provide sessions, as they hold trivially by induction and the fact that the span of inputs, outputs, external choices and internal choices is either $0$ or the span of their continuations, depending on the environment.
% % \begin{description}
% % %\item[$\runa{TS-$\mathbf{1}$L}$] 

% % \item[$\runa{TS-$\mathbf{1}$R}$] The span of inaction $\mathbf{0}$ is $0$, and by $\runa{TS-$\mathbf{1}R$}$, $\mathbf{0}$ provides a session of type $\mathbf{1}$. We have that $\text{time}(\mathbf{1}) = 0$.

% % %\item[$\runa{TS-$\otimes$L}$]

% % \item[$\runa{TS-$\otimes$R}$] The span of a synchronous output $\outputch{a}{v}{}{P}$ is 0 when it is not in parallel with a corresponding input, and equal to the span of its continuation $P$ when it is. By $\runa{TS-$\otimes$R}$, $\Delta,v : A\vdash\outputch{a}{v}{}{P}:: a\!:\! A\otimes B$ such that $\Delta\vdash P :: a\!:\!B$, and so by induction, $\text{time}(B)$ is an upper bound on the span of $P$. We have that $\text{time}(A \otimes B) = \text{time}(B)$, which is an upper bound in either case.

% % %\item[$\runa{TS-$\multimap$L}$]

% % \item[$\runa{TS-$\multimap$R}$] The span of a synchronous input $\inputch{a}{v}{}{P}$ is 0 when it is not in parallel with a corresponding output, and equal to the span of its continuation $P$ when it is. By $\runa{TS-$\multimap$R}$, $\Delta\vdash\inputch{a}{v}{}{P}:: a\!:\! A\multimap B$ such that $\Delta,v:A\vdash P :: a\!:\!B$, and so by induction, $\text{time}(B)$ is an upper bound on the span of $P$. We have that $\text{time}(A \multimap B) = \text{time}(B)$, which is an upper bound in either case.

% % \item[$\runa{TS-cut}$] The span of a parallel composition depends on the parallelism of reductions of ticks with respect to synchronizations in its two parallel subprocesses, i.e. it depends on the protocols of channels used in the parallel composition. By $\runa{TS-cut}$ we have that $\Delta',\Delta''\vdash \newvar{a}{(P'\mid P'')::c\!:\!C}$ such that $\Delta'\vdash P' :: a\!:\!A$ and $\Delta'',a:A\vdash P'' :: c\!:\!C$. As $P'$ and $P''$ only communicate using one channel (i.e. $a$ when no identity constructs are used, or some fresh name otherwise), the parallelism entirely depends on the session type $A$. By induction $\text{time}(A)$ is an upper bound on the span of $P'$ when session $a:A$ is used. Similarly, $\text{time}(C)$ is a bound on the span of $P''$ including when session $c:C$ is consumed. By Lemma \ref{}, $a:A\vdash P'' :: c\!:\!C$ implies $\text{time}(C) \geq \text{time}(A)$, and so $\text{time}(C)$ is an upper bound on the span of $\newvar{a}{(P'\mid P'')}$.

% % \item[$\runa{TS-id}$] The span of an identity construct $a \leftarrow b$ is 0, as it has no continuation, and so for $b:A\vdash a\leftarrow b :: a\!:\!A$ it trivially holds that $\text{time}(A)$ is an upper bound on the span, as $\text{time}(A) \geq 0$.

% % %\item[$\runa{TS-$\oplus$L}$]

% % \item[$\runa{TS-$\oplus$R}$] The span of an internal choice $a.k; P$ is 0 when it is not in parallel with a corresponding external choice, and equal to the span of its continuation $P$ when it is. By $\runa{TS-$\oplus$R}$, $\Delta\vdash a.k; P ::a\!:\!\oplus\{l:A_l\}_{l\in L}$ such that $k \in L$ and $\Delta\vdash P ::a\!:\!A_k$, and so by induction, $\text{time}(A_k)$ is an upper bound on the span of $P$. We have that $\text{time}(\oplus\{l:A_l\}_{l\in L}) = \text{max}(\text{time}(A_l) \mid l \in L)$. As $k \in L$, $\text{max}(\text{time}(A_l) \mid l \in L)\geq \text{time}(A_k)$, which is an upper bound in either case.

% % %\item[$\runa{TS-$\&$L}$]

% % \item[$\runa{TS-$\&$R}$] The span of an external choice $a.\texttt{case}\{l \Rightarrow P_l\}_{l\in L}$ is 0 when it is not in parallel with a corresponding internal choice, and equal to the maximum span amongst its possible continuations $P_l$ for $l\in L$ when it is. By $\runa{TS-$\&$R}$, $\Delta\vdash a.\texttt{case}\{l \Rightarrow P_l\}_{l\in L} ::a\!:\!\&\{l:A_l\}_{l\in L}$ such that for all $l \in L$ $\Delta\vdash P_l ::a\!:\!A_l$, and so by induction, $\text{time}(A_l)$ is an upper bound on the span of $P_l$. We have that $\text{time}(\&\{l:A_l\}_{l\in L}) = \text{max}(\text{time}(A_l) \mid l \in L)$, which is an upper bound in either case.

% % \item[$\runa{TS-def}$]

% % \item[$\runa{TS-$\ocircle$LR'}$]

% % \item[$\runa{TS-$\ocircle$LR}$]

% % \item[$\runa{TS-$\lozenge$L}$]

% % \item[$\runa{TS-$\lozenge$R}$]

% % \item[$\runa{TS-$\Box$L}$]

% % \item[$\runa{TS-$\Box$R}$]

% % \end{description}
% \end{proof}
% \end{theorem}
\end{document}
%%% Local Variables:
%%% mode: latex
%%% TeX-master: t
%%% End:
