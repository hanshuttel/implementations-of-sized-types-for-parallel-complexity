%\documentclass[fleqn, 11pt]{article}
\documentclass[11pt,a4paper,oldfontcommands]{memoir}
%\usepackage[danish]{babel}
%\usepackage{boisik}
%\usepackage[T1]{fontenc}
\usepackage{mathtools}
\usepackage{stackengine}
\newcommand\letvdash[1]{\mathrel{
  \stackengine{1.2ex}{\vdash}{\;\;\scriptscriptstyle#1}{O}{c}{F}{T}{L}}}
\stackMath
%\usepackage{stix}
\usepackage{upgreek}%alpha symbol
\usepackage{adjmulticol}
\usepackage[utf8]{inputenc}
% xspace sørger for rigtigt mellemrum efter en kommando-anvendelse
\setlength\parindent{0pt}
\usepackage{xspace}
\usepackage{comment}
\usepackage{multicol}
\usepackage{wasysym}
\usepackage{float}
\usepackage{array}

% Memoir stuff
\usepackage[T1]{fontenc}
\usepackage{microtype}
\usepackage[dvips]{graphicx}
\usepackage[dvipsnames]{xcolor}
\usepackage{times}

\usepackage{listings}
\usepackage{tikz}
\usepackage{pgfplots}
\usepackage{mathrsfs}
\usepgfplotslibrary{external}
\usetikzlibrary{external}
\tikzexternalize[prefix=tikz/]

\usepackage[
breaklinks=true,colorlinks=true,
%linkcolor=blue,urlcolor=blue,citecolor=blue,% PDF VIEW
linkcolor=black,urlcolor=black,citecolor=black,% PRINT
bookmarks=true,bookmarksopenlevel=2]{hyperref}

\usepackage{geometry}
% PDF VIEW
% \geometry{total={210mm,297mm},
% left=25mm,right=25mm,%
% bindingoffset=0mm, top=25mm,bottom=25mm}
% PRINT
\geometry{total={210mm,297mm},
left=20mm,right=20mm,
bindingoffset=10mm, top=25mm,bottom=25mm}

\OnehalfSpacing
%\linespread{1.3}

\usepackage{bm}

%%% CHAPTER'S STYLE
\chapterstyle{bianchi}
%\chapterstyle{ger}
%\chapterstyle{madsen}
%\chapterstyle{ell}
%%% STYLE OF SECTIONS, SUBSECTIONS, AND SUBSUBSECTIONS
\setsecheadstyle{\Large\bfseries\sffamily\raggedright}
\setsubsecheadstyle{\large\bfseries\sffamily\raggedright}
\setsubsubsecheadstyle{\bfseries\sffamily\raggedright}


%%% STYLE OF PAGES NUMBERING
%\pagestyle{companion}\nouppercaseheads 
%\pagestyle{headings}
%\pagestyle{Ruled}
\pagestyle{plain}
\makepagestyle{plain}
\makeevenfoot{plain}{\thepage}{}{}
\makeoddfoot{plain}{}{}{\thepage}
\makeevenhead{plain}{}{}{}
\makeoddhead{plain}{}{}{}


\maxsecnumdepth{subsection} % chapters, sections, and subsections are numbered
\maxtocdepth{subsection} % chapters, sections, and subsections are in the Table of Contents

%Memoir stuff end

%\usepackage{algorithm}
%\usepackage{algpseudocode}
\usepackage{algorithm2e}

%\usepackage{breqn}

% amsmath giver align-environments og meget mere
\usepackage{amsmath}
\newcommand\numberthis{\addtocounter{equation}{1}\tag{\theequation}}
\usepackage{amsthm}

\usepackage{stmaryrd}

\theoremstyle{proposition}
\newtheorem{prop}{Proposition}

\theoremstyle{proposition}
\newtheorem{conj}{Conjecture}[section]


\theoremstyle{definition}
\newtheorem{definition}{Definition}[section]
\newtheorem{exmp}{Example}[section]
\newtheorem{theorem}{Theorem}
\newtheorem{corollary}{Corollary}[theorem]

\theoremstyle{plain}
\newtheorem{lemma}[theorem]{Lemma}

\usepackage{framed}
\theoremstyle{remark}
\newtheorem*{remark}{Remark}

\theoremstyle{definition}
\newtheorem{defi/}{Definition}[section]

\newenvironment{defi}
  {\renewcommand{\qedsymbol}{$\triangleleft$}%
   \pushQED{\qed}\begin{defi/}}
  {\popQED\end{defi/}}

%\theoremstyle{definition}
\newtheorem{examp/}{Example}[section]
\newenvironment{examp}
  {\renewcommand{\qedsymbol}{$\triangleright$}%
   \pushQED{\qed}\begin{examp/}}
  {\popQED\end{examp/}}



% \addtolength{\oddsidemargin}{-.475in}
% \addtolength{\evensidemargin}{-.475in}
% \addtolength{\textwidth}{1.05in}

% amssymb giver mulighed for "fede" symboler i mathmode til N, Z mm.

\usepackage{amssymb}

\usepackage{enumerate}
\usepackage{ebproof}
% Fede mængder

\newcommand{\skat}[1]{\textbf{#1}\xspace}
\newcommand{\EnvV}{\skat{EnvV}}
\newcommand{\Var}{\skat{Var}}
\newcommand{\Store}{\skat{Store}}
\newcommand{\Loc}{\skat{Loc}}

% De hele tal

\newcommand{\Z}{\ensuremath{\mathbb{Z}}}

% Inferensregler

\newcommand{\condinfrule}[3]
          {\parbox{5.5cm}{$$ {\frac{#1}{#2}}{\qquad
            #3} \hfill  $$}}

\newcommand{\infrule}[2]
          {\parbox{4.5cm}{$$ \frac{#1}{#2}\hspace{.5cm}$$}}

% Regelnavne
           
\newcommand{\runa}[1]{\mbox{\textsc{\protect{(#1})}}}
\newcommand{\runatt}[2]{$[{\mbox{\textsc{#1}}}_{\mbox{\textsc{\small
        #2}}}]$\xspace}

% Pile

\newcommand{\ra}[1][\relax]{\ensuremath \rightarrow_{#1}}
\newcommand{\lra}{\longrightarrow}
\newcommand{\Ra}{\Rightarrow}
\newcommand{\pra}{\ensuremath \rightharpoonup }

% Kantede parenteser

\newcommand{\lag}{\langle}
\newcommand{\rag}{\rangle}
\newcommand{\conf}[1]{\ensuremath{\lag #1 \rag}}

% Mængder

\newcommand{\setof}[2]{\ensuremath{\{ #1 \mid #2 \}}}
\newcommand{\set}[1]{\ensuremath{\{ #1 \}}}

% Kommandoer i Bims

\newcommand{\skib}{\texttt{skip}}
%\newcommand{\ifthenelse}[3]{\texttt{if}\; #1 \; \texttt{then}\; #2 \;
%  \texttt{else}\; #3}
%\newcommand{\whiledo}[2]{\texttt{while}\; #1 \; \texttt{do}\; #2}

% Neu commands
\newcommand{\hole}{(\!\mid\mid\!)}
\newcommand{\nehole}[1]{\mathord{(\!\mid\!\!#1\!\!\mid\!)}}
\newcommand{\prerep}{[^{m}\!\!\//_{\!\!D}]}
\newcommand{\replace}[1]{\lbrbrak #1 \rbrbrak}
\newcommand{\cursor}[1]{\mathord{
    \lBrack\mspace{1mu}#1\mspace{1mu}\rBrack
}}
\newcommand{\breakpoint}[1]{
\mathord{
    \lAngle\mspace{1mu}#1\mspace{1mu}\rAngle
}}
\newcommand{\wellformed}[1]{\letvdash{wf}#1}
\newcommand{\complete}[1]{\letvdash{c}#1}
\newcommand{\cursorexc}[1]{\letvdash{ce}#1}
\newcommand{\nodevalid}[1]{\letvdash{nv}#1}
\newcommand{\cursorctx}[1]{C\!\left[#1\right]}
\newcommand{\cursorctxhole}{\left[\cdot\right]}
\newcommand{\recursion}[2]{#1\!\left(#2\right)}
% Next

\newcommand{\nexte}{\textrm{next}\xspace}

\newcommand{\editelig}[1]{\mathcal{E}#1}
\newcommand{\noteditelig}[1]{\overline{\mathcal{E}#1}}

\newcommand{\consistent}[2]{#1 \sim #2}

% Omvendt \vdash
\usepackage{graphicx}

\makeatletter
\providecommand*{\dashv}{%
  \mathrel{%
    \mathpalette\@dashv\vdash
  }%
}
\newcommand*{\@dashv}[2]{%
  \reflectbox{$\m@th#1#2$}%
}

\newcommand{\f}{\mkern-2mu f\mkern-3mu}%fix f i mathmode

\newcommand{\cmdchild}[1]{\texttt{child}\; #1}
\newcommand{\cmdparent}{\texttt{parent}}
\newcommand{\cmdlambda}[1]{\texttt{lambda}\; #1}
\newcommand{\cmdhole}{\texttt{hole}}
\newcommand{\cmdrec}[1]{\texttt{rec}\; #1.}
\newcommand{\cmdapp}{\texttt{app}}
\newcommand{\cmdbreak}{\texttt{break}}


\newcommand{\condexp}[2]{#1 \Rightarrow #2}
\newcommand{\condexpto}[3]{#1 \Rightarrow #2 \vert #3}

\definecolor{codegreen}{rgb}{0,0.6,0}
\definecolor{codegray}{rgb}{0.5,0.5,0.5}
\definecolor{codepurple}{rgb}{0.58,0,0.82}
\definecolor{backcolour}{rgb}{0.95,0.95,0.92}

\lstdefinestyle{mystyle}{
    backgroundcolor=\color{backcolour},   
    commentstyle=\color{codegreen},
    keywordstyle=\color{magenta},
    numberstyle=\tiny\color{codegray},
    stringstyle=\color{codepurple},
    basicstyle=\ttfamily\footnotesize,
    breakatwhitespace=false,         
    breaklines=true,                 
    captionpos=b,                    
    keepspaces=true,                 
    numbers=left,                    
    numbersep=5pt,                  
    showspaces=false,                
    showstringspaces=false,
    showtabs=false,                  
    tabsize=2
}

\lstset{style=mystyle}

\makeatother

% HER KOMMER TITLEN

\usepackage{authblk}

\title{Implementations of Sized Types for Parallel Complexity of Message-passing Processes}

%\usepackage{newpxmath} % math font is Palatino compatible

%\author{Herrmann, Thomas. Lauridsen, Mikkel Korup}

%\def\email#1{{\tt#1}}

\author{Thomas Herrmann}
\author{Mikkel Korup Lauridsen}
\affil{\{therrm17,mkla17\}@student.aau.dk}
%\affil{mkla17@student.aau.dk}
\affil{Department of Computer Science, Aalborg University}
%\affil{Selma Lagerlöfs Vej 300}
\affil{Aalborg, Denmark}
\date{June 17, 2022}
%\affil{Denmark}
%\email{therrm17@student.aau.dk}
%\authornotemark[1]
%\affiliation{%
%  \institution{Department of Computer Science, Aalborg University}
%  \streetaddress{Selma Lagerlöfs Vej 300}
%  \city{Aalborg}
%  \country{Denmark}}
%
%\author[2]{Lauridsen, Mikkel Korup}
%\email{mkla17@student.aau.dk}
%\authornotemark[1]
%\affiliation{%
%  \institution{Department of Computer Science, Aalborg University}
%  \streetaddress{Selma Lagerlöfs Vej 300}
%  \city{Aalborg}
%  \country{Denmark}}

\newcommand*\bang[1]{! #1}
%\newcommand{\ifthenelse}[3]{\texttt{if}\; #1 \texttt{ then}\; #2 \texttt{ else}\; #3}
\newcommand{\match}[4]{\texttt{match}\; #1\; \{ 0 \mapsto #2;\; s(#3) \mapsto #4 \}}

\newcommand{\outputch}[4]{\overline{#1}\langle #2\rangle^{#3}\!.#4}
\newcommand{\inputch}[4]{#1(#2)^{#3}.#4}

\newcommand{\asyncoutputch}[3]{\overline{#1}\!\langle #2\rangle^{#3}}
\newcommand{\asyncinputch}[3]{#1\!\left( #2\right)^{#3}}
\newcommand{\asyncrepinputch}[3]{\bang{\asyncinputch{#1}{#2}{#3}}}


\newcommand{\splitatcommas}[1]{%
  \begingroup
  \begingroup\lccode`~=`, \lowercase{\endgroup
    \edef~{\mathchar\the\mathcode`, \penalty0 \noexpand\hspace{0pt plus 1em}}%
  }\mathcode`,="8000 #1%
  \endgroup
}



\newcommand{\parcomp}[2]{#1 \mid #2}
\newcommand{\parcompthree}[3]{\parcomp{\parcomp{#1}{#2}}{#3}}
\newcommand{\parcompfour}[4]{\parcompthree{#1}{#2}{\parcomp{#3}{#4}}}
\newcommand{\parcompseven}[7]{#1 \mid #2 \mid #3 \mid #4 \mid #5 \mid #6 \mid #7}

\newcommand{\uparcomp}[2]{\parcomp{#1}{#2}}
\newcommand{\uparcompthree}[3]{\uparcomp{\uparcomp{#1}{#2}}{#3}}
\newcommand{\uparcompfour}[4]{\uparcompthree{#1}{#2}{\uparcomp{#3}{#4}}}
\newcommand{\uparcompfive}[5]{\uparcompfour{#1}{#2}{#3}{\uparcomp{#4}{#5}}}

\newcommand{\newvar}[2]{(\nu #1) #2}
\newcommand{\newvarU}[1]{\left(\nu #1\right)}

\newcommand{\nil}{\mathbf{0}}

\newcommand{\freemodule}[0]{\mathbb{Z}[i_1,\dots,i_n]}

\newcommand{\true}{\textit{true}}
\newcommand{\false}{\textit{false}}

\newcommand{\succeeds}{\mathbf{c}}

\newcommand{\succc}[1]{s(#1)}

\newcommand{\tick}[1]{\texttt{tick}.#1}


\newcommand{\dasfwr}[1]{[#1]^{-1}_R}


\newcommand{\subst}[2]{#1\!\left[#2\right]}

\newcommand{\substi}[2]{\{#1/#2\}}

% Type stuff
\newcommand{\withusage}[2]{#1/#2}
\newcommand{\withtype}[2]{#1:#2}
\newcommand{\channeltype}[1]{\texttt{ch}(#1)}
\newcommand{\channeltypeusage}[2]{\withusage{\channeltype{#1}}{#2}}
\newcommand{\inchanneltypeS}[2]{\texttt{in}_{#1}(#2)}
\newcommand{\outchanneltypeS}[2]{\texttt{out}_{#1}(#2)}
\newcommand{\channeltypeS}[2]{\texttt{ch}_{#1}(#2)}
\newcommand{\tparcomp}[2]{\parcomp{#1}{#2}}
\newcommand{\withdelay}[2]{\uparrow^{#1}\!\!#2}
\newcommand{\usagepref}[3]{#1^{#2}_{#3}}
\newcommand{\inusagesym}[0]{\texttt{In}}
\newcommand{\outusagesym}[0]{\texttt{Out}}
\newcommand{\inusagepref}[2]{\usagepref{\inusagesym}{#1}{#2}}
\newcommand{\outusagepref}[2]{\usagepref{\outusagesym}{#1}{#2}}
\newcommand{\repinusagepref}[2]{\bang{\usagepref{\inusagesym}{#1}{#2}}}
\newcommand{\usagerep}[1]{\;*\!#1}
\newcommand{\errres}[0]{\textbf{\texttt{err}}}
\newcommand{\comlabel}[0]{\textbf{com}}

\newcommand{\withcomplex}[2]{#1 \triangleleft #2}

\newcommand{\typenat}[0]{\texttt{Nat}}
\newcommand{\typechanusage}[2]{\withusage{\channeltype{#1}}{#2}}
\newcommand{\kinterval}[2]{\left[#1,#2\right]}
\newcommand{\kintervalsingle}[1]{\left[#1\right]}
\newcommand{\natinterval}[2]{\typenat\!\kinterval{#1}{#2}}
\newcommand{\natintervalsingle}[1]{\typenat\!\left[#1\right]}

\newcommand{\encoding}[1]{\left[\!\left[#1\right]\!\right]}


\newcommand\defeq{\stackrel{\mathclap{\normalfont\tiny\mbox{def}}}{=}}


\newcommand{\servt}[5]{\forall_{#1} #2.\texttt{serv}^{#3}_{#4}(#5)}
\newcommand{\chant}[3]{\texttt{ch}^{#1}_{#2}(#3)}

\newcommand{\servU}[4]{\withusage{\forall #1.\texttt{serv}^{#2}(#3)}{#4}}

\newcommand{\reliableU}[1]{#1\;\text{reliable}\;}
\newcommand{\reliableT}[1]{#1\;\text{reliable}\;}

%\newcommand{\susume}[4]{\langle #1 \rangle^{#2;#3}_{#4}} % old
\newcommand{\susume}[4]{\downarrow^{#2;#3}_{#4}\!\!(#1)} % new
\newcommand{\susumesim}[2]{\downarrow_{#2}\!\!#1}
\newcommand{\tforward}[4]{\susume{#1}{#2}{#3}{#4}}
\newcommand{\tforwardsim}[2]{\susumesim{#1}{#2}}

\newcommand{\vect}[1]{\texttt{(#1)}}
\newcommand{\evect}[2][]{\vect{#2}_{\!#1}}
\newcommand{\cvect}[2][\varphi]{\vect{#2}_{\!#1}}

\newcommand{\normlinearindex}[3][\mathcal{E}(I)]{#2 + \sum_{\alpha\in #1} #3_\alpha i_\alpha}

\newcommand{\monus}[0]{\dot -}
\newcommand{\monusE}[1][\varphi;\Phi]{\monus_{#1}}

% Fede mængder

\newcommand{\skat}[1]{\textbf{#1}\xspace}
\newcommand{\EnvV}{\skat{EnvV}}
\newcommand{\Var}{\skat{Var}}
\newcommand{\Store}{\skat{Store}}
\newcommand{\Loc}{\skat{Loc}}

% De hele tal

\newcommand{\Z}{\ensuremath{\mathbb{Z}}}

% Inferensregler

\newcommand{\condinfrule}[3]
          {\parbox{5.5cm}{$$ {\frac{#1}{#2}}{\qquad
            #3} \hfill  $$}}

\newcommand{\infrule}[2]
          {\parbox{4.5cm}{$$ \frac{#1}{#2}\hspace{.5cm}$$}}

% Regelnavne
           
\newcommand{\runa}[1]{\mbox{\textsc{\protect{(#1})}}}
\newcommand{\runatt}[2]{$[{\mbox{\textsc{#1}}}_{\mbox{\textsc{\small
        #2}}}]$\xspace}

% Pile

\newcommand{\ra}[1][\relax]{\ensuremath \rightarrow_{#1}}
\newcommand{\lra}{\longrightarrow}
\newcommand{\Ra}{\Rightarrow}
\newcommand{\pra}{\ensuremath \rightharpoonup }

% Kantede parenteser

\newcommand{\lag}{\langle}
\newcommand{\rag}{\rangle}
\newcommand{\conf}[1]{\ensuremath{\lag #1 \rag}}

% Mængder

\newcommand{\setof}[2]{\ensuremath{\{ #1 \mid #2 \}}}
\newcommand{\set}[1]{\ensuremath{\{ #1 \}}}

% Kommandoer i Bims

\newcommand{\skib}{\texttt{skip}}
%\newcommand{\ifthenelse}[3]{\texttt{if}\; #1 \; \texttt{then}\; #2 \;
%  \texttt{else}\; #3}
%\newcommand{\whiledo}[2]{\texttt{while}\; #1 \; \texttt{do}\; #2}

% Neu commands
\newcommand{\hole}{(\!\mid\mid\!)}
\newcommand{\nehole}[1]{\mathord{(\!\mid\!\!#1\!\!\mid\!)}}
\newcommand{\prerep}{[^{m}\!\!\//_{\!\!D}]}
\newcommand{\replace}[1]{\lbrbrak #1 \rbrbrak}
\newcommand{\cursor}[1]{\mathord{
    \lBrack\mspace{1mu}#1\mspace{1mu}\rBrack
}}
\newcommand{\breakpoint}[1]{
\mathord{
    \lAngle\mspace{1mu}#1\mspace{1mu}\rAngle
}}
\newcommand{\wellformed}[1]{\letvdash{wf}#1}
\newcommand{\complete}[1]{\letvdash{c}#1}
\newcommand{\cursorexc}[1]{\letvdash{ce}#1}
\newcommand{\nodevalid}[1]{\letvdash{nv}#1}
\newcommand{\cursorctx}[1]{C\!\left[#1\right]}
\newcommand{\cursorctxhole}{\left[\cdot\right]}
\newcommand{\recursion}[2]{#1\!\left(#2\right)}
% Next

\newcommand{\nexte}{\textrm{next}\xspace}

\newcommand{\editelig}[1]{\mathcal{E}#1}
\newcommand{\noteditelig}[1]{\overline{\mathcal{E}#1}}

\newcommand{\consistent}[2]{#1 \sim #2}

% Omvendt \vdash
\usepackage{graphicx}

\makeatletter
\providecommand*{\dashv}{%
  \mathrel{%
    \mathpalette\@dashv\vdash
  }%
}
\newcommand*{\@dashv}[2]{%
  \reflectbox{$\m@th#1#2$}%
}

\newcommand{\f}{\mkern-2mu f\mkern-3mu}%fix f i mathmode

\newcommand{\cmdchild}[1]{\texttt{child}\; #1}
\newcommand{\cmdparent}{\texttt{parent}}
\newcommand{\cmdlambda}[1]{\texttt{lambda}\; #1}
\newcommand{\cmdhole}{\texttt{hole}}
\newcommand{\cmdrec}[1]{\texttt{rec}\; #1.}
\newcommand{\cmdapp}{\texttt{app}}
\newcommand{\cmdbreak}{\texttt{break}}


\newcommand{\condexp}[2]{#1 \Rightarrow #2}
\newcommand{\condexpto}[3]{#1 \Rightarrow #2 \vert #3}

\definecolor{codegreen}{rgb}{0,0.6,0}
\definecolor{codegray}{rgb}{0.5,0.5,0.5}
\definecolor{codepurple}{rgb}{0.58,0,0.82}
\definecolor{backcolour}{rgb}{0.95,0.95,0.92}

\lstdefinestyle{mystyle}{
    backgroundcolor=\color{backcolour},   
    commentstyle=\color{codegreen},
    keywordstyle=\color{magenta},
    numberstyle=\tiny\color{codegray},
    stringstyle=\color{codepurple},
    basicstyle=\ttfamily\footnotesize,
    breakatwhitespace=false,         
    breaklines=true,                 
    captionpos=b,                    
    keepspaces=true,                 
    numbers=left,                    
    numbersep=5pt,                  
    showspaces=false,                
    showstringspaces=false,
    showtabs=false,                  
    tabsize=2
}

\lstset{style=mystyle}

\makeatother


\begin{document}

\section*{Summary}

Type systems have been studied extensively in the domain of static complexity analysis, to formalize rules that can describe the relationships between a program and its resource use in terms of time and memory (space). Formal methods such as type systems have the advantage that they are typically proved sound. In this thesis, we explore the challenges of implementing both type checking and type inference for a type system for parallel complexity of message-passing processes introduced by Baillot and Ghyselen \cite{BaillotGhyselen2021} that has until now not been implemented. Such implementations of type checking and type inference, paired with corresponding soundness proofs, enable verification or inference of correct complexity bounds, respectively. The type system builds on sized types to express parametric complexity, combined with input/output types to bound synchronizations on channels. \\

After briefly presenting the variant of the pi-calculus considered by Baillot and Ghyselen as well as a more formal description of parallel complexity, we provide an overview of their type system. An important concept is that of \textit{constraint judgements}. These judgements allow us to compare parametric sizes and complexity bounds under a set of known constraints on otherwise unknown sizes. We show both types and type rules of the type system, as well as present examples of typings for processes.\\

Constructing a type checker for the type system by Baillot and Ghyselen introduces a number of challenges. In creating algorithmic type rules for their type system, a notable challenge is maintaining the subject reduction property of the type system. That is, if we are not careful, the reduction relation of $\pi$-calculus processes may not be type-preserving under our modified type rules. We solve this problem by introducing the idea of \textit{combined complexities} consisting of a set of intersecting parametric complexity bounds. As such, a combined complexity represents the maximum of all the complexities contained within. We also define the accompanying function \textit{basis} that keeps a combined complexity as a minimal set by removing complexities that never contribute to the actual bound. Finally, we prove our algorithmic type rules sound, to this effort proving a weaker subject reduction property. We also show how the type checker may be extended with more practical constructs, and present an example of a parallel merge sort encoding that would then be typable using our modified rules.\\

We next show how constraint judgements may be verified. As these judgements are universally quantified over size variables (referred to as index variables), comparison of sizes and complexity bounds is a partial order. We first limit ourselves to constraint judgements over linear functions that may be verified by reduction to integer programs, or alternatively by over-approximation using linear programming. To increase expressiveness, we then show how we can reduce constraint judgements on monotonic monovariate polynomial functions to linear constraint judgements.\\

% After shortly introducing the work by Baillot and Ghyselen and the variant of the pi-calculus they consider, we focus on so-called \textit{constraint judgements} that are particularly relevant for their type system. We give different interpretations of the judgements to give both a formal and an intuitive understanding. Judgements are first limited to linear judgements and are verified by reducing the problem to an integer programming problem. To increase expressiveness, we then show how we can reduce monotonic monovariate polynomial constraints to linear constraints.\\

% Constructing a type checker for the type system by Baillot and Ghyselen introduces a number of challenges. One such challenge is checking for the existence of a specific substitution needed during type checking of outputs on \textit{servers}. We prove that this is NP-complete by reducing it to the NP-complete 3-SAT problem. To get around this, we limit ourselves to type checking processes with expressions of a particular form and introduce a function \textit{instantiate} that uses a greedy strategy to find substitutions.\\

% Another problem encountered during construction of the type checker is that of soundness of its type rules. More specifically, if we are not careful, reduction of processes checked by the type checker may no longer type check and the subject reduction property is therefore lost. We solve this problem by introducing the idea of \textit{combined complexities} consisting of a set of intersecting parametric complexity bounds. We also define the accompanying function \textit{basis} that keeps a combined complexity as a minimal set. Finally, we introduce algorithmic type rules for the type checker and prove them sound, including subject reduction. We show how an example process corresponding to the sequential merge function can be type checked using our type checker.\\

Our type inference algorithm is constraint-based, and as such generates constraints that enforce premises of the type rules are satisfied, based on a provided process. Once these constraints have been generated, they are reduced to simpler constraints that may be checked using an off-the-shelf SMT solver. As many of the premises in the type rules are universally quantified constraint judgements, our generated constraints contain both existential quantifiers over unknown coefficients and universal quantifiers over index variables. Such constraints are often not solvable using SMT solvers. As such, we perform a number of over-approximations during constraint reduction that put some limitations on which processes we may infer bounds on the parallel complexity for. We implement type inference in Haskell using the Z3 SMT solver. We find that we can bound the parallel complexity of some linear time and many constant time servers in reasonable time.\\

In conclusion, we have implemented type checking and type inference for the type system by Baillot and Ghyselen, and find that we can type check some polynomial and linear time primitive recursive processes, as well as infer precise bounds on the parallel complexity of some linear time processes and many constant time processes. To increase the expressiveness of our type inference algorithm, future work may include relaxing the over-approximations made during constraint reduction, such that they more closely reflect the original constraints, while still being satisfiable in reasonable time. Furthermore, it may be interesting to see how our type inference algorithm extends to \textit{usage} types, as in the type system by Baillot et al. \cite{BaillotEtAl2021}, which is a generalization of the type system by Baillot and Ghyselen that increases the expressiveness and precision.
\maketitle

\begin{abstract}
    Type systems using sized types have been studied extensively in the context of complexity analysis of functional and parallel programs to formally express, verify and infer complexity bounds on programs. Recent contributions have extended this study to message-passing processes using behavioral types to bound channel synchronizations, providing a sound framework for parallel complexity of pi-calculus processes. We explore the challenges of implementing this work, and present a type checker and a type inference algorithm. Our type checker can verify complexity analyses of some polynomial- and linear time primitive recursive functions, encoded as replicated channel inputs (servers), by using integer programming to bound channel synchronizations. Comparison of parametric complexities is a partial order, so to bound parallel complexities, we introduce the notion of combined complexity: A set of intersecting parametric complexity bounds. Our type inference algorithm first generates a constraint satisfaction problem on sized input/output types that we reduce to a set of polynomial inequality constraints, a solution to which provides a parametric bound on the parallel complexity of a server. We show how our constraint satisfaction problems can be over-approximated and provide a Haskell implementation using the Z3 SMT solver that can provide reasonable bounds on the parallel complexity of some linear time servers.
    %
    % Complexity analysis has long been a central part of algorithm design and is paramount to determine the efficiency of algorithms and other systems. Existing type systems formalize rules determining the complexity of both functional and parallel programs. We extend the work by Baillot and Ghyselen \cite{BaillotGhyselen2021} and implement both type checking and type inference for their type system for parallel complexity with message-passing processes in the pi-calculus. We show how we determine if a linear constraint is always satisfied given some other constraints using linear programming, which is an important step during type checking. To increase expressiveness, we allow certain polynomial constraints that we can reduce to linear ones. We also introduce algorithmic type rules, where we introduce combined complexities to preserve the subject reduction property. Soundness of the algorithmic type rules is proved. We use a constraint-based system to infer types representing linear bounds. This system is two-phased and sets up constraints based on linear templates that are later reduced into simple constraints on coefficients from the templates. The constraints on coefficients are then checked for satisfiability by an off-the-shelf SMT solver, and types are inferred based on an interpretation of the constraints. We implement type inference in Haskell using Z3 as the SMT solver. We find that we can type check many linear processes and infer bounds for constant and linear time processes in a reasonable amount of time.
\end{abstract}
\clearpage

\text{ }
\vspace{50em}
\section*{Acknowledgments}
We give special thanks to our supervisor Hans Hüttel for his excellent supervision and perspectives. We also thank the people at Kobayashi Laboratory for their hospitality during our stay at the University of Tokyo, and especially Naoki Kobayashi for his excellent guidance and passionate input on type inference.
\clearpage

\setcounter{tocdepth}{1}
\tableofcontents*
\clearpage


\section{Introduction}\label{ch:introduction}

Static analysis of computational complexity has long been a central part of algorithm design and computer science as a whole. One way of performing such a static analysis on a program is by means of type-based techniques. Traditionally, soundness properties have pertained to the absence of certain run-time errors, i.e. \textit{well-typed programs do not go wrong} \cite{Milner1978} but  with the advent of behavioral type disciplines, soundness properties that for instance ensure bounds on the resource use of programs have been proved. % If combined with an implementation, i.e. an algorithm that specifies how the rules of a type system are to be used, we may either verify that a specified type based complexity analysis of a program is correct (type-checking), or automate the complexity analysis and infer a complexity bound for a program, if it can be bounded by the type system (type inference).

Research on type systems for computational complexity has originally focused on sequential programs, using a notion of types that can express sizes of terms in a program, referred to as sized types, which have been both formalized and implemented \cite{HofmannAndJost2003,HofmannAndHoffmann2010,HoffmannEtAl2012,LagoGaboardi2012,AvanziniLago2017}. However, there is particularly interest in static complexity analysis of parallel and concurrent computation, following the trend for programs to increase in size, as distributed systems scale better. Moreover, parallel and especially concurrent computation is significantly more difficult to analyze, and so the work on type systems for static complexity analysis has been extended to parallel computation \cite{HoffmannShao2015}, and more recently to the more intricate domain of message-passing processes, using behavioral type disciplines to bound message-passing \cite{BaillotGhyselen2021,BaillotEtAl2021}.

Baillot and Ghyselen \cite{BaillotGhyselen2021} introduce a type system for parallel computational complexity of $\pi$-calculus processes, extended with naturals and pattern matching as a computational model, combining sized types and input/output types to bound synchronizations on channels, and thereby bound the parallel time complexity of a process. Baillot et al. \cite{BaillotEtAl2021} generalize the type system using the more expressive behavioral type discipline usage types \cite{Kobayashi1998,KobayashiEtAl2000}. However, these type systems are quite abstract and build on a notion of sized types that has yet to be implemented in the context of message-passing, and so neither type checking nor type inference has until now been realized for either type system.

In this thesis, we explore the challenges of implementing type checking and type inference for the type system by Baillot and Ghyselen \cite{BaillotGhyselen2021}. An important part of this type system is the concept of indices. That is, arithmetic expressions that may contain index variables that represent unknown sizes, thereby enabling a notion of size polymorphism. Indices appear in sized types, to for instance express the timesteps at which a channel must synchronize, which may depend on the size of a value received on a replicated input. To compute an upper bound on the parallel complexity, a partial order on channel synchronizations is required, represented as index comparisons that we refer to as constraint judgements and read as: \textit{Provided a set of constraints on valuations of index variables, is one index always less than or equal to another?} Many of the challenges that arise for both type checking and type inference are related to either verification or satisfaction of such judgements.%Finally, the type system relies on on a special form of minus that can never give negative results, which breaks many useful mathematical properties.\\

We implement a type checker for the type system by Baillot and Ghyselen. This effort is two-fold: We define algorithmic type rules and show how constraint judgements on linear indices can be verified using integer programming or alternatively be over-approximated as linear programs. The type system makes heavy use of subtyping, which we partially account for using \textit{combined complexities} that are effectively sets of indices with a number of associated functions that enable us to discard indices we can guarantee to be bounded by other indices. Combined complexities have the advantage that we can defer finding a single index representing a least upper bound until a later time, and we can in fact show that the combined complexity of a closed process can always be reduced to a singleton, i.e. a singular complexity bound.

We prove that our type checker is sound with regards to time complexity, and to this effort we prove a subject reduction property. Our soundness results guarantee that the bounds assigned to well-typed processes by our type checker are indeed upper bounds on the parallel complexity. To increase the expressiveness of the type checker, we also show how constraints on monotonic univariate polynomial indices can be reduced to linear constraints.

We also define a type inference algorithm for the type system by Baillot and Ghyselen. We take a constraint based approach akin to that of \cite{HofmannAndJost2003,HofmannAndHoffmann2010,HoffmannEtAl2012,KobayashiEtAl2000,Kobayashi2005,Lhoussaine2004}, where unknown indices are represented by \textit{templates}: linear functions over a set of known index variables with unknown coefficients represented by coefficient variables. Inspired by  Kobayashi et al. \cite{KobayashiEtAl2000}, we first infer simple types, which are then used to infer a constraint satisfaction problem on use-capabilities and subtyping which we then reduce to constraints of the form $\exists\alpha_1,\dots,\alpha_n.\forall i_1,\dots,i_m.C_1\land\cdots\land C_k \implies I \leq J$ where $\alpha_1,\dots,\alpha_n$ are coefficient variables, $i_1,\dots,i_m$ are index variables, $C_1,\dots,C_k$ are inequality constraints on indices and $I$ and $J$ are indices.

We provide a Haskell implementation of our type inference algorithm using the Z3 SMT solver \cite{Z3}. We naively eliminate universal quantifiers by over-approximating our constraints using coefficient-wise inequality constraints. We account for antecedents $C_1,\dots,C_k$ by substitution. For instance, if we can deduce that coefficient $c$ is positive in the constraint $\exists\alpha_1,\dots,\alpha_n.\forall i_1,\dots,i_j,\dots,i_m.K \leq L + ci_j \land C_1 \land \cdots \land C_k \implies I \leq J$, then we can simulate the antecedent by substituting $\frac{K-L}{c} + i_j$ for $i_j$, i.e. $\exists\alpha_1,\dots,\alpha_n.\forall i_1,\dots,i_j,\dots,i_m.C_1\{\frac{K-L}{c} + i_j/i_j\} \land \cdots \land C_k\{\frac{K-L}{c} + i_j/i_j\} \implies I\{\frac{K-L}{c} + i_j/i_j\} \leq J\{\frac{K-L}{c} + i_j/i_j\}$. Using these over-approximations, our implementation is able to infer precise bounds on several processes containing replicated inputs with linear time complexity.


%%% Local Variables:
%%% mode: latex
%%% TeX-master: "../esop2023"
%%% End:

\chapter{The $\pi$-calculus}\label{ch:picalc}

In this chapter, we introduce an extended $\pi$-calculus definition, and introduce a cost model for time that uses process annotations to represent incurred reduction costs. Finally, we formalize the parallel complexity of a process.

%%

\section{Syntax and semantics}
We consider an asynchronous polyadic $\pi$-calculus extended with naturals as algebraic terms and a pattern matching construct that enables deconstruction of such terms. The languages of processes and expressions are defined by the syntax 
%
\begin{align*}
    P,Q \text{ (processes) } ::=&\; \nil \mid \left(\parcomp{P}{Q} \right) \mid \inputch{a}{\widetilde{v}}{}{P} \mid \asyncoutputch{a}{\widetilde{e}}{} \mid\; \bang{\inputch{a}{\widetilde{v}}{}{P}} \mid \newvar{a}{P} \mid \tick{P} \mid \\
    &\; \match{e}{P}{x}{Q}\\
    e \text{ (expressions) } ::=&\; 0 \mid \succc{e} \mid v
\end{align*}
%
where we assume a countably infinite set of names $\textbf{Var}$, such that $a,b,c\in\textbf{Var}$ represent channels, $x,y,z\in\textbf{Var}$ are bound to algebraic terms and the meta-variable $v\in\textbf{Var}$ may be bound to any expression. As usual, we have inaction $\mathbf{0}$ representing a terminated process, the parallel composition of two processes $P \mid Q$ and restrictions $\newvar{a}{P}$. 
%Here, $T$ is a type annotation the implications of which we shall defer until Section \ref{sec:typesandsubs}.
For polyadic inputs and outputs $\inputch{a}{\widetilde{v}}{}{P}$ and $\asyncoutputch{a}{\widetilde{e}}{}$, we use the notation $\widetilde{v}$ and $\widetilde{e}$, respectively, to denote sequences of names $v_1,\dots,v_n$ and expressions $e_1,\dots,e_n$, and we write $P[\widetilde{v}\mapsto\widetilde{e}]$ to denote the substitution $P[v_1\mapsto e_1,\dots,v_n\mapsto e_n]$ for names in process $P$. Similarly, for nested restrictions $\newvar{a_1}{\cdots \newvar{a_n}{P}}$ we may write $\newvar{\widetilde{a}}{P}$. For technical convenience, we only enable replication on inputs, i.e. $!\inputch{a}{\widetilde{v}}{}{P}$, such that we can more easily determine which channels induce recursive behavior. Note that any replicated process $!P$ can be simulated using a replicated input $!\inputch{a}{}{}{(P \mid \asyncoutputch{a}{}{})} \mid \asyncoutputch{a}{}{})$.\\ 

Naturals are represented by a zero constructor $0$, representing the smallest natural number, and a successor constructor $\succc{e}$ that represents the successor of the natural number $e$. Thus, algebraic terms have a clearly distinguishable base case. We use this for the pattern matching constructor that deconstructs such terms, by branching based on the shapes of the expressions. This will be useful later, as this enables us to analyze termination conditions for some processes with recursive behavior. The \texttt{tick} constructor represents a cost in time complexity of one. We provide a detailed account of this constructor in Section \ref{sec:parcomplex}.\\

In Definition \ref{def:structcong1}, we introduce the usual structural congruence relation \cite{Milner1993}. We omit rules for replication, as these will be covered by the reduction relation. The congruence relation essentially introduces associative and commutative properties to parallel compositions and enables widening or narrowing of the scopes of restrictions, thereby simplifying the semantics.
%
\begin{defi}[Structural congruence]
We define structural congruence $\equiv$ as the least congruence relation that satisfies the rules
%
\begin{align*}
    &\kern6em\runa{SC-nil}\;\;\parcomp{P}{\nil} \equiv P\kern2em \runa{SC-commu}\;\; \parcomp{P}{Q} \equiv \parcomp{Q}{P}\\ &\kern7em\runa{SC-assoc}\;\; \parcomp{P}{\left(\parcomp{Q}{R}\right)} \equiv \parcomp{\left(\parcomp{P}{Q}\right)}{R}\\
    %
    &\kern2em\runa{SC-scope}\;\;\newvar{a}{\left(\parcomp{P}{Q}\right)} \equiv \parcomp{\newvar{a}{P}}{Q}\;\; \text{ if } a \text{ is not free in } Q\\
    %
    &\kern2.5em\runa{SC-par}\;\;\infrule{P \equiv P'}{\parcomp{P}{Q} \equiv \parcomp{P'}{Q}} \kern2em
    \runa{SC-res}\;\;\infrule{P \equiv Q}{\newvar{a}{P} \equiv \newvar{a}{Q}}
\end{align*}
\label{def:structcong1}
\end{defi}
%
We extend the usual definition of structural congruence with rules that enable us to group restrictions. This allows us to introduce a normal form for processes that simplifies the definition of parallel complexity or span, and consequently makes it easier to prove various properties for typed $\pi$-calculi. We formalize the normal form in Definition \ref{def:canonform1}, referring to it as the canonical form. Essentially, we can view a process in canonical form as a sequence of bound names and a multiset of guarded processes. In Lemma \ref{lemma:cannform}, we prove that an arbitrary process is structurally congruent to a process in this form.
%
\begin{defi}[Canonical form]
We say that a process $P$ is in canonical form if is has the shape $\newvar{\widetilde{a}}{\left(\parcomp{G_1}{\parcomp{G_2}{\parcomp{\dots}{G_n}}}\right)}$
 where $G_1,G_2,\dots,G_n$ are referred to as guarded processes which may be of any of the forms
\begin{equation*}
    G ::=\; \bang{\inputch{a}{\widetilde{y}}{}{P}} \mid \inputch{a}{\widetilde{v}}{}{P} \mid \asyncoutputch{a}{\widetilde{e}}{} \mid \match{e}{P}{x}{Q} \mid \tick P
\end{equation*}
If $n = 0$ then the canonical form is $\newvar{\widetilde{a}}{\nil}$.
\label{def:canonform1}
\end{defi}
%
\begin{lemma}[Existence of a canonical form]\label{lemma:cannform}
Let $P$ be a process. Then there exists a process $Q$ in canonical form such that $P \equiv Q$.
\begin{proof}
Suppose by $\alpha$-renaming that all names are unique, then by using rule $\runa{SC-res}$ from left to right and $\runa{SC-scope}$ from right to left, we can widen the scope of all unguarded restrictions, such that all unguarded restrictions are outmost. Then using rule $\runa{SC-res}$ and $\runa{SC-zero}$ from left to right, we remove all unguarded inactions. If the whole process is unguarded, one inaction will remain and we have the canonical form $\newvar{\widetilde{a} : \widetilde{T}}{\mathbf{0}}$.
\end{proof}
\end{lemma}
%
In Table \ref{tab:redurules}, we define the reduction relation $\longrightarrow$ for $\pi$-calculus processes, such that $P \longrightarrow Q$ denotes that process $P$ reduces to $Q$ in one reduction step \cite{Milner1993}. We have the usual rules for the asynchronous polyadic $\pi$-calculus, enriched with rules for replicated inputs, pattern matching and temporal reductions. The former is covered by rule $\runa{R-rep}$ that synchronizes a replicated input with an output, preserving the replicated input for subsequent synchronizations. Rule $\runa{R-zero}$ considers the base case for pattern matching on naturals. For pattern matches on successors, we substitute the \textit{deconstructed} terms for variables bound in the patterns of the pattern match constructors. Finally, rule $\runa{R-tick}$ reduces a tick prefix, representing a cost of one in time complexity.
%
\begin{table*}[ht]
    \centering
    \begin{framed}\vspace{-1em}\begin{tabular}{l}
        \kern0em\runa{R-rep}\;\;\infrule{}{\parcomp{\;\bang{\inputch{a}{\widetilde{v}}{}{P}}}{\asyncoutputch{a}{\widetilde{e}}{}} \longrightarrow \parcomp{\;\bang{\inputch{a}{\widetilde{v}}{}{P}}}{\subst{P}{\widetilde{v}\mapsto \widetilde{e}}}}
        %
        \kern7em\runa{R-comm}\;\;\infrule{}{\parcomp{\inputch{a}{\widetilde{v}}{}{P}}{\asyncoutputch{a}{\widetilde{e}}{}} \longrightarrow \subst{P}{\widetilde{v}\mapsto \widetilde{e}}}\\[-1em]
        %
        \kern0em\runa{R-zero}\;\;\infrule{}{\match{0}{P}{x}{Q} \longrightarrow P}
        %
        \kern7em\runa{R-par}\;\;\infrule{P \longrightarrow Q}{\parcomp{P}{R} \longrightarrow \parcomp{Q}{R}}
        \\[-1em]
        %
        \kern0em\runa{R-succ}\;\;\infrule{}{\match{\succc{e}}{P}{x}{Q} \longrightarrow \subst{Q}{x \mapsto e}} \kern9.5em \runa{R-tick}\kern-1.5em\infrule{}{\tick P \longrightarrow P} \\[-1em]
        %
        %\kern0em\runa{R-empty}\;\;\infrule{}{\texttt{match}\; []\; \{ [] \mapsto P; x :: y \mapsto Q \} \longrightarrow P}
        %
        \kern-0em \runa{R-res}\kern-1em\infrule{P \longrightarrow Q}{\newvar{a}{P} \longrightarrow \newvar{a}{Q}}
        %
        %\kern7em\runa{R-cons}\;\;\infrule{}{\texttt{match}\; e :: e'\; \{ [] \mapsto P; x :: y \mapsto Q \} \longrightarrow Q[x \mapsto e,y \mapsto e']}\\[-1em]
        %
        %\kern3em\runa{R-par}\;\;\infrule{P \longrightarrow Q}{\parcomp{P}{R} \longrightarrow \parcomp{Q}{R}} \kern-0em \runa{R-res}\;\;\infrule{P \longrightarrow Q}{\newvar{a}{P} \longrightarrow \newvar{a}{Q}}\\
        %
        \kern2em\runa{R-struct}\infrule{P \equiv P'\quad P' \longrightarrow Q'\quad Q' \equiv Q}{P \longrightarrow Q} 
    \end{tabular}\end{framed}
    \smallskip
    \caption{The reduction rules defining $\longrightarrow$.}
    \label{tab:redurules}
\end{table*}
%
%
\section{Parallel complexity}\label{sec:parcomplex}
There are several tried approaches to modeling time complexity for the $\pi$-calculus. One way is to incur a cost for every axiomatic reduction, in our case whenever a channel synchronizes or a pattern match branches. A widely used alternative, including the one chosen by Baillot and Ghyselen, is to introduce an explicit process prefix constructor $\tick P$ denoting that the continuation $P$ is preceded by a cost in time complexity \cite{BaillotGhyselen2021,BaillotEtAl2021,DasEtAl2018}. This approach is more flexible, as we can simulate many different cost models, based on placement patterns of tick prefixes. We now present some of the properties of the tick constructor with respect to parallel complexity. We are interested in maximizing the parallelism, and to do so we must reduce ticks in parallel. Consider the process
\begin{align*}
    \overbrace{\tick\nil \mid \tick\nil \mid \cdots \mid \tick\nil}^{n}
\end{align*}
The sequential complexity or work is $n$, whereas the parallel complexity or span is $1$, as we can reduce the $n$ ticks in parallel. To keep track of the span during reduction, we introduce integer annotations to processes, such that process $P$ can be represented as $m : P$ where $m$ represents the time already incurred. We refer to such processes as \textit{annotated processes}, and we enrich the definition of structural congruence with four additional rules that include time annotations in Definition \ref{def:structuralcongruenceanno}. Intuitively, this means that process annotations can be moved outward and may be summed. Any process is also structurally congruent to one with an extra annotation of $0$.

\begin{defi}\label{def:structuralcongruenceanno}
    We enrich the definition of structural congruence with the four rules
    %
    \begin{align*}
        &\runa{SC-dis}\;\; m : (P \mid Q) \equiv (m : P) \mid (m : Q) \kern3em \runa{SC-ares}\;\; m : \newvar{a}{P} \equiv \newvar{a}{(m : P)} \\
        &\kern4em\runa{SC-sum}\;\; m : (n : P) \equiv (m + n) : P \kern3em \runa{SC-zero}\;\; 0 : P \equiv P
    \end{align*}
\end{defi}

Annotations on processes introduce another notion of canonical process that can be seen in Definition \ref{def:annotatedcanonical}. That is, an annotated process in canonical form is a sequence of nested restrictions on a parallel composition of annotated guarded processes.

% canonical form for annotated processes
\begin{defi}[Annotated canonical form]\label{def:annotatedcanonical}
An annotated process is in canonical form if it is of the form
\begin{align*}
    \newvar{\widetilde{a}}{(n_1 : G_1 \mid \cdots \mid n_m : G_m)}
\end{align*}
where $G_1, \dots, G_m$ are guarded annotated processes. If $m=0$, its canonical form is $\newvar{\widetilde{a}}{(0 : \nil)}$.
\end{defi}

\begin{lemma}[Existence of an annotated canonical form]\label{lemma:anncannform}
Let $P$ be an annotated process. Then there exists an annotated process $Q$ in annotated canonical form such that $P \equiv Q$.
\begin{proof}
Suppose by $\alpha$-renaming that all names are unique, then by using rule $\runa{SC-res}$, $\runa{SC-ares}$, $\runa{SC-dis}$ and $\runa{SC-sum}$ from left to right, and $\runa{SC-scope}$ from right to left, we can widen the scope of all unguarded restrictions, such that all unguarded restrictions are outmost, and all non-guarded annotations are prefixes to guarded processes. Then using rule $\runa{SC-res}$ and $\runa{SC-zero}$ from left to right, we remove all unguarded inactions. For all remaining guarded processes that are not prefixed with an annotation, we use $\runa{SC-res}$, $\runa{SC-par}$ and $\runa{SC-zero}$ to introduce prefixing $0$-annotations. If the whole process is unguarded, one inaction will remain and we have the canonical form $\newvar{\widetilde{a} : \widetilde{T}}{\mathbf{0}}$.
\end{proof}
\end{lemma}

With annotated processes, we can define the parallel reduction relation $\Longrightarrow$ in Table \ref{tab:redurulesanno}. Most notably, we can see that the tick constructor reduces to an annotation of $1$, and during communication we choose the maximum annotation amongst the two endpoints.\\

\begin{table*}[ht]
    \centering
    \begin{framed}\begin{tabular}{l}
        \vspace{-1.0em}
        \kern0em\runa{PR-rep}\;\infrule{}{\parcomp{(n :\;\bang{\inputch{a}{\widetilde{v}}{}{P}})}{(m : \asyncoutputch{a}{\widetilde{e}}{})} \Longrightarrow \parcomp{(n :\;\bang{\inputch{a}{\widetilde{v}}{}{P}})}{(\text{max}(n,m) : \subst{P}{\widetilde{v}\mapsto \widetilde{e}}})}\\ 
        %
        \kern1.5em\runa{PR-comm}\;\infrule{}{\parcomp{(n : \inputch{a}{\widetilde{v}}{}{P})}{(m :\asyncoutputch{a}{\widetilde{e}}{})} \Longrightarrow \text{max}(n,m) : \subst{P}{\widetilde{v}\mapsto \widetilde{e}}}        \vspace{-1em}\\
        %
        \kern-1em\runa{PR-tick}\;\infrule{}{\tick{P} \Longrightarrow 1 : P}
        %
        %\vspace{-1.5em}
        \kern-2em\runa{PR-zero}\;\infrule{}{\match{0}{P}{x}{Q} \Longrightarrow P}\vspace{-1em}\\
        %
        \kern4em\runa{PR-succ}\;\infrule{}{\match{\succc{e}}{P}{x}{Q} \Longrightarrow \subst{Q}{x \mapsto e}}\vspace{-1.0em}\\
        %
        \kern1em\runa{PR-par}\;\infrule{P \Longrightarrow Q}{\parcomp{P}{R} \Longrightarrow \parcomp{Q}{R}} \kern0em \runa{PR-res}\;\infrule{P \Longrightarrow Q}{\newvar{a}{P} \Longrightarrow \newvar{a}{Q}}\\
        \kern1em\runa{PR-annot}\;\infrule{P \Longrightarrow Q}{n : P \Longrightarrow n : Q}
        %
        \kern-1em\runa{PR-struct}\;\infrule{P \equiv P'\quad P' \Longrightarrow Q'\quad Q' \equiv Q}{P \Longrightarrow Q}
    \end{tabular}\end{framed}
    \smallskip
    \caption{The reduction rules defining $\Rightarrow$.}
    \label{tab:redurulesanno}
\end{table*}

In Definition \ref{def:bglcsim}, we define the local parallel complexity $C_\ell(P)$ of a process $P$. Intuitively, the local complexity is the maximal integer annotation in the canonical form of $P$.

\begin{defi}[Local complexity]\label{def:bglcsim}
    We define the local parallel complexity $\mathcal{C}_\ell(P)$ of a process $P$ by the following rules
    \begin{align*}
        \mathcal{C}_\ell(n:P) &= n + \mathcal{C}_\ell(P)\quad\quad
        \mathcal{C}_\ell(P \mid Q) = \text{max}(\mathcal{C}_\ell(P), \mathcal{C}_\ell(Q))\\
        \mathcal{C}_\ell(\newvar{a}{P}) &= \mathcal{C}_\ell(P)\quad\quad
        \mathcal{C}_\ell(G) = 0 \text{ if } G \text{ is a guarded process}
    \end{align*}
\end{defi}

We now formalize the parallel complexity or span of a process in Definition \ref{def:spancomp}. To account for the non-determinism of the $\pi$-calculus, the parallel complexity of a process is defined as the maximal integer annotation in any reduction sequence. To see why this is necessary, consider the process
%
\begin{align*}
    \inputch{a}{}{}{\tick} \mid \inputch{a}{}{}{\tick\asyncoutputch{a}{}{}} \mid \asyncoutputch{a}{}{}
\end{align*}
%
We have two possible reduction sequences with different integer annotations. That is, if we were to reduce the left-most input first, we have a single time reduction, as the second tick will be guarded. If we instead reduce the second input first, then we can synchronize of channel $a$ again after one time reduction, thus yielding two time reductions.
%
\begin{defi}[Parallel complexity]\label{def:spancomp}
We define the parallel complexity (or span) $\mathcal{C}_{\mathcal{P}}(P)$ of process $P$ as the maximal local complexity of any reduction sequence from $P$.
\begin{align*}
    \mathcal{C}_{\mathcal{P}}(P) = \text{max}\{n \mid P \Longrightarrow^* Q \land \mathcal{C}_\ell(Q) = n\}
\end{align*}
where $\Longrightarrow^*$ is the reflexive and transitive closure of $\Longrightarrow$.
\end{defi}


% As the reduction relation $\longrightarrow$ is sequential, we define a parallel reduction relation $\Longrightarrow^{-1}\subseteq\longrightarrow^*$ that reduces all outer-most ticks in parallel. Here, $\longrightarrow^*$ is the transitive and reflexive closure of $\longrightarrow$, and $P \Longrightarrow^{-1} Q$ is called a \textit{time reduction}. We formalize the relation in Definition \ref{def:timereduction}.
% %
% %
% \begin{defi}[Time reduction]\label{def:timereduction}
% Let $P$ be a process with canonical form $P \equiv \newvar{\widetilde{a}}{(G_1\mid\dots\mid G_n \mid \tick R_1 \mid\dots\mid\tick R_m)}$ such that $G_1\dots G_n$ are not guarded by ticks. We refer to the reduction sequence from $P$ to $Q\equiv \newvar{\widetilde{a}}{(G_1\mid\dots\mid G_n \mid R_1 \mid\dots\mid R_m)}$ $P\longrightarrow^* Q$ as a \textit{time reduction}, denoted $P \Longrightarrow^{-1} Q$. We say that a time reduction $P\Longrightarrow^{-1} Q$ is \textit{productive} if $P\neq Q$, i.e. $P$ is not invariant to $\Longrightarrow^{-1}$.
% \end{defi}
% %
% Another concern with respect to maximal parallelism is the reduction order. Consider the process
% %
% \begin{align*}
%     \tick \nil \mid \inputch{a}{}{}{\tick\nil} \mid \asyncoutputch{a}{}{}
% \end{align*}
% %
% We have two valid reduction sequences, we can either reduce the left-most tick first yielding a reduction sequence with two time reductions, or synchronize on channel $a$ first, such that the second tick is unguarded, thereby enabling a single time reduction. Thus, to maximize the parallelism, we must prioritize reductions on channels and pattern matches, to maximize the number of ticks in parallel, before performing a time reduction. Therefore, we introduce a reduction relation $\leadsto\subseteq\longrightarrow$ that represents \textit{non-temporal reductions}, i.e. reductions that do not use rule $\runa{R-tick}$. We formalize this relation in Definition \ref{def:nontempreduction}.
% %
% \begin{defi}[Non-temporal reduction]\label{def:nontempreduction}
% We define a relation $\leadsto\subseteq\longrightarrow$ such that $P\leadsto Q$ if $P \longrightarrow Q$ without using $\runa{R-tick}$. If there exists $Q$ such that $P\leadsto Q$, we write $P\!\!\leadsto$, and conversely, $P\!\not\!\leadsto$ if no such $Q$ exists. We write $P \leadsto^* Q$ for the transitive and reflexive closure of $\leadsto$.
% \end{defi}
% %
% Much inspired by a similar reduction strategy in Baillot and Ghyselen \cite{BaillotGhyselen2021}, we introduce the tick-last reduction strategy in Definition \ref{def:ticklaststrat}. To maximize the parallelism of reduction of process $P$, we perform non-temporal reductions until no non-temporal reduction is defined, and then we perform a single time reduction. If a tick was reduced, we proceed to perform non-temporal reduction again. If no tick was reduced, we terminate reduction, as the process then cannot reduce with reduction relation $\longrightarrow$.
% %
% \begin{defi}[Tick-last strategy]\label{def:ticklaststrat}
% We define a reduction strategy for processes called the tick-last strategy. We reduce a process $P$ in two steps
% \begin{enumerate}
%     \item We perform a sequence of non-temporal reductions on $P$ such that $P\leadsto^* Q$  and $Q\!\not\!\leadsto$.
    
%     \item We perform a single time reduction on $Q$ such that if
%     \begin{itemize}
%         \item $Q \Longrightarrow^{-1} Q$ we stop, as $Q$ cannot reduce any further.
%         \item $Q \Longrightarrow^{-1} R$ with $Q \neq R$ we proceed to step one starting from $R$.  
%     \end{itemize}
% \end{enumerate}
% We say that a reduction sequence $P_1 \longrightarrow^* P_{n+1}$ adheres to the tick-last strategy if it is of the form
% \begin{align*}
%     P_1 \leadsto^{*} P_1' \Longrightarrow^{-1} P_2 \leadsto^{*} \cdots \Longrightarrow^{-1} P_n \leadsto^{*} P_n' \Longrightarrow^{-1} P_{n+1}
% \end{align*}
% such that $P_1'\!\not\!\leadsto$, $P_1' \neq P_2$, $P_n'\!\not\!\leadsto$ and $P_n' = P_{n+1}$. Moreover, we write $P_1 \hookrightarrow^n P_{n+1}$ to denote that $P_1$ reduces to $P_{n+1}$ by the tick-last strategy using $n$ productive time reductions.
% \end{defi}
% %
% We now formalize the parallel complexity or span of a process in Definition \ref{def:spancomp}. To account for the non-determinism of the $\pi$-calculus, the parallel complexity of a process is defined as the maximal number of productive time reductions in any reduction sequence that adheres to the tick-last strategy. To see why this is necessary, consider the process
% %
% \begin{align*}
%     \inputch{a}{}{}{\tick} \mid \inputch{a}{}{}{\tick\asyncoutputch{a}{}{}} \mid \asyncoutputch{a}{}{}
% \end{align*}
% %
% We have two reduction sequences that adhere to the tick-last strategy, yet have different numbers of productive time reductions. That is, if we were to reduce the left-most input first, we have a single time reduction, as the second tick will be guarded. If we instead reduce the second input first, then we can synchronize of channel $a$ again after one time reduction, thus yielding two time reductions.
% %
% \begin{defi}[Parallel complexity]\label{def:spancomp}
% Let $P$ be a process. The parallel complexity (or span) of $P$ is given as the maximum number of productive time reductions in any reduction sequence from $P$ that adheres to the tick-last strategy 
% \begin{align*}
%     \mathcal{C}(P) = \text{max}\{n \mid P \hookrightarrow^n Q\}
% \end{align*}
% \end{defi}

%%
\chapter{Sized types for parallel complexity}\label{ch:bgts}
In this chapter, we briefly discuss the type system for parallel complexity of message-passing processes introduced in Baillot and Ghyselen \cite{BaillotGhyselen2021}. This type system builds on the foundations of indices and constraint judgements and formalizes parallel complexity analysis of $\pi$-calculus processes. Due to extensive use of subtyping and the challenges involved in verifying and satisfying constraint judgements, substantial modifications must be made to enable type checking and type inference of processes. We address these topics in Chapter \ref{ch:typecheck} and \ref{ch:timeinference}, respectively.\\

The type system for parallel complexity of message-passing processes introduced by Baillot and Ghyselen uses sized types to express parametric complexity of replicated input invocation, and thereby achieves precise bounds on primitively recursive processes: A class of processes behaving as primitively recursive functions. This requires a notion of polymorphism in the message types of replicated inputs. Baillot and Ghyselen introduce size polymorphism by bounding sizes of algebraic terms and synchronizations on channels with indices that may contain index variables representing unknown sizes. We may interpret an index with an index valuation that maps its index variables to naturals, such that the index may be evaluated to a natural number.\\

We first formally define indices and constraints on the valuations of indices. We give both a predicate logic and a model-theoretic interpretation of judgements on such constraints, referring to these as \textit{constraint judgements}. We then define sized types, the subtyping relation and introduce non-algorithmic type rules.

%\section{A type checker}\label{Sec:typesystembg}
\section{Indices and constraint judgements}\label{sec:indicesandjudgements}
In the type system by Baillot and Ghyselen, indices are used to keep track of sizes of inputs received on replicated inputs. As these sizes may be parametric, in that they may be dependent on the sizes of values received on replicated inputs, we view indices as algebraic expressions consisting of index variables $i,j,k\in\mathcal{V}$ ranging over a countable set, and function symbols, using meta-variable $f$, that may represent natural number constants as nullary functions as well as algebraic operators
\begin{align*}
    I,J ::= i \mid f(I_1,I_2,\dots,I_n)
\end{align*}
Each function symbol $f$ has an arity $\text{ar}(f)$ and an interpretation $[\![f]\!] : \mathbb{N}^{\text{ar}(f)} \rightarrow \mathbb{N}$. For the interpretation of binary difference, we assume that $[\![-]\!](n,m) = 0$ when $m \geq n$, which we refer to as the \textit{monus} operator. As indices may contain index variables, we assume some index valuation $\rho : \mathcal{V} \rightarrow \mathbb{N}$, and extend the definition of interpretations to indices, such that $[\![I]\!]_\rho$ is a natural number instance of index $I$, according to index valuation $\rho$, where for all $i$ in $I$, $\rho(i)$ substitutes for $i$ denoted $I\{\rho(i)/i\}$. Index substitution is defined in Definition \ref{def:indexsubstitution}. Based on the structure of the process that indices are used in the typing of, we may be able to establish relationships between the instances of these indices. For instance, a replicated input may receive values of sizes defined by an interval of two indices $[I,J]$. Then, we are only interested in index valuations $\rho$ that satisfy $[\![I]\!]_\rho \leq [\![J]\!]_\rho$. To express such relationships, we define binary constraints on indices in Definition \ref{def:indexconstr}.

\begin{defi}\label{def:indexsubstitution}
    We define index substitution by the following rules
    \begin{align*}
        i\substi{I}{j} &= j \text{ if } i = j\\
        i\substi{I}{j} &= i \text{ if } i \not = j\\
        f(I_1, I_2, \dots, I_n)\substi{J}{i} &= f(I_1\substi{J}{i}, I_2\substi{J}{i}, \dots, I_n\substi{J}{i})
    \end{align*}
\end{defi}

\begin{defi}[Index constraints]\label{def:indexconstr}
    Given a finite set of index variables $\varphi\subset \mathcal{V}$, we define a constraint $C$ on $\varphi$ to be an expression of the form $I \bowtie J$, where $I$ and $J$ are indices with all free index variables in $\varphi$ and $\bowtie\;\in\{\leq,=,\geq\}$ is a binary relation on $\mathbb{N}$. A finite set of constraints is represented by meta-variable $\Phi$.
\end{defi}
%
A constraint $I \bowtie J$ on $\varphi$ is satisfied given an index valuation $\rho : \varphi \longrightarrow \mathbb{N}$ when $[\![I]\!]_\rho \bowtie [\![J]\!]_\rho$ is satisfied, denoted $\rho \vDash I \bowtie J$. For a finite set of constraints $\Phi$, we write $\rho\vDash \Phi$ when $\rho \vDash C$ holds for all $C \in \Phi$. Finally, $\varphi;\Phi\vDash C$ holds when for all index valuations $\rho$ such that $\rho\vDash \Phi$ holds, we also have $\rho\vDash C$. That is, $\varphi;\Phi\vDash C$ holds exactly when $C$ does not impose further restrictions on index valuations on $\varphi$. Such judgements are fundamental to the type system by Baillot and Ghyselen, especially ones of the form $\varphi;\Phi\vDash I \leq J$, as they impose a partial order on indices wrt. how indices may be interpreted. This enables a notion of subtyping for parametric complexities, such that only indices that are greater or equal may substitute, thus preserving upper bounds on the global parallel complexity, as we shall see in the following sections.
%
%\section{The typechecker}





\begin{table*}[!ht]
    \begin{framed}\vspace{-1em}\begin{align*}
        %
        % S-nil
        &\kern-0.5em\runa{U-nil}\infrule{}{\varphi;\Phi;\Gamma \vdash \nil \triangleleft \{0\}}
        % S-nu
        \kern-2em\runa{U-nu}\infrule{\varphi;\Phi;\Gamma, a:T \vdash P \triangleleft \kappa}{\varphi;\Phi;\Gamma \vdash \newvar{a:T}{P} \triangleleft \kappa}
        % S-par
        \kern-1em\runa{U-par}\infrule{\varphi;\Phi;\Gamma \vdash P \triangleleft \kappa \quad \varphi;\Phi;\Delta \vdash Q \triangleleft \kappa'}{\varphi;\Phi;\Gamma \mid \Delta \vdash P \mid Q \triangleleft \text{basis}(\varphi, \Phi,\kappa \cup \kappa')}\\
        % S-match
        &\kern-0.5em\runa{U-match}\infrule{
        \begin{matrix}
            \varphi;\Phi;\Gamma \vdash e:\natinterval{I}{J} \quad \varphi;\Phi, I \leq 0;\Gamma \vdash P \triangleleft \kappa\\
            \varphi;\Phi, J \geq 1;\Gamma, x:\natinterval{I-1}{J-1} \vdash Q \triangleleft \kappa'
        \end{matrix}}{\varphi;\Phi;\Gamma \vdash \match{e}{P}{x}{Q} \triangleleft \text{basis}(\varphi, \Phi, \kappa \cup \kappa')}
        % S-tick
        \kern16em\runa{S-tick}\infrule{\varphi;\Phi;\Gamma \vdash P \triangleleft \kappa}{\varphi;\Phi;\uparrow^1\!\!\Gamma \vdash \tick P \triangleleft \kappa + 1}\\
        % S-iserv
        &\runa{S-iserv}\infrule{\begin{matrix}
            \texttt{in} \in \sigma\quad \varphi;\Phi;\Gamma\vdash a:\servt{I}{i}{\sigma}{K}{\widetilde{T}}\\
            (\varphi, \widetilde{i}); \Phi; \text{ready}(\varphi,\Phi,\tforwardsim{\Gamma}{I}), \widetilde{v} : \widetilde{T} \vdash P \triangleleft \kappa \quad (\varphi,\widetilde{i});\Phi\vDash\kappa \leq K
        \end{matrix}}
        {\varphi;\Phi;\Gamma \vdash \;\bang\inputch{a}{\widetilde{v}}{}{P}\triangleleft \{I\}}
         % S-ich
        \kern15em\runa{S-ich}\infrule{\begin{matrix}
            \texttt{in} \in \sigma\quad \varphi;\Phi;\Gamma \vdash a:\chant{\sigma}{I}{\widetilde{T}}\\
            \varphi; \Phi; \tforwardsim{\Gamma}{I}, \widetilde{v}:\widetilde{T} \vdash P \triangleleft \kappa
        \end{matrix}}
        {\varphi;\Phi;\Gamma \vdash \inputch{a}{\widetilde{v}}{}{P} \triangleleft \kappa + I}\\
        % S-oserv
        &\runa{S-oserv}\infrule{\begin{matrix}
            \texttt{out} \in \sigma\quad \varphi;\Phi;\Gamma\vdash a:\servt{I}{i}{\sigma}{K}{\widetilde{T}}\\
            \varphi; \Phi;(\tforwardsim{\Gamma}{I}) \vdash \widetilde{e}:\widetilde{S} \quad \text{instantiate}(\widetilde{i}, \widetilde{S}) = \{\widetilde{J}/\widetilde{i}\} \quad \varphi;\Phi \vDash \widetilde{S} \sqsubseteq \widetilde{T}
        \end{matrix}}
        {\varphi;\Phi;\Gamma \vdash \asyncoutputch{a}{\widetilde{e}}{}\triangleleft \{K\{\widetilde{J}/\widetilde{i}\} + I\}}
        % S-och
        \kern15em\runa{S-och}\infrule{\begin{matrix}
            \texttt{out} \in \sigma\quad \varphi;\Phi;\Gamma \vdash a:\chant{\sigma}{I}{\widetilde{T}}\\
            \varphi; \Phi; \tforwardsim{\Gamma}{I} \vdash \widetilde{e}:\widetilde{S} \quad \varphi;\Phi \vDash \widetilde{S} \sqsubseteq \widetilde{T}
        \end{matrix}}
        {\varphi;\Phi;\Gamma \vdash \asyncoutputch{a}{\widetilde{e}}{} \triangleleft \{I\}}\\
        % S-annot
        &\runa{S-annot}\infrule{\varphi;\Phi;\tforwardsim{\Gamma}{n}\vdash P \triangleleft \kappa}{\varphi;\Phi;\Gamma\vdash n:P \triangleleft \kappa + n}
    \end{align*}\vspace{-1em}\end{framed}
    \smallskip
    \caption{Usage typing rules for parallel complexity of processes.}
    \label{tab:usageprocesstypingrules}
\end{table*}
\subsection{Alternative formulations of constraint judgements}\label{sec:cjalternativeform}
There are several equivalent formulations of the problem of verifying the judgement $\varphi;\Phi\vDash C_0$. One such formulation is that the judgement holds, when the conjunction of constraints in $\Phi$ implies $C_0$, i.e. assuming that $n \bowtie m$ evaluates to a truth value based on membership in the relation $\bowtie$, the predicate formula $C_1 \land \cdots \land C_n \implies C_0$, where $\Phi = \{C_1,\dots,C_n\}$, must be satisfied for all valuations $\rho$ over $\varphi$. That is, let $C_i = I_i \bowtie_i J_i$, then for any valuation $\rho : \varphi \rightarrow \mathbb{N}$, the formula $([\![I_1]\!]_\rho \bowtie_1 [\![J_1]\!]_\rho) \land \cdots \land ([\![I_n]\!]_\rho \bowtie_n [\![J_n]\!]_\rho) \implies [\![I_0]\!]_\rho \bowtie_0 [\![J_0]\!]_\rho$ must be satisfied. Another interpretation of the problem is that the intersection of the feasible regions of all (inequality) constraints in $\Phi$ must be contained in the feasible region of $C_0$, or equivalently, the set of all valuations over $\varphi$ that satisfy all the constraints in $\Phi$, referred to as the model space of $\Phi$ wrt. $\varphi$, $\mathcal{M}_\varphi(\Phi)$ must be a subset of the model space of $C_0$ wrt. $\varphi$
\begin{equation*}
    \mathcal{M}_\varphi(\Phi) \subseteq \mathcal{M}_\varphi(\{C_0\})\quad\text{where}\quad\mathcal{M}_\varphi(\Phi)=\{\rho : \varphi \rightarrow \mathbb{N} \mid \rho \vDash C\;\text{for}\; C \in \Phi\}
\end{equation*}
or equivalently
\begin{equation*}
    \forall \rho \in \mathcal{M}_\varphi(\Phi) (\rho \in \mathcal{M}_\varphi(\{C_0\}))
\end{equation*}

Finally, given the fact that the current statement of the problem is expressed using a universal quantifier, we can negate the problem, obtaining a problem that can instead be expressed using an existential quantifier by the fact that $\neg \forall x P(x)$ is equivalent to $\exists x \neg P(x)$. This means the problem can also be expressed as 
%
\begin{equation*}
    \neg (\exists \rho \in \mathcal{M}_\varphi(\Phi) (\rho \not\in \mathcal{M}_\varphi(\{C_0\})))
\end{equation*}
or equivalently
\begin{equation*}
    \mathcal{M}_\varphi(\Phi) \cap \mathcal{M}_\varphi'(\{C_0\}) = \emptyset \quad\text{where}\quad
    \begin{matrix}
        \mathcal{M}_\varphi(\Phi)=\{\rho : \varphi \rightarrow \mathbb{N} \mid \rho \vDash C\;\text{for all}\; C \in \Phi\}\\
        \mathcal{M'}_\varphi(\Phi)=\{\rho : \varphi \rightarrow \mathbb{N} \mid \rho \not\vDash C\;\text{for some}\; C \in \Phi\}
    \end{matrix}
\end{equation*}
Notice that $\mathcal{M}_\varphi'(\{C\})$ is equivalent to $\mathcal{M}_\varphi(\{C'\})$ where $C'$ is the inverse constraint of constraint $C$, and so $\mathcal{M}_\varphi(\Phi) \cap \mathcal{M}_\varphi'(\{C_0\}$) = $\mathcal{M}_\varphi(\Phi \cup \{C_0'\})$ given some method to invert constraints. Thus, the problem can also be expressed simply as
\begin{equation*}
    \mathcal{M}_\varphi(\Phi \cup \{C_0'\}) = \emptyset \quad \text{where } C_0' = \text{inverse of } C_0
\end{equation*}

In Example \ref{exmp:judgementsatisfaction}, we show how a judgement can be verified manually using the predicate logic and model-theoretic interpretations of judgements provided above.
%
\begin{examp}\label{exmp:judgementsatisfaction}
    Given index variables $\varphi = \{i, j, k\}$ and constraints $\Phi = \{C_1, C_2, C_3, C_4\}$ where
    \begin{align*}
        C_1 &= i \geq 4\\
        C_2 &= j \geq 2\\
        C_3 &= -k + 3 < 0\\ % k \leq 4
        C_4 &= i + j + k \leq 11
    \end{align*}
    we want to check if $\varphi; \Phi \vDash 2i + j^2 + 3k \geq 20$ always holds. %For this example we assume interpretations are as expected from usual mathematical notation.\\
    Namely, we are interested in verifying whether the constraint $2i + j^2 + 3k \geq 20$ imposes any additional constraints to the index variables $i$, $j$ and $k$ given the existing constraints $C_1$, $C_2$, $C_3$ and $C_4$. In this case, we can notice that the minimum values of $i$, $j$ and $k$ are $4$, $2$ and $4$ respectively. As such, given these constraints, the minimum value $2i + j^2 + 3k$ may evaluate to is $2 \cdot 4 + 2^2 + 3 \cdot 4 = 24$. As such, we can conclude that $\varphi; \Phi \vDash 2i + j^2 + 3k \geq 20$ always holds.\\
    
    We can also consider the predicate logic interpretation of the example. It suffices to only consider the index valuations that satisfy the conjunction of constraints, of which there are four. Here, we represent a valuation $\rho$ as a set of pairs of the form $\{(i,\rho(i)) \mid i\in\varphi\}$, and so we have $\{(i,4),(j,2),(k,4)\}$, $\{(i,5),(j,2),(k,4)\}$, $\{(i,4),(j,3),(k,4)\}$ and $\{(i,4),(j,2),(k,5)\}$. We can then verify the corresponding implications to show that the judgement holds
    %
    \begin{align*}
        (4 \geq 4) \land (2 \geq 2) \land ({-4}+3 < 0) \implies 4+2+4 \leq 11\\
        %
        (5 \geq 4) \land (2 \geq 2) \land ({-4}+3 < 0) \implies 5+2+4 \leq 11\\
        %
        (4 \geq 4) \land (3 \geq 2) \land ({-4}+3 < 0) \implies 4+3+4 \leq 11\\
        %
        (4 \geq 4) \land (2 \geq 2) \land ({-5}+3 < 0) \implies 4+2+5 \leq 11
    \end{align*}
    Or correspondingly in model-theoretic notation
    {\small
    \begin{align*}
        \mathcal{M}_\varphi(\Phi) =&\; \{\{(i,4),(j,2),(k,4)\}, \{(i,5),(j,2),(k,4)\}, \{(i,4),(j,3),(k,4)\}, \{(i,4),(j,2),(k,5)\}\}\\
        \mathcal{M}_\varphi(\{2i+j^2+3k\geq 20\}) =&\; \{\{(i,n_1),(j,n_2),(k,n_3)\} \mid n_1,n_2,n_3\in\mathbb{N},\; 2n_1 + n_2^2 + 3n_3 \geq 20 \}\\
        \mathcal{M}_\varphi(\Phi) \subseteq&\; \mathcal{M}_\varphi(\{2i+j^2+3k\geq 20\})
    \end{align*}}
    % \begin{align*}
    %     \mathcal{M}_\varphi(\Phi) = \left\{\{(i,4),(j,2),(k,4)\}, \{(i,5),(j,2),(k,4)\}, \{(i,4),(j,3),(k,4)\}, \{(i,4),(j,2),(k,5)\}\right\}
    % \end{align*}
    
    We can also solve the inverse of the mode-theoretic interpretation of the problem. Then we want to show that $\mathcal{M}_\varphi(\Phi) \cap \mathcal{M}_\varphi'(\{2i+j^2+3k\geq 20\}) = \emptyset$ or equivalently $\mathcal{M}_\varphi(\Phi \cup \{2i+j^2+3k < 20\}) = \emptyset$. 
    %
    \begin{align*}
        \mathcal{M}_\varphi(\{2i+j^2+3k < 20\}) =&\; \{\{(i,n_1),(j,n_2),(k,n_3)\} \mid n_1,n_2,n_3\in\mathbb{N},\; 2n_1 + n_2^2 + 3n_3 < 20 \}\\
        &\kern-9em\mathcal{M}_\varphi(\Phi) \cap \mathcal{M}_\varphi'(\{2i+j^2+3k\geq 20\}) = \mathcal{M}_\varphi(\Phi \cup \{2i+j^2+3k < 20\}) = \emptyset
    \end{align*}
\end{examp}
\section{Types and subtyping}\label{sec:typesandsubs}
We now introduce the types from the type system of Baillot and Ghyselen. The types include a base type describing naturals as algebraic terms with sizes bounded by an interval consisting of two indices. This enables us to statically reason about how sizes of data structures change throughout reduction of processes, providing us termination guarantees for some forms of recursion. The type system of Baillot and Ghyselen contains lists as an additional base type, however for conciseness of the type system, we only consider naturals.
%
\begin{align*}
    T,S ::=&\; \texttt{Nat}[I,J] \mid \texttt{ch}_I^\sigma(\widetilde{T}) \mid \forall_I\widetilde{i}.\texttt{serv}_K^\sigma(\widetilde{T})
\end{align*}
%
We use input/output types for channels, and we further distinguish between channels that have replicated inputs, i.e. channels that have recursive behavior, and those that do not. We refer to the former as \textit{servers}, and we more specifically require all inputs on such channels to be replicated for technical convenience. Both servers and normal channels are annotated with an index $I$ that for a normal channel represents the number of time steps remaining before the channel synchronizes, and for a server the remaining time before it becomes available. Note that this imposes a temporal linearity constraint onto normal channels, as such channels can synchronize at exactly one time step. For servers we have an additional index $K$ that represents the parametric complexity of invoking the continuation of a replicated input on the server. Finally, the set $\sigma \subseteq \{\texttt{in},\texttt{out}\}$ is a subset of use-capabilities. Since types consist partly of indices, we define index substitution on types in Definition \ref{def:typeindexsubstitution}.\\

\begin{defi}\label{def:typeindexsubstitution}
    We define index substitution on types by the following rules
    \begin{align*}
        \natinterval{I}{J}\substi{K}{i} &= \natinterval{I\substi{K}{i}}{J\substi{K}{i}}\\
        \chant{\sigma}{I}{\widetilde{T}}\substi{J}{i} &= \chant{\sigma}{I\substi{J}{i}}{\widetilde{T}\substi{J}{i}}\\
        \servt{I}{\widetilde{i}}{\sigma}{K}{\widetilde{T}}\substi{J}{j} &= \servt{I\substi{J}{j}}{\widetilde{i}}{\sigma}{K}{\widetilde{T}} \text{ if } j \in \widetilde{i}\\
        \servt{I}{\widetilde{i}}{\sigma}{K}{\widetilde{T}}\substi{J}{j} &= \servt{I\substi{J}{j}}{\widetilde{i}}{\sigma}{K\substi{J}{j}}{\widetilde{T}\substi{J}{j}} \text{ if } j \not\in \widetilde{i}
    \end{align*}
\end{defi}


Subtyping for base types and types is the least reflexive relation $\sqsubseteq$ that satisfies the subtyping rules in Table \ref{tab:subtypeSized}. As the type system should provide upper bounds on the parallel complexity of processes, it is safe to weaken the bounds on the sizes of natural types. That is, we may decrease the lower bound and increase the upper bound on the sizes of such terms. For server and channel types, we may relax use-capabilities and use the subtyping relation on parameter types as well as modify the complexity bounds on servers, depending on the use-capabilities. Servers and channels of input/output capability are invariant, those of input capability are covariant and those of output capability are contravariant. That is, if a server or channel that inputs a value of type $T$ is required, then we can safely use a server or channel that inputs a subtype of $T$, respectively. Conversely, when a server or channel of output capability is required, we can safely use a channel or server that outputs a supertype of the required parameter type \cite{PierceSangiorgi1996}. This becomes apparent when we assume types $\texttt{Integer}$ and $\texttt{Real}$ such that $\texttt{Integer} \sqsubseteq \texttt{Real}$, as any process that receives reals can also safely receive integers, and any process that output reals can also safely output integers. Unlike Baillot and Ghyselen \cite{BaillotGhyselen2021}, we do not discard associations from our type contexts, rather we discard use-capabilities from channels and servers. Thus, to ensure the type checker is sound, we introduce rules $\runa{BGS-cempty}$ and $\runa{BGS-sempty}$ such that channel and server types are super types of ones with no use-capabilities.

%
\begin{table*}[h!]
    \begin{framed}\vspace{-1em}\begin{align*}
        &\kern0em\runa{BGS-nweak}\;\infrule{\varphi;\Phi\vDash I' \leq I\quad\quad \varphi;\Phi\vDash J \leq J'}{
        \varphi;\Phi\vdash \texttt{Nat}[I,J] \sqsubseteq \texttt{Nat}[I',J']}
        %
        \kern3em\runa{BGS-cinvar}\;\infrule{\varphi;\Phi\vdash \widetilde{T}\sqsubseteq\widetilde{S}\quad\quad \varphi;\Phi\vdash \widetilde{S}\sqsubseteq\widetilde{T}}{\varphi;\Phi\vdash\texttt{ch}_I^{\{\texttt{in},\texttt{out}\}}(\widetilde{T}) \sqsubseteq \texttt{ch}_I^{\{\texttt{in},\texttt{out}\}}(\widetilde{S})}\kern7em\\[-1em]
        %
        \vspace{-0.5em}
        &\kern-0em\runa{BGS-ccovar}\;\infrule{\{\texttt{in}\}\subseteq\sigma\quad\varphi;\Phi\vdash \widetilde{T}\sqsubseteq\widetilde{S}}{\varphi;\Phi\vdash \texttt{ch}_I^{\sigma}(\widetilde{T})\sqsubseteq\texttt{ch}_I^{\{\texttt{in}\}}(\widetilde{S})}\quad\quad\runa{BGS-ccontra}\;\infrule{\{\texttt{out}\}\subseteq\sigma\quad\varphi;\Phi\vdash \widetilde{S}\sqsubseteq\widetilde{T}}{\varphi;\Phi\vdash \texttt{ch}_I^{\sigma}(\widetilde{T})\sqsubseteq \texttt{ch}_I^{\{\texttt{out}\}}(\widetilde{S})}\\[-1em]
        %
        &\kern4em\runa{BGS-sinvar}\;\infrule{(\varphi,\widetilde{i});\Phi\vdash \widetilde{T}\sqsubseteq\widetilde{S}\quad\quad (\varphi,\widetilde{i});\Phi\vdash \widetilde{S}\sqsubseteq\widetilde{T}\quad\quad (\varphi,\widetilde{i});\Phi\vDash K = K'}{\varphi;\Phi\vdash
        \forall_I\widetilde{i}.\texttt{serv}^{\{\texttt{in},\texttt{out}\}}_K(\widetilde{T})
        \sqsubseteq \forall_I\widetilde{i}.\texttt{serv}^{\{\texttt{in},\texttt{out}\}}_{K'}(\widetilde{S})}\\[-1em]
        %
        \vspace{-0.5em}
        &\kern5em\runa{BGS-scovar}\;\infrule{\{\texttt{in}\}\subseteq\sigma\quad(\varphi,\widetilde{i});\Phi\vdash \widetilde{T}\sqsubseteq\widetilde{S}\quad (\varphi,\widetilde{i});\Phi\vDash K' \leq K}{\varphi;\Phi\vdash \forall_I\widetilde{i}.\texttt{serv}^{\sigma}_K(\widetilde{T})\sqsubseteq\forall_I\widetilde{i}.\texttt{serv}^{\{\texttt{in}\}}_{K'}(\widetilde{S})}\\[-1em]
        &\kern4.5em\runa{BGS-scontra}\;\infrule{\{\texttt{out}\}\subseteq\sigma\quad(\varphi,\widetilde{i});\Phi\vdash \widetilde{S}\sqsubseteq\widetilde{T}\quad (\varphi,\widetilde{i});\Phi\vDash K \leq K'}{\varphi;\Phi\vdash \forall_I\widetilde{i}.\texttt{serv}^{\sigma}_K(\widetilde{T})\sqsubseteq \forall_I\widetilde{i}.\texttt{serv}^{\{\texttt{out}\}}_{K'}(\widetilde{S})}\\[-1em]
        %
        &\kern0em\runa{BGS-cempty}\;\infrule{}{\varphi;\Phi\vdash \texttt{ch}^\sigma_I(\widetilde{S}) \sqsubseteq \texttt{ch}^\emptyset_I(\widetilde{T})}\quad\runa{BGS-sempty}\;\infrule{}{\varphi;\Phi\vdash \forall_I\widetilde{i}.\texttt{serv}^\sigma_K(\widetilde{S}) \sqsubseteq \forall_I\widetilde{i}.\texttt{serv}^\emptyset_{K'}(\widetilde{T})}
    \end{align*}\vspace{-1em}\end{framed}
    \smallskip
    \caption{Rules for subtyping of base types and types.}
    \label{tab:subtypeSized}
\end{table*}
\section{Non-algorithmic type rules}

We first consider the type rules for expressions, which are shown in Table \ref{tab:sizedtypedexpressiontypes}. The zero term $0$ intuitively receives the type $\texttt{Nat}[0,0]$ and a successor to a natural term has the same type as its predecessor, but with 1 added to its lower and upper bounds. Finally, a variable receives a type if it is bound in the type context.\\
%Lists are typed similarly, aside from the addition of an element base type. For the element type of a list, we simply use the least lower bound and greatest upper bound on the size amongst the elements of the list.

\begin{table*}[ht]
    \begin{framed}\vspace{-1em}\begin{align*}
        &\kern2em
        \runa{BG-nzero}\;\infrule{}{\varphi;\Phi;\Gamma\vdash\withtype{0}{\typenat[0,0]}}\kern0em
        \runa{BG-nsucc}\;\infrule{\varphi;\Phi;\Gamma \vdash \withtype{e}{\typenat[I, J]}}{\varphi;\Phi;\Gamma \vdash \withtype{\succc{e}}{\typenat[I + 1, J + 1]}}\\[-1em]
        %
        &\kern1em\runa{BG-sub}\;\infrule{\varphi;\Phi;\Delta\vdash e : S\quad\quad \varphi;\Phi\vdash \Gamma\sqsubseteq \Delta\quad\quad \varphi;\Phi\vdash S \sqsubseteq T}{\varphi;\Phi;\Gamma\vdash e : T}\kern11em\runa{BG-var}\;\infrule{}{\varphi;\Phi;\Gamma, \withtype{v}{T} \vdash \withtype{v}{T}}
    \end{align*}\vspace{-1em}\end{framed}
    \smallskip
    \caption{Type rules for expressions.}
    \label{tab:sizedtypedexpressiontypes}
\end{table*}

Before introducing the type rules for processes, we first introduce a function $\downarrow^{\varphi;\Phi}_I\!\!(T)$ in Definition \ref{def:delaysized} that \textit{advances} the time of type $T$ by $I$ units of time complexity. For a channel type $\texttt{ch}^\sigma_J(\widetilde{S})$, we subtract $I$ from $J$ whenever we can guarantee that $J\geq I$ under the constraints imposed on $\varphi$ by $\Phi$. Otherwise, the advancement of $I$ units of time complexity is undefined for type $\texttt{ch}^\sigma_J(\widetilde{S})$, to ensure bounds on communication are not violated. For a server type $\forall_J\widetilde{i}.\texttt{serv}^\sigma_K(\widetilde{S})$, corresponding outputs are well-typed for any timestep $I$ with $I\geq J$, and so a server simply loses input capability whenever we cannot guarantee that $J \geq I$. We extend advancement of time to contexts such that $\downarrow^{\varphi;\Phi}_I(\Gamma)(v)=\;\downarrow^{\varphi;\Phi}_I(\Gamma(v))$. When it is clear from context, we may omit $\varphi$ and $\Phi$.

\begin{defi}[Advancement of Time]\label{def:delaysized}
Let $\varphi$ be a set of index variables, $\Phi$ a set of constraints on indices, $T$ a type and $J$ an index. Then $T$ after $J$ units of time complexity, $\susume{T}{\varphi}{\Phi}{I}$, is given by the rules below
\begin{align*}
    \susume{\natinterval{I}{J}}{\varphi}{\Phi}{I} =&\; \natinterval{I}{J}\\
    %
    %\susume{\texttt{List}[J,K](\mathcal{B})}{\varphi}{\Phi}{I} =&\; \texttt{List}[J,K](\mathcal{B})\\
    %
    \susume{\texttt{ch}^\sigma_J(\widetilde{T})}{\varphi}{\Phi}{I} =&\; \left\{ \begin{matrix}
        %\texttt{ch}^\emptyset_J(\widetilde{T}) & \text{if}\; \sigma = \emptyset\\
        %
        \texttt{ch}^\sigma_{J-I}(\widetilde{T}) & \text{if}\; \varphi;\Phi \vDash J \geq I\\
        %
        \texttt{ch}^\emptyset_{0}(\widetilde{T}) & \text{if}\; \varphi;\Phi \nvDash J \geq I
    \end{matrix} \right.\\
    %
    %\texttt{ch}^\sigma_{J-I}(\widetilde{T}) & \text{if}\; \varphi;\Phi \vDash J \geq I\\
    %
    % \susume{\inchanneltypeS{J}{\widetilde{T}}}{\varphi}{\Phi}{I} =&\; 
    %  \inchanneltypeS{J-I}{\widetilde{T}} & \text{if}\; \varphi;\Phi \vDash J \geq I \\
    % %
    % \susume{\outchanneltypeS{J}{\widetilde{T}}}{\varphi}{\Phi}{I} =&\; 
    %  \outchanneltypeS{J-I}{\widetilde{T}} & \text{if}\; \varphi;\Phi \vDash J \geq I \\
    %
    \susume{\forall_J\widetilde{i}.\texttt{serv}^\sigma_K(\widetilde{T})}{\varphi}{\Phi}{I} =&\; \left\{ \begin{matrix}
        \forall_{J-I}\widetilde{i}.\texttt{serv}^\sigma_K(\widetilde{T}) & \text{if}\; \varphi;\Phi \vDash J \geq I\\
        %
        \forall_{J-I}\widetilde{i}.\texttt{serv}^{\sigma \cap \{\texttt{out}\}}_K(\widetilde{T}) & \text{if}\; \varphi;\Phi \nvDash J \geq I
    \end{matrix} \right.
    %  \servS{J - I}{\widetilde{i}}{K}{\widetilde{T}} & \text{if}\; \varphi;\Phi \vDash J \geq I \\
    % %
    % \susume{\servS{J}{\widetilde{i}}{K}{\widetilde{T}}}{\varphi}{\Phi}{I} =&\; \oservS{J - I}{\widetilde{i}}{K}{\widetilde{T}} & \text{if}\; \varphi;\Phi \vDash J \not\geq I \\          
    % %
    % \susume{\iservS{J}{\widetilde{i}}{K}{\widetilde{T}}}{\varphi}{\Phi}{I} =&\; 
    %  \iservS{J - I}{\widetilde{i}}{K}{\widetilde{T}} & \text{if}\; \varphi;\Phi \vDash J \geq I \\
    % %
    % \susume{\oservS{J}{\widetilde{i}}{K}{\widetilde{T}}}{\varphi}{\Phi}{I} =&\; \oservS{J - I}{\widetilde{i}}{K}{\widetilde{T}}
\end{align*}
\end{defi}

\begin{defi}[Time invariance]\label{def:timeinvariance}
Let $\Gamma$ be a type context. We say that $\Gamma$ is \textit{time invariant} if it only contains variables of either base types or server type with time $0$ and use-capabilities $\sigma$ such that $\sigma\subseteq\{\texttt{out}\}$, i.e. $\forall_0\widetilde{i}.\texttt{serv}^{\sigma}_K(\widetilde{T})$ for some index variables $\widetilde{i}$, types $\widetilde{T}$ and index $K$.
\end{defi}

We now present the type rules of the type system by Baillot and Ghyselen, adapted to fit our syntax. Type judgements are of the form $\varphi;\Phi;\Gamma \vdash P \triangleleft K$, which means that process $P$ has complexity $K$ given constraints $\Phi$ with index variables in $\varphi$ and given a type environment $\Gamma$. The type rules are defined in Table \ref{tab:bgprocesstypingrules}. Rule $\runa{BG-iserv}$ handles replicated inputs and ensures that name $a$ is bound to a server type with input capability in the type context. We must also make sure that in the continuation $P$, the type context must be time invariant as the replicated input may be invoked any number of times after $I$ units of time have elapsed. Thus, only free naturals and servers with no input capability are safe. Rule $\runa{BS-ich}$ is similar except we do not require the type context in the continuation to be time invariant as it is only used once. Rule $\runa{BG-oserv}$ types output servers and most notably uses polymorphism in the index variables $\widetilde{i}$. As such, when typing the expressions sent on the server, we must ensure that we can \textit{instantiate} the index variables of the server using a substitution. Finally, type rule $\runa{BG-match}$ shows how index constraints are introduced when typing processes by utilizing information gained from the two branches of the match expression.\\

% Examples
We now show how a server calculating the $n$th digit of the Fibonacci sequence can be typed. Before presenting the process for the implementation of Fibonacci's sequence, we first need to encode addition in the $\pi$-calculus, which we do using the \textit{add} server as follows.
%
\begin{align*}
    P_\text{add}\defeq&\;\bang\inputch{\text{add}}{x,y,r}{}{
        \texttt{match}\; x\; \{
             0 \mapsto \asyncoutputch{r}{y}{};
            \succc{z} \mapsto \asyncoutputch{\text{add}}{z,\succc{y},r}{}\}}
    %
\end{align*}

The \textit{add} server needs three inputs $x$, $y$, and $r$. The parameters $x$ and $y$ represent two naturals to be added, and $r$ represents the channel intended for receiving the result. Note that no ticks are included in the server as we assume that addition can be done in constant time. The following process for calculating the $n$th number of the Fibonacci sequence is a naïve recursive implementation calculating $\textit{fib}(n)=\textit{fib}(n-1)+\textit{fib}(n-2)$. The server takes two parameters $n$ and $r$ where $n$ is the number of the Fibonacci sequence to calculate and $r$ represents the channel intended for receiving the result.
%
\newcommand{\funcf}[0]{l}
\newcommand{\funcg}[0]{l}
\newcommand{\funcgp}[0]{l-1}
\newcommand{\funcgpp}[0]{l-2}
\newcommand{\funcgppp}[0]{l-1}
\begin{align*}
    P_\text{fib}\defeq&\; \bang\inputch{\text{fib}}{n,r}{}{
         \texttt{match}\; n\; \{ 0 \mapsto \asyncoutputch{r}{0}{}\!;\;
              \succc{n_1} \mapsto\\ 
              &\quad\texttt{match}\; n_1\; \{
                    0 \mapsto \asyncoutputch{r}{\succc{0}}{}\!;\;
                    \succc{n_2} \mapsto\\ &\quad\quad\newvar{r_1,r_2,r_3}{(\asyncoutputch{\text{fib}}{n_1,r_1}{}\mid\asyncoutputch{\text{fib}}{n_2,r_2}{}\\
    &\quad\quad\mid\inputch{r_2}{m_2}{}{\inputch{r_1}{m_1}{}{\tick{\asyncoutputch{\text{add}}{m_1,m_2,r_3}{}}}\mid \inputch{r_3}{m_3}{}{\asyncoutputch{r}{m_3}{}}})}\}\}
    }
\end{align*}

Finally we present a type context $\Gamma$ under which the two servers \textit{add} and \textit{fib} are well-typed. Note that even though we use a naïve implementation of the Fibonacci sequence, we can still get a linear bound as the program runs in parallel.
%
\begin{align*}
    \Gamma \defeq&\; \text{add} : \servt{0}{i,j,k}{\{\texttt{in},\texttt{out}\}}{0}{\texttt{Nat}[0,i],\texttt{Nat}[j,k],\channeltypeS{0}{\texttt{Nat}[j,i+k]}},\\
    &\;\text{fib} : \servt{0}{l}{\{\texttt{in},\texttt{out}\}}{\funcf}{\texttt{Nat}[0,l],\channeltypeS{\funcg}{\texttt{Nat}[0,\textit{fib}(l)]}}
\end{align*}

\begin{table*}
    \begin{framed}\vspace{-1em}\begin{align*}
        &\kern46em\\[-2em] % Stretch frame
        &\kern0em\runa{BG-zero}\infrule{}{\varphi;\Phi;\Gamma \vdash \withcomplex{\nil}{0}}\!\!
        \runa{BG-subtype}\;\infrule{\varphi;\Phi;\Delta \vdash \withcomplex{P}{K} \quad \varphi;\Phi \vdash \Gamma \sqsubseteq \Delta \quad \varphi; \Phi \vDash K \leq K'}{\varphi;\Phi;\Gamma \vdash \withcomplex{P}{K'}}
        \\[-1em]
        %
        &\kern-0em\runa{BG-match}\;\infrule{\varphi;\Phi;\Gamma \vdash \withtype{e}{\natinterval{I}{J}} \quad \varphi;\Phi, I \leq 0;\Gamma \vdash \withcomplex{P}{K} \quad \varphi;\Phi, J \geq 1;\Gamma, \withtype{x}{\natinterval{I\monus 1}{J\monus 1}} \vdash \withcomplex{Q}{K}}{\varphi;\Phi;\Gamma \vdash \withcomplex{\match{e}{P}{x}{Q}}{K}}\\[-1em]
        %
        &\kern4em\runa{BG-par}\;\infrule{\varphi;\Phi;\Gamma\vdash P \triangleleft K\quad \varphi;\Phi;\Gamma\vdash Q \triangleleft K}{\varphi;\Phi;\Gamma\vdash \parcomp{P}{Q} \triangleleft K}\quad\quad\quad\quad\quad\quad \runa{BG-tick}\;\infrule{\varphi;\Phi;\susumesim{\Gamma}{1}\vdash P \triangleleft K}{\varphi;\Phi;\Gamma\vdash \tick P \triangleleft K + 1}\\[-1em]
        %
        &\kern-0em\runa{BG-iserv}\;\infrule{\texttt{in}\in\sigma\quad \varphi;\Phi\vdash\;\susumesim{\Gamma}{I},a:\forall_0\widetilde{i}.\texttt{serv}^\sigma_K(\widetilde{T}) \sqsubseteq \Gamma'\;\text{and}\; \Gamma'\;\text{time invariant}\quad \varphi,\widetilde{i};\Phi;\Gamma',\widetilde{v} : \widetilde{T}\vdash P \triangleleft K}{\varphi;\Phi;\Gamma,\Delta,a : \servt{I}{\widetilde{i}}{\sigma}{K}{\widetilde{T}}\vdash\; \bang\inputch{a}{\widetilde{v}}{}{P}\triangleleft I}\\[-1em]
        %
        &\kern-0em\runa{BG-ich}\;\infrule{\texttt{in}\in\sigma\quad \varphi;\Phi;\susumesim{\Gamma}{I},\widetilde{v} : \widetilde{T}, a : \chant{\sigma}{0}{\widetilde{T}}\vdash P \triangleleft K}{\varphi;\Phi;\Gamma, a : \chant{\sigma}{I}{\widetilde{T}}\vdash \inputch{a}{\widetilde{v}}{}{P}\triangleleft K + I}\kern8.5em \runa{BG-och}\;\infrule{\texttt{out}\in\sigma\quad \varphi;\Phi;\susumesim{\Gamma}{I}\vdash \widetilde{e} : \widetilde{T}}{\varphi;\Phi;\Gamma,a:\chant{\sigma}{I}{\widetilde{T}}\vdash \asyncoutputch{a}{\widetilde{e}}{} \triangleleft I}\\[-1em]
        %
        &\kern2em\runa{BG-oserv}\;\infrule{\texttt{out}\in\sigma\quad \varphi;\Phi;\susumesim{\Gamma}{I}\vdash \widetilde{e} : \widetilde{T}\substi{\widetilde{J}}{\widetilde{i}}}{\varphi;\Phi;\Gamma, a : \servt{I}{\widetilde{i}}{\sigma}{K}{\widetilde{T}}\vdash \asyncoutputch{a}{\widetilde{e}}{} \triangleleft K\!\substi{\widetilde{J}}{\widetilde{i}} + I}\kern12em \runa{BG-nu}\;\infrule{\varphi;\Phi;\Gamma,\withtype{a}{T} \vdash \withcomplex{P}{K}}{\varphi;\Phi;\Gamma \vdash \newvar{a}{\withcomplex{P}{K}}}
    \end{align*}\vspace{-1em}\end{framed}
    \smallskip
    \caption{Sized typing rules for parallel complexity of processes.}
    \label{tab:bgprocesstypingrules}
\end{table*}
\section{Examples of invalid configurations}
The following examples are written in the format $\conf{E, a}$, where $E$ is an editor expression and $a$ is the AST on which we apply the editor expression. \\

In equation \ref{condsubproblem} we show how conditioned substitution can cause problems.
\begin{equation}
    \conf{\left(@\texttt{break} \Rightarrow \replace{\texttt{break}}\right) \ggg \texttt{child}\; 1,\; \lambda x.\hole\; \cursor{\breakpoint{c}}} \label{condsubproblem}
\end{equation}
 In the example we check if the cursor is at a breakpoint, and since the check is true we \textit{toggle} the breakpoint thereby making the following \texttt{child} 1 command problematic. The constant c cannot have a child which means this configuration would cause a run-time error. \\
 
In equation \ref{parentproblem} we show how using the \texttt{parent} command can cause problem when the root is unknown.
\begin{equation}
    \conf{\left(\lozenge\texttt{hole} \Rightarrow \texttt{parent}\right) \ggg \texttt{parent},\; \cursor{\lambda x.\hole}\; c} \label{parentproblem}
\end{equation}
In the example we first check if there is a hole in some subtree of the current cursor. This condition holds and we therefore evaluate the \texttt{parent} command resulting in the AST $\cursor{\lambda x.\hole\; c}$. When the next \texttt{parent} command is evaluated we have a run-time error since we are already situated at the root.\\

In equation \ref{astproblem} we show how an editor expression can result in an AST that would cause a run-time error when evaluated.
\begin{equation}
    \conf{\left(\neg\Box(\texttt{lambda}\; x) \Rightarrow \texttt{child}\; 1\right) \ggg \replace{\texttt{var}\; x}.\texttt{eval},\; \cursor{\lambda x.\hole}\; c} \label{astproblem}
\end{equation}
In the example we first check if it is \textbf{not} necessary that the subtree of the cursor contains a lambda expression. This condition does not hold since it is necessary. Since the condition does not hold we do not evaluate the \texttt{child} 1 command, which means the following substitution of \texttt{var} x is problematic. The substitution results in the AST $\cursor{\texttt{var}\; x}\; c$, which causes a run-time error when the command \texttt{eval} is evaluated, since the left child of the function application is no longer a function.
%
\section{Over-approximations}
As we cannot determine statically whether a condition holds, we establish over-approximations to ensure run-time errors cannot occur in well-typed configurations. As equation \ref{parentproblem} shows, conditioned expressions can result in loss of information about the cursor location. As such, we enforce the cursor \textit{depth} in the tree to be the same before and after a conditioned expression. Furthermore, the first cursor movement in a conditioned expression must be a \texttt{child} prefix. As equation \ref{condsubproblem} shows, conditioned substitution also results in loss of information. Thus, we can no longer guarantee that subsequent substitution at a deeper level is well-typed. Similarly, we no longer know of the structure of the subtree, such that we must condition \texttt{child} prefixes.\\

The above discussion leads to the following list of over-approximations:
\begin{itemize}
    \item In conditioned and recursive expressions, the cursor depth must be the same before and after.
    \item In conditioned and recursive expressions, only the subtree encapsulated by the cursor is accessible.
    \item After conditioned substitution, subsequent substitution at a deeper level is no longer valid, and the \texttt{child} prefix command must be conditioned.
\end{itemize}
%
\section{AST type rules}
\begin{table*}[htp]
    \centering
    \begin{align*}
        \runa{t-var} &\; \infrule{\Gamma_a\left(x\right)=\tau}{\Gamma_a \vdash x : \tau}\\
        %
        \runa{t-const} &\; \infrule{}{\Gamma_a \vdash c : b}\\
        %
        \runa{t-app} &\; \infrule{\Gamma_a \vdash a_1 : \tau_1 \rightarrow \tau_2 \quad \Gamma_a \vdash a_2 : \tau_1}{\Gamma_a \vdash a_1\; a_2 : \tau_2}\\
        %
        \runa{t-lambda} &\; \infrule{\Gamma_a\left[x \mapsto \tau_1\right] \vdash a : \tau_2}{\Gamma_a \vdash \lambda x:\tau_1.a : \tau_1 \rightarrow \tau_2} \\
        %
        \runa{t-break} &\; \infrule{\Gamma_a \vdash a : \tau}{\Gamma_a \vdash \breakpoint{a} : \tau} \\
        %
        \runa{t-hole} &\; \infrule{}{\Gamma_a \vdash \left(\hole : \tau\right) : \tau}
        %
    \end{align*}
    \caption{Type rules for abstract syntax trees.}
    \label{tab:typerules}
\end{table*}

%\section{Type context format}
%Here, we propose a format for type contexts of editor expressions. The context of an editor expression could be a triple $\Psi = (\Gamma_a, \tau, \Gamma)$, where $\Gamma_a$ is the type context for the subtree encapsulated by the cursor, $\tau$ is the type of the subtree and $\Gamma$ is a function or map from prefix command types to editor expression contexts. That is, contexts for editor expressions are recursive. Say we have context $(\Gamma_a, \tau, \Gamma)$. Upon a $\texttt{child}\; 1$ prefix, we \textit{look up} $\texttt{one}$ in $\Gamma$. If $\Gamma(\texttt{one}) = undef$, the expression is not well-typed. Otherwise, we evaluate the prefixed expression in the new context $\Gamma(\texttt{one})$.\\

%We construct the initial context based on the AST in the configuration $\conf{E,\; a}$. Upon a substitution prefix, we modify the context, upon a child or parent prefix, we \textit{move} in the context, and upon a conditioned or recursive expression, we set some of the bindings to $undef$: $\Gamma(T)=undef$.\\

%$\Gamma = T_1 : \Psi_1,...,T_n : \Psi_n$ \\
%$\Psi = (\Gamma_a, \tau, \Gamma)$
%Γ = T1 : Ψ1,..,Tn : Ψn
%Ψ = (Γa, τ, Γ)

\section{Experimental type system}

In this section, we introduce a type system for our editor-calculus. For the type system, we introduce the syntactic categories $\tau \in \mathbf{ATyp}$ to denote types of AST nodes, $T \in \mathbf{CTyp}$ to denote \textit{child} types, and p $\in \mathbf{Pth}$ to denote AST paths.
%
\begin{align*}
    \tau ::=&\; b \mid \tau_1 \rightarrow \tau_2 \mid \breakpoint{\tau} \mid \texttt{indet}\\
    T ::=&\; \texttt{one} \mid \texttt{two}\\
    p ::=&\; p\; T \mid \epsilon
\end{align*}

In addition to the basic and arrow types in $\mathbf{ATyp}$, we include a type for breakpoints, $\breakpoint{\tau}$, and a type to denote indeterminate types, \texttt{indet}. We use $\mathbf{Pth}$ to denote paths in an AST by storing a sequence of \textbb{one} and \textbb{two} which denote if the path goes through the first or second child.\\

We define two sets for contexts in our type system. The first context, $\mathbf{ACtx}$, stores type bindings for variables in the AST. The second context, $\mathbf{ECtx}$, stores, for all available paths so far, a pair of an AST context and the type of the node at the end of the path. We use $\Gamma_a \in \mathbf{ACtx}$ and $\Gamma_e \in \mathbf{ECtx}$ as metavariables for the two contexts. To check if a path $p$ is available in a context $\Gamma_e$, we use the notion $\Gamma_e(p) \neq \text{undef}$. $\mathbf{ACtx}$ and $\mathbf{ECtx}$ are thus defined as the following.
%
\begin{align*}
\mathbf{ACtx} &= \mathbf{Var} \rightharpoonup \mathbf{ATyp}\\
\mathbf{ECtx} &= \mathbf{Pth} \rightharpoonup \left(\mathbf{ACtx} \times \mathbf{ATyp}\right)
\end{align*}

To support our type system, we modify the syntax for AST node modifications by including type annotations for application, abstraction and holes. The new syntax thus becomes the following.
%
\begin{align*}
  D ::= \; & \texttt{var}\;x \mid \texttt{const}\;c \mid \texttt{app} : \tau_1 \rightarrow \tau_2, \tau_1 \mid \texttt{lambda}\; x : \tau_1 \rightarrow \tau_2 \mid \texttt{break} \mid \texttt{hole} : \tau
\end{align*}

To support breakpoint types, we introduce the notion of type consistency into our typesystem. The purpose of consistency in our type system is to ensure breakpoints types are consistent with their respective type, as defined below.
%
\begin{definition}{(Type consistency)}
    We define two types $\tau_1, \tau_2$ to be \textit{consistent}, denoted $\tau_1 \sim \tau_2$, by the following rules.
    \begin{align*}
        \runa{cons-1} \hspace{-1cm}
        \infrule{}{\tau \sim \tau} \hspace{-1cm}
        \runa{cons-2} \hspace{-1cm}
        \infrule{}{\breakpoint{\tau} \sim \tau} \hspace{-1cm}
        \runa{cons-3} \hspace{-1cm}
        \infrule{}{\tau \sim \breakpoint{\tau}} \hspace{-1cm}
        \runa{cons-4}
        \infrule{\tau_1 \sim \tau_1' \quad \tau_2 \sim \tau_2'}{(\tau_1 \rightarrow \tau_2) \sim (\tau_1' \rightarrow \tau_2')}
    \end{align*}
\end{definition}


\begin{table*}[htp]
    \centering
    \begin{align*}
        \runa{ctx-split-1}&\; \infrule{}{\emptyset = p \left(\emptyset\; \circ\; \emptyset\right)}\\
        \runa{ctx-split-2}&\; \infrule{\Gamma_e = p \left({\Gamma_e}_1\; \circ\; {\Gamma_e}_2\right)}{\Gamma_e,\; p\; T_1..T_n: (\Gamma_a,\; \tau) = p \left(\left({\Gamma_e}_1,\; p\; T_1..T_n: (\Gamma_a,\; \tau)\right)\; \circ\; {\Gamma_e}_2\right)}\\
        \runa{ctx-split-3}&\; \infrule{p_1 \neq p_2 \quad \Gamma_e = p_2 \left({\Gamma_e}_1\; \circ\; {\Gamma_e}_2\right)}{\Gamma_e,\; p_1\; T_1..T_n: (\Gamma_a,\; \tau) = p_2 \left({\Gamma_e}_1\; \circ\; \left({\Gamma_e}_2,\; p_1\; T_1..T_n: (\Gamma_a,\; \tau)\right)\right)}\\
        %
        \runa{ctx-update-1}&\; \infrule{}{\Gamma_e = \Gamma_e + \emptyset}\\
        \runa{ctx-update-2}&\; \infrule{\Gamma_e = \left({\Gamma_e}_1,\; p: ({\Gamma_a}_2,\; \tau_2)\right) + {\Gamma_e}_2}{\Gamma_e,\; p: ({\Gamma_a}_1,\; \tau_1) = \left({\Gamma_e}_1,\; p: ({\Gamma_a}_2,\; \tau_2)\right) + {\Gamma_e}_2}\\
        \runa{ctx-update-3}&\; \infrule{\Gamma_e = {\Gamma_e}_1 + {\Gamma_e}_2}{\Gamma_e,\; p: (\Gamma_a,\; \tau) = {\Gamma_e}_1 + \left({\Gamma_e}_2,\; p: (\Gamma_a,\; \tau)\right)}
    \end{align*}
    \caption{Context split and context update for editor contexts.}
    \label{tab:context}
\end{table*}
% We define \textit{type contexts}, $\Gamma_e$ in Table \ref{tab:context} as a mapping from a path $p$ to a pair consisting of an AST context $\Gamma_a$ and AST type $\tau$. We denote the $\Gamma_e, p : (\Gamma_a, \tau)$ as the type context equal to the paths not in the domain of map $\Gamma_e$ except for $p$, where $\Gamma_e(p) = (\Gamma_a, \tau)$. For type contexts we introduce the concept of \textit{context splitting} on a path in terms of $\Gamma_e$ maintained through two sub-contexts $\Gamma_{e1}$ and $\Gamma_{e2}$. For this we require a split-operation $\circ$, defined for two sub-contexts on a path as $\Gamma_e = p(\Gamma_{e1}\; \circ \; \Gamma_{e2})$. Notice the empty context is defined with the symbol $\emptyset$ as in \runa{ctx-split-1}. In rule \runa{ctx-split-2} we have that $p$ is in $\Gamma_{e1}$, but not in $\Gamma_{e2}$. Thus, $p$ is not in $\Gamma = \Gamma_{e1}\; \circ \; \Gamma_{e2}$, which is similarly done for the \runa{ctx-split-3} in terms of $\Gamma_{e1}$.\\

Next we introduce the notion of \textit{context updates} to update bindings in a context with new types for the associated path $p$. We use the addition operator $+$, to denote sum-context $\Gamma$ of two compatible type contexts $\Gamma_{e1}$ and $\Gamma_{e2}$. The rules require linear paths to not have bindings exist in another context. Thus, we can only update a context $\Gamma_{e2}$ iff no bindings for a given path is in context $\Gamma_{e1}$. In rule \runa{ctx-update-2} we have bindings in $\Gamma_{e1}$, which means we cannot add bindings to $\Gamma_{e2}$. However, in rule \runa{ctx-update-3} we allow path bindings in $\Gamma_{e2}$ since no such bindings are in context $\Gamma_{e1}$.

% \begin{equation}
%     depth(e) = \left\{
%         \begin{array}{ll}
%             depth(E) + 1            & \quad if e = (\texttt{child}\; n).E \\
%             depth(E) - 1            & \quad if e = \texttt{parent}.E\\
%             depth(E_1) + depth(E_2) & \quad if e = E_1 \ggg E_2\\
%             depth(E)                & \quad if e = \texttt{rec}\; x.E\\
%             depth(E)                & \quad if e = \pi.E\\
%             0                       & \quad otherwise
%         \end{array}
%     \right.
% \end{equation}

\begin{definition}{(Relative cursor depth)}
    We define the function $depth : \mathbf{Edt} \rightarrow \mathbb{Z}$, from the set of atomic editor expression to the set of integers.
    \begin{align*}
    depth((\texttt{child}\; n).E) &= depth(E) + 1 \\
    depth(\texttt{parent}.E) &= depth(E) - 1 \\
    depth(E_1 \ggg E_2) &= depth(E_1) + depth(E_2) \\
    depth(\texttt{rec}\; x.E) &= depth(E) \\
    depth(\pi.E) &= depth(E) \\
    depth(E) &= 0 
\end{align*}
\end{definition}
The $depth$ function statically analyses the structure of an editor expression to determine the relative depth of the cursor after evaluation of the expression. This function is used to make sure the position of the cursor before and after evaluation of an expression is the same. As the function performs a static analysis, we do not consider conditioned subexpressions. Later, in the type rules, we will see why we can safely ignore conditioned subexpressions. \\


% Next we define the function $match : \mathbf{Aam} \times \mathbf{ACtx} \times \mathbf{ATyp} \rightarrow \{tt, f\!\!f\}$. This function returns true if the type of the given AST modification $D$, is equal to the given AST type $\tau$.  
% \begin{align*}
%     match(\texttt{var}\; x,\;\Gamma_a,\;\tau) &= \left\{\begin{matrix}
%  tt & \text{if}\; \Gamma_a(x) = \tau\\ 
%  f\!\!f & \text{otherwise}
% \end{matrix}\right.\\
%     match(\texttt{const}\; c,\;\Gamma_a,\; b) &= tt\\
%     match(\texttt{app} : \tau_1 \rightarrow \tau_2,\; \tau_1,\;\Gamma_a,\; \tau_2) &= tt\\
%     match(\texttt{lambda}\; x : \tau_1 \rightarrow \tau_2,\;\Gamma_a,\; \tau_1 \rightarrow \tau_2) &= tt\\
%     match(\texttt{break},\;\Gamma_a,\; \tau) &= tt\\
%     match(\texttt{hole} : \tau,\;\Gamma_a,\; \tau) &= tt\\
%     match(D,\; \Gamma_a,\; \tau) &= f\!\!f
% \end{align*}

%\begin{equation*}
%    %context : \left(\mathbf{Aam} \times \mathbf{ACtx}\right) \rightharpoonup %\left(\left(\mathbf{Pth} \rightarrow \left(\left(\mathbf{Var} \rightharpoonup %\mathbf{ATyp}\right) \times \mathbf{ATyp}\right)\right) \cup \{error\}\right)
    %context : \left(\mathbf{Aam} \times \mathbf{ACtx} \times \mathbf{Pth} \right) %\rightharpoonup \mathbf{ECtx}
%\end{equation*}
%\begin{align*}
% context(\texttt{const}\; c,\; \Gamma_a,\; p) =&\; \emptyset\\
%  context(\texttt{hole} : \tau,\; \Gamma_a,\; p) =&\; \emptyset\\
%context(\texttt{var}\; x,\; \Gamma_a,\; p) =&\; \emptyset\\
 %context((\texttt{app} : \tau_1 \rightarrow \tau2,\; \tau_1),\; \Gamma_a,\; p) =&\; %\emptyset,\; p\; \texttt{one} : (\Gamma_a,\; \tau_1 \rightarrow \tau_2),\; p\; \texttt{two} : %(\Gamma_a,\; \tau_1)\\
 %context(\texttt{lambda}\; x : \tau_1 \rightarrow \tau_2,\; \Gamma_a,\; p) =&\; \emptyset,\; %p\; \texttt{one} : ((\Gamma_a,\; x : \tau_1),\; \tau_2)
%\end{align*}
%
%

We define functions \textit{limits} and \textit{follows} to analyze which cursor movement is safe given a condition holds. \textit{limits} finds the set of possible AST node modifiers, on which the cursor may sit, given the condition holds. \textit{follows} gives a set of editor type context bindings guaranteed to be safe, given the cursor sits on AST node modifier $D$. Note that the AST type context is empty and that the node type is $\texttt{indet}$, as we cannot determine such information based on a condition. Thus, besides toggling of breakpoints, substitution is not well-typed at path $p$ if $\Gamma_e(p)=(\emptyset,\; \texttt{indet})$. We can combine functions \textit{limits} and \textit{follows} to provide additional bindings to the editor type context of a conditioned expression $\phi \Rightarrow E$. The intersection of \textit{follows} applied to each AST node modifier $D$ in the set $limits(\phi)$ is the set of bindings guaranteed to be safe, given $\phi$ holds.

\theoremstyle{definition}
\begin{definition}{(Condition constraints)}
We define a function $limits: \mathbf{Eed} \rightarrow \mathcal{P}(\mathbf{Aam})$ from the set of conditions to the power set of the set of AST node modifiers. We assume conditions are in conjunctive normal form.
\begin{align*}
    limits(@D)=&\;\{D\}\\
    limits(\neg @D)=&\;\mathbf{Aam}\setminus \{D\}\\
    limits(\lozenge D)=&\;\{D\} \cup \{\texttt{app},\; \texttt{lambda}\; x,\; \texttt{break}\}\\
    limits(\neg \lozenge D)=&\;\mathbf{Aam}\setminus \{D\}\\
    limits(\Box D)=&\;\{D\} \cup \{\texttt{app},\; \texttt{lambda}\; x,\; \texttt{break}\}\\
    limits(\neg \Box D)=&\;\mathbf{Aam}\setminus \{D\}\\
    limits(\phi_1 \land \phi_2)=&\;limits(\phi_1) \cap limits(\phi_2)\\
    limits(\phi_1 \lor \phi_2)=&\;limits(\phi_1) \cup limits(\phi_2)
\end{align*}
\end{definition}


\theoremstyle{definition}
\begin{definition}{(Safe movement)}
We define a function $follows: \mathbf{Aam} \times \mathbf{Pth} \rightarrow \mathcal{P}\left(\mathbf{Pth} \times \left(\mathbf{ACtx} \times \mathbf{ATyp}\right)\right)$ from the set of pairs of AST node modifiers and paths to the power set of editor context bindings.
\begin{align*}
    \textit{follows}(\texttt{var}\; x,\; p)=&\; \emptyset\\
    \textit{follows}(\texttt{const}\; c,\; p)=&\; \emptyset\\
    \textit{follows}(\texttt{app},\; p)=&\; \{p\; \texttt{one} : (\emptyset,\; \texttt{indet}),\; p\; \texttt{two} : (\emptyset,\; \texttt{indet})\}\\
    \textit{follows}(\texttt{lambda}\; x,\; p)=&\; \{p\; \texttt{one} : (\emptyset,\; \texttt{indet})\}\\
    \textit{follows}(\texttt{break},\; p)=&\; \{p\; \texttt{one} : (\emptyset,\; \texttt{indet})\}\\
    \textit{follows}(\texttt{hole},\; p)=&\; \emptyset
\end{align*}
\end{definition}

%
%
We now introduce the type rules for editor expressions. Type rules for substitution are shown in table \ref{tab:typerulesv2sub} and the remaining rules are shown in table \ref{tab:typerulesv2}. The \texttt{child} n prefix is handled by \runa{t-child-1} and \runa{t-child-2}. Here we check that the cursor movement is viable by looking up the new path in $\Gamma_e$. Notice that the remaining editor expression $E$, is evaluated using the new path. The \texttt{parent} prefix is handled similarly in \runa{t-parent} with the exception being that we deconstruct the path instead of building it. When using recursion we require that the depth of the cursor is unchanged after evaluating the expression. We ensure this in \runa{t-rec} with the side condition $depth(E) = 0$. Similarly, \runa{t-cond} utilizes the same side condition to ensure that the cursor is unaffected by whether the condition holds or not. Notice here that evaluation of the conditioned expression is limited by what can follow the condition if it holds, denoted by $\delta$. Sequential composition is handled by the type rule \runa{t-seq}. Here we split the type context into $\Gamma_{e1}$, which contains information about the current subtree, and $\Gamma{e2}$, which contains information about the rest of the tree. This split ensures that the potentially hazardous evaluation of $E_1$ is kept separate from the evaluation of $E_2$.\\

\begin{table*}[htp]
    \centering
    \begin{align*}
        %
        \runa{t-eval} &\; \infrule{p,\; \Gamma_e \vdash E : ok}{p,\; \Gamma_e \vdash \texttt{eval}.E : ok}\\
        %
        \runa{t-child-1}&\; \infrule{\Gamma_e(p\; \texttt{one}) \neq \text{undef} \quad p\; \texttt{one},\; \Gamma_e \vdash E : ok}{p,\; \Gamma_e \vdash \left(\texttt{child}\; 1\right).E : ok}\\
        %
        \runa{t-child-2}&\; \infrule{\Gamma_e(p\; \texttt{two}) \neq \text{undef} \quad p\; \texttt{one},\; \Gamma_e \vdash E : ok}{p,\; \Gamma_e \vdash \left(\texttt{child}\; 2\right).E : ok}\\
        %
        \runa{t-parent}&\; \infrule{\Gamma_e(p) \neq \text{undef} \quad p,\; \Gamma_e \vdash E : ok}{p\; T,\; \Gamma_e \vdash \texttt{parent}.E : ok}\\
        %
        \runa{t-rec} &\; \condinfrule{p,\; \Gamma_e \vdash E : ok}{p,\; \Gamma_e \vdash \texttt{rec} x.E : ok}{\text{if}\; depth(E) = 0}\\
        %
        \runa{t-cond} &\; \condinfrule{p,\; \Gamma_e + \delta \vdash E : ok}{p,\; \Gamma_e \vdash \phi \Rightarrow E : ok}{\begin{align*}
            \text{if}\; &depth(E) = 0\;\\
            \text{and}\; &\delta = \bigcap_{D \in limits(\phi)}follows(D,\; p)\\
        \end{align*}}\\
        %
        \runa{t-seq} &\; \condinfrule{p,\; {\Gamma_e}_1 \vdash E_1 : ok \quad p,\; {\Gamma_e}_2 \vdash E_2 : ok}{p,\; \Gamma_e \vdash E_1 \ggg E_2 : ok}{\text{where}\; \Gamma_e = p\; ({\Gamma_e}_1\; \circ\; {\Gamma_e}_2)}\\
        %
        \runa{t-ref} &\; \infrule{}{p,\;\Gamma_e \vdash x : ok}\\
        %
        \runa{t-nil} &\; \infrule{}{p,\;\Gamma_e \vdash \mathbf{0} : ok}
    \end{align*}
    \caption{Type rules for editor expressions.}
    \label{tab:typerulesv2}
\end{table*}
%
%
Table \ref{tab:typerulesv2sub} shows the type rules for substitution. For substitution to be well-typed, the AST node type $\tau$ in the type context binding associated with the current path $p$ must be consistent with the type of the AST node modifier to be inserted. In \runa{t-sub-var}, we handle the special case where we insert a variable reference $x$. For this to be well-typed, a binding $\Gamma_a(x)=\tau'$ must exist, such that $\consistent{\tau}{\tau'}$. Note that substitution replaces a subtree of the AST. Thus, the bindings in the editor type context with paths starting with $p$ are no longer valid. Therefore, we split the type context on path $p$, such that $\Gamma_e = p\left({\Gamma_e}_1\;\circ\;{\Gamma_e}_2\right)$, and evaluate the prefixed expression $E$ in the type context ${\Gamma_e}_2$. That is, the type context containing all bindings of $\Gamma_e$ not starting with $p$. Note that the binding with path exactly $p$ is in both ${\Gamma_e}_1$ and ${\Gamma_e}_2$, however. We add bindings to ${\Gamma_e}_2$ in rules $\runa{t-sub-app}$ and $\runa{t-sub-abs}$. Particularly, we expand the AST type context upon substitution for an abstraction.\\

We treat substitution of breakpoints differently, as we can either toggle breakpoints on or off. Furthermore, we do not replace the subtree upon substitution for breakpoints. Instead, we must modify the bindings with paths starting with $p$, to either include or remove a $\texttt{one}$. Additionally, we change the type in the binding at the current path $p$ to indicate whether it has a breakpoint. Note that we toggle off the breakpoint if the type is of the form $\breakpoint{\tau}$, and toggle it on otherwise. Thus, the type indicates the structure of the tree.
%
%
\begin{table}
    \begin{flalign*}
        %
        \runa{t-sub-var} &\; \condinfrule{\Gamma_e(p)=(\Gamma_a,\;\tau) \quad \Gamma_a(x) = \tau' \quad \consistent{\tau}{\tau'} \quad p,\;{\Gamma_e}_2 \vdash E : ok}{p,\; \Gamma_e \vdash \replace{\texttt{var}\; x}.E : ok}{\text{where}\; \Gamma_e = p\; ({\Gamma_e}_1\; \circ\; {\Gamma_e}_2)} \\
        %
        \runa{t-sub-const} &\; \condinfrule{\Gamma_e(p)=(\Gamma_a,\;b) \quad p,\;{\Gamma_e}_2 \vdash E : ok}{p,\; \Gamma_e \vdash \replace{\texttt{const}\; c}.E : ok}{\text{where}\; \Gamma_e = p\; ({\Gamma_e}_1\; \circ\; {\Gamma_e}_2)}\\
        %
        \runa{t-sub-app} &\; \condinfrule{\Gamma_e(p)=(\Gamma_a,\; \tau_2') \quad \consistent{\tau_2}{\tau_2'} \quad p,\; \Gamma_e' \vdash E : ok}{p,\; \Gamma_e \vdash \replace{\texttt{app} : \tau_1 \rightarrow \tau_2,\; \tau_1}.E : ok}{\begin{align*}
            &\text{where}\; \Gamma_e = p\; ({\Gamma_e}_1\; \circ\; {\Gamma_e}_2)\;\\
            &\text{and}\; \Gamma_e' = {\Gamma_e}_2,\; p\; \texttt{one} : (\Gamma_a,\; \tau_1 \rightarrow \tau_2),\; p\; \texttt{two} : (\Gamma_a,\; \tau_1)
        \end{align*}}\\
        %
        \runa{t-sub-abs} &\; \condinfrule{\Gamma_e(p)=(\Gamma_a,\; \tau_1' \rightarrow \tau_2') \quad \consistent{\tau_1 \rightarrow \tau_2}{\tau_1' \rightarrow \tau_2'} \quad p,\; \Gamma_e' \vdash E : ok}{p,\; \Gamma_e \vdash \replace{\texttt{lambda}\; x : \tau_1 \rightarrow \tau_2}.E : ok}{\begin{align*}
        &\text{where}\;\Gamma_e = p\; ({\Gamma_e}_1\; \circ\; {\Gamma_e}_2)\\
        &\text{and}\;\Gamma_e' = {\Gamma_e}_2, p\; \texttt{one} : ((\Gamma_a,\; x : \tau_1),\; \tau_2)\end{align*}} \\
        %
        %\runa{t-sub} &\; \infrule{match(D,\; \Gamma_a,\; \tau) = tt \quad p,\;\Gamma_e' \vdash %E : ok}{p,\;\Gamma_e \vdash \replace{D}.E : ok} \\
        %&\text{if}\; D \neq \texttt{break}\\
        %&\text{and}\; \Gamma_e(p)=(\Gamma_a,\;\tau) \\
        %&\text{and}\; \Gamma_e = p\; ({\Gamma_e}_1\; \circ\; {\Gamma_e}_2)\\
        %&\text{and}\; \Gamma_e' = {\Gamma_e}_2 + context(D,\; \Gamma_a)\\
        %
        \runa{t-sub-break-1} &\; \infrule{\Gamma_e(p)=(\Gamma_a,\; \breakpoint{\tau}) \quad p,\; \Gamma_e' \vdash E : ok}{p,\; \Gamma_e \vdash \replace{\texttt{break}} : ok} \\
        &\text{where}\; \Gamma_e = p\; ({\Gamma_e}_1\; \circ\; {\Gamma_e}_2)\\
        &\text{and}\; {\Gamma_e}_1 = \emptyset,\; p\; \texttt{one}\; T_1..T_{n_1} : ({\Gamma_a}_1,\; \tau_1),..,p\; \texttt{one}\; T_1..T_{n_m} : ({\Gamma_a}_m,\; \tau_m)\\
        &\text{and}\; {\Gamma_e}_1' =\emptyset,\; p\; T_1..T_{n_1} : ({\Gamma_a}_1,\; \tau_1),..,p\; T_1..T_{n_m} : ({\Gamma_a}_m,\; \tau_m)\\
        &\text{and}\; \Gamma_e' = \left({\Gamma_e}_2 + {\Gamma_e}_1'\right),\; p : (\Gamma_a,\; \tau)\\
        %
        \runa{t-sub-break-2} &\; \infrule{\Gamma_e(p)=(\Gamma_a,\;\tau)\quad  p,\; \Gamma_e' \vdash E : ok}{p,\; \Gamma_e \vdash \replace{\texttt{break}} : ok} \\
        &\text{where}\; \Gamma_e = p\; ({\Gamma_e}_1\; \circ\; {\Gamma_e}_2)\\
        &\text{and}\; {\Gamma_e}_1 =\emptyset,\; p\; T_1..T_{n_1} : ({\Gamma_a}_1,\; \tau_1),..,p\; T_1..T_{n_m} : ({\Gamma_a}_m,\; \tau_m)\\
        &\text{and}\; {\Gamma_e}_1' = \emptyset,\; p\; \texttt{one}\; T_1..T_{n_1} : ({\Gamma_a}_1,\; \tau_1),..,p\; \texttt{one}\; T_1..T_{n_m} : ({\Gamma_a}_m,\; \tau_m)\\
        &\text{and}\; \Gamma_e' = \left({\Gamma_e}_2 + {\Gamma_e}_1'\right),\; p : (\Gamma_a,\; \breakpoint{\tau})\\
        %
        \runa{t-sub-hole} &\; \condinfrule{\Gamma_e(p)=(\Gamma_a,\;\tau') \quad \consistent{\tau}{\tau'} \quad p,\;{\Gamma_e}_2 \vdash E : ok}{p,\; \Gamma_e \vdash \replace{\texttt{hole} : \tau}.E : ok}{\text{where}\; \Gamma_e = p\; ({\Gamma_e}_1\; \circ\; {\Gamma_e}_2)}
        %
    \end{flalign*}
    \caption{Type rules for substitution.}
    \label{tab:typerulesv2sub}
\end{table}

%\begin{table*}[htp]
%    \centering
%    \begin{align*}
        %%
        %\runa{t-eval} &\; \infrule{p,\; \Gamma_e \vdash E : ok \dashv p',\; \Gamma_e'}{p,\; \Gamma_e \vdash \texttt{eval}.E : %ok \dashv p',\; \Gamma_e'}\\
        %%
        %\runa{t-sub} &\; \infrule{T=\tau \quad p,\;\Gamma_e'' \vdash E : ok \dashv p',\;\Gamma_e'}{p,\;\Gamma_e \vdash %\replace{D}.E : ok \dashv p',\;\Gamma_e'} \\
        %&\text{where}\; \Gamma_e(p)=(\Gamma_a,\;\tau) \\
        %&\text{and}\; T = type(D,\;\Gamma_a) \\
        %&\text{and}\; \Gamma_e = p\; ({\Gamma_e}_1\; \circ\; {\Gamma_e}_2)\\
        %&\text{and}\; \Gamma_e'' = {\Gamma_e}_1 + context(D,\; \Gamma_a)\\
        %%
        %\runa{t-child-1}&\; \infrule{\Gamma_e(p\; \texttt{one}) \neq undef \quad p,\; \texttt{one},\; \Gamma_e \vdash E : ok %\dashv p',\; \Gamma_e'}{p,\; \Gamma_e \vdash \left(\texttt{child}\; 1\right).E : ok \dashv p',\; \Gamma_e'}\\
        %%
        %\runa{t-child-2}&\; \infrule{\Gamma_e(p\; \texttt{two}) \neq undef \quad p,\; \texttt{one},\; \Gamma_e \vdash E : ok %\dashv p',\; \Gamma_e'}{p,\; \Gamma_e \vdash \left(\texttt{child}\; 2\right).E : ok \dashv p',\; \Gamma_e'}\\
        %%
        %\runa{t-parent}&\; \infrule{\Gamma_e(p) \neq undef \quad p,\; \Gamma_e \vdash E : ok \dashv p',\; \Gamma_e'}{p\; T,\; %\Gamma_e \vdash \texttt{parent}.E : ok \dashv p',\; \Gamma_e'}\\
        %%
        %\runa{t-rec} &\; \condinfrule{p,\; {\Gamma_e}_1 \vdash E : ok \dashv p,\; \Gamma_e'}{p,\; \Gamma_e \vdash \texttt{rec} %x.E : ok \dashv p,\; {\Gamma_e}_2}{\text{where}\; \Gamma_e = p\; ({\Gamma_e}_1\; \circ\; {\Gamma_e}_2)}\\
        %%
        %\runa{t-seq} &\; \infrule{p,\; \Gamma_e \vdash E_1 : ok \dashv p'',\; \Gamma_e'' \quad p'',\; \Gamma_e'' \vdash E_2 : %ok \dashv p',\; \Gamma_e'}{p,\; \Gamma_e \vdash E_1 \ggg E_2 : ok \dashv p',\; \Gamma_e'}\\
        %%
        %\runa{t-cond} &\; \infrule{p,\; {\Gamma_e}_1 + \delta \vdash E : ok \dashv p,\; \Gamma_e'}{p,\; \Gamma_e \vdash \phi %\Rightarrow E : ok \dashv p,\; {\Gamma_e}_2}\\
%        &\text{where}\; \Gamma_e = p\; ({\Gamma_e}_1\; \circ\; {\Gamma_e}_2)\\
%        &\text{and}\; \delta = \bigcap_{D \in limits(\phi)}follows(D)\\
%        %
%        \runa{t-ref} &\; \infrule{}{p,\;\Gamma_e \vdash x : ok \dashv p,\;\Gamma_e}\\
%        %
%        \runa{t-nil} &\; \infrule{}{p,\;\Gamma_e \vdash \mathbf{0} : ok \dashv p,\;\Gamma_e}\\
%    \end{align*}
%    \caption{Type rules for editor expressions.}
%    \label{tab:typerules}
%\end{table*}

\begin{theorem} (Subject reduction)
If $\Gamma_e, \;\Gamma_a \vdash \conf{E,\;a} : ok$ and $\conf{E, a} \xrightarrow{\alpha} \conf{E', a'}$ then $\Gamma_e, \;\Gamma_a \vdash \conf{E',\;a'} : ok$.
\end{theorem}

We define \textit{well-typedness} of a configuration $\conf{E,\;a}$ by the following rule: \\
$\condinfrule{\Gamma_a \vdash a : \tau \quad p,\; \Gamma_e \vdash E : ok}{\Gamma_e, \;\Gamma_a \vdash \conf{E,\;a} : ok}{\begin{align*}
        &\text{where}\;\\
        &\text{and}\;\end{align*}}$
        
        

\chapter{Sized types for parallel complexity}\label{ch:bgts}
In this chapter, we briefly discuss the type system for parallel complexity of message-passing processes introduced in Baillot and Ghyselen \cite{BaillotGhyselen2021}. This type system builds on the foundations of indices and constraint judgements and formalizes parallel complexity analysis of $\pi$-calculus processes. Due to extensive use of subtyping and the challenges involved in verifying and satisfying constraint judgements, substantial modifications must be made to enable type checking and type inference of processes. We address these topics in Chapter \ref{ch:typecheck} and \ref{ch:timeinference}, respectively.\\

The type system for parallel complexity of message-passing processes introduced by Baillot and Ghyselen uses sized types to express parametric complexity of replicated input invocation, and thereby achieves precise bounds on primitively recursive processes: A class of processes behaving as primitively recursive functions. This requires a notion of polymorphism in the message types of replicated inputs. Baillot and Ghyselen introduce size polymorphism by bounding sizes of algebraic terms and synchronizations on channels with indices that may contain index variables representing unknown sizes. We may interpret an index with an index valuation that maps its index variables to naturals, such that the index may be evaluated to a natural number.\\

We first formally define indices and constraints on the valuations of indices. We give both a predicate logic and a model-theoretic interpretation of judgements on such constraints, referring to these as \textit{constraint judgements}. We then define sized types, the subtyping relation and introduce non-algorithmic type rules.

%\section{A type checker}\label{Sec:typesystembg}
\section{Indices and constraint judgements}\label{sec:indicesandjudgements}
In the type system by Baillot and Ghyselen, indices are used to keep track of sizes of inputs received on replicated inputs. As these sizes may be parametric, in that they may be dependent on the sizes of values received on replicated inputs, we view indices as algebraic expressions consisting of index variables $i,j,k\in\mathcal{V}$ ranging over a countable set, and function symbols, using meta-variable $f$, that may represent natural number constants as nullary functions as well as algebraic operators
\begin{align*}
    I,J ::= i \mid f(I_1,I_2,\dots,I_n)
\end{align*}
Each function symbol $f$ has an arity $\text{ar}(f)$ and an interpretation $[\![f]\!] : \mathbb{N}^{\text{ar}(f)} \rightarrow \mathbb{N}$. For the interpretation of binary difference, we assume that $[\![-]\!](n,m) = 0$ when $m \geq n$, which we refer to as the \textit{monus} operator. As indices may contain index variables, we assume some index valuation $\rho : \mathcal{V} \rightarrow \mathbb{N}$, and extend the definition of interpretations to indices, such that $[\![I]\!]_\rho$ is a natural number instance of index $I$, according to index valuation $\rho$, where for all $i$ in $I$, $\rho(i)$ substitutes for $i$ denoted $I\{\rho(i)/i\}$. Index substitution is defined in Definition \ref{def:indexsubstitution}. Based on the structure of the process that indices are used in the typing of, we may be able to establish relationships between the instances of these indices. For instance, a replicated input may receive values of sizes defined by an interval of two indices $[I,J]$. Then, we are only interested in index valuations $\rho$ that satisfy $[\![I]\!]_\rho \leq [\![J]\!]_\rho$. To express such relationships, we define binary constraints on indices in Definition \ref{def:indexconstr}.

\begin{defi}\label{def:indexsubstitution}
    We define index substitution by the following rules
    \begin{align*}
        i\substi{I}{j} &= j \text{ if } i = j\\
        i\substi{I}{j} &= i \text{ if } i \not = j\\
        f(I_1, I_2, \dots, I_n)\substi{J}{i} &= f(I_1\substi{J}{i}, I_2\substi{J}{i}, \dots, I_n\substi{J}{i})
    \end{align*}
\end{defi}

\begin{defi}[Index constraints]\label{def:indexconstr}
    Given a finite set of index variables $\varphi\subset \mathcal{V}$, we define a constraint $C$ on $\varphi$ to be an expression of the form $I \bowtie J$, where $I$ and $J$ are indices with all free index variables in $\varphi$ and $\bowtie\;\in\{\leq,=,\geq\}$ is a binary relation on $\mathbb{N}$. A finite set of constraints is represented by meta-variable $\Phi$.
\end{defi}
%
A constraint $I \bowtie J$ on $\varphi$ is satisfied given an index valuation $\rho : \varphi \longrightarrow \mathbb{N}$ when $[\![I]\!]_\rho \bowtie [\![J]\!]_\rho$ is satisfied, denoted $\rho \vDash I \bowtie J$. For a finite set of constraints $\Phi$, we write $\rho\vDash \Phi$ when $\rho \vDash C$ holds for all $C \in \Phi$. Finally, $\varphi;\Phi\vDash C$ holds when for all index valuations $\rho$ such that $\rho\vDash \Phi$ holds, we also have $\rho\vDash C$. That is, $\varphi;\Phi\vDash C$ holds exactly when $C$ does not impose further restrictions on index valuations on $\varphi$. Such judgements are fundamental to the type system by Baillot and Ghyselen, especially ones of the form $\varphi;\Phi\vDash I \leq J$, as they impose a partial order on indices wrt. how indices may be interpreted. This enables a notion of subtyping for parametric complexities, such that only indices that are greater or equal may substitute, thus preserving upper bounds on the global parallel complexity, as we shall see in the following sections.
%
%\section{The typechecker}





\begin{table*}[!ht]
    \begin{framed}\vspace{-1em}\begin{align*}
        %
        % S-nil
        &\kern-0.5em\runa{U-nil}\infrule{}{\varphi;\Phi;\Gamma \vdash \nil \triangleleft \{0\}}
        % S-nu
        \kern-2em\runa{U-nu}\infrule{\varphi;\Phi;\Gamma, a:T \vdash P \triangleleft \kappa}{\varphi;\Phi;\Gamma \vdash \newvar{a:T}{P} \triangleleft \kappa}
        % S-par
        \kern-1em\runa{U-par}\infrule{\varphi;\Phi;\Gamma \vdash P \triangleleft \kappa \quad \varphi;\Phi;\Delta \vdash Q \triangleleft \kappa'}{\varphi;\Phi;\Gamma \mid \Delta \vdash P \mid Q \triangleleft \text{basis}(\varphi, \Phi,\kappa \cup \kappa')}\\
        % S-match
        &\kern-0.5em\runa{U-match}\infrule{
        \begin{matrix}
            \varphi;\Phi;\Gamma \vdash e:\natinterval{I}{J} \quad \varphi;\Phi, I \leq 0;\Gamma \vdash P \triangleleft \kappa\\
            \varphi;\Phi, J \geq 1;\Gamma, x:\natinterval{I-1}{J-1} \vdash Q \triangleleft \kappa'
        \end{matrix}}{\varphi;\Phi;\Gamma \vdash \match{e}{P}{x}{Q} \triangleleft \text{basis}(\varphi, \Phi, \kappa \cup \kappa')}
        % S-tick
        \kern16em\runa{S-tick}\infrule{\varphi;\Phi;\Gamma \vdash P \triangleleft \kappa}{\varphi;\Phi;\uparrow^1\!\!\Gamma \vdash \tick P \triangleleft \kappa + 1}\\
        % S-iserv
        &\runa{S-iserv}\infrule{\begin{matrix}
            \texttt{in} \in \sigma\quad \varphi;\Phi;\Gamma\vdash a:\servt{I}{i}{\sigma}{K}{\widetilde{T}}\\
            (\varphi, \widetilde{i}); \Phi; \text{ready}(\varphi,\Phi,\tforwardsim{\Gamma}{I}), \widetilde{v} : \widetilde{T} \vdash P \triangleleft \kappa \quad (\varphi,\widetilde{i});\Phi\vDash\kappa \leq K
        \end{matrix}}
        {\varphi;\Phi;\Gamma \vdash \;\bang\inputch{a}{\widetilde{v}}{}{P}\triangleleft \{I\}}
         % S-ich
        \kern15em\runa{S-ich}\infrule{\begin{matrix}
            \texttt{in} \in \sigma\quad \varphi;\Phi;\Gamma \vdash a:\chant{\sigma}{I}{\widetilde{T}}\\
            \varphi; \Phi; \tforwardsim{\Gamma}{I}, \widetilde{v}:\widetilde{T} \vdash P \triangleleft \kappa
        \end{matrix}}
        {\varphi;\Phi;\Gamma \vdash \inputch{a}{\widetilde{v}}{}{P} \triangleleft \kappa + I}\\
        % S-oserv
        &\runa{S-oserv}\infrule{\begin{matrix}
            \texttt{out} \in \sigma\quad \varphi;\Phi;\Gamma\vdash a:\servt{I}{i}{\sigma}{K}{\widetilde{T}}\\
            \varphi; \Phi;(\tforwardsim{\Gamma}{I}) \vdash \widetilde{e}:\widetilde{S} \quad \text{instantiate}(\widetilde{i}, \widetilde{S}) = \{\widetilde{J}/\widetilde{i}\} \quad \varphi;\Phi \vDash \widetilde{S} \sqsubseteq \widetilde{T}
        \end{matrix}}
        {\varphi;\Phi;\Gamma \vdash \asyncoutputch{a}{\widetilde{e}}{}\triangleleft \{K\{\widetilde{J}/\widetilde{i}\} + I\}}
        % S-och
        \kern15em\runa{S-och}\infrule{\begin{matrix}
            \texttt{out} \in \sigma\quad \varphi;\Phi;\Gamma \vdash a:\chant{\sigma}{I}{\widetilde{T}}\\
            \varphi; \Phi; \tforwardsim{\Gamma}{I} \vdash \widetilde{e}:\widetilde{S} \quad \varphi;\Phi \vDash \widetilde{S} \sqsubseteq \widetilde{T}
        \end{matrix}}
        {\varphi;\Phi;\Gamma \vdash \asyncoutputch{a}{\widetilde{e}}{} \triangleleft \{I\}}\\
        % S-annot
        &\runa{S-annot}\infrule{\varphi;\Phi;\tforwardsim{\Gamma}{n}\vdash P \triangleleft \kappa}{\varphi;\Phi;\Gamma\vdash n:P \triangleleft \kappa + n}
    \end{align*}\vspace{-1em}\end{framed}
    \smallskip
    \caption{Usage typing rules for parallel complexity of processes.}
    \label{tab:usageprocesstypingrules}
\end{table*}
\subsection{Alternative formulations of constraint judgements}\label{sec:cjalternativeform}
There are several equivalent formulations of the problem of verifying the judgement $\varphi;\Phi\vDash C_0$. One such formulation is that the judgement holds, when the conjunction of constraints in $\Phi$ implies $C_0$, i.e. assuming that $n \bowtie m$ evaluates to a truth value based on membership in the relation $\bowtie$, the predicate formula $C_1 \land \cdots \land C_n \implies C_0$, where $\Phi = \{C_1,\dots,C_n\}$, must be satisfied for all valuations $\rho$ over $\varphi$. That is, let $C_i = I_i \bowtie_i J_i$, then for any valuation $\rho : \varphi \rightarrow \mathbb{N}$, the formula $([\![I_1]\!]_\rho \bowtie_1 [\![J_1]\!]_\rho) \land \cdots \land ([\![I_n]\!]_\rho \bowtie_n [\![J_n]\!]_\rho) \implies [\![I_0]\!]_\rho \bowtie_0 [\![J_0]\!]_\rho$ must be satisfied. Another interpretation of the problem is that the intersection of the feasible regions of all (inequality) constraints in $\Phi$ must be contained in the feasible region of $C_0$, or equivalently, the set of all valuations over $\varphi$ that satisfy all the constraints in $\Phi$, referred to as the model space of $\Phi$ wrt. $\varphi$, $\mathcal{M}_\varphi(\Phi)$ must be a subset of the model space of $C_0$ wrt. $\varphi$
\begin{equation*}
    \mathcal{M}_\varphi(\Phi) \subseteq \mathcal{M}_\varphi(\{C_0\})\quad\text{where}\quad\mathcal{M}_\varphi(\Phi)=\{\rho : \varphi \rightarrow \mathbb{N} \mid \rho \vDash C\;\text{for}\; C \in \Phi\}
\end{equation*}
or equivalently
\begin{equation*}
    \forall \rho \in \mathcal{M}_\varphi(\Phi) (\rho \in \mathcal{M}_\varphi(\{C_0\}))
\end{equation*}

Finally, given the fact that the current statement of the problem is expressed using a universal quantifier, we can negate the problem, obtaining a problem that can instead be expressed using an existential quantifier by the fact that $\neg \forall x P(x)$ is equivalent to $\exists x \neg P(x)$. This means the problem can also be expressed as 
%
\begin{equation*}
    \neg (\exists \rho \in \mathcal{M}_\varphi(\Phi) (\rho \not\in \mathcal{M}_\varphi(\{C_0\})))
\end{equation*}
or equivalently
\begin{equation*}
    \mathcal{M}_\varphi(\Phi) \cap \mathcal{M}_\varphi'(\{C_0\}) = \emptyset \quad\text{where}\quad
    \begin{matrix}
        \mathcal{M}_\varphi(\Phi)=\{\rho : \varphi \rightarrow \mathbb{N} \mid \rho \vDash C\;\text{for all}\; C \in \Phi\}\\
        \mathcal{M'}_\varphi(\Phi)=\{\rho : \varphi \rightarrow \mathbb{N} \mid \rho \not\vDash C\;\text{for some}\; C \in \Phi\}
    \end{matrix}
\end{equation*}
Notice that $\mathcal{M}_\varphi'(\{C\})$ is equivalent to $\mathcal{M}_\varphi(\{C'\})$ where $C'$ is the inverse constraint of constraint $C$, and so $\mathcal{M}_\varphi(\Phi) \cap \mathcal{M}_\varphi'(\{C_0\}$) = $\mathcal{M}_\varphi(\Phi \cup \{C_0'\})$ given some method to invert constraints. Thus, the problem can also be expressed simply as
\begin{equation*}
    \mathcal{M}_\varphi(\Phi \cup \{C_0'\}) = \emptyset \quad \text{where } C_0' = \text{inverse of } C_0
\end{equation*}

In Example \ref{exmp:judgementsatisfaction}, we show how a judgement can be verified manually using the predicate logic and model-theoretic interpretations of judgements provided above.
%
\begin{examp}\label{exmp:judgementsatisfaction}
    Given index variables $\varphi = \{i, j, k\}$ and constraints $\Phi = \{C_1, C_2, C_3, C_4\}$ where
    \begin{align*}
        C_1 &= i \geq 4\\
        C_2 &= j \geq 2\\
        C_3 &= -k + 3 < 0\\ % k \leq 4
        C_4 &= i + j + k \leq 11
    \end{align*}
    we want to check if $\varphi; \Phi \vDash 2i + j^2 + 3k \geq 20$ always holds. %For this example we assume interpretations are as expected from usual mathematical notation.\\
    Namely, we are interested in verifying whether the constraint $2i + j^2 + 3k \geq 20$ imposes any additional constraints to the index variables $i$, $j$ and $k$ given the existing constraints $C_1$, $C_2$, $C_3$ and $C_4$. In this case, we can notice that the minimum values of $i$, $j$ and $k$ are $4$, $2$ and $4$ respectively. As such, given these constraints, the minimum value $2i + j^2 + 3k$ may evaluate to is $2 \cdot 4 + 2^2 + 3 \cdot 4 = 24$. As such, we can conclude that $\varphi; \Phi \vDash 2i + j^2 + 3k \geq 20$ always holds.\\
    
    We can also consider the predicate logic interpretation of the example. It suffices to only consider the index valuations that satisfy the conjunction of constraints, of which there are four. Here, we represent a valuation $\rho$ as a set of pairs of the form $\{(i,\rho(i)) \mid i\in\varphi\}$, and so we have $\{(i,4),(j,2),(k,4)\}$, $\{(i,5),(j,2),(k,4)\}$, $\{(i,4),(j,3),(k,4)\}$ and $\{(i,4),(j,2),(k,5)\}$. We can then verify the corresponding implications to show that the judgement holds
    %
    \begin{align*}
        (4 \geq 4) \land (2 \geq 2) \land ({-4}+3 < 0) \implies 4+2+4 \leq 11\\
        %
        (5 \geq 4) \land (2 \geq 2) \land ({-4}+3 < 0) \implies 5+2+4 \leq 11\\
        %
        (4 \geq 4) \land (3 \geq 2) \land ({-4}+3 < 0) \implies 4+3+4 \leq 11\\
        %
        (4 \geq 4) \land (2 \geq 2) \land ({-5}+3 < 0) \implies 4+2+5 \leq 11
    \end{align*}
    Or correspondingly in model-theoretic notation
    {\small
    \begin{align*}
        \mathcal{M}_\varphi(\Phi) =&\; \{\{(i,4),(j,2),(k,4)\}, \{(i,5),(j,2),(k,4)\}, \{(i,4),(j,3),(k,4)\}, \{(i,4),(j,2),(k,5)\}\}\\
        \mathcal{M}_\varphi(\{2i+j^2+3k\geq 20\}) =&\; \{\{(i,n_1),(j,n_2),(k,n_3)\} \mid n_1,n_2,n_3\in\mathbb{N},\; 2n_1 + n_2^2 + 3n_3 \geq 20 \}\\
        \mathcal{M}_\varphi(\Phi) \subseteq&\; \mathcal{M}_\varphi(\{2i+j^2+3k\geq 20\})
    \end{align*}}
    % \begin{align*}
    %     \mathcal{M}_\varphi(\Phi) = \left\{\{(i,4),(j,2),(k,4)\}, \{(i,5),(j,2),(k,4)\}, \{(i,4),(j,3),(k,4)\}, \{(i,4),(j,2),(k,5)\}\right\}
    % \end{align*}
    
    We can also solve the inverse of the mode-theoretic interpretation of the problem. Then we want to show that $\mathcal{M}_\varphi(\Phi) \cap \mathcal{M}_\varphi'(\{2i+j^2+3k\geq 20\}) = \emptyset$ or equivalently $\mathcal{M}_\varphi(\Phi \cup \{2i+j^2+3k < 20\}) = \emptyset$. 
    %
    \begin{align*}
        \mathcal{M}_\varphi(\{2i+j^2+3k < 20\}) =&\; \{\{(i,n_1),(j,n_2),(k,n_3)\} \mid n_1,n_2,n_3\in\mathbb{N},\; 2n_1 + n_2^2 + 3n_3 < 20 \}\\
        &\kern-9em\mathcal{M}_\varphi(\Phi) \cap \mathcal{M}_\varphi'(\{2i+j^2+3k\geq 20\}) = \mathcal{M}_\varphi(\Phi \cup \{2i+j^2+3k < 20\}) = \emptyset
    \end{align*}
\end{examp}
\section{Types and subtyping}\label{sec:typesandsubs}
We now introduce the types from the type system of Baillot and Ghyselen. The types include a base type describing naturals as algebraic terms with sizes bounded by an interval consisting of two indices. This enables us to statically reason about how sizes of data structures change throughout reduction of processes, providing us termination guarantees for some forms of recursion. The type system of Baillot and Ghyselen contains lists as an additional base type, however for conciseness of the type system, we only consider naturals.
%
\begin{align*}
    T,S ::=&\; \texttt{Nat}[I,J] \mid \texttt{ch}_I^\sigma(\widetilde{T}) \mid \forall_I\widetilde{i}.\texttt{serv}_K^\sigma(\widetilde{T})
\end{align*}
%
We use input/output types for channels, and we further distinguish between channels that have replicated inputs, i.e. channels that have recursive behavior, and those that do not. We refer to the former as \textit{servers}, and we more specifically require all inputs on such channels to be replicated for technical convenience. Both servers and normal channels are annotated with an index $I$ that for a normal channel represents the number of time steps remaining before the channel synchronizes, and for a server the remaining time before it becomes available. Note that this imposes a temporal linearity constraint onto normal channels, as such channels can synchronize at exactly one time step. For servers we have an additional index $K$ that represents the parametric complexity of invoking the continuation of a replicated input on the server. Finally, the set $\sigma \subseteq \{\texttt{in},\texttt{out}\}$ is a subset of use-capabilities. Since types consist partly of indices, we define index substitution on types in Definition \ref{def:typeindexsubstitution}.\\

\begin{defi}\label{def:typeindexsubstitution}
    We define index substitution on types by the following rules
    \begin{align*}
        \natinterval{I}{J}\substi{K}{i} &= \natinterval{I\substi{K}{i}}{J\substi{K}{i}}\\
        \chant{\sigma}{I}{\widetilde{T}}\substi{J}{i} &= \chant{\sigma}{I\substi{J}{i}}{\widetilde{T}\substi{J}{i}}\\
        \servt{I}{\widetilde{i}}{\sigma}{K}{\widetilde{T}}\substi{J}{j} &= \servt{I\substi{J}{j}}{\widetilde{i}}{\sigma}{K}{\widetilde{T}} \text{ if } j \in \widetilde{i}\\
        \servt{I}{\widetilde{i}}{\sigma}{K}{\widetilde{T}}\substi{J}{j} &= \servt{I\substi{J}{j}}{\widetilde{i}}{\sigma}{K\substi{J}{j}}{\widetilde{T}\substi{J}{j}} \text{ if } j \not\in \widetilde{i}
    \end{align*}
\end{defi}


Subtyping for base types and types is the least reflexive relation $\sqsubseteq$ that satisfies the subtyping rules in Table \ref{tab:subtypeSized}. As the type system should provide upper bounds on the parallel complexity of processes, it is safe to weaken the bounds on the sizes of natural types. That is, we may decrease the lower bound and increase the upper bound on the sizes of such terms. For server and channel types, we may relax use-capabilities and use the subtyping relation on parameter types as well as modify the complexity bounds on servers, depending on the use-capabilities. Servers and channels of input/output capability are invariant, those of input capability are covariant and those of output capability are contravariant. That is, if a server or channel that inputs a value of type $T$ is required, then we can safely use a server or channel that inputs a subtype of $T$, respectively. Conversely, when a server or channel of output capability is required, we can safely use a channel or server that outputs a supertype of the required parameter type \cite{PierceSangiorgi1996}. This becomes apparent when we assume types $\texttt{Integer}$ and $\texttt{Real}$ such that $\texttt{Integer} \sqsubseteq \texttt{Real}$, as any process that receives reals can also safely receive integers, and any process that output reals can also safely output integers. Unlike Baillot and Ghyselen \cite{BaillotGhyselen2021}, we do not discard associations from our type contexts, rather we discard use-capabilities from channels and servers. Thus, to ensure the type checker is sound, we introduce rules $\runa{BGS-cempty}$ and $\runa{BGS-sempty}$ such that channel and server types are super types of ones with no use-capabilities.

%
\begin{table*}[h!]
    \begin{framed}\vspace{-1em}\begin{align*}
        &\kern0em\runa{BGS-nweak}\;\infrule{\varphi;\Phi\vDash I' \leq I\quad\quad \varphi;\Phi\vDash J \leq J'}{
        \varphi;\Phi\vdash \texttt{Nat}[I,J] \sqsubseteq \texttt{Nat}[I',J']}
        %
        \kern3em\runa{BGS-cinvar}\;\infrule{\varphi;\Phi\vdash \widetilde{T}\sqsubseteq\widetilde{S}\quad\quad \varphi;\Phi\vdash \widetilde{S}\sqsubseteq\widetilde{T}}{\varphi;\Phi\vdash\texttt{ch}_I^{\{\texttt{in},\texttt{out}\}}(\widetilde{T}) \sqsubseteq \texttt{ch}_I^{\{\texttt{in},\texttt{out}\}}(\widetilde{S})}\kern7em\\[-1em]
        %
        \vspace{-0.5em}
        &\kern-0em\runa{BGS-ccovar}\;\infrule{\{\texttt{in}\}\subseteq\sigma\quad\varphi;\Phi\vdash \widetilde{T}\sqsubseteq\widetilde{S}}{\varphi;\Phi\vdash \texttt{ch}_I^{\sigma}(\widetilde{T})\sqsubseteq\texttt{ch}_I^{\{\texttt{in}\}}(\widetilde{S})}\quad\quad\runa{BGS-ccontra}\;\infrule{\{\texttt{out}\}\subseteq\sigma\quad\varphi;\Phi\vdash \widetilde{S}\sqsubseteq\widetilde{T}}{\varphi;\Phi\vdash \texttt{ch}_I^{\sigma}(\widetilde{T})\sqsubseteq \texttt{ch}_I^{\{\texttt{out}\}}(\widetilde{S})}\\[-1em]
        %
        &\kern4em\runa{BGS-sinvar}\;\infrule{(\varphi,\widetilde{i});\Phi\vdash \widetilde{T}\sqsubseteq\widetilde{S}\quad\quad (\varphi,\widetilde{i});\Phi\vdash \widetilde{S}\sqsubseteq\widetilde{T}\quad\quad (\varphi,\widetilde{i});\Phi\vDash K = K'}{\varphi;\Phi\vdash
        \forall_I\widetilde{i}.\texttt{serv}^{\{\texttt{in},\texttt{out}\}}_K(\widetilde{T})
        \sqsubseteq \forall_I\widetilde{i}.\texttt{serv}^{\{\texttt{in},\texttt{out}\}}_{K'}(\widetilde{S})}\\[-1em]
        %
        \vspace{-0.5em}
        &\kern5em\runa{BGS-scovar}\;\infrule{\{\texttt{in}\}\subseteq\sigma\quad(\varphi,\widetilde{i});\Phi\vdash \widetilde{T}\sqsubseteq\widetilde{S}\quad (\varphi,\widetilde{i});\Phi\vDash K' \leq K}{\varphi;\Phi\vdash \forall_I\widetilde{i}.\texttt{serv}^{\sigma}_K(\widetilde{T})\sqsubseteq\forall_I\widetilde{i}.\texttt{serv}^{\{\texttt{in}\}}_{K'}(\widetilde{S})}\\[-1em]
        &\kern4.5em\runa{BGS-scontra}\;\infrule{\{\texttt{out}\}\subseteq\sigma\quad(\varphi,\widetilde{i});\Phi\vdash \widetilde{S}\sqsubseteq\widetilde{T}\quad (\varphi,\widetilde{i});\Phi\vDash K \leq K'}{\varphi;\Phi\vdash \forall_I\widetilde{i}.\texttt{serv}^{\sigma}_K(\widetilde{T})\sqsubseteq \forall_I\widetilde{i}.\texttt{serv}^{\{\texttt{out}\}}_{K'}(\widetilde{S})}\\[-1em]
        %
        &\kern0em\runa{BGS-cempty}\;\infrule{}{\varphi;\Phi\vdash \texttt{ch}^\sigma_I(\widetilde{S}) \sqsubseteq \texttt{ch}^\emptyset_I(\widetilde{T})}\quad\runa{BGS-sempty}\;\infrule{}{\varphi;\Phi\vdash \forall_I\widetilde{i}.\texttt{serv}^\sigma_K(\widetilde{S}) \sqsubseteq \forall_I\widetilde{i}.\texttt{serv}^\emptyset_{K'}(\widetilde{T})}
    \end{align*}\vspace{-1em}\end{framed}
    \smallskip
    \caption{Rules for subtyping of base types and types.}
    \label{tab:subtypeSized}
\end{table*}
\section{Non-algorithmic type rules}

We first consider the type rules for expressions, which are shown in Table \ref{tab:sizedtypedexpressiontypes}. The zero term $0$ intuitively receives the type $\texttt{Nat}[0,0]$ and a successor to a natural term has the same type as its predecessor, but with 1 added to its lower and upper bounds. Finally, a variable receives a type if it is bound in the type context.\\
%Lists are typed similarly, aside from the addition of an element base type. For the element type of a list, we simply use the least lower bound and greatest upper bound on the size amongst the elements of the list.

\begin{table*}[ht]
    \begin{framed}\vspace{-1em}\begin{align*}
        &\kern2em
        \runa{BG-nzero}\;\infrule{}{\varphi;\Phi;\Gamma\vdash\withtype{0}{\typenat[0,0]}}\kern0em
        \runa{BG-nsucc}\;\infrule{\varphi;\Phi;\Gamma \vdash \withtype{e}{\typenat[I, J]}}{\varphi;\Phi;\Gamma \vdash \withtype{\succc{e}}{\typenat[I + 1, J + 1]}}\\[-1em]
        %
        &\kern1em\runa{BG-sub}\;\infrule{\varphi;\Phi;\Delta\vdash e : S\quad\quad \varphi;\Phi\vdash \Gamma\sqsubseteq \Delta\quad\quad \varphi;\Phi\vdash S \sqsubseteq T}{\varphi;\Phi;\Gamma\vdash e : T}\kern11em\runa{BG-var}\;\infrule{}{\varphi;\Phi;\Gamma, \withtype{v}{T} \vdash \withtype{v}{T}}
    \end{align*}\vspace{-1em}\end{framed}
    \smallskip
    \caption{Type rules for expressions.}
    \label{tab:sizedtypedexpressiontypes}
\end{table*}

Before introducing the type rules for processes, we first introduce a function $\downarrow^{\varphi;\Phi}_I\!\!(T)$ in Definition \ref{def:delaysized} that \textit{advances} the time of type $T$ by $I$ units of time complexity. For a channel type $\texttt{ch}^\sigma_J(\widetilde{S})$, we subtract $I$ from $J$ whenever we can guarantee that $J\geq I$ under the constraints imposed on $\varphi$ by $\Phi$. Otherwise, the advancement of $I$ units of time complexity is undefined for type $\texttt{ch}^\sigma_J(\widetilde{S})$, to ensure bounds on communication are not violated. For a server type $\forall_J\widetilde{i}.\texttt{serv}^\sigma_K(\widetilde{S})$, corresponding outputs are well-typed for any timestep $I$ with $I\geq J$, and so a server simply loses input capability whenever we cannot guarantee that $J \geq I$. We extend advancement of time to contexts such that $\downarrow^{\varphi;\Phi}_I(\Gamma)(v)=\;\downarrow^{\varphi;\Phi}_I(\Gamma(v))$. When it is clear from context, we may omit $\varphi$ and $\Phi$.

\begin{defi}[Advancement of Time]\label{def:delaysized}
Let $\varphi$ be a set of index variables, $\Phi$ a set of constraints on indices, $T$ a type and $J$ an index. Then $T$ after $J$ units of time complexity, $\susume{T}{\varphi}{\Phi}{I}$, is given by the rules below
\begin{align*}
    \susume{\natinterval{I}{J}}{\varphi}{\Phi}{I} =&\; \natinterval{I}{J}\\
    %
    %\susume{\texttt{List}[J,K](\mathcal{B})}{\varphi}{\Phi}{I} =&\; \texttt{List}[J,K](\mathcal{B})\\
    %
    \susume{\texttt{ch}^\sigma_J(\widetilde{T})}{\varphi}{\Phi}{I} =&\; \left\{ \begin{matrix}
        %\texttt{ch}^\emptyset_J(\widetilde{T}) & \text{if}\; \sigma = \emptyset\\
        %
        \texttt{ch}^\sigma_{J-I}(\widetilde{T}) & \text{if}\; \varphi;\Phi \vDash J \geq I\\
        %
        \texttt{ch}^\emptyset_{0}(\widetilde{T}) & \text{if}\; \varphi;\Phi \nvDash J \geq I
    \end{matrix} \right.\\
    %
    %\texttt{ch}^\sigma_{J-I}(\widetilde{T}) & \text{if}\; \varphi;\Phi \vDash J \geq I\\
    %
    % \susume{\inchanneltypeS{J}{\widetilde{T}}}{\varphi}{\Phi}{I} =&\; 
    %  \inchanneltypeS{J-I}{\widetilde{T}} & \text{if}\; \varphi;\Phi \vDash J \geq I \\
    % %
    % \susume{\outchanneltypeS{J}{\widetilde{T}}}{\varphi}{\Phi}{I} =&\; 
    %  \outchanneltypeS{J-I}{\widetilde{T}} & \text{if}\; \varphi;\Phi \vDash J \geq I \\
    %
    \susume{\forall_J\widetilde{i}.\texttt{serv}^\sigma_K(\widetilde{T})}{\varphi}{\Phi}{I} =&\; \left\{ \begin{matrix}
        \forall_{J-I}\widetilde{i}.\texttt{serv}^\sigma_K(\widetilde{T}) & \text{if}\; \varphi;\Phi \vDash J \geq I\\
        %
        \forall_{J-I}\widetilde{i}.\texttt{serv}^{\sigma \cap \{\texttt{out}\}}_K(\widetilde{T}) & \text{if}\; \varphi;\Phi \nvDash J \geq I
    \end{matrix} \right.
    %  \servS{J - I}{\widetilde{i}}{K}{\widetilde{T}} & \text{if}\; \varphi;\Phi \vDash J \geq I \\
    % %
    % \susume{\servS{J}{\widetilde{i}}{K}{\widetilde{T}}}{\varphi}{\Phi}{I} =&\; \oservS{J - I}{\widetilde{i}}{K}{\widetilde{T}} & \text{if}\; \varphi;\Phi \vDash J \not\geq I \\          
    % %
    % \susume{\iservS{J}{\widetilde{i}}{K}{\widetilde{T}}}{\varphi}{\Phi}{I} =&\; 
    %  \iservS{J - I}{\widetilde{i}}{K}{\widetilde{T}} & \text{if}\; \varphi;\Phi \vDash J \geq I \\
    % %
    % \susume{\oservS{J}{\widetilde{i}}{K}{\widetilde{T}}}{\varphi}{\Phi}{I} =&\; \oservS{J - I}{\widetilde{i}}{K}{\widetilde{T}}
\end{align*}
\end{defi}

\begin{defi}[Time invariance]\label{def:timeinvariance}
Let $\Gamma$ be a type context. We say that $\Gamma$ is \textit{time invariant} if it only contains variables of either base types or server type with time $0$ and use-capabilities $\sigma$ such that $\sigma\subseteq\{\texttt{out}\}$, i.e. $\forall_0\widetilde{i}.\texttt{serv}^{\sigma}_K(\widetilde{T})$ for some index variables $\widetilde{i}$, types $\widetilde{T}$ and index $K$.
\end{defi}

We now present the type rules of the type system by Baillot and Ghyselen, adapted to fit our syntax. Type judgements are of the form $\varphi;\Phi;\Gamma \vdash P \triangleleft K$, which means that process $P$ has complexity $K$ given constraints $\Phi$ with index variables in $\varphi$ and given a type environment $\Gamma$. The type rules are defined in Table \ref{tab:bgprocesstypingrules}. Rule $\runa{BG-iserv}$ handles replicated inputs and ensures that name $a$ is bound to a server type with input capability in the type context. We must also make sure that in the continuation $P$, the type context must be time invariant as the replicated input may be invoked any number of times after $I$ units of time have elapsed. Thus, only free naturals and servers with no input capability are safe. Rule $\runa{BS-ich}$ is similar except we do not require the type context in the continuation to be time invariant as it is only used once. Rule $\runa{BG-oserv}$ types output servers and most notably uses polymorphism in the index variables $\widetilde{i}$. As such, when typing the expressions sent on the server, we must ensure that we can \textit{instantiate} the index variables of the server using a substitution. Finally, type rule $\runa{BG-match}$ shows how index constraints are introduced when typing processes by utilizing information gained from the two branches of the match expression.\\

% Examples
We now show how a server calculating the $n$th digit of the Fibonacci sequence can be typed. Before presenting the process for the implementation of Fibonacci's sequence, we first need to encode addition in the $\pi$-calculus, which we do using the \textit{add} server as follows.
%
\begin{align*}
    P_\text{add}\defeq&\;\bang\inputch{\text{add}}{x,y,r}{}{
        \texttt{match}\; x\; \{
             0 \mapsto \asyncoutputch{r}{y}{};
            \succc{z} \mapsto \asyncoutputch{\text{add}}{z,\succc{y},r}{}\}}
    %
\end{align*}

The \textit{add} server needs three inputs $x$, $y$, and $r$. The parameters $x$ and $y$ represent two naturals to be added, and $r$ represents the channel intended for receiving the result. Note that no ticks are included in the server as we assume that addition can be done in constant time. The following process for calculating the $n$th number of the Fibonacci sequence is a naïve recursive implementation calculating $\textit{fib}(n)=\textit{fib}(n-1)+\textit{fib}(n-2)$. The server takes two parameters $n$ and $r$ where $n$ is the number of the Fibonacci sequence to calculate and $r$ represents the channel intended for receiving the result.
%
\newcommand{\funcf}[0]{l}
\newcommand{\funcg}[0]{l}
\newcommand{\funcgp}[0]{l-1}
\newcommand{\funcgpp}[0]{l-2}
\newcommand{\funcgppp}[0]{l-1}
\begin{align*}
    P_\text{fib}\defeq&\; \bang\inputch{\text{fib}}{n,r}{}{
         \texttt{match}\; n\; \{ 0 \mapsto \asyncoutputch{r}{0}{}\!;\;
              \succc{n_1} \mapsto\\ 
              &\quad\texttt{match}\; n_1\; \{
                    0 \mapsto \asyncoutputch{r}{\succc{0}}{}\!;\;
                    \succc{n_2} \mapsto\\ &\quad\quad\newvar{r_1,r_2,r_3}{(\asyncoutputch{\text{fib}}{n_1,r_1}{}\mid\asyncoutputch{\text{fib}}{n_2,r_2}{}\\
    &\quad\quad\mid\inputch{r_2}{m_2}{}{\inputch{r_1}{m_1}{}{\tick{\asyncoutputch{\text{add}}{m_1,m_2,r_3}{}}}\mid \inputch{r_3}{m_3}{}{\asyncoutputch{r}{m_3}{}}})}\}\}
    }
\end{align*}

Finally we present a type context $\Gamma$ under which the two servers \textit{add} and \textit{fib} are well-typed. Note that even though we use a naïve implementation of the Fibonacci sequence, we can still get a linear bound as the program runs in parallel.
%
\begin{align*}
    \Gamma \defeq&\; \text{add} : \servt{0}{i,j,k}{\{\texttt{in},\texttt{out}\}}{0}{\texttt{Nat}[0,i],\texttt{Nat}[j,k],\channeltypeS{0}{\texttt{Nat}[j,i+k]}},\\
    &\;\text{fib} : \servt{0}{l}{\{\texttt{in},\texttt{out}\}}{\funcf}{\texttt{Nat}[0,l],\channeltypeS{\funcg}{\texttt{Nat}[0,\textit{fib}(l)]}}
\end{align*}

\begin{table*}
    \begin{framed}\vspace{-1em}\begin{align*}
        &\kern46em\\[-2em] % Stretch frame
        &\kern0em\runa{BG-zero}\infrule{}{\varphi;\Phi;\Gamma \vdash \withcomplex{\nil}{0}}\!\!
        \runa{BG-subtype}\;\infrule{\varphi;\Phi;\Delta \vdash \withcomplex{P}{K} \quad \varphi;\Phi \vdash \Gamma \sqsubseteq \Delta \quad \varphi; \Phi \vDash K \leq K'}{\varphi;\Phi;\Gamma \vdash \withcomplex{P}{K'}}
        \\[-1em]
        %
        &\kern-0em\runa{BG-match}\;\infrule{\varphi;\Phi;\Gamma \vdash \withtype{e}{\natinterval{I}{J}} \quad \varphi;\Phi, I \leq 0;\Gamma \vdash \withcomplex{P}{K} \quad \varphi;\Phi, J \geq 1;\Gamma, \withtype{x}{\natinterval{I\monus 1}{J\monus 1}} \vdash \withcomplex{Q}{K}}{\varphi;\Phi;\Gamma \vdash \withcomplex{\match{e}{P}{x}{Q}}{K}}\\[-1em]
        %
        &\kern4em\runa{BG-par}\;\infrule{\varphi;\Phi;\Gamma\vdash P \triangleleft K\quad \varphi;\Phi;\Gamma\vdash Q \triangleleft K}{\varphi;\Phi;\Gamma\vdash \parcomp{P}{Q} \triangleleft K}\quad\quad\quad\quad\quad\quad \runa{BG-tick}\;\infrule{\varphi;\Phi;\susumesim{\Gamma}{1}\vdash P \triangleleft K}{\varphi;\Phi;\Gamma\vdash \tick P \triangleleft K + 1}\\[-1em]
        %
        &\kern-0em\runa{BG-iserv}\;\infrule{\texttt{in}\in\sigma\quad \varphi;\Phi\vdash\;\susumesim{\Gamma}{I},a:\forall_0\widetilde{i}.\texttt{serv}^\sigma_K(\widetilde{T}) \sqsubseteq \Gamma'\;\text{and}\; \Gamma'\;\text{time invariant}\quad \varphi,\widetilde{i};\Phi;\Gamma',\widetilde{v} : \widetilde{T}\vdash P \triangleleft K}{\varphi;\Phi;\Gamma,\Delta,a : \servt{I}{\widetilde{i}}{\sigma}{K}{\widetilde{T}}\vdash\; \bang\inputch{a}{\widetilde{v}}{}{P}\triangleleft I}\\[-1em]
        %
        &\kern-0em\runa{BG-ich}\;\infrule{\texttt{in}\in\sigma\quad \varphi;\Phi;\susumesim{\Gamma}{I},\widetilde{v} : \widetilde{T}, a : \chant{\sigma}{0}{\widetilde{T}}\vdash P \triangleleft K}{\varphi;\Phi;\Gamma, a : \chant{\sigma}{I}{\widetilde{T}}\vdash \inputch{a}{\widetilde{v}}{}{P}\triangleleft K + I}\kern8.5em \runa{BG-och}\;\infrule{\texttt{out}\in\sigma\quad \varphi;\Phi;\susumesim{\Gamma}{I}\vdash \widetilde{e} : \widetilde{T}}{\varphi;\Phi;\Gamma,a:\chant{\sigma}{I}{\widetilde{T}}\vdash \asyncoutputch{a}{\widetilde{e}}{} \triangleleft I}\\[-1em]
        %
        &\kern2em\runa{BG-oserv}\;\infrule{\texttt{out}\in\sigma\quad \varphi;\Phi;\susumesim{\Gamma}{I}\vdash \widetilde{e} : \widetilde{T}\substi{\widetilde{J}}{\widetilde{i}}}{\varphi;\Phi;\Gamma, a : \servt{I}{\widetilde{i}}{\sigma}{K}{\widetilde{T}}\vdash \asyncoutputch{a}{\widetilde{e}}{} \triangleleft K\!\substi{\widetilde{J}}{\widetilde{i}} + I}\kern12em \runa{BG-nu}\;\infrule{\varphi;\Phi;\Gamma,\withtype{a}{T} \vdash \withcomplex{P}{K}}{\varphi;\Phi;\Gamma \vdash \newvar{a}{\withcomplex{P}{K}}}
    \end{align*}\vspace{-1em}\end{framed}
    \smallskip
    \caption{Sized typing rules for parallel complexity of processes.}
    \label{tab:bgprocesstypingrules}
\end{table*}
\section{Examples of invalid configurations}
The following examples are written in the format $\conf{E, a}$, where $E$ is an editor expression and $a$ is the AST on which we apply the editor expression. \\

In equation \ref{condsubproblem} we show how conditioned substitution can cause problems.
\begin{equation}
    \conf{\left(@\texttt{break} \Rightarrow \replace{\texttt{break}}\right) \ggg \texttt{child}\; 1,\; \lambda x.\hole\; \cursor{\breakpoint{c}}} \label{condsubproblem}
\end{equation}
 In the example we check if the cursor is at a breakpoint, and since the check is true we \textit{toggle} the breakpoint thereby making the following \texttt{child} 1 command problematic. The constant c cannot have a child which means this configuration would cause a run-time error. \\
 
In equation \ref{parentproblem} we show how using the \texttt{parent} command can cause problem when the root is unknown.
\begin{equation}
    \conf{\left(\lozenge\texttt{hole} \Rightarrow \texttt{parent}\right) \ggg \texttt{parent},\; \cursor{\lambda x.\hole}\; c} \label{parentproblem}
\end{equation}
In the example we first check if there is a hole in some subtree of the current cursor. This condition holds and we therefore evaluate the \texttt{parent} command resulting in the AST $\cursor{\lambda x.\hole\; c}$. When the next \texttt{parent} command is evaluated we have a run-time error since we are already situated at the root.\\

In equation \ref{astproblem} we show how an editor expression can result in an AST that would cause a run-time error when evaluated.
\begin{equation}
    \conf{\left(\neg\Box(\texttt{lambda}\; x) \Rightarrow \texttt{child}\; 1\right) \ggg \replace{\texttt{var}\; x}.\texttt{eval},\; \cursor{\lambda x.\hole}\; c} \label{astproblem}
\end{equation}
In the example we first check if it is \textbf{not} necessary that the subtree of the cursor contains a lambda expression. This condition does not hold since it is necessary. Since the condition does not hold we do not evaluate the \texttt{child} 1 command, which means the following substitution of \texttt{var} x is problematic. The substitution results in the AST $\cursor{\texttt{var}\; x}\; c$, which causes a run-time error when the command \texttt{eval} is evaluated, since the left child of the function application is no longer a function.
%
\section{Over-approximations}
As we cannot determine statically whether a condition holds, we establish over-approximations to ensure run-time errors cannot occur in well-typed configurations. As equation \ref{parentproblem} shows, conditioned expressions can result in loss of information about the cursor location. As such, we enforce the cursor \textit{depth} in the tree to be the same before and after a conditioned expression. Furthermore, the first cursor movement in a conditioned expression must be a \texttt{child} prefix. As equation \ref{condsubproblem} shows, conditioned substitution also results in loss of information. Thus, we can no longer guarantee that subsequent substitution at a deeper level is well-typed. Similarly, we no longer know of the structure of the subtree, such that we must condition \texttt{child} prefixes.\\

The above discussion leads to the following list of over-approximations:
\begin{itemize}
    \item In conditioned and recursive expressions, the cursor depth must be the same before and after.
    \item In conditioned and recursive expressions, only the subtree encapsulated by the cursor is accessible.
    \item After conditioned substitution, subsequent substitution at a deeper level is no longer valid, and the \texttt{child} prefix command must be conditioned.
\end{itemize}
%
\section{AST type rules}
\begin{table*}[htp]
    \centering
    \begin{align*}
        \runa{t-var} &\; \infrule{\Gamma_a\left(x\right)=\tau}{\Gamma_a \vdash x : \tau}\\
        %
        \runa{t-const} &\; \infrule{}{\Gamma_a \vdash c : b}\\
        %
        \runa{t-app} &\; \infrule{\Gamma_a \vdash a_1 : \tau_1 \rightarrow \tau_2 \quad \Gamma_a \vdash a_2 : \tau_1}{\Gamma_a \vdash a_1\; a_2 : \tau_2}\\
        %
        \runa{t-lambda} &\; \infrule{\Gamma_a\left[x \mapsto \tau_1\right] \vdash a : \tau_2}{\Gamma_a \vdash \lambda x:\tau_1.a : \tau_1 \rightarrow \tau_2} \\
        %
        \runa{t-break} &\; \infrule{\Gamma_a \vdash a : \tau}{\Gamma_a \vdash \breakpoint{a} : \tau} \\
        %
        \runa{t-hole} &\; \infrule{}{\Gamma_a \vdash \left(\hole : \tau\right) : \tau}
        %
    \end{align*}
    \caption{Type rules for abstract syntax trees.}
    \label{tab:typerules}
\end{table*}

%\section{Type context format}
%Here, we propose a format for type contexts of editor expressions. The context of an editor expression could be a triple $\Psi = (\Gamma_a, \tau, \Gamma)$, where $\Gamma_a$ is the type context for the subtree encapsulated by the cursor, $\tau$ is the type of the subtree and $\Gamma$ is a function or map from prefix command types to editor expression contexts. That is, contexts for editor expressions are recursive. Say we have context $(\Gamma_a, \tau, \Gamma)$. Upon a $\texttt{child}\; 1$ prefix, we \textit{look up} $\texttt{one}$ in $\Gamma$. If $\Gamma(\texttt{one}) = undef$, the expression is not well-typed. Otherwise, we evaluate the prefixed expression in the new context $\Gamma(\texttt{one})$.\\

%We construct the initial context based on the AST in the configuration $\conf{E,\; a}$. Upon a substitution prefix, we modify the context, upon a child or parent prefix, we \textit{move} in the context, and upon a conditioned or recursive expression, we set some of the bindings to $undef$: $\Gamma(T)=undef$.\\

%$\Gamma = T_1 : \Psi_1,...,T_n : \Psi_n$ \\
%$\Psi = (\Gamma_a, \tau, \Gamma)$
%Γ = T1 : Ψ1,..,Tn : Ψn
%Ψ = (Γa, τ, Γ)

\section{Experimental type system}

In this section, we introduce a type system for our editor-calculus. For the type system, we introduce the syntactic categories $\tau \in \mathbf{ATyp}$ to denote types of AST nodes, $T \in \mathbf{CTyp}$ to denote \textit{child} types, and p $\in \mathbf{Pth}$ to denote AST paths.
%
\begin{align*}
    \tau ::=&\; b \mid \tau_1 \rightarrow \tau_2 \mid \breakpoint{\tau} \mid \texttt{indet}\\
    T ::=&\; \texttt{one} \mid \texttt{two}\\
    p ::=&\; p\; T \mid \epsilon
\end{align*}

In addition to the basic and arrow types in $\mathbf{ATyp}$, we include a type for breakpoints, $\breakpoint{\tau}$, and a type to denote indeterminate types, \texttt{indet}. We use $\mathbf{Pth}$ to denote paths in an AST by storing a sequence of \textbb{one} and \textbb{two} which denote if the path goes through the first or second child.\\

We define two sets for contexts in our type system. The first context, $\mathbf{ACtx}$, stores type bindings for variables in the AST. The second context, $\mathbf{ECtx}$, stores, for all available paths so far, a pair of an AST context and the type of the node at the end of the path. We use $\Gamma_a \in \mathbf{ACtx}$ and $\Gamma_e \in \mathbf{ECtx}$ as metavariables for the two contexts. To check if a path $p$ is available in a context $\Gamma_e$, we use the notion $\Gamma_e(p) \neq \text{undef}$. $\mathbf{ACtx}$ and $\mathbf{ECtx}$ are thus defined as the following.
%
\begin{align*}
\mathbf{ACtx} &= \mathbf{Var} \rightharpoonup \mathbf{ATyp}\\
\mathbf{ECtx} &= \mathbf{Pth} \rightharpoonup \left(\mathbf{ACtx} \times \mathbf{ATyp}\right)
\end{align*}

To support our type system, we modify the syntax for AST node modifications by including type annotations for application, abstraction and holes. The new syntax thus becomes the following.
%
\begin{align*}
  D ::= \; & \texttt{var}\;x \mid \texttt{const}\;c \mid \texttt{app} : \tau_1 \rightarrow \tau_2, \tau_1 \mid \texttt{lambda}\; x : \tau_1 \rightarrow \tau_2 \mid \texttt{break} \mid \texttt{hole} : \tau
\end{align*}

To support breakpoint types, we introduce the notion of type consistency into our typesystem. The purpose of consistency in our type system is to ensure breakpoints types are consistent with their respective type, as defined below.
%
\begin{definition}{(Type consistency)}
    We define two types $\tau_1, \tau_2$ to be \textit{consistent}, denoted $\tau_1 \sim \tau_2$, by the following rules.
    \begin{align*}
        \runa{cons-1} \hspace{-1cm}
        \infrule{}{\tau \sim \tau} \hspace{-1cm}
        \runa{cons-2} \hspace{-1cm}
        \infrule{}{\breakpoint{\tau} \sim \tau} \hspace{-1cm}
        \runa{cons-3} \hspace{-1cm}
        \infrule{}{\tau \sim \breakpoint{\tau}} \hspace{-1cm}
        \runa{cons-4}
        \infrule{\tau_1 \sim \tau_1' \quad \tau_2 \sim \tau_2'}{(\tau_1 \rightarrow \tau_2) \sim (\tau_1' \rightarrow \tau_2')}
    \end{align*}
\end{definition}


\begin{table*}[htp]
    \centering
    \begin{align*}
        \runa{ctx-split-1}&\; \infrule{}{\emptyset = p \left(\emptyset\; \circ\; \emptyset\right)}\\
        \runa{ctx-split-2}&\; \infrule{\Gamma_e = p \left({\Gamma_e}_1\; \circ\; {\Gamma_e}_2\right)}{\Gamma_e,\; p\; T_1..T_n: (\Gamma_a,\; \tau) = p \left(\left({\Gamma_e}_1,\; p\; T_1..T_n: (\Gamma_a,\; \tau)\right)\; \circ\; {\Gamma_e}_2\right)}\\
        \runa{ctx-split-3}&\; \infrule{p_1 \neq p_2 \quad \Gamma_e = p_2 \left({\Gamma_e}_1\; \circ\; {\Gamma_e}_2\right)}{\Gamma_e,\; p_1\; T_1..T_n: (\Gamma_a,\; \tau) = p_2 \left({\Gamma_e}_1\; \circ\; \left({\Gamma_e}_2,\; p_1\; T_1..T_n: (\Gamma_a,\; \tau)\right)\right)}\\
        %
        \runa{ctx-update-1}&\; \infrule{}{\Gamma_e = \Gamma_e + \emptyset}\\
        \runa{ctx-update-2}&\; \infrule{\Gamma_e = \left({\Gamma_e}_1,\; p: ({\Gamma_a}_2,\; \tau_2)\right) + {\Gamma_e}_2}{\Gamma_e,\; p: ({\Gamma_a}_1,\; \tau_1) = \left({\Gamma_e}_1,\; p: ({\Gamma_a}_2,\; \tau_2)\right) + {\Gamma_e}_2}\\
        \runa{ctx-update-3}&\; \infrule{\Gamma_e = {\Gamma_e}_1 + {\Gamma_e}_2}{\Gamma_e,\; p: (\Gamma_a,\; \tau) = {\Gamma_e}_1 + \left({\Gamma_e}_2,\; p: (\Gamma_a,\; \tau)\right)}
    \end{align*}
    \caption{Context split and context update for editor contexts.}
    \label{tab:context}
\end{table*}
% We define \textit{type contexts}, $\Gamma_e$ in Table \ref{tab:context} as a mapping from a path $p$ to a pair consisting of an AST context $\Gamma_a$ and AST type $\tau$. We denote the $\Gamma_e, p : (\Gamma_a, \tau)$ as the type context equal to the paths not in the domain of map $\Gamma_e$ except for $p$, where $\Gamma_e(p) = (\Gamma_a, \tau)$. For type contexts we introduce the concept of \textit{context splitting} on a path in terms of $\Gamma_e$ maintained through two sub-contexts $\Gamma_{e1}$ and $\Gamma_{e2}$. For this we require a split-operation $\circ$, defined for two sub-contexts on a path as $\Gamma_e = p(\Gamma_{e1}\; \circ \; \Gamma_{e2})$. Notice the empty context is defined with the symbol $\emptyset$ as in \runa{ctx-split-1}. In rule \runa{ctx-split-2} we have that $p$ is in $\Gamma_{e1}$, but not in $\Gamma_{e2}$. Thus, $p$ is not in $\Gamma = \Gamma_{e1}\; \circ \; \Gamma_{e2}$, which is similarly done for the \runa{ctx-split-3} in terms of $\Gamma_{e1}$.\\

Next we introduce the notion of \textit{context updates} to update bindings in a context with new types for the associated path $p$. We use the addition operator $+$, to denote sum-context $\Gamma$ of two compatible type contexts $\Gamma_{e1}$ and $\Gamma_{e2}$. The rules require linear paths to not have bindings exist in another context. Thus, we can only update a context $\Gamma_{e2}$ iff no bindings for a given path is in context $\Gamma_{e1}$. In rule \runa{ctx-update-2} we have bindings in $\Gamma_{e1}$, which means we cannot add bindings to $\Gamma_{e2}$. However, in rule \runa{ctx-update-3} we allow path bindings in $\Gamma_{e2}$ since no such bindings are in context $\Gamma_{e1}$.

% \begin{equation}
%     depth(e) = \left\{
%         \begin{array}{ll}
%             depth(E) + 1            & \quad if e = (\texttt{child}\; n).E \\
%             depth(E) - 1            & \quad if e = \texttt{parent}.E\\
%             depth(E_1) + depth(E_2) & \quad if e = E_1 \ggg E_2\\
%             depth(E)                & \quad if e = \texttt{rec}\; x.E\\
%             depth(E)                & \quad if e = \pi.E\\
%             0                       & \quad otherwise
%         \end{array}
%     \right.
% \end{equation}

\begin{definition}{(Relative cursor depth)}
    We define the function $depth : \mathbf{Edt} \rightarrow \mathbb{Z}$, from the set of atomic editor expression to the set of integers.
    \begin{align*}
    depth((\texttt{child}\; n).E) &= depth(E) + 1 \\
    depth(\texttt{parent}.E) &= depth(E) - 1 \\
    depth(E_1 \ggg E_2) &= depth(E_1) + depth(E_2) \\
    depth(\texttt{rec}\; x.E) &= depth(E) \\
    depth(\pi.E) &= depth(E) \\
    depth(E) &= 0 
\end{align*}
\end{definition}
The $depth$ function statically analyses the structure of an editor expression to determine the relative depth of the cursor after evaluation of the expression. This function is used to make sure the position of the cursor before and after evaluation of an expression is the same. As the function performs a static analysis, we do not consider conditioned subexpressions. Later, in the type rules, we will see why we can safely ignore conditioned subexpressions. \\


% Next we define the function $match : \mathbf{Aam} \times \mathbf{ACtx} \times \mathbf{ATyp} \rightarrow \{tt, f\!\!f\}$. This function returns true if the type of the given AST modification $D$, is equal to the given AST type $\tau$.  
% \begin{align*}
%     match(\texttt{var}\; x,\;\Gamma_a,\;\tau) &= \left\{\begin{matrix}
%  tt & \text{if}\; \Gamma_a(x) = \tau\\ 
%  f\!\!f & \text{otherwise}
% \end{matrix}\right.\\
%     match(\texttt{const}\; c,\;\Gamma_a,\; b) &= tt\\
%     match(\texttt{app} : \tau_1 \rightarrow \tau_2,\; \tau_1,\;\Gamma_a,\; \tau_2) &= tt\\
%     match(\texttt{lambda}\; x : \tau_1 \rightarrow \tau_2,\;\Gamma_a,\; \tau_1 \rightarrow \tau_2) &= tt\\
%     match(\texttt{break},\;\Gamma_a,\; \tau) &= tt\\
%     match(\texttt{hole} : \tau,\;\Gamma_a,\; \tau) &= tt\\
%     match(D,\; \Gamma_a,\; \tau) &= f\!\!f
% \end{align*}

%\begin{equation*}
%    %context : \left(\mathbf{Aam} \times \mathbf{ACtx}\right) \rightharpoonup %\left(\left(\mathbf{Pth} \rightarrow \left(\left(\mathbf{Var} \rightharpoonup %\mathbf{ATyp}\right) \times \mathbf{ATyp}\right)\right) \cup \{error\}\right)
    %context : \left(\mathbf{Aam} \times \mathbf{ACtx} \times \mathbf{Pth} \right) %\rightharpoonup \mathbf{ECtx}
%\end{equation*}
%\begin{align*}
% context(\texttt{const}\; c,\; \Gamma_a,\; p) =&\; \emptyset\\
%  context(\texttt{hole} : \tau,\; \Gamma_a,\; p) =&\; \emptyset\\
%context(\texttt{var}\; x,\; \Gamma_a,\; p) =&\; \emptyset\\
 %context((\texttt{app} : \tau_1 \rightarrow \tau2,\; \tau_1),\; \Gamma_a,\; p) =&\; %\emptyset,\; p\; \texttt{one} : (\Gamma_a,\; \tau_1 \rightarrow \tau_2),\; p\; \texttt{two} : %(\Gamma_a,\; \tau_1)\\
 %context(\texttt{lambda}\; x : \tau_1 \rightarrow \tau_2,\; \Gamma_a,\; p) =&\; \emptyset,\; %p\; \texttt{one} : ((\Gamma_a,\; x : \tau_1),\; \tau_2)
%\end{align*}
%
%

We define functions \textit{limits} and \textit{follows} to analyze which cursor movement is safe given a condition holds. \textit{limits} finds the set of possible AST node modifiers, on which the cursor may sit, given the condition holds. \textit{follows} gives a set of editor type context bindings guaranteed to be safe, given the cursor sits on AST node modifier $D$. Note that the AST type context is empty and that the node type is $\texttt{indet}$, as we cannot determine such information based on a condition. Thus, besides toggling of breakpoints, substitution is not well-typed at path $p$ if $\Gamma_e(p)=(\emptyset,\; \texttt{indet})$. We can combine functions \textit{limits} and \textit{follows} to provide additional bindings to the editor type context of a conditioned expression $\phi \Rightarrow E$. The intersection of \textit{follows} applied to each AST node modifier $D$ in the set $limits(\phi)$ is the set of bindings guaranteed to be safe, given $\phi$ holds.

\theoremstyle{definition}
\begin{definition}{(Condition constraints)}
We define a function $limits: \mathbf{Eed} \rightarrow \mathcal{P}(\mathbf{Aam})$ from the set of conditions to the power set of the set of AST node modifiers. We assume conditions are in conjunctive normal form.
\begin{align*}
    limits(@D)=&\;\{D\}\\
    limits(\neg @D)=&\;\mathbf{Aam}\setminus \{D\}\\
    limits(\lozenge D)=&\;\{D\} \cup \{\texttt{app},\; \texttt{lambda}\; x,\; \texttt{break}\}\\
    limits(\neg \lozenge D)=&\;\mathbf{Aam}\setminus \{D\}\\
    limits(\Box D)=&\;\{D\} \cup \{\texttt{app},\; \texttt{lambda}\; x,\; \texttt{break}\}\\
    limits(\neg \Box D)=&\;\mathbf{Aam}\setminus \{D\}\\
    limits(\phi_1 \land \phi_2)=&\;limits(\phi_1) \cap limits(\phi_2)\\
    limits(\phi_1 \lor \phi_2)=&\;limits(\phi_1) \cup limits(\phi_2)
\end{align*}
\end{definition}


\theoremstyle{definition}
\begin{definition}{(Safe movement)}
We define a function $follows: \mathbf{Aam} \times \mathbf{Pth} \rightarrow \mathcal{P}\left(\mathbf{Pth} \times \left(\mathbf{ACtx} \times \mathbf{ATyp}\right)\right)$ from the set of pairs of AST node modifiers and paths to the power set of editor context bindings.
\begin{align*}
    \textit{follows}(\texttt{var}\; x,\; p)=&\; \emptyset\\
    \textit{follows}(\texttt{const}\; c,\; p)=&\; \emptyset\\
    \textit{follows}(\texttt{app},\; p)=&\; \{p\; \texttt{one} : (\emptyset,\; \texttt{indet}),\; p\; \texttt{two} : (\emptyset,\; \texttt{indet})\}\\
    \textit{follows}(\texttt{lambda}\; x,\; p)=&\; \{p\; \texttt{one} : (\emptyset,\; \texttt{indet})\}\\
    \textit{follows}(\texttt{break},\; p)=&\; \{p\; \texttt{one} : (\emptyset,\; \texttt{indet})\}\\
    \textit{follows}(\texttt{hole},\; p)=&\; \emptyset
\end{align*}
\end{definition}

%
%
We now introduce the type rules for editor expressions. Type rules for substitution are shown in table \ref{tab:typerulesv2sub} and the remaining rules are shown in table \ref{tab:typerulesv2}. The \texttt{child} n prefix is handled by \runa{t-child-1} and \runa{t-child-2}. Here we check that the cursor movement is viable by looking up the new path in $\Gamma_e$. Notice that the remaining editor expression $E$, is evaluated using the new path. The \texttt{parent} prefix is handled similarly in \runa{t-parent} with the exception being that we deconstruct the path instead of building it. When using recursion we require that the depth of the cursor is unchanged after evaluating the expression. We ensure this in \runa{t-rec} with the side condition $depth(E) = 0$. Similarly, \runa{t-cond} utilizes the same side condition to ensure that the cursor is unaffected by whether the condition holds or not. Notice here that evaluation of the conditioned expression is limited by what can follow the condition if it holds, denoted by $\delta$. Sequential composition is handled by the type rule \runa{t-seq}. Here we split the type context into $\Gamma_{e1}$, which contains information about the current subtree, and $\Gamma{e2}$, which contains information about the rest of the tree. This split ensures that the potentially hazardous evaluation of $E_1$ is kept separate from the evaluation of $E_2$.\\

\begin{table*}[htp]
    \centering
    \begin{align*}
        %
        \runa{t-eval} &\; \infrule{p,\; \Gamma_e \vdash E : ok}{p,\; \Gamma_e \vdash \texttt{eval}.E : ok}\\
        %
        \runa{t-child-1}&\; \infrule{\Gamma_e(p\; \texttt{one}) \neq \text{undef} \quad p\; \texttt{one},\; \Gamma_e \vdash E : ok}{p,\; \Gamma_e \vdash \left(\texttt{child}\; 1\right).E : ok}\\
        %
        \runa{t-child-2}&\; \infrule{\Gamma_e(p\; \texttt{two}) \neq \text{undef} \quad p\; \texttt{one},\; \Gamma_e \vdash E : ok}{p,\; \Gamma_e \vdash \left(\texttt{child}\; 2\right).E : ok}\\
        %
        \runa{t-parent}&\; \infrule{\Gamma_e(p) \neq \text{undef} \quad p,\; \Gamma_e \vdash E : ok}{p\; T,\; \Gamma_e \vdash \texttt{parent}.E : ok}\\
        %
        \runa{t-rec} &\; \condinfrule{p,\; \Gamma_e \vdash E : ok}{p,\; \Gamma_e \vdash \texttt{rec} x.E : ok}{\text{if}\; depth(E) = 0}\\
        %
        \runa{t-cond} &\; \condinfrule{p,\; \Gamma_e + \delta \vdash E : ok}{p,\; \Gamma_e \vdash \phi \Rightarrow E : ok}{\begin{align*}
            \text{if}\; &depth(E) = 0\;\\
            \text{and}\; &\delta = \bigcap_{D \in limits(\phi)}follows(D,\; p)\\
        \end{align*}}\\
        %
        \runa{t-seq} &\; \condinfrule{p,\; {\Gamma_e}_1 \vdash E_1 : ok \quad p,\; {\Gamma_e}_2 \vdash E_2 : ok}{p,\; \Gamma_e \vdash E_1 \ggg E_2 : ok}{\text{where}\; \Gamma_e = p\; ({\Gamma_e}_1\; \circ\; {\Gamma_e}_2)}\\
        %
        \runa{t-ref} &\; \infrule{}{p,\;\Gamma_e \vdash x : ok}\\
        %
        \runa{t-nil} &\; \infrule{}{p,\;\Gamma_e \vdash \mathbf{0} : ok}
    \end{align*}
    \caption{Type rules for editor expressions.}
    \label{tab:typerulesv2}
\end{table*}
%
%
Table \ref{tab:typerulesv2sub} shows the type rules for substitution. For substitution to be well-typed, the AST node type $\tau$ in the type context binding associated with the current path $p$ must be consistent with the type of the AST node modifier to be inserted. In \runa{t-sub-var}, we handle the special case where we insert a variable reference $x$. For this to be well-typed, a binding $\Gamma_a(x)=\tau'$ must exist, such that $\consistent{\tau}{\tau'}$. Note that substitution replaces a subtree of the AST. Thus, the bindings in the editor type context with paths starting with $p$ are no longer valid. Therefore, we split the type context on path $p$, such that $\Gamma_e = p\left({\Gamma_e}_1\;\circ\;{\Gamma_e}_2\right)$, and evaluate the prefixed expression $E$ in the type context ${\Gamma_e}_2$. That is, the type context containing all bindings of $\Gamma_e$ not starting with $p$. Note that the binding with path exactly $p$ is in both ${\Gamma_e}_1$ and ${\Gamma_e}_2$, however. We add bindings to ${\Gamma_e}_2$ in rules $\runa{t-sub-app}$ and $\runa{t-sub-abs}$. Particularly, we expand the AST type context upon substitution for an abstraction.\\

We treat substitution of breakpoints differently, as we can either toggle breakpoints on or off. Furthermore, we do not replace the subtree upon substitution for breakpoints. Instead, we must modify the bindings with paths starting with $p$, to either include or remove a $\texttt{one}$. Additionally, we change the type in the binding at the current path $p$ to indicate whether it has a breakpoint. Note that we toggle off the breakpoint if the type is of the form $\breakpoint{\tau}$, and toggle it on otherwise. Thus, the type indicates the structure of the tree.
%
%
\begin{table}
    \begin{flalign*}
        %
        \runa{t-sub-var} &\; \condinfrule{\Gamma_e(p)=(\Gamma_a,\;\tau) \quad \Gamma_a(x) = \tau' \quad \consistent{\tau}{\tau'} \quad p,\;{\Gamma_e}_2 \vdash E : ok}{p,\; \Gamma_e \vdash \replace{\texttt{var}\; x}.E : ok}{\text{where}\; \Gamma_e = p\; ({\Gamma_e}_1\; \circ\; {\Gamma_e}_2)} \\
        %
        \runa{t-sub-const} &\; \condinfrule{\Gamma_e(p)=(\Gamma_a,\;b) \quad p,\;{\Gamma_e}_2 \vdash E : ok}{p,\; \Gamma_e \vdash \replace{\texttt{const}\; c}.E : ok}{\text{where}\; \Gamma_e = p\; ({\Gamma_e}_1\; \circ\; {\Gamma_e}_2)}\\
        %
        \runa{t-sub-app} &\; \condinfrule{\Gamma_e(p)=(\Gamma_a,\; \tau_2') \quad \consistent{\tau_2}{\tau_2'} \quad p,\; \Gamma_e' \vdash E : ok}{p,\; \Gamma_e \vdash \replace{\texttt{app} : \tau_1 \rightarrow \tau_2,\; \tau_1}.E : ok}{\begin{align*}
            &\text{where}\; \Gamma_e = p\; ({\Gamma_e}_1\; \circ\; {\Gamma_e}_2)\;\\
            &\text{and}\; \Gamma_e' = {\Gamma_e}_2,\; p\; \texttt{one} : (\Gamma_a,\; \tau_1 \rightarrow \tau_2),\; p\; \texttt{two} : (\Gamma_a,\; \tau_1)
        \end{align*}}\\
        %
        \runa{t-sub-abs} &\; \condinfrule{\Gamma_e(p)=(\Gamma_a,\; \tau_1' \rightarrow \tau_2') \quad \consistent{\tau_1 \rightarrow \tau_2}{\tau_1' \rightarrow \tau_2'} \quad p,\; \Gamma_e' \vdash E : ok}{p,\; \Gamma_e \vdash \replace{\texttt{lambda}\; x : \tau_1 \rightarrow \tau_2}.E : ok}{\begin{align*}
        &\text{where}\;\Gamma_e = p\; ({\Gamma_e}_1\; \circ\; {\Gamma_e}_2)\\
        &\text{and}\;\Gamma_e' = {\Gamma_e}_2, p\; \texttt{one} : ((\Gamma_a,\; x : \tau_1),\; \tau_2)\end{align*}} \\
        %
        %\runa{t-sub} &\; \infrule{match(D,\; \Gamma_a,\; \tau) = tt \quad p,\;\Gamma_e' \vdash %E : ok}{p,\;\Gamma_e \vdash \replace{D}.E : ok} \\
        %&\text{if}\; D \neq \texttt{break}\\
        %&\text{and}\; \Gamma_e(p)=(\Gamma_a,\;\tau) \\
        %&\text{and}\; \Gamma_e = p\; ({\Gamma_e}_1\; \circ\; {\Gamma_e}_2)\\
        %&\text{and}\; \Gamma_e' = {\Gamma_e}_2 + context(D,\; \Gamma_a)\\
        %
        \runa{t-sub-break-1} &\; \infrule{\Gamma_e(p)=(\Gamma_a,\; \breakpoint{\tau}) \quad p,\; \Gamma_e' \vdash E : ok}{p,\; \Gamma_e \vdash \replace{\texttt{break}} : ok} \\
        &\text{where}\; \Gamma_e = p\; ({\Gamma_e}_1\; \circ\; {\Gamma_e}_2)\\
        &\text{and}\; {\Gamma_e}_1 = \emptyset,\; p\; \texttt{one}\; T_1..T_{n_1} : ({\Gamma_a}_1,\; \tau_1),..,p\; \texttt{one}\; T_1..T_{n_m} : ({\Gamma_a}_m,\; \tau_m)\\
        &\text{and}\; {\Gamma_e}_1' =\emptyset,\; p\; T_1..T_{n_1} : ({\Gamma_a}_1,\; \tau_1),..,p\; T_1..T_{n_m} : ({\Gamma_a}_m,\; \tau_m)\\
        &\text{and}\; \Gamma_e' = \left({\Gamma_e}_2 + {\Gamma_e}_1'\right),\; p : (\Gamma_a,\; \tau)\\
        %
        \runa{t-sub-break-2} &\; \infrule{\Gamma_e(p)=(\Gamma_a,\;\tau)\quad  p,\; \Gamma_e' \vdash E : ok}{p,\; \Gamma_e \vdash \replace{\texttt{break}} : ok} \\
        &\text{where}\; \Gamma_e = p\; ({\Gamma_e}_1\; \circ\; {\Gamma_e}_2)\\
        &\text{and}\; {\Gamma_e}_1 =\emptyset,\; p\; T_1..T_{n_1} : ({\Gamma_a}_1,\; \tau_1),..,p\; T_1..T_{n_m} : ({\Gamma_a}_m,\; \tau_m)\\
        &\text{and}\; {\Gamma_e}_1' = \emptyset,\; p\; \texttt{one}\; T_1..T_{n_1} : ({\Gamma_a}_1,\; \tau_1),..,p\; \texttt{one}\; T_1..T_{n_m} : ({\Gamma_a}_m,\; \tau_m)\\
        &\text{and}\; \Gamma_e' = \left({\Gamma_e}_2 + {\Gamma_e}_1'\right),\; p : (\Gamma_a,\; \breakpoint{\tau})\\
        %
        \runa{t-sub-hole} &\; \condinfrule{\Gamma_e(p)=(\Gamma_a,\;\tau') \quad \consistent{\tau}{\tau'} \quad p,\;{\Gamma_e}_2 \vdash E : ok}{p,\; \Gamma_e \vdash \replace{\texttt{hole} : \tau}.E : ok}{\text{where}\; \Gamma_e = p\; ({\Gamma_e}_1\; \circ\; {\Gamma_e}_2)}
        %
    \end{flalign*}
    \caption{Type rules for substitution.}
    \label{tab:typerulesv2sub}
\end{table}

%\begin{table*}[htp]
%    \centering
%    \begin{align*}
        %%
        %\runa{t-eval} &\; \infrule{p,\; \Gamma_e \vdash E : ok \dashv p',\; \Gamma_e'}{p,\; \Gamma_e \vdash \texttt{eval}.E : %ok \dashv p',\; \Gamma_e'}\\
        %%
        %\runa{t-sub} &\; \infrule{T=\tau \quad p,\;\Gamma_e'' \vdash E : ok \dashv p',\;\Gamma_e'}{p,\;\Gamma_e \vdash %\replace{D}.E : ok \dashv p',\;\Gamma_e'} \\
        %&\text{where}\; \Gamma_e(p)=(\Gamma_a,\;\tau) \\
        %&\text{and}\; T = type(D,\;\Gamma_a) \\
        %&\text{and}\; \Gamma_e = p\; ({\Gamma_e}_1\; \circ\; {\Gamma_e}_2)\\
        %&\text{and}\; \Gamma_e'' = {\Gamma_e}_1 + context(D,\; \Gamma_a)\\
        %%
        %\runa{t-child-1}&\; \infrule{\Gamma_e(p\; \texttt{one}) \neq undef \quad p,\; \texttt{one},\; \Gamma_e \vdash E : ok %\dashv p',\; \Gamma_e'}{p,\; \Gamma_e \vdash \left(\texttt{child}\; 1\right).E : ok \dashv p',\; \Gamma_e'}\\
        %%
        %\runa{t-child-2}&\; \infrule{\Gamma_e(p\; \texttt{two}) \neq undef \quad p,\; \texttt{one},\; \Gamma_e \vdash E : ok %\dashv p',\; \Gamma_e'}{p,\; \Gamma_e \vdash \left(\texttt{child}\; 2\right).E : ok \dashv p',\; \Gamma_e'}\\
        %%
        %\runa{t-parent}&\; \infrule{\Gamma_e(p) \neq undef \quad p,\; \Gamma_e \vdash E : ok \dashv p',\; \Gamma_e'}{p\; T,\; %\Gamma_e \vdash \texttt{parent}.E : ok \dashv p',\; \Gamma_e'}\\
        %%
        %\runa{t-rec} &\; \condinfrule{p,\; {\Gamma_e}_1 \vdash E : ok \dashv p,\; \Gamma_e'}{p,\; \Gamma_e \vdash \texttt{rec} %x.E : ok \dashv p,\; {\Gamma_e}_2}{\text{where}\; \Gamma_e = p\; ({\Gamma_e}_1\; \circ\; {\Gamma_e}_2)}\\
        %%
        %\runa{t-seq} &\; \infrule{p,\; \Gamma_e \vdash E_1 : ok \dashv p'',\; \Gamma_e'' \quad p'',\; \Gamma_e'' \vdash E_2 : %ok \dashv p',\; \Gamma_e'}{p,\; \Gamma_e \vdash E_1 \ggg E_2 : ok \dashv p',\; \Gamma_e'}\\
        %%
        %\runa{t-cond} &\; \infrule{p,\; {\Gamma_e}_1 + \delta \vdash E : ok \dashv p,\; \Gamma_e'}{p,\; \Gamma_e \vdash \phi %\Rightarrow E : ok \dashv p,\; {\Gamma_e}_2}\\
%        &\text{where}\; \Gamma_e = p\; ({\Gamma_e}_1\; \circ\; {\Gamma_e}_2)\\
%        &\text{and}\; \delta = \bigcap_{D \in limits(\phi)}follows(D)\\
%        %
%        \runa{t-ref} &\; \infrule{}{p,\;\Gamma_e \vdash x : ok \dashv p,\;\Gamma_e}\\
%        %
%        \runa{t-nil} &\; \infrule{}{p,\;\Gamma_e \vdash \mathbf{0} : ok \dashv p,\;\Gamma_e}\\
%    \end{align*}
%    \caption{Type rules for editor expressions.}
%    \label{tab:typerules}
%\end{table*}

\begin{theorem} (Subject reduction)
If $\Gamma_e, \;\Gamma_a \vdash \conf{E,\;a} : ok$ and $\conf{E, a} \xrightarrow{\alpha} \conf{E', a'}$ then $\Gamma_e, \;\Gamma_a \vdash \conf{E',\;a'} : ok$.
\end{theorem}

We define \textit{well-typedness} of a configuration $\conf{E,\;a}$ by the following rule: \\
$\condinfrule{\Gamma_a \vdash a : \tau \quad p,\; \Gamma_e \vdash E : ok}{\Gamma_e, \;\Gamma_a \vdash \conf{E,\;a} : ok}{\begin{align*}
        &\text{where}\;\\
        &\text{and}\;\end{align*}}$
        
        

\chapter{Sized types for parallel complexity}\label{ch:bgts}
In this chapter, we briefly discuss the type system for parallel complexity of message-passing processes introduced in Baillot and Ghyselen \cite{BaillotGhyselen2021}. This type system builds on the foundations of indices and constraint judgements and formalizes parallel complexity analysis of $\pi$-calculus processes. Due to extensive use of subtyping and the challenges involved in verifying and satisfying constraint judgements, substantial modifications must be made to enable type checking and type inference of processes. We address these topics in Chapter \ref{ch:typecheck} and \ref{ch:timeinference}, respectively.\\

The type system for parallel complexity of message-passing processes introduced by Baillot and Ghyselen uses sized types to express parametric complexity of replicated input invocation, and thereby achieves precise bounds on primitively recursive processes: A class of processes behaving as primitively recursive functions. This requires a notion of polymorphism in the message types of replicated inputs. Baillot and Ghyselen introduce size polymorphism by bounding sizes of algebraic terms and synchronizations on channels with indices that may contain index variables representing unknown sizes. We may interpret an index with an index valuation that maps its index variables to naturals, such that the index may be evaluated to a natural number.\\

We first formally define indices and constraints on the valuations of indices. We give both a predicate logic and a model-theoretic interpretation of judgements on such constraints, referring to these as \textit{constraint judgements}. We then define sized types, the subtyping relation and introduce non-algorithmic type rules.

%\section{A type checker}\label{Sec:typesystembg}
\section{Indices and constraint judgements}\label{sec:indicesandjudgements}
In the type system by Baillot and Ghyselen, indices are used to keep track of sizes of inputs received on replicated inputs. As these sizes may be parametric, in that they may be dependent on the sizes of values received on replicated inputs, we view indices as algebraic expressions consisting of index variables $i,j,k\in\mathcal{V}$ ranging over a countable set, and function symbols, using meta-variable $f$, that may represent natural number constants as nullary functions as well as algebraic operators
\begin{align*}
    I,J ::= i \mid f(I_1,I_2,\dots,I_n)
\end{align*}
Each function symbol $f$ has an arity $\text{ar}(f)$ and an interpretation $[\![f]\!] : \mathbb{N}^{\text{ar}(f)} \rightarrow \mathbb{N}$. For the interpretation of binary difference, we assume that $[\![-]\!](n,m) = 0$ when $m \geq n$, which we refer to as the \textit{monus} operator. As indices may contain index variables, we assume some index valuation $\rho : \mathcal{V} \rightarrow \mathbb{N}$, and extend the definition of interpretations to indices, such that $[\![I]\!]_\rho$ is a natural number instance of index $I$, according to index valuation $\rho$, where for all $i$ in $I$, $\rho(i)$ substitutes for $i$ denoted $I\{\rho(i)/i\}$. Index substitution is defined in Definition \ref{def:indexsubstitution}. Based on the structure of the process that indices are used in the typing of, we may be able to establish relationships between the instances of these indices. For instance, a replicated input may receive values of sizes defined by an interval of two indices $[I,J]$. Then, we are only interested in index valuations $\rho$ that satisfy $[\![I]\!]_\rho \leq [\![J]\!]_\rho$. To express such relationships, we define binary constraints on indices in Definition \ref{def:indexconstr}.

\begin{defi}\label{def:indexsubstitution}
    We define index substitution by the following rules
    \begin{align*}
        i\substi{I}{j} &= j \text{ if } i = j\\
        i\substi{I}{j} &= i \text{ if } i \not = j\\
        f(I_1, I_2, \dots, I_n)\substi{J}{i} &= f(I_1\substi{J}{i}, I_2\substi{J}{i}, \dots, I_n\substi{J}{i})
    \end{align*}
\end{defi}

\begin{defi}[Index constraints]\label{def:indexconstr}
    Given a finite set of index variables $\varphi\subset \mathcal{V}$, we define a constraint $C$ on $\varphi$ to be an expression of the form $I \bowtie J$, where $I$ and $J$ are indices with all free index variables in $\varphi$ and $\bowtie\;\in\{\leq,=,\geq\}$ is a binary relation on $\mathbb{N}$. A finite set of constraints is represented by meta-variable $\Phi$.
\end{defi}
%
A constraint $I \bowtie J$ on $\varphi$ is satisfied given an index valuation $\rho : \varphi \longrightarrow \mathbb{N}$ when $[\![I]\!]_\rho \bowtie [\![J]\!]_\rho$ is satisfied, denoted $\rho \vDash I \bowtie J$. For a finite set of constraints $\Phi$, we write $\rho\vDash \Phi$ when $\rho \vDash C$ holds for all $C \in \Phi$. Finally, $\varphi;\Phi\vDash C$ holds when for all index valuations $\rho$ such that $\rho\vDash \Phi$ holds, we also have $\rho\vDash C$. That is, $\varphi;\Phi\vDash C$ holds exactly when $C$ does not impose further restrictions on index valuations on $\varphi$. Such judgements are fundamental to the type system by Baillot and Ghyselen, especially ones of the form $\varphi;\Phi\vDash I \leq J$, as they impose a partial order on indices wrt. how indices may be interpreted. This enables a notion of subtyping for parametric complexities, such that only indices that are greater or equal may substitute, thus preserving upper bounds on the global parallel complexity, as we shall see in the following sections.
%
%\section{The typechecker}





\begin{table*}[!ht]
    \begin{framed}\vspace{-1em}\begin{align*}
        %
        % S-nil
        &\kern-0.5em\runa{U-nil}\infrule{}{\varphi;\Phi;\Gamma \vdash \nil \triangleleft \{0\}}
        % S-nu
        \kern-2em\runa{U-nu}\infrule{\varphi;\Phi;\Gamma, a:T \vdash P \triangleleft \kappa}{\varphi;\Phi;\Gamma \vdash \newvar{a:T}{P} \triangleleft \kappa}
        % S-par
        \kern-1em\runa{U-par}\infrule{\varphi;\Phi;\Gamma \vdash P \triangleleft \kappa \quad \varphi;\Phi;\Delta \vdash Q \triangleleft \kappa'}{\varphi;\Phi;\Gamma \mid \Delta \vdash P \mid Q \triangleleft \text{basis}(\varphi, \Phi,\kappa \cup \kappa')}\\
        % S-match
        &\kern-0.5em\runa{U-match}\infrule{
        \begin{matrix}
            \varphi;\Phi;\Gamma \vdash e:\natinterval{I}{J} \quad \varphi;\Phi, I \leq 0;\Gamma \vdash P \triangleleft \kappa\\
            \varphi;\Phi, J \geq 1;\Gamma, x:\natinterval{I-1}{J-1} \vdash Q \triangleleft \kappa'
        \end{matrix}}{\varphi;\Phi;\Gamma \vdash \match{e}{P}{x}{Q} \triangleleft \text{basis}(\varphi, \Phi, \kappa \cup \kappa')}
        % S-tick
        \kern16em\runa{S-tick}\infrule{\varphi;\Phi;\Gamma \vdash P \triangleleft \kappa}{\varphi;\Phi;\uparrow^1\!\!\Gamma \vdash \tick P \triangleleft \kappa + 1}\\
        % S-iserv
        &\runa{S-iserv}\infrule{\begin{matrix}
            \texttt{in} \in \sigma\quad \varphi;\Phi;\Gamma\vdash a:\servt{I}{i}{\sigma}{K}{\widetilde{T}}\\
            (\varphi, \widetilde{i}); \Phi; \text{ready}(\varphi,\Phi,\tforwardsim{\Gamma}{I}), \widetilde{v} : \widetilde{T} \vdash P \triangleleft \kappa \quad (\varphi,\widetilde{i});\Phi\vDash\kappa \leq K
        \end{matrix}}
        {\varphi;\Phi;\Gamma \vdash \;\bang\inputch{a}{\widetilde{v}}{}{P}\triangleleft \{I\}}
         % S-ich
        \kern15em\runa{S-ich}\infrule{\begin{matrix}
            \texttt{in} \in \sigma\quad \varphi;\Phi;\Gamma \vdash a:\chant{\sigma}{I}{\widetilde{T}}\\
            \varphi; \Phi; \tforwardsim{\Gamma}{I}, \widetilde{v}:\widetilde{T} \vdash P \triangleleft \kappa
        \end{matrix}}
        {\varphi;\Phi;\Gamma \vdash \inputch{a}{\widetilde{v}}{}{P} \triangleleft \kappa + I}\\
        % S-oserv
        &\runa{S-oserv}\infrule{\begin{matrix}
            \texttt{out} \in \sigma\quad \varphi;\Phi;\Gamma\vdash a:\servt{I}{i}{\sigma}{K}{\widetilde{T}}\\
            \varphi; \Phi;(\tforwardsim{\Gamma}{I}) \vdash \widetilde{e}:\widetilde{S} \quad \text{instantiate}(\widetilde{i}, \widetilde{S}) = \{\widetilde{J}/\widetilde{i}\} \quad \varphi;\Phi \vDash \widetilde{S} \sqsubseteq \widetilde{T}
        \end{matrix}}
        {\varphi;\Phi;\Gamma \vdash \asyncoutputch{a}{\widetilde{e}}{}\triangleleft \{K\{\widetilde{J}/\widetilde{i}\} + I\}}
        % S-och
        \kern15em\runa{S-och}\infrule{\begin{matrix}
            \texttt{out} \in \sigma\quad \varphi;\Phi;\Gamma \vdash a:\chant{\sigma}{I}{\widetilde{T}}\\
            \varphi; \Phi; \tforwardsim{\Gamma}{I} \vdash \widetilde{e}:\widetilde{S} \quad \varphi;\Phi \vDash \widetilde{S} \sqsubseteq \widetilde{T}
        \end{matrix}}
        {\varphi;\Phi;\Gamma \vdash \asyncoutputch{a}{\widetilde{e}}{} \triangleleft \{I\}}\\
        % S-annot
        &\runa{S-annot}\infrule{\varphi;\Phi;\tforwardsim{\Gamma}{n}\vdash P \triangleleft \kappa}{\varphi;\Phi;\Gamma\vdash n:P \triangleleft \kappa + n}
    \end{align*}\vspace{-1em}\end{framed}
    \smallskip
    \caption{Usage typing rules for parallel complexity of processes.}
    \label{tab:usageprocesstypingrules}
\end{table*}
\subsection{Alternative formulations of constraint judgements}\label{sec:cjalternativeform}
There are several equivalent formulations of the problem of verifying the judgement $\varphi;\Phi\vDash C_0$. One such formulation is that the judgement holds, when the conjunction of constraints in $\Phi$ implies $C_0$, i.e. assuming that $n \bowtie m$ evaluates to a truth value based on membership in the relation $\bowtie$, the predicate formula $C_1 \land \cdots \land C_n \implies C_0$, where $\Phi = \{C_1,\dots,C_n\}$, must be satisfied for all valuations $\rho$ over $\varphi$. That is, let $C_i = I_i \bowtie_i J_i$, then for any valuation $\rho : \varphi \rightarrow \mathbb{N}$, the formula $([\![I_1]\!]_\rho \bowtie_1 [\![J_1]\!]_\rho) \land \cdots \land ([\![I_n]\!]_\rho \bowtie_n [\![J_n]\!]_\rho) \implies [\![I_0]\!]_\rho \bowtie_0 [\![J_0]\!]_\rho$ must be satisfied. Another interpretation of the problem is that the intersection of the feasible regions of all (inequality) constraints in $\Phi$ must be contained in the feasible region of $C_0$, or equivalently, the set of all valuations over $\varphi$ that satisfy all the constraints in $\Phi$, referred to as the model space of $\Phi$ wrt. $\varphi$, $\mathcal{M}_\varphi(\Phi)$ must be a subset of the model space of $C_0$ wrt. $\varphi$
\begin{equation*}
    \mathcal{M}_\varphi(\Phi) \subseteq \mathcal{M}_\varphi(\{C_0\})\quad\text{where}\quad\mathcal{M}_\varphi(\Phi)=\{\rho : \varphi \rightarrow \mathbb{N} \mid \rho \vDash C\;\text{for}\; C \in \Phi\}
\end{equation*}
or equivalently
\begin{equation*}
    \forall \rho \in \mathcal{M}_\varphi(\Phi) (\rho \in \mathcal{M}_\varphi(\{C_0\}))
\end{equation*}

Finally, given the fact that the current statement of the problem is expressed using a universal quantifier, we can negate the problem, obtaining a problem that can instead be expressed using an existential quantifier by the fact that $\neg \forall x P(x)$ is equivalent to $\exists x \neg P(x)$. This means the problem can also be expressed as 
%
\begin{equation*}
    \neg (\exists \rho \in \mathcal{M}_\varphi(\Phi) (\rho \not\in \mathcal{M}_\varphi(\{C_0\})))
\end{equation*}
or equivalently
\begin{equation*}
    \mathcal{M}_\varphi(\Phi) \cap \mathcal{M}_\varphi'(\{C_0\}) = \emptyset \quad\text{where}\quad
    \begin{matrix}
        \mathcal{M}_\varphi(\Phi)=\{\rho : \varphi \rightarrow \mathbb{N} \mid \rho \vDash C\;\text{for all}\; C \in \Phi\}\\
        \mathcal{M'}_\varphi(\Phi)=\{\rho : \varphi \rightarrow \mathbb{N} \mid \rho \not\vDash C\;\text{for some}\; C \in \Phi\}
    \end{matrix}
\end{equation*}
Notice that $\mathcal{M}_\varphi'(\{C\})$ is equivalent to $\mathcal{M}_\varphi(\{C'\})$ where $C'$ is the inverse constraint of constraint $C$, and so $\mathcal{M}_\varphi(\Phi) \cap \mathcal{M}_\varphi'(\{C_0\}$) = $\mathcal{M}_\varphi(\Phi \cup \{C_0'\})$ given some method to invert constraints. Thus, the problem can also be expressed simply as
\begin{equation*}
    \mathcal{M}_\varphi(\Phi \cup \{C_0'\}) = \emptyset \quad \text{where } C_0' = \text{inverse of } C_0
\end{equation*}

In Example \ref{exmp:judgementsatisfaction}, we show how a judgement can be verified manually using the predicate logic and model-theoretic interpretations of judgements provided above.
%
\begin{examp}\label{exmp:judgementsatisfaction}
    Given index variables $\varphi = \{i, j, k\}$ and constraints $\Phi = \{C_1, C_2, C_3, C_4\}$ where
    \begin{align*}
        C_1 &= i \geq 4\\
        C_2 &= j \geq 2\\
        C_3 &= -k + 3 < 0\\ % k \leq 4
        C_4 &= i + j + k \leq 11
    \end{align*}
    we want to check if $\varphi; \Phi \vDash 2i + j^2 + 3k \geq 20$ always holds. %For this example we assume interpretations are as expected from usual mathematical notation.\\
    Namely, we are interested in verifying whether the constraint $2i + j^2 + 3k \geq 20$ imposes any additional constraints to the index variables $i$, $j$ and $k$ given the existing constraints $C_1$, $C_2$, $C_3$ and $C_4$. In this case, we can notice that the minimum values of $i$, $j$ and $k$ are $4$, $2$ and $4$ respectively. As such, given these constraints, the minimum value $2i + j^2 + 3k$ may evaluate to is $2 \cdot 4 + 2^2 + 3 \cdot 4 = 24$. As such, we can conclude that $\varphi; \Phi \vDash 2i + j^2 + 3k \geq 20$ always holds.\\
    
    We can also consider the predicate logic interpretation of the example. It suffices to only consider the index valuations that satisfy the conjunction of constraints, of which there are four. Here, we represent a valuation $\rho$ as a set of pairs of the form $\{(i,\rho(i)) \mid i\in\varphi\}$, and so we have $\{(i,4),(j,2),(k,4)\}$, $\{(i,5),(j,2),(k,4)\}$, $\{(i,4),(j,3),(k,4)\}$ and $\{(i,4),(j,2),(k,5)\}$. We can then verify the corresponding implications to show that the judgement holds
    %
    \begin{align*}
        (4 \geq 4) \land (2 \geq 2) \land ({-4}+3 < 0) \implies 4+2+4 \leq 11\\
        %
        (5 \geq 4) \land (2 \geq 2) \land ({-4}+3 < 0) \implies 5+2+4 \leq 11\\
        %
        (4 \geq 4) \land (3 \geq 2) \land ({-4}+3 < 0) \implies 4+3+4 \leq 11\\
        %
        (4 \geq 4) \land (2 \geq 2) \land ({-5}+3 < 0) \implies 4+2+5 \leq 11
    \end{align*}
    Or correspondingly in model-theoretic notation
    {\small
    \begin{align*}
        \mathcal{M}_\varphi(\Phi) =&\; \{\{(i,4),(j,2),(k,4)\}, \{(i,5),(j,2),(k,4)\}, \{(i,4),(j,3),(k,4)\}, \{(i,4),(j,2),(k,5)\}\}\\
        \mathcal{M}_\varphi(\{2i+j^2+3k\geq 20\}) =&\; \{\{(i,n_1),(j,n_2),(k,n_3)\} \mid n_1,n_2,n_3\in\mathbb{N},\; 2n_1 + n_2^2 + 3n_3 \geq 20 \}\\
        \mathcal{M}_\varphi(\Phi) \subseteq&\; \mathcal{M}_\varphi(\{2i+j^2+3k\geq 20\})
    \end{align*}}
    % \begin{align*}
    %     \mathcal{M}_\varphi(\Phi) = \left\{\{(i,4),(j,2),(k,4)\}, \{(i,5),(j,2),(k,4)\}, \{(i,4),(j,3),(k,4)\}, \{(i,4),(j,2),(k,5)\}\right\}
    % \end{align*}
    
    We can also solve the inverse of the mode-theoretic interpretation of the problem. Then we want to show that $\mathcal{M}_\varphi(\Phi) \cap \mathcal{M}_\varphi'(\{2i+j^2+3k\geq 20\}) = \emptyset$ or equivalently $\mathcal{M}_\varphi(\Phi \cup \{2i+j^2+3k < 20\}) = \emptyset$. 
    %
    \begin{align*}
        \mathcal{M}_\varphi(\{2i+j^2+3k < 20\}) =&\; \{\{(i,n_1),(j,n_2),(k,n_3)\} \mid n_1,n_2,n_3\in\mathbb{N},\; 2n_1 + n_2^2 + 3n_3 < 20 \}\\
        &\kern-9em\mathcal{M}_\varphi(\Phi) \cap \mathcal{M}_\varphi'(\{2i+j^2+3k\geq 20\}) = \mathcal{M}_\varphi(\Phi \cup \{2i+j^2+3k < 20\}) = \emptyset
    \end{align*}
\end{examp}
\section{Types and subtyping}\label{sec:typesandsubs}
We now introduce the types from the type system of Baillot and Ghyselen. The types include a base type describing naturals as algebraic terms with sizes bounded by an interval consisting of two indices. This enables us to statically reason about how sizes of data structures change throughout reduction of processes, providing us termination guarantees for some forms of recursion. The type system of Baillot and Ghyselen contains lists as an additional base type, however for conciseness of the type system, we only consider naturals.
%
\begin{align*}
    T,S ::=&\; \texttt{Nat}[I,J] \mid \texttt{ch}_I^\sigma(\widetilde{T}) \mid \forall_I\widetilde{i}.\texttt{serv}_K^\sigma(\widetilde{T})
\end{align*}
%
We use input/output types for channels, and we further distinguish between channels that have replicated inputs, i.e. channels that have recursive behavior, and those that do not. We refer to the former as \textit{servers}, and we more specifically require all inputs on such channels to be replicated for technical convenience. Both servers and normal channels are annotated with an index $I$ that for a normal channel represents the number of time steps remaining before the channel synchronizes, and for a server the remaining time before it becomes available. Note that this imposes a temporal linearity constraint onto normal channels, as such channels can synchronize at exactly one time step. For servers we have an additional index $K$ that represents the parametric complexity of invoking the continuation of a replicated input on the server. Finally, the set $\sigma \subseteq \{\texttt{in},\texttt{out}\}$ is a subset of use-capabilities. Since types consist partly of indices, we define index substitution on types in Definition \ref{def:typeindexsubstitution}.\\

\begin{defi}\label{def:typeindexsubstitution}
    We define index substitution on types by the following rules
    \begin{align*}
        \natinterval{I}{J}\substi{K}{i} &= \natinterval{I\substi{K}{i}}{J\substi{K}{i}}\\
        \chant{\sigma}{I}{\widetilde{T}}\substi{J}{i} &= \chant{\sigma}{I\substi{J}{i}}{\widetilde{T}\substi{J}{i}}\\
        \servt{I}{\widetilde{i}}{\sigma}{K}{\widetilde{T}}\substi{J}{j} &= \servt{I\substi{J}{j}}{\widetilde{i}}{\sigma}{K}{\widetilde{T}} \text{ if } j \in \widetilde{i}\\
        \servt{I}{\widetilde{i}}{\sigma}{K}{\widetilde{T}}\substi{J}{j} &= \servt{I\substi{J}{j}}{\widetilde{i}}{\sigma}{K\substi{J}{j}}{\widetilde{T}\substi{J}{j}} \text{ if } j \not\in \widetilde{i}
    \end{align*}
\end{defi}


Subtyping for base types and types is the least reflexive relation $\sqsubseteq$ that satisfies the subtyping rules in Table \ref{tab:subtypeSized}. As the type system should provide upper bounds on the parallel complexity of processes, it is safe to weaken the bounds on the sizes of natural types. That is, we may decrease the lower bound and increase the upper bound on the sizes of such terms. For server and channel types, we may relax use-capabilities and use the subtyping relation on parameter types as well as modify the complexity bounds on servers, depending on the use-capabilities. Servers and channels of input/output capability are invariant, those of input capability are covariant and those of output capability are contravariant. That is, if a server or channel that inputs a value of type $T$ is required, then we can safely use a server or channel that inputs a subtype of $T$, respectively. Conversely, when a server or channel of output capability is required, we can safely use a channel or server that outputs a supertype of the required parameter type \cite{PierceSangiorgi1996}. This becomes apparent when we assume types $\texttt{Integer}$ and $\texttt{Real}$ such that $\texttt{Integer} \sqsubseteq \texttt{Real}$, as any process that receives reals can also safely receive integers, and any process that output reals can also safely output integers. Unlike Baillot and Ghyselen \cite{BaillotGhyselen2021}, we do not discard associations from our type contexts, rather we discard use-capabilities from channels and servers. Thus, to ensure the type checker is sound, we introduce rules $\runa{BGS-cempty}$ and $\runa{BGS-sempty}$ such that channel and server types are super types of ones with no use-capabilities.

%
\begin{table*}[h!]
    \begin{framed}\vspace{-1em}\begin{align*}
        &\kern0em\runa{BGS-nweak}\;\infrule{\varphi;\Phi\vDash I' \leq I\quad\quad \varphi;\Phi\vDash J \leq J'}{
        \varphi;\Phi\vdash \texttt{Nat}[I,J] \sqsubseteq \texttt{Nat}[I',J']}
        %
        \kern3em\runa{BGS-cinvar}\;\infrule{\varphi;\Phi\vdash \widetilde{T}\sqsubseteq\widetilde{S}\quad\quad \varphi;\Phi\vdash \widetilde{S}\sqsubseteq\widetilde{T}}{\varphi;\Phi\vdash\texttt{ch}_I^{\{\texttt{in},\texttt{out}\}}(\widetilde{T}) \sqsubseteq \texttt{ch}_I^{\{\texttt{in},\texttt{out}\}}(\widetilde{S})}\kern7em\\[-1em]
        %
        \vspace{-0.5em}
        &\kern-0em\runa{BGS-ccovar}\;\infrule{\{\texttt{in}\}\subseteq\sigma\quad\varphi;\Phi\vdash \widetilde{T}\sqsubseteq\widetilde{S}}{\varphi;\Phi\vdash \texttt{ch}_I^{\sigma}(\widetilde{T})\sqsubseteq\texttt{ch}_I^{\{\texttt{in}\}}(\widetilde{S})}\quad\quad\runa{BGS-ccontra}\;\infrule{\{\texttt{out}\}\subseteq\sigma\quad\varphi;\Phi\vdash \widetilde{S}\sqsubseteq\widetilde{T}}{\varphi;\Phi\vdash \texttt{ch}_I^{\sigma}(\widetilde{T})\sqsubseteq \texttt{ch}_I^{\{\texttt{out}\}}(\widetilde{S})}\\[-1em]
        %
        &\kern4em\runa{BGS-sinvar}\;\infrule{(\varphi,\widetilde{i});\Phi\vdash \widetilde{T}\sqsubseteq\widetilde{S}\quad\quad (\varphi,\widetilde{i});\Phi\vdash \widetilde{S}\sqsubseteq\widetilde{T}\quad\quad (\varphi,\widetilde{i});\Phi\vDash K = K'}{\varphi;\Phi\vdash
        \forall_I\widetilde{i}.\texttt{serv}^{\{\texttt{in},\texttt{out}\}}_K(\widetilde{T})
        \sqsubseteq \forall_I\widetilde{i}.\texttt{serv}^{\{\texttt{in},\texttt{out}\}}_{K'}(\widetilde{S})}\\[-1em]
        %
        \vspace{-0.5em}
        &\kern5em\runa{BGS-scovar}\;\infrule{\{\texttt{in}\}\subseteq\sigma\quad(\varphi,\widetilde{i});\Phi\vdash \widetilde{T}\sqsubseteq\widetilde{S}\quad (\varphi,\widetilde{i});\Phi\vDash K' \leq K}{\varphi;\Phi\vdash \forall_I\widetilde{i}.\texttt{serv}^{\sigma}_K(\widetilde{T})\sqsubseteq\forall_I\widetilde{i}.\texttt{serv}^{\{\texttt{in}\}}_{K'}(\widetilde{S})}\\[-1em]
        &\kern4.5em\runa{BGS-scontra}\;\infrule{\{\texttt{out}\}\subseteq\sigma\quad(\varphi,\widetilde{i});\Phi\vdash \widetilde{S}\sqsubseteq\widetilde{T}\quad (\varphi,\widetilde{i});\Phi\vDash K \leq K'}{\varphi;\Phi\vdash \forall_I\widetilde{i}.\texttt{serv}^{\sigma}_K(\widetilde{T})\sqsubseteq \forall_I\widetilde{i}.\texttt{serv}^{\{\texttt{out}\}}_{K'}(\widetilde{S})}\\[-1em]
        %
        &\kern0em\runa{BGS-cempty}\;\infrule{}{\varphi;\Phi\vdash \texttt{ch}^\sigma_I(\widetilde{S}) \sqsubseteq \texttt{ch}^\emptyset_I(\widetilde{T})}\quad\runa{BGS-sempty}\;\infrule{}{\varphi;\Phi\vdash \forall_I\widetilde{i}.\texttt{serv}^\sigma_K(\widetilde{S}) \sqsubseteq \forall_I\widetilde{i}.\texttt{serv}^\emptyset_{K'}(\widetilde{T})}
    \end{align*}\vspace{-1em}\end{framed}
    \smallskip
    \caption{Rules for subtyping of base types and types.}
    \label{tab:subtypeSized}
\end{table*}
\section{Non-algorithmic type rules}

We first consider the type rules for expressions, which are shown in Table \ref{tab:sizedtypedexpressiontypes}. The zero term $0$ intuitively receives the type $\texttt{Nat}[0,0]$ and a successor to a natural term has the same type as its predecessor, but with 1 added to its lower and upper bounds. Finally, a variable receives a type if it is bound in the type context.\\
%Lists are typed similarly, aside from the addition of an element base type. For the element type of a list, we simply use the least lower bound and greatest upper bound on the size amongst the elements of the list.

\begin{table*}[ht]
    \begin{framed}\vspace{-1em}\begin{align*}
        &\kern2em
        \runa{BG-nzero}\;\infrule{}{\varphi;\Phi;\Gamma\vdash\withtype{0}{\typenat[0,0]}}\kern0em
        \runa{BG-nsucc}\;\infrule{\varphi;\Phi;\Gamma \vdash \withtype{e}{\typenat[I, J]}}{\varphi;\Phi;\Gamma \vdash \withtype{\succc{e}}{\typenat[I + 1, J + 1]}}\\[-1em]
        %
        &\kern1em\runa{BG-sub}\;\infrule{\varphi;\Phi;\Delta\vdash e : S\quad\quad \varphi;\Phi\vdash \Gamma\sqsubseteq \Delta\quad\quad \varphi;\Phi\vdash S \sqsubseteq T}{\varphi;\Phi;\Gamma\vdash e : T}\kern11em\runa{BG-var}\;\infrule{}{\varphi;\Phi;\Gamma, \withtype{v}{T} \vdash \withtype{v}{T}}
    \end{align*}\vspace{-1em}\end{framed}
    \smallskip
    \caption{Type rules for expressions.}
    \label{tab:sizedtypedexpressiontypes}
\end{table*}

Before introducing the type rules for processes, we first introduce a function $\downarrow^{\varphi;\Phi}_I\!\!(T)$ in Definition \ref{def:delaysized} that \textit{advances} the time of type $T$ by $I$ units of time complexity. For a channel type $\texttt{ch}^\sigma_J(\widetilde{S})$, we subtract $I$ from $J$ whenever we can guarantee that $J\geq I$ under the constraints imposed on $\varphi$ by $\Phi$. Otherwise, the advancement of $I$ units of time complexity is undefined for type $\texttt{ch}^\sigma_J(\widetilde{S})$, to ensure bounds on communication are not violated. For a server type $\forall_J\widetilde{i}.\texttt{serv}^\sigma_K(\widetilde{S})$, corresponding outputs are well-typed for any timestep $I$ with $I\geq J$, and so a server simply loses input capability whenever we cannot guarantee that $J \geq I$. We extend advancement of time to contexts such that $\downarrow^{\varphi;\Phi}_I(\Gamma)(v)=\;\downarrow^{\varphi;\Phi}_I(\Gamma(v))$. When it is clear from context, we may omit $\varphi$ and $\Phi$.

\begin{defi}[Advancement of Time]\label{def:delaysized}
Let $\varphi$ be a set of index variables, $\Phi$ a set of constraints on indices, $T$ a type and $J$ an index. Then $T$ after $J$ units of time complexity, $\susume{T}{\varphi}{\Phi}{I}$, is given by the rules below
\begin{align*}
    \susume{\natinterval{I}{J}}{\varphi}{\Phi}{I} =&\; \natinterval{I}{J}\\
    %
    %\susume{\texttt{List}[J,K](\mathcal{B})}{\varphi}{\Phi}{I} =&\; \texttt{List}[J,K](\mathcal{B})\\
    %
    \susume{\texttt{ch}^\sigma_J(\widetilde{T})}{\varphi}{\Phi}{I} =&\; \left\{ \begin{matrix}
        %\texttt{ch}^\emptyset_J(\widetilde{T}) & \text{if}\; \sigma = \emptyset\\
        %
        \texttt{ch}^\sigma_{J-I}(\widetilde{T}) & \text{if}\; \varphi;\Phi \vDash J \geq I\\
        %
        \texttt{ch}^\emptyset_{0}(\widetilde{T}) & \text{if}\; \varphi;\Phi \nvDash J \geq I
    \end{matrix} \right.\\
    %
    %\texttt{ch}^\sigma_{J-I}(\widetilde{T}) & \text{if}\; \varphi;\Phi \vDash J \geq I\\
    %
    % \susume{\inchanneltypeS{J}{\widetilde{T}}}{\varphi}{\Phi}{I} =&\; 
    %  \inchanneltypeS{J-I}{\widetilde{T}} & \text{if}\; \varphi;\Phi \vDash J \geq I \\
    % %
    % \susume{\outchanneltypeS{J}{\widetilde{T}}}{\varphi}{\Phi}{I} =&\; 
    %  \outchanneltypeS{J-I}{\widetilde{T}} & \text{if}\; \varphi;\Phi \vDash J \geq I \\
    %
    \susume{\forall_J\widetilde{i}.\texttt{serv}^\sigma_K(\widetilde{T})}{\varphi}{\Phi}{I} =&\; \left\{ \begin{matrix}
        \forall_{J-I}\widetilde{i}.\texttt{serv}^\sigma_K(\widetilde{T}) & \text{if}\; \varphi;\Phi \vDash J \geq I\\
        %
        \forall_{J-I}\widetilde{i}.\texttt{serv}^{\sigma \cap \{\texttt{out}\}}_K(\widetilde{T}) & \text{if}\; \varphi;\Phi \nvDash J \geq I
    \end{matrix} \right.
    %  \servS{J - I}{\widetilde{i}}{K}{\widetilde{T}} & \text{if}\; \varphi;\Phi \vDash J \geq I \\
    % %
    % \susume{\servS{J}{\widetilde{i}}{K}{\widetilde{T}}}{\varphi}{\Phi}{I} =&\; \oservS{J - I}{\widetilde{i}}{K}{\widetilde{T}} & \text{if}\; \varphi;\Phi \vDash J \not\geq I \\          
    % %
    % \susume{\iservS{J}{\widetilde{i}}{K}{\widetilde{T}}}{\varphi}{\Phi}{I} =&\; 
    %  \iservS{J - I}{\widetilde{i}}{K}{\widetilde{T}} & \text{if}\; \varphi;\Phi \vDash J \geq I \\
    % %
    % \susume{\oservS{J}{\widetilde{i}}{K}{\widetilde{T}}}{\varphi}{\Phi}{I} =&\; \oservS{J - I}{\widetilde{i}}{K}{\widetilde{T}}
\end{align*}
\end{defi}

\begin{defi}[Time invariance]\label{def:timeinvariance}
Let $\Gamma$ be a type context. We say that $\Gamma$ is \textit{time invariant} if it only contains variables of either base types or server type with time $0$ and use-capabilities $\sigma$ such that $\sigma\subseteq\{\texttt{out}\}$, i.e. $\forall_0\widetilde{i}.\texttt{serv}^{\sigma}_K(\widetilde{T})$ for some index variables $\widetilde{i}$, types $\widetilde{T}$ and index $K$.
\end{defi}

We now present the type rules of the type system by Baillot and Ghyselen, adapted to fit our syntax. Type judgements are of the form $\varphi;\Phi;\Gamma \vdash P \triangleleft K$, which means that process $P$ has complexity $K$ given constraints $\Phi$ with index variables in $\varphi$ and given a type environment $\Gamma$. The type rules are defined in Table \ref{tab:bgprocesstypingrules}. Rule $\runa{BG-iserv}$ handles replicated inputs and ensures that name $a$ is bound to a server type with input capability in the type context. We must also make sure that in the continuation $P$, the type context must be time invariant as the replicated input may be invoked any number of times after $I$ units of time have elapsed. Thus, only free naturals and servers with no input capability are safe. Rule $\runa{BS-ich}$ is similar except we do not require the type context in the continuation to be time invariant as it is only used once. Rule $\runa{BG-oserv}$ types output servers and most notably uses polymorphism in the index variables $\widetilde{i}$. As such, when typing the expressions sent on the server, we must ensure that we can \textit{instantiate} the index variables of the server using a substitution. Finally, type rule $\runa{BG-match}$ shows how index constraints are introduced when typing processes by utilizing information gained from the two branches of the match expression.\\

% Examples
We now show how a server calculating the $n$th digit of the Fibonacci sequence can be typed. Before presenting the process for the implementation of Fibonacci's sequence, we first need to encode addition in the $\pi$-calculus, which we do using the \textit{add} server as follows.
%
\begin{align*}
    P_\text{add}\defeq&\;\bang\inputch{\text{add}}{x,y,r}{}{
        \texttt{match}\; x\; \{
             0 \mapsto \asyncoutputch{r}{y}{};
            \succc{z} \mapsto \asyncoutputch{\text{add}}{z,\succc{y},r}{}\}}
    %
\end{align*}

The \textit{add} server needs three inputs $x$, $y$, and $r$. The parameters $x$ and $y$ represent two naturals to be added, and $r$ represents the channel intended for receiving the result. Note that no ticks are included in the server as we assume that addition can be done in constant time. The following process for calculating the $n$th number of the Fibonacci sequence is a naïve recursive implementation calculating $\textit{fib}(n)=\textit{fib}(n-1)+\textit{fib}(n-2)$. The server takes two parameters $n$ and $r$ where $n$ is the number of the Fibonacci sequence to calculate and $r$ represents the channel intended for receiving the result.
%
\newcommand{\funcf}[0]{l}
\newcommand{\funcg}[0]{l}
\newcommand{\funcgp}[0]{l-1}
\newcommand{\funcgpp}[0]{l-2}
\newcommand{\funcgppp}[0]{l-1}
\begin{align*}
    P_\text{fib}\defeq&\; \bang\inputch{\text{fib}}{n,r}{}{
         \texttt{match}\; n\; \{ 0 \mapsto \asyncoutputch{r}{0}{}\!;\;
              \succc{n_1} \mapsto\\ 
              &\quad\texttt{match}\; n_1\; \{
                    0 \mapsto \asyncoutputch{r}{\succc{0}}{}\!;\;
                    \succc{n_2} \mapsto\\ &\quad\quad\newvar{r_1,r_2,r_3}{(\asyncoutputch{\text{fib}}{n_1,r_1}{}\mid\asyncoutputch{\text{fib}}{n_2,r_2}{}\\
    &\quad\quad\mid\inputch{r_2}{m_2}{}{\inputch{r_1}{m_1}{}{\tick{\asyncoutputch{\text{add}}{m_1,m_2,r_3}{}}}\mid \inputch{r_3}{m_3}{}{\asyncoutputch{r}{m_3}{}}})}\}\}
    }
\end{align*}

Finally we present a type context $\Gamma$ under which the two servers \textit{add} and \textit{fib} are well-typed. Note that even though we use a naïve implementation of the Fibonacci sequence, we can still get a linear bound as the program runs in parallel.
%
\begin{align*}
    \Gamma \defeq&\; \text{add} : \servt{0}{i,j,k}{\{\texttt{in},\texttt{out}\}}{0}{\texttt{Nat}[0,i],\texttt{Nat}[j,k],\channeltypeS{0}{\texttt{Nat}[j,i+k]}},\\
    &\;\text{fib} : \servt{0}{l}{\{\texttt{in},\texttt{out}\}}{\funcf}{\texttt{Nat}[0,l],\channeltypeS{\funcg}{\texttt{Nat}[0,\textit{fib}(l)]}}
\end{align*}

\begin{table*}
    \begin{framed}\vspace{-1em}\begin{align*}
        &\kern46em\\[-2em] % Stretch frame
        &\kern0em\runa{BG-zero}\infrule{}{\varphi;\Phi;\Gamma \vdash \withcomplex{\nil}{0}}\!\!
        \runa{BG-subtype}\;\infrule{\varphi;\Phi;\Delta \vdash \withcomplex{P}{K} \quad \varphi;\Phi \vdash \Gamma \sqsubseteq \Delta \quad \varphi; \Phi \vDash K \leq K'}{\varphi;\Phi;\Gamma \vdash \withcomplex{P}{K'}}
        \\[-1em]
        %
        &\kern-0em\runa{BG-match}\;\infrule{\varphi;\Phi;\Gamma \vdash \withtype{e}{\natinterval{I}{J}} \quad \varphi;\Phi, I \leq 0;\Gamma \vdash \withcomplex{P}{K} \quad \varphi;\Phi, J \geq 1;\Gamma, \withtype{x}{\natinterval{I\monus 1}{J\monus 1}} \vdash \withcomplex{Q}{K}}{\varphi;\Phi;\Gamma \vdash \withcomplex{\match{e}{P}{x}{Q}}{K}}\\[-1em]
        %
        &\kern4em\runa{BG-par}\;\infrule{\varphi;\Phi;\Gamma\vdash P \triangleleft K\quad \varphi;\Phi;\Gamma\vdash Q \triangleleft K}{\varphi;\Phi;\Gamma\vdash \parcomp{P}{Q} \triangleleft K}\quad\quad\quad\quad\quad\quad \runa{BG-tick}\;\infrule{\varphi;\Phi;\susumesim{\Gamma}{1}\vdash P \triangleleft K}{\varphi;\Phi;\Gamma\vdash \tick P \triangleleft K + 1}\\[-1em]
        %
        &\kern-0em\runa{BG-iserv}\;\infrule{\texttt{in}\in\sigma\quad \varphi;\Phi\vdash\;\susumesim{\Gamma}{I},a:\forall_0\widetilde{i}.\texttt{serv}^\sigma_K(\widetilde{T}) \sqsubseteq \Gamma'\;\text{and}\; \Gamma'\;\text{time invariant}\quad \varphi,\widetilde{i};\Phi;\Gamma',\widetilde{v} : \widetilde{T}\vdash P \triangleleft K}{\varphi;\Phi;\Gamma,\Delta,a : \servt{I}{\widetilde{i}}{\sigma}{K}{\widetilde{T}}\vdash\; \bang\inputch{a}{\widetilde{v}}{}{P}\triangleleft I}\\[-1em]
        %
        &\kern-0em\runa{BG-ich}\;\infrule{\texttt{in}\in\sigma\quad \varphi;\Phi;\susumesim{\Gamma}{I},\widetilde{v} : \widetilde{T}, a : \chant{\sigma}{0}{\widetilde{T}}\vdash P \triangleleft K}{\varphi;\Phi;\Gamma, a : \chant{\sigma}{I}{\widetilde{T}}\vdash \inputch{a}{\widetilde{v}}{}{P}\triangleleft K + I}\kern8.5em \runa{BG-och}\;\infrule{\texttt{out}\in\sigma\quad \varphi;\Phi;\susumesim{\Gamma}{I}\vdash \widetilde{e} : \widetilde{T}}{\varphi;\Phi;\Gamma,a:\chant{\sigma}{I}{\widetilde{T}}\vdash \asyncoutputch{a}{\widetilde{e}}{} \triangleleft I}\\[-1em]
        %
        &\kern2em\runa{BG-oserv}\;\infrule{\texttt{out}\in\sigma\quad \varphi;\Phi;\susumesim{\Gamma}{I}\vdash \widetilde{e} : \widetilde{T}\substi{\widetilde{J}}{\widetilde{i}}}{\varphi;\Phi;\Gamma, a : \servt{I}{\widetilde{i}}{\sigma}{K}{\widetilde{T}}\vdash \asyncoutputch{a}{\widetilde{e}}{} \triangleleft K\!\substi{\widetilde{J}}{\widetilde{i}} + I}\kern12em \runa{BG-nu}\;\infrule{\varphi;\Phi;\Gamma,\withtype{a}{T} \vdash \withcomplex{P}{K}}{\varphi;\Phi;\Gamma \vdash \newvar{a}{\withcomplex{P}{K}}}
    \end{align*}\vspace{-1em}\end{framed}
    \smallskip
    \caption{Sized typing rules for parallel complexity of processes.}
    \label{tab:bgprocesstypingrules}
\end{table*}
\section{Examples of invalid configurations}
The following examples are written in the format $\conf{E, a}$, where $E$ is an editor expression and $a$ is the AST on which we apply the editor expression. \\

In equation \ref{condsubproblem} we show how conditioned substitution can cause problems.
\begin{equation}
    \conf{\left(@\texttt{break} \Rightarrow \replace{\texttt{break}}\right) \ggg \texttt{child}\; 1,\; \lambda x.\hole\; \cursor{\breakpoint{c}}} \label{condsubproblem}
\end{equation}
 In the example we check if the cursor is at a breakpoint, and since the check is true we \textit{toggle} the breakpoint thereby making the following \texttt{child} 1 command problematic. The constant c cannot have a child which means this configuration would cause a run-time error. \\
 
In equation \ref{parentproblem} we show how using the \texttt{parent} command can cause problem when the root is unknown.
\begin{equation}
    \conf{\left(\lozenge\texttt{hole} \Rightarrow \texttt{parent}\right) \ggg \texttt{parent},\; \cursor{\lambda x.\hole}\; c} \label{parentproblem}
\end{equation}
In the example we first check if there is a hole in some subtree of the current cursor. This condition holds and we therefore evaluate the \texttt{parent} command resulting in the AST $\cursor{\lambda x.\hole\; c}$. When the next \texttt{parent} command is evaluated we have a run-time error since we are already situated at the root.\\

In equation \ref{astproblem} we show how an editor expression can result in an AST that would cause a run-time error when evaluated.
\begin{equation}
    \conf{\left(\neg\Box(\texttt{lambda}\; x) \Rightarrow \texttt{child}\; 1\right) \ggg \replace{\texttt{var}\; x}.\texttt{eval},\; \cursor{\lambda x.\hole}\; c} \label{astproblem}
\end{equation}
In the example we first check if it is \textbf{not} necessary that the subtree of the cursor contains a lambda expression. This condition does not hold since it is necessary. Since the condition does not hold we do not evaluate the \texttt{child} 1 command, which means the following substitution of \texttt{var} x is problematic. The substitution results in the AST $\cursor{\texttt{var}\; x}\; c$, which causes a run-time error when the command \texttt{eval} is evaluated, since the left child of the function application is no longer a function.
%
\section{Over-approximations}
As we cannot determine statically whether a condition holds, we establish over-approximations to ensure run-time errors cannot occur in well-typed configurations. As equation \ref{parentproblem} shows, conditioned expressions can result in loss of information about the cursor location. As such, we enforce the cursor \textit{depth} in the tree to be the same before and after a conditioned expression. Furthermore, the first cursor movement in a conditioned expression must be a \texttt{child} prefix. As equation \ref{condsubproblem} shows, conditioned substitution also results in loss of information. Thus, we can no longer guarantee that subsequent substitution at a deeper level is well-typed. Similarly, we no longer know of the structure of the subtree, such that we must condition \texttt{child} prefixes.\\

The above discussion leads to the following list of over-approximations:
\begin{itemize}
    \item In conditioned and recursive expressions, the cursor depth must be the same before and after.
    \item In conditioned and recursive expressions, only the subtree encapsulated by the cursor is accessible.
    \item After conditioned substitution, subsequent substitution at a deeper level is no longer valid, and the \texttt{child} prefix command must be conditioned.
\end{itemize}
%
\section{AST type rules}
\begin{table*}[htp]
    \centering
    \begin{align*}
        \runa{t-var} &\; \infrule{\Gamma_a\left(x\right)=\tau}{\Gamma_a \vdash x : \tau}\\
        %
        \runa{t-const} &\; \infrule{}{\Gamma_a \vdash c : b}\\
        %
        \runa{t-app} &\; \infrule{\Gamma_a \vdash a_1 : \tau_1 \rightarrow \tau_2 \quad \Gamma_a \vdash a_2 : \tau_1}{\Gamma_a \vdash a_1\; a_2 : \tau_2}\\
        %
        \runa{t-lambda} &\; \infrule{\Gamma_a\left[x \mapsto \tau_1\right] \vdash a : \tau_2}{\Gamma_a \vdash \lambda x:\tau_1.a : \tau_1 \rightarrow \tau_2} \\
        %
        \runa{t-break} &\; \infrule{\Gamma_a \vdash a : \tau}{\Gamma_a \vdash \breakpoint{a} : \tau} \\
        %
        \runa{t-hole} &\; \infrule{}{\Gamma_a \vdash \left(\hole : \tau\right) : \tau}
        %
    \end{align*}
    \caption{Type rules for abstract syntax trees.}
    \label{tab:typerules}
\end{table*}

%\section{Type context format}
%Here, we propose a format for type contexts of editor expressions. The context of an editor expression could be a triple $\Psi = (\Gamma_a, \tau, \Gamma)$, where $\Gamma_a$ is the type context for the subtree encapsulated by the cursor, $\tau$ is the type of the subtree and $\Gamma$ is a function or map from prefix command types to editor expression contexts. That is, contexts for editor expressions are recursive. Say we have context $(\Gamma_a, \tau, \Gamma)$. Upon a $\texttt{child}\; 1$ prefix, we \textit{look up} $\texttt{one}$ in $\Gamma$. If $\Gamma(\texttt{one}) = undef$, the expression is not well-typed. Otherwise, we evaluate the prefixed expression in the new context $\Gamma(\texttt{one})$.\\

%We construct the initial context based on the AST in the configuration $\conf{E,\; a}$. Upon a substitution prefix, we modify the context, upon a child or parent prefix, we \textit{move} in the context, and upon a conditioned or recursive expression, we set some of the bindings to $undef$: $\Gamma(T)=undef$.\\

%$\Gamma = T_1 : \Psi_1,...,T_n : \Psi_n$ \\
%$\Psi = (\Gamma_a, \tau, \Gamma)$
%Γ = T1 : Ψ1,..,Tn : Ψn
%Ψ = (Γa, τ, Γ)

\section{Experimental type system}

In this section, we introduce a type system for our editor-calculus. For the type system, we introduce the syntactic categories $\tau \in \mathbf{ATyp}$ to denote types of AST nodes, $T \in \mathbf{CTyp}$ to denote \textit{child} types, and p $\in \mathbf{Pth}$ to denote AST paths.
%
\begin{align*}
    \tau ::=&\; b \mid \tau_1 \rightarrow \tau_2 \mid \breakpoint{\tau} \mid \texttt{indet}\\
    T ::=&\; \texttt{one} \mid \texttt{two}\\
    p ::=&\; p\; T \mid \epsilon
\end{align*}

In addition to the basic and arrow types in $\mathbf{ATyp}$, we include a type for breakpoints, $\breakpoint{\tau}$, and a type to denote indeterminate types, \texttt{indet}. We use $\mathbf{Pth}$ to denote paths in an AST by storing a sequence of \textbb{one} and \textbb{two} which denote if the path goes through the first or second child.\\

We define two sets for contexts in our type system. The first context, $\mathbf{ACtx}$, stores type bindings for variables in the AST. The second context, $\mathbf{ECtx}$, stores, for all available paths so far, a pair of an AST context and the type of the node at the end of the path. We use $\Gamma_a \in \mathbf{ACtx}$ and $\Gamma_e \in \mathbf{ECtx}$ as metavariables for the two contexts. To check if a path $p$ is available in a context $\Gamma_e$, we use the notion $\Gamma_e(p) \neq \text{undef}$. $\mathbf{ACtx}$ and $\mathbf{ECtx}$ are thus defined as the following.
%
\begin{align*}
\mathbf{ACtx} &= \mathbf{Var} \rightharpoonup \mathbf{ATyp}\\
\mathbf{ECtx} &= \mathbf{Pth} \rightharpoonup \left(\mathbf{ACtx} \times \mathbf{ATyp}\right)
\end{align*}

To support our type system, we modify the syntax for AST node modifications by including type annotations for application, abstraction and holes. The new syntax thus becomes the following.
%
\begin{align*}
  D ::= \; & \texttt{var}\;x \mid \texttt{const}\;c \mid \texttt{app} : \tau_1 \rightarrow \tau_2, \tau_1 \mid \texttt{lambda}\; x : \tau_1 \rightarrow \tau_2 \mid \texttt{break} \mid \texttt{hole} : \tau
\end{align*}

To support breakpoint types, we introduce the notion of type consistency into our typesystem. The purpose of consistency in our type system is to ensure breakpoints types are consistent with their respective type, as defined below.
%
\begin{definition}{(Type consistency)}
    We define two types $\tau_1, \tau_2$ to be \textit{consistent}, denoted $\tau_1 \sim \tau_2$, by the following rules.
    \begin{align*}
        \runa{cons-1} \hspace{-1cm}
        \infrule{}{\tau \sim \tau} \hspace{-1cm}
        \runa{cons-2} \hspace{-1cm}
        \infrule{}{\breakpoint{\tau} \sim \tau} \hspace{-1cm}
        \runa{cons-3} \hspace{-1cm}
        \infrule{}{\tau \sim \breakpoint{\tau}} \hspace{-1cm}
        \runa{cons-4}
        \infrule{\tau_1 \sim \tau_1' \quad \tau_2 \sim \tau_2'}{(\tau_1 \rightarrow \tau_2) \sim (\tau_1' \rightarrow \tau_2')}
    \end{align*}
\end{definition}


\begin{table*}[htp]
    \centering
    \begin{align*}
        \runa{ctx-split-1}&\; \infrule{}{\emptyset = p \left(\emptyset\; \circ\; \emptyset\right)}\\
        \runa{ctx-split-2}&\; \infrule{\Gamma_e = p \left({\Gamma_e}_1\; \circ\; {\Gamma_e}_2\right)}{\Gamma_e,\; p\; T_1..T_n: (\Gamma_a,\; \tau) = p \left(\left({\Gamma_e}_1,\; p\; T_1..T_n: (\Gamma_a,\; \tau)\right)\; \circ\; {\Gamma_e}_2\right)}\\
        \runa{ctx-split-3}&\; \infrule{p_1 \neq p_2 \quad \Gamma_e = p_2 \left({\Gamma_e}_1\; \circ\; {\Gamma_e}_2\right)}{\Gamma_e,\; p_1\; T_1..T_n: (\Gamma_a,\; \tau) = p_2 \left({\Gamma_e}_1\; \circ\; \left({\Gamma_e}_2,\; p_1\; T_1..T_n: (\Gamma_a,\; \tau)\right)\right)}\\
        %
        \runa{ctx-update-1}&\; \infrule{}{\Gamma_e = \Gamma_e + \emptyset}\\
        \runa{ctx-update-2}&\; \infrule{\Gamma_e = \left({\Gamma_e}_1,\; p: ({\Gamma_a}_2,\; \tau_2)\right) + {\Gamma_e}_2}{\Gamma_e,\; p: ({\Gamma_a}_1,\; \tau_1) = \left({\Gamma_e}_1,\; p: ({\Gamma_a}_2,\; \tau_2)\right) + {\Gamma_e}_2}\\
        \runa{ctx-update-3}&\; \infrule{\Gamma_e = {\Gamma_e}_1 + {\Gamma_e}_2}{\Gamma_e,\; p: (\Gamma_a,\; \tau) = {\Gamma_e}_1 + \left({\Gamma_e}_2,\; p: (\Gamma_a,\; \tau)\right)}
    \end{align*}
    \caption{Context split and context update for editor contexts.}
    \label{tab:context}
\end{table*}
% We define \textit{type contexts}, $\Gamma_e$ in Table \ref{tab:context} as a mapping from a path $p$ to a pair consisting of an AST context $\Gamma_a$ and AST type $\tau$. We denote the $\Gamma_e, p : (\Gamma_a, \tau)$ as the type context equal to the paths not in the domain of map $\Gamma_e$ except for $p$, where $\Gamma_e(p) = (\Gamma_a, \tau)$. For type contexts we introduce the concept of \textit{context splitting} on a path in terms of $\Gamma_e$ maintained through two sub-contexts $\Gamma_{e1}$ and $\Gamma_{e2}$. For this we require a split-operation $\circ$, defined for two sub-contexts on a path as $\Gamma_e = p(\Gamma_{e1}\; \circ \; \Gamma_{e2})$. Notice the empty context is defined with the symbol $\emptyset$ as in \runa{ctx-split-1}. In rule \runa{ctx-split-2} we have that $p$ is in $\Gamma_{e1}$, but not in $\Gamma_{e2}$. Thus, $p$ is not in $\Gamma = \Gamma_{e1}\; \circ \; \Gamma_{e2}$, which is similarly done for the \runa{ctx-split-3} in terms of $\Gamma_{e1}$.\\

Next we introduce the notion of \textit{context updates} to update bindings in a context with new types for the associated path $p$. We use the addition operator $+$, to denote sum-context $\Gamma$ of two compatible type contexts $\Gamma_{e1}$ and $\Gamma_{e2}$. The rules require linear paths to not have bindings exist in another context. Thus, we can only update a context $\Gamma_{e2}$ iff no bindings for a given path is in context $\Gamma_{e1}$. In rule \runa{ctx-update-2} we have bindings in $\Gamma_{e1}$, which means we cannot add bindings to $\Gamma_{e2}$. However, in rule \runa{ctx-update-3} we allow path bindings in $\Gamma_{e2}$ since no such bindings are in context $\Gamma_{e1}$.

% \begin{equation}
%     depth(e) = \left\{
%         \begin{array}{ll}
%             depth(E) + 1            & \quad if e = (\texttt{child}\; n).E \\
%             depth(E) - 1            & \quad if e = \texttt{parent}.E\\
%             depth(E_1) + depth(E_2) & \quad if e = E_1 \ggg E_2\\
%             depth(E)                & \quad if e = \texttt{rec}\; x.E\\
%             depth(E)                & \quad if e = \pi.E\\
%             0                       & \quad otherwise
%         \end{array}
%     \right.
% \end{equation}

\begin{definition}{(Relative cursor depth)}
    We define the function $depth : \mathbf{Edt} \rightarrow \mathbb{Z}$, from the set of atomic editor expression to the set of integers.
    \begin{align*}
    depth((\texttt{child}\; n).E) &= depth(E) + 1 \\
    depth(\texttt{parent}.E) &= depth(E) - 1 \\
    depth(E_1 \ggg E_2) &= depth(E_1) + depth(E_2) \\
    depth(\texttt{rec}\; x.E) &= depth(E) \\
    depth(\pi.E) &= depth(E) \\
    depth(E) &= 0 
\end{align*}
\end{definition}
The $depth$ function statically analyses the structure of an editor expression to determine the relative depth of the cursor after evaluation of the expression. This function is used to make sure the position of the cursor before and after evaluation of an expression is the same. As the function performs a static analysis, we do not consider conditioned subexpressions. Later, in the type rules, we will see why we can safely ignore conditioned subexpressions. \\


% Next we define the function $match : \mathbf{Aam} \times \mathbf{ACtx} \times \mathbf{ATyp} \rightarrow \{tt, f\!\!f\}$. This function returns true if the type of the given AST modification $D$, is equal to the given AST type $\tau$.  
% \begin{align*}
%     match(\texttt{var}\; x,\;\Gamma_a,\;\tau) &= \left\{\begin{matrix}
%  tt & \text{if}\; \Gamma_a(x) = \tau\\ 
%  f\!\!f & \text{otherwise}
% \end{matrix}\right.\\
%     match(\texttt{const}\; c,\;\Gamma_a,\; b) &= tt\\
%     match(\texttt{app} : \tau_1 \rightarrow \tau_2,\; \tau_1,\;\Gamma_a,\; \tau_2) &= tt\\
%     match(\texttt{lambda}\; x : \tau_1 \rightarrow \tau_2,\;\Gamma_a,\; \tau_1 \rightarrow \tau_2) &= tt\\
%     match(\texttt{break},\;\Gamma_a,\; \tau) &= tt\\
%     match(\texttt{hole} : \tau,\;\Gamma_a,\; \tau) &= tt\\
%     match(D,\; \Gamma_a,\; \tau) &= f\!\!f
% \end{align*}

%\begin{equation*}
%    %context : \left(\mathbf{Aam} \times \mathbf{ACtx}\right) \rightharpoonup %\left(\left(\mathbf{Pth} \rightarrow \left(\left(\mathbf{Var} \rightharpoonup %\mathbf{ATyp}\right) \times \mathbf{ATyp}\right)\right) \cup \{error\}\right)
    %context : \left(\mathbf{Aam} \times \mathbf{ACtx} \times \mathbf{Pth} \right) %\rightharpoonup \mathbf{ECtx}
%\end{equation*}
%\begin{align*}
% context(\texttt{const}\; c,\; \Gamma_a,\; p) =&\; \emptyset\\
%  context(\texttt{hole} : \tau,\; \Gamma_a,\; p) =&\; \emptyset\\
%context(\texttt{var}\; x,\; \Gamma_a,\; p) =&\; \emptyset\\
 %context((\texttt{app} : \tau_1 \rightarrow \tau2,\; \tau_1),\; \Gamma_a,\; p) =&\; %\emptyset,\; p\; \texttt{one} : (\Gamma_a,\; \tau_1 \rightarrow \tau_2),\; p\; \texttt{two} : %(\Gamma_a,\; \tau_1)\\
 %context(\texttt{lambda}\; x : \tau_1 \rightarrow \tau_2,\; \Gamma_a,\; p) =&\; \emptyset,\; %p\; \texttt{one} : ((\Gamma_a,\; x : \tau_1),\; \tau_2)
%\end{align*}
%
%

We define functions \textit{limits} and \textit{follows} to analyze which cursor movement is safe given a condition holds. \textit{limits} finds the set of possible AST node modifiers, on which the cursor may sit, given the condition holds. \textit{follows} gives a set of editor type context bindings guaranteed to be safe, given the cursor sits on AST node modifier $D$. Note that the AST type context is empty and that the node type is $\texttt{indet}$, as we cannot determine such information based on a condition. Thus, besides toggling of breakpoints, substitution is not well-typed at path $p$ if $\Gamma_e(p)=(\emptyset,\; \texttt{indet})$. We can combine functions \textit{limits} and \textit{follows} to provide additional bindings to the editor type context of a conditioned expression $\phi \Rightarrow E$. The intersection of \textit{follows} applied to each AST node modifier $D$ in the set $limits(\phi)$ is the set of bindings guaranteed to be safe, given $\phi$ holds.

\theoremstyle{definition}
\begin{definition}{(Condition constraints)}
We define a function $limits: \mathbf{Eed} \rightarrow \mathcal{P}(\mathbf{Aam})$ from the set of conditions to the power set of the set of AST node modifiers. We assume conditions are in conjunctive normal form.
\begin{align*}
    limits(@D)=&\;\{D\}\\
    limits(\neg @D)=&\;\mathbf{Aam}\setminus \{D\}\\
    limits(\lozenge D)=&\;\{D\} \cup \{\texttt{app},\; \texttt{lambda}\; x,\; \texttt{break}\}\\
    limits(\neg \lozenge D)=&\;\mathbf{Aam}\setminus \{D\}\\
    limits(\Box D)=&\;\{D\} \cup \{\texttt{app},\; \texttt{lambda}\; x,\; \texttt{break}\}\\
    limits(\neg \Box D)=&\;\mathbf{Aam}\setminus \{D\}\\
    limits(\phi_1 \land \phi_2)=&\;limits(\phi_1) \cap limits(\phi_2)\\
    limits(\phi_1 \lor \phi_2)=&\;limits(\phi_1) \cup limits(\phi_2)
\end{align*}
\end{definition}


\theoremstyle{definition}
\begin{definition}{(Safe movement)}
We define a function $follows: \mathbf{Aam} \times \mathbf{Pth} \rightarrow \mathcal{P}\left(\mathbf{Pth} \times \left(\mathbf{ACtx} \times \mathbf{ATyp}\right)\right)$ from the set of pairs of AST node modifiers and paths to the power set of editor context bindings.
\begin{align*}
    \textit{follows}(\texttt{var}\; x,\; p)=&\; \emptyset\\
    \textit{follows}(\texttt{const}\; c,\; p)=&\; \emptyset\\
    \textit{follows}(\texttt{app},\; p)=&\; \{p\; \texttt{one} : (\emptyset,\; \texttt{indet}),\; p\; \texttt{two} : (\emptyset,\; \texttt{indet})\}\\
    \textit{follows}(\texttt{lambda}\; x,\; p)=&\; \{p\; \texttt{one} : (\emptyset,\; \texttt{indet})\}\\
    \textit{follows}(\texttt{break},\; p)=&\; \{p\; \texttt{one} : (\emptyset,\; \texttt{indet})\}\\
    \textit{follows}(\texttt{hole},\; p)=&\; \emptyset
\end{align*}
\end{definition}

%
%
We now introduce the type rules for editor expressions. Type rules for substitution are shown in table \ref{tab:typerulesv2sub} and the remaining rules are shown in table \ref{tab:typerulesv2}. The \texttt{child} n prefix is handled by \runa{t-child-1} and \runa{t-child-2}. Here we check that the cursor movement is viable by looking up the new path in $\Gamma_e$. Notice that the remaining editor expression $E$, is evaluated using the new path. The \texttt{parent} prefix is handled similarly in \runa{t-parent} with the exception being that we deconstruct the path instead of building it. When using recursion we require that the depth of the cursor is unchanged after evaluating the expression. We ensure this in \runa{t-rec} with the side condition $depth(E) = 0$. Similarly, \runa{t-cond} utilizes the same side condition to ensure that the cursor is unaffected by whether the condition holds or not. Notice here that evaluation of the conditioned expression is limited by what can follow the condition if it holds, denoted by $\delta$. Sequential composition is handled by the type rule \runa{t-seq}. Here we split the type context into $\Gamma_{e1}$, which contains information about the current subtree, and $\Gamma{e2}$, which contains information about the rest of the tree. This split ensures that the potentially hazardous evaluation of $E_1$ is kept separate from the evaluation of $E_2$.\\

\begin{table*}[htp]
    \centering
    \begin{align*}
        %
        \runa{t-eval} &\; \infrule{p,\; \Gamma_e \vdash E : ok}{p,\; \Gamma_e \vdash \texttt{eval}.E : ok}\\
        %
        \runa{t-child-1}&\; \infrule{\Gamma_e(p\; \texttt{one}) \neq \text{undef} \quad p\; \texttt{one},\; \Gamma_e \vdash E : ok}{p,\; \Gamma_e \vdash \left(\texttt{child}\; 1\right).E : ok}\\
        %
        \runa{t-child-2}&\; \infrule{\Gamma_e(p\; \texttt{two}) \neq \text{undef} \quad p\; \texttt{one},\; \Gamma_e \vdash E : ok}{p,\; \Gamma_e \vdash \left(\texttt{child}\; 2\right).E : ok}\\
        %
        \runa{t-parent}&\; \infrule{\Gamma_e(p) \neq \text{undef} \quad p,\; \Gamma_e \vdash E : ok}{p\; T,\; \Gamma_e \vdash \texttt{parent}.E : ok}\\
        %
        \runa{t-rec} &\; \condinfrule{p,\; \Gamma_e \vdash E : ok}{p,\; \Gamma_e \vdash \texttt{rec} x.E : ok}{\text{if}\; depth(E) = 0}\\
        %
        \runa{t-cond} &\; \condinfrule{p,\; \Gamma_e + \delta \vdash E : ok}{p,\; \Gamma_e \vdash \phi \Rightarrow E : ok}{\begin{align*}
            \text{if}\; &depth(E) = 0\;\\
            \text{and}\; &\delta = \bigcap_{D \in limits(\phi)}follows(D,\; p)\\
        \end{align*}}\\
        %
        \runa{t-seq} &\; \condinfrule{p,\; {\Gamma_e}_1 \vdash E_1 : ok \quad p,\; {\Gamma_e}_2 \vdash E_2 : ok}{p,\; \Gamma_e \vdash E_1 \ggg E_2 : ok}{\text{where}\; \Gamma_e = p\; ({\Gamma_e}_1\; \circ\; {\Gamma_e}_2)}\\
        %
        \runa{t-ref} &\; \infrule{}{p,\;\Gamma_e \vdash x : ok}\\
        %
        \runa{t-nil} &\; \infrule{}{p,\;\Gamma_e \vdash \mathbf{0} : ok}
    \end{align*}
    \caption{Type rules for editor expressions.}
    \label{tab:typerulesv2}
\end{table*}
%
%
Table \ref{tab:typerulesv2sub} shows the type rules for substitution. For substitution to be well-typed, the AST node type $\tau$ in the type context binding associated with the current path $p$ must be consistent with the type of the AST node modifier to be inserted. In \runa{t-sub-var}, we handle the special case where we insert a variable reference $x$. For this to be well-typed, a binding $\Gamma_a(x)=\tau'$ must exist, such that $\consistent{\tau}{\tau'}$. Note that substitution replaces a subtree of the AST. Thus, the bindings in the editor type context with paths starting with $p$ are no longer valid. Therefore, we split the type context on path $p$, such that $\Gamma_e = p\left({\Gamma_e}_1\;\circ\;{\Gamma_e}_2\right)$, and evaluate the prefixed expression $E$ in the type context ${\Gamma_e}_2$. That is, the type context containing all bindings of $\Gamma_e$ not starting with $p$. Note that the binding with path exactly $p$ is in both ${\Gamma_e}_1$ and ${\Gamma_e}_2$, however. We add bindings to ${\Gamma_e}_2$ in rules $\runa{t-sub-app}$ and $\runa{t-sub-abs}$. Particularly, we expand the AST type context upon substitution for an abstraction.\\

We treat substitution of breakpoints differently, as we can either toggle breakpoints on or off. Furthermore, we do not replace the subtree upon substitution for breakpoints. Instead, we must modify the bindings with paths starting with $p$, to either include or remove a $\texttt{one}$. Additionally, we change the type in the binding at the current path $p$ to indicate whether it has a breakpoint. Note that we toggle off the breakpoint if the type is of the form $\breakpoint{\tau}$, and toggle it on otherwise. Thus, the type indicates the structure of the tree.
%
%
\begin{table}
    \begin{flalign*}
        %
        \runa{t-sub-var} &\; \condinfrule{\Gamma_e(p)=(\Gamma_a,\;\tau) \quad \Gamma_a(x) = \tau' \quad \consistent{\tau}{\tau'} \quad p,\;{\Gamma_e}_2 \vdash E : ok}{p,\; \Gamma_e \vdash \replace{\texttt{var}\; x}.E : ok}{\text{where}\; \Gamma_e = p\; ({\Gamma_e}_1\; \circ\; {\Gamma_e}_2)} \\
        %
        \runa{t-sub-const} &\; \condinfrule{\Gamma_e(p)=(\Gamma_a,\;b) \quad p,\;{\Gamma_e}_2 \vdash E : ok}{p,\; \Gamma_e \vdash \replace{\texttt{const}\; c}.E : ok}{\text{where}\; \Gamma_e = p\; ({\Gamma_e}_1\; \circ\; {\Gamma_e}_2)}\\
        %
        \runa{t-sub-app} &\; \condinfrule{\Gamma_e(p)=(\Gamma_a,\; \tau_2') \quad \consistent{\tau_2}{\tau_2'} \quad p,\; \Gamma_e' \vdash E : ok}{p,\; \Gamma_e \vdash \replace{\texttt{app} : \tau_1 \rightarrow \tau_2,\; \tau_1}.E : ok}{\begin{align*}
            &\text{where}\; \Gamma_e = p\; ({\Gamma_e}_1\; \circ\; {\Gamma_e}_2)\;\\
            &\text{and}\; \Gamma_e' = {\Gamma_e}_2,\; p\; \texttt{one} : (\Gamma_a,\; \tau_1 \rightarrow \tau_2),\; p\; \texttt{two} : (\Gamma_a,\; \tau_1)
        \end{align*}}\\
        %
        \runa{t-sub-abs} &\; \condinfrule{\Gamma_e(p)=(\Gamma_a,\; \tau_1' \rightarrow \tau_2') \quad \consistent{\tau_1 \rightarrow \tau_2}{\tau_1' \rightarrow \tau_2'} \quad p,\; \Gamma_e' \vdash E : ok}{p,\; \Gamma_e \vdash \replace{\texttt{lambda}\; x : \tau_1 \rightarrow \tau_2}.E : ok}{\begin{align*}
        &\text{where}\;\Gamma_e = p\; ({\Gamma_e}_1\; \circ\; {\Gamma_e}_2)\\
        &\text{and}\;\Gamma_e' = {\Gamma_e}_2, p\; \texttt{one} : ((\Gamma_a,\; x : \tau_1),\; \tau_2)\end{align*}} \\
        %
        %\runa{t-sub} &\; \infrule{match(D,\; \Gamma_a,\; \tau) = tt \quad p,\;\Gamma_e' \vdash %E : ok}{p,\;\Gamma_e \vdash \replace{D}.E : ok} \\
        %&\text{if}\; D \neq \texttt{break}\\
        %&\text{and}\; \Gamma_e(p)=(\Gamma_a,\;\tau) \\
        %&\text{and}\; \Gamma_e = p\; ({\Gamma_e}_1\; \circ\; {\Gamma_e}_2)\\
        %&\text{and}\; \Gamma_e' = {\Gamma_e}_2 + context(D,\; \Gamma_a)\\
        %
        \runa{t-sub-break-1} &\; \infrule{\Gamma_e(p)=(\Gamma_a,\; \breakpoint{\tau}) \quad p,\; \Gamma_e' \vdash E : ok}{p,\; \Gamma_e \vdash \replace{\texttt{break}} : ok} \\
        &\text{where}\; \Gamma_e = p\; ({\Gamma_e}_1\; \circ\; {\Gamma_e}_2)\\
        &\text{and}\; {\Gamma_e}_1 = \emptyset,\; p\; \texttt{one}\; T_1..T_{n_1} : ({\Gamma_a}_1,\; \tau_1),..,p\; \texttt{one}\; T_1..T_{n_m} : ({\Gamma_a}_m,\; \tau_m)\\
        &\text{and}\; {\Gamma_e}_1' =\emptyset,\; p\; T_1..T_{n_1} : ({\Gamma_a}_1,\; \tau_1),..,p\; T_1..T_{n_m} : ({\Gamma_a}_m,\; \tau_m)\\
        &\text{and}\; \Gamma_e' = \left({\Gamma_e}_2 + {\Gamma_e}_1'\right),\; p : (\Gamma_a,\; \tau)\\
        %
        \runa{t-sub-break-2} &\; \infrule{\Gamma_e(p)=(\Gamma_a,\;\tau)\quad  p,\; \Gamma_e' \vdash E : ok}{p,\; \Gamma_e \vdash \replace{\texttt{break}} : ok} \\
        &\text{where}\; \Gamma_e = p\; ({\Gamma_e}_1\; \circ\; {\Gamma_e}_2)\\
        &\text{and}\; {\Gamma_e}_1 =\emptyset,\; p\; T_1..T_{n_1} : ({\Gamma_a}_1,\; \tau_1),..,p\; T_1..T_{n_m} : ({\Gamma_a}_m,\; \tau_m)\\
        &\text{and}\; {\Gamma_e}_1' = \emptyset,\; p\; \texttt{one}\; T_1..T_{n_1} : ({\Gamma_a}_1,\; \tau_1),..,p\; \texttt{one}\; T_1..T_{n_m} : ({\Gamma_a}_m,\; \tau_m)\\
        &\text{and}\; \Gamma_e' = \left({\Gamma_e}_2 + {\Gamma_e}_1'\right),\; p : (\Gamma_a,\; \breakpoint{\tau})\\
        %
        \runa{t-sub-hole} &\; \condinfrule{\Gamma_e(p)=(\Gamma_a,\;\tau') \quad \consistent{\tau}{\tau'} \quad p,\;{\Gamma_e}_2 \vdash E : ok}{p,\; \Gamma_e \vdash \replace{\texttt{hole} : \tau}.E : ok}{\text{where}\; \Gamma_e = p\; ({\Gamma_e}_1\; \circ\; {\Gamma_e}_2)}
        %
    \end{flalign*}
    \caption{Type rules for substitution.}
    \label{tab:typerulesv2sub}
\end{table}

%\begin{table*}[htp]
%    \centering
%    \begin{align*}
        %%
        %\runa{t-eval} &\; \infrule{p,\; \Gamma_e \vdash E : ok \dashv p',\; \Gamma_e'}{p,\; \Gamma_e \vdash \texttt{eval}.E : %ok \dashv p',\; \Gamma_e'}\\
        %%
        %\runa{t-sub} &\; \infrule{T=\tau \quad p,\;\Gamma_e'' \vdash E : ok \dashv p',\;\Gamma_e'}{p,\;\Gamma_e \vdash %\replace{D}.E : ok \dashv p',\;\Gamma_e'} \\
        %&\text{where}\; \Gamma_e(p)=(\Gamma_a,\;\tau) \\
        %&\text{and}\; T = type(D,\;\Gamma_a) \\
        %&\text{and}\; \Gamma_e = p\; ({\Gamma_e}_1\; \circ\; {\Gamma_e}_2)\\
        %&\text{and}\; \Gamma_e'' = {\Gamma_e}_1 + context(D,\; \Gamma_a)\\
        %%
        %\runa{t-child-1}&\; \infrule{\Gamma_e(p\; \texttt{one}) \neq undef \quad p,\; \texttt{one},\; \Gamma_e \vdash E : ok %\dashv p',\; \Gamma_e'}{p,\; \Gamma_e \vdash \left(\texttt{child}\; 1\right).E : ok \dashv p',\; \Gamma_e'}\\
        %%
        %\runa{t-child-2}&\; \infrule{\Gamma_e(p\; \texttt{two}) \neq undef \quad p,\; \texttt{one},\; \Gamma_e \vdash E : ok %\dashv p',\; \Gamma_e'}{p,\; \Gamma_e \vdash \left(\texttt{child}\; 2\right).E : ok \dashv p',\; \Gamma_e'}\\
        %%
        %\runa{t-parent}&\; \infrule{\Gamma_e(p) \neq undef \quad p,\; \Gamma_e \vdash E : ok \dashv p',\; \Gamma_e'}{p\; T,\; %\Gamma_e \vdash \texttt{parent}.E : ok \dashv p',\; \Gamma_e'}\\
        %%
        %\runa{t-rec} &\; \condinfrule{p,\; {\Gamma_e}_1 \vdash E : ok \dashv p,\; \Gamma_e'}{p,\; \Gamma_e \vdash \texttt{rec} %x.E : ok \dashv p,\; {\Gamma_e}_2}{\text{where}\; \Gamma_e = p\; ({\Gamma_e}_1\; \circ\; {\Gamma_e}_2)}\\
        %%
        %\runa{t-seq} &\; \infrule{p,\; \Gamma_e \vdash E_1 : ok \dashv p'',\; \Gamma_e'' \quad p'',\; \Gamma_e'' \vdash E_2 : %ok \dashv p',\; \Gamma_e'}{p,\; \Gamma_e \vdash E_1 \ggg E_2 : ok \dashv p',\; \Gamma_e'}\\
        %%
        %\runa{t-cond} &\; \infrule{p,\; {\Gamma_e}_1 + \delta \vdash E : ok \dashv p,\; \Gamma_e'}{p,\; \Gamma_e \vdash \phi %\Rightarrow E : ok \dashv p,\; {\Gamma_e}_2}\\
%        &\text{where}\; \Gamma_e = p\; ({\Gamma_e}_1\; \circ\; {\Gamma_e}_2)\\
%        &\text{and}\; \delta = \bigcap_{D \in limits(\phi)}follows(D)\\
%        %
%        \runa{t-ref} &\; \infrule{}{p,\;\Gamma_e \vdash x : ok \dashv p,\;\Gamma_e}\\
%        %
%        \runa{t-nil} &\; \infrule{}{p,\;\Gamma_e \vdash \mathbf{0} : ok \dashv p,\;\Gamma_e}\\
%    \end{align*}
%    \caption{Type rules for editor expressions.}
%    \label{tab:typerules}
%\end{table*}

\begin{theorem} (Subject reduction)
If $\Gamma_e, \;\Gamma_a \vdash \conf{E,\;a} : ok$ and $\conf{E, a} \xrightarrow{\alpha} \conf{E', a'}$ then $\Gamma_e, \;\Gamma_a \vdash \conf{E',\;a'} : ok$.
\end{theorem}

We define \textit{well-typedness} of a configuration $\conf{E,\;a}$ by the following rule: \\
$\condinfrule{\Gamma_a \vdash a : \tau \quad p,\; \Gamma_e \vdash E : ok}{\Gamma_e, \;\Gamma_a \vdash \conf{E,\;a} : ok}{\begin{align*}
        &\text{where}\;\\
        &\text{and}\;\end{align*}}$
        
        

\chapter{Sized types for parallel complexity}\label{ch:bgts}
In this chapter, we briefly discuss the type system for parallel complexity of message-passing processes introduced in Baillot and Ghyselen \cite{BaillotGhyselen2021}. This type system builds on the foundations of indices and constraint judgements and formalizes parallel complexity analysis of $\pi$-calculus processes. Due to extensive use of subtyping and the challenges involved in verifying and satisfying constraint judgements, substantial modifications must be made to enable type checking and type inference of processes. We address these topics in Chapter \ref{ch:typecheck} and \ref{ch:timeinference}, respectively.\\

The type system for parallel complexity of message-passing processes introduced by Baillot and Ghyselen uses sized types to express parametric complexity of replicated input invocation, and thereby achieves precise bounds on primitively recursive processes: A class of processes behaving as primitively recursive functions. This requires a notion of polymorphism in the message types of replicated inputs. Baillot and Ghyselen introduce size polymorphism by bounding sizes of algebraic terms and synchronizations on channels with indices that may contain index variables representing unknown sizes. We may interpret an index with an index valuation that maps its index variables to naturals, such that the index may be evaluated to a natural number.\\

We first formally define indices and constraints on the valuations of indices. We give both a predicate logic and a model-theoretic interpretation of judgements on such constraints, referring to these as \textit{constraint judgements}. We then define sized types, the subtyping relation and introduce non-algorithmic type rules.

%\section{A type checker}\label{Sec:typesystembg}
\section{Indices and constraint judgements}\label{sec:indicesandjudgements}
In the type system by Baillot and Ghyselen, indices are used to keep track of sizes of inputs received on replicated inputs. As these sizes may be parametric, in that they may be dependent on the sizes of values received on replicated inputs, we view indices as algebraic expressions consisting of index variables $i,j,k\in\mathcal{V}$ ranging over a countable set, and function symbols, using meta-variable $f$, that may represent natural number constants as nullary functions as well as algebraic operators
\begin{align*}
    I,J ::= i \mid f(I_1,I_2,\dots,I_n)
\end{align*}
Each function symbol $f$ has an arity $\text{ar}(f)$ and an interpretation $[\![f]\!] : \mathbb{N}^{\text{ar}(f)} \rightarrow \mathbb{N}$. For the interpretation of binary difference, we assume that $[\![-]\!](n,m) = 0$ when $m \geq n$, which we refer to as the \textit{monus} operator. As indices may contain index variables, we assume some index valuation $\rho : \mathcal{V} \rightarrow \mathbb{N}$, and extend the definition of interpretations to indices, such that $[\![I]\!]_\rho$ is a natural number instance of index $I$, according to index valuation $\rho$, where for all $i$ in $I$, $\rho(i)$ substitutes for $i$ denoted $I\{\rho(i)/i\}$. Index substitution is defined in Definition \ref{def:indexsubstitution}. Based on the structure of the process that indices are used in the typing of, we may be able to establish relationships between the instances of these indices. For instance, a replicated input may receive values of sizes defined by an interval of two indices $[I,J]$. Then, we are only interested in index valuations $\rho$ that satisfy $[\![I]\!]_\rho \leq [\![J]\!]_\rho$. To express such relationships, we define binary constraints on indices in Definition \ref{def:indexconstr}.

\begin{defi}\label{def:indexsubstitution}
    We define index substitution by the following rules
    \begin{align*}
        i\substi{I}{j} &= j \text{ if } i = j\\
        i\substi{I}{j} &= i \text{ if } i \not = j\\
        f(I_1, I_2, \dots, I_n)\substi{J}{i} &= f(I_1\substi{J}{i}, I_2\substi{J}{i}, \dots, I_n\substi{J}{i})
    \end{align*}
\end{defi}

\begin{defi}[Index constraints]\label{def:indexconstr}
    Given a finite set of index variables $\varphi\subset \mathcal{V}$, we define a constraint $C$ on $\varphi$ to be an expression of the form $I \bowtie J$, where $I$ and $J$ are indices with all free index variables in $\varphi$ and $\bowtie\;\in\{\leq,=,\geq\}$ is a binary relation on $\mathbb{N}$. A finite set of constraints is represented by meta-variable $\Phi$.
\end{defi}
%
A constraint $I \bowtie J$ on $\varphi$ is satisfied given an index valuation $\rho : \varphi \longrightarrow \mathbb{N}$ when $[\![I]\!]_\rho \bowtie [\![J]\!]_\rho$ is satisfied, denoted $\rho \vDash I \bowtie J$. For a finite set of constraints $\Phi$, we write $\rho\vDash \Phi$ when $\rho \vDash C$ holds for all $C \in \Phi$. Finally, $\varphi;\Phi\vDash C$ holds when for all index valuations $\rho$ such that $\rho\vDash \Phi$ holds, we also have $\rho\vDash C$. That is, $\varphi;\Phi\vDash C$ holds exactly when $C$ does not impose further restrictions on index valuations on $\varphi$. Such judgements are fundamental to the type system by Baillot and Ghyselen, especially ones of the form $\varphi;\Phi\vDash I \leq J$, as they impose a partial order on indices wrt. how indices may be interpreted. This enables a notion of subtyping for parametric complexities, such that only indices that are greater or equal may substitute, thus preserving upper bounds on the global parallel complexity, as we shall see in the following sections.
%
%\section{The typechecker}





\begin{table*}[!ht]
    \begin{framed}\vspace{-1em}\begin{align*}
        %
        % S-nil
        &\kern-0.5em\runa{U-nil}\infrule{}{\varphi;\Phi;\Gamma \vdash \nil \triangleleft \{0\}}
        % S-nu
        \kern-2em\runa{U-nu}\infrule{\varphi;\Phi;\Gamma, a:T \vdash P \triangleleft \kappa}{\varphi;\Phi;\Gamma \vdash \newvar{a:T}{P} \triangleleft \kappa}
        % S-par
        \kern-1em\runa{U-par}\infrule{\varphi;\Phi;\Gamma \vdash P \triangleleft \kappa \quad \varphi;\Phi;\Delta \vdash Q \triangleleft \kappa'}{\varphi;\Phi;\Gamma \mid \Delta \vdash P \mid Q \triangleleft \text{basis}(\varphi, \Phi,\kappa \cup \kappa')}\\
        % S-match
        &\kern-0.5em\runa{U-match}\infrule{
        \begin{matrix}
            \varphi;\Phi;\Gamma \vdash e:\natinterval{I}{J} \quad \varphi;\Phi, I \leq 0;\Gamma \vdash P \triangleleft \kappa\\
            \varphi;\Phi, J \geq 1;\Gamma, x:\natinterval{I-1}{J-1} \vdash Q \triangleleft \kappa'
        \end{matrix}}{\varphi;\Phi;\Gamma \vdash \match{e}{P}{x}{Q} \triangleleft \text{basis}(\varphi, \Phi, \kappa \cup \kappa')}
        % S-tick
        \kern16em\runa{S-tick}\infrule{\varphi;\Phi;\Gamma \vdash P \triangleleft \kappa}{\varphi;\Phi;\uparrow^1\!\!\Gamma \vdash \tick P \triangleleft \kappa + 1}\\
        % S-iserv
        &\runa{S-iserv}\infrule{\begin{matrix}
            \texttt{in} \in \sigma\quad \varphi;\Phi;\Gamma\vdash a:\servt{I}{i}{\sigma}{K}{\widetilde{T}}\\
            (\varphi, \widetilde{i}); \Phi; \text{ready}(\varphi,\Phi,\tforwardsim{\Gamma}{I}), \widetilde{v} : \widetilde{T} \vdash P \triangleleft \kappa \quad (\varphi,\widetilde{i});\Phi\vDash\kappa \leq K
        \end{matrix}}
        {\varphi;\Phi;\Gamma \vdash \;\bang\inputch{a}{\widetilde{v}}{}{P}\triangleleft \{I\}}
         % S-ich
        \kern15em\runa{S-ich}\infrule{\begin{matrix}
            \texttt{in} \in \sigma\quad \varphi;\Phi;\Gamma \vdash a:\chant{\sigma}{I}{\widetilde{T}}\\
            \varphi; \Phi; \tforwardsim{\Gamma}{I}, \widetilde{v}:\widetilde{T} \vdash P \triangleleft \kappa
        \end{matrix}}
        {\varphi;\Phi;\Gamma \vdash \inputch{a}{\widetilde{v}}{}{P} \triangleleft \kappa + I}\\
        % S-oserv
        &\runa{S-oserv}\infrule{\begin{matrix}
            \texttt{out} \in \sigma\quad \varphi;\Phi;\Gamma\vdash a:\servt{I}{i}{\sigma}{K}{\widetilde{T}}\\
            \varphi; \Phi;(\tforwardsim{\Gamma}{I}) \vdash \widetilde{e}:\widetilde{S} \quad \text{instantiate}(\widetilde{i}, \widetilde{S}) = \{\widetilde{J}/\widetilde{i}\} \quad \varphi;\Phi \vDash \widetilde{S} \sqsubseteq \widetilde{T}
        \end{matrix}}
        {\varphi;\Phi;\Gamma \vdash \asyncoutputch{a}{\widetilde{e}}{}\triangleleft \{K\{\widetilde{J}/\widetilde{i}\} + I\}}
        % S-och
        \kern15em\runa{S-och}\infrule{\begin{matrix}
            \texttt{out} \in \sigma\quad \varphi;\Phi;\Gamma \vdash a:\chant{\sigma}{I}{\widetilde{T}}\\
            \varphi; \Phi; \tforwardsim{\Gamma}{I} \vdash \widetilde{e}:\widetilde{S} \quad \varphi;\Phi \vDash \widetilde{S} \sqsubseteq \widetilde{T}
        \end{matrix}}
        {\varphi;\Phi;\Gamma \vdash \asyncoutputch{a}{\widetilde{e}}{} \triangleleft \{I\}}\\
        % S-annot
        &\runa{S-annot}\infrule{\varphi;\Phi;\tforwardsim{\Gamma}{n}\vdash P \triangleleft \kappa}{\varphi;\Phi;\Gamma\vdash n:P \triangleleft \kappa + n}
    \end{align*}\vspace{-1em}\end{framed}
    \smallskip
    \caption{Usage typing rules for parallel complexity of processes.}
    \label{tab:usageprocesstypingrules}
\end{table*}
\subsection{Alternative formulations of constraint judgements}\label{sec:cjalternativeform}
There are several equivalent formulations of the problem of verifying the judgement $\varphi;\Phi\vDash C_0$. One such formulation is that the judgement holds, when the conjunction of constraints in $\Phi$ implies $C_0$, i.e. assuming that $n \bowtie m$ evaluates to a truth value based on membership in the relation $\bowtie$, the predicate formula $C_1 \land \cdots \land C_n \implies C_0$, where $\Phi = \{C_1,\dots,C_n\}$, must be satisfied for all valuations $\rho$ over $\varphi$. That is, let $C_i = I_i \bowtie_i J_i$, then for any valuation $\rho : \varphi \rightarrow \mathbb{N}$, the formula $([\![I_1]\!]_\rho \bowtie_1 [\![J_1]\!]_\rho) \land \cdots \land ([\![I_n]\!]_\rho \bowtie_n [\![J_n]\!]_\rho) \implies [\![I_0]\!]_\rho \bowtie_0 [\![J_0]\!]_\rho$ must be satisfied. Another interpretation of the problem is that the intersection of the feasible regions of all (inequality) constraints in $\Phi$ must be contained in the feasible region of $C_0$, or equivalently, the set of all valuations over $\varphi$ that satisfy all the constraints in $\Phi$, referred to as the model space of $\Phi$ wrt. $\varphi$, $\mathcal{M}_\varphi(\Phi)$ must be a subset of the model space of $C_0$ wrt. $\varphi$
\begin{equation*}
    \mathcal{M}_\varphi(\Phi) \subseteq \mathcal{M}_\varphi(\{C_0\})\quad\text{where}\quad\mathcal{M}_\varphi(\Phi)=\{\rho : \varphi \rightarrow \mathbb{N} \mid \rho \vDash C\;\text{for}\; C \in \Phi\}
\end{equation*}
or equivalently
\begin{equation*}
    \forall \rho \in \mathcal{M}_\varphi(\Phi) (\rho \in \mathcal{M}_\varphi(\{C_0\}))
\end{equation*}

Finally, given the fact that the current statement of the problem is expressed using a universal quantifier, we can negate the problem, obtaining a problem that can instead be expressed using an existential quantifier by the fact that $\neg \forall x P(x)$ is equivalent to $\exists x \neg P(x)$. This means the problem can also be expressed as 
%
\begin{equation*}
    \neg (\exists \rho \in \mathcal{M}_\varphi(\Phi) (\rho \not\in \mathcal{M}_\varphi(\{C_0\})))
\end{equation*}
or equivalently
\begin{equation*}
    \mathcal{M}_\varphi(\Phi) \cap \mathcal{M}_\varphi'(\{C_0\}) = \emptyset \quad\text{where}\quad
    \begin{matrix}
        \mathcal{M}_\varphi(\Phi)=\{\rho : \varphi \rightarrow \mathbb{N} \mid \rho \vDash C\;\text{for all}\; C \in \Phi\}\\
        \mathcal{M'}_\varphi(\Phi)=\{\rho : \varphi \rightarrow \mathbb{N} \mid \rho \not\vDash C\;\text{for some}\; C \in \Phi\}
    \end{matrix}
\end{equation*}
Notice that $\mathcal{M}_\varphi'(\{C\})$ is equivalent to $\mathcal{M}_\varphi(\{C'\})$ where $C'$ is the inverse constraint of constraint $C$, and so $\mathcal{M}_\varphi(\Phi) \cap \mathcal{M}_\varphi'(\{C_0\}$) = $\mathcal{M}_\varphi(\Phi \cup \{C_0'\})$ given some method to invert constraints. Thus, the problem can also be expressed simply as
\begin{equation*}
    \mathcal{M}_\varphi(\Phi \cup \{C_0'\}) = \emptyset \quad \text{where } C_0' = \text{inverse of } C_0
\end{equation*}

In Example \ref{exmp:judgementsatisfaction}, we show how a judgement can be verified manually using the predicate logic and model-theoretic interpretations of judgements provided above.
%
\begin{examp}\label{exmp:judgementsatisfaction}
    Given index variables $\varphi = \{i, j, k\}$ and constraints $\Phi = \{C_1, C_2, C_3, C_4\}$ where
    \begin{align*}
        C_1 &= i \geq 4\\
        C_2 &= j \geq 2\\
        C_3 &= -k + 3 < 0\\ % k \leq 4
        C_4 &= i + j + k \leq 11
    \end{align*}
    we want to check if $\varphi; \Phi \vDash 2i + j^2 + 3k \geq 20$ always holds. %For this example we assume interpretations are as expected from usual mathematical notation.\\
    Namely, we are interested in verifying whether the constraint $2i + j^2 + 3k \geq 20$ imposes any additional constraints to the index variables $i$, $j$ and $k$ given the existing constraints $C_1$, $C_2$, $C_3$ and $C_4$. In this case, we can notice that the minimum values of $i$, $j$ and $k$ are $4$, $2$ and $4$ respectively. As such, given these constraints, the minimum value $2i + j^2 + 3k$ may evaluate to is $2 \cdot 4 + 2^2 + 3 \cdot 4 = 24$. As such, we can conclude that $\varphi; \Phi \vDash 2i + j^2 + 3k \geq 20$ always holds.\\
    
    We can also consider the predicate logic interpretation of the example. It suffices to only consider the index valuations that satisfy the conjunction of constraints, of which there are four. Here, we represent a valuation $\rho$ as a set of pairs of the form $\{(i,\rho(i)) \mid i\in\varphi\}$, and so we have $\{(i,4),(j,2),(k,4)\}$, $\{(i,5),(j,2),(k,4)\}$, $\{(i,4),(j,3),(k,4)\}$ and $\{(i,4),(j,2),(k,5)\}$. We can then verify the corresponding implications to show that the judgement holds
    %
    \begin{align*}
        (4 \geq 4) \land (2 \geq 2) \land ({-4}+3 < 0) \implies 4+2+4 \leq 11\\
        %
        (5 \geq 4) \land (2 \geq 2) \land ({-4}+3 < 0) \implies 5+2+4 \leq 11\\
        %
        (4 \geq 4) \land (3 \geq 2) \land ({-4}+3 < 0) \implies 4+3+4 \leq 11\\
        %
        (4 \geq 4) \land (2 \geq 2) \land ({-5}+3 < 0) \implies 4+2+5 \leq 11
    \end{align*}
    Or correspondingly in model-theoretic notation
    {\small
    \begin{align*}
        \mathcal{M}_\varphi(\Phi) =&\; \{\{(i,4),(j,2),(k,4)\}, \{(i,5),(j,2),(k,4)\}, \{(i,4),(j,3),(k,4)\}, \{(i,4),(j,2),(k,5)\}\}\\
        \mathcal{M}_\varphi(\{2i+j^2+3k\geq 20\}) =&\; \{\{(i,n_1),(j,n_2),(k,n_3)\} \mid n_1,n_2,n_3\in\mathbb{N},\; 2n_1 + n_2^2 + 3n_3 \geq 20 \}\\
        \mathcal{M}_\varphi(\Phi) \subseteq&\; \mathcal{M}_\varphi(\{2i+j^2+3k\geq 20\})
    \end{align*}}
    % \begin{align*}
    %     \mathcal{M}_\varphi(\Phi) = \left\{\{(i,4),(j,2),(k,4)\}, \{(i,5),(j,2),(k,4)\}, \{(i,4),(j,3),(k,4)\}, \{(i,4),(j,2),(k,5)\}\right\}
    % \end{align*}
    
    We can also solve the inverse of the mode-theoretic interpretation of the problem. Then we want to show that $\mathcal{M}_\varphi(\Phi) \cap \mathcal{M}_\varphi'(\{2i+j^2+3k\geq 20\}) = \emptyset$ or equivalently $\mathcal{M}_\varphi(\Phi \cup \{2i+j^2+3k < 20\}) = \emptyset$. 
    %
    \begin{align*}
        \mathcal{M}_\varphi(\{2i+j^2+3k < 20\}) =&\; \{\{(i,n_1),(j,n_2),(k,n_3)\} \mid n_1,n_2,n_3\in\mathbb{N},\; 2n_1 + n_2^2 + 3n_3 < 20 \}\\
        &\kern-9em\mathcal{M}_\varphi(\Phi) \cap \mathcal{M}_\varphi'(\{2i+j^2+3k\geq 20\}) = \mathcal{M}_\varphi(\Phi \cup \{2i+j^2+3k < 20\}) = \emptyset
    \end{align*}
\end{examp}
\section{Types and subtyping}\label{sec:typesandsubs}
We now introduce the types from the type system of Baillot and Ghyselen. The types include a base type describing naturals as algebraic terms with sizes bounded by an interval consisting of two indices. This enables us to statically reason about how sizes of data structures change throughout reduction of processes, providing us termination guarantees for some forms of recursion. The type system of Baillot and Ghyselen contains lists as an additional base type, however for conciseness of the type system, we only consider naturals.
%
\begin{align*}
    T,S ::=&\; \texttt{Nat}[I,J] \mid \texttt{ch}_I^\sigma(\widetilde{T}) \mid \forall_I\widetilde{i}.\texttt{serv}_K^\sigma(\widetilde{T})
\end{align*}
%
We use input/output types for channels, and we further distinguish between channels that have replicated inputs, i.e. channels that have recursive behavior, and those that do not. We refer to the former as \textit{servers}, and we more specifically require all inputs on such channels to be replicated for technical convenience. Both servers and normal channels are annotated with an index $I$ that for a normal channel represents the number of time steps remaining before the channel synchronizes, and for a server the remaining time before it becomes available. Note that this imposes a temporal linearity constraint onto normal channels, as such channels can synchronize at exactly one time step. For servers we have an additional index $K$ that represents the parametric complexity of invoking the continuation of a replicated input on the server. Finally, the set $\sigma \subseteq \{\texttt{in},\texttt{out}\}$ is a subset of use-capabilities. Since types consist partly of indices, we define index substitution on types in Definition \ref{def:typeindexsubstitution}.\\

\begin{defi}\label{def:typeindexsubstitution}
    We define index substitution on types by the following rules
    \begin{align*}
        \natinterval{I}{J}\substi{K}{i} &= \natinterval{I\substi{K}{i}}{J\substi{K}{i}}\\
        \chant{\sigma}{I}{\widetilde{T}}\substi{J}{i} &= \chant{\sigma}{I\substi{J}{i}}{\widetilde{T}\substi{J}{i}}\\
        \servt{I}{\widetilde{i}}{\sigma}{K}{\widetilde{T}}\substi{J}{j} &= \servt{I\substi{J}{j}}{\widetilde{i}}{\sigma}{K}{\widetilde{T}} \text{ if } j \in \widetilde{i}\\
        \servt{I}{\widetilde{i}}{\sigma}{K}{\widetilde{T}}\substi{J}{j} &= \servt{I\substi{J}{j}}{\widetilde{i}}{\sigma}{K\substi{J}{j}}{\widetilde{T}\substi{J}{j}} \text{ if } j \not\in \widetilde{i}
    \end{align*}
\end{defi}


Subtyping for base types and types is the least reflexive relation $\sqsubseteq$ that satisfies the subtyping rules in Table \ref{tab:subtypeSized}. As the type system should provide upper bounds on the parallel complexity of processes, it is safe to weaken the bounds on the sizes of natural types. That is, we may decrease the lower bound and increase the upper bound on the sizes of such terms. For server and channel types, we may relax use-capabilities and use the subtyping relation on parameter types as well as modify the complexity bounds on servers, depending on the use-capabilities. Servers and channels of input/output capability are invariant, those of input capability are covariant and those of output capability are contravariant. That is, if a server or channel that inputs a value of type $T$ is required, then we can safely use a server or channel that inputs a subtype of $T$, respectively. Conversely, when a server or channel of output capability is required, we can safely use a channel or server that outputs a supertype of the required parameter type \cite{PierceSangiorgi1996}. This becomes apparent when we assume types $\texttt{Integer}$ and $\texttt{Real}$ such that $\texttt{Integer} \sqsubseteq \texttt{Real}$, as any process that receives reals can also safely receive integers, and any process that output reals can also safely output integers. Unlike Baillot and Ghyselen \cite{BaillotGhyselen2021}, we do not discard associations from our type contexts, rather we discard use-capabilities from channels and servers. Thus, to ensure the type checker is sound, we introduce rules $\runa{BGS-cempty}$ and $\runa{BGS-sempty}$ such that channel and server types are super types of ones with no use-capabilities.

%
\begin{table*}[h!]
    \begin{framed}\vspace{-1em}\begin{align*}
        &\kern0em\runa{BGS-nweak}\;\infrule{\varphi;\Phi\vDash I' \leq I\quad\quad \varphi;\Phi\vDash J \leq J'}{
        \varphi;\Phi\vdash \texttt{Nat}[I,J] \sqsubseteq \texttt{Nat}[I',J']}
        %
        \kern3em\runa{BGS-cinvar}\;\infrule{\varphi;\Phi\vdash \widetilde{T}\sqsubseteq\widetilde{S}\quad\quad \varphi;\Phi\vdash \widetilde{S}\sqsubseteq\widetilde{T}}{\varphi;\Phi\vdash\texttt{ch}_I^{\{\texttt{in},\texttt{out}\}}(\widetilde{T}) \sqsubseteq \texttt{ch}_I^{\{\texttt{in},\texttt{out}\}}(\widetilde{S})}\kern7em\\[-1em]
        %
        \vspace{-0.5em}
        &\kern-0em\runa{BGS-ccovar}\;\infrule{\{\texttt{in}\}\subseteq\sigma\quad\varphi;\Phi\vdash \widetilde{T}\sqsubseteq\widetilde{S}}{\varphi;\Phi\vdash \texttt{ch}_I^{\sigma}(\widetilde{T})\sqsubseteq\texttt{ch}_I^{\{\texttt{in}\}}(\widetilde{S})}\quad\quad\runa{BGS-ccontra}\;\infrule{\{\texttt{out}\}\subseteq\sigma\quad\varphi;\Phi\vdash \widetilde{S}\sqsubseteq\widetilde{T}}{\varphi;\Phi\vdash \texttt{ch}_I^{\sigma}(\widetilde{T})\sqsubseteq \texttt{ch}_I^{\{\texttt{out}\}}(\widetilde{S})}\\[-1em]
        %
        &\kern4em\runa{BGS-sinvar}\;\infrule{(\varphi,\widetilde{i});\Phi\vdash \widetilde{T}\sqsubseteq\widetilde{S}\quad\quad (\varphi,\widetilde{i});\Phi\vdash \widetilde{S}\sqsubseteq\widetilde{T}\quad\quad (\varphi,\widetilde{i});\Phi\vDash K = K'}{\varphi;\Phi\vdash
        \forall_I\widetilde{i}.\texttt{serv}^{\{\texttt{in},\texttt{out}\}}_K(\widetilde{T})
        \sqsubseteq \forall_I\widetilde{i}.\texttt{serv}^{\{\texttt{in},\texttt{out}\}}_{K'}(\widetilde{S})}\\[-1em]
        %
        \vspace{-0.5em}
        &\kern5em\runa{BGS-scovar}\;\infrule{\{\texttt{in}\}\subseteq\sigma\quad(\varphi,\widetilde{i});\Phi\vdash \widetilde{T}\sqsubseteq\widetilde{S}\quad (\varphi,\widetilde{i});\Phi\vDash K' \leq K}{\varphi;\Phi\vdash \forall_I\widetilde{i}.\texttt{serv}^{\sigma}_K(\widetilde{T})\sqsubseteq\forall_I\widetilde{i}.\texttt{serv}^{\{\texttt{in}\}}_{K'}(\widetilde{S})}\\[-1em]
        &\kern4.5em\runa{BGS-scontra}\;\infrule{\{\texttt{out}\}\subseteq\sigma\quad(\varphi,\widetilde{i});\Phi\vdash \widetilde{S}\sqsubseteq\widetilde{T}\quad (\varphi,\widetilde{i});\Phi\vDash K \leq K'}{\varphi;\Phi\vdash \forall_I\widetilde{i}.\texttt{serv}^{\sigma}_K(\widetilde{T})\sqsubseteq \forall_I\widetilde{i}.\texttt{serv}^{\{\texttt{out}\}}_{K'}(\widetilde{S})}\\[-1em]
        %
        &\kern0em\runa{BGS-cempty}\;\infrule{}{\varphi;\Phi\vdash \texttt{ch}^\sigma_I(\widetilde{S}) \sqsubseteq \texttt{ch}^\emptyset_I(\widetilde{T})}\quad\runa{BGS-sempty}\;\infrule{}{\varphi;\Phi\vdash \forall_I\widetilde{i}.\texttt{serv}^\sigma_K(\widetilde{S}) \sqsubseteq \forall_I\widetilde{i}.\texttt{serv}^\emptyset_{K'}(\widetilde{T})}
    \end{align*}\vspace{-1em}\end{framed}
    \smallskip
    \caption{Rules for subtyping of base types and types.}
    \label{tab:subtypeSized}
\end{table*}
\section{Non-algorithmic type rules}

We first consider the type rules for expressions, which are shown in Table \ref{tab:sizedtypedexpressiontypes}. The zero term $0$ intuitively receives the type $\texttt{Nat}[0,0]$ and a successor to a natural term has the same type as its predecessor, but with 1 added to its lower and upper bounds. Finally, a variable receives a type if it is bound in the type context.\\
%Lists are typed similarly, aside from the addition of an element base type. For the element type of a list, we simply use the least lower bound and greatest upper bound on the size amongst the elements of the list.

\begin{table*}[ht]
    \begin{framed}\vspace{-1em}\begin{align*}
        &\kern2em
        \runa{BG-nzero}\;\infrule{}{\varphi;\Phi;\Gamma\vdash\withtype{0}{\typenat[0,0]}}\kern0em
        \runa{BG-nsucc}\;\infrule{\varphi;\Phi;\Gamma \vdash \withtype{e}{\typenat[I, J]}}{\varphi;\Phi;\Gamma \vdash \withtype{\succc{e}}{\typenat[I + 1, J + 1]}}\\[-1em]
        %
        &\kern1em\runa{BG-sub}\;\infrule{\varphi;\Phi;\Delta\vdash e : S\quad\quad \varphi;\Phi\vdash \Gamma\sqsubseteq \Delta\quad\quad \varphi;\Phi\vdash S \sqsubseteq T}{\varphi;\Phi;\Gamma\vdash e : T}\kern11em\runa{BG-var}\;\infrule{}{\varphi;\Phi;\Gamma, \withtype{v}{T} \vdash \withtype{v}{T}}
    \end{align*}\vspace{-1em}\end{framed}
    \smallskip
    \caption{Type rules for expressions.}
    \label{tab:sizedtypedexpressiontypes}
\end{table*}

Before introducing the type rules for processes, we first introduce a function $\downarrow^{\varphi;\Phi}_I\!\!(T)$ in Definition \ref{def:delaysized} that \textit{advances} the time of type $T$ by $I$ units of time complexity. For a channel type $\texttt{ch}^\sigma_J(\widetilde{S})$, we subtract $I$ from $J$ whenever we can guarantee that $J\geq I$ under the constraints imposed on $\varphi$ by $\Phi$. Otherwise, the advancement of $I$ units of time complexity is undefined for type $\texttt{ch}^\sigma_J(\widetilde{S})$, to ensure bounds on communication are not violated. For a server type $\forall_J\widetilde{i}.\texttt{serv}^\sigma_K(\widetilde{S})$, corresponding outputs are well-typed for any timestep $I$ with $I\geq J$, and so a server simply loses input capability whenever we cannot guarantee that $J \geq I$. We extend advancement of time to contexts such that $\downarrow^{\varphi;\Phi}_I(\Gamma)(v)=\;\downarrow^{\varphi;\Phi}_I(\Gamma(v))$. When it is clear from context, we may omit $\varphi$ and $\Phi$.

\begin{defi}[Advancement of Time]\label{def:delaysized}
Let $\varphi$ be a set of index variables, $\Phi$ a set of constraints on indices, $T$ a type and $J$ an index. Then $T$ after $J$ units of time complexity, $\susume{T}{\varphi}{\Phi}{I}$, is given by the rules below
\begin{align*}
    \susume{\natinterval{I}{J}}{\varphi}{\Phi}{I} =&\; \natinterval{I}{J}\\
    %
    %\susume{\texttt{List}[J,K](\mathcal{B})}{\varphi}{\Phi}{I} =&\; \texttt{List}[J,K](\mathcal{B})\\
    %
    \susume{\texttt{ch}^\sigma_J(\widetilde{T})}{\varphi}{\Phi}{I} =&\; \left\{ \begin{matrix}
        %\texttt{ch}^\emptyset_J(\widetilde{T}) & \text{if}\; \sigma = \emptyset\\
        %
        \texttt{ch}^\sigma_{J-I}(\widetilde{T}) & \text{if}\; \varphi;\Phi \vDash J \geq I\\
        %
        \texttt{ch}^\emptyset_{0}(\widetilde{T}) & \text{if}\; \varphi;\Phi \nvDash J \geq I
    \end{matrix} \right.\\
    %
    %\texttt{ch}^\sigma_{J-I}(\widetilde{T}) & \text{if}\; \varphi;\Phi \vDash J \geq I\\
    %
    % \susume{\inchanneltypeS{J}{\widetilde{T}}}{\varphi}{\Phi}{I} =&\; 
    %  \inchanneltypeS{J-I}{\widetilde{T}} & \text{if}\; \varphi;\Phi \vDash J \geq I \\
    % %
    % \susume{\outchanneltypeS{J}{\widetilde{T}}}{\varphi}{\Phi}{I} =&\; 
    %  \outchanneltypeS{J-I}{\widetilde{T}} & \text{if}\; \varphi;\Phi \vDash J \geq I \\
    %
    \susume{\forall_J\widetilde{i}.\texttt{serv}^\sigma_K(\widetilde{T})}{\varphi}{\Phi}{I} =&\; \left\{ \begin{matrix}
        \forall_{J-I}\widetilde{i}.\texttt{serv}^\sigma_K(\widetilde{T}) & \text{if}\; \varphi;\Phi \vDash J \geq I\\
        %
        \forall_{J-I}\widetilde{i}.\texttt{serv}^{\sigma \cap \{\texttt{out}\}}_K(\widetilde{T}) & \text{if}\; \varphi;\Phi \nvDash J \geq I
    \end{matrix} \right.
    %  \servS{J - I}{\widetilde{i}}{K}{\widetilde{T}} & \text{if}\; \varphi;\Phi \vDash J \geq I \\
    % %
    % \susume{\servS{J}{\widetilde{i}}{K}{\widetilde{T}}}{\varphi}{\Phi}{I} =&\; \oservS{J - I}{\widetilde{i}}{K}{\widetilde{T}} & \text{if}\; \varphi;\Phi \vDash J \not\geq I \\          
    % %
    % \susume{\iservS{J}{\widetilde{i}}{K}{\widetilde{T}}}{\varphi}{\Phi}{I} =&\; 
    %  \iservS{J - I}{\widetilde{i}}{K}{\widetilde{T}} & \text{if}\; \varphi;\Phi \vDash J \geq I \\
    % %
    % \susume{\oservS{J}{\widetilde{i}}{K}{\widetilde{T}}}{\varphi}{\Phi}{I} =&\; \oservS{J - I}{\widetilde{i}}{K}{\widetilde{T}}
\end{align*}
\end{defi}

\begin{defi}[Time invariance]\label{def:timeinvariance}
Let $\Gamma$ be a type context. We say that $\Gamma$ is \textit{time invariant} if it only contains variables of either base types or server type with time $0$ and use-capabilities $\sigma$ such that $\sigma\subseteq\{\texttt{out}\}$, i.e. $\forall_0\widetilde{i}.\texttt{serv}^{\sigma}_K(\widetilde{T})$ for some index variables $\widetilde{i}$, types $\widetilde{T}$ and index $K$.
\end{defi}

We now present the type rules of the type system by Baillot and Ghyselen, adapted to fit our syntax. Type judgements are of the form $\varphi;\Phi;\Gamma \vdash P \triangleleft K$, which means that process $P$ has complexity $K$ given constraints $\Phi$ with index variables in $\varphi$ and given a type environment $\Gamma$. The type rules are defined in Table \ref{tab:bgprocesstypingrules}. Rule $\runa{BG-iserv}$ handles replicated inputs and ensures that name $a$ is bound to a server type with input capability in the type context. We must also make sure that in the continuation $P$, the type context must be time invariant as the replicated input may be invoked any number of times after $I$ units of time have elapsed. Thus, only free naturals and servers with no input capability are safe. Rule $\runa{BS-ich}$ is similar except we do not require the type context in the continuation to be time invariant as it is only used once. Rule $\runa{BG-oserv}$ types output servers and most notably uses polymorphism in the index variables $\widetilde{i}$. As such, when typing the expressions sent on the server, we must ensure that we can \textit{instantiate} the index variables of the server using a substitution. Finally, type rule $\runa{BG-match}$ shows how index constraints are introduced when typing processes by utilizing information gained from the two branches of the match expression.\\

% Examples
We now show how a server calculating the $n$th digit of the Fibonacci sequence can be typed. Before presenting the process for the implementation of Fibonacci's sequence, we first need to encode addition in the $\pi$-calculus, which we do using the \textit{add} server as follows.
%
\begin{align*}
    P_\text{add}\defeq&\;\bang\inputch{\text{add}}{x,y,r}{}{
        \texttt{match}\; x\; \{
             0 \mapsto \asyncoutputch{r}{y}{};
            \succc{z} \mapsto \asyncoutputch{\text{add}}{z,\succc{y},r}{}\}}
    %
\end{align*}

The \textit{add} server needs three inputs $x$, $y$, and $r$. The parameters $x$ and $y$ represent two naturals to be added, and $r$ represents the channel intended for receiving the result. Note that no ticks are included in the server as we assume that addition can be done in constant time. The following process for calculating the $n$th number of the Fibonacci sequence is a naïve recursive implementation calculating $\textit{fib}(n)=\textit{fib}(n-1)+\textit{fib}(n-2)$. The server takes two parameters $n$ and $r$ where $n$ is the number of the Fibonacci sequence to calculate and $r$ represents the channel intended for receiving the result.
%
\newcommand{\funcf}[0]{l}
\newcommand{\funcg}[0]{l}
\newcommand{\funcgp}[0]{l-1}
\newcommand{\funcgpp}[0]{l-2}
\newcommand{\funcgppp}[0]{l-1}
\begin{align*}
    P_\text{fib}\defeq&\; \bang\inputch{\text{fib}}{n,r}{}{
         \texttt{match}\; n\; \{ 0 \mapsto \asyncoutputch{r}{0}{}\!;\;
              \succc{n_1} \mapsto\\ 
              &\quad\texttt{match}\; n_1\; \{
                    0 \mapsto \asyncoutputch{r}{\succc{0}}{}\!;\;
                    \succc{n_2} \mapsto\\ &\quad\quad\newvar{r_1,r_2,r_3}{(\asyncoutputch{\text{fib}}{n_1,r_1}{}\mid\asyncoutputch{\text{fib}}{n_2,r_2}{}\\
    &\quad\quad\mid\inputch{r_2}{m_2}{}{\inputch{r_1}{m_1}{}{\tick{\asyncoutputch{\text{add}}{m_1,m_2,r_3}{}}}\mid \inputch{r_3}{m_3}{}{\asyncoutputch{r}{m_3}{}}})}\}\}
    }
\end{align*}

Finally we present a type context $\Gamma$ under which the two servers \textit{add} and \textit{fib} are well-typed. Note that even though we use a naïve implementation of the Fibonacci sequence, we can still get a linear bound as the program runs in parallel.
%
\begin{align*}
    \Gamma \defeq&\; \text{add} : \servt{0}{i,j,k}{\{\texttt{in},\texttt{out}\}}{0}{\texttt{Nat}[0,i],\texttt{Nat}[j,k],\channeltypeS{0}{\texttt{Nat}[j,i+k]}},\\
    &\;\text{fib} : \servt{0}{l}{\{\texttt{in},\texttt{out}\}}{\funcf}{\texttt{Nat}[0,l],\channeltypeS{\funcg}{\texttt{Nat}[0,\textit{fib}(l)]}}
\end{align*}

\begin{table*}
    \begin{framed}\vspace{-1em}\begin{align*}
        &\kern46em\\[-2em] % Stretch frame
        &\kern0em\runa{BG-zero}\infrule{}{\varphi;\Phi;\Gamma \vdash \withcomplex{\nil}{0}}\!\!
        \runa{BG-subtype}\;\infrule{\varphi;\Phi;\Delta \vdash \withcomplex{P}{K} \quad \varphi;\Phi \vdash \Gamma \sqsubseteq \Delta \quad \varphi; \Phi \vDash K \leq K'}{\varphi;\Phi;\Gamma \vdash \withcomplex{P}{K'}}
        \\[-1em]
        %
        &\kern-0em\runa{BG-match}\;\infrule{\varphi;\Phi;\Gamma \vdash \withtype{e}{\natinterval{I}{J}} \quad \varphi;\Phi, I \leq 0;\Gamma \vdash \withcomplex{P}{K} \quad \varphi;\Phi, J \geq 1;\Gamma, \withtype{x}{\natinterval{I\monus 1}{J\monus 1}} \vdash \withcomplex{Q}{K}}{\varphi;\Phi;\Gamma \vdash \withcomplex{\match{e}{P}{x}{Q}}{K}}\\[-1em]
        %
        &\kern4em\runa{BG-par}\;\infrule{\varphi;\Phi;\Gamma\vdash P \triangleleft K\quad \varphi;\Phi;\Gamma\vdash Q \triangleleft K}{\varphi;\Phi;\Gamma\vdash \parcomp{P}{Q} \triangleleft K}\quad\quad\quad\quad\quad\quad \runa{BG-tick}\;\infrule{\varphi;\Phi;\susumesim{\Gamma}{1}\vdash P \triangleleft K}{\varphi;\Phi;\Gamma\vdash \tick P \triangleleft K + 1}\\[-1em]
        %
        &\kern-0em\runa{BG-iserv}\;\infrule{\texttt{in}\in\sigma\quad \varphi;\Phi\vdash\;\susumesim{\Gamma}{I},a:\forall_0\widetilde{i}.\texttt{serv}^\sigma_K(\widetilde{T}) \sqsubseteq \Gamma'\;\text{and}\; \Gamma'\;\text{time invariant}\quad \varphi,\widetilde{i};\Phi;\Gamma',\widetilde{v} : \widetilde{T}\vdash P \triangleleft K}{\varphi;\Phi;\Gamma,\Delta,a : \servt{I}{\widetilde{i}}{\sigma}{K}{\widetilde{T}}\vdash\; \bang\inputch{a}{\widetilde{v}}{}{P}\triangleleft I}\\[-1em]
        %
        &\kern-0em\runa{BG-ich}\;\infrule{\texttt{in}\in\sigma\quad \varphi;\Phi;\susumesim{\Gamma}{I},\widetilde{v} : \widetilde{T}, a : \chant{\sigma}{0}{\widetilde{T}}\vdash P \triangleleft K}{\varphi;\Phi;\Gamma, a : \chant{\sigma}{I}{\widetilde{T}}\vdash \inputch{a}{\widetilde{v}}{}{P}\triangleleft K + I}\kern8.5em \runa{BG-och}\;\infrule{\texttt{out}\in\sigma\quad \varphi;\Phi;\susumesim{\Gamma}{I}\vdash \widetilde{e} : \widetilde{T}}{\varphi;\Phi;\Gamma,a:\chant{\sigma}{I}{\widetilde{T}}\vdash \asyncoutputch{a}{\widetilde{e}}{} \triangleleft I}\\[-1em]
        %
        &\kern2em\runa{BG-oserv}\;\infrule{\texttt{out}\in\sigma\quad \varphi;\Phi;\susumesim{\Gamma}{I}\vdash \widetilde{e} : \widetilde{T}\substi{\widetilde{J}}{\widetilde{i}}}{\varphi;\Phi;\Gamma, a : \servt{I}{\widetilde{i}}{\sigma}{K}{\widetilde{T}}\vdash \asyncoutputch{a}{\widetilde{e}}{} \triangleleft K\!\substi{\widetilde{J}}{\widetilde{i}} + I}\kern12em \runa{BG-nu}\;\infrule{\varphi;\Phi;\Gamma,\withtype{a}{T} \vdash \withcomplex{P}{K}}{\varphi;\Phi;\Gamma \vdash \newvar{a}{\withcomplex{P}{K}}}
    \end{align*}\vspace{-1em}\end{framed}
    \smallskip
    \caption{Sized typing rules for parallel complexity of processes.}
    \label{tab:bgprocesstypingrules}
\end{table*}
\section{Examples of invalid configurations}
The following examples are written in the format $\conf{E, a}$, where $E$ is an editor expression and $a$ is the AST on which we apply the editor expression. \\

In equation \ref{condsubproblem} we show how conditioned substitution can cause problems.
\begin{equation}
    \conf{\left(@\texttt{break} \Rightarrow \replace{\texttt{break}}\right) \ggg \texttt{child}\; 1,\; \lambda x.\hole\; \cursor{\breakpoint{c}}} \label{condsubproblem}
\end{equation}
 In the example we check if the cursor is at a breakpoint, and since the check is true we \textit{toggle} the breakpoint thereby making the following \texttt{child} 1 command problematic. The constant c cannot have a child which means this configuration would cause a run-time error. \\
 
In equation \ref{parentproblem} we show how using the \texttt{parent} command can cause problem when the root is unknown.
\begin{equation}
    \conf{\left(\lozenge\texttt{hole} \Rightarrow \texttt{parent}\right) \ggg \texttt{parent},\; \cursor{\lambda x.\hole}\; c} \label{parentproblem}
\end{equation}
In the example we first check if there is a hole in some subtree of the current cursor. This condition holds and we therefore evaluate the \texttt{parent} command resulting in the AST $\cursor{\lambda x.\hole\; c}$. When the next \texttt{parent} command is evaluated we have a run-time error since we are already situated at the root.\\

In equation \ref{astproblem} we show how an editor expression can result in an AST that would cause a run-time error when evaluated.
\begin{equation}
    \conf{\left(\neg\Box(\texttt{lambda}\; x) \Rightarrow \texttt{child}\; 1\right) \ggg \replace{\texttt{var}\; x}.\texttt{eval},\; \cursor{\lambda x.\hole}\; c} \label{astproblem}
\end{equation}
In the example we first check if it is \textbf{not} necessary that the subtree of the cursor contains a lambda expression. This condition does not hold since it is necessary. Since the condition does not hold we do not evaluate the \texttt{child} 1 command, which means the following substitution of \texttt{var} x is problematic. The substitution results in the AST $\cursor{\texttt{var}\; x}\; c$, which causes a run-time error when the command \texttt{eval} is evaluated, since the left child of the function application is no longer a function.
%
\section{Over-approximations}
As we cannot determine statically whether a condition holds, we establish over-approximations to ensure run-time errors cannot occur in well-typed configurations. As equation \ref{parentproblem} shows, conditioned expressions can result in loss of information about the cursor location. As such, we enforce the cursor \textit{depth} in the tree to be the same before and after a conditioned expression. Furthermore, the first cursor movement in a conditioned expression must be a \texttt{child} prefix. As equation \ref{condsubproblem} shows, conditioned substitution also results in loss of information. Thus, we can no longer guarantee that subsequent substitution at a deeper level is well-typed. Similarly, we no longer know of the structure of the subtree, such that we must condition \texttt{child} prefixes.\\

The above discussion leads to the following list of over-approximations:
\begin{itemize}
    \item In conditioned and recursive expressions, the cursor depth must be the same before and after.
    \item In conditioned and recursive expressions, only the subtree encapsulated by the cursor is accessible.
    \item After conditioned substitution, subsequent substitution at a deeper level is no longer valid, and the \texttt{child} prefix command must be conditioned.
\end{itemize}
%
\section{AST type rules}
\begin{table*}[htp]
    \centering
    \begin{align*}
        \runa{t-var} &\; \infrule{\Gamma_a\left(x\right)=\tau}{\Gamma_a \vdash x : \tau}\\
        %
        \runa{t-const} &\; \infrule{}{\Gamma_a \vdash c : b}\\
        %
        \runa{t-app} &\; \infrule{\Gamma_a \vdash a_1 : \tau_1 \rightarrow \tau_2 \quad \Gamma_a \vdash a_2 : \tau_1}{\Gamma_a \vdash a_1\; a_2 : \tau_2}\\
        %
        \runa{t-lambda} &\; \infrule{\Gamma_a\left[x \mapsto \tau_1\right] \vdash a : \tau_2}{\Gamma_a \vdash \lambda x:\tau_1.a : \tau_1 \rightarrow \tau_2} \\
        %
        \runa{t-break} &\; \infrule{\Gamma_a \vdash a : \tau}{\Gamma_a \vdash \breakpoint{a} : \tau} \\
        %
        \runa{t-hole} &\; \infrule{}{\Gamma_a \vdash \left(\hole : \tau\right) : \tau}
        %
    \end{align*}
    \caption{Type rules for abstract syntax trees.}
    \label{tab:typerules}
\end{table*}

%\section{Type context format}
%Here, we propose a format for type contexts of editor expressions. The context of an editor expression could be a triple $\Psi = (\Gamma_a, \tau, \Gamma)$, where $\Gamma_a$ is the type context for the subtree encapsulated by the cursor, $\tau$ is the type of the subtree and $\Gamma$ is a function or map from prefix command types to editor expression contexts. That is, contexts for editor expressions are recursive. Say we have context $(\Gamma_a, \tau, \Gamma)$. Upon a $\texttt{child}\; 1$ prefix, we \textit{look up} $\texttt{one}$ in $\Gamma$. If $\Gamma(\texttt{one}) = undef$, the expression is not well-typed. Otherwise, we evaluate the prefixed expression in the new context $\Gamma(\texttt{one})$.\\

%We construct the initial context based on the AST in the configuration $\conf{E,\; a}$. Upon a substitution prefix, we modify the context, upon a child or parent prefix, we \textit{move} in the context, and upon a conditioned or recursive expression, we set some of the bindings to $undef$: $\Gamma(T)=undef$.\\

%$\Gamma = T_1 : \Psi_1,...,T_n : \Psi_n$ \\
%$\Psi = (\Gamma_a, \tau, \Gamma)$
%Γ = T1 : Ψ1,..,Tn : Ψn
%Ψ = (Γa, τ, Γ)

\section{Experimental type system}

In this section, we introduce a type system for our editor-calculus. For the type system, we introduce the syntactic categories $\tau \in \mathbf{ATyp}$ to denote types of AST nodes, $T \in \mathbf{CTyp}$ to denote \textit{child} types, and p $\in \mathbf{Pth}$ to denote AST paths.
%
\begin{align*}
    \tau ::=&\; b \mid \tau_1 \rightarrow \tau_2 \mid \breakpoint{\tau} \mid \texttt{indet}\\
    T ::=&\; \texttt{one} \mid \texttt{two}\\
    p ::=&\; p\; T \mid \epsilon
\end{align*}

In addition to the basic and arrow types in $\mathbf{ATyp}$, we include a type for breakpoints, $\breakpoint{\tau}$, and a type to denote indeterminate types, \texttt{indet}. We use $\mathbf{Pth}$ to denote paths in an AST by storing a sequence of \textbb{one} and \textbb{two} which denote if the path goes through the first or second child.\\

We define two sets for contexts in our type system. The first context, $\mathbf{ACtx}$, stores type bindings for variables in the AST. The second context, $\mathbf{ECtx}$, stores, for all available paths so far, a pair of an AST context and the type of the node at the end of the path. We use $\Gamma_a \in \mathbf{ACtx}$ and $\Gamma_e \in \mathbf{ECtx}$ as metavariables for the two contexts. To check if a path $p$ is available in a context $\Gamma_e$, we use the notion $\Gamma_e(p) \neq \text{undef}$. $\mathbf{ACtx}$ and $\mathbf{ECtx}$ are thus defined as the following.
%
\begin{align*}
\mathbf{ACtx} &= \mathbf{Var} \rightharpoonup \mathbf{ATyp}\\
\mathbf{ECtx} &= \mathbf{Pth} \rightharpoonup \left(\mathbf{ACtx} \times \mathbf{ATyp}\right)
\end{align*}

To support our type system, we modify the syntax for AST node modifications by including type annotations for application, abstraction and holes. The new syntax thus becomes the following.
%
\begin{align*}
  D ::= \; & \texttt{var}\;x \mid \texttt{const}\;c \mid \texttt{app} : \tau_1 \rightarrow \tau_2, \tau_1 \mid \texttt{lambda}\; x : \tau_1 \rightarrow \tau_2 \mid \texttt{break} \mid \texttt{hole} : \tau
\end{align*}

To support breakpoint types, we introduce the notion of type consistency into our typesystem. The purpose of consistency in our type system is to ensure breakpoints types are consistent with their respective type, as defined below.
%
\begin{definition}{(Type consistency)}
    We define two types $\tau_1, \tau_2$ to be \textit{consistent}, denoted $\tau_1 \sim \tau_2$, by the following rules.
    \begin{align*}
        \runa{cons-1} \hspace{-1cm}
        \infrule{}{\tau \sim \tau} \hspace{-1cm}
        \runa{cons-2} \hspace{-1cm}
        \infrule{}{\breakpoint{\tau} \sim \tau} \hspace{-1cm}
        \runa{cons-3} \hspace{-1cm}
        \infrule{}{\tau \sim \breakpoint{\tau}} \hspace{-1cm}
        \runa{cons-4}
        \infrule{\tau_1 \sim \tau_1' \quad \tau_2 \sim \tau_2'}{(\tau_1 \rightarrow \tau_2) \sim (\tau_1' \rightarrow \tau_2')}
    \end{align*}
\end{definition}


\begin{table*}[htp]
    \centering
    \begin{align*}
        \runa{ctx-split-1}&\; \infrule{}{\emptyset = p \left(\emptyset\; \circ\; \emptyset\right)}\\
        \runa{ctx-split-2}&\; \infrule{\Gamma_e = p \left({\Gamma_e}_1\; \circ\; {\Gamma_e}_2\right)}{\Gamma_e,\; p\; T_1..T_n: (\Gamma_a,\; \tau) = p \left(\left({\Gamma_e}_1,\; p\; T_1..T_n: (\Gamma_a,\; \tau)\right)\; \circ\; {\Gamma_e}_2\right)}\\
        \runa{ctx-split-3}&\; \infrule{p_1 \neq p_2 \quad \Gamma_e = p_2 \left({\Gamma_e}_1\; \circ\; {\Gamma_e}_2\right)}{\Gamma_e,\; p_1\; T_1..T_n: (\Gamma_a,\; \tau) = p_2 \left({\Gamma_e}_1\; \circ\; \left({\Gamma_e}_2,\; p_1\; T_1..T_n: (\Gamma_a,\; \tau)\right)\right)}\\
        %
        \runa{ctx-update-1}&\; \infrule{}{\Gamma_e = \Gamma_e + \emptyset}\\
        \runa{ctx-update-2}&\; \infrule{\Gamma_e = \left({\Gamma_e}_1,\; p: ({\Gamma_a}_2,\; \tau_2)\right) + {\Gamma_e}_2}{\Gamma_e,\; p: ({\Gamma_a}_1,\; \tau_1) = \left({\Gamma_e}_1,\; p: ({\Gamma_a}_2,\; \tau_2)\right) + {\Gamma_e}_2}\\
        \runa{ctx-update-3}&\; \infrule{\Gamma_e = {\Gamma_e}_1 + {\Gamma_e}_2}{\Gamma_e,\; p: (\Gamma_a,\; \tau) = {\Gamma_e}_1 + \left({\Gamma_e}_2,\; p: (\Gamma_a,\; \tau)\right)}
    \end{align*}
    \caption{Context split and context update for editor contexts.}
    \label{tab:context}
\end{table*}
% We define \textit{type contexts}, $\Gamma_e$ in Table \ref{tab:context} as a mapping from a path $p$ to a pair consisting of an AST context $\Gamma_a$ and AST type $\tau$. We denote the $\Gamma_e, p : (\Gamma_a, \tau)$ as the type context equal to the paths not in the domain of map $\Gamma_e$ except for $p$, where $\Gamma_e(p) = (\Gamma_a, \tau)$. For type contexts we introduce the concept of \textit{context splitting} on a path in terms of $\Gamma_e$ maintained through two sub-contexts $\Gamma_{e1}$ and $\Gamma_{e2}$. For this we require a split-operation $\circ$, defined for two sub-contexts on a path as $\Gamma_e = p(\Gamma_{e1}\; \circ \; \Gamma_{e2})$. Notice the empty context is defined with the symbol $\emptyset$ as in \runa{ctx-split-1}. In rule \runa{ctx-split-2} we have that $p$ is in $\Gamma_{e1}$, but not in $\Gamma_{e2}$. Thus, $p$ is not in $\Gamma = \Gamma_{e1}\; \circ \; \Gamma_{e2}$, which is similarly done for the \runa{ctx-split-3} in terms of $\Gamma_{e1}$.\\

Next we introduce the notion of \textit{context updates} to update bindings in a context with new types for the associated path $p$. We use the addition operator $+$, to denote sum-context $\Gamma$ of two compatible type contexts $\Gamma_{e1}$ and $\Gamma_{e2}$. The rules require linear paths to not have bindings exist in another context. Thus, we can only update a context $\Gamma_{e2}$ iff no bindings for a given path is in context $\Gamma_{e1}$. In rule \runa{ctx-update-2} we have bindings in $\Gamma_{e1}$, which means we cannot add bindings to $\Gamma_{e2}$. However, in rule \runa{ctx-update-3} we allow path bindings in $\Gamma_{e2}$ since no such bindings are in context $\Gamma_{e1}$.

% \begin{equation}
%     depth(e) = \left\{
%         \begin{array}{ll}
%             depth(E) + 1            & \quad if e = (\texttt{child}\; n).E \\
%             depth(E) - 1            & \quad if e = \texttt{parent}.E\\
%             depth(E_1) + depth(E_2) & \quad if e = E_1 \ggg E_2\\
%             depth(E)                & \quad if e = \texttt{rec}\; x.E\\
%             depth(E)                & \quad if e = \pi.E\\
%             0                       & \quad otherwise
%         \end{array}
%     \right.
% \end{equation}

\begin{definition}{(Relative cursor depth)}
    We define the function $depth : \mathbf{Edt} \rightarrow \mathbb{Z}$, from the set of atomic editor expression to the set of integers.
    \begin{align*}
    depth((\texttt{child}\; n).E) &= depth(E) + 1 \\
    depth(\texttt{parent}.E) &= depth(E) - 1 \\
    depth(E_1 \ggg E_2) &= depth(E_1) + depth(E_2) \\
    depth(\texttt{rec}\; x.E) &= depth(E) \\
    depth(\pi.E) &= depth(E) \\
    depth(E) &= 0 
\end{align*}
\end{definition}
The $depth$ function statically analyses the structure of an editor expression to determine the relative depth of the cursor after evaluation of the expression. This function is used to make sure the position of the cursor before and after evaluation of an expression is the same. As the function performs a static analysis, we do not consider conditioned subexpressions. Later, in the type rules, we will see why we can safely ignore conditioned subexpressions. \\


% Next we define the function $match : \mathbf{Aam} \times \mathbf{ACtx} \times \mathbf{ATyp} \rightarrow \{tt, f\!\!f\}$. This function returns true if the type of the given AST modification $D$, is equal to the given AST type $\tau$.  
% \begin{align*}
%     match(\texttt{var}\; x,\;\Gamma_a,\;\tau) &= \left\{\begin{matrix}
%  tt & \text{if}\; \Gamma_a(x) = \tau\\ 
%  f\!\!f & \text{otherwise}
% \end{matrix}\right.\\
%     match(\texttt{const}\; c,\;\Gamma_a,\; b) &= tt\\
%     match(\texttt{app} : \tau_1 \rightarrow \tau_2,\; \tau_1,\;\Gamma_a,\; \tau_2) &= tt\\
%     match(\texttt{lambda}\; x : \tau_1 \rightarrow \tau_2,\;\Gamma_a,\; \tau_1 \rightarrow \tau_2) &= tt\\
%     match(\texttt{break},\;\Gamma_a,\; \tau) &= tt\\
%     match(\texttt{hole} : \tau,\;\Gamma_a,\; \tau) &= tt\\
%     match(D,\; \Gamma_a,\; \tau) &= f\!\!f
% \end{align*}

%\begin{equation*}
%    %context : \left(\mathbf{Aam} \times \mathbf{ACtx}\right) \rightharpoonup %\left(\left(\mathbf{Pth} \rightarrow \left(\left(\mathbf{Var} \rightharpoonup %\mathbf{ATyp}\right) \times \mathbf{ATyp}\right)\right) \cup \{error\}\right)
    %context : \left(\mathbf{Aam} \times \mathbf{ACtx} \times \mathbf{Pth} \right) %\rightharpoonup \mathbf{ECtx}
%\end{equation*}
%\begin{align*}
% context(\texttt{const}\; c,\; \Gamma_a,\; p) =&\; \emptyset\\
%  context(\texttt{hole} : \tau,\; \Gamma_a,\; p) =&\; \emptyset\\
%context(\texttt{var}\; x,\; \Gamma_a,\; p) =&\; \emptyset\\
 %context((\texttt{app} : \tau_1 \rightarrow \tau2,\; \tau_1),\; \Gamma_a,\; p) =&\; %\emptyset,\; p\; \texttt{one} : (\Gamma_a,\; \tau_1 \rightarrow \tau_2),\; p\; \texttt{two} : %(\Gamma_a,\; \tau_1)\\
 %context(\texttt{lambda}\; x : \tau_1 \rightarrow \tau_2,\; \Gamma_a,\; p) =&\; \emptyset,\; %p\; \texttt{one} : ((\Gamma_a,\; x : \tau_1),\; \tau_2)
%\end{align*}
%
%

We define functions \textit{limits} and \textit{follows} to analyze which cursor movement is safe given a condition holds. \textit{limits} finds the set of possible AST node modifiers, on which the cursor may sit, given the condition holds. \textit{follows} gives a set of editor type context bindings guaranteed to be safe, given the cursor sits on AST node modifier $D$. Note that the AST type context is empty and that the node type is $\texttt{indet}$, as we cannot determine such information based on a condition. Thus, besides toggling of breakpoints, substitution is not well-typed at path $p$ if $\Gamma_e(p)=(\emptyset,\; \texttt{indet})$. We can combine functions \textit{limits} and \textit{follows} to provide additional bindings to the editor type context of a conditioned expression $\phi \Rightarrow E$. The intersection of \textit{follows} applied to each AST node modifier $D$ in the set $limits(\phi)$ is the set of bindings guaranteed to be safe, given $\phi$ holds.

\theoremstyle{definition}
\begin{definition}{(Condition constraints)}
We define a function $limits: \mathbf{Eed} \rightarrow \mathcal{P}(\mathbf{Aam})$ from the set of conditions to the power set of the set of AST node modifiers. We assume conditions are in conjunctive normal form.
\begin{align*}
    limits(@D)=&\;\{D\}\\
    limits(\neg @D)=&\;\mathbf{Aam}\setminus \{D\}\\
    limits(\lozenge D)=&\;\{D\} \cup \{\texttt{app},\; \texttt{lambda}\; x,\; \texttt{break}\}\\
    limits(\neg \lozenge D)=&\;\mathbf{Aam}\setminus \{D\}\\
    limits(\Box D)=&\;\{D\} \cup \{\texttt{app},\; \texttt{lambda}\; x,\; \texttt{break}\}\\
    limits(\neg \Box D)=&\;\mathbf{Aam}\setminus \{D\}\\
    limits(\phi_1 \land \phi_2)=&\;limits(\phi_1) \cap limits(\phi_2)\\
    limits(\phi_1 \lor \phi_2)=&\;limits(\phi_1) \cup limits(\phi_2)
\end{align*}
\end{definition}


\theoremstyle{definition}
\begin{definition}{(Safe movement)}
We define a function $follows: \mathbf{Aam} \times \mathbf{Pth} \rightarrow \mathcal{P}\left(\mathbf{Pth} \times \left(\mathbf{ACtx} \times \mathbf{ATyp}\right)\right)$ from the set of pairs of AST node modifiers and paths to the power set of editor context bindings.
\begin{align*}
    \textit{follows}(\texttt{var}\; x,\; p)=&\; \emptyset\\
    \textit{follows}(\texttt{const}\; c,\; p)=&\; \emptyset\\
    \textit{follows}(\texttt{app},\; p)=&\; \{p\; \texttt{one} : (\emptyset,\; \texttt{indet}),\; p\; \texttt{two} : (\emptyset,\; \texttt{indet})\}\\
    \textit{follows}(\texttt{lambda}\; x,\; p)=&\; \{p\; \texttt{one} : (\emptyset,\; \texttt{indet})\}\\
    \textit{follows}(\texttt{break},\; p)=&\; \{p\; \texttt{one} : (\emptyset,\; \texttt{indet})\}\\
    \textit{follows}(\texttt{hole},\; p)=&\; \emptyset
\end{align*}
\end{definition}

%
%
We now introduce the type rules for editor expressions. Type rules for substitution are shown in table \ref{tab:typerulesv2sub} and the remaining rules are shown in table \ref{tab:typerulesv2}. The \texttt{child} n prefix is handled by \runa{t-child-1} and \runa{t-child-2}. Here we check that the cursor movement is viable by looking up the new path in $\Gamma_e$. Notice that the remaining editor expression $E$, is evaluated using the new path. The \texttt{parent} prefix is handled similarly in \runa{t-parent} with the exception being that we deconstruct the path instead of building it. When using recursion we require that the depth of the cursor is unchanged after evaluating the expression. We ensure this in \runa{t-rec} with the side condition $depth(E) = 0$. Similarly, \runa{t-cond} utilizes the same side condition to ensure that the cursor is unaffected by whether the condition holds or not. Notice here that evaluation of the conditioned expression is limited by what can follow the condition if it holds, denoted by $\delta$. Sequential composition is handled by the type rule \runa{t-seq}. Here we split the type context into $\Gamma_{e1}$, which contains information about the current subtree, and $\Gamma{e2}$, which contains information about the rest of the tree. This split ensures that the potentially hazardous evaluation of $E_1$ is kept separate from the evaluation of $E_2$.\\

\begin{table*}[htp]
    \centering
    \begin{align*}
        %
        \runa{t-eval} &\; \infrule{p,\; \Gamma_e \vdash E : ok}{p,\; \Gamma_e \vdash \texttt{eval}.E : ok}\\
        %
        \runa{t-child-1}&\; \infrule{\Gamma_e(p\; \texttt{one}) \neq \text{undef} \quad p\; \texttt{one},\; \Gamma_e \vdash E : ok}{p,\; \Gamma_e \vdash \left(\texttt{child}\; 1\right).E : ok}\\
        %
        \runa{t-child-2}&\; \infrule{\Gamma_e(p\; \texttt{two}) \neq \text{undef} \quad p\; \texttt{one},\; \Gamma_e \vdash E : ok}{p,\; \Gamma_e \vdash \left(\texttt{child}\; 2\right).E : ok}\\
        %
        \runa{t-parent}&\; \infrule{\Gamma_e(p) \neq \text{undef} \quad p,\; \Gamma_e \vdash E : ok}{p\; T,\; \Gamma_e \vdash \texttt{parent}.E : ok}\\
        %
        \runa{t-rec} &\; \condinfrule{p,\; \Gamma_e \vdash E : ok}{p,\; \Gamma_e \vdash \texttt{rec} x.E : ok}{\text{if}\; depth(E) = 0}\\
        %
        \runa{t-cond} &\; \condinfrule{p,\; \Gamma_e + \delta \vdash E : ok}{p,\; \Gamma_e \vdash \phi \Rightarrow E : ok}{\begin{align*}
            \text{if}\; &depth(E) = 0\;\\
            \text{and}\; &\delta = \bigcap_{D \in limits(\phi)}follows(D,\; p)\\
        \end{align*}}\\
        %
        \runa{t-seq} &\; \condinfrule{p,\; {\Gamma_e}_1 \vdash E_1 : ok \quad p,\; {\Gamma_e}_2 \vdash E_2 : ok}{p,\; \Gamma_e \vdash E_1 \ggg E_2 : ok}{\text{where}\; \Gamma_e = p\; ({\Gamma_e}_1\; \circ\; {\Gamma_e}_2)}\\
        %
        \runa{t-ref} &\; \infrule{}{p,\;\Gamma_e \vdash x : ok}\\
        %
        \runa{t-nil} &\; \infrule{}{p,\;\Gamma_e \vdash \mathbf{0} : ok}
    \end{align*}
    \caption{Type rules for editor expressions.}
    \label{tab:typerulesv2}
\end{table*}
%
%
Table \ref{tab:typerulesv2sub} shows the type rules for substitution. For substitution to be well-typed, the AST node type $\tau$ in the type context binding associated with the current path $p$ must be consistent with the type of the AST node modifier to be inserted. In \runa{t-sub-var}, we handle the special case where we insert a variable reference $x$. For this to be well-typed, a binding $\Gamma_a(x)=\tau'$ must exist, such that $\consistent{\tau}{\tau'}$. Note that substitution replaces a subtree of the AST. Thus, the bindings in the editor type context with paths starting with $p$ are no longer valid. Therefore, we split the type context on path $p$, such that $\Gamma_e = p\left({\Gamma_e}_1\;\circ\;{\Gamma_e}_2\right)$, and evaluate the prefixed expression $E$ in the type context ${\Gamma_e}_2$. That is, the type context containing all bindings of $\Gamma_e$ not starting with $p$. Note that the binding with path exactly $p$ is in both ${\Gamma_e}_1$ and ${\Gamma_e}_2$, however. We add bindings to ${\Gamma_e}_2$ in rules $\runa{t-sub-app}$ and $\runa{t-sub-abs}$. Particularly, we expand the AST type context upon substitution for an abstraction.\\

We treat substitution of breakpoints differently, as we can either toggle breakpoints on or off. Furthermore, we do not replace the subtree upon substitution for breakpoints. Instead, we must modify the bindings with paths starting with $p$, to either include or remove a $\texttt{one}$. Additionally, we change the type in the binding at the current path $p$ to indicate whether it has a breakpoint. Note that we toggle off the breakpoint if the type is of the form $\breakpoint{\tau}$, and toggle it on otherwise. Thus, the type indicates the structure of the tree.
%
%
\begin{table}
    \begin{flalign*}
        %
        \runa{t-sub-var} &\; \condinfrule{\Gamma_e(p)=(\Gamma_a,\;\tau) \quad \Gamma_a(x) = \tau' \quad \consistent{\tau}{\tau'} \quad p,\;{\Gamma_e}_2 \vdash E : ok}{p,\; \Gamma_e \vdash \replace{\texttt{var}\; x}.E : ok}{\text{where}\; \Gamma_e = p\; ({\Gamma_e}_1\; \circ\; {\Gamma_e}_2)} \\
        %
        \runa{t-sub-const} &\; \condinfrule{\Gamma_e(p)=(\Gamma_a,\;b) \quad p,\;{\Gamma_e}_2 \vdash E : ok}{p,\; \Gamma_e \vdash \replace{\texttt{const}\; c}.E : ok}{\text{where}\; \Gamma_e = p\; ({\Gamma_e}_1\; \circ\; {\Gamma_e}_2)}\\
        %
        \runa{t-sub-app} &\; \condinfrule{\Gamma_e(p)=(\Gamma_a,\; \tau_2') \quad \consistent{\tau_2}{\tau_2'} \quad p,\; \Gamma_e' \vdash E : ok}{p,\; \Gamma_e \vdash \replace{\texttt{app} : \tau_1 \rightarrow \tau_2,\; \tau_1}.E : ok}{\begin{align*}
            &\text{where}\; \Gamma_e = p\; ({\Gamma_e}_1\; \circ\; {\Gamma_e}_2)\;\\
            &\text{and}\; \Gamma_e' = {\Gamma_e}_2,\; p\; \texttt{one} : (\Gamma_a,\; \tau_1 \rightarrow \tau_2),\; p\; \texttt{two} : (\Gamma_a,\; \tau_1)
        \end{align*}}\\
        %
        \runa{t-sub-abs} &\; \condinfrule{\Gamma_e(p)=(\Gamma_a,\; \tau_1' \rightarrow \tau_2') \quad \consistent{\tau_1 \rightarrow \tau_2}{\tau_1' \rightarrow \tau_2'} \quad p,\; \Gamma_e' \vdash E : ok}{p,\; \Gamma_e \vdash \replace{\texttt{lambda}\; x : \tau_1 \rightarrow \tau_2}.E : ok}{\begin{align*}
        &\text{where}\;\Gamma_e = p\; ({\Gamma_e}_1\; \circ\; {\Gamma_e}_2)\\
        &\text{and}\;\Gamma_e' = {\Gamma_e}_2, p\; \texttt{one} : ((\Gamma_a,\; x : \tau_1),\; \tau_2)\end{align*}} \\
        %
        %\runa{t-sub} &\; \infrule{match(D,\; \Gamma_a,\; \tau) = tt \quad p,\;\Gamma_e' \vdash %E : ok}{p,\;\Gamma_e \vdash \replace{D}.E : ok} \\
        %&\text{if}\; D \neq \texttt{break}\\
        %&\text{and}\; \Gamma_e(p)=(\Gamma_a,\;\tau) \\
        %&\text{and}\; \Gamma_e = p\; ({\Gamma_e}_1\; \circ\; {\Gamma_e}_2)\\
        %&\text{and}\; \Gamma_e' = {\Gamma_e}_2 + context(D,\; \Gamma_a)\\
        %
        \runa{t-sub-break-1} &\; \infrule{\Gamma_e(p)=(\Gamma_a,\; \breakpoint{\tau}) \quad p,\; \Gamma_e' \vdash E : ok}{p,\; \Gamma_e \vdash \replace{\texttt{break}} : ok} \\
        &\text{where}\; \Gamma_e = p\; ({\Gamma_e}_1\; \circ\; {\Gamma_e}_2)\\
        &\text{and}\; {\Gamma_e}_1 = \emptyset,\; p\; \texttt{one}\; T_1..T_{n_1} : ({\Gamma_a}_1,\; \tau_1),..,p\; \texttt{one}\; T_1..T_{n_m} : ({\Gamma_a}_m,\; \tau_m)\\
        &\text{and}\; {\Gamma_e}_1' =\emptyset,\; p\; T_1..T_{n_1} : ({\Gamma_a}_1,\; \tau_1),..,p\; T_1..T_{n_m} : ({\Gamma_a}_m,\; \tau_m)\\
        &\text{and}\; \Gamma_e' = \left({\Gamma_e}_2 + {\Gamma_e}_1'\right),\; p : (\Gamma_a,\; \tau)\\
        %
        \runa{t-sub-break-2} &\; \infrule{\Gamma_e(p)=(\Gamma_a,\;\tau)\quad  p,\; \Gamma_e' \vdash E : ok}{p,\; \Gamma_e \vdash \replace{\texttt{break}} : ok} \\
        &\text{where}\; \Gamma_e = p\; ({\Gamma_e}_1\; \circ\; {\Gamma_e}_2)\\
        &\text{and}\; {\Gamma_e}_1 =\emptyset,\; p\; T_1..T_{n_1} : ({\Gamma_a}_1,\; \tau_1),..,p\; T_1..T_{n_m} : ({\Gamma_a}_m,\; \tau_m)\\
        &\text{and}\; {\Gamma_e}_1' = \emptyset,\; p\; \texttt{one}\; T_1..T_{n_1} : ({\Gamma_a}_1,\; \tau_1),..,p\; \texttt{one}\; T_1..T_{n_m} : ({\Gamma_a}_m,\; \tau_m)\\
        &\text{and}\; \Gamma_e' = \left({\Gamma_e}_2 + {\Gamma_e}_1'\right),\; p : (\Gamma_a,\; \breakpoint{\tau})\\
        %
        \runa{t-sub-hole} &\; \condinfrule{\Gamma_e(p)=(\Gamma_a,\;\tau') \quad \consistent{\tau}{\tau'} \quad p,\;{\Gamma_e}_2 \vdash E : ok}{p,\; \Gamma_e \vdash \replace{\texttt{hole} : \tau}.E : ok}{\text{where}\; \Gamma_e = p\; ({\Gamma_e}_1\; \circ\; {\Gamma_e}_2)}
        %
    \end{flalign*}
    \caption{Type rules for substitution.}
    \label{tab:typerulesv2sub}
\end{table}

%\begin{table*}[htp]
%    \centering
%    \begin{align*}
        %%
        %\runa{t-eval} &\; \infrule{p,\; \Gamma_e \vdash E : ok \dashv p',\; \Gamma_e'}{p,\; \Gamma_e \vdash \texttt{eval}.E : %ok \dashv p',\; \Gamma_e'}\\
        %%
        %\runa{t-sub} &\; \infrule{T=\tau \quad p,\;\Gamma_e'' \vdash E : ok \dashv p',\;\Gamma_e'}{p,\;\Gamma_e \vdash %\replace{D}.E : ok \dashv p',\;\Gamma_e'} \\
        %&\text{where}\; \Gamma_e(p)=(\Gamma_a,\;\tau) \\
        %&\text{and}\; T = type(D,\;\Gamma_a) \\
        %&\text{and}\; \Gamma_e = p\; ({\Gamma_e}_1\; \circ\; {\Gamma_e}_2)\\
        %&\text{and}\; \Gamma_e'' = {\Gamma_e}_1 + context(D,\; \Gamma_a)\\
        %%
        %\runa{t-child-1}&\; \infrule{\Gamma_e(p\; \texttt{one}) \neq undef \quad p,\; \texttt{one},\; \Gamma_e \vdash E : ok %\dashv p',\; \Gamma_e'}{p,\; \Gamma_e \vdash \left(\texttt{child}\; 1\right).E : ok \dashv p',\; \Gamma_e'}\\
        %%
        %\runa{t-child-2}&\; \infrule{\Gamma_e(p\; \texttt{two}) \neq undef \quad p,\; \texttt{one},\; \Gamma_e \vdash E : ok %\dashv p',\; \Gamma_e'}{p,\; \Gamma_e \vdash \left(\texttt{child}\; 2\right).E : ok \dashv p',\; \Gamma_e'}\\
        %%
        %\runa{t-parent}&\; \infrule{\Gamma_e(p) \neq undef \quad p,\; \Gamma_e \vdash E : ok \dashv p',\; \Gamma_e'}{p\; T,\; %\Gamma_e \vdash \texttt{parent}.E : ok \dashv p',\; \Gamma_e'}\\
        %%
        %\runa{t-rec} &\; \condinfrule{p,\; {\Gamma_e}_1 \vdash E : ok \dashv p,\; \Gamma_e'}{p,\; \Gamma_e \vdash \texttt{rec} %x.E : ok \dashv p,\; {\Gamma_e}_2}{\text{where}\; \Gamma_e = p\; ({\Gamma_e}_1\; \circ\; {\Gamma_e}_2)}\\
        %%
        %\runa{t-seq} &\; \infrule{p,\; \Gamma_e \vdash E_1 : ok \dashv p'',\; \Gamma_e'' \quad p'',\; \Gamma_e'' \vdash E_2 : %ok \dashv p',\; \Gamma_e'}{p,\; \Gamma_e \vdash E_1 \ggg E_2 : ok \dashv p',\; \Gamma_e'}\\
        %%
        %\runa{t-cond} &\; \infrule{p,\; {\Gamma_e}_1 + \delta \vdash E : ok \dashv p,\; \Gamma_e'}{p,\; \Gamma_e \vdash \phi %\Rightarrow E : ok \dashv p,\; {\Gamma_e}_2}\\
%        &\text{where}\; \Gamma_e = p\; ({\Gamma_e}_1\; \circ\; {\Gamma_e}_2)\\
%        &\text{and}\; \delta = \bigcap_{D \in limits(\phi)}follows(D)\\
%        %
%        \runa{t-ref} &\; \infrule{}{p,\;\Gamma_e \vdash x : ok \dashv p,\;\Gamma_e}\\
%        %
%        \runa{t-nil} &\; \infrule{}{p,\;\Gamma_e \vdash \mathbf{0} : ok \dashv p,\;\Gamma_e}\\
%    \end{align*}
%    \caption{Type rules for editor expressions.}
%    \label{tab:typerules}
%\end{table*}

\begin{theorem} (Subject reduction)
If $\Gamma_e, \;\Gamma_a \vdash \conf{E,\;a} : ok$ and $\conf{E, a} \xrightarrow{\alpha} \conf{E', a'}$ then $\Gamma_e, \;\Gamma_a \vdash \conf{E',\;a'} : ok$.
\end{theorem}

We define \textit{well-typedness} of a configuration $\conf{E,\;a}$ by the following rule: \\
$\condinfrule{\Gamma_a \vdash a : \tau \quad p,\; \Gamma_e \vdash E : ok}{\Gamma_e, \;\Gamma_a \vdash \conf{E,\;a} : ok}{\begin{align*}
        &\text{where}\;\\
        &\text{and}\;\end{align*}}$
        
        

\section{Conclusion}\label{ch:conclusion}
In this paper, we have explored the challenges of implementing both a type checker and type inference for the type system for parallel complexity of $\pi$-calculus processes by Baillot and Ghyselen \cite{BaillotGhyselen2021}. In this chapter, we first present and discuss our results, after which we discuss limitations of our implementations. Finally, we consider future work, and discuss how some of the limitations may be relaxed.
%
\subsection{Results}
Type checking and type inference are limited by our ability to respectively verify or satisfy constraint judgements. As such judgements are universally quantified over index variables, this becomes non-trivial. We can reduce verification of constraint judgements to linear programming, where a constraint judgement holds if there is no solution to the corresponding linear program. We have also shown that certain polynomial constraint judgements can be reduced to linear constraint judgements, enabling type checking of some processes with polynomial time behavior. We have introduced algorithmic type rules for type checking, and proved their soundness with respect to parallel complexity.\\ %To ensure we maintain the subject reduction property, we introduced combined complexities and the associated function \textit{basis}, such that we effectively defer checks of constraint judgements introduced by parallel composition until a later time. The type checker has been proved sound.\\

Our type inference algorithm performs multiple passes over a program to first infer simple types, which are then used to infer constraints on the variance of input/output types, sizes of naturals, bounds on channel synchronization and complexity bounds. We can reduce such constraints to a set of constraint judgements with existentially quantified variables representing coefficients that when solved provide a bound on the span. These constraints are significantly more difficult to solve than those that emerge for type checking, and so we over-approximate them using naive quantifier elimination. We provide a Haskell implementation based on the Z3 SMT solver.\\

%We have introduced constraint based type inference to the type system by Baillot and Ghyselen, which is inspired by the work of Kobayashi et al. \cite{KobayashiEtAl2000}. To infer types, we do multiple passes over the program to infer different kinds of constraints, whose solution corresponds to a valid typing of the program. To solve the constraints, we reduce them into simple constraints on coefficients that are solved by an off-the-shelf SMT solver. However, we found that by naively generating constraints, we get constraints consisting of nested existential and universal quantifiers that are very difficult to solve even for state-of-the-art SMT solvers. So, to solve these constraints we make a number of over-approximations during reduction of constraints. We implement type inference in Haskell using the Z3 SMT solver to solve constraints.\\

Overall, we find that we can type check many constant and linear time processes and some polynomial time processes. Similarly, we can infer precise bounds on many constant time and some linear time processes in reasonable time. However, both type checking and type inference are limited not only by the expressiveness of complexity bounds, but also by size bounds on naturals, as some encoded algorithms may return naturals with size bounds that exceed their complexity bounds. As our type inference algorithm infers polynomial constraints, we are also limited by the ability to find solutions, which we find to be difficult for many simple processes.

\subsection{Discussion}

During type checking, we limit ourselves to linear indices such that we can reduce constraint judgements to a linear program that can be solved by an algorithm such as the simplex algorithm. We have primarily made this choice for the sake of simplicity, however, tools such as \textit{Gröbner bases} exist that can be used to help solving systems of polynomial equations. Even more generally, existing provers such as the Z3 prover, which utilizes, among other things, Gröbner bases, could be used to solve both linear and some super-linear systems.\\

As for type inference, we are able to infer precise bounds on some constant and linear time servers. We notice that inferred constraint satisfaction problems for servers quickly grow in difficulty based on the number of index variables. Even simple processes with no ticks take significantly longer time to solve, when an extra index variable is introduced. We ascribe this to the polynomial constraints introduced upon instantiating servers, and so it may make sense to consider further over-approximations of substitutions.\\

An interesting observation is that our restriction to linear indices makes us unable to infer bounds on some linear time processes. One such example is that of the Fibonacci number encoding, where we are unable to express size bounds on the \textit{returned} Fibonacci number, which grows according to the Fibonacci sequence, yet we can express the complexity of the server. Although the complexity is not directly affected by the size of this natural, we are thus unable to infer a sized type for the server. As the complexity of a process does not necessarily depend on all size bounds, it may be possible to let some of them be unknown, and thereby relax the restriction.\\

We also notice the importance of antecedents in constraint judgements. When all antecedents are discarded, we are unable to infer bounds on any linear time server. Moreover, we observe that coefficient variables do not take negative valuations unless we simulate antecedents. We believe this is due to our over-approximations of index inequality constraints. That is, we implicitly infer the constraint $\varphi;\Phi\vDash 0 \leq I$ for all indices $I$, which we over-approximate using coefficient-wise inequality constraints, which means all coefficients must be non-negative. This seems to translate to all coefficient variables being non-negative as well. However, when antecedents are simulated, we may substitute a positive term into an index, providing more flexibility to coefficient variable valuations. We believe this is why our simulation of inequality antecedents greatly increases expressiveness of our type inference algorithm.

% type check
%   better ways to reduce (more) polynomial constraints to linear: Hoffmann and Hofmann
%   Gröbner basis for solving polynomial constraint judgements?
% type inference
%   what can we infer bounds on: Constant time processes, some linear time processes; Processes with different span and work!
%   what can we not infer bounds on: fib; not only a question of complexity but also size of outputs (non-linear); What can we do about it? - better use of index variables ..


%   The importance of antecedents
%   Consequences of our over-approximations:  0 <= I means all coefficients in I are non-negative!

\subsection{Future work}
We have introduced algorithmic type rules for the type system by Baillot and Ghyselen \cite{BaillotGhyselen2021}, enabling us to implement a type checker that given some type environment and process, can verify if process is well-typed under the environment, and thereby provide us a bound on the parallel complexity. As such, a natural next step is to implement the type checker in for instance Haskell.\\

We have primarily limited ourselves to linear indices, and thereby linear complexity bounds, as the corresponding constraint judgements are then simpler to verify or satisfy. It may be worthwhile to explore how this restriction may be relaxed. Hofmann and Hoffmann \cite{HofmannAndHoffmann2010} and Hoffmann et al. \cite{HoffmannEtAl2012}, show how linear constraints can be derived from polynomial bounds, by representing polynomials as sums of binomial coefficients. However, this is in the setting of first-order functional programs, and so it may be interesting to see if this method can be applied to message-passing processes.\\

Our type inference algorithm for parallel complexity of $\pi$-calculus processes provides correct bounds on the parallel complexity of the processes we have tried (when a bound can be inferred in reasonable time). However, we have yet to formally prove its correctness. In particular, we are interested in proving that our algorithm always infers principal typings. We expect this property to be quite straightforward to prove, as our inference rules are based on the type rules. One exception to this is our treatment of time invariance.\\

% Better support for lower-bounds: antecedents; I = 0, can be solved in the same way as I >= 1 with proper index variable use !!

In this thesis, we have limited ourselves to the type system for parallel complexity by Baillot and Ghyselen \cite{BaillotGhyselen2021}. However, the usage-based generalization of the type system provided by Baillot et al. \cite{BaillotEtAl2021} increases the expressiveness and precision. Usages allow for more precise description of the behavior of channels, for instance making processes with some forms of indeterministic communication typable. Usages are well-suited for inference, and as this type system shares many similarities with the type system by Baillot and Ghyselen, it may be interesting to see if our type inference algorithm can be extended to usages. Constraint-based inference of usage types have been studied previously, and judging from this work, we expect usage reliability to be the main challenge due to universal quantification over index variables \cite{KobayashiEtAl2000, Kobayashi2005}.

% implementation of type check
% extension of type inference to polynomial bounds alá Hoffmann and Hofmann and Hoffmann et al. (their representation of polynomials)
% prove soundness of type inference
% Extending type inference to usages
% Better support for lower-bounds: antecedents; I = 0, can be solved in the same way as I >= 1 with proper index variable use !!
%%% Local Variables:
%%% mode: latex
%%% TeX-master: "../esop2023"
%%% End:


%
%
\bibliographystyle{compj}
\bibliography{mybib}

\appendix
\chapter{Soundness of sized type implementation}\label{app:sizedtypesoundness}
\setcounter{theorem}{10}


\begin{lemma}[Additive advancement of time]
Let $\Phi$ be a set of constraints with unknowns in $\varphi$ and let $T$ be a type then $\susume{\susume{T}{\varphi}{\Phi}{J}}{\varphi}{\Phi}{I} =\; \susume{T}{\varphi}{\Phi}{I+J}$.
\begin{proof} On the structure of $T$.
    \begin{description}
    \item[$(\susume{\susume{\texttt{Nat}[K,L]}{\varphi}{\Phi}{J}}{\varphi}{\Phi}{I})$] obtained directly from  $\susume{\texttt{Nat}[K,L]}{\varphi}{\Phi}{J} = \texttt{Nat}[K,L]$ and $\susume{\texttt{Nat}[K,L]}{\varphi}{\Phi}{I} = \texttt{Nat}[K,L]$.
    %
    \item[$(\susume{\susume{\texttt{ch}^\sigma_L(\widetilde{T})}{\varphi}{\Phi}{J}}{\varphi}{\Phi}{I})$] We either have that
    \begin{enumerate}
        \item $\varphi;\Phi\vDash J \leq L$ and so we have that $\susume{\texttt{ch}^\sigma_L(\widetilde{T})}{\varphi}{\Phi}{J}=\texttt{ch}^\sigma_{L-J}(\widetilde{T})$. Then if $\varphi;\Phi\vDash I \leq L-J$ we also have $\varphi;\Phi\vDash I+J \leq L$ as $\varphi;\Phi\vDash J \leq L$, and so we obtain $\susume{\susume{\texttt{ch}^\sigma_L(\widetilde{T})}{\varphi}{\Phi}{J}}{\varphi}{\Phi}{I}=\susume{\texttt{ch}^\sigma_L(\widetilde{T})}{\varphi}{\Phi}{I+J}=\texttt{ch}^\sigma_{L-(I+J)}(\widetilde{T})$. Otherwise, we have that $\varphi;\Phi\nvDash I \leq L-J$, implying that $\varphi;\Phi\nvDash I+J \leq L$ as $\varphi;\Phi\vDash J \leq L$, and so we obtain $\susume{\susume{\texttt{ch}^\sigma_L(\widetilde{T})}{\varphi}{\Phi}{J}}{\varphi}{\Phi}{I}=\susume{\texttt{ch}^\sigma_L(\widetilde{T})}{\varphi}{\Phi}{I+J}=\texttt{ch}^\emptyset_{L-(I+J)}(\widetilde{T})$.
        %
        \item $\varphi;\Phi\nvDash J \leq L$ and so we have that $\susume{\texttt{ch}^\sigma_L(\widetilde{T})}{\varphi}{\Phi}{J}=\texttt{ch}^\emptyset_{L-J}(\widetilde{T})$ and $\susume{\texttt{ch}^\emptyset_{L-J}(\widetilde{T})}{\varphi}{\Phi}{I}=\texttt{ch}^\emptyset_{(L-J)-I}(\widetilde{T})$. It follows from the fact that $I$ is non-negative that also $\varphi;\Phi\nvDash I+J \leq L$ and so we obtain $\susume{\susume{\texttt{ch}^\sigma_L(\widetilde{T})}{\varphi}{\Phi}{J}}{\varphi}{\Phi}{I}=\texttt{ch}^\emptyset_{L-(J+I)}(\widetilde{T})=\texttt{ch}^\emptyset_{(L-J)-I}(\widetilde{T})$.
    \end{enumerate}
    %
    \item[$(\susume{\susume{\forall_L\widetilde{i}.\texttt{serv}^\sigma_K(\widetilde{T})}{\varphi}{\Phi}{J}}{\varphi}{\Phi}{I})$] We either have that
    \begin{enumerate}
        \item $\varphi;\Phi\vDash J \leq L$ and so we have that $\susume{\forall_L\widetilde{i}.\texttt{serv}^\sigma_K(\widetilde{T})}{\varphi}{\Phi}{J}=\forall_{L-J}\widetilde{i}.\texttt{serv}^\sigma_K(\widetilde{T})$. Then if $\varphi;\Phi\vDash I \leq L-J$ we also have $\varphi;\Phi\vDash I+J \leq L$ as $\varphi;\Phi\vDash J \leq L$, and so we obtain $\susume{\susume{\forall_L\widetilde{i}.\texttt{serv}^\sigma_K(\widetilde{T})}{\varphi}{\Phi}{J}}{\varphi}{\Phi}{I}=\susume{\forall_L\widetilde{i}.\texttt{serv}^\sigma_K(\widetilde{T})}{\varphi}{\Phi}{I+J}=\forall_{L-(I+J)}\widetilde{i}.\texttt{serv}^\sigma_K(\widetilde{T})$. Otherwise, we have that $\varphi;\Phi\nvDash I \leq L-J$, implying that $\varphi;\Phi\nvDash I+J \leq L$ as $\varphi;\Phi\vDash J \leq L$, and so we obtain $\susume{\susume{\forall_L\widetilde{i}.\texttt{serv}^\sigma_K(\widetilde{T})}{\varphi}{\Phi}{J}}{\varphi}{\Phi}{I}=\susume{\forall_L\widetilde{i}.\texttt{serv}^\sigma_K(\widetilde{T})}{\varphi}{\Phi}{I+J}=\forall_{L-(I+J)}\widetilde{i}.\texttt{serv}^{\sigma\cap\{\texttt{out}\}}_K(\widetilde{T})$.
        %
        \item $\varphi;\Phi\nvDash J \leq L$ and so we have that $\susume{\forall_L\widetilde{i}.\texttt{serv}^\sigma_K(\widetilde{T})}{\varphi}{\Phi}{J}=\forall_{L-J}\widetilde{i}.\texttt{serv}^{\sigma\cap\{\texttt{out}\}}_K(\widetilde{T})$ and $\susume{\forall_{L-J}\widetilde{i}.\texttt{serv}^{\sigma\cap\{\texttt{out}\}}_K(\widetilde{T})}{\varphi}{\Phi}{I}=\forall_{(L-J)-I}\widetilde{i}.\texttt{serv}^{\sigma\cap\{\texttt{out}\}}_K(\widetilde{T})$. It follows from the fact that $I$ is non-negative that also $\varphi;\Phi\nvDash I+J \leq L$ and so we obtain $\susume{\susume{\forall_L\widetilde{i}.\texttt{serv}^\sigma_K(\widetilde{T})}{\varphi}{\Phi}{J}}{\varphi}{\Phi}{I}=\forall_{L-(I+J)}\widetilde{i}.\texttt{serv}^{\sigma\cap\{\texttt{out}\}}_K(\widetilde{T})=\forall_{(L-J)-I}\widetilde{i}.\texttt{serv}^{\sigma\cap\{\texttt{out}\}}_K(\widetilde{T})$.
    \end{enumerate}
    \end{description}
\end{proof}
\end{lemma}
%

% \begin{lemma}
% If $\varphi;\Phi;\Gamma;\Delta\vdash e : T$ then $\varphi;\Phi;\Gamma;\cdot\vdash \circledcirc e : S$ with $\varphi;\Phi\vdash S \sqsubseteq T$.
% \begin{proof} By induction on $e$. We only show the interesting cases
%     \begin{description}
%     %
%     \item[$(e_\theta)$] We have two cases
%     \begin{enumerate}
%         \item $\theta = v$ By $\runa{S-avar}$ we have that $\varphi;\Phi;\Gamma;\Delta,v:T\vdash e_v : T$ and $\varphi;\Phi;\Gamma;\Delta,v:T\vdash e : S$ such that $\varphi;\Phi\vdash S \sqsubseteq T$. As $\circledcirc e_v = \circledcirc e$, we obtain by induction that $\varphi;\Phi;\Gamma;\cdot\vdash \circledcirc e : S'$ with $\varphi;\Phi\vdash S' \sqsubseteq S$ and by transitivity we have that $\varphi;\Phi\vdash S'\sqsubseteq T$.
%         %
%         \item $\theta = e'$ By $\runa{S-strength}$ we have that $\varphi;\Phi;\Gamma;\Delta\vdash e_{e'} : \texttt{Nat}[I-1,J-1]$,
%         $\varphi;\Phi;\Gamma;\Delta\vdash e : \texttt{Nat}[I',J']$ and $\varphi;\Phi;\Gamma;\Delta\vdash e' : \texttt{Nat}[I,J]$ such that $\varphi;\Phi\vdash \texttt{Nat}[I',J']\sqsubseteq\texttt{Nat}[I-1,J-1]$. By induction we obtain $\varphi;\Phi;\Gamma;\Delta\vdash \circledcirc e : \texttt{Nat}[I'',J'']$ with $\varphi;\Phi\vdash \texttt{Nat}[I'',J''] \sqsubseteq \texttt{Nat}[I',J']$. By the transitive property of $\leq$ it follows that $\varphi;\Phi\vdash \texttt{Nat}[I'',J''] \sqsubseteq \texttt{Nat}[I-1,J-1]$. 
%     \end{enumerate}
%     %
%     %\item[$(e :: e')$] By $\runa{S-cons}$ we have that $\varphi;\Phi;\Gamma\vdash_\Delta e :: e' : \texttt{List}[I+1,J+1](\mathcal{B}_1 \uplus_{\varphi;\Phi} \mathcal{B}_2)$, $\varphi;\Phi;\Gamma\vdash_\Delta e : \mathcal{B}_1$ and $\varphi;\Phi;\Gamma\vdash_\Delta e' : \texttt{List}[I,J](\mathcal{B}_2)$. As $\circledcirc (e :: e') = (\circledcirc e) :: (\circledcirc e')$, we have by induction and by $\runa{SS-lweak}$ that $\varphi;\Phi;\Gamma\vdash_\Delta \circledcirc e : \mathcal{B}_1'$ and $\varphi;\Phi;\Gamma\vdash_\emptyset \circledcirc e' : \texttt{List}[I',J'](\mathcal{B}_2')$ such that $\varphi;\Phi\vDash I \leq I'$, $\varphi;\Phi\vDash J' \leq J$, $\varphi;\Phi\vdash \mathcal{B}_1' \sqsubseteq \mathcal{B}_1$ and $\varphi;\Phi\vdash \mathcal{B}_2' \sqsubseteq \mathcal{B}_2$. By application of $\runa{S-cons}$ we obtain $\varphi;\Phi;\Gamma\vdash_\emptyset (\circledcirc e) :: (\circledcirc e') : \texttt{List}[I' + 1, J' + 1](\mathcal{B}_1' \uplus_{\varphi;\Phi} \mathcal{B}_2')$. It follows from $\varphi;\Phi\vDash I \leq I'$ and $\varphi;\Phi\vDash J' \leq J$ that also $\varphi;\Phi\vDash I+1 \leq I'+1$ and $\varphi;\Phi\vDash J'+1 \leq J+1$
%     %%and so by $\runa{SS-nweak}$ $\varphi;\Phi\vdash\texttt{Nat}[I'+1,J'+1] \sqsubseteq \texttt{Nat}[I+1,J+1]$.
%     %
%     \item[$(s(e))$] By $\runa{S-succ}$ we have that $\varphi;\Phi;\Gamma;\Delta\vdash s(e) : \texttt{Nat}[I+1,J+1]$ and $\varphi;\Phi;\Gamma;\Delta\vdash e : \texttt{Nat}[I,J]$. As $\circledcirc s(e) = s(\circledcirc e)$, we have by induction and by $\runa{SS-nweak}$ that $\varphi;\Phi;\Gamma\vdash_\emptyset \circledcirc e : \texttt{Nat}[I',J']$ such that $\varphi;\Phi\vDash I \leq I'$ and $\varphi;\Phi\vDash J' \leq J$. By application of $\runa{S-succ}$ we obtain $\varphi;\Phi;\Gamma;\cdot\vdash s(\circledcirc e) : \texttt{Nat}[I' + 1, J' + 1]$. It follows from $\varphi;\Phi\vDash I \leq I'$ and $\varphi;\Phi\vDash J' \leq J$ that also $\varphi;\Phi\vDash I+1 \leq I'+1$ and $\varphi;\Phi\vDash J'+1 \leq J+1$ and so by $\runa{SS-nweak}$ $\varphi;\Phi\vdash\texttt{Nat}[I'+1,J'+1] \sqsubseteq \texttt{Nat}[I+1,J+1]$.
%     %
%     \end{description}
% \end{proof}
% \end{lemma}
\setcounter{theorem}{16}
\begin{lemma}[Substitution]\text{ }
\begin{enumerate}
    \item If $\varphi;\Phi;\Gamma,v:T\vdash e' : S$ and $\varphi;\Phi;\Gamma\vdash e : T$ then $\varphi;\Phi;\Gamma\vdash e'[v\mapsto e] : S$.
    \item If $\varphi;\Phi;\Gamma,v:T\vdash P \triangleleft \kappa$ and $\varphi;\Phi;\Gamma\vdash e : T$ then $\varphi;\Phi;\Gamma\vdash P[v\mapsto e] \triangleleft \kappa$.
\end{enumerate}
\begin{proof} The first point is proved by induction on the type rules of expressions, and the second by induction on the type rules for processes. We consider them separately
\begin{enumerate}
    \item 
\begin{description}
%
\item[$\runa{S-zero}$] We have that $\varphi;\Phi;\Gamma,v:T\vdash 0 : \texttt{Nat}[0,0]$. We obtain $\varphi;\Phi;\Gamma\vdash 0[v\mapsto e] : \texttt{Nat}[0,0]$ directly from $0[v\mapsto e] = 0$ and $\varphi;\Phi;\Gamma\vdash 0 : \texttt{Nat}[0,0]$.
%
\item[$\runa{S-succ}$] We have that $\varphi;\Phi;\Gamma,v:T\vdash e' : \texttt{Nat}[I,J]$, $\varphi;\Phi;\Gamma,v:T\vdash s(e') : \texttt{Nat}[I+1,J+1]$ and $\varphi;\Phi;\Gamma\vdash e : T$. By induction we obtain $\varphi;\Phi;\Gamma\vdash e'[v\mapsto e] : \texttt{Nat}[I,J]$, and so by application of $\runa{S-succ}$ we derive $\varphi;\Phi;\Gamma\vdash s(e'[v\mapsto e]) : \texttt{Nat}[I+1,J+1]$.
%
\item[$\runa{S-var}$] We have two cases. Either we have that $\varphi;\Phi;\Gamma,v:T\vdash v : T$ and we substitute $e$ for $v$, or we have that $\varphi;\Phi;\Gamma,v:T,w:S\vdash v : T$. The first case is obtained directly from the assumption that $\varphi;\Phi;\Gamma\vdash e : T$. The second case is obtained directly from $v[w\mapsto e] = v$ when $v\neq w$ and $\varphi;\Phi;\Gamma,v:T\vdash v : T$ by $\runa{S-var}$.
%
\item[$\runa{S-subsumption}$] We have that $\varphi;\Phi;\Gamma,v:T\vdash e' : S'$ and $\varphi;\Phi\vdash S' \sqsubseteq S$ such that $\varphi;\Phi;\Gamma,v:T\vdash e' : S$. By the assumption we have that $\varphi;\Phi;\Gamma\vdash e : T$, and so by induction we obtain $\varphi;\Phi;\Gamma\vdash e'[v\mapsto e] : S'$, and so by application of $\runa{S-subsumption}$, we derive $\varphi;\Phi;\Gamma\vdash e'[v\mapsto e] : S$.
%
% \item[$\runa{S-avar}$] We have that $\varphi;\Phi;\Gamma,v:T;\Delta,w:S'\vdash e' : S$, $\varphi;\Phi\vdash S \sqsubseteq S'$, $\varphi;\Phi;\Gamma,v:T;\Delta\vdash {e'}_w : S'$ and $\varphi;\Phi;\Gamma;\Delta,w:S'\vdash e : T$. By induction we obtain $\varphi;\Phi;\Gamma;\Delta,w:S'\vdash e'[v \mapsto T] : S$, and by application of $\runa{S-avar}$ we derive $\varphi;\Phi;\Gamma,v:T;\Delta\vdash {e'}_w[v\mapsto T] : S'$. 
% %
% \item[$\runa{S-strength}$] We have that $\varphi;\Phi;\Gamma,v:T;\Delta\vdash e' : \texttt{Nat}[I',J']$, $\varphi;\Phi;\Gamma,v:T;\Delta\vdash e'' : \texttt{Nat}[I,J]$, $\varphi;\Phi\vdash \texttt{Nat}[I',J'] \sqsubseteq \texttt{Nat}[I-1,J-1]$, $\varphi;\Phi;\Gamma,v:T;\Delta\vdash {e'}_{e_''} : \texttt{Nat}[I-1,J-1]$ and $\varphi;\Phi;\Gamma;\Delta\vdash e : T$. By induction we obtain $\varphi;\Phi;\Gamma,v:T;\Delta\vdash e'[v\mapsto e] : \texttt{Nat}[I',J']$ and $\varphi;\Phi;\Gamma,v:T;\Delta\vdash e''[v\mapsto e] : \texttt{Nat}[I,J]$. By application of $\runa{S-strength}$ we then obtain $\varphi;\Phi;\Gamma;\Delta\vdash {e'}_{e_''}[v\mapsto e] : \texttt{Nat}[I-1,J-1]$.
%
\end{description}
    %
    \item 
\begin{description}
%
\item[$\runa{S-nil}$] We have that $\varphi;\Phi;\Gamma,v:T\vdash \nil \triangleleft \{0\}$. We obtain $\varphi;\Phi;\Gamma\vdash \nil[v\mapsto e] \triangleleft \{0\}$ directly from $\nil[v\mapsto e] = \nil$ and $\varphi;\Phi;\Gamma\vdash \nil \triangleleft \{0\}$.
%
\item[$\runa{S-tick}$] We have that $\varphi;\Phi;\downarrow_1\!\!(\Gamma,v:T)\vdash P \triangleleft \kappa$ and $\varphi;\Phi;\Gamma,v:T\vdash \tick{P} \triangleleft \kappa + 1$. By Lemma \ref{lemma:susumedefer}, we have that $\varphi;\Phi;\downarrow_1\!\!\Gamma\vdash e :\; \susume{T}{\varphi}{\Phi}{1}$, and so by induction we obtain $\varphi;\Phi;\downarrow_1\!\!\Gamma\vdash P[v\mapsto e] \triangleleft \kappa$. By application of $\runa{S-tick}$ we then derive $\varphi;\Phi;\Gamma\vdash \tick{P[v\mapsto e]} \triangleleft \kappa + 1$.
%
\item[$\runa{S-match}$] We have that $\varphi;\Phi;\Gamma,v:T\vdash e' : \texttt{Nat}[I,J]$, $\varphi;(\Phi,I\leq 0);\Gamma,v:T\vdash P \triangleleft \kappa$, $\varphi;(\Phi,J\geq 1);\Gamma,v:T,x:\texttt{Nat}[I-1,J-1]\vdash Q \triangleleft \kappa'$, $\varphi;\Phi;\Gamma,v:T\vdash \match{e}{P}{x}{Q} \triangleleft \text{basis}(\varphi,\Phi,\kappa\cup\kappa')$ and $\varphi;\Phi;\Gamma\vdash e : T$. From point 1 we obtain $\varphi;\Phi;\Gamma\vdash e'[v\mapsto e] : \texttt{Nat}[I,J]$ and by weakening (Lemma \ref{lemma:weakening}) and induction we derive $\varphi;(\Phi,I\leq 0);\Gamma\vdash P[v\mapsto e] \triangleleft \kappa$ and $\varphi;(\Phi,J\geq 1);\Gamma,x:\texttt{Nat}[I-1,J-1]\vdash Q[v\mapsto e] \triangleleft \kappa'$. Thus, by application of $\runa{S-match}$, we obtain $\varphi;\Phi;\Gamma\vdash \match{e}{P}{x}{Q}[v\mapsto e] \triangleleft \text{basis}(\varphi,\Phi,\kappa\cup\kappa')$. 
%
\item[$\runa{S-par}$] We have that $\varphi;\Phi;\Gamma,v:T\vdash P \triangleleft \kappa$, $\varphi;\Phi;\Gamma,v:T\vdash Q \triangleleft \kappa'$, $\varphi;\Phi;\Gamma,v:T\vdash P \mid Q \triangleleft \text{basis}(\varphi,\Phi,\kappa\cup\kappa')$ and $\varphi;\Phi;\Gamma\vdash e : T$. By induction we obtain $\varphi;\Phi;\Gamma\vdash P[v\mapsto e] \triangleleft \kappa$ and $\varphi;\Phi;\Gamma\vdash Q[v\mapsto e] \triangleleft \kappa'$. Thus, by application of $\runa{S-par}$, we derive $\varphi;\Phi;\Gamma\vdash (P \mid Q)[v\mapsto e] \triangleleft \text{basis}(\varphi,\Phi,\kappa\cup\kappa')$.
%
\item[$\runa{S-nu}$] We have that $\varphi;\Phi;\Gamma,v:T,a:S;\Delta\vdash P \triangleleft \kappa$, $\varphi;\Phi;\Gamma,v:T\vdash \newvar{a}{P} \triangleleft \kappa$ and $\varphi;\Phi;\Gamma\vdash e : T$. By weakening (Lemma \ref{lemma:weakening}) we obtain $\varphi;\Phi;\Gamma,a:S\vdash e : T$, and so by induction we have that $\varphi;\Phi;\Gamma,a:S\vdash P[v\mapsto e] \triangleleft \kappa$. Thus, by application of $\runa{S-nu}$ we derive $\varphi;\Phi;\Gamma\vdash (\newvar{a}{P})[v\mapsto e] \triangleleft \kappa$.
%
\item[$\runa{S-iserv}$] We have that $\varphi;\Phi;\Gamma,w:S\vdash a : \forall_0\widetilde{i}.\texttt{serv}^\sigma_K(\widetilde{T})$, $(\varphi,\widetilde{i});\Phi;\text{ready}(\varphi,\Phi,\downarrow_I\!\!(\Gamma,w:S)),\widetilde{v}:\widetilde{T}\vdash P \triangleleft \kappa$, $\varphi;\Phi;\Gamma,w:S\vdash\; !\inputch{a}{\widetilde{v}}{}{P} \triangleleft \{I\}$ and $\varphi;\Phi;\Gamma\vdash e : S$. By Lemma \ref{lemma:susumedefer} this implies $\varphi;\Phi;\text{ready}(\varphi,\Phi,\downarrow_I\!\!\Gamma)\vdash e : \text{ready}(\varphi,\Phi,\downarrow_I\!\!S)$, and from point $1$ we obtain $\varphi;\Phi;\Gamma\vdash a[w\mapsto e] : \forall_0\widetilde{i}.\texttt{serv}^\sigma_K(\widetilde{T})$. By weakening (Lemma \ref{lemma:weakening}) we then derive $\varphi;\Phi;\text{ready}(\varphi,\Phi,\downarrow_I\!\!\Gamma),\widetilde{v}:\widetilde{T}\vdash e : \text{ready}(\varphi,\Phi,\downarrow_I\!\!S)$, and so by induction we obtain $(\varphi,\widetilde{i});\Phi;\text{ready}(\varphi,\Phi,\downarrow_I\!\!\Gamma),\widetilde{v}:\widetilde{T}\vdash P[w\mapsto e] \triangleleft \kappa$. Finally, by application of $\runa{S-iserv}$, we derive $\varphi;\Phi;\Gamma\vdash\; !\inputch{a}{\widetilde{v}}{}{P}[w\mapsto e] \triangleleft \{I\}$.
%
\item[$\runa{S-ich}$] We have that $\varphi;\Phi;\Gamma,v:T\vdash a : \texttt{ch}^\sigma_I(\widetilde{S})$, $\varphi;\Phi;\downarrow_I\;\;(\Gamma,v:T),\widetilde{w}:\widetilde{S}\vdash P \triangleleft \kappa$, $\varphi;\Phi;\Gamma,v:T\vdash \inputch{a}{\widetilde{w}}{}{P} \triangleleft \kappa + I$ and $\varphi;\Phi;\Gamma\vdash e : T$. From point $1$ we obtain $\varphi;\Phi;\Gamma\vdash a[v\mapsto e] : \texttt{ch}^\sigma_I(\widetilde{S})$ (Note that it may be that $v=a$). By Lemma \ref{lemma:susumedefer}, we have that $\varphi;\Phi;\downarrow_I\!\!\Gamma\vdash e :\; \susume{T}{\varphi}{\Phi}{I}$, and so by weakening (Lemma \ref{lemma:weakening}) and induction we derive $\varphi;\Phi;\downarrow_I\;\;\Gamma,\widetilde{w}:\widetilde{S}\vdash P[v\mapsto e] \triangleleft \kappa$. Thus, by application of $\runa{S-ich}$ we obtain $\varphi;\Phi;\Gamma\vdash (\inputch{a}{\widetilde{w}}{}{P})[v\mapsto e] \triangleleft \kappa + I$. 
%
\item[$\runa{S-och}$] We have that $\varphi;\Phi;\Gamma,v:T\vdash a : \texttt{ch}^\sigma_I(\widetilde{S})$, $\varphi;\Phi;\downarrow_I\!\!(\Gamma,v:T)\vdash \widetilde{e}' : \widetilde{S}'$, $\varphi;\Phi;\Gamma,v:T\vdash \asyncoutputch{a}{\widetilde{e}'}{} \triangleleft \{I\}$ and $\varphi;\Phi;\Gamma\vdash e : T$. By Lemma \ref{lemma:susumedefer}, we have that $\varphi;\Phi;\downarrow_I\!\!\Gamma\vdash e :\; \susume{T}{\varphi}{\Phi}{I}$, and so from point $1$ we obtain $\varphi;\Phi;\downarrow_I\!\!\Gamma\vdash \widetilde{e}'[v\mapsto e] : \widetilde{S}'$ and $\varphi;\Phi;\Gamma\vdash a[v\mapsto e] : \texttt{ch}^\sigma_I(\widetilde{S})$. By application of $\runa{S-och}$ we thus obtain $\varphi;\Phi;\Gamma\vdash \asyncoutputch{a}{\widetilde{e}'}{}[v\mapsto e] \triangleleft \{I\}$.
%
\item[$\runa{S-annot}$] We have that $\varphi;\Phi;\downarrow_n\!\!(\Gamma,v:T)\vdash P \triangleleft \kappa$ and $\varphi;\Phi;\Gamma,v:T\vdash n : P \triangleleft \kappa + n$. By Lemma \ref{lemma:susumedefer}, we have that $\varphi;\Phi;\downarrow_n\!\!\Gamma\vdash e :\; \susume{T}{\varphi}{\Phi}{n}$, and so by induction we obtain $\varphi;\Phi;\downarrow_n\!\!\Gamma\vdash P[v\mapsto e] \triangleleft \kappa$. By application of $\runa{S-annot}$ we then derive $\varphi;\Phi;\Gamma\vdash n : P[v\mapsto e] \triangleleft \kappa + n$.
%
\item[$\runa{S-oserv}$] We have that $\varphi;\Phi;\Gamma,v:T\vdash a : \forall_0\widetilde{i}.\texttt{serv}^\sigma_K(\widetilde{S})$, $\varphi;\Phi;\downarrow_I\!\!(\Gamma,v:T)\vdash \widetilde{e}' : \widetilde{S}'$, $\varphi;\Phi;\Gamma,v:T\vdash \asyncoutputch{a}{\widetilde{e}}{} \triangleleft \{K\{\widetilde{J}/\widetilde{i}\}+I\}$ and $\varphi;\Phi;\Gamma\vdash e : T$, where $\text{instantiate}(\widetilde{i},\widetilde{S}')=\{\widetilde{J}/\widetilde{i}\}$. From point $1$ we obtain $\varphi;\Phi;\Gamma\vdash a[v\mapsto e] : \forall_0\widetilde{i}.\texttt{serv}^\sigma_K(\widetilde{S})$, and by Lemma \ref{lemma:basisdefer} we derive $\varphi;\Phi;\downarrow_I\!\!\Gamma\vdash e :\; \susume{T}{\varphi}{\Phi}{I}$. Thus, by induction we obtain $\varphi;\Phi;\downarrow_I\!\!\Gamma\vdash \widetilde{e}'[v\mapsto e] : \widetilde{S}'$. Finally, by application of $\runa{S-oserv}$, we obtain $\varphi;\Phi;\Gamma\vdash \asyncoutputch{a}{\widetilde{e}[v\mapsto e]}{} \triangleleft \{K\{\widetilde{J}/\widetilde{i}\}+I\}$.
%
%
\end{description}
\end{enumerate}
\end{proof}
\end{lemma}

\begin{lemma}[Subject congruence]
Let $P$ and $Q$ be processes such that $P\equiv Q$ then $\varphi;\Phi;\Gamma\vdash P \triangleleft \kappa$ if and only if $\varphi;\Phi;\Gamma\vdash Q \triangleleft \kappa'$ with $\varphi;\Phi\vDash \kappa = \kappa'$.
\begin{proof} By induction on the rules defining $\equiv$.
\begin{description}
\item[$\runa{SC-nil}$] We have that $P \mid \nil \equiv P$. We either have that $\varphi;\Phi;\Gamma\vdash P \mid \nil \triangleleft \kappa'$ or $\varphi;\Phi;\Gamma\vdash P \triangleleft \kappa$. In the former case, we must use type rule $\runa{S-par}$, and so we derive $\varphi;\Phi;\Gamma\vdash P \triangleleft \kappa$. Thus, it suffices to show that $\varphi;\Phi\vDash \kappa = \kappa'$. By $\runa{S-nil}$ we have that $\varphi;\Phi;\Gamma\vdash \nil \triangleleft \{0\}$. By $\runa{S-par}$ we have that $\kappa'=\text{basis}(\varphi,\Phi,\kappa \cup \{0\}) = \text{basis}(\varphi,\Phi,\kappa)$, as $\varphi;\Phi\vDash 0 \leq \kappa$. By Lemma \ref{lemma:basisdefer} we have that $\varphi;\Phi\vDash\text{basis}(\varphi,\Phi,\kappa)=\kappa$.
%
\item[$\runa{SC-commu}$] We have that $P\mid Q \equiv Q\mid P$. In either case we must use type rule $\runa{S-par}$ and so we have that $\varphi;\Phi;\Gamma\vdash P \triangleleft \kappa$ and $\varphi;\Phi;\Gamma\vdash Q \triangleleft \kappa'$. By the commutative law of set union, $\kappa\cup\kappa'=\kappa'\cup\kappa$ and so by extension, $\text{basis}(\varphi,\Phi,\kappa\cup\kappa')=\text{basis}(\varphi,\Phi,\kappa'\cup\kappa)$. Thus, by application of $\runa{S-par}$ we obtain $\varphi;\Phi;\Gamma\vdash Q \mid P \triangleleft \text{basis}(\varphi,\Phi,\kappa\cup\kappa')$ and $\varphi;\Phi;\Gamma\vdash P \mid Q \triangleleft \text{basis}(\varphi,\Phi,\kappa\cup\kappa')$.
%
\item[$\runa{SC-assoc}$] We have that $P\mid (Q \mid R) \equiv (P\mid Q) \mid R$. In either case we must use type rule $\runa{S-par}$ twice such that $\varphi;\Phi;\Gamma\vdash P \triangleleft \kappa$, $\varphi;\Phi;\Gamma\vdash Q \triangleleft \kappa'$ and
$\varphi;\Phi;\Gamma\vdash R \triangleleft \kappa''$. From this we obtain two derivation trees of the form in both cases
    \begin{align*}
        \begin{prooftree}
        \Infer0{\pi_P}
        \Infer1{\varphi;\Phi;\Gamma\vdash P \triangleleft \kappa}
        %
        \Infer0{\pi_Q}
        \Infer1{\varphi;\Phi;\Gamma\vdash Q \triangleleft \kappa'}
        %
        \Infer0{\pi_R}
        \Infer1{\varphi;\Phi;\Gamma\vdash R \triangleleft \kappa''}
        %
        \Infer2{\varphi;\Phi;\Gamma\vdash Q \mid R \triangleleft \text{basis}(\varphi,\Phi,\kappa'\cup\kappa'')}
        %
        \Infer2{\varphi;\Phi;\Gamma\vdash P \mid (Q \mid R) \triangleleft \text{basis}(\varphi,\Phi,\kappa\cup\text{basis}(\varphi,\Phi,\kappa'\cup\kappa''))}
        \end{prooftree}\\
        %
        \\
        %
        \begin{prooftree}
        \Infer0{\pi_P}
        \Infer1{\varphi;\Phi;\Gamma\vdash P \triangleleft \kappa}
        %
        \Infer0{\pi_Q}
        \Infer1{\varphi;\Phi;\Gamma\vdash Q \triangleleft \kappa'}
        %
        \Infer2{\varphi;\Phi;\Gamma\vdash P \mid Q \triangleleft \text{basis}(\varphi,\Phi,\kappa\cup\kappa')}
        %
        \Infer0{\pi_R}
        \Infer1{\varphi;\Phi;\Gamma\vdash R \triangleleft \kappa''}
        %
        \Infer2{\varphi;\Phi;\Gamma\vdash (P \mid Q) \mid R \triangleleft \text{basis}(\varphi,\Phi,\text{basis}(\varphi,\Phi,\kappa\cup\kappa')\cup\kappa'')}
        \end{prooftree}
    \end{align*}
Thus, it suffices to show that $\text{basis}(\varphi,\Phi,\kappa\cup\text{basis}(\varphi,\Phi,\kappa'\cup\kappa''))=\text{basis}(\varphi,\Phi,\text{basis}(\varphi,\Phi,\kappa\cup\kappa')\cup\kappa'')$. We obtain this directly from Lemma \ref{lemma:basisdefer}.
%
\item[$\runa{SC-scope}$] We have that $\newvar{a}{(P \mid Q)} \equiv \newvar{a}{P\mid Q}$ and that $a$ is not free in $Q$. We consider the implications separately
\begin{enumerate}
    \item We have that $\varphi;\Phi;\Gamma\vdash \newvar{a}{(P \mid Q)} \triangleleft \kappa''$. Thus, we must use type rule $\runa{S-nu}$ and $\runa{S-par}$ such that $\varphi;\Phi;\Gamma,a:T\vdash P \mid Q \triangleleft \kappa''$, $\varphi;\Phi;\Gamma,a:T\vdash P \triangleleft \kappa$ and $\varphi;\Phi;\Gamma,a:T\vdash Q \triangleleft \kappa'$. By strengthening (Lemma \ref{lemma:strengthening}) we obtain $\varphi;\Phi;\Gamma\vdash Q \triangleleft \kappa'$, and by application of $\runa{S-nu}$ we derive $\varphi;\Phi;\Gamma\vdash \newvar{a}{P} \triangleleft \kappa$. Thus, by application of $\runa{S-par}$ we obtain $\varphi;\Phi;\Gamma\vdash \newvar{a}{P} \mid Q \triangleleft \kappa''$.
    %
    \item We have that $\varphi;\Phi;\Gamma\vdash \newvar{a}{P} \mid Q \triangleleft \kappa''$. Thus, we must use type rule $\runa{S-par}$ and $\runa{S-nu}$ such that $\varphi;\Phi;\Gamma\vdash \newvar{a}{P} \triangleleft \kappa$, $\varphi;\Phi;\Gamma,a:T\vdash P \triangleleft \kappa$ and $\varphi;\Phi;\Gamma\vdash Q \triangleleft \kappa'$. By weakening (Lemma \ref{lemma:weakening}) we obtain $\varphi;\Phi;\Gamma,a:T\vdash Q \triangleleft \kappa'$ and so by application of $\runa{S-par}$ and $\runa{S-nu}$ we derive $\varphi;\Phi;\Gamma,a:T\vdash P \mid Q \triangleleft \kappa''$ and $\varphi;\Phi;\Gamma\vdash \newvar{a}{(P \mid Q)} \triangleleft \kappa''$.
\end{enumerate}
%
\item[$\runa{SC-par}$] We have that $P\mid Q \equiv P' \mid Q$ with $P\equiv P'$. We must use type rule $\runa{S-par}$ and so we either have that $\varphi;\Phi;\Gamma\vdash P \mid Q \triangleleft \kappa''$ or $\varphi;\Phi;\Gamma\vdash P' \mid Q \triangleleft \kappa''$ with $\varphi;\Phi;\Gamma\vdash Q \triangleleft \kappa'$. When $P$ is well-typed we obtain an equivalent typing for $P'$ and vice-versa by induction. Thus, we have that $\varphi;\Phi;\Gamma\vdash P \triangleleft \kappa$ and $\varphi;\Phi;\Gamma\vdash P' \triangleleft \kappa'$ with $\varphi;\Phi\vDash \kappa = \kappa'$, and so in either case, it suffices to apply $\runa{S-par}$.
%
\item[$\runa{SC-res}$] We have that $\newvar{a}{P} \equiv \newvar{a}{Q}$ with $P \equiv Q$. We must use type rule $\runa{S-nu}$ and so we either have that $\varphi;\Phi;\Gamma\vdash \newvar{a}{P} \triangleleft \kappa$ with $\varphi;\Phi;\Gamma,a:T\vdash P \triangleleft \kappa$ or $\varphi;\Phi;\Gamma\vdash \newvar{a}{Q} \triangleleft \kappa'$ with $\varphi;\Phi;\Gamma,a:T\vdash Q \triangleleft \kappa'$. In either case we use induction to obtain an equivalent typing for $Q$ when we have the same typing for $P$ and vice-versa, i.e. $\varphi;\Phi\vDash \kappa = \kappa'$. Thus in either case, it suffices to apply $\runa{S-nu}$.
%
\item[$\runa{SC-zero}$] This result is obtained directly from $\susume{\Gamma}{\varphi}{\Phi}{0}=\Gamma$.% We have that $P \equiv 0 : P$, and so we must use type rule $\runa{S-annot}$. 
%
\item[$\runa{SC-sum}$] We have that $n : m : P \equiv n+m : P$, and so we must use type rule $\runa{S-annot}$. In the first case we have that $\varphi;\Phi;\downarrow_m\!\!(\downarrow_n\!\!\Gamma)\vdash P \triangleleft \kappa$, $\varphi;\Phi;\downarrow_n\!\!\Gamma\vdash m : P \triangleleft \kappa + m$ and $\varphi;\Phi;\Gamma\vdash n : m : P \triangleleft \kappa + m + n$. In the second case we have that $\varphi;\Phi;\downarrow_{n+m}\!\!\Gamma\vdash P \triangleleft \kappa$ and $\varphi;\Phi;\Gamma\vdash (n+m) : P \triangleleft \kappa + m + n$. Thus, it suffices to show that $\susume{\Gamma}{\varphi}{\Phi}{n+m} = \susume{\susume{\Gamma}{\varphi}{\Phi}{n}}{\varphi}{\Phi}{m}$. We obtain this directly from Lemma \ref{lemma:addsusume}.
%
\item[$\runa{SC-dis}$] We have that $n : (P \mid Q) \equiv (n : P) \mid (n : Q)$, and so we must use type rule $\runa{S-par}$ and $\runa{S-annot}$. We have the two derivation trees 
{\small
\begin{align*}
    \begin{prooftree}
    \Infer0{\pi_P}
    \Infer1{\varphi;\Phi;\downarrow_n\!\!\Gamma\vdash P \triangleleft \kappa}
    %
    \Infer0{\pi_Q}
    \Infer1{\varphi;\Phi;\downarrow_n\!\!\Gamma\vdash Q \triangleleft \kappa'}
    %
    \Infer2{\varphi;\Phi;\downarrow_n\!\!\Gamma\vdash P \mid Q \triangleleft \text{basis}(\varphi,\Phi,\kappa\cup\kappa')}
    %
    \Infer1{\varphi;\Phi;\Gamma\vdash n : (P \mid Q) \triangleleft \text{basis}(\varphi,\Phi,\kappa\cup\kappa') + n}
    \end{prooftree}\quad
    %
    \begin{prooftree}
    \Infer0{\pi_P}
    \Infer1{\varphi;\Phi;\downarrow_n\!\!\Gamma\vdash P \triangleleft \kappa}
    \Infer1{\varphi;\Phi;\Gamma\vdash n : P \triangleleft \kappa + n}
    %
    \Infer0{\pi_Q}
    \Infer1{\varphi;\Phi;\downarrow_n\!\!\Gamma\vdash Q \triangleleft \kappa'}
    \Infer1{\varphi;\Phi;\Gamma\vdash n : Q \triangleleft \kappa' + n}
    %
    \Infer2{\varphi;\Phi;\Gamma\vdash (n : P) \mid (n:Q) \triangleleft \text{basis}(\varphi,\Phi,(\kappa+n)\cup(\kappa'+n))}
    \end{prooftree}
\end{align*}}
Thus, it suffices to show that $\varphi;\Phi\vDash \text{basis}(\varphi,\Phi,\kappa\cup\kappa') + n = \text{basis}(\varphi,\Phi,(\kappa+n)\cup(\kappa'+n))$. We obtain this directly from Lemma \ref{lemma:basisdefer}.
%
\item[$\runa{SC-ares}$] We have that $n : \newvar{a}{P} \equiv \newvar{a}{n : P}$, and so we must use type rule $\runa{S-annot}$ and $\runa{S-nu}$. We have the two derivation trees
\begin{align*}
    \begin{prooftree}
    \Infer0{\pi_P}
    \Infer1{\varphi;\Phi;\downarrow_n\!\!\Gamma,a:T\vdash P \triangleleft \kappa}
    \Infer1{\varphi;\Phi;\downarrow_n\!\!\Gamma\vdash \newvar{a}{P} \triangleleft \kappa}
    \Infer1{\varphi;\Phi;\Gamma\vdash n : \newvar{a}{P} \triangleleft \kappa + n}
    \end{prooftree}\quad
    %
    \begin{prooftree}
    \Infer0{\pi_P}
    \Infer1{\varphi;\Phi;\downarrow_n\!\!(\Gamma,a:\uparrow^n\!\!T)\vdash P \triangleleft \kappa'}
    \Infer1{\varphi;\Phi;\Gamma,a:\uparrow^n\!\!T\vdash n : P \triangleleft \kappa' + n}
    \Infer1{\varphi;\Phi;\Gamma\vdash \newvar{a}{n : P} \triangleleft \kappa' + n}
    \end{prooftree}
\end{align*}
From Lemma \ref{lemma:delayingg}, we have that $\susume{\uparrow^n\!\!T}{\varphi}{\Phi}{n}=T$, and so $\susume{\Gamma,a:\uparrow^n\!\!T^}{\varphi}{\Phi}{n}=\;\susume{\Gamma}{\varphi}{\Phi}{n},a:T$. This implies that $\kappa=\kappa'$, and so from either typing we can reach the other by application of type rule $\runa{S-nu}$ and $\runa{S-annot}$.
\end{description}
\end{proof}
\end{lemma}
\chapter{Type inference examples}\label{app:runningexm}

We first show the inferred and reduced constraint satisfaction problems for the process
\begin{align*}
    &\kern0em P_{\text{npar}} \defeq\\
    &(\nu \text{npar})(\\
    &\kern2em !\text{npar}(n,r).\texttt{match}\; n\; \{\\
    &\kern3em 0 \mapsto \asyncoutputch{r}{}{}\\
    &\kern3em s(x) \mapsto (\nu r' )(\nu r'' )(\\
    &\kern4em {\color{blue} \tick{{\color{black} \asyncoutputch{r'}{}{}}} } \mid
 \asyncoutputch{\text{npar}}{x,r''}{} \mid \inputch{r'}{}{}{\inputch{r''}{}{}{\asyncoutputch{r}{}{}}}) \} \\
    &\kern2em \mid \\
    &(\nu r)( \asyncoutputch{\text{npar}}{10,r}{} \mid \inputch{r}{}{}{\nil} ))
\end{align*}
We infer the following constraints

{
\tiny

\begin{align*}
    \texttt{Nat}[0, \alpha_{52}] \sim \texttt{Nat}[0, \alpha_{53}]\\ \texttt{Nat}[0, \alpha_{62}] \sim \texttt{Nat}[0, (1\alpha_{62}+0)]\\ \texttt{Nat}[0, \alpha_{24} + \alpha_{25}i_{0}] \sim \texttt{Nat}[0, (1\alpha_{24}+0) + 1\alpha_{25}i_{0}]\\ \texttt{ch}^{\gamma_{0}}_{\alpha_{2} + \alpha_{3}i_{0}}() \sim \texttt{ch}^{\gamma_{6}}_{((\alpha_{28}+\alpha_{32})+\alpha_{36}) + ((\alpha_{29}+\alpha_{33})+\alpha_{37})i_{0}}()\\ \texttt{ch}^{\gamma_{1}}_{(\alpha_{6}+1) + \alpha_{7}i_{0}}() \sim \texttt{ch}^{\gamma_{8}}_{\alpha_{36} + \alpha_{37}i_{0}}()\\ \texttt{ch}^{\gamma_{2}}_{(\alpha_{14}+\alpha_{16}) + (\alpha_{15}+\alpha_{17})i_{0}}() \sim \texttt{ch}^{\gamma_{7}}_{(\alpha_{32}+\alpha_{36}) + (\alpha_{33}+\alpha_{37})i_{0}}()\\ \texttt{ch}^{\gamma_{5}}_{\alpha_{26} + \alpha_{27}i_{0}}() \sim \texttt{ch}^{\gamma_{4}}_{(\alpha_{21}\alpha_{24}+\alpha_{20}) + \alpha_{21}\alpha_{25}i_{0}}()\\ \texttt{ch}^{\gamma_{13}}_{(\alpha_{55}+\alpha_{56})}() \sim \texttt{ch}^{\gamma_{17}}_{\alpha_{65}}()\\ \texttt{ch}^{\gamma_{16}}_{\alpha_{63}}() \sim \texttt{ch}^{\gamma_{15}}_{(\alpha_{60}\alpha_{62}+\alpha_{59})}()\\ \forall_{\alpha_{40}}{i_{0}}.\texttt{serv}^{\gamma_{9}}_{\alpha_{41} + \alpha_{42}i_{0}}(\texttt{Nat}[0, 0 + 1i_{0}], \texttt{ch}^{\gamma_{10}}_{\alpha_{43} + \alpha_{44}i_{0}}()) \sim \forall_{\alpha_{56}}{i_{0}}.\texttt{serv}^{\gamma_{14}}_{\alpha_{57} + \alpha_{58}i_{0}}(\texttt{Nat}[0, 0 + 1i_{0}], \texttt{ch}^{\gamma_{15}}_{\alpha_{59} + \alpha_{60}i_{0}}())\\ \{\};\{\} \vDash \texttt{inv}(\forall_{\alpha_{16} + \alpha_{17}i_{0}}{i_{0}}.\texttt{serv}^{\gamma_{3}}_{\alpha_{18} + \alpha_{19}i_{0}}(\texttt{Nat}[0, 0 + 1i_{0}], \texttt{ch}^{\gamma_{4}}_{\alpha_{20} + \alpha_{21}i_{0}}()))\\ () \implies (\{\};\{\}  \vdash \texttt{Nat}[0, \alpha_{54}] \sqsubseteq \texttt{Nat}[0, \alpha_{62}])\\ () \implies (\{\};\{\}  \vdash \texttt{Nat}[0, (\alpha_{53}+10)] \sqsubseteq \texttt{Nat}[0, \alpha_{54}])\\ () \implies (\{\};\{\}  \vdash \texttt{ch}^{\gamma_{13}}_{\alpha_{55}}() \sqsubseteq \texttt{ch}^{\gamma_{16}}_{\alpha_{63}}())\\ () \implies (\{\};\{\}  \vdash \forall_{\alpha_{47}}{i_{0}}.\texttt{serv}^{\gamma_{11}}_{\alpha_{48} + \alpha_{49}i_{0}}(\texttt{Nat}[0, 0 + 1i_{0}], \texttt{ch}^{\gamma_{12}}_{\alpha_{50} + \alpha_{51}i_{0}}()) \sqsubseteq \forall_{\alpha_{46}}{i_{0}}.\texttt{serv}^{\gamma_{3}}_{\alpha_{18} + \alpha_{19}i_{0}}(\texttt{Nat}[0, 0 + 1i_{0}], \texttt{ch}^{\gamma_{4}}_{\alpha_{20} + \alpha_{21}i_{0}}()))\\ () \implies (\{i_{0}\};\{\}  \vdash \texttt{Nat}[0, \alpha_{0} + \alpha_{1}i_{0}] \sqsubseteq \texttt{Nat}[0, (\alpha_{12}+1) + \alpha_{13}i_{0}])\\ () \implies (\{i_{0}\};\{\}  \vdash \texttt{Nat}[0, 0 + 1i_{0}] \sqsubseteq \texttt{Nat}[0, \alpha_{0} + \alpha_{1}i_{0}])\\ () \implies (\{i_{0}\};\{\}  \vdash \texttt{Nat}[0, 0 + 1i_{0}] \sqsubseteq \texttt{Nat}[0, 0 + 1i_{0}])\\ () \implies (\{i_{0}\};\{\}  \vdash \texttt{ch}^{\gamma_{4}}_{\alpha_{20} + \alpha_{21}i_{0}}() \sqsubseteq \texttt{ch}^{\gamma_{10}}_{\alpha_{43} + \alpha_{44}i_{0}}())\\ () \implies (\{i_{0}\};\{\}  \vdash \texttt{ch}^{\gamma_{10}}_{\alpha_{43} + \alpha_{44}i_{0}}() \sqsubseteq \texttt{ch}^{\gamma_{0}}_{\alpha_{2} + \alpha_{3}i_{0}}())\\ () \implies (\{i_{0}\};\{1 \leq \alpha_{0} + \alpha_{1}i_{0}\}  \vdash \texttt{Nat}[0, \alpha_{12} + \alpha_{13}i_{0}] \sqsubseteq \texttt{Nat}[0, \alpha_{24} + \alpha_{25}i_{0}])\\ () \implies (\{i_{0}\};\{1 \leq \alpha_{0} + \alpha_{1}i_{0}\}  \vdash \texttt{ch}^{\gamma_{2}}_{\alpha_{14} + \alpha_{15}i_{0}}() \sqsubseteq \texttt{ch}^{\gamma_{5}}_{\alpha_{26} + \alpha_{27}i_{0}}())\\ () \implies (\{\};\{\}  \vDash \alpha_{40} \leq \alpha_{45})\\ () \implies (\{\};\{\}  \vDash \alpha_{46} \leq \alpha_{40})\\ () \implies (\{\};\{\}  \vDash 0 \leq \alpha_{40}), () \implies (\{\};\{\}  \vDash 0 \leq \alpha_{45}), () \implies (\{\};\{\}  \vDash 0 \leq \alpha_{52})\\
    () \implies (\{\};\{\}  \vDash 0 \leq \alpha_{56}), () \implies (\{\};\{\}  \vDash 0 \leq \alpha_{64}), () \implies (\{\};\{\}  \vDash 0 \leq \alpha_{65}),
    () \implies (\{\};\{\}  \vDash 0 \leq \alpha_{57} + \alpha_{58}i_{0})\\ () \implies (\{\};\{\}  \vDash (\alpha_{56}+(\alpha_{58}\alpha_{62}+\alpha_{57})) \leq \alpha_{61})\\ () \implies (\{\};\{\}  \vDash (\alpha_{64}+\alpha_{65}) \leq \alpha_{66})\\ () \implies (\{i_{0}\};\{\}  \vDash 0 \leq \alpha_{0} + \alpha_{1}i_{0})\\ () \implies (\{i_{0}\};\{\}  \vDash \alpha_{41} + \alpha_{42}i_{0} \leq \alpha_{18} + \alpha_{19}i_{0})\\ () \implies (\{i_{0}\};\{0 \leq 0\}  \vDash 0 \leq \alpha_{2} + \alpha_{3}i_{0}), () \implies (\{i_{0}\};\{0 \leq 0\}  \vDash \alpha_{2} + \alpha_{3}i_{0} \leq \alpha_{4} + \alpha_{5}i_{0})\\ () \implies (\{i_{0}\};\{1 \leq \alpha_{0} + \alpha_{1}i_{0}\}  \vDash 0 \leq \alpha_{6} + \alpha_{7}i_{0}), () \implies (\{i_{0}\};\{1 \leq \alpha_{0} + \alpha_{1}i_{0}\}  \vDash 0 \leq \alpha_{12} + \alpha_{13}i_{0})\\ () \implies (\{i_{0}\};\{1 \leq \alpha_{0} + \alpha_{1}i_{0}\}  \vDash 0 \leq \alpha_{18} + \alpha_{19}i_{0}), () \implies (\{i_{0}\};\{1 \leq \alpha_{0} + \alpha_{1}i_{0}\}  \vDash 0 \leq \alpha_{28} + \alpha_{29}i_{0})\\ () \implies (\{i_{0}\};\{1 \leq \alpha_{0} + \alpha_{1}i_{0}\}  \vDash 0 \leq \alpha_{32} + \alpha_{33}i_{0}), () \implies (\{i_{0}\};\{1 \leq \alpha_{0} + \alpha_{1}i_{0}\}  \vDash 0 \leq \alpha_{36} + \alpha_{37}i_{0})\\ () \implies (\{i_{0}\};\{1 \leq \alpha_{0} + \alpha_{1}i_{0}\}  \vDash \alpha_{6} + \alpha_{7}i_{0} \leq \alpha_{8} + \alpha_{9}i_{0})\\ () \implies (\{i_{0}\};\{1 \leq \alpha_{0} + \alpha_{1}i_{0}\}  \vDash (\alpha_{8}+1) + \alpha_{9}i_{0} \leq \alpha_{10} + \alpha_{11}i_{0})\\ () \implies (\{i_{0}\};\{1 \leq \alpha_{0} + \alpha_{1}i_{0}\}  \vDash \alpha_{28} + \alpha_{29}i_{0} \leq \alpha_{30} + \alpha_{31}i_{0})\\ () \implies (\{i_{0}\};\{1 \leq \alpha_{0} + \alpha_{1}i_{0}\}  \vDash (\alpha_{16}+(\alpha_{19}\alpha_{24}+\alpha_{18})) + (\alpha_{17}+\alpha_{19}\alpha_{25})i_{0} \leq \alpha_{22} + \alpha_{23}i_{0})\\ () \implies (\{i_{0}\};\{1 \leq \alpha_{0} + \alpha_{1}i_{0}\}  \vDash (\alpha_{30}+\alpha_{32}) + (\alpha_{31}+\alpha_{33})i_{0} \leq \alpha_{34} + \alpha_{35}i_{0})\\ () \implies (\{i_{0}\};\{1 \leq \alpha_{0} + \alpha_{1}i_{0}\}  \vDash (\alpha_{34}+\alpha_{36}) + (\alpha_{35}+\alpha_{37})i_{0} \leq \alpha_{38} + \alpha_{39}i_{0})\\ 
    () \implies \gamma_{3} \subseteq \gamma_{9},() \implies \{\texttt{in}\} \subseteq \gamma_{7}, () \implies \{\texttt{in}\} \subseteq \gamma_{8}\\ () \implies \{\texttt{in}\} \subseteq \gamma_{9}, () \implies \{\texttt{in}\} \subseteq \gamma_{17}, () \implies \{\texttt{out}\} \subseteq \gamma_{0}\\ () \implies \{\texttt{out}\} \subseteq \gamma_{1}, () \implies \{\texttt{out}\} \subseteq \gamma_{3}, () \implies \{\texttt{out}\} \subseteq \gamma_{6}, () \implies \{\texttt{out}\} \subseteq \gamma_{14}\\ \alpha_{45} \sim \alpha_{61}\\ \alpha_{61} \sim \alpha_{66}\\ 0 \sim \alpha_{16} + \alpha_{17}i_{0}\\ \alpha_{4} + \alpha_{5}i_{0} \sim \alpha_{10} + \alpha_{11}i_{0}\\ \alpha_{4} + \alpha_{5}i_{0} \sim \alpha_{41} + \alpha_{42}i_{0}\\ \alpha_{10} + \alpha_{11}i_{0} \sim \alpha_{22} + \alpha_{23}i_{0}\\ \alpha_{10} + \alpha_{11}i_{0} \sim \alpha_{38} + \alpha_{39}i_{0}
\end{align*}

}

We reduce the constraints to

{
\tiny

\begin{align*}
    \alpha_{40} \sim \alpha_{56}, \alpha_{45} \sim \alpha_{61}, \alpha_{47} \sim \alpha_{46}, \alpha_{52} \sim \alpha_{53}, \alpha_{55} \sim \alpha_{63},, \alpha_{61} \sim \alpha_{66}\\ \alpha_{62} \sim (1\alpha_{62}+0), \alpha_{63} \sim (\alpha_{60}\alpha_{62}+\alpha_{59}), 0 \sim 0, 0 \sim \alpha_{16} + \alpha_{17}i_{0},
    (\alpha_{55}+\alpha_{56}) \sim \alpha_{65}\\
    %
    \alpha_{2} + \alpha_{3}i_{0} \sim ((\alpha_{28}+\alpha_{32})+\alpha_{36}) + ((\alpha_{29}+\alpha_{33})+\alpha_{37})i_{0}\\
    \alpha_{4} + \alpha_{5}i_{0} \sim \alpha_{10} + \alpha_{11}i_{0}, \alpha_{4} + \alpha_{5}i_{0} \sim \alpha_{41} + \alpha_{42}i_{0}\\
    (\alpha_{6}+1) + \alpha_{7}i_{0} \sim \alpha_{36} + \alpha_{37}i_{0}\\ \alpha_{10} + \alpha_{11}i_{0} \sim \alpha_{22} + \alpha_{23}i_{0}\\ \alpha_{10} + \alpha_{11}i_{0} \sim \alpha_{38} + \alpha_{39}i_{0}\\ \alpha_{14} + \alpha_{15}i_{0} \sim \alpha_{26} + \alpha_{27}i_{0}\\ \alpha_{20} + \alpha_{21}i_{0} \sim \alpha_{43} + \alpha_{44}i_{0}\\ \alpha_{20} + \alpha_{21}i_{0} \sim \alpha_{50} + \alpha_{51}i_{0}\\ \alpha_{24} + \alpha_{25}i_{0} \sim (1\alpha_{24}+0) + 1\alpha_{25}i_{0}\\ \alpha_{26} + \alpha_{27}i_{0} \sim (\alpha_{21}\alpha_{24}+\alpha_{20}) + \alpha_{21}\alpha_{25}i_{0}\\ \alpha_{41} + \alpha_{42}i_{0} \sim \alpha_{57} + \alpha_{58}i_{0}\\ \alpha_{43} + \alpha_{44}i_{0} \sim \alpha_{2} + \alpha_{3}i_{0}\\ \alpha_{43} + \alpha_{44}i_{0} \sim \alpha_{59} + \alpha_{60}i_{0}\\ \alpha_{50} + \alpha_{51}i_{0} \sim \alpha_{20} + \alpha_{21}i_{0}\\ 0 + 1i_{0} \sim 0 + 1i_{0}\\ (\alpha_{14}+\alpha_{16}) + (\alpha_{15}+\alpha_{17})i_{0} \sim (\alpha_{32}+\alpha_{36}) + (\alpha_{33}+\alpha_{37})i_{0}\\ \{\};\{\}  \vDash \alpha_{40} \leq \alpha_{45}\\ \{\};\{\}  \vDash \alpha_{46} \leq \alpha_{40}\\ \{\};\{\}  \vDash \alpha_{54} \leq \alpha_{62}\\ \{\};\{\}  \vDash 0 \leq \alpha_{40}\\ \{\};\{\}  \vDash 0 \leq \alpha_{45}\\ \{\};\{\}  \vDash 0 \leq \alpha_{52}\\ \{\};\{\}  \vDash 0 \leq \alpha_{56}\\ \{\};\{\}  \vDash 0 \leq \alpha_{64}\\ \{\};\{\}  \vDash 0 \leq \alpha_{65}\\ \{\};\{\}  \vDash 0 \leq 0\\ \{\};\{\}  \vDash 0 \leq \alpha_{57} + \alpha_{58}i_{0}\\ \{\};\{\}  \vDash (\alpha_{53}+10) \leq \alpha_{54}\\ \{\};\{\}  \vDash (\alpha_{56}+(\alpha_{58}\alpha_{62}+\alpha_{57})) \leq \alpha_{61}\\ \{\};\{\}  \vDash (\alpha_{64}+\alpha_{65}) \leq \alpha_{66}\\ \{i_{0}\};\{\}  \vDash 0 \leq 0\\ \{i_{0}\};\{\}  \vDash 0 \leq \alpha_{0} + \alpha_{1}i_{0}\\ \{i_{0}\};\{\}  \vDash \alpha_{0} + \alpha_{1}i_{0} \leq (\alpha_{12}+1) + \alpha_{13}i_{0}\\ \{i_{0}\};\{\}  \vDash \alpha_{18} + \alpha_{19}i_{0} \leq \alpha_{48} + \alpha_{49}i_{0}\\ \{i_{0}\};\{\}  \vDash \alpha_{41} + \alpha_{42}i_{0} \leq \alpha_{18} + \alpha_{19}i_{0}\\ \{i_{0}\};\{\}  \vDash 0 + 1i_{0} \leq \alpha_{0} + \alpha_{1}i_{0}\\ \{i_{0}\};\{\}  \vDash 0 + 1i_{0} \leq 0 + 1i_{0}\\ \{i_{0}\};\{0 \leq 0\}  \vDash 0 \leq \alpha_{2} + \alpha_{3}i_{0}\\ \{i_{0}\};\{0 \leq 0\}  \vDash \alpha_{2} + \alpha_{3}i_{0} \leq \alpha_{4} + \alpha_{5}i_{0}\\ \{i_{0}\};\{1 \leq \alpha_{0} + \alpha_{1}i_{0}\}  \vDash 0 \leq 0\\ \{i_{0}\};\{1 \leq \alpha_{0} + \alpha_{1}i_{0}\}  \vDash 0 \leq \alpha_{6} + \alpha_{7}i_{0}\\ \{i_{0}\};\{1 \leq \alpha_{0} + \alpha_{1}i_{0}\}  \vDash 0 \leq \alpha_{12} + \alpha_{13}i_{0}\\ \{i_{0}\};\{1 \leq \alpha_{0} + \alpha_{1}i_{0}\}  \vDash 0 \leq \alpha_{18} + \alpha_{19}i_{0}\\ \{i_{0}\};\{1 \leq \alpha_{0} + \alpha_{1}i_{0}\}  \vDash 0 \leq \alpha_{28} + \alpha_{29}i_{0}\\ \{i_{0}\};\{1 \leq \alpha_{0} + \alpha_{1}i_{0}\}  \vDash 0 \leq \alpha_{32} + \alpha_{33}i_{0}\\ \{i_{0}\};\{1 \leq \alpha_{0} + \alpha_{1}i_{0}\}  \vDash 0 \leq \alpha_{36} + \alpha_{37}i_{0}\\ \{i_{0}\};\{1 \leq \alpha_{0} + \alpha_{1}i_{0}\}  \vDash \alpha_{6} + \alpha_{7}i_{0} \leq \alpha_{8} + \alpha_{9}i_{0}\\ \{i_{0}\};\{1 \leq \alpha_{0} + \alpha_{1}i_{0}\}  \vDash (\alpha_{8}+1) + \alpha_{9}i_{0} \leq \alpha_{10} + \alpha_{11}i_{0}\\ \{i_{0}\};\{1 \leq \alpha_{0} + \alpha_{1}i_{0}\}  \vDash \alpha_{12} + \alpha_{13}i_{0} \leq \alpha_{24} + \alpha_{25}i_{0}\\ \{i_{0}\};\{1 \leq \alpha_{0} + \alpha_{1}i_{0}\}  \vDash \alpha_{28} + \alpha_{29}i_{0} \leq \alpha_{30} + \alpha_{31}i_{0}\\ \{i_{0}\};\{1 \leq \alpha_{0} + \alpha_{1}i_{0}\}  \vDash (\alpha_{16}+(\alpha_{19}\alpha_{24}+\alpha_{18})) + (\alpha_{17}+\alpha_{19}\alpha_{25})i_{0} \leq \alpha_{22} + \alpha_{23}i_{0}\\ \{i_{0}\};\{1 \leq \alpha_{0} + \alpha_{1}i_{0}\}  \vDash (\alpha_{30}+\alpha_{32}) + (\alpha_{31}+\alpha_{33})i_{0} \leq \alpha_{34} + \alpha_{35}i_{0}\\ \{i_{0}\};\{1 \leq \alpha_{0} + \alpha_{1}i_{0}\}  \vDash (\alpha_{34}+\alpha_{36}) + (\alpha_{35}+\alpha_{37})i_{0} \leq \alpha_{38} + \alpha_{39}i_{0}
\end{align*}

}

We were able to find the following solution to the reduced constraint satisfaction problem (written as pairs of coefficient variables and naturals)\\

$\splitatcommas{
(\alpha_{0},0),(\alpha_{1},1),(\alpha_{2},1),(\alpha_{3},0),(\alpha_{4},1),(\alpha_{5},0),(\alpha_{6},0),(\alpha_{7},0),(\alpha_{8},0),(\alpha_{9},0),(\alpha_{10},1),(\alpha_{11},0),(\alpha_{12},0),(\alpha_{13},1),(\alpha_{14},1),(\alpha_{15},0),(\alpha_{16},0),(\alpha_{17},0),(\alpha_{18},1),(\alpha_{19},0),(\alpha_{20},1),(\alpha_{21},0),(\alpha_{22},1),(\alpha_{23},0),(\alpha_{24},0),(\alpha_{25},1),(\alpha_{26},1),(\alpha_{27},0),(\alpha_{28},0),(\alpha_{29},0),(\alpha_{30},0),(\alpha_{31},0),(\alpha_{32},0),(\alpha_{33},0),(\alpha_{34},0),(\alpha_{35},0),(\alpha_{36},1),(\alpha_{37},0),(\alpha_{38},1),(\alpha_{39},0),(\alpha_{40},0),(\alpha_{41},1),(\alpha_{42},0),(\alpha_{43},1),(\alpha_{44},0),(\alpha_{45},1),(\alpha_{46},0),(\alpha_{47},0),(\alpha_{48},1),(\alpha_{49},0),(\alpha_{50},1),(\alpha_{51},0),(\alpha_{52},0),(\alpha_{53},0),(\alpha_{54},10),(\alpha_{55},1),(\alpha_{56},0),(\alpha_{57},1),(\alpha_{58},0),(\alpha_{59},1),(\alpha_{60},0),(\alpha_{61},1),(\alpha_{62},10),(\alpha_{63},1),(\alpha_{64},0),(\alpha_{65},1),(\alpha_{66},1)
}$\\

We next consider the process 
\begin{align*}
    &\kern0em P_{\text{npar}}' \defeq\\
    &(\nu \text{npar})(\\
    &\kern2em !\text{npar}(n,r).\texttt{match}\; n\; \{\\
    &\kern3em 0 \mapsto \asyncoutputch{r}{}{}\\
    &\kern3em s(x) \mapsto (\nu r' )(\nu r'' ) {\color{blue}\texttt{tick}}. (\\ &\kern4em {\asyncoutputch{r'}{}{}} \mid
 \asyncoutputch{\text{npar}}{x,r''}{} \mid \inputch{r'}{}{}{\inputch{r''}{}{}{\asyncoutputch{r}{}{}}}) \} \\
    &\kern2em \mid \\
    &(\nu r)( \asyncoutputch{\text{npar}}{m,r}{} \mid \inputch{r}{}{}{\nil} ))
\end{align*}
We infer the following constraints

{ \tiny

\begin{align*}
    \texttt{Nat}[0, \alpha_{52}] \sim \texttt{Nat}[0, \alpha_{53}]\\ \texttt{Nat}[0, \alpha_{62}] \sim \texttt{Nat}[0, (1\alpha_{62}+0)]\\ \texttt{Nat}[0, \alpha_{22} + \alpha_{23}i_{0}] \sim \texttt{Nat}[0, (1\alpha_{22}+0) + 1\alpha_{23}i_{0}]\\ \texttt{ch}^{\gamma_{0}}_{\alpha_{2} + \alpha_{3}i_{0}}() \sim \texttt{ch}^{\gamma_{6}}_{(((\alpha_{26}+\alpha_{30})+\alpha_{34})+1) + ((\alpha_{27}+\alpha_{31})+\alpha_{35})i_{0}}()\\ \texttt{ch}^{\gamma_{1}}_{\alpha_{6} + \alpha_{7}i_{0}}() \sim \texttt{ch}^{\gamma_{8}}_{\alpha_{34} + \alpha_{35}i_{0}}()\\ \texttt{ch}^{\gamma_{2}}_{(\alpha_{12}+\alpha_{14}) + (\alpha_{13}+\alpha_{15})i_{0}}() \sim \texttt{ch}^{\gamma_{7}}_{(\alpha_{30}+\alpha_{34}) + (\alpha_{31}+\alpha_{35})i_{0}}()\\ \texttt{ch}^{\gamma_{5}}_{\alpha_{24} + \alpha_{25}i_{0}}() \sim \texttt{ch}^{\gamma_{4}}_{(\alpha_{19}\alpha_{22}+\alpha_{18}) + \alpha_{19}\alpha_{23}i_{0}}()\\ \texttt{ch}^{\gamma_{13}}_{(\alpha_{55}+\alpha_{56})}() \sim \texttt{ch}^{\gamma_{17}}_{\alpha_{65}}()\\ \texttt{ch}^{\gamma_{16}}_{\alpha_{63}}() \sim \texttt{ch}^{\gamma_{15}}_{(\alpha_{60}\alpha_{62}+\alpha_{59})}()\\ \forall_{\alpha_{40}}{i_{0}}.\texttt{serv}^{\gamma_{9}}_{\alpha_{41} + \alpha_{42}i_{0}}(\texttt{Nat}[0, 0 + 1i_{0}], \texttt{ch}^{\gamma_{10}}_{\alpha_{43} + \alpha_{44}i_{0}}()) \sim \forall_{\alpha_{56}}{i_{0}}.\texttt{serv}^{\gamma_{14}}_{\alpha_{57} + \alpha_{58}i_{0}}(\texttt{Nat}[0, 0 + 1i_{0}], \texttt{ch}^{\gamma_{15}}_{\alpha_{59} + \alpha_{60}i_{0}}())\\ \{\};\{\} \vDash \texttt{inv}(\forall_{(\alpha_{14}+1) + \alpha_{15}i_{0}}{i_{0}}.\texttt{serv}^{\gamma_{3}}_{\alpha_{16} + \alpha_{17}i_{0}}(\texttt{Nat}[0, 0 + 1i_{0}], \texttt{ch}^{\gamma_{4}}_{\alpha_{18} + \alpha_{19}i_{0}}()))\\ () \implies (\{\};\{\}  \vdash \texttt{Nat}[0, \alpha_{54}] \sqsubseteq \texttt{Nat}[0, \alpha_{62}])\\ () \implies (\{\};\{\}  \vdash \texttt{Nat}[0, (\alpha_{53}+10)] \sqsubseteq \texttt{Nat}[0, \alpha_{54}])\\ () \implies (\{\};\{\}  \vdash \texttt{ch}^{\gamma_{13}}_{\alpha_{55}}() \sqsubseteq \texttt{ch}^{\gamma_{16}}_{\alpha_{63}}())\\ () \implies (\{\};\{\}  \vdash \forall_{\alpha_{47}}{i_{0}}.\texttt{serv}^{\gamma_{11}}_{\alpha_{48} + \alpha_{49}i_{0}}(\texttt{Nat}[0, 0 + 1i_{0}], \texttt{ch}^{\gamma_{12}}_{\alpha_{50} + \alpha_{51}i_{0}}()) \sqsubseteq \forall_{\alpha_{46}}{i_{0}}.\texttt{serv}^{\gamma_{3}}_{\alpha_{16} + \alpha_{17}i_{0}}(\texttt{Nat}[0, 0 + 1i_{0}], \texttt{ch}^{\gamma_{4}}_{\alpha_{18} + \alpha_{19}i_{0}}()))\\ () \implies (\{i_{0}\};\{\}  \vdash \texttt{Nat}[0, \alpha_{0} + \alpha_{1}i_{0}] \sqsubseteq \texttt{Nat}[0, (\alpha_{10}+1) + \alpha_{11}i_{0}])\\ () \implies (\{i_{0}\};\{\}  \vdash \texttt{Nat}[0, 0 + 1i_{0}] \sqsubseteq \texttt{Nat}[0, \alpha_{0} + \alpha_{1}i_{0}])\\ () \implies (\{i_{0}\};\{\}  \vdash \texttt{Nat}[0, 0 + 1i_{0}] \sqsubseteq \texttt{Nat}[0, 0 + 1i_{0}])\\ () \implies (\{i_{0}\};\{\}  \vdash \texttt{ch}^{\gamma_{4}}_{\alpha_{18} + \alpha_{19}i_{0}}() \sqsubseteq \texttt{ch}^{\gamma_{10}}_{\alpha_{43} + \alpha_{44}i_{0}}())\\ () \implies (\{i_{0}\};\{\}  \vdash \texttt{ch}^{\gamma_{10}}_{\alpha_{43} + \alpha_{44}i_{0}}() \sqsubseteq \texttt{ch}^{\gamma_{0}}_{\alpha_{2} + \alpha_{3}i_{0}}())\\ () \implies (\{i_{0}\};\{1 \leq \alpha_{0} + \alpha_{1}i_{0}\}  \vdash \texttt{Nat}[0, \alpha_{10} + \alpha_{11}i_{0}] \sqsubseteq \texttt{Nat}[0, \alpha_{22} + \alpha_{23}i_{0}])\\ () \implies (\{i_{0}\};\{1 \leq \alpha_{0} + \alpha_{1}i_{0}\}  \vdash \texttt{ch}^{\gamma_{2}}_{\alpha_{12} + \alpha_{13}i_{0}}() \sqsubseteq \texttt{ch}^{\gamma_{5}}_{\alpha_{24} + \alpha_{25}i_{0}}())\\ () \implies (\{\};\{\}  \vDash \alpha_{40} \leq \alpha_{45})\\ () \implies (\{\};\{\}  \vDash \alpha_{46} \leq \alpha_{40})\\ () \implies (\{\};\{\}  \vDash 0 \leq \alpha_{40}), () \implies (\{\};\{\}  \vDash 0 \leq \alpha_{45}), () \implies (\{\};\{\}  \vDash 0 \leq \alpha_{52})\\ () \implies (\{\};\{\}  \vDash 0 \leq \alpha_{56}), () \implies (\{\};\{\}  \vDash 0 \leq \alpha_{64}), () \implies (\{\};\{\}  \vDash 0 \leq \alpha_{65})\\ () \implies (\{\};\{\}  \vDash 0 \leq \alpha_{57} + \alpha_{58}i_{0})\\ () \implies (\{\};\{\}  \vDash (\alpha_{56}+(\alpha_{58}\alpha_{62}+\alpha_{57})) \leq \alpha_{61})\\ () \implies (\{\};\{\}  \vDash (\alpha_{64}+\alpha_{65}) \leq \alpha_{66})\\ () \implies (\{i_{0}\};\{\}  \vDash 0 \leq \alpha_{0} + \alpha_{1}i_{0})\\ () \implies (\{i_{0}\};\{\}  \vDash \alpha_{41} + \alpha_{42}i_{0} \leq \alpha_{16} + \alpha_{17}i_{0})\\ () \implies (\{i_{0}\};\{0 \leq 0\}  \vDash 0 \leq \alpha_{2} + \alpha_{3}i_{0})\\ () \implies (\{i_{0}\};\{0 \leq 0\}  \vDash \alpha_{2} + \alpha_{3}i_{0} \leq \alpha_{4} + \alpha_{5}i_{0})\\ () \implies (\{i_{0}\};\{1 \leq \alpha_{0} + \alpha_{1}i_{0}\}  \vDash 0 \leq \alpha_{6} + \alpha_{7}i_{0})\\ () \implies (\{i_{0}\};\{1 \leq \alpha_{0} + \alpha_{1}i_{0}\}  \vDash 0 \leq \alpha_{10} + \alpha_{11}i_{0})\\ () \implies (\{i_{0}\};\{1 \leq \alpha_{0} + \alpha_{1}i_{0}\}  \vDash 0 \leq \alpha_{16} + \alpha_{17}i_{0})\\ () \implies (\{i_{0}\};\{1 \leq \alpha_{0} + \alpha_{1}i_{0}\}  \vDash 0 \leq \alpha_{26} + \alpha_{27}i_{0})\\ () \implies (\{i_{0}\};\{1 \leq \alpha_{0} + \alpha_{1}i_{0}\}  \vDash 0 \leq \alpha_{30} + \alpha_{31}i_{0})\\ () \implies (\{i_{0}\};\{1 \leq \alpha_{0} + \alpha_{1}i_{0}\}  \vDash 0 \leq \alpha_{34} + \alpha_{35}i_{0})\\ () \implies (\{i_{0}\};\{1 \leq \alpha_{0} + \alpha_{1}i_{0}\}  \vDash \alpha_{6} + \alpha_{7}i_{0} \leq \alpha_{8} + \alpha_{9}i_{0})\\ () \implies (\{i_{0}\};\{1 \leq \alpha_{0} + \alpha_{1}i_{0}\}  \vDash (\alpha_{8}+1) + \alpha_{9}i_{0} \leq \alpha_{38} + \alpha_{39}i_{0})\\ () \implies (\{i_{0}\};\{1 \leq \alpha_{0} + \alpha_{1}i_{0}\}  \vDash \alpha_{26} + \alpha_{27}i_{0} \leq \alpha_{28} + \alpha_{29}i_{0})\\ () \implies (\{i_{0}\};\{1 \leq \alpha_{0} + \alpha_{1}i_{0}\}  \vDash (\alpha_{14}+(\alpha_{17}\alpha_{22}+\alpha_{16})) + (\alpha_{15}+\alpha_{17}\alpha_{23})i_{0} \leq \alpha_{20} + \alpha_{21}i_{0})\\ () \implies (\{i_{0}\};\{1 \leq \alpha_{0} + \alpha_{1}i_{0}\}  \vDash (\alpha_{28}+\alpha_{30}) + (\alpha_{29}+\alpha_{31})i_{0} \leq \alpha_{32} + \alpha_{33}i_{0})\\ () \implies (\{i_{0}\};\{1 \leq \alpha_{0} + \alpha_{1}i_{0}\}  \vDash (\alpha_{32}+\alpha_{34}) + (\alpha_{33}+\alpha_{35})i_{0} \leq \alpha_{36} + \alpha_{37}i_{0})\\ 
    () \implies \gamma_{3} \subseteq \gamma_{9}, () \implies \{\texttt{in}\} \subseteq \gamma_{7}, () \implies \{\texttt{in}\} \subseteq \gamma_{8}\\ 
    () \implies \{\texttt{in}\} \subseteq \gamma_{9}, () \implies \{\texttt{in}\} \subseteq \gamma_{17}, () \implies \{\texttt{out}\} \subseteq \gamma_{0}, () \implies \{\texttt{out}\} \subseteq \gamma_{1}\\ 
    () \implies \{\texttt{out}\} \subseteq \gamma_{3}, () \implies \{\texttt{out}\} \subseteq \gamma_{6}, () \implies \{\texttt{out}\} \subseteq \gamma_{14}\\ \alpha_{45} \sim \alpha_{61}\\ \alpha_{61} \sim \alpha_{66}\\ 0 \sim \alpha_{14} + \alpha_{15}i_{0}\\ \alpha_{4} + \alpha_{5}i_{0} \sim \alpha_{38} + \alpha_{39}i_{0}, \alpha_{4} + \alpha_{5}i_{0} \sim \alpha_{41} + \alpha_{42}i_{0}, \alpha_{8} + \alpha_{9}i_{0} \sim \alpha_{20} + \alpha_{21}i_{0}, \alpha_{8} + \alpha_{9}i_{0} \sim \alpha_{36} + \alpha_{37}i_{0}
\end{align*}

}

We reduce the constraints to

{\tiny
\begin{align*}
    \alpha_{40} \sim \alpha_{56}, \alpha_{45} \sim \alpha_{61}, \alpha_{47} \sim \alpha_{46}, \alpha_{52} \sim \alpha_{53}, \alpha_{55} \sim \alpha_{63}, \alpha_{61} \sim \alpha_{66}, \alpha_{62} \sim (1\alpha_{62}+0)\\ 
    \alpha_{63} \sim (\alpha_{60}\alpha_{62}+\alpha_{59}), 0 \sim 0, 0 \sim \alpha_{14} + \alpha_{15}i_{0}, (\alpha_{55}+\alpha_{56}) \sim \alpha_{65}\\ \alpha_{2} + \alpha_{3}i_{0} \sim (((\alpha_{26}+\alpha_{30})+\alpha_{34})+1) + ((\alpha_{27}+\alpha_{31})+\alpha_{35})i_{0}\\ \alpha_{4} + \alpha_{5}i_{0} \sim \alpha_{38} + \alpha_{39}i_{0}\\ \alpha_{4} + \alpha_{5}i_{0} \sim \alpha_{41} + \alpha_{42}i_{0}\\ \alpha_{6} + \alpha_{7}i_{0} \sim \alpha_{34} + \alpha_{35}i_{0}\\ \alpha_{8} + \alpha_{9}i_{0} \sim \alpha_{20} + \alpha_{21}i_{0}\\ \alpha_{8} + \alpha_{9}i_{0} \sim \alpha_{36} + \alpha_{37}i_{0}\\ \alpha_{12} + \alpha_{13}i_{0} \sim \alpha_{24} + \alpha_{25}i_{0}\\ \alpha_{18} + \alpha_{19}i_{0} \sim \alpha_{43} + \alpha_{44}i_{0}\\ \alpha_{18} + \alpha_{19}i_{0} \sim \alpha_{50} + \alpha_{51}i_{0}\\ \alpha_{22} + \alpha_{23}i_{0} \sim (1\alpha_{22}+0) + 1\alpha_{23}i_{0}\\ \alpha_{24} + \alpha_{25}i_{0} \sim (\alpha_{19}\alpha_{22}+\alpha_{18}) + \alpha_{19}\alpha_{23}i_{0}\\ \alpha_{41} + \alpha_{42}i_{0} \sim \alpha_{57} + \alpha_{58}i_{0}\\ \alpha_{43} + \alpha_{44}i_{0} \sim \alpha_{2} + \alpha_{3}i_{0}\\ \alpha_{43} + \alpha_{44}i_{0} \sim \alpha_{59} + \alpha_{60}i_{0}\\ \alpha_{50} + \alpha_{51}i_{0} \sim \alpha_{18} + \alpha_{19}i_{0}\\ 0 + 1i_{0} \sim 0 + 1i_{0}\\ (\alpha_{12}+\alpha_{14}) + (\alpha_{13}+\alpha_{15})i_{0} \sim (\alpha_{30}+\alpha_{34}) + (\alpha_{31}+\alpha_{35})i_{0}\\ \{\};\{\}  \vDash \alpha_{40} \leq \alpha_{45}\\ \{\};\{\}  \vDash \alpha_{46} \leq \alpha_{40}\\ \{\};\{\}  \vDash \alpha_{54} \leq \alpha_{62}\\ \{\};\{\}  \vDash 0 \leq \alpha_{40}, \{\};\{\}  \vDash 0 \leq \alpha_{45}, \{\};\{\}  \vDash 0 \leq \alpha_{52}\\ \{\};\{\}  \vDash 0 \leq \alpha_{56}, \{\};\{\}  \vDash 0 \leq \alpha_{64}, \{\};\{\}  \vDash 0 \leq \alpha_{65}\\ \{\};\{\}  \vDash 0 \leq 0\\ \{\};\{\}  \vDash 0 \leq \alpha_{57} + \alpha_{58}i_{0}\\ \{\};\{\}  \vDash (\alpha_{53}+10) \leq \alpha_{54}\\ \{\};\{\}  \vDash (\alpha_{56}+(\alpha_{58}\alpha_{62}+\alpha_{57})) \leq \alpha_{61}\\ \{\};\{\}  \vDash (\alpha_{64}+\alpha_{65}) \leq \alpha_{66}\\ \{i_{0}\};\{\}  \vDash 0 \leq 0\\ \{i_{0}\};\{\}  \vDash 0 \leq \alpha_{0} + \alpha_{1}i_{0}\\ \{i_{0}\};\{\}  \vDash \alpha_{0} + \alpha_{1}i_{0} \leq (\alpha_{10}+1) + \alpha_{11}i_{0}\\ \{i_{0}\};\{\}  \vDash \alpha_{16} + \alpha_{17}i_{0} \leq \alpha_{48} + \alpha_{49}i_{0}\\ \{i_{0}\};\{\}  \vDash \alpha_{41} + \alpha_{42}i_{0} \leq \alpha_{16} + \alpha_{17}i_{0}\\ \{i_{0}\};\{\}  \vDash 0 + 1i_{0} \leq \alpha_{0} + \alpha_{1}i_{0}\\ \{i_{0}\};\{\}  \vDash 0 + 1i_{0} \leq 0 + 1i_{0}\\ \{i_{0}\};\{0 \leq 0\}  \vDash 0 \leq \alpha_{2} + \alpha_{3}i_{0}\\ \{i_{0}\};\{0 \leq 0\}  \vDash \alpha_{2} + \alpha_{3}i_{0} \leq \alpha_{4} + \alpha_{5}i_{0}\\ \{i_{0}\};\{1 \leq \alpha_{0} + \alpha_{1}i_{0}\}  \vDash 0 \leq 0\\ \{i_{0}\};\{1 \leq \alpha_{0} + \alpha_{1}i_{0}\}  \vDash 0 \leq \alpha_{6} + \alpha_{7}i_{0}\\ \{i_{0}\};\{1 \leq \alpha_{0} + \alpha_{1}i_{0}\}  \vDash 0 \leq \alpha_{10} + \alpha_{11}i_{0}\\ \{i_{0}\};\{1 \leq \alpha_{0} + \alpha_{1}i_{0}\}  \vDash 0 \leq \alpha_{16} + \alpha_{17}i_{0}\\ \{i_{0}\};\{1 \leq \alpha_{0} + \alpha_{1}i_{0}\}  \vDash 0 \leq \alpha_{26} + \alpha_{27}i_{0}\\ \{i_{0}\};\{1 \leq \alpha_{0} + \alpha_{1}i_{0}\}  \vDash 0 \leq \alpha_{30} + \alpha_{31}i_{0}\\ \{i_{0}\};\{1 \leq \alpha_{0} + \alpha_{1}i_{0}\}  \vDash 0 \leq \alpha_{34} + \alpha_{35}i_{0}\\ \{i_{0}\};\{1 \leq \alpha_{0} + \alpha_{1}i_{0}\}  \vDash \alpha_{6} + \alpha_{7}i_{0} \leq \alpha_{8} + \alpha_{9}i_{0}\\ \{i_{0}\};\{1 \leq \alpha_{0} + \alpha_{1}i_{0}\}  \vDash (\alpha_{8}+1) + \alpha_{9}i_{0} \leq \alpha_{38} + \alpha_{39}i_{0}\\ \{i_{0}\};\{1 \leq \alpha_{0} + \alpha_{1}i_{0}\}  \vDash \alpha_{10} + \alpha_{11}i_{0} \leq \alpha_{22} + \alpha_{23}i_{0}\\ \{i_{0}\};\{1 \leq \alpha_{0} + \alpha_{1}i_{0}\}  \vDash \alpha_{26} + \alpha_{27}i_{0} \leq \alpha_{28} + \alpha_{29}i_{0}\\ \{i_{0}\};\{1 \leq \alpha_{0} + \alpha_{1}i_{0}\}  \vDash (\alpha_{14}+(\alpha_{17}\alpha_{22}+\alpha_{16})) + (\alpha_{15}+\alpha_{17}\alpha_{23})i_{0} \leq \alpha_{20} + \alpha_{21}i_{0}\\ \{i_{0}\};\{1 \leq \alpha_{0} + \alpha_{1}i_{0}\}  \vDash (\alpha_{28}+\alpha_{30}) + (\alpha_{29}+\alpha_{31})i_{0} \leq \alpha_{32} + \alpha_{33}i_{0}\\ \{i_{0}\};\{1 \leq \alpha_{0} + \alpha_{1}i_{0}\}  \vDash (\alpha_{32}+\alpha_{34}) + (\alpha_{33}+\alpha_{35})i_{0} \leq \alpha_{36} + \alpha_{37}i_{0}
\end{align*}
}

We find the following solution to the reduced constraint satisfaction problem (written as pairs of coefficient variables and natural numbers)\\

$\splitatcommas{(\alpha_{0},0),(\alpha_{1},1),(\alpha_{2},0),(\alpha_{3},1),(\alpha_{4},0),(\alpha_{5},1),(\alpha_{6},0),(\alpha_{7},0),(\alpha_{8},0),(\alpha_{9},0),(\alpha_{10},-1),(\alpha_{11},1),(\alpha_{12},-2),(\alpha_{13},2),(\alpha_{14},0),(\alpha_{15},0),(\alpha_{16},0),(\alpha_{17},2),(\alpha_{18},0),(\alpha_{19},1),(\alpha_{20},0),(\alpha_{21},0),(\alpha_{22},-2),(\alpha_{23},2),(\alpha_{24},-2),(\alpha_{25},2),(\alpha_{26},1),(\alpha_{27},-1),(\alpha_{28},0),(\alpha_{29},0),(\alpha_{30},-2),(\alpha_{31},2),(\alpha_{32},-1),(\alpha_{33},1),(\alpha_{34},0),(\alpha_{35},0),(\alpha_{36},0),(\alpha_{37},0),(\alpha_{38},0),(\alpha_{39},1),(\alpha_{40},0),(\alpha_{41},0),(\alpha_{42},1),(\alpha_{43},0),(\alpha_{44},1),(\alpha_{45},10),(\alpha_{46},0),(\alpha_{47},0),(\alpha_{48},0),(\alpha_{49},2),(\alpha_{50},0),(\alpha_{51},1),(\alpha_{52},0),(\alpha_{53},0),(\alpha_{54},10),(\alpha_{55},10),(\alpha_{56},0),(\alpha_{57},0),(\alpha_{58},1),(\alpha_{59},0),(\alpha_{60},1),(\alpha_{61},10),(\alpha_{62},10),(\alpha_{63},10),(\alpha_{64},0),(\alpha_{65},10),(\alpha_{66},10)}$

%\input{sections/turingcompleteness}
%\appendix
%\chapter{Semantic equivalence}\label{app:languageequiv}
\setcounter{theorem}{1}

\begin{lemma}
Let $P$ and $Q$ be processes such that $P \equiv Q$, then also $P \succeq_K Q$ and $Q \succeq_K P$.\\

where $\succeq_{K}$ is the structural preorder from Kobayashi \cite{Kobayashi2000}.
\begin{proof} By induction on the rules defining $\equiv$. We only consider the rules that are unique to $\equiv$.
\begin{description}
    \item[$\runa{SC-par}$] By rule $\runa{SC-par}$ we have that $P \equiv P'$ such that $P \mid Q \equiv P' \mid Q$. By induction $P \succeq_K P'$ and $P' \succeq_K P$. Then it follows from $\runa{SP-par}$ that $P \mid Q \succeq_K P' \mid Q$ and $P' \mid Q \succeq_k P \mid Q$.
    
    \item[$\runa{SC-res}$] By rule $\runa{SC-res}$ we have that $P \equiv Q$ such that $\newvar{x}{P} \equiv \newvar{x}{Q}$. By induction $P \succeq_K Q$ and $Q \succeq_K P$. Then it follows from $\runa{SP-res}$ that $\newvar{x}{P} \succeq_K \newvar{x}{Q}$ and $\newvar{x}{Q} \succeq_k \newvar{x}{P}$.
\end{description}

\end{proof}%\label{lemma:SCImpliesSP}
\end{lemma}

\begin{lemma}[Semantic equivalence]
Let $P$ and $P'$ be processes. We have that if
\begin{enumerate}
    \item $P \longrightarrow Q$ then $P \longrightarrow_{K} Q$
    \item $P \longrightarrow_{K} Q$ and $P' \succeq_K P$ then $P' \longrightarrow R$ and $R \succeq_{K} Q$
\end{enumerate}
where $\longrightarrow_{K}$ and $\succeq_{K}$ are the reduction relation and structural preorder from Kobayashi \cite{Kobayashi2000}, respectively.
\begin{proof} By induction on the rules defining $\longrightarrow$ and $\longrightarrow_{K}$. We consider the two cases separately, and we only show the clauses for non-trivial rules.
\begin{enumerate}
    \item 
    \begin{description}
        \item[\runa{R-rep}] Assume that $\bang\inputch{a}{\widetilde{v}}{}{P} \mid \outputch{a}{\widetilde{e}}{}{Q} \longrightarrow\; \bang\inputch{a}{\widetilde{v}}{}{P} \mid P[\widetilde{v}\mapsto\widetilde{e}] \mid Q$. We have that $\bang\inputch{a}{\widetilde{v}}{}{P} \mid \outputch{a}{\widetilde{e}}{}{Q}\succeq_K\; \bang\inputch{a}{\widetilde{v}}{}{P} \mid \inputch{a}{\widetilde{v}}{}{P} \mid \outputch{a}{\widetilde{e}}{}{Q}$, and so it follows from application of $\runa{K-struct}$ and $\runa{K-comm}$ that $\bang\inputch{a}{\widetilde{v}}{}{P} \mid \outputch{a}{\widetilde{e}}{}{Q} \longrightarrow_K\; \bang\inputch{a}{\widetilde{v}}{}{P} \mid P[\widetilde{v}\mapsto\widetilde{e}] \mid Q$.
        %
        %\item[\runa{R-comm}] Assume that $\inputch{a}{\widetilde{v}}{}{P} \mid \outputch{a}{\widetilde{e}}{}{Q} \longrightarrow P[\widetilde{v}\mapsto\widetilde{e}] \mid Q$. It trivially follows from application of $\runa{K-comm}$ that $\inputch{a}{\widetilde{v}}{}{P} \mid \outputch{a}{\widetilde{e}}{}{Q} \longrightarrow_K P[\widetilde{v}\mapsto\widetilde{e}] \mid Q$, as $\succeq_K$ is reflexive.
        %
        %\item[\runa{R-zero}] Assume that $\match{e}{P}{x}{Q} \longrightarrow P$
        %
        %\item[\runa{R-succ}]
        %
        %\item[\runa{R-par}]
        %
        %\item[\runa{R-res}]
        
        \item[\runa{R-struct}] Assume that $P \longrightarrow Q$ by rule $\runa{R-struct}$. Then we have that $P \equiv P_1$, $P_1 \longrightarrow P_1'$ and $P_1' \equiv Q$. By Lemma \ref{lemma:SCImpliesSP} it follows that $P \succeq_K P_1$ and $P_1' \succeq_K Q$. By induction $P' \longrightarrow_K Q'$, and so by rule $\runa{K-struct}$, $P \longrightarrow_K Q$.
            %\begin{description}
            %    \item[\runa{SC-par}]
            %    
            %    \item[\runa{SC-res}]
            %\end{description}
        
        
    \end{description} 
    
    \item
    \begin{description}
        %\item[\runa{K-comm}]
        %    
        %\item[\runa{K-zero}]
        %
        %\item[\runa{K-succ}]
        %
        %\item[\runa{K-par}]
        %
        %\item[\runa{K-res-1}]
        %
        %\item[\runa{K-res-2}]
        %
        \item[\runa{K-struct}] Assume that $P \longrightarrow_K Q$ by rule $\runa{K-struct}$ and $P' \succeq_K P$. Then we have that $P \succeq_K P_2$, $P_2 \longrightarrow_K P_2'$ and $P_2' \succeq_K Q$. We show that $P'$ can match this reduction by rule $\runa{R-struct}$ such that $P' \longrightarrow R$ and $R \succeq_K Q$. Rule $\runa{R-struct}$ implies $P' \equiv P_1$ such that $P_1 \longrightarrow P_1'$ and $P_1' \equiv R$. By Lemma \ref{lemma:SCImpliesSP}, $P' \equiv P_1$ implies $P' \succeq_K P_1$ and $P_1 \succeq_K P'$. Thus by the assumption, it must be that $P_1 \succeq_K P$, and by transitivity $P \succeq_K P_2$ implies $P_1 \succeq_K P_2$. By induction $P_1 \succeq_K P_2$ and $P_2 \longrightarrow_K P_2'$ implies $P_1 \longrightarrow P_1'$ and $P_1' \succeq_K P_2'$. Now, it suffices to show that $P_1' \equiv R$ and $P_2' \succeq_K Q$ implies $R \succeq_K Q$. By Lemma \ref{lemma:SCImpliesSP}, $P_1' \equiv R$ implies $P_1' \succeq_K R$ and $R \succeq_K P_1'$, and so it follows from $P_1' \succeq_K P_2'$ that $R \succeq_K P_2'$. By transitivity $P_2' \succeq_K Q$ implies $R \succeq_K Q$, concluding the proof.
        
        
        %By the definition of $\succeq_K$, we have that $P \succeq_K P'' %\mid P_1 \mid \dots \mid P_n$ such that $P' \succeq P''$ and %$P_1,\dots,P_n$ are copies of replicated inputs in $P'$.
        %    \begin{description}
        %        \item[\runa{SP-rep-1}]
        %        
        %        \item[\runa{SP-par}]
        %        
        %        \item[\runa{SP-res}]
        %        
        %        \item[\runa{SP-rep-2}]
        %    \end{description}
            
    \end{description}
\end{enumerate}
\end{proof}
\end{lemma}
%
%%
%
% \begin{lemma}
% Let $P$ and $Q$ be processes such that $P \succeq_{K} Q$. We have that if
% \begin{enumerate}
%     \item $P \longrightarrow R$ then $Q \longrightarrow C$ and $R \succeq_K C$
%     \item $Q \longrightarrow R$ then $P \longrightarrow C$ and $C \succeq_K R$
%     \item $P \longrightarrow_{K} R$ then $Q \longrightarrow_K C$ and $R \succeq_K C$ or $C \succeq_K R$
%     \item $Q \longrightarrow_{K} R$ then $P \longrightarrow_{K} R$
% \end{enumerate}
% where $\longrightarrow_{K}$ and $\succeq_{K}$ is the reduction relation and structural preorder from Kobayashi \cite{Kobayashi2000}, respectively.
% \begin{proof}
% Proof by induction on the rules defining $\longrightarrow$ and $\longrightarrow_{K}$. We consider each case separately.
% \begin{enumerate}
%     \item
%         \begin{description}
%             \item[\runa{R-rep}]
            
%             \item[\runa{R-comm}]
            
%             \item[\runa{R-zero}]
            
%             \item[\runa{R-succ}]
            
%             \item[\runa{R-par}]
            
%             \item[\runa{R-res}]
            
%             \item[\runa{R-struct}]
            
%         \end{description}
        
%     \item 
%         \begin{description}
%             \item[\runa{R-rep}]
            
%             \item[\runa{R-comm}]
            
%             \item[\runa{R-zero}]
            
%             \item[\runa{R-succ}]
            
%             \item[\runa{R-par}]
            
%             \item[\runa{R-res}]
            
%             \item[\runa{R-struct}]
            
%         \end{description}
%     \item 
%         \begin{description}
%             \item[\runa{K-comm}]
            
%             \item[\runa{K-zero}]
            
%             \item[\runa{K-succ}]
            
%             \item[\runa{K-par}]
            
%             \item[\runa{K-res-1}]
            
%             \item[\runa{K-res-2}]
            
%             \item[\runa{K-struct}]
            
%         \end{description}
        
%     \item As $P \succeq_K Q$, this trivially holds by applying rule $\runa{K-struct}$ with a corresponding reduction, such that $P \longrightarrow_K R$.
% \end{enumerate}
% \end{proof}
% \end{lemma}
%\chapter{Session type soundness}\label{app:dasetallsoundness}
\setcounter{theorem}{11}
%

\begin{lemma}
Let $P$ be an arbitrary process such that $b$ is not free in $P$. 
\begin{enumerate}
\item If $\Delta,a:A\vdash P :: c\!:\!C$ then $\Delta,b:A\vdash P[a\mapsto b] :: c\!:\!C$.

\item If $\Delta\vdash P :: a\!:\!A$ then $\Delta\vdash P[a\mapsto b] :: b\!:\!A$.
\end{enumerate}
\begin{proof}
By induction on the type rules
\begin{description}
\item[$\runa{TS-$\mathbf{1}$L}$] We have that $\Delta,a:\mathbf{1} \vdash P :: c\!:\!C$ because $\Delta \vdash P :: c\!:\!C$. We consider the cases separately
\begin{enumerate}
    \item Let $d\in\text{dom}(\Delta)$. Then there exists $\Delta'$ and $D$ such that $\Delta=\Delta',d:D$. By induction we then have that $\Delta',b:D\vdash P[d\mapsto b] :: c\!:\!C$, and by application of $\runa{TS-$\mathbf{1}$L}$ we obtain $\Delta',b:D,a:\mathbf{1}\vdash P[d\mapsto b] :: c\!:\!C$ as required. We obtain the special case $\Delta,b:\mathbf{1}\vdash P[a\mapsto b] :: c\!:\!C$ directly from application of \runa{TS-$\mathbf{1}$L}.
    
    \item By induction we have that $\Delta\vdash P[c\mapsto b] :: b\!:\!C$, then by application of $\runa{TS-$\mathbf{1}$L}$ we obtain $\Delta,a:\mathbf{1}\vdash P[c\mapsto b] :: b\!:\!C$, as required.
\end{enumerate}

% We consider (1) first. Then for $P[d\mapsto b]$ we either have $d=a$ or $\Delta = \Delta',d:D$. In the first case we obtain $\Delta,b:\mathbf{1}\vdash P[a\mapsto b] :: c\!:\!C$ directly from \runa{TS-$\mathbf{1}$L} as $P[a\mapsto b]=P$ and $P$ can consume a session of type $\mathbf{1}$ for any name. For the second case we have by induction that $\Delta',b:D\vdash P[d\mapsto b] :: c\!:\!C$, and from $\runa{TS-$\mathbf{1}$L}$ we obtain $\Delta',b:D,a:\mathbf{1}\vdash P[d\mapsto b] :: c\!:\!C$.\\

% We then consider (2). For $P[d\mapsto b]$ we have $d=c$. Then we have by induction that $\Delta\vdash P[c\mapsto b] :: b\!:\!C$, and it follows from $\runa{TS-$\mathbf{1}$L}$ that also $\Delta,a:\mathbf{1}\vdash P[c\mapsto b] :: b\!:\!C$.

\item[$\runa{TS-$\mathbf{1}$R}$] We have that $\cdot\vdash \mathbf{0} :: a\!:\!\mathbf{1}$ and $\mathbf{0}[a\mapsto b]=\mathbf{0}$. (1) does not apply, as $\cdot$ is the empty type context. For (2) we obtain $\cdot \vdash \mathbf{0}[a\mapsto b] :: b\!:\!\mathbf{1}$ directly by application of $\runa{TS-$\mathbf{1}$R}$, as $\mathbf{0}$ can provide a session on any name.

\item[$\runa{TS-$\otimes$L}$] We have that $\Delta,a:A\otimes B \vdash \inputch{a}{v}{}{P'} :: c\!:\!C$ because $\Delta,v:A,a:B\vdash P' :: c\!:\!C$. We consider the cases separately
\begin{enumerate}
    \item We either replace $a$ or some $d\in\text{dom}(\Delta)$. We consider them separately
    \begin{itemize}
        \item We have that $(\inputch{a}{v}{}{P'})[a\mapsto b]=\inputch{b}{v}{}{P'}[a\mapsto b]$. By induction $\Delta,v:A,b:B\vdash P[a\mapsto b] :: c\!:\!C$ and by application of $\runa{TS-$\otimes$L}$ we obtain $\Delta,b:A\otimes B\vdash \inputch{b}{v}{}{P[a\mapsto b]} :: c\!:\!C$ as required.
        
        \item There exists $\Delta'$ and $D$ such that $\Delta=\Delta',d:D$ and we have that $(\inputch{a}{v}{}{P'})[d\mapsto b]=\inputch{a}{v}{}{P'}[d\mapsto b]$. By induction $\Delta',b:D,v:a,a:B\vdash P[d\mapsto b] :: c\!:\!C$ and by application of $\runa{TS-$\otimes$L}$ we obtain $\Delta',b:D,a:A\otimes B\vdash \inputch{a}{v}{}{P[d\mapsto b]} :: c\!:\!C$ as required. 
    \end{itemize}
    
    \item We have that $(\inputch{a}{v}{}{P'})[c\mapsto b]=\inputch{a}{v}{}{P'[c\mapsto b]}$. By induction we have that $\Delta,v:A,a:B\vdash P'[c\mapsto b] :: b\!:\!C$, and by application of $\runa{TS-$\otimes$L}$ we obtain $\Delta,a:A\otimes B \vdash \inputch{a}{v}{}{P'[c\mapsto b]} :: b\!:\!C$ as required.
\end{enumerate}

\item[$\runa{TS-$\otimes$R}$] We have that $\Delta,v:A\vdash \outputch{a}{v}{}{P'} :: a\!:\!A\otimes B$ because $\Delta\vdash P' :: a\!:\!B$. We consider the cases separately
\begin{enumerate}
    \item We replace some $d\in\text{dom}(\Delta)$ or $v$. We consider them separately
    \begin{itemize}
        \item There exists $\Delta'$ and $D$ such that $\Delta=\Delta',d:D$ and we have that $(\outputch{a}{v}{}{P'})[d\mapsto b]=\outputch{a}{v}{}{P'}[d\mapsto b]$. By induction $\Delta',b:D\vdash P[d\mapsto b] :: a\!:\!B$ and by application of $\runa{TS-$\otimes$R}$ we obtain $\Delta',b:D,v:A\vdash \outputch{a}{v}{}{P[d\mapsto b]} :: a\!:\!A\otimes B$ as required. 
        
        \item We have that $(\outputch{a}{v}{}{P'})[v\mapsto b]=\outputch{a}{b}{}{P'}$. From $\Delta\vdash P' :: a\!:\!B$ we then obtain $\Delta,b:A\vdash \outputch{a}{b}{}{P'} :: a\!:\!A\otimes B$ directly by application of $\runa{TS-$\otimes$R}$.
    \end{itemize}
    
    \item We have that $(\outputch{a}{v}{}{P'})[a\mapsto b]=\outputch{b}{v}{}{P'[a\mapsto b]}$. By induction we have that $\Delta\vdash P'[a\mapsto b] :: b\!:\!B$, and by application of $\runa{TS-$\otimes$R}$ we obtain $\Delta,v:A \vdash \outputch{b}{v}{}{P'[a\mapsto b]} :: b\!:\!A\otimes B$ as required.
\end{enumerate}

%We have that $\Delta,v:A \vdash \outputch{a}{v}{}{P'} :: a\!:\!A\otimes B$ and $\Delta\vdash P' :: a\!:\!B$. The first part of the lemma applies to $(\outputch{a}{v}{}{P'})[v\mapsto b]=\outputch{a}{b}{}{P'[v\mapsto b]}$ and to $(\outputch{a}{v}{}{P'})[d\mapsto b]=\outputch{a}{v}{}{P'[d\mapsto b]}$ when $\Delta=\Delta',d:D$. For the first case we obtain $\Delta,b:A\vdash \outputch{a}{b}{}{P'[v\mapsto b]} :: a\!:\!A\otimes B$ directly from $\runa{TS-$\otimes$R}$, as it must be that $P'[v\mapsto b]=P'$ since $\Delta\vdash P' :: a\!:\!B$. In the second case we have by induction that $\Delta',b:D\vdash P'[d\mapsto b] :: a\!:\!B$ and from $\runa{TS-$\otimes$R}$ we obtain $\Delta',b:D,v:A \vdash \outputch{a}{v}{}{P'[d\mapsto b]} :: a\!:\!A\otimes B$. The second part of the lemma applies to $(\outputch{a}{v}{}{P'})[a\mapsto b]=\outputch{b}{v}{}{P'[a\mapsto b]}$. Then we have by induction that $\Delta\vdash P'[a\mapsto b] :: b\!:\!B$, and it follows from $\runa{TS-$\otimes$R}$ that also $\Delta,v:A \vdash \outputch{b}{v}{}{P'[a\mapsto b]} :: b\!:\!A\otimes B$.

\item[$\runa{TS-$\multimap$L}$] We have that $\Delta,a:A\multimap B,v:a \vdash \outputch{a}{v}{}{P'} :: c\!:\!C$ because $\Delta,a:B\vdash P' :: c\!:\!C$. We consider the cases separately
\begin{enumerate}
    \item We either replace $a$, some $d\in\text{dom}(\Delta)$ or $v$. We consider them separately
    \begin{itemize}
        \item We have that $(\outputch{a}{v}{}{P'})[a\mapsto b]=\outputch{b}{v}{}{P'}[a\mapsto b]$. By induction $\Delta,b:B\vdash P[a\mapsto b] :: c\!:\!C$ and by application of $\runa{TS-$\multimap$L}$ we obtain $\Delta,b:A\multimap B,v:A\vdash \outputch{b}{v}{}{P[a\mapsto b]} :: c\!:\!C$ as required.
        
        \item There exists $\Delta'$ and $D$ such that $\Delta=\Delta',d:D$ and we have that $(\outputch{a}{v}{}{P'})[d\mapsto b]=\outputch{a}{v}{}{P'}[d\mapsto b]$. By induction $\Delta',b:D,a:B\vdash P[d\mapsto b] :: c\!:\!C$ and by application of $\runa{TS-$\multimap$L}$ we obtain $\Delta',b:D,a:A\multimap B,v:A\vdash \outputch{a}{v}{}{P[d\mapsto b]} :: c\!:\!C$ as required. 
        
        \item We have that $(\outputch{a}{v}{}{P'})[v\mapsto b]=\outputch{a}{b}{}{P'}$. From $\Delta,a:B\vdash P' :: c\!:\!C$ we then obtain $\Delta,a:A\multimap B,b:A\vdash \outputch{a}{b}{}{P'} :: c\!:\!C$ directly by application of $\runa{TS-$\multimap$L}$.
    \end{itemize}
    
    \item We have that $(\outputch{a}{v}{}{P'})[c\mapsto b]=\outputch{a}{v}{}{P'[c\mapsto b]}$. By induction we have that $\Delta,a:B\vdash P'[c\mapsto b] :: b\!:\!C$, and by application of $\runa{TS-$\multimap$L}$ we obtain $\Delta,a:A\multimap B,v:A \vdash \outputch{a}{v}{}{P'[c\mapsto b]} :: b\!:\!C$ as required.
\end{enumerate}

%We have that $\Delta,a:A\multimap B,v:A \vdash \outputch{a}{v}{}{P'} :: c\!:\!C$ and $\Delta,a:B\vdash P' :: c\!:\!C$. The first part of the lemma applies to $(\outputch{a}{v}{}{P'})[v\mapsto b]=\outputch{a}{b}{}{P'[v\mapsto b]}$, $(\outputch{a}{v}{}{P'})[a\mapsto b]=\outputch{b}{v}{}{P'[a\mapsto b]}$ and to $(\outputch{a}{v}{}{P'})[d\mapsto b]=\outputch{a}{v}{}{P'[d\mapsto b]}$ when $\Delta=\Delta',d:D$. For the first case we obtain $\Delta,a:A\multimap B, b:A\vdash \outputch{a}{b}{}{P'[v\mapsto b]} :: c\!:\!C$ directly from $\runa{TS-$\multimap$L}$, as it must be that $P'[v\mapsto b]=P'$ since $\Delta,a:B\vdash P' :: c\!:\!C$. In the second case we have by induction that $\Delta,b:B\vdash P'[a\mapsto b] :: c\!:\!C$ and from $\runa{TS-$\multimap$L}$ we obtain $\Delta,b:A\multimap B,v:A \vdash \outputch{b}{v}{}{P'[a\mapsto b]} :: c\!:\!C$. For the third case we have by induction $\Delta',b:D,a:B\vdash P'[d\mapsto b] :: c\!:\!C$ and from $\runa{TS-$\multimap$L}$ we obtain $\Delta',b:D,a:A\multimap B,v:A \vdash \outputch{a}{v}{}{P'[d\mapsto b]} :: c\!:\!C$. The second part of the lemma applies to $(\outputch{a}{v}{}{P'})[c\mapsto b]=\outputch{a}{v}{}{P'[c\mapsto b]}$. Then we have by induction that $\Delta,a:B\vdash P'[c\mapsto b] :: b\!:\!C$, and it follows from $\runa{TS-$\multimap$L}$ that also $\Delta,a:A\otimes B,v:A \vdash \outputch{a}{v}{}{P'[c\mapsto b]} :: b\!:\!C$.

\item[$\runa{TS-$\multimap$R}$] We have that $\Delta\vdash \inputch{a}{v}{}{P'} :: a\!:\!A\multimap B$ because $\Delta,v:A\vdash P' :: a\!:\!B$. We consider the cases separately
\begin{enumerate}
    \item We replace some $d\in\text{dom}(\Delta)$. There exists $\Delta'$ and $D$ such that $\Delta=\Delta',d:D$ and we have that $(\inputch{a}{v}{}{P'})[d\mapsto b]=\inputch{a}{v}{}{P'}[d\mapsto b]$. By induction $\Delta',b:D,v:A\vdash P[d\mapsto b] :: a\!:\!B$ and by application of $\runa{TS-$\multimap$R}$ we obtain $\Delta',b:D\vdash \inputch{a}{v}{}{P[d\mapsto b]} :: a\!:\!A\otimes B$ as required. 
    
    \item We have that $(\inputch{a}{v}{}{P'})[a\mapsto b]=\inputch{b}{v}{}{P'[a\mapsto b]}$. By induction we have that $\Delta,v:A\vdash P'[a\mapsto b] :: b\!:\!B$, and by application of $\runa{TS-$\multimap$R}$ we obtain $\Delta\vdash \inputch{b}{v}{}{P'[a\mapsto b]} :: b\!:\!A\otimes B$ as required.
\end{enumerate}

%We have that $\Delta\vdash \inputch{a}{v}{}{P'} :: a\!:\!A\multimap B$ and $\Delta,v:A\vdash P' :: a\!:\!B$. The first part of the lemma applies to $(\inputch{a}{v}{}{P'})[d\mapsto b]=\inputch{a}{v}{}{P'[d\mapsto b]}$ when $\Delta=\Delta',d:D$. We have by induction that $\Delta',b:D,v:A\vdash P'[d\mapsto b] :: a\!:\!B$ and from $\runa{TS-$\multimap$R}$ we obtain $\Delta',b:D\vdash \inputch{a}{v}{}{P'[d\mapsto b]} :: a\!:\!A\multimap B$. The second part of the lemma applies to $(\inputch{a}{v}{}{P'})[a\mapsto b]=\inputch{b}{v}{}{P'[a\mapsto b]}$. Then we have by induction that $\Delta,v:A\vdash P'[a\mapsto b] :: b\!:\!B$, and it follows from $\runa{TS-$\multimap$R}$ that also $\Delta \vdash \inputch{b}{v}{}{P'[a\mapsto b]} :: b\!:\!A\otimes B$.

\item[$\runa{TS-cut}$] We have that $\Delta_1,\Delta_2\vdash \newvar{a}{(P' \mid P'') :: c\!:\!C}$ because $\Delta_1\vdash P' :: a\!:\!A$ and $\Delta_2,a:A\vdash P'' :: c\!:\!C$. Then $(\newvar{a}{(P' \mid P'')})[d\mapsto b]=\newvar{a}{(P'[d\mapsto b] \mid P''[d\mapsto b])}$ and we can assume that $d\neq a$, as $\Delta_1,\Delta_2\vdash \newvar{a}{(P' \mid P'') :: c\!:\!C}$ does not hold when $a\in \text{dom}(\Delta_1,\Delta_2)$ or $a=c$. We consider the cases separately
\begin{enumerate}
    \item We replace some $d\in\text{dom}(\Delta_1)$ or $d\in\text{dom}(\Delta_2)$, such that either $\Delta_1=\Delta_1',d:D$ or $\Delta_2=\Delta_2',d:D$, and so by induction we have either $\Delta_1',b:D\vdash P'[d\mapsto b] :: a\!:\!A$ or $\Delta_2',a:A,b:D\vdash P''[d\mapsto b] :: c\!:\!C$. Thus we obtain either $\Delta_1',b:D,\Delta_2\vdash \newvar{a}{(P'[d\mapsto b] \mid P''[d\mapsto b])} :: c\!:\!C$ or $\Delta_1,\Delta_2',b:D\vdash \newvar{a}{(P'[d\mapsto b] \mid P''[d\mapsto b])} :: c\!:\!C$ by application of $\runa{TS-cut}$ as required.
    
    \item We need only consider $P''$ as $d\neq a$, and so if $d=c$ we have by induction that $\Delta_2,a:A\vdash P''[d\mapsto b] :: b\!:\!C$. Then we obtain $\Delta_1,\Delta_2\vdash \newvar{a}{(P'[c\mapsto b] \mid P''[c\mapsto b])} :: b\!:\!C$ directly by application of $\runa{TS-cut}$ as required. 
\end{enumerate}

%The first part of the lemma applies when either $\Delta_1=\Delta_1',d:D$ or $\Delta_2=\Delta_2',d:D$, and so by induction we have either $\Delta_1',b:D\vdash P'[d\mapsto b] :: a\!:\!A$ or $\Delta_2',a:A,b:D\vdash P''[d\mapsto b] :: c\!:\!C$. Thus we obtain either $\Delta_1',b:D,\Delta_2\vdash \newvar{a}{(P'[d\mapsto b] \mid P''[d\mapsto b])} :: c\!:\!C$ or $\Delta_1,\Delta_2',b:D\vdash \newvar{a}{(P'[d\mapsto b] \mid P''[d\mapsto b])} :: c\!:\!C$ by $\runa{TS-cut}$.

%The second part of the lemma can only apply to $P''$ as $d\neq a$, and so if $d=c$ we have by induction that $\Delta_2,a:A\vdash P''[d\mapsto b] :: b\!:\!C$. Then we obtain $\Delta_1,\Delta_2\vdash \newvar{a}{(P'[c\mapsto b] \mid P''[c\mapsto b])} :: b\!:\!C$ directly from $\runa{TS-cut}$. 

\item[$\runa{TS-id}$] We have that $b:A\vdash a \leftarrow b :: a\!:\!A$, $(a \leftarrow b)[a\mapsto c]= c \leftarrow b$ and $(a\leftarrow b)[b\mapsto c]=a\leftarrow c$. We obtain $c:A\vdash a\leftarrow c :: a\!:\!A$ and $b:A\vdash c \leftarrow b :: c\!:\!A$ directly by application of $\runa{TS-id}$ as required.

\item[$\runa{TS-$\oplus$L}$] We have that $\Delta,a : \oplus\{l:A_l\}_{l\in L}\vdash a.\texttt{case}\{l\Rightarrow P_l\}_{l\in L} :: c\!:\!C$ because for $l \in L$ we also have $\Delta,a:A_l \vdash P_l :: c\!:\!C$. We consider the cases separately
\begin{enumerate}
    \item We either replace $a$ or some $d\in\text{dom}(\Delta)$. We consider them separately
    \begin{itemize}
        \item We have that $(a.\texttt{case}\{l\Rightarrow P_l\}_{l\in L})[a\mapsto b]=b.\texttt{case}\{l\Rightarrow P_l[a\mapsto b]\}_{l\in L}$. Then for $l\in L$ we have by induction that $\Delta,b:A_l\vdash P_l[a\mapsto b] :: c\!:\!C$, and by application of $\runa{TS-$\oplus$L}$ we obtain $\Delta,b:\oplus\{l:A_l\}_{l\in L}\vdash b.\texttt{case}\{l\Rightarrow P_l[a\mapsto b]\}_{l\in L} :: c\!:\!C$ as required.
        
        \item There exists $\Delta'$ and $D$ such that $\Delta=\Delta',d:D$ and we have that $(a.\texttt{case}\{l\Rightarrow P_l\}_{l\in L})[d\mapsto b]=a.\texttt{case}\{l\Rightarrow P_l[d\mapsto b]\}_{l\in L}$. Then for $l\in L$ we have by induction that $\Delta',b:D,a:A_l\vdash P_l[d\mapsto b] :: c\!:\!C$, and by application of $\runa{TS-$\oplus$L}$ we obtain $\Delta',b:D,a:\oplus\{l:A_l\}_{l\in L}\vdash a.\texttt{case}\{l\Rightarrow P_l[d\mapsto b]\}_{l\in L} :: c\!:\!C$ as required.
    \end{itemize}
    
    \item We have that $(a.\texttt{case}\{l\Rightarrow P_l\}_{l\in L})[c\mapsto b]=a.\texttt{case}\{l\Rightarrow P_l[c\mapsto b]\}_{l\in L}$. Then for $l\in L$ we have by induction that $\Delta',b:D,a:A_l\vdash P_l[d\mapsto b] :: c\!:\!C$, and by application of $\runa{TS-$\oplus$L}$ we obtain $\Delta',b:D,a:\oplus\{l:A_l\}_{l\in L}\vdash a.\texttt{case}\{l\Rightarrow P_l[d\mapsto b]\}_{l\in L} :: c\!:\!C$ as required.
\end{enumerate}

% The first part of the lemma applies to $(a.\texttt{case}\{l\Rightarrow P_l\}_{l\in L})[a\mapsto b]=b.\texttt{case}\{l\Rightarrow P_l[a\mapsto b]\}_{l\in L}$ and to $(a.\texttt{case}\{l\Rightarrow P_l\}_{l\in L})[d\mapsto b]=a.\texttt{case}\{l\Rightarrow P_l[d\mapsto b]\}_{l\in L}$ when $\Delta=\Delta',d:D$. In the first case we have by induction for $l\in L$ that $\Delta,b:A_l\vdash P_l[a\mapsto b] :: c\!:\!C$, and so by $\runa{TS-$\oplus$L}$ we obtain $\Delta,b:\oplus\{l:A_l\}_{l\in L}\vdash b.\texttt{case}\{l\Rightarrow P_l[a\mapsto b]\}_{l\in L} :: c\!:\!C$.
% For the second case we have by induction for $l\in L$ that $\Delta',b:D,a:A_l\vdash P_l[d\mapsto b] :: c\!:\!C$, and so by $\runa{TS-$\oplus$L}$ we obtain $\Delta',b:D,a:\oplus\{l:A_l\}_{l\in L}\vdash a.\texttt{case}\{l\Rightarrow P_l[d\mapsto b]\}_{l\in L} :: c\!:\!C$.
% The second part of the lemma applies to $(a.\texttt{case}\{l\Rightarrow P_l\}_{l\in L})[c\mapsto b]=a.\texttt{case}\{l\Rightarrow P_l[c\mapsto b]\}_{l\in L}$. We have by induction for $l\in L$ that $\Delta,a:A_l\vdash P_l[c\mapsto b] :: b\!:\!C$, and so by $\runa{TS-$\oplus$L}$ we obtain $\Delta,a:\oplus\{l:A_l\}_{l\in L}\vdash a.\texttt{case}\{l\Rightarrow P_l[c\mapsto b]\}_{l\in L} :: b\!:\!C$.

\item[$\runa{TS-$\oplus$R}$] We have that $\Delta\vdash a.k; P' :: a\!:\!\oplus\{l:A_l\}_{l\in L}$ because $k \in L$ and $\Delta \vdash P' :: a\!:\!A_k$. We consider the cases separately
\begin{enumerate}
    \item We replace some $d\in\text{dom}(\Delta)$. There exists $\Delta'$ and $D$ such that $\Delta=\Delta',d:D$ and we have that $(a.k; P')[d\mapsto b]=a.k; P'[d\mapsto b]$. Then we have by induction that $\Delta',b:D\vdash P'[d\mapsto b] :: a\!:\!a_k$, and by application of $\runa{TS-$\oplus$R}$ we obtain $\Delta',b:D\vdash a.k; P'[d\mapsto b] :: a\!:\!\oplus\{l:A_l\}_{l\in L}$ as required.
    
    \item We have that $(a.k; P')[a\mapsto b]=b.k; P'[a\mapsto b]$. Then we have by induction that $\Delta\vdash P'[a\mapsto b] :: b\!:\!A_k$, and by application of $\runa{TS-$\oplus$L}$ we obtain $\Delta\vdash b.k; P'[a\mapsto b] :: b\!:\!\oplus\{l:A_l\}_{l\in L}$ as required.
\end{enumerate}

%We have that $\Delta\vdash a.k; P' :: a\!:\!\oplus\{l:A_l\}_{l\in L}$ such that $k \in L$ and $\Delta \vdash P' :: a\!:\!A_k$. The first part of the lemma applies to $(a.k; P')[d\mapsto b]=a.k; P'[d\mapsto b]$ when $\Delta=\Delta',d:D$. We have by induction that $\Delta',b:D\vdash P'[d\mapsto b] :: a\!:\!a_k$, and so by $\runa{TS-$\oplus$R}$ we obtain $\Delta',b:D\vdash a.k; P'[d\mapsto b] :: a\!:\!\oplus\{l:A_l\}_{l\in L}$. The second part of the lemma applies to $(a.k; P')[a\mapsto b]=b.k; P'[a\mapsto b]$. We have by induction that $\Delta\vdash P'[a\mapsto b] :: b\!:\!A_k$, and so by $\runa{TS-$\oplus$L}$ we obtain $\Delta\vdash b.k; P'[a\mapsto b] :: b\!:\!\oplus\{l:A_l\}_{l\in L}$.

\item[$\runa{TS-$\&$L}$] We have that $\Delta,a : \&\{l:A_l\}_{l\in L}\vdash a.k; P' :: c\!:\!C$ because $k \in L$ and $\Delta,a:A_k \vdash P' :: c\!:\!C$. We consider the cases separately
\begin{enumerate}
    \item We either replace $a$ or some $d\in\text{dom}(\Delta)$. We consider them separately
    \begin{itemize}
        \item We have that $(a.k; P')[a\mapsto b]=b.k; P'[a\mapsto b]$. Then we have by induction that $\Delta,b:A_k\vdash P'[a\mapsto b] :: c\!:\!C$, and by application of $\runa{TS-$\&$L}$ we obtain $\Delta,b:\&\{l:A_l\}_{l\in L}\vdash b.k; P'[a\mapsto b] :: c\!:\!C$ as required.
        
        \item There exists $\Delta'$ and $D$ such that $\Delta=\Delta',d:D$ and we have that $(a.k; P')[d\mapsto b]=a.k; P'[d\mapsto b]$. Then we have by induction that $\Delta',b:D,a:A_k\vdash P'[d\mapsto b] :: c\!:\!C$, and by application of $\runa{TS-$\&$L}$ we obtain $\Delta',b:D,a:\&\{l:A_l\}_{l\in L}\vdash a.k; P'[d\mapsto b] :: c\!:\!C$ as required.
    \end{itemize}
    
    \item We have that $(a.k; P')[c\mapsto b]=a.k; P'[c\mapsto b]$. Then we have by induction that $\Delta,a:A_k\vdash P'[c\mapsto b] :: b\!:\!C$, and by application of $\runa{TS-$\&$L}$ we obtain $\Delta,a:\&\{l:A_l\}_{l\in L}\vdash a.k; P'[c\mapsto b] :: b\!:\!C$ as required.
\end{enumerate}

%We have that $\Delta,a : \&\{l:A_l\}_{l\in L}\vdash a.k; P' :: c\!:\!C$ such that $k \in L$ and $\Delta,a:A_k \vdash P' :: c\!:\!C$. The first part of the lemma applies to $(a.k; P')[a\mapsto b]=b.k; P'[a\mapsto b]$ and to $(a.k; P')[d\mapsto b]=a.k; P'[d\mapsto b]$ when $\Delta=\Delta',d:D$. In the first case we have by induction that $\Delta,b:A_k\vdash P'[a\mapsto b] :: c\!:\!C$, and so by $\runa{TS-$\&$L}$ we obtain $\Delta,b:\&\{l:A_l\}_{l\in L}\vdash b.k; P'[a\mapsto b] :: c\!:\!C$. For the second case we have by induction that $\Delta',b:D,a:A_k\vdash P'[d\mapsto b] :: c\!:\!C$, and so by $\runa{TS-$\&$L}$ we obtain $\Delta',b:D,a:\&\{l:A_l\}_{l\in L}\vdash a.k; P'[d\mapsto b] :: c\!:\!C$. The second part of the lemma applies to $(a.k; P')[c\mapsto b]=a.k; P'[c\mapsto b]$. We have by induction that $\Delta,a:A_k\vdash P'[c\mapsto b] :: b\!:\!C$, and so by $\runa{TS-$\&$L}$ we obtain $\Delta,a:\&\{l:A_l\}_{l\in L}\vdash a.k; P'[c\mapsto b] :: b\!:\!C$.

\item[$\runa{TS-$\&$R}$] We have that $\Delta\vdash a.\texttt{case}\{l\Rightarrow P_l\}_{l\in L} :: a\!:\!\&\{l:A_l\}_{l\in L}$ because for $l\in L$ we also have that $\Delta \vdash P_l :: a\!:\!A_l$. We consider the cases separately
\begin{enumerate}
    \item We replace some $d\in\text{dom}(\Delta)$. There exists $\Delta'$ and $D$ such that $\Delta=\Delta',d:D$ and we have that $(a.\texttt{case}\{l\Rightarrow P_l\}_{l\in L})[d\mapsto b]=a.\texttt{case}\{l\Rightarrow P_l[d\mapsto b]\}_{l\in L}$. Then for $l\in L$ we have by induction that $\Delta',b:D\vdash P_l[d\mapsto b] :: a\!:\!a_l$, and by application of $\runa{TS-$\&$R}$ we obtain $\Delta',b:D\vdash a.\texttt{case}\{l\Rightarrow P_l[d\mapsto b]\}_{l\in L} :: a\!:\!\&\{l:A_l\}_{l\in L}$ as required.
    
    \item We have that $(a.\texttt{case}\{l\Rightarrow P_l\}_{l\in L})[a\mapsto b]=b.\texttt{case}\{l\Rightarrow P_l[a\mapsto b]\}_{l\in L}$. Then for $l\in L$ we have by induction that $\Delta\vdash P_l[a\mapsto b] :: b\!:\!A_l$, and by application of $\runa{TS-$\&$L}$ we obtain $\Delta\vdash b.\texttt{case}\{l\Rightarrow P_l[a\mapsto b]\}_{l\in L} :: b\!:\!\&\{l:A_l\}_{l\in L}$ as required.
\end{enumerate}

%We have that $\Delta\vdash a.\texttt{case}\{l\Rightarrow P_l\}_{l\in L} :: a\!:\!\&\{l:A_l\}_{l\in L}$ such that for $l\in L$ we also have that $\Delta \vdash P_l :: a\!:\!A_l$. The first part of the lemma applies to $(a.\texttt{case}\{l\Rightarrow P_l\}_{l\in L})[d\mapsto b]=a.\texttt{case}\{l\Rightarrow P_l[d\mapsto b]\}_{l\in L}$ when $\Delta=\Delta',d:D$. We have by induction for $l\in L$ that $\Delta',b:D\vdash P_l[d\mapsto b] :: a\!:\!a_l$, and so by $\runa{TS-$\&$R}$ we obtain $\Delta',b:D\vdash a.\texttt{case}\{l\Rightarrow P_l[d\mapsto b]\}_{l\in L} :: a\!:\!\&\{l:A_l\}_{l\in L}$. The second part of the lemma applies to $(a.\texttt{case}\{l\Rightarrow P_l\}_{l\in L})[a\mapsto b]=b.\texttt{case}\{l\Rightarrow P_l[a\mapsto b]\}_{l\in L}$. We have by induction for $l\in L$ that $\Delta\vdash P_l[a\mapsto b] :: b\!:\!A_l$, and so by $\runa{TS-$\&$L}$ we obtain $\Delta\vdash b.\texttt{case}\{l\Rightarrow P_l[a\mapsto b]\}_{l\in L} :: b\!:\!\&\{l:A_l\}_{l\in L}$.

\item[$\runa{TS-def}$] We have that $\Delta,\widetilde{b}:\widetilde{B}\vdash \newvar{a}{(a \leftarrow f \leftarrow \widetilde{b} \mid P'')} :: c\!:\!C$ because $(\widetilde{v}:\widetilde{B}\vdash f = Q' :: g\!:\!A)\in\Sigma$ and $\Delta,a:A\vdash P'' :: c\!:\!C$. Then $(\newvar{a}{(a\leftarrow f \leftarrow\widetilde{b} \mid P'')})[d\mapsto h]=\newvar{a}{((a\leftarrow f \leftarrow\widetilde{b})[d\mapsto h] \mid P''[d\mapsto h])}$ and we can assume that $d\neq a$, as $\Delta,\widetilde{b}:\widetilde{B}\vdash \newvar{a}{(a\leftarrow f \leftarrow\widetilde{b} \mid P'') :: c\!:\!C}$ does not hold when $a\in \text{dom}(\Delta,\widetilde{b}:\widetilde{B})$ or $a=c$. We consider the cases separately
\begin{enumerate}
    \item We either have that $d\in\text{dom}(\Delta)$ or $d\in\text{dom}(\widetilde{b}:\widetilde{B})$. We consider them separately
    \begin{itemize}
        \item There exists $\Delta'$ and $D$ such that $\Delta=\Delta',d:D$ and by induction we have that $\Delta',h:D,a:A\vdash P''[d\mapsto h] :: c\!:C$. By application of $\runa{TS-def}$ we obtain $\Delta',h:D,\widetilde{b}:\widetilde{B}\vdash \newvar{a}{((a\leftarrow f \leftarrow\widetilde{b})[d\mapsto h] \mid P''[d\mapsto h])} :: c\!:\!C$ as required.
        
        \item There exists $\widetilde{b'}$, $\widetilde{B'}$ and $D$ such that $\widetilde{b}:\widetilde{B}=\widetilde{b'}:\widetilde{B'},d:D$ and by induction we have that $\Delta',h:D,a:A\vdash P''[d\mapsto h] :: c\!:C$. By application of $\runa{TS-def}$ we obtain $\Delta',h:D,\widetilde{b}:\widetilde{B}\vdash \newvar{a}{((a\leftarrow f \leftarrow\widetilde{b})[d\mapsto h] \mid P''[d\mapsto h])} :: c\!:\!C$ as required.
    \end{itemize}
    
    \item We need only consider $P''$ as $d\neq a$, and so $d=c$. By induction we have that $\Delta,a:A\vdash P''[c\mapsto b] :: b\!:\!C$, and by application of $\runa{TS-def}$ we obtain $\Delta',h:D,\widetilde{b}:\widetilde{B}\vdash \newvar{a}{((a\leftarrow f \leftarrow\widetilde{b})[d\mapsto h] \mid P''[d\mapsto h])} :: c\!:\!C$ as required.
\end{enumerate}

% The first part of the lemma applies when either
% \begin{enumerate}
%     \item $\widetilde{b}:\widetilde{B}=\widetilde{b'}:\widetilde{B'},d:D$ such that $(a\leftarrow f \leftarrow\widetilde{b})[d\mapsto h]=a\leftarrow f \leftarrow \widetilde{b'}:\widetilde{B'},h:D$, and so we obtain $\Delta,\widetilde{b'}:\widetilde{B'},h:D\vdash \newvar{a}{(a\leftarrow f \leftarrow \widetilde{b'}:\widetilde{B'},h:D \mid P''[d\mapsto h]) :: c\!:\!C}$ from $\runa{TS-def}$.
    
%     \item $\Delta=\Delta',d:D$, and so by induction we obtain $\Delta',h:D,a:A\vdash P''[d\mapsto h] :: c\!:C$. It follows from $\runa{TS-def}$ that $\Delta',h:D,\widetilde{b}:\widetilde{B}\vdash \newvar{a}{((a\leftarrow f \leftarrow\widetilde{b})[d\mapsto h] \mid P''[d\mapsto h])} :: c\!:\!C$.
% \end{enumerate}
% The second part of the lemma can only apply to $P''$ as $d\neq a$, and so if $d=c$ we have by induction that $\Delta,a:A\vdash P''[d\mapsto b] :: b\!:\!C$. Then we obtain $\Delta,\widetilde{b}:\widetilde{B}\vdash \newvar{a}{((a\leftarrow f \leftarrow\widetilde{b})[d\mapsto h] \mid P''[d\mapsto h])} :: b\!:\!C$ directly from $\runa{TS-def}$. 

\item[$\runa{TS-$\ocircle$LR'}$] We have that $\Delta\vdash \tick P' :: a\!:\!A$ because $[\Delta]^{-1}_L\vdash P' :: a\!:\![A]^{-1}_R$. We consider the cases separately
\begin{enumerate}
    \item We replace some $d\in\text{dom}(\Delta)$, and so there exists $\Delta'$ and $D$ such that $\Delta=\Delta',d:D$. Then we have that $(\tick P')[d\mapsto b]=\tick P'[d\mapsto b]$, and by induction we have that $[\Delta']^{-1}_L,b:[D]^{-1}_L\vdash P'[d\mapsto b] :: a\!:\![A]^{-1}_R$. By application of $\runa{TS-$\ocircle$LR'}$ we obtain $\Delta',b:D\vdash \tick P'[d\mapsto b] :: a\!:\!A$ as required.
    
    \item We have that $(\tick P')[a\mapsto b]=\tick P'[a\mapsto b]$, and so by induction we have that $[\Delta]^{-1}_L\vdash P'[a\mapsto b] :: b\!:\![A]^{-1}_R$. By application of $\runa{TS-$\ocircle$LR'}$ we obtain $\Delta\vdash \tick P'[a\mapsto b] :: b\!:\!A$ as required.
\end{enumerate}

%The first part of the lemma applies to $(\tick P')[d\mapsto b]=\tick P'[d\mapsto b]$ when $\Delta=\Delta',d:D$. We have by induction that $[\Delta']^{-1}_L,b:[D]^{-1}_L\vdash P'[d\mapsto b] :: a\!:\![A]^{-1}_R$, and so by $\runa{TS-$\ocircle$LR'}$ we obtain $\Delta',b:D\vdash \tick P'[d\mapsto b] :: a\!:\!A$. The second part of the lemma applies to $(\tick P')[a\mapsto b]=\tick P'[a\mapsto b]$. We have by induction that $[\Delta]^{-1}_L\vdash P'[a\mapsto b] :: b\!:\![A]^{-1}_R$, and so by $\runa{TS-$\ocircle$LR'}$ we obtain $\Delta\vdash \tick P'[a\mapsto b] :: b\!:\!A$.

\item[$\runa{TS-$\ocircle$LR}$] We have that $\Delta\vdash P :: a\!:\!A$ because $[\Delta]^{-1}_L\vdash P :: a\!:\![A]^{-1}_R$. We consider the cases separately
\begin{enumerate}
    \item We replace some $d\in\text{dom}(\Delta)$, and so there exists $\Delta'$ and $D$ such that $\Delta=\Delta',d:D$. Then by induction we have that $[\Delta']^{-1}_L,b:[D]^{-1}_L\vdash P[d\mapsto b] :: a\!:\![A]^{-1}_R$. By application of $\runa{TS-$\ocircle$LR}$ we obtain $\Delta',b:D\vdash P[d\mapsto b] :: a\!:\!A$ as required.
    
    \item We replace $a$, and so by induction we have that $[\Delta]^{-1}_L\vdash P[a\mapsto b] :: b\!:\![A]^{-1}_R$. By application of $\runa{TS-$\ocircle$LR}$ we obtain $\Delta\vdash P[a\mapsto b] :: b\!:\!A$ as required.
\end{enumerate}

%The first part of the lemma applies to $P[d\mapsto b]$ when $\Delta=\Delta',d:D$. We have by induction that $[\Delta']^{-1}_L,b:[D]^{-1}_L\vdash P[d\mapsto b] :: a\!:\![A]^{-1}_R$, and so by $\runa{TS-$\ocircle$LR}$ we obtain $\Delta',b:D\vdash P[d\mapsto b] :: a\!:\!A$. The second part of the lemma applies to $P[a\mapsto b]$. We have by induction that $[\Delta]^{-1}_L\vdash P[a\mapsto b] :: b\!:\![A]^{-1}_R$, and so by $\runa{TS-$\ocircle$LR}$ we obtain $\Delta\vdash P[a\mapsto b] :: b\!:\!A$.

\item[$\runa{TS-$\lozenge$L}$] We have that $\Delta,a:\lozenge A\vdash P :: c\!:\!C$ because $\Delta\;\texttt{delayed}^\Box$, $C\;\texttt{delayed}^\lozenge$ and $\Delta,a:A\vdash P :: c\!:\!C$. We consider the cases separately
\begin{enumerate}
    \item We either replace $a$ or some $d\in\text{dom}(\Delta)$. We consider them separately
    \begin{itemize}
        \item By induction we have that $\Delta,b:A\vdash P[a\mapsto b] :: c\!:\!C$, and by application of $\runa{TS-$\lozenge$L}$ we obtain $\Delta,b:\lozenge A\vdash P[a\mapsto b] :: c\!:\!C$ as required.
        
        \item There exists $\Delta'$ and $D$ such that $\Delta=\Delta',d:D$, and so by induction we have that $\Delta',b:D,a:A\vdash P[d\mapsto b] :: c\!:\!C$. As the types are unchanged it follows that also $\Delta',b:D\;\texttt{delayed}^\Box$. Then by application of $\runa{TS-$\lozenge$L}$ we obtain $\Delta',b:D,a:\lozenge A\vdash P[d\mapsto b] :: c\!:\!C$ as required.
    \end{itemize}
    
    \item We replace $c$, and so by induction we have that $\Delta, a:A\vdash P[c\mapsto b] :: b\!:\!C$. By application of $\runa{TS-$\lozenge$L}$ we obtain $\Delta,a:\lozenge A\vdash P[c\mapsto b] :: b\!:\!C$ as required.
\end{enumerate}

%The first part of the lemma applies to $P[d\mapsto b]$ when either $d=a$ or $\Delta = \Delta',d:D$. In the first case we have by induction that $\Delta,b:A\vdash P[a\mapsto b] :: c\!:\!C$, and so from $\runa{TS-$\lozenge$L}$ we obtain $\Delta,b:\lozenge A\vdash P[a\mapsto b] :: c\!:\!C$. For the second case we have by induction that $\Delta',b:D,a:A\vdash P[d\mapsto b] :: c\!:\!C$, and as the types are unchanged it follows that also $\Delta',b:D\;\texttt{delayed}^\Box$. Then from  and from $\runa{TS-$\lozenge$L}$ we obtain $\Delta',b:D,a:\lozenge A\vdash P[d\mapsto b] :: c\!:\!C$. The second part of the lemma applies to $P[d\mapsto b]$ when $d=c$. Then we have by induction that $\Delta, a:A\vdash P[c\mapsto b] :: b\!:\!C$, and it follows from $\runa{TS-$\lozenge$L}$ that also $\Delta,a:\lozenge A\vdash P[c\mapsto b] :: b\!:\!C$.

\item[$\runa{TS-$\lozenge$R}$] We have that $\Delta\vdash P :: c\!:\!\lozenge C$ because $\Delta\vdash P :: c\!:\!C$. We consider the cases separately
\begin{enumerate}
    \item We replace some $d\in\text{dom}(\Delta)$, and so there exists $\Delta'$ and $D$ such that $\Delta=\Delta',d:D$. By induction we have that $\Delta',b:D\vdash P[d\mapsto b] :: c\!:\!C$, and by application of $\runa{TS-$\lozenge$R}$ we obtain $\Delta',b:D\vdash P[d\mapsto b] :: c\!:\!C$ as required.
    
    \item We replace $c$, and so by induction we have that $\Delta\vdash P[c\mapsto b] :: b\!:\!C$. By application of $\runa{TS-$\lozenge$R}$ we obtain $\Delta\vdash P[c\mapsto b] :: b\!:\!\lozenge C$ as required.
\end{enumerate}

%The first part of the lemma applies to $P[d\mapsto b]$ when $\Delta = \Delta',d:D$. We have by induction that $\Delta',b:D\vdash P[d\mapsto b] :: c\!:\!C$, and from $\runa{TS-$\lozenge$R}$ we obtain $\Delta',b:D\vdash P[d\mapsto b] :: c\!:\!C$. The second part of the lemma applies to $P[d\mapsto b]$ when $d=c$. Then we have by induction that $\Delta\vdash P[c\mapsto b] :: b\!:\!C$, and it follows from $\runa{TS-$\lozenge$R}$ that also $\Delta\vdash P[c\mapsto b] :: b\!:\!\lozenge C$.

\item[$\runa{TS-$\Box$L}$] We have that $\Delta,a:\Box A\vdash P :: c\!:\!C$ because $\Delta,a:A\vdash P :: c\!:\!C$. We consider the cases separately
\begin{enumerate}
    \item We either replace $a$ or some $d\in\text{dom}(\Delta)$. We consider them separately
    \begin{itemize}
        \item By induction we have that $\Delta,b:A\vdash P[a\mapsto b] :: c\!:\!C$ and by application of $\runa{TS-$\Box$L}$ we obtain $\Delta,b:\Box A\vdash P[a\mapsto b] :: c\!:\!C$ as required.
        
        \item There exists $\Delta'$ and $D$ such that $\Delta=\Delta',d:D$, and so by induction we have that $\Delta',b:D,a:A\vdash P[d\mapsto b] :: c\!:\!C$. By application of $\runa{TS-$\Box$L}$ we obtain $\Delta',b:D,a:\Box A\vdash P[d\mapsto b] :: c\!:\!C$ as required.
    \end{itemize}
    
    \item We replace $c$, and so by induction we have that $\Delta, a:A\vdash P[c\mapsto b] :: b\!:\!C$, and by application of $\runa{TS-$\Box$L}$ we obtain $\Delta,a:\Box A\vdash P[c\mapsto b] :: b\!:\!C$ as required.
\end{enumerate}

%The first part of the lemma applies to $P[d\mapsto b]$ when either $d=a$ or $\Delta = \Delta',d:D$. In the first case we have by induction that $\Delta,b:A\vdash P[a\mapsto b] :: c\!:\!C$ and so we obtain from $\runa{TS-$\Box$L}$ that also $\Delta,b:\Box A\vdash P[a\mapsto b] :: c\!:\!C$. For the second case we have by induction that $\Delta',b:D,a:A\vdash P[d\mapsto b] :: c\!:\!C$, and from $\runa{TS-$\Box$L}$ we obtain $\Delta',b:D,a:\Box A\vdash P[d\mapsto b] :: c\!:\!C$. The second part of the lemma applies to $P[d\mapsto b]$ when $d=c$. Then we have by induction that $\Delta, a:A\vdash P[c\mapsto b] :: b\!:\!C$, and it follows from $\runa{TS-$\Box$L}$ that also $\Delta,a:\Box A\vdash P[c\mapsto b] :: b\!:\!C$.

\item[$\runa{TS-$\Box$R}$] We have that $\Delta\vdash P :: c\!:\!\Box C$ because $\Delta\;\texttt{delayed}^\Box$ and $\Delta\vdash P :: c\!:\!C$. We consider the cases separately
\begin{enumerate}
    \item We replace some $d\in\text{dom}(\Delta)$, and so there exists $\Delta'$ and $D$ such that $\Delta=\Delta',d:D$. By induction we have that $\Delta',b:D\vdash P[d\mapsto b] :: c\!:\!C$. Then, as the types are unchanged, it follows that also $\Delta',b:D\;\texttt{delayed}^\Box$. By application of $\runa{TS-$\Box$R}$ we then obtain $\Delta',b:D\vdash P[d\mapsto b] :: c\!:\!\Box C$ as required.
    
    \item We replace $c$, and so by induction we have that $\Delta\vdash P[c\mapsto b] :: b\!:\!C$, and by application of $\runa{TS-$\Box$R}$ we obtain $\Delta\vdash P[c\mapsto b] :: b\!:\!\Box C$ as required.
\end{enumerate}

%The first part of the lemma applies to $P[d\mapsto b]$ when $\Delta = \Delta',d:D$. We have by induction that $\Delta',b:D\vdash P[d\mapsto b] :: c\!:\!C$, and as the types are unchanged it follows that also $\Delta',b:D\;\texttt{delayed}^\Box$. Then from $\runa{TS-$\Box$R}$ we obtain $\Delta',b:D\vdash P[d\mapsto b] :: c\!:\!\Box C$. The second part of the lemma applies to $P[d\mapsto b]$ when $d=c$. Then we have by induction that $\Delta\vdash P[c\mapsto b] :: b\!:\!C$, and it follows from $\runa{TS-$\Box$R}$ that also $\Delta\vdash P[c\mapsto b] :: b\!:\!\Box C$.

\end{description}
\end{proof}
\end{lemma}


% \begin{lemma}\label{lemma:contextredex}
% If $P$ is a redex and $\Delta\vdash C[P] :: a\!:\!A$ such that $P\longrightarrow P'$ then $\Delta\vdash C[P'] :: a\!:\!A$.
% \begin{proof}
% By induction on the reduction rules 
% \end{proof}
% \end{lemma}




\begin{theorem}[Subject reduction]
If $\Delta \vdash P :: a\!:\!A$ and $P \longrightarrow Q$ then $\Delta\vdash Q :: a\!:\!A$.
\begin{proof}
By induction on the reduction rules. For a process to be well-typed and reduce, it must be typed with either $\runa{TS-$\ocircle$LR'}$, $\runa{TS-cut}$ or $\runa{TS-def}$, and so it suffices to consider $\runa{R-tick}$, $\runa{R-res}$, $\runa{R-id}$ and $\runa{R-struct}$. We omit $\runa{R-struct}$ as typability is closed under structural congruence. We consider the cases separately
\begin{description}
\item[$\runa{R-tick}$] We have that $P=\texttt{tick}.P'$ and $Q = P'$. Then by $\runa{TS-$\ocircle$LR'}$, we have that $[\Delta]^{-1}_L \vdash P' :: [a:A]^{-1}_R$ such that $\Delta \vdash \texttt{tick}.P' :: a\!:\!A$. It follows from type rule $\runa{TS-$\ocircle$LR}$ that also $\Delta \vdash P' :: a\!:\!A$.

\item[$\runa{R-res}$] We have that $P\equiv\newvar{a}{(P'' \mid P'')}$. Then for $P$ to be well-typed, we must use either $\runa{TS-cut}$ or $\runa{TS-def}$. We consider the cases separately
\begin{description}
    \item[$\runa{TS-cut}$] We have that $\Delta\vdash \newvar{a}{(P' \mid P'') :: c\!:\!C}$ with $\Delta=\Delta_1,\Delta_2$ such that $\Delta_1\vdash P' :: a\!:\!A$ and $\Delta_2,a:A\vdash P'' :: c\!:\!C$. Then either $P' \longrightarrow Q'$ or $P''\longrightarrow Q''$ by $\runa{R-par}$ or $P' \mid P'' \longrightarrow Q'\mid Q''$ such that $P'\neq Q'$ and $P''\neq Q''$. In the two first cases we obtain $\Delta_1\vdash Q' :: a\!:\!A$ by induction from $\Delta_1\vdash P' :: a\!:\!A$ and $\Delta_2,a:A\vdash Q'' :: c\!:\!C$ by induction from $\Delta_2,a:A\vdash P'' :: c\!:\!C$, respectively, from which we obtain $\Delta\vdash Q :: c\!:\!C$ by $\runa{TS-cut}$. In the third case as $P' \mid P''$ reduces, $P'$ and $P''$ must synchronize. We then have the canonical form $P\equiv\newvar{\widetilde{d}}{(R_1 \mid R_2 \mid \cdots \mid R_n)}$ such that $R_1$ and $R_2$ correspond to the prefixes that synchronize in $P'$ and $P''$, and so we have $R_1 \mid R_2 \longrightarrow R_1' \mid R_2'$. Then for $P$ to be well-typed, each parallel composition must be wrapped in a restriction and be typed with either $\runa{TS-cut}$ or $\runa{TS-def}$, such that $\widetilde{d}$ contains $n-1$ names. By premise of these rules, there must be some partitions $\Delta_1=\Delta_1',\Delta_1''$ and $\Delta_2=\Delta_2',\Delta_2''$ such that $\Delta_1'\vdash R_1 :: a\!:\!A$ and $\Delta_2',a:A\vdash R_2 :: b\!:\!B$. For $Q$ to also be well-typed under the same typing as $P$, it then suffices to show that there exists some new partition $\Delta_1',\Delta_2'=\Delta_3,\Delta_4$ and type $A'$ such that $\Delta_3\vdash R_1' :: a\!:\!A'$ and $\Delta_4,a:A'\vdash R_2' :: b\!:\!B$. We consider the cases of the reduction 
    \begin{description}
    \item[$\runa{R-comm}$]
    Assume we reduce by $\runa{R-comm}$ then $R_1 \mid R_2 \equiv \inputch{a}{v}{}{R_1'} \mid \outputch{a}{w}{}{R_2'}$ for some name $w$ and processes $R_1'$ and $R_2'$, such that $\inputch{a}{v}{}{R_1'} \mid \outputch{a}{b}{}{R_2'} \longrightarrow R_1'[v\mapsto b] \mid R_2'$. Then $R_1$ and $R_2$ have two possible typings
    \begin{enumerate}
    \item $A=A'\multimap A''$ and $\Delta_2'=\Delta_3',w:A'$ such that $\Delta_1' \vdash \inputch{a}{v}{}{R_1'} :: a\!:\!A' \multimap A''$ and $\Delta_3',a : A'\multimap A'', w : A' \vdash \outputch{a}{w}{}{R_2'} :: b\!:\!B$ by $\runa{TS-$\multimap$R}$ and $\runa{TS-$\multimap$L}$. By the premises to these rules we have that $\Delta_1',v : A' \vdash R_1' :: a\!:\!A''$ and $\Delta_3',a:A''\vdash R_2' :: b\!:\!B$. This implies $\Delta_1',w : A'\vdash R'[v\mapsto w] :: a\!:\!A''$ by Lemma \ref{lemma:substlem}, and $\Delta_1',\Delta_2'=\Delta_1',w:A',\Delta_3'$. %so by $\runa{TS-cut}$ it follows that $(\Delta',b : A'),\Delta_3\vdash \newvar{a}{(R'[v\mapsto b] \mid R'') :: c\!:\!C}$ and $\Delta = (\Delta',b : A'),\Delta_3$.
    
    %
    
    \item $A=A'\otimes A''$ and $\Delta_1'=\Delta_3',w:A$ such that $\Delta_3,w:A' \vdash \outputch{a}{w}{}{R_1'} :: a\!:\!A'\otimes A''$ and $\Delta_2',a : A'\otimes A''\vdash \inputch{a}{v}{}{R_2'} :: b\!:\!B$ by $\runa{TS-$\otimes$R}$ and $\runa{TS-$\otimes$L}$. By the premises to these rules we have that $\Delta_3'\vdash R_1' :: a\!:\!A''$ and $\Delta_2',a:A'',v:A'\vdash R_2' :: b\!:\!B$. This implies $\Delta_2',a:A'',w:A'\vdash R_2'[v\mapsto w] :: b\!:\!B$ by Lemma \ref{lemma:substlem}, and $\Delta_1',\Delta_2'=\Delta_3',\Delta_2',w:A'$. %so by $\runa{TS-cut}$ it follows that $\Delta_3,(\Delta'',b : A')\vdash \newvar{a}{(R'' \mid R'[v\mapsto b])} :: c\!:\!C$ and $\Delta = \Delta_3,(\Delta'',b : A')$.
\end{enumerate}
    
    %
    
    
    \item[$\runa{R-choice}$] Assume we reduce by $\runa{R-choice}$ then $R_1 \mid R_2 \equiv a.\texttt{case}\{ l \Rightarrow P_l \}_{l\in L} \mid a.k; R$ for some label $k$ and set of labels $L$, such that $k\in L$ and $a.\texttt{case}\{ l \Rightarrow P_l \}_{l\in L} \mid a.k; R \longrightarrow P_k \mid R$. Then $R_1$ and $R_2$ have two possible typings
\begin{enumerate}
    \item $A=\&\{l : A_l\}_{l\in L}$ and $\Delta_1'\vdash a.\texttt{case}\{l \Rightarrow P_l\}_{l\in L} :: a\!:\!\&\{l : A_l\}_{l\in L}$ and $\Delta_2', a : \&\{l : A_l\}_{l\in L}\vdash a.k; R :: b\!:\!B$ by $\runa{TS-$\&$R}$ and $\runa{TS-$\&$L}$. By the premises of these rules we have that $\Delta_1' \vdash P_k :: a\!:\!A_k$ and $\Delta_2',a : A_k\vdash R :: b\!:\!B$, such that $R_1' = P_k$ and $R_2'=R$ and so we obtain $\Delta_1',\Delta_2'=\Delta_1',\Delta_2'$ directly.
        
    %
    
    \item $A=\oplus\{l : A_l\}_{l\in L}$ and $\Delta_1'\vdash a.k; R :: a\!:\!\oplus\{l : A_l\}_{l\in L}$ and $\Delta_2',a : \oplus\{l : A_l\}_{l\in L}\vdash a.\texttt{case}\{l\Rightarrow P_l\}_{l\in L} :: b\!:\!B$ by $\runa{TS-$\oplus$R}$ and $\runa{TS-$\oplus$L}$. By the premises of these rules we have that $\Delta_1'\vdash R :: a\!:\!A_k$ and $\Delta_2',a : A_k\vdash P_k :: b\!:\!B$, such that $R_1'=R$ and $R_2'=P_k$ and so we obtain $\Delta_1',\Delta_2'=\Delta_1',\Delta_2'$ directly.
    
\end{enumerate}
    
    \end{description}
    
    
    %$R_1 \mid R_2$ must be a redex such that $R_1 \mid R_2 \longrightarrow R_1' \mid R_2'$, and so we obtain $\Delta\vdash\newvar{\widetilde{b}}{([R_1' \mid R_2'] \mid R_\text{rem})} :: c\!:\!C$ from $\Delta\vdash\newvar{\widetilde{b}}{([R_1 \mid R_2] \mid R_\text{rem})} :: c\!:\!C$ by Lemma \ref{lemma:contextredex}.
    
    \item[$\runa{TS-def}$] We have that $\Delta\vdash \newvar{a}{(a \leftarrow f \leftarrow \widetilde{b} \mid P'')} :: c\!:\!C$ with $\Delta=\Delta',\widetilde{b} : \widetilde{B}$ such that $(\widetilde{d} :\widetilde{B} \vdash f = Q' :: g\!:\!A) \in \Sigma$ and $\Delta',a:A\vdash P'' :: c\!:\!C$. Then either $a \leftarrow f \leftarrow \widetilde{b} \longrightarrow Q'[g\mapsto a, \widetilde{d}\mapsto\widetilde{b}]$ by $\runa{R-par}$ and $\runa{R-def}$ or $P''\longrightarrow Q''$ by $\runa{R-par}$. In the first case we obtain $\widetilde{b}:\widetilde{B}\vdash Q'[g\mapsto a, \widetilde{d}\mapsto\widetilde{b}] :: a\!:\!A$ from $\widetilde{d} :\widetilde{B} \vdash Q' :: g\!:\!A$ by Lemma \ref{lemma:substlem}. It follows from $\runa{TS-cut}$ that $\Delta\vdash \newvar{a}{(Q'[g\mapsto a, \widetilde{d}\mapsto\widetilde{b}] \mid P'')} :: c\!:\!C$. In the second case we obtain $\Delta',a:A\vdash Q'' :: c\!:\!C$ by induction from $\Delta',a:A\vdash P'' :: c\!:\!C$. It follows from $\runa{TS-def}$ that $\Delta\vdash \newvar{a}{(a \leftarrow f \leftarrow \widetilde{b} \mid Q'')} :: c\!:\!C$.
\end{description}

\item[$\runa{R-id}$] We have that $P \equiv \newvar{a}{\newvar{b}{(P' \mid a \leftarrow b)}}$ such that $Q \equiv \newvar{h}{(P'[a\mapsto h,b\mapsto h])}$ for some name $h$ not free in $P'$. Then, as restrictions are only typable by $\runa{TS-cut}$ and $\runa{TS-def}$, $P'$ must be of the form $R' \mid R''$ such that $P \equiv \newvar{a}{(R' \mid \newvar{b}{(R'' \mid a \leftarrow b)})}$ or $P \equiv \newvar{b}{(R' \mid \newvar{a}{(R'' \mid a \leftarrow b)})}$. We consider the cases separately
\begin{enumerate}
    \item $\Delta'',a:A \vdash R' :: c\!:\!C$ such that $\Delta'\vdash \newvar{b}{(R'' \mid a \leftarrow b)} :: a\!:\!A$ and $\Delta',\Delta''\vdash P :: c\!:\!C$ using $\runa{TS-cut}$. Then we can type $\newvar{b}{(R'' \mid a \leftarrow b)}$ with either $\runa{TS-cut}$ or $\runa{TS-def}$
    \begin{enumerate}
        \item $\Delta' \vdash R'' :: b\!:\!A$ such that $b:A\vdash a \leftarrow b :: a\!:\!A$ and $\Delta' \vdash \newvar{b}{(R'' \mid a \leftarrow b)} :: a\!:\!A$. Then it follows by Lemma \ref{lemma:substlem} that $\Delta''\vdash R''[a\mapsto h,b\mapsto h] :: h\!:\!A$ and $\Delta'',h:A \vdash R' :: c\!:\!C$ such that $\Delta',\Delta''\vdash\newvar{h}{(R'[a\mapsto h,b\mapsto h] \mid R''[a\mapsto h,b\mapsto h]) :: c\!:\!C}$.
        
        \item $R'' = b \leftarrow f \leftarrow \widetilde{d}$ and $(\widetilde{e} : \widetilde{B}\vdash f = R :: g\!:\!A) \in \Sigma$ such that $\Delta' = \widetilde{d}:\widetilde{B}$, $b:A\vdash a \leftarrow b :: a\!:\!A$ and $\Delta' \vdash \newvar{b}{(R'' \mid a \leftarrow b)} :: a\!:\!A$. Then it follows by Lemma \ref{lemma:substlem} that $\Delta'',h:A \vdash R' :: c\!:\!C$ such that $\Delta',\Delta''\vdash\newvar{h}{(R'[a\mapsto h,b\mapsto h] \mid h \leftarrow f \leftarrow \widetilde{d}) :: c\!:\!C}$.
    \end{enumerate}
    
    %
    
    \item Either $\Delta' \vdash R' :: b\!:\!A$ or $b \leftarrow f \leftarrow \widetilde{d}$, $\Delta' = \widetilde{d}:\widetilde{B}$ and $(\widetilde{e} : \widetilde{B}\vdash f = R :: g\!:\!A) \in \Sigma$ such that $\Delta'',b:A\vdash \newvar{a}{(R'' \mid a \leftarrow b)} :: c\!:\!C$ and $\Delta',\Delta''\vdash P :: c\!:\!C$ using $\runa{TS-cut}$ or $\runa{TS-def}$, respectively. In either case we must use $\runa{TS-cut}$ to get $\Delta'',b:A\vdash \newvar{a}{(R'' \mid a \leftarrow b)} :: c\!:\!C$, as we have that $b:A\vdash a\leftarrow b :: a\!:\!A$ and $\Delta'',a:A\vdash R'' :: c\!:\!C$. Then we reach $\Delta',\Delta''\vdash\newvar{h}{(R'[a\mapsto h,b\mapsto h] \mid R''[a\mapsto h,b\mapsto h])} :: c\!:\!C$ by either $\runa{TS-cut}$ or $\runa{TS-def}$. In either case we have that $\Delta'',h:A\vdash R''[a\mapsto h,b\mapsto h] :: c\!:\!C$. In the first case we have that $\Delta' \vdash R'[a\mapsto h,b\mapsto h] :: h\!:\!A$ and the latter case trivially follows by $R'[a\mapsto h,b\mapsto h] = h \leftarrow f \leftarrow \widetilde{d}$, concluding the proof.
\end{enumerate}
\end{description}

% OLD BEGIN !

% by induction on the extended reduction rules. The proof uses the fact that a well-typed process cannot \textit{consume} the session it provides on reduction, by type rules $\runa{TS-cut}$ and $\runa{TS-def}$. The proof is slightly tedious, as the type rules are not syntax directed.
% \begin{description}
% \item[$\runa{R-tick}$] Assume that $P$ reduces by $\runa{R-tick}$, such that $P$ is of the form $\texttt{tick}.P'$ and $Q = P'$. Then by $\runa{TS-$\ocircle$LR'}$, we have that $[\Delta]^{-1}_L \vdash P' :: [a:A]^{-1}_R$ such that $\Delta \vdash \texttt{tick}.P' :: a\!:\!A$. It follows from type rule $\runa{TS-$\ocircle$LR}$ that also $\Delta \vdash P' :: a\!:\!A$.

% %

% \item[$\runa{R-id}$] Assume that $P$ reduces by $\runa{R-id}$ then we have that $P \equiv \newvar{a}{\newvar{b}{(P' \mid a \leftarrow b)}}$ such that $Q \equiv \newvar{h}{(P'[a\mapsto h,b\mapsto h])}$ for some name $h \notin fv(P')$. Then, as restrictions are only typable by $\runa{TS-cut}$ and $\runa{TS-def}$, $P'$ must be of the form $R' \mid R''$ such that $P \equiv \newvar{a}{(R' \mid \newvar{b}{(R'' \mid a \leftarrow b)})}$ or $P \equiv \newvar{b}{(R' \mid \newvar{a}{(R'' \mid a \leftarrow b)})}$. We consider the cases separately
% \begin{enumerate}
%     \item $\Delta'',a:A \vdash R' :: c\!:\!C$ such that $\Delta'\vdash \newvar{b}{(R'' \mid a \leftarrow b)} :: a\!:\!A$ and $\Delta',\Delta''\vdash P :: c\!:\!C$ using $\runa{TS-cut}$. Then we can type $\newvar{b}{(R'' \mid a \leftarrow b)}$ with either $\runa{TS-cut}$ or $\runa{TS-def}$
%     \begin{enumerate}
%         \item $\Delta' \vdash R'' :: b\!:\!A$ such that $b:A\vdash a \leftarrow b :: a\!:\!A$ and $\Delta' \vdash \newvar{b}{(R'' \mid a \leftarrow b)} :: a\!:\!A$. Then it follows by Lemma \ref{lemma:substlem} that $\Delta''\vdash R''[a\mapsto h,b\mapsto h] :: h\!:\!A$ and $\Delta'',h:A \vdash R' :: c\!:\!C$ such that $\Delta',\Delta''\vdash\newvar{h}{(R'[a\mapsto h,b\mapsto h] \mid R''[a\mapsto h,b\mapsto h]) :: c\!:\!C}$.
        
%         \item $R'' = b \leftarrow f \leftarrow \widetilde{d}$ and $(\widetilde{e} : \widetilde{B}\vdash f = R :: g\!:\!A) \in \Sigma$ such that $\Delta' = \widetilde{d}:\widetilde{B}$, $b:A\vdash a \leftarrow b :: a\!:\!A$ and $\Delta' \vdash \newvar{b}{(R'' \mid a \leftarrow b)} :: a\!:\!A$. Then it follows by Lemma \ref{lemma:substlem} that $\Delta'',h:A \vdash R' :: c\!:\!C$ such that $\Delta',\Delta''\vdash\newvar{h}{(R'[a\mapsto h,b\mapsto h] \mid h \leftarrow f \leftarrow \widetilde{d}) :: c\!:\!C}$.
%     \end{enumerate}
    
%     %
    
%     \item Either $\Delta' \vdash R' :: b\!:\!A$ or $b \leftarrow f \leftarrow \widetilde{d}$, $\Delta' = \widetilde{d}:\widetilde{B}$ and $(\widetilde{e} : \widetilde{B}\vdash f = R :: g\!:\!A) \in \Sigma$ such that $\Delta'',b:A\vdash \newvar{a}{(R'' \mid a \leftarrow b)} :: c\!:\!C$ and $\Delta',\Delta''\vdash P :: c\!:\!C$ using $\runa{TS-cut}$ or $\runa{TS-def}$, respectively. In either case we must use $\runa{TS-cut}$ to get $\Delta'',b:A\vdash \newvar{a}{(R'' \mid a \leftarrow b)} :: c\!:\!C$, as we have that $b:A\vdash a\leftarrow b :: a\!:\!A$ and $\Delta'',a:A\vdash R'' :: c\!:\!C$. Then we reach $\Delta',\Delta''\vdash\newvar{h}{(R'[a\mapsto h,b\mapsto h] \mid R''[a\mapsto h,b\mapsto h])} :: c\!:\!C$ by either $\runa{TS-cut}$ or $\runa{TS-def}$. In either case we have that $\Delta'',h:A\vdash R''[a\mapsto h,b\mapsto h] :: c\!:\!C$. In the first case we have that $\Delta' \vdash R'[a\mapsto h,b\mapsto h] :: h\!:\!A$ and the latter case trivially follows by $R'[a\mapsto h,b\mapsto h] = h \leftarrow f \leftarrow \widetilde{d}$.
% \end{enumerate}

% %

% \item[$\runa{R-comm}$] Assume we reduce $P$ by $\runa{R-comm}$ then $P \equiv \inputch{a}{v}{}{R'} \mid \outputch{a}{b}{}{R''}$ for some name $b$ and processes $R'$ and $R''$, such that $\inputch{a}{v}{}{R'} \mid \outputch{a}{b}{}{R''} \longrightarrow R'[v\mapsto b] \mid R''$. For $P$ to be well-typed, it must be part of a larger process $\Delta',\Delta''\vdash\newvar{a}{P} :: c\!:\!C$ typed with $\runa{TS-cut}$ for which we have two cases
% \begin{enumerate}
%     \item $\Delta' \vdash \inputch{a}{v}{}{R'} :: a\!:\!A' \multimap A''$ and $\Delta_3,a : A'\multimap A'', b : A' \vdash \outputch{a}{b}{}{R''} :: c\!:\!C$ by $\runa{TS-$\multimap$R}$ and $\runa{TS-$\multimap$L}$ such that $\Delta'' = \Delta_3,b:A'$. By the premises to these rules we have that $\Delta',v : A' \vdash R' :: a\!:\!A''$ and $\Delta_3,a:A''\vdash R'' :: c\!:\!C$. This implies $\Delta',b : A'\vdash R'[v\mapsto b] :: a\!:\!A''$, and so by $\runa{TS-cut}$ it follows that $(\Delta',b : A'),\Delta_3\vdash \newvar{a}{(R'[v\mapsto b] \mid R'') :: c\!:\!C}$ and $\Delta = (\Delta',b : A'),\Delta_3$.
    
%     %
    
%     \item $\Delta_3,b:A' \vdash \outputch{a}{b}{}{R''} :: a\!:\!A'\otimes A''$ and $\Delta'',a : A'\otimes A''\vdash \inputch{a}{v}{}{R'} :: c\!:\!C$ by $\runa{TS-$\otimes$R}$ and $\runa{TS-$\otimes$L}$ such that $\Delta' = \Delta_3,b:A'$. By the premises to these rules we have that $\Delta_3\vdash R'' :: a\!:\!A''$ and $\Delta'',a:A'',v:A'\vdash R' :: c\!:\!C$. This implies $\Delta'',a:A'',b:A'\vdash R'[v\mapsto b] :: c\!:\!C$, and so by $\runa{TS-cut}$ it follows that $\Delta_3,(\Delta'',b : A')\vdash \newvar{a}{(R'' \mid R'[v\mapsto b])} :: c\!:\!C$ and $\Delta = \Delta_3,(\Delta'',b : A')$.
% \end{enumerate}

% %

% \item[$\runa{R-choice}$] Assume we reduce $P$ by $\runa{R-choice}$ then $P \equiv a.\texttt{case}\{ l \Rightarrow P_l \}_{l\in L} \mid a.k; R$ for some label $k$ and set of labels $L$, such that $k\in L$ and $a.\texttt{case}\{ l \Rightarrow P_l \}_{l\in L} \mid a.k; R \longrightarrow P_k \mid R$. For $P$ to be well-typed, it must be part of a larger process $\Delta',\Delta''\vdash \newvar{a}{P} :: c\!:\!C$ typed with $\runa{TS-cut}$ for which we have two cases
% \begin{enumerate}
%     \item $\Delta'\vdash a.\texttt{case}\{l \Rightarrow P_l\}_{l\in L} :: a\!:\!\&\{l : A_l\}_{l\in L}$ and $\Delta'', a : \&\{l : A_l\}_{l\in L}\vdash a.k; R :: c\!:\!C$ by $\runa{TS-$\&$R}$ and $\runa{TS-$\&$L}$. By the premises of these rules we have that $\Delta' \vdash P_k :: a\!:\!A_k$ and $\Delta'',a : A_k\vdash R :: c\!:\!C$, and so it follows by $\runa{TS-cut}$ that $\Delta',\Delta''\vdash \newvar{a}{(P_k \mid R) :: c\!:\!C}$.
        
%     %
    
%     \item $\Delta'\vdash a.k; R :: a\!:\!\oplus\{l : A_l\}_{l\in L}$ and $\Delta'',a : \oplus\{l : A_l\}_{l\in L}\vdash a.\texttt{case}\{l\Rightarrow P_l\}_{l\in L} :: c\!:\!C$ by $\runa{TS-$\oplus$R}$ and $\runa{TS-$\oplus$L}$. By the premises of these rules we have that $\Delta'\vdash R :: a\!:\!A_k$ and $\Delta'',a : A_k\vdash P_k :: c\!:\!C$, and so it follows by $\runa{TS-cut}$ that $\Delta',\Delta''\vdash \newvar{a}{(R \mid P_k)} :: c\!:\!C$.
    
% \end{enumerate}

% %

% \item[$\runa{R-def}$] Assume $P$ reduces by $\runa{R-def}$ then $P = b \leftarrow f \leftarrow \widetilde{d}$ and $(\widetilde{c}:\widetilde{B}\vdash f = P' :: a\!:\!A) \in \Sigma$, such that $Q = P'[a\mapsto b,\widetilde{c}\mapsto\widetilde{d}]$. For $P$ to be well-typed it must be part of a larger process $\widetilde{d}:\widetilde{B},\Delta'\vdash \newvar{b}{(P \mid R)} :: c\!:\!C$ typed with $\runa{TS-def}$ such that $\Delta',b:A\vdash R :: c\!:\!C$. By Lemma \ref{lemma:substlem} we have that $\widetilde{d}:\widetilde{B}\vdash P'[a\mapsto b,\widetilde{c}\mapsto\widetilde{d}] :: b\!:\!B$ and so by $\runa{TS-cut}$ we have that $\widetilde{d}:\widetilde{B},\Delta'\vdash \newvar{b}{(Q \mid R)} :: c\!:\!C$.

% %

% \item[$\runa{R-res}$] Assume that $P$ reduces by $\runa{R-res}$ then we have that $P \equiv \newvar{a}{P'}$ for some name $a$ such that $P' \longrightarrow Q'$. Then $P$ must be typed either with $\runa{TS-cut}$ or $\runa{TS-def}$ and so $P' \equiv R' \mid R''$ yielding two cases
% \begin{enumerate}
%     \item $\Delta'\vdash R' :: a\!:\!A$ such that $\Delta'',a:A\vdash R'' :: c\!:\!C$ and $\Delta',\Delta''\vdash \newvar{a}{P'}::c\!:\!C$. Either $R' \mid R''$ reduces by $\runa{R-par}$, $\runa{R-comm}$, $\runa{R-choice}$ or $\runa{R-struct}$. The first three cases are covered by the clauses for the corresponding rules, and the last case holds by induction as typability is closed under structural congruence.
    
%     \item $R' = a \leftarrow f \leftarrow \widetilde{b}$ and $(\widetilde{e} : \widetilde{B}\vdash f = R :: g\!:\!A) \in \Sigma$ such that $\Delta' = \widetilde{b}:\widetilde{B}$, $\Delta'',a:A\vdash R'' :: c\!:\!C$ and $\Delta',\Delta''\vdash \newvar{a}{P'}::c\!:\!C$. Then either $R' \mid R''$ reduces by $\runa{R-par}$ or $\runa{R-struct}$. The first case is covered by the clause for $\runa{R-par}$, and the last case holds by induction as typability is closed under structural congruence.
% \end{enumerate}

% %

% \item[$\runa{R-par}$] Assume that $P$ reduces by $\runa{R-par}$ then we have that $P \equiv P' \mid P''$ such that $P' \longrightarrow Q'$. For $P$ to be well-typed, it must be part of a larger well-typed process $\newvar{a}{(P'\mid P'')}$ typed with either $\runa{TS-cut}$ or $\runa{TS-def}$ such that either
% \begin{enumerate}
%     \item $\Delta'\vdash P' :: a\!:\!A$ such that $\Delta'',a:A\vdash P'' :: c\!:\!C$ and $\Delta',\Delta''\vdash \newvar{a}{(P'\mid P'')}::c\!:\!C$. Then by induction we have that $\Delta'\vdash Q' :: a\!:\!A$ and so it follows that $\Delta',\Delta''\vdash \newvar{a}{(Q' \mid P'')}::c\!:\!C$
    
%     \item $P' = a \leftarrow f \leftarrow \widetilde{b}$ and $(\widetilde{e} : \widetilde{B}\vdash f = R :: g\!:\!A) \in \Sigma$ such that $\Delta' = \widetilde{b}:\widetilde{B}$, $\Delta'',a:A\vdash P'' :: c\!:\!C$ and $\widetilde{b}:\widetilde{B},\Delta''\vdash \newvar{a}{P' \mid P''}::c\!:\!C$. Then it must be that $P'$ reduces to $Q'$ by $\runa{TS-def}$ such that $Q' = R[g\mapsto a,\widetilde{e}\mapsto\widetilde{b}]$. By renaming $\widetilde{e} : \widetilde{B}\vdash R :: g\!:\!A$ implies $\widetilde{b} : \widetilde{B}\vdash Q' :: a\!:\!A$ such that $\widetilde{b}:\widetilde{B},\Delta''\vdash \newvar{a}{(Q' \mid P''):: c\!:\!C}$ by $\runa{T-cut}$.
% \end{enumerate}

%%%
%%
%%
%%%

%when they contain no named processes, for $P$ to be well-typed, $P$ must be a subprocess of a larger well-typed process $R \equiv \newvar{a}{\newvar{b}{P}} \equiv \newvar{a}{(\outputch{a}{d}{}{P'} \mid \newvar{b}{(\inputch{b}{v}{}{P''} \mid b \leftarrow a}))}$ such that $\Delta',\Delta''\vdash R :: c\!:\!C$. Then from the premises of $\runa{TS-cut}$, we have that $\Delta'',a:A\vdash \outputch{a}{d}{}{P'} ::c\!:\!C$ and (by $\runa{TS-cut}$ again) $\Delta'\vdash \newvar{b}{(\inputch{b}{v}{}{P''} \mid b \leftarrow a}) :: a\!:\!A$ such that $\Delta' \vdash \inputch{b}{v}{}{P''} :: b\!:\!A$ by $\runa{TS-$\multimap$R}$ and $b : A\vdash b \leftarrow a :: a\!:\!A$ by $\runa{TS-id}$. The full reduced process is then $\newvar{a}{\newvar{b}{(P' \mid P''[v\mapsto d])}}$

%
%%%%%%%%%%
%

% \item[$\runa{R-res}$] Assume that $P$ reduces by $\runa{R-res}$. Then for $P$ to be well-typed, $P$ must be typed by either $\runa{TS-cut}$ or $\runa{TS-def}$. We consider the cases separately
% \begin{description}
% \item[$\runa{TS-cut}$] We have that $P$ is of the form $\newvar{a}{(P'\mid P'')}$ such that $\Delta' \vdash P' :: a\!:\!A$, $\Delta'', a : A\vdash P'' :: c\!:\!C$ and $\Delta',\Delta'' \vdash \newvar{a}{(P'\mid P'')} :: c\!:\!C$. By $\runa{R-res}$ we have that $P' \mid P''$ must reduce, for which several rules apply
% \begin{description}
% \item[$\runa{R-comm}$] If we reduce the parallel composition by $\runa{R-comm}$ then $P' \mid P'' \equiv \inputch{a}{v}{}{R'} \mid \outputch{a}{b}{}{R''}$ for some name $b$ and processes $R'$ and $R''$, such that $\inputch{a}{v}{}{R'} \mid \outputch{a}{b}{}{R''} \longrightarrow R'[v\mapsto b] \mid R''$. We have two cases
% \begin{enumerate}
%     \item $\Delta' \vdash \inputch{a}{v}{}{R'} :: a\!:\!A' \multimap A''$ and $\Delta_3,a : A'\multimap A'', b : A' \vdash \outputch{a}{b}{}{R''} :: c\!:\!C$ by $\runa{TS-$\multimap$R}$ and $\runa{TS-$\multimap$L}$ such that $\Delta'' = \Delta_3,b:A'$. By the premises to these rules we have that $\Delta',v : A' \vdash R' :: a\!:\!A''$ and $\Delta_3,a:A''\vdash R'' :: c\!:\!C$. This implies $\Delta',b : A'\vdash R'[v\mapsto b] :: a\!:\!A''$, and so by $\runa{TS-cut}$ it follows that $(\Delta',b : A'),\Delta_3\vdash \newvar{a}{(R'[v\mapsto b] \mid R'') :: c\!:\!C}$ and $\Delta = (\Delta',b : A'),\Delta_3$.
    
%     %
    
%     \item $\Delta_3,b:A' \vdash \outputch{a}{b}{}{R''} :: a\!:\!A'\otimes A''$ and $\Delta'',a : A'\otimes A''\vdash \inputch{a}{v}{}{R'} :: c\!:\!C$ by $\runa{TS-$\otimes$R}$ and $\runa{TS-$\otimes$L}$ such that $\Delta' = \Delta_3,b:A'$. By the premises to these rules we have that $\Delta_3\vdash R'' :: a\!:\!A''$ and $\Delta'',a:A'',v:A'\vdash R' :: c\!:\!C$. This implies $\Delta'',a:A'',b:A'\vdash R'[v\mapsto b] :: c\!:\!C$, and so by $\runa{TS-cut}$ it follows that $\Delta_3,(\Delta'',b : A')\vdash \newvar{a}{(R'' \mid R'[v\mapsto b])} :: c\!:\!C$ and $\Delta = \Delta_3,(\Delta'',b : A')$.
% \end{enumerate}

% \item[$\runa{R-choice}$] If we reduce the parallel composition by $\runa{R-choice}$ then $P' \mid P'' \equiv a.\texttt{case}\{ l \Rightarrow P_l \}_{l\in L} \mid a.k; R$ for some label and set of labels $k$ and $L$, such that $k\in L$ and $a.\texttt{case}\{ l \Rightarrow P_l \}_{l\in L} \mid a.k; R \longrightarrow P_k \mid R$. We have two cases
% \begin{enumerate}
%     \item $\Delta'\vdash a.\texttt{case}\{l \Rightarrow P_l\}_{l\in L} :: a\!:\!\&\{l : A_l\}_{l\in L}$ and $\Delta'', a : \&\{l : A_l\}_{l\in L}\vdash a.k; R :: c\!:\!C$ by $\runa{TS-$\&$R}$ and $\runa{TS-$\&$L}$. By the premises of these rules we have that $\Delta' \vdash P_k :: a\!:\!A_k$ and $\Delta'',a : A_k\vdash R :: c\!:\!C$, and so it follows by $\runa{TS-cut}$ that $\Delta',\Delta''\vdash \newvar{a}{(P_k \mid R) :: c\!:\!C}$.
        
%     %
    
%     \item $\Delta'\vdash a.k; R :: a\!:\!\oplus\{l : A_l\}_{l\in L}$ and $\Delta'',a : \oplus\{l : A_l\}_{l\in L}\vdash a.\texttt{case}\{l\Rightarrow P_l\}_{l\in L} :: c\!:\!C$ by $\runa{TS-$\oplus$R}$ and $\runa{TS-$\oplus$L}$. By the premises of these rules we have that $\Delta'\vdash R :: a\!:\!A_k$ and $\Delta'',a : A_k\vdash P_k :: c\!:\!C$, and so it follows by $\runa{TS-cut}$ that $\Delta',\Delta''\vdash \newvar{a}{(R \mid P_k)} :: c\!:\!C$.
    
% \end{enumerate}

% \item[$\runa{R-id-1}$]
% \item[$\runa{R-id-2}$]
% \item[$\runa{R-par}$] If we reduce the parallel composition by $\runa{R-par}$ then $P' \longrightarrow Q'$. Here we can apply induction, as $P'$ cannot be typed as $\Delta' \vdash P' :: a\!:\!A$ and reduce unless it is prefixed by a tick or is wrapped with a restriction (or is structurally congruent to such a process by $\runa{R-struct}$). And so, it follows that $\Delta' \vdash Q' :: a\!:\!A$, such that $\Delta',\Delta'' \vdash \newvar{a}{(Q'\mid P'')} :: c\!:\!C$.
% \item[$\runa{R-struct}$] todo: induction (with R-par after).
% \end{description}
% \item[$\runa{TS-def}$] We have that $P$ is of the form $\newvar{a}{(a\leftarrow f \leftarrow \widetilde{b} \mid P')}$ such that $(\widetilde{d} : \widetilde{B}\vdash f = P :: g\!:\!A) \in \Sigma$, $\Delta',a : A \vdash P' :: c\!:\!C$ and $\Delta',\widetilde{b} : \widetilde{B}\vdash \newvar{a}{(a\leftarrow f \leftarrow \widetilde{b} \mid P') :: c\!:\!C}$. By $\runa{R-res}$ we have that $a\leftarrow f \leftarrow \widetilde{b} \mid P'$ must reduce, for which $\runa{R-par}$ and $\runa{R-struct}$ apply. Note that the parallel composition cannot reduce by $\runa{R-par}$, as  does not  several rules apply.
% \begin{description}
% \item[$\runa{R-par}$] todo: R-def --> can type with R-cut after.
% \item[$\runa{R-struct}$] todo: induction (with R-par after). 
% \end{description}
% \end{description}


% \item[$\runa{R-struct}$] Assume that $P$ reduces by $\runa{R-struct}$. Then $P \equiv P'$, $P' \longrightarrow Q'$ and $Q' \equiv Q$. As typability is closed under structural congruence and $\Delta \vdash P :: c\!:\!C$ it follows that $\Delta \vdash P' :: c\!:\!C$. By induction this implies $\Delta \vdash Q' :: c\!:\!C$, and as $Q' \equiv Q$ we have that $\Delta\vdash Q :: c\!:\!C$.
% \end{description}
\end{proof}
\end{theorem}
%
%
% \begin{lemma}
% For any session type $A$ for which $[A]^{-1}_R$ is defined, $\text{time}(A) - 1 \geq \text{time}([A]^{-1}_R)$.
% \begin{proof}
% by case analysis on $[A]^{-1}_R$
% \begin{description}
% \item[$\dasfwr{\ocircle A'}$] We have that $\dasfwr{\ocircle A'} = A'$ and $\text{time}(\ocircle A') = 1 + \text{time}(A')$. It follows that $\text{time}(\ocircle A') - 1 \geq \text{time}(A')$.

% \item[$\dasfwr{\lozenge A'}$] We have that $\dasfwr{\lozenge A'} = \lozenge A'$ and $\text{time}(\lozenge A') = \infty$. As $\infty - 1 = \infty$ it follows that $\text{time}(\lozenge A') - 1 \geq \text{time}(\lozenge A')$.
% \end{description}
% \end{proof}
% \end{lemma}

\begin{lemma}
Let $\hat{A}[A]$ and $\hat{A}[[A]^{-1}_R]$ be session types then $\text{time}(\hat{A}[A])-1\geq\text{time}(\hat{A[[A]^{-1}_R]})$.
\begin{proof}
On the shape of $\hat{A}[\cdot]$. By definition $\hat{A}[\cdot]$ is a prefix of modalities. If the prefix contains an $\lozenge$ or $\Box$ modality, then $\text{time}(\hat{A}[A])=\text{time}(\hat{A}[[A]^{-1}_R])=\infty$ for any two session types $A$ and $[A]^{-1}_R$. As $\infty-1 = \infty$ we obtain $\text{time}(\hat{A}[A])-1\geq\text{time}(\hat{A}[[A]^{-1}_R])$. Otherwise, $\hat{A}[\cdot]$ only contains $\ocircle$ modalities, and so $\text{time}(\hat{A}[A])$ is equal to $\text{time}(A)$ plus the count of $\ocircle$ modalities in $\hat{A}[\cdot]$ which is constant. Then it remains to show that $\text{time}(A)-1\geq\text{time}([A]^{-1}_R)$. As $[A]^{-1}_R$ is defined we have that either
\begin{enumerate}
    \item $A=\dasfwr{\ocircle A'}$ with $\dasfwr{\ocircle A'} = A'$ and $\text{time}(\ocircle A') = 1 + \text{time}(A')$. It follows that $\text{time}(\ocircle A') - 1 \geq \text{time}(A')$.
    
    \item $A=\dasfwr{\lozenge A'}$ with $\dasfwr{\lozenge A'} = \lozenge A'$ and $\text{time}(\lozenge A') = \infty$. As $\infty - 1 = \infty$ it follows that $\text{time}(\lozenge A') - 1 \geq \text{time}(\lozenge A')$.
\end{enumerate}
% \item[$\dasfwr{\ocircle A'}$] We have that $\dasfwr{\ocircle A'} = A'$ and $\text{time}(\ocircle A') = 1 + \text{time}(A')$. It follows that $\text{time}(\ocircle A') - 1 \geq \text{time}(A')$.

% \item[$\dasfwr{\lozenge A'}$] We have that $\dasfwr{\lozenge A'} = \lozenge A'$ and $\text{time}(\lozenge A') = \infty$. As $\infty - 1 = \infty$ it follows that $\text{time}(\lozenge A') - 1 \geq \text{time}(\lozenge A')$.
% \end{description}
\end{proof}
\end{lemma}

\begin{lemma}
If $\hat{B}[B]\;\texttt{delayed}^\Box$ then also $\hat{B}[[B]^{-1}_L]\;\texttt{delayed}^\Box$ and if $\hat{A}[A]\;\texttt{delayed}^\lozenge$ then also $\hat{A}[[A]^{-1}_R]\;\texttt{delayed}^\lozenge$.
\begin{proof}
On the shapes of $\hat{B}[B]$, $\hat{B}[[B]^{-1}_L]$, $\hat{A}[A]$ and $\hat{A}[[A]^{-1}_R]$. We consider $\texttt{delayed}^\Box$ and $\texttt{delayed}^\lozenge$ separately
\begin{enumerate}
    \item If $\hat{B}[B]\;\texttt{delayed}^\Box$ then either $\hat{B}[\cdot]=\ocircle^*\Box$ or $\hat{B}[\cdot]=\ocircle^*$ and $B=\ocircle^*\Box$. The first case is obtained directly and the second case holds by the fact that $[\cdot]^{-1}_L$ preserves $\Box$ by definition.
    
    \item If $\hat{A}[A]\;\texttt{delayed}^\lozenge$ then either $\hat{A}[\cdot]=\ocircle^*\lozenge$ or $\hat{A}[\cdot]=\ocircle^*$ and $A=\ocircle^*\lozenge$. The first case is obtained directly and the second case holds by the fact that $[\cdot]^{-1}_R$ preserves $\lozenge$ by definition.
    
\end{enumerate}
\end{proof}
\end{lemma}


% \begin{lemma}
% Let $P$ be an arbitrary process.
% \begin{enumerate}
%     \item If $\Delta,a:\hat{A}[\lozenge A']\vdash P :: c\!:\!C$ then there exists $\hat{\Delta'}[\Delta']=\Delta$ and $\hat{C'}[C']$ such that $\Delta'\;\texttt{delayed}^\Box$ and $C'\;\texttt{delayed}^\lozenge$.
    
%     \item If $\Delta\vdash P :: a\!:\!\hat{A}[\Box A']$ then there exists $\hat{\Delta'}[\Delta']=\Delta$ such that $\Delta'\;\texttt{delayed}^\Box$.
% \end{enumerate}
% \begin{proof}
% By induction on the type rules. We only show the interesting cases
% \begin{description}
% \item[$\runa{TS-$\ocircle$LR'}$] Consider first (1).\\

% Consider then (2). If $P$ is well-typed with $\runa{TS-$\ocircle$LR'}$ then $P = \tick P'$ such that $\Delta\vdash \tick P' :: a\!:\!\hat{A}[\Box A']$ and $[\Delta]^{-1}_L\vdash P' :: a\!:\![\hat{A}[\Box A']^{-1}_R]$. By induction we have $\hat{\Delta'}[\Delta']=[\Delta]^{-1}_L$ such that $\Delta'\;\texttt{delayed}^\Box$. Then there also exists $\hat{\Delta''}[\Delta']=\Delta$.

% \item[$\runa{TS-$\ocircle$LR}$] Consider first (1).\\

% Consider then (2). If $P$ is well-typed with $\runa{TS-$\ocircle$LR}$ then $\Delta\vdash P :: a\!:\!\hat{A}[\Box A']$ and $[\Delta]^{-1}_L\vdash P :: a\!:\![\hat{A}[\Box A']^{-1}_R]$. By induction we have $\hat{\Delta'}[\Delta']=[\Delta]^{-1}_L$ such that $\Delta'\;\texttt{delayed}^\Box$. Then there also exists $\hat{\Delta''}[\Delta']=\Delta$.

% \item[$\runa{TS-$\lozenge$L}$] Consider first (1).\\

% Consider then (2). If $P$ is well-typed with $\runa{TS-$\lozenge$L}$ then $\Delta,b:\lozenge B \vdash P :: a\!:\!\hat{A}[\Box A']$ and $\Delta,b:B\vdash P :: a\!:\!\hat{A}[\Box A']$. By induction we have $\hat{\Delta'}[\Delta'],b:\hat{B'}[B']=\Delta,b:B$ such that $\Delta',b:B'\;\texttt{delayed}^\Box$. Then there also exists $\hat{\Delta'}[\Delta'],b:\lozenge\hat{B'}[B']=\Delta,b:\lozenge B$.

% \item[$\runa{TS-$\lozenge$R}$] Consider first (1).\\

% Consider then (2). If $P$ is well-typed with $\runa{TS-$\lozenge$R}$ then $\Delta \vdash P :: a\!:\!\lozenge\hat{A}[\Box A']$ and $\Delta \vdash P :: a\!:\!\hat{A}[\Box A']$. By induction we have $\hat{\Delta'}[\Delta']=\Delta$ such that $\Delta'\;\texttt{delayed}^\Box$. 

% \item[$\runa{TS-$\Box$L}$] Consider first (1).\\

% Consider then (2). If $P$ is well-typed with $\runa{TS-$\Box$L}$ then $\Delta,b:\Box B \vdash P :: a\!:\!\hat{A}[\Box A']$ and $\Delta,b:B \vdash P :: a\!:\!\hat{A}[\Box A']$. By induction we have $\hat{\Delta'}[\Delta'],b:\hat{B'}[B']=\Delta,b:B$ such that $\Delta',b:B'\;\texttt{delayed}^\Box$. Then there also exists $\hat{\Delta'}[\Delta'],b:\Box\hat{B'}[B']=\Delta,b:\Box B$. 

% \item[$\runa{TS-$\Box$R}$] Consider first (1).\\

% Consider then (2).

% \item[$\runa{TS-cut}$] Consider first (1).\\

% Consider then (2). 

% \end{description}
% \end{proof}
% \end{lemma}


% \begin{lemma}
% If $\Delta\vdash P :: a\!:\!A$ such that $P$ is not prefixed on $a$ and $A$ contains no $\lozenge$ with $P \Longrightarrow^{-1} Q$ such that $P \neq Q$ and $P\!\not\!\leadsto$ then there exists $\hat{\Delta'}[\Delta']=\Delta$ and $\hat{A'}[A']=A$ such that $\hat{\Delta'}[[\Delta']^{-1}_L]\vdash Q :: a\!:\!\hat{A'}[[A']^{-1}_R]$.
% \begin{proof}
% By induction on the type rules. We need not consider type rule $\runa{TS-def}$ as that would imply $P\!\!\leadsto$ by $\runa{R-res}$, $\runa{R-par}$ and $\runa{R-def}$. We also need not consider type rules for process prefixes except for $\runa{TS-$\ocircle$LR'}$, as $P\Longrightarrow^{-1} Q$ is productive iff at least one tick is not prefixed. We consider the cases
% \begin{description}
% \item[$\runa{TS-$\ocircle$LR'}$] We have that $\Delta\vdash \tick P' :: a\!:\!A$ such that $[\Delta]^{-1}_L\vdash P' :: a\!:\![A]^{-1}_R$. Then as $Q = P'$, we obtain $\hat{\Delta'}[[\Delta']^{-1}_L]\vdash Q :: a\!:\!\hat{A'}[[A']^{-1}_R]$ where $\hat{A'}=[\cdot]$, $\text{dom}(\Delta)=\text{dom}(\hat{\Delta'})$ and for $b\in\text{dom}(\hat{\Delta'})$ we have $\hat{\Delta'}(b)=[\cdot]$.

% \item[$\runa{TS-$\ocircle$LR}$] We have that $\Delta\vdash P :: a\!:\!A$ and $[\Delta]^{-1}_L\vdash P :: a\!:\![A]^{-1}_R$. By induction there exists $\hat{\Delta'}[\Delta']=[\Delta]^{-1}_L$ and $\hat{A'}[A']=[A]^{-1}_R$ such that $\hat{\Delta'}[[\Delta']^{-1}_L]\vdash Q :: a\!:\!\hat{A'}[[A']^{-1}_R]$. As $\hat{\Delta'}[\Delta']=[\Delta]^{-1}_L$ and $\hat{A'}[A']=[A]^{-1}_R$ it must be that $\hat{\Delta'}[\Delta']=[\hat{\Delta''}[\Delta'']]^{-1}_L$ and $\hat{A'}[A']=[\hat{A''}[A'']]^{-1}_R$ for some $\hat{\Delta''}[\Delta'']$ and $\hat{A''}[A'']$ such that also $[\hat{\Delta''}[[\Delta'']^{-1}_L]]^{-1}_L\vdash Q :: a\!:\![\hat{A''}[[A'']^{-1}_R]]^{-1}_R$. It follows from $\runa{TS-$\ocircle$LR}$ that also $\hat{\Delta''}[[\Delta'']^{-1}_L]\vdash Q :: a\!:\!\hat{A''}[[A'']^{-1}_R]$.

% \item[$\runa{TS-$\lozenge$L}$] We have that $\Delta,a:\lozenge A\vdash P :: b\!:\!B$, such that $\Delta\;\texttt{delayed}^\Box$, $B\;\texttt{delayed}^\lozenge$ and $\Delta,a:A\vdash P :: b\!:\!B$. By induction there exists $\hat{\Delta'}[\Delta'],a:\hat{A'}[A']=\Delta,a:A$ and $\hat{B'}[B']=B$ such that $\hat{\Delta'}[[\Delta']^{-1}_L],a:\hat{A'}[[A']^{-1}_L]\vdash Q :: b\!:\!\hat{B'}[[B']^{-1}_R]$ and by Lemma \ref{lemma:progdel} $\hat{\Delta'}[\Delta']\;\texttt{delayed}^\Box$ implies $\hat{\Delta'}[[\Delta']^{-1}_L]\;\texttt{delayed}^\Box$ and $\hat{B'}[B']\;\texttt{delayed}^\lozenge$ implies $\hat{B'}[[B']^{-1}_R]\;\texttt{delayed}^\lozenge$ and so by $\runa{TS-$\lozenge$L}$ we obtain $\hat{\Delta'}[[\Delta']^{-1}_L],a:\lozenge\hat{A'}[[A']^{-1}_L]\vdash Q :: b\!:\!\hat{B'}[[B']^{-1}_R]$.

% \item[$\runa{TS-$\lozenge$R}$] We have that $\Delta\vdash P :: a\!:\!\lozenge A$ and $\Delta\vdash P :: a\!:\!A$. By induction there exists $\hat{\Delta'}[\Delta']=\Delta$ and $\hat{A'}[A']=A$ such that $\hat{\Delta'}[[\Delta']^{-1}_L]\vdash Q :: a\!:\!\hat{A'}[[A']^{-1}_R]$. It follows directly from $\runa{TS-$\lozenge$R}$ that also $\hat{\Delta'}[[\Delta']^{-1}_L]\vdash Q :: a\!:\!\lozenge\hat{A'}[[A']^{-1}_R]$.

% \item[$\runa{TS-$\Box$L}$] We have that $\Delta,a:\Box A\vdash P :: b\!:\!B$ and $\Delta,a:A\vdash P :: b\!:\!B$. By induction there exists $\hat{\Delta'}[\Delta'],a:\hat{A'}[A']=\Delta,a:A$ and $\hat{B'}[B']=B$ such that $\hat{\Delta'}[[\Delta']^{-1}_L],a:\hat{A'}[[A']^{-1}_L]\vdash Q :: b\!:\!\hat{B'}[[B']^{-1}_R]$. It follows directly from $\runa{TS-$\Box$L}$ that also $\hat{\Delta'}[[\Delta']^{-1}_L],a:\Box\hat{A'}[[A']^{-1}_L]\vdash Q :: b\!:\!\hat{B'}[[B']^{-1}_R]$.

% \item[$\runa{TS-$\Box$R}$] We have that $\Delta\vdash P :: a\!:\!\Box A$, $\Delta\;\texttt{delayed}^\Box$ and $\Delta\vdash P :: a\!:\!A$. By induction there exists $\hat{\Delta'}[\Delta']=\Delta$ and $\hat{A'}[A']=A$ such that $\hat{\Delta'}[[\Delta']^{-1}_L]\vdash Q :: a\!:\!\hat{A'}[[A']^{-1}_R]$ and from Lemma \ref{lemma:progdel} $\hat{\Delta'}[[\Delta']^{-1}_L]\;\texttt{delayed}^\Box$ follows from $\hat{\Delta'}[\Delta']\;\texttt{delayed}^\Box$, and so by $\runa{TS-$\Box$R}$ we obtain $\hat{\Delta'}[[\Delta']^{-1}_L]\vdash Q :: a\!:\!\Box\hat{A'}[[A']^{-1}_R]$.

% \item[$\runa{TS-cut}$] We have that $\Delta_1,\Delta_2\vdash \newvar{a}{(P'\mid 
% P'')} :: c\!:\!C$ with $\Delta_1\vdash P' :: a\!:\!A$ and $\Delta_2,a:A\vdash P'' :: c\!:\!C$. We identify three cases where $P \Longrightarrow^{-1} Q$ is productive
% \begin{enumerate}
%     \item $\newvar{a}{(P'\mid P'')} \Longrightarrow^{-1} \newvar{a}{(Q' \mid P'')}$ with $P' \neq Q'$. For $P$ to not be prefixed on $c$, $P''$ also cannot be prefixed on $c$. Then as $P'' \Longrightarrow^{-1} P''$ and $P\!\not\!\leadsto$ the subprocess that provides a session on $A$ in $P'$ must be prefixed with a tick or be $\mathbf{0}$. Thus, $P'$ cannot be prefixed on $a$. Then by induction there exists $\hat{\Delta_1'}[\Delta_1']=\Delta_1$ and $\hat{A'}[A']=A$ such that $\hat{\Delta_1'}[[\Delta_1']^{-1}_L]\vdash Q' :: a\!:\!\hat{A'}[[A']^{-1}_R]$. As $\hat{A'}[[A']^{-1}_R]$ is defined it must be that either
%     \begin{itemize}
%         \item $\hat{A'}[A']=\hat{A'}[\lozenge A'']$ for some session type $A''$ and so $\hat{A'}[A']=\hat{A'}[[A']^{-1}_R]$. From this we obtain $\Delta_2,a:\hat{A'}[\lozenge A'']\vdash P'' :: c\!:\!C$. However, by Lemma \ref{lemma:deldiaimp} $C$ then contains an $\lozenge$ modality, contradicting our assumption. %there exists $\Delta_2=\hat{\Delta_2'}[\Delta_2']$ and $C=\hat{C'}[C']$ such that $\Delta_2'\;\texttt{delayed}^\Box$ and $C'\;\texttt{delayed}^\lozenge$. Then $C'=\ocircle^n\lozenge C''$ for some $n\geq 0$ and session type $C''$ and so we obtain $C=\hat{C'}[\ocircle^n[\lozenge C'']]=\hat{C'}[\ocircle^n[[\lozenge C'']^{-1}_R]]=\hat{C''}[[\lozenge C'']^{-1}_R]$ for some $\hat{C''}[\cdot]$, and for $b\in\text{dom}(\Delta_2')$ we have $\Delta_2'(b)=\ocircle^m\Box B'$ for some $m\geq 0$ and session type $B'$ and so $\Delta_2'(b)=\ocircle^m[\Box B']=\ocircle^m[[\Box B']^{-1}_L]$. Thus, there exists $\hat{\Delta_2''}[\Delta_2'']=\Delta_2$ such that $\hat{\Delta_2''}[[\Delta_2'']^{-1}_L],\hat{A'}[[A']^{-1}_R]\vdash P'' :: c\!:\!\hat{C''}[[\lozenge C'']^{-1}_R]$. From $\runa{TS-cut}$ we then obtain $\hat{\Delta_1'}[[\Delta_1']^{-1}_L],\hat{\Delta_2''}[[\Delta_2'']^{-1}_L]\vdash \newvar{a}{(Q'\mid P'') :: c\!:\!\hat{C''}[[\lozenge C'']^{-1}_R]}$.
        
%         \item $\hat{A'}[A']=\hat{A'}[\ocircle A'']$ for some session type $A''$ and so $\hat{A'}[[\ocircle A'']^{-1}_R] = \hat{A'}[A'']$. As $P''$ is not prefixed on $c$ and $\Delta_2,a:\hat{A'}[\ocircle A'']\vdash P'' :: c\!:\!C$ with $P''\Longrightarrow^{-1} P''$, $C$ must be prefixed with at least one modality typed with $\runa{TS-$\ocircle$LR}$ corresponding to the $\ocircle$ modality in the prefix of $A$. This implies $\Delta_2=\hat{\Delta_2'}[\Delta_2']$ and $C=\hat{C'}[C']$ such that $\hat{\Delta_2'}[[\Delta_2']^{-1}_L],a:\hat{A'}[A'']\vdash P'' :: c\!:\!\hat{C'}[[C']^{-1}_R]$, and so it follows from $\runa{TS-cut}$ that $\hat{\Delta_1'}[[\Delta_1']^{-1}_L],\hat{\Delta_2'}[[\Delta_2']^{-1}_L]\vdash \newvar{a}{(Q'\mid P'') :: c\!:\!\hat{C'}[[C']^{-1}_R]}$.
%     \end{itemize}
    
%     \item $\newvar{a}{(P'\mid P'')} \Longrightarrow^{-1} \newvar{a}{(P' \mid Q'')}$ with $P'' \neq Q''$. For $P$ to not be prefixed on $c$ $P''$ also cannot be prefixed on $c$, and $C$ contains no $\lozenge$ by assumption. Then by induction there exists $\hat{\Delta_2'}[\Delta_2'],a:\hat{A''}[A'']=\Delta_2,a:A$ and $\hat{C'}[C']=C$ such that $\hat{\Delta_2'}[[\Delta_2']^{-1}_L],a:\hat{A''}[[A'']^{-1}_L]\vdash P'' :: c\!:\!\hat{C'}[[C']^{-1}_R]$. As $\hat{A'}[[A']^{-1}_L]$ is defined it must be that either
%     \begin{itemize}
%         \item $\hat{A'}[A']=\hat{A'}[\Box A'']$ for some session type $A''$ and so $\hat{A'}[A']=\hat{A'}[[A']^{-1}_L]$. 
        
%         Then as $P' \Longrightarrow^{-1} P'$ and $P\!\not\!\leadsto$ the subprocess that consumes a session on $A$ in $P'$ must be prefixed with a tick or be $\mathbf{0}$. Thus, $P''$ cannot be prefixed on $a$.
        
%         From this we obtain $\Delta_1\vdash P' :: a\!:\!\hat{A'}[[A']^{-1}_L]$. Then by Lemma \ref{lemma:deldiaimp} there exists $\Delta_1=\hat{\Delta_1'}[\Delta_1']$ such that $\Delta_1'\;\texttt{delayed}^\Box$. Then for $b\in\text{dom}(\Delta_1')$ we have $\Delta_1'(b)=\ocircle^m\Box B'$ for some $m\geq 0$ and session type $B'$ and so $\Delta_1'(b)=\ocircle^m[\Box B']=\ocircle^m[[\Box B']^{-1}_L]$. Thus, there exists $\hat{\Delta_1''}[\Delta_1'']=\Delta_1$ such that $\hat{\Delta_1''}[[\Delta_1'']^{-1}_L]\vdash P' :: a\!:\!\hat{A'}[[A']^{-1}_L]$. From $\runa{TS-cut}$ we then obtain $\hat{\Delta_1''}[[\Delta_1'']^{-1}_L],\hat{\Delta_2'}[[\Delta_2']^{-1}_L]\vdash \newvar{a}{(P'\mid Q'') :: c\!:\!\hat{C''}[[\lozenge C'']^{-1}_R]}$.
        
%         \item $\hat{A'}[A']=\hat{A'}[\ocircle A'']$ for some session type $A''$ and so $\hat{A'}[[\ocircle A'']^{-1}_L] = \hat{A'}[A'']$. As $\Delta_1\vdash P' :: a:\hat{A'}[\ocircle A'']$ with $P' \Longrightarrow^{-1} P'$, we must use $\runa{TS-$\ocircle$LR}$ to consume the $\ocircle$ modality. This implies $\Delta_1=\hat{\Delta_1'}[\Delta_1']$ such that $\hat{\Delta_1'}[[\Delta_1']^{-1}_L] \vdash P' :: a:\hat{A'}[A'']$, and so it follows from $\runa{TS-cut}$ that $\hat{\Delta_1'}[[\Delta_1']^{-1}_L],\hat{\Delta_2'}[[\Delta_2']^{-1}_L]\vdash \newvar{a}{(P'\mid Q'') :: c\!:\!\hat{C'}[[C']^{-1}_R]}$.
        
%     \end{itemize}
    
%     \item $\newvar{a}{(P'\mid P'')} \Longrightarrow^{-1} \newvar{a}{(Q' \mid Q'')}$ with $P' \neq Q'$ and $P'' \neq Q''$. For $P$ to not be prefixed on $c$ $P''$ also cannot be prefixed on $c$. Then by induction there exists $\hat{\Delta_2'}[\Delta_2'],a:\hat{A_2}[A_2]=\Delta_2,a:A$ and $\hat{C'}[C']=C$ such that $\hat{\Delta_2'}[[\Delta_2']^{-1}_L],a:\hat{A_2}[[A_2]^{-1}_L]\vdash Q'' :: c\!:\!\hat{C'}[[C']^{-1}_R]$.
    
%     As $P\!\not\!\leadsto$ either $P'$ or $P''$ is not prefixed on $a$. We consider the cases
%     \begin{itemize}
%         \item $P'$ is not prefixed on $a$. Then by induction we have $\hat{\Delta_1'}[\Delta_1']=\Delta_1$ and $\hat{A_1}[A_1]=A$ such that $\hat{\Delta_1'}[[\Delta_1']^{-1}_L]\vdash Q' :: a\!:\!\hat{A_1}[[A_1]^{-1}_R]$. As $\hat{A_1}[[A_1]^{-1}_R]$ and $\hat{A_2}[[A_2]^{-1}_L]$ are defined it must be that either
        
%         \item
        
%         \item
%     \end{itemize}
    
%     \begin{itemize}
%         %\item $\hat{A_1}[A_1]=\hat{A_1}[\lozenge A_1']$ and $\hat{A_2}[A_2]=\hat{A_2}[\Box A_2']$ for some session types $A_1'$ and $A_2'$. We obtain $\hat{\Delta_1'}[[\Delta_1']^{-1}_L]\vdash Q' :: a\!:\!\hat{A_2}[[\Box A_2']^{-1}_L]$ directly from $\hat{A_1}[[\lozenge A_1']^{-1}_R]=\hat{A_1}[A_1]$ and $\hat{A_2}[[\Box A_2']^{-1}_L]=\hat{A_2}[A_2]$ as $\hat{A_1}[A_1]=\hat{A_2}[A_2]$.
        
%         \item $\hat{A_1}[A_1]=\hat{A_1}[\lozenge A_1']$ and so $\hat{A_1}[[\lozenge A_1']^{-1}_R]=\hat{A_1}[A_1]$. From this we obtain $\Delta_2,a:\hat{A_1}[[\lozenge A_1']^{-1}_R]\vdash P'' :: c\!:\!C$ and by Theorem \ref{theorem:sr} $\Delta_2,a:\hat{A_1}[[\lozenge A_1']^{-1}_R]\vdash Q'' :: c\!:\!C$. Then by Lemma \ref{lemma:deldiaimp} there exists $\hat{\Delta_2''}[\Delta_2'']=\Delta_2$ and $\hat{C''}[C'']=C$ such that $\Delta_2''\;\texttt{delayed}^\Box$ and $C''\;\texttt{delayed}^\lozenge$. Then $C''=\ocircle^n\lozenge C_3$ for some $n\geq 0$ and session type $C_3$ and so we obtain $C=\hat{C''}[\ocircle^n[\lozenge C_3]]=\hat{C''}[\ocircle^n[[\lozenge C_3]^{-1}_R]]=\hat{C_3}[[\lozenge C_3]^{-1}_R]$ for some $\hat{C_3}[\cdot]$, and for $b\in\text{dom}(\Delta_2'')$ we have $\Delta_2''(b)=\ocircle^m\Box B'$ for some $m\geq 0$ and session type $B'$ and so $\Delta_2''(b)=\ocircle^m[\Box B']=\ocircle^m[[\Box B']^{-1}_L]$. Thus, there exists $\hat{\Delta_3}[\Delta_3]=\Delta_2$ such that $\hat{\Delta_3}[[\Delta_3]^{-1}_L],\hat{A_1}[[A_1]^{-1}_R]\vdash Q'' :: c\!:\!\hat{C_3}[[\lozenge C_3]^{-1}_R]$. From $\runa{TS-cut}$ we then obtain $\hat{\Delta_1'}[[\Delta_1']^{-1}_L],\hat{\Delta_3}[[\Delta_3]^{-1}_L]\vdash \newvar{a}{(Q'\mid Q'') :: c\!:\!\hat{C_3}[[\lozenge C_3]^{-1}_R]}$.
        
%         %As $\hat{A_1}[A_1]=\hat{A_2}[A_2]=A$ for $\Delta_1\vdash P' :: a\!:\!A$ to hold, the $\ocircle$ modality removed from $\hat{A_2}[A_2']$ must be typed with $\runa{TS-$\ocircle$LR}$ for $P'$. By premise there must then exist $\hat{\Delta_1''}[\Delta_1'']=\Delta_1$ such that $\hat{\Delta_1''}[[\Delta_1'']^{-1}_L]\vdash P' :: a\!:\!\hat{A_2}[A_2']$, but then we could have used $\runa{TS-$\ocircle$LR'}$ here instead, from which we obtain $\hat{\Delta_1''}[[\Delta_1'']^{-1}_L]\vdash Q' :: a\!:\!\hat{A_2}[A_2']$.
        
%         \item $\hat{A_2}[A_2]=\hat{A_2}[\Box A_2']$ and so $\hat{A_2}[[\Box A_2']^{-1}_L]=\hat{A_2}[A_2]$. From this we obtain $\Delta_1\vdash P' :: a\!:\!\hat{A_2}[[\Box A_2']^{-1}_L]$ and by Theorem \ref{theorem:sr} $\Delta_1\vdash Q' :: a\!:\!\hat{A_2}[[\Box A_2']^{-1}_L]$. Then by Lemma \ref{lemma:deldiaimp} there exists $\hat{\Delta_1''}[\Delta_1'']=\Delta_1$ such that $\Delta_1''\;\texttt{delayed}^\Box$. Then for $b\in\text{dom}(\Delta_1'')$ we have $\Delta_1''(b)=\ocircle^m\Box B'$ for some $m\geq 0$ and session type $B'$ and so $\Delta_1''(b)=\ocircle^m[\Box B']=\ocircle^m[[\Box B']^{-1}_L]$. Thus, there exists $\hat{\Delta_3}[\Delta_3]=\Delta_1$ such that $\hat{\Delta_3}[[\Delta_3]^{-1}_L]\vdash Q' :: a\!:\!\hat{A_2}[[A_2]^{-1}_L]$. From $\runa{TS-cut}$ we then obtain $\hat{\Delta_3}[[\Delta_3]^{-1}_L],\hat{\Delta_2'}[[\Delta_2']^{-1}_L]\vdash \newvar{a}{(Q'\mid Q'') :: c\!:\!\hat{C''}[[\lozenge C'']^{-1}_R]}$.
        
        
        
%         %As $\hat{A_1}[A_1]=\hat{A_2}[A_2]=A$ for $\Delta_2,a:A\vdash P'' :: c\!:\!C$ to hold, the $\ocircle$ modality removed from $\hat{A_1}[A_1']$ must be typed with $\runa{TS-$\ocircle$LR}$ for $P''$. By premise there must then exist $\hat{\Delta_2''}[\Delta_2'']=\Delta_2$ and $\hat{C''}[C'']=C$ such that $\hat{\Delta_2''}[[\Delta_2'']^{-1}_L],a:\hat{A_1}[A_1']\vdash P'' :: a\!:\!\hat{C''}[[C'']^{-1}_R]$, but then we could have used $\runa{TS-$\ocircle$LR'}$ here instead, from which we obtain $\hat{\Delta_2''}[[\Delta_2'']^{-1}_L],a:\hat{A_1}[A_1']\vdash Q'' :: a\!:\!\hat{C''}[[C'']^{-1}_R]$.
        
%         \item $\hat{A_1}[A_1]=\hat{A_1}[\ocircle A_1']$ and $\hat{A_2}[A_2]=\hat{A_2}[\ocircle A_2']$ for some session types $A_1'$ and $A_2'$ and so $\hat{A_1}[[\ocircle A_1']^{-1}_R]=\hat{A_1}[A_1']$ and $\hat{A_2}[[\ocircle A_2']^{-1}_L]=\hat{A_2}[A_2']$. Either $\hat{A_1}[A_1']=\hat{A_2}[A_2']$ and we obtain $\hat{\Delta_2'}[[\Delta_2']^{-1}_L],a:\hat{A_1}[A_1']\vdash Q'':: c\!:\!\hat{C'}[[C']^{-1}_R]$ directly, or there is at least one $\Box$ or $\lozenge$ modality between the two $\ocircle$ modalities removed from $A_1$ and $A_2$, respectively. Then the remaining $\ocircle$ modality in $Q'$ and $Q''$ must be typed with either $\runa{TS-$\ocircle$LR'}$ or $\runa{TS-$\ocircle$LR}$. As $\Box$ and $\lozenge$ are not syntax directed, and as $\runa{TS-$\ocircle$LR'}$ and $\runa{TS-$\ocircle$LR}$ do not affect $\texttt{delayed}^\Box$ and $\texttt{delayed}^\lozenge$, by definition of $[\cdot]^{-1}_L$ and $[\cdot]^{-1}_R$, we can rearrange the prefixes of modalities in $\hat{A_1}[[\ocircle A_1']^{-1}_R]$ and $\hat{A_2}[[\ocircle A_2']^{-1}_L]$ without affecting typability such that we have $\hat{A_1'}[[\ocircle A_1'']^{-1}_R]=\hat{A_2'}[[\ocircle A_2'']^{-1}_L]$ with $\hat{\Delta_1'}[[\Delta_1']^{-1}_L]\vdash Q' :: a\!:\hat{A_1}[[\ocircle A_1']^{-1}_R]$, and $\hat{\Delta_2'}[[\Delta_2']^{-1}_L],a:\hat{A_2}[[\ocircle A_2']^{-1}_L]\vdash Q'' :: c\!:\!\hat{C'}[[C']^{-1}_R]$. Thus, by application of $\runa{TS-cut}$ we obtain $\hat{\Delta_1'}[[\Delta_1']^{-1}_L],\hat{\Delta_2'}[[\Delta_2']^{-1}_L]\vdash \newvar{a}{(Q' \mid Q'')} :: c\!:\!\hat{C'}[[C']^{-1}_R]$.
        
        
%         %there are only $\ocircle$ modalities between the two modalities removed, without affecting typability.  move the $\Box$ and $\lozenge$ modalities between the two $\ocircle$ modalities outward,  
        
        
%         %two distinct $\ocircle$ modalities were removed. Then as $\hat{A_1}[A_1]=\hat{A_2}[A_2]$ for $\Delta_2,a:A\vdash P'' :: c\!:\!C$ to hold, the $\ocircle$ modality must be typed with $\runa{TS-$\ocircle$LR}$. By premise there must then exist $\hat{\Delta_2''}[\Delta_2'']=\Delta_2$ and $\hat{C''}[C'']=C$ such that $\hat{\Delta_2''}[[\Delta_2'']^{-1}_L],a:\hat{A_1}[A_1']\vdash P'':: c\!:\!\hat{C''}[[C'']^{-1}_R]$, but then we could have used $\runa{TS-$\ocircle$LR'}$ here instead, from which we obtain $\hat{\Delta_2''}[[\Delta_2'']^{-1}_L],a:\hat{A_1}[A_1']\vdash Q'':: c\!:\!\hat{C''}[[C'']^{-1}_R]$.
        
%     \end{itemize}
    
% \end{enumerate}

% \end{description}
% %As $P \Longrightarrow Q$ we know that $P$ cannot be prefixed with anything except for a tick, and so it is sufficient to consider $\runa{TS-$\ocircle$LR'}$, $\runa{TS-cut}$, $\runa{TS-def}$, $\runa{TS-$\ocircle$LR}$, $\runa{TS-$\lozenge$L}$, $\runa{TS-$\lozenge$R}$, $\runa{TS-$\Box$L}$ and $\runa{TS-$\Box$R}$
% % \begin{description}
% % \item[$\runa{TS-$\ocircle$LR'}$] We have that $P=\tick P'$, $Q=P'$ and $[\Delta]^{-1}_L\vdash P' :: a\!:\![A]^{-1}_R$. By lemma \ref{lemma:timegeq} $\text{time}(A)-1\geq\text{time}([A]^{-1}_R)$, and by Lemma \ref{TODO}, $\texttt{delayed}^\Box$ and $\texttt{delayed}^\lozenge$ are invariant to $[\cdot]^{-1}_L$ and $[\cdot]^{-1}_R$, respectively.

% % \item[$\runa{TS-cut}$]

% % %

% % \item[$\runa{TS-def}$]

% % %

% % \item[$\runa{TS-$\ocircle$LR}$] If $\Delta \vdash P :: a\!:\!A$ by $\runa{TS-$\ocircle$LR}$ then $[\Delta]^{-1}_L\vdash P :: a\!:\![A]^{-1}_R$. By Lemma \ref{lemma:timegeq} $\text{time}(A)-1\geq\text{time}([A]^{-1}_R)$ and by Lemma \ref{TODO}, $\texttt{delayed}^\Box$ and $\texttt{delayed}^\lozenge$ are invariant to $[\cdot]^{-1}_L$ and $[\cdot]^{-1}_R$, respectively. By induction, $\Delta''\vdash Q :: a\!:\!A''$ such that $\text{time}([A]^{-1}_R)-1\geq\text{time}(A'')$. 

% % %

% % \item[$\runa{TS-$\lozenge$L}$]

% % %

% % \item[$\runa{TS-$\lozenge$R}$]

% % %

% % \item[$\runa{TS-$\Box$L}$]

% % %

% % \item[$\runa{TS-$\Box$R}$]


% % \end{description}
% % \begin{enumerate}
% %     \item $P = \tick P'$ and $\Delta'\vdash P :: a\!:\!A'$ by $\runa{TS-$\ocircle$LR'}$. Then $P$ is in canonical form and $Q = P'$. By $\runa{TS-$\ocircle$LR'}$ we obtain that $[\Delta']^{-1}_L\vdash P' :: a\!:\![A']^{-1}_R$. We show that this implies $[\Delta]^{-1}_L\vdash P' :: a\!:\![A]^{-1}_R$ by induction on the temporal type rules
% %     \begin{description}
% %     \item[hmm]
    
    
% %     \end{description}
    
% %     \item $P \equiv \newvar{a}{(Q \mid R)}$ and $\Delta P\vdash P :: a\!:\!A$ by $\runa{TS-cut}$ or $\runa{TS-def}$.
% % \end{enumerate}
% \end{proof}
% \end{lemma}


% %
% % \begin{theorem}[Subject Reduction]
% % If $\Delta \vdash P :: a\!:\!A$ and $P \longrightarrow Q$ then $\Delta\vdash Q :: a\!:\!A$.
% % \begin{proof}
% % Proof by induction on the extended reduction rules. The proof uses the fact that a well-typed process cannot \textit{consume} the session it provides on reduction, by type rules $\runa{TS-cut}$ and $\runa{TS-def}$. The proof is slightly tedious, as the type rules are not syntax directed.
% % \begin{description}
% % \item[$\runa{R-tick}$] Assume that $P$ reduces by $\runa{R-tick}$, such that $P$ is of the form $\texttt{tick}.P'$ and $Q = P'$. Then by $\runa{TS-$\ocircle$LR'}$, we have that $[\Delta]^{-1}_L \vdash P' :: [a:A]^{-1}_R$ such that $\Delta \vdash \texttt{tick}.P' :: a\!:\!A$. It follows from type rule $\runa{TS-$\ocircle$LR}$ that also $\Delta \vdash P' :: a\!:\!A$.

% % %

% % \item[$\runa{R-id}$] Assume that $P$ reduces by $\runa{R-id}$ then we have that $P \equiv \newvar{a}{\newvar{b}{(P' \mid a \leftarrow b)}}$ such that $Q \equiv \newvar{h}{(P'[a\mapsto h,b\mapsto h])}$ for some name $h \notin fv(P')$. Then, as restrictions are only typable by $\runa{TS-cut}$ and $\runa{TS-def}$, $P'$ must be of the form $R' \mid R''$ such that $P \equiv \newvar{a}{(R' \mid \newvar{b}{(R'' \mid a \leftarrow b)})}$ or $P \equiv \newvar{b}{(R' \mid \newvar{a}{(R'' \mid a \leftarrow b)})}$. We consider the cases separately
% % \begin{enumerate}
% %     \item $\Delta'',a:A \vdash R' :: c\!:\!C$ such that $\Delta'\vdash \newvar{b}{(R'' \mid a \leftarrow b)} :: a\!:\!A$ and $\Delta',\Delta''\vdash P :: c\!:\!C$ using $\runa{TS-cut}$. Then we can type $\newvar{b}{(R'' \mid a \leftarrow b)}$ with either $\runa{TS-cut}$ or $\runa{TS-def}$
% %     \begin{enumerate}
% %         \item $\Delta' \vdash R'' :: b\!:\!A$ such that $b:A\vdash a \leftarrow b :: a\!:\!A$ and $\Delta' \vdash \newvar{b}{(R'' \mid a \leftarrow b)} :: a\!:\!A$. Then it follows by renaming that $\Delta''\vdash R''[a\mapsto h,b\mapsto h] :: h\!:\!A$ and $\Delta'',h:A \vdash R' :: c\!:\!C$ such that $\Delta',\Delta''\vdash\newvar{h}{(R'[a\mapsto h,b\mapsto h] \mid R''[a\mapsto h,b\mapsto h]) :: c\!:\!C}$.
        
% %         \item $R'' = b \leftarrow f \leftarrow \widetilde{d}$ and $(\widetilde{e} : \widetilde{B}\vdash f = R :: g\!:\!A) \in \Sigma$ such that $\Delta' = \widetilde{d}:\widetilde{B}$, $b:A\vdash a \leftarrow b :: a\!:\!A$ and $\Delta' \vdash \newvar{b}{(R'' \mid a \leftarrow b)} :: a\!:\!A$. Then it follows by renaming that $\Delta'',h:A \vdash R' :: c\!:\!C$ such that $\Delta',\Delta''\vdash\newvar{h}{(R'[a\mapsto h,b\mapsto h] \mid h \leftarrow f \leftarrow \widetilde{d}) :: c\!:\!C}$.
% %     \end{enumerate}
    
% %     %
    
% %     \item Either $\Delta' \vdash R' :: b\!:\!A$ or $b \leftarrow f \leftarrow \widetilde{d}$, $\Delta' = \widetilde{d}:\widetilde{B}$ and $(\widetilde{e} : \widetilde{B}\vdash f = R :: g\!:\!A) \in \Sigma$ such that $\Delta'',b:A\vdash \newvar{a}{(R'' \mid a \leftarrow b)} :: c\!:\!C$ and $\Delta',\Delta''\vdash P :: c\!:\!C$ using $\runa{TS-cut}$ or $\runa{TS-def}$, respectively. In either case we must use $\runa{TS-cut}$ to get $\Delta'',b:A\vdash \newvar{a}{(R'' \mid a \leftarrow b)} :: c\!:\!C$, as we have that $b:A\vdash a\leftarrow b :: a\!:\!A$ and $\Delta'',a:A\vdash R'' :: c\!:\!C$. Then we reach $\Delta',\Delta''\vdash\newvar{h}{(R'[a\mapsto h,b\mapsto h] \mid R''[a\mapsto h,b\mapsto h])} :: c\!:\!C$ by either $\runa{TS-cut}$ or $\runa{TS-def}$. In either case we have that $\Delta'',h:A\vdash R''[a\mapsto h,b\mapsto h] :: c\!:\!C$. In the first case we have that $\Delta' \vdash R'[a\mapsto h,b\mapsto h] :: h\!:\!A$ and the latter case trivially follows by $R'[a\mapsto h,b\mapsto h] = h \leftarrow f \leftarrow \widetilde{d}$.
% % \end{enumerate}

% % %

% % \item[$\runa{R-comm}$] Assume we reduce $P$ by $\runa{R-comm}$ then $P \equiv \inputch{a}{v}{}{R'} \mid \outputch{a}{b}{}{R''}$ for some name $b$ and processes $R'$ and $R''$, such that $\inputch{a}{v}{}{R'} \mid \outputch{a}{b}{}{R''} \longrightarrow R'[v\mapsto b] \mid R''$. For $P$ to be well-typed, it must be part of a larger process $\Delta',\Delta''\vdash\newvar{a}{P} :: c\!:\!C$ typed with $\runa{TS-cut}$ for which we have two cases
% % \begin{enumerate}
% %     \item $\Delta' \vdash \inputch{a}{v}{}{R'} :: a\!:\!A' \multimap A''$ and $\Delta_3,a : A'\multimap A'', b : A' \vdash \outputch{a}{b}{}{R''} :: c\!:\!C$ by $\runa{TS-$\multimap$R}$ and $\runa{TS-$\multimap$L}$ such that $\Delta'' = \Delta_3,b:A'$. By the premises to these rules we have that $\Delta',v : A' \vdash R' :: a\!:\!A''$ and $\Delta_3,a:A''\vdash R'' :: c\!:\!C$. This implies $\Delta',b : A'\vdash R'[v\mapsto b] :: a\!:\!A''$, and so by $\runa{TS-cut}$ it follows that $(\Delta',b : A'),\Delta_3\vdash \newvar{a}{(R'[v\mapsto b] \mid R'') :: c\!:\!C}$ and $\Delta = (\Delta',b : A'),\Delta_3$.
    
% %     %
    
% %     \item $\Delta_3,b:A' \vdash \outputch{a}{b}{}{R''} :: a\!:\!A'\otimes A''$ and $\Delta'',a : A'\otimes A''\vdash \inputch{a}{v}{}{R'} :: c\!:\!C$ by $\runa{TS-$\otimes$R}$ and $\runa{TS-$\otimes$L}$ such that $\Delta' = \Delta_3,b:A'$. By the premises to these rules we have that $\Delta_3\vdash R'' :: a\!:\!A''$ and $\Delta'',a:A'',v:A'\vdash R' :: c\!:\!C$. This implies $\Delta'',a:A'',b:A'\vdash R'[v\mapsto b] :: c\!:\!C$, and so by $\runa{TS-cut}$ it follows that $\Delta_3,(\Delta'',b : A')\vdash \newvar{a}{(R'' \mid R'[v\mapsto b])} :: c\!:\!C$ and $\Delta = \Delta_3,(\Delta'',b : A')$.
% % \end{enumerate}

% % %

% % \item[$\runa{R-choice}$] Assume we reduce $P$ by $\runa{R-choice}$ then $P \equiv a.\texttt{case}\{ l \Rightarrow P_l \}_{l\in L} \mid a.k; R$ for some label $k$ and set of labels $L$, such that $k\in L$ and $a.\texttt{case}\{ l \Rightarrow P_l \}_{l\in L} \mid a.k; R \longrightarrow P_k \mid R$. For $P$ to be well-typed, it must be part of a larger process $\Delta',\Delta''\vdash \newvar{a}{P} :: c\!:\!C$ typed with $\runa{TS-cut}$ for which we have two cases
% % \begin{enumerate}
% %     \item $\Delta'\vdash a.\texttt{case}\{l \Rightarrow P_l\}_{l\in L} :: a\!:\!\&\{l : A_l\}_{l\in L}$ and $\Delta'', a : \&\{l : A_l\}_{l\in L}\vdash a.k; R :: c\!:\!C$ by $\runa{TS-$\&$R}$ and $\runa{TS-$\&$L}$. By the premises of these rules we have that $\Delta' \vdash P_k :: a\!:\!A_k$ and $\Delta'',a : A_k\vdash R :: c\!:\!C$, and so it follows by $\runa{TS-cut}$ that $\Delta',\Delta''\vdash \newvar{a}{(P_k \mid R) :: c\!:\!C}$.
        
% %     %
    
% %     \item $\Delta'\vdash a.k; R :: a\!:\!\oplus\{l : A_l\}_{l\in L}$ and $\Delta'',a : \oplus\{l : A_l\}_{l\in L}\vdash a.\texttt{case}\{l\Rightarrow P_l\}_{l\in L} :: c\!:\!C$ by $\runa{TS-$\oplus$R}$ and $\runa{TS-$\oplus$L}$. By the premises of these rules we have that $\Delta'\vdash R :: a\!:\!A_k$ and $\Delta'',a : A_k\vdash P_k :: c\!:\!C$, and so it follows by $\runa{TS-cut}$ that $\Delta',\Delta''\vdash \newvar{a}{(R \mid P_k)} :: c\!:\!C$.
    
% % \end{enumerate}

% % %

% % \item[$\runa{R-def}$] Assume $P$ reduces by $\runa{R-def}$ then $P = b \leftarrow f \leftarrow \widetilde{d}$ and $(\widetilde{c}:\widetilde{B}\vdash f = P' :: a\!:\!A) \in \Sigma$, such that $Q = P'[a\mapsto b,\widetilde{c}\mapsto\widetilde{d}]$. For $P$ to be well-typed it must be part of a larger process $\widetilde{d}:\widetilde{B},\Delta'\vdash \newvar{b}{(P \mid R)} :: c\!:\!C$ typed with $\runa{TS-def}$ such that $\Delta',b:A\vdash R :: c\!:\!C$. By renaming we have that $\widetilde{d}:\widetilde{B}\vdash P'[a\mapsto b,\widetilde{c}\mapsto\widetilde{d}] :: b\!:\!B$ and so by $\runa{TS-cut}$ we have that $\widetilde{d}:\widetilde{B},\Delta'\vdash \newvar{b}{(Q \mid R)} :: c\!:\!C$.

% % %

% % \item[$\runa{R-res}$] Assume that $P$ reduces by $\runa{R-res}$ then we have that $P \equiv \newvar{a}{P'}$ for some name $a$ such that $P' \longrightarrow Q'$. Then $P$ must be typed either with $\runa{TS-cut}$ or $\runa{TS-def}$ and so $P' \equiv R' \mid R''$ yielding two cases
% % \begin{enumerate}
% %     \item $\Delta'\vdash R' :: a\!:\!A$ such that $\Delta'',a:A\vdash R'' :: c\!:\!C$ and $\Delta',\Delta''\vdash \newvar{a}{P'}::c\!:\!C$. Either $R' \mid R''$ reduces by $\runa{R-par}$, $\runa{R-comm}$, $\runa{R-choice}$ or $\runa{R-struct}$. The first three cases are covered by the clauses for the corresponding rules, and the last case holds by induction as typability is closed under structural congruence.
    
% %     \item $R' = a \leftarrow f \leftarrow \widetilde{b}$ and $(\widetilde{e} : \widetilde{B}\vdash f = R :: g\!:\!A) \in \Sigma$ such that $\Delta' = \widetilde{b}:\widetilde{B}$, $\Delta'',a:A\vdash R'' :: c\!:\!C$ and $\Delta',\Delta''\vdash \newvar{a}{P'}::c\!:\!C$. Then either $R' \mid R''$ reduces by $\runa{R-par}$ or $\runa{R-struct}$. The first case is covered by the clause for $\runa{R-par}$, and the last case holds by induction as typability is closed under structural congruence.
% % \end{enumerate}

% % %

% % \item[$\runa{R-par}$] Assume that $P$ reduces by $\runa{R-par}$ then we have that $P \equiv P' \mid P''$ such that $P' \longrightarrow Q'$. For $P$ to be well-typed, it must be part of a larger well-typed process $\newvar{a}{(P'\mid P'')}$ typed with either $\runa{TS-cut}$ or $\runa{TS-def}$ such that either
% % \begin{enumerate}
% %     \item $\Delta'\vdash P' :: a\!:\!A$ such that $\Delta'',a:A\vdash P'' :: c\!:\!C$ and $\Delta',\Delta''\vdash \newvar{a}{(P'\mid P'')}::c\!:\!C$. Then by induction we have that $\Delta'\vdash Q' :: a\!:\!A$ and so it follows that $\Delta',\Delta''\vdash \newvar{a}{(Q' \mid P'')}::c\!:\!C$
    
% %     \item $P' = a \leftarrow f \leftarrow \widetilde{b}$ and $(\widetilde{e} : \widetilde{B}\vdash f = R :: g\!:\!A) \in \Sigma$ such that $\Delta' = \widetilde{b}:\widetilde{B}$, $\Delta'',a:A\vdash P'' :: c\!:\!C$ and $\widetilde{b}:\widetilde{B},\Delta''\vdash \newvar{a}{P' \mid P''}::c\!:\!C$. Then it must be that $P'$ reduces to $Q'$ by $\runa{TS-def}$ such that $Q' = R[g\mapsto a,\widetilde{e}\mapsto\widetilde{b}]$. By renaming $\widetilde{e} : \widetilde{B}\vdash R :: g\!:\!A$ implies $\widetilde{b} : \widetilde{B}\vdash Q' :: a\!:\!A$ such that $\widetilde{b}:\widetilde{B},\Delta''\vdash \newvar{a}{(Q' \mid P''):: c\!:\!C}$ by $\runa{T-cut}$.
% % \end{enumerate}

% % %%%
% % %%
% % %%
% % %%%

% % %when they contain no named processes, for $P$ to be well-typed, $P$ must be a subprocess of a larger well-typed process $R \equiv \newvar{a}{\newvar{b}{P}} \equiv \newvar{a}{(\outputch{a}{d}{}{P'} \mid \newvar{b}{(\inputch{b}{v}{}{P''} \mid b \leftarrow a}))}$ such that $\Delta',\Delta''\vdash R :: c\!:\!C$. Then from the premises of $\runa{TS-cut}$, we have that $\Delta'',a:A\vdash \outputch{a}{d}{}{P'} ::c\!:\!C$ and (by $\runa{TS-cut}$ again) $\Delta'\vdash \newvar{b}{(\inputch{b}{v}{}{P''} \mid b \leftarrow a}) :: a\!:\!A$ such that $\Delta' \vdash \inputch{b}{v}{}{P''} :: b\!:\!A$ by $\runa{TS-$\multimap$R}$ and $b : A\vdash b \leftarrow a :: a\!:\!A$ by $\runa{TS-id}$. The full reduced process is then $\newvar{a}{\newvar{b}{(P' \mid P''[v\mapsto d])}}$

% % %
% % %%%%%%%%%%
% % %

% % % \item[$\runa{R-res}$] Assume that $P$ reduces by $\runa{R-res}$. Then for $P$ to be well-typed, $P$ must be typed by either $\runa{TS-cut}$ or $\runa{TS-def}$. We consider the cases separately
% % % \begin{description}
% % % \item[$\runa{TS-cut}$] We have that $P$ is of the form $\newvar{a}{(P'\mid P'')}$ such that $\Delta' \vdash P' :: a\!:\!A$, $\Delta'', a : A\vdash P'' :: c\!:\!C$ and $\Delta',\Delta'' \vdash \newvar{a}{(P'\mid P'')} :: c\!:\!C$. By $\runa{R-res}$ we have that $P' \mid P''$ must reduce, for which several rules apply
% % % \begin{description}
% % % \item[$\runa{R-comm}$] If we reduce the parallel composition by $\runa{R-comm}$ then $P' \mid P'' \equiv \inputch{a}{v}{}{R'} \mid \outputch{a}{b}{}{R''}$ for some name $b$ and processes $R'$ and $R''$, such that $\inputch{a}{v}{}{R'} \mid \outputch{a}{b}{}{R''} \longrightarrow R'[v\mapsto b] \mid R''$. We have two cases
% % % \begin{enumerate}
% % %     \item $\Delta' \vdash \inputch{a}{v}{}{R'} :: a\!:\!A' \multimap A''$ and $\Delta_3,a : A'\multimap A'', b : A' \vdash \outputch{a}{b}{}{R''} :: c\!:\!C$ by $\runa{TS-$\multimap$R}$ and $\runa{TS-$\multimap$L}$ such that $\Delta'' = \Delta_3,b:A'$. By the premises to these rules we have that $\Delta',v : A' \vdash R' :: a\!:\!A''$ and $\Delta_3,a:A''\vdash R'' :: c\!:\!C$. This implies $\Delta',b : A'\vdash R'[v\mapsto b] :: a\!:\!A''$, and so by $\runa{TS-cut}$ it follows that $(\Delta',b : A'),\Delta_3\vdash \newvar{a}{(R'[v\mapsto b] \mid R'') :: c\!:\!C}$ and $\Delta = (\Delta',b : A'),\Delta_3$.
    
% % %     %
    
% % %     \item $\Delta_3,b:A' \vdash \outputch{a}{b}{}{R''} :: a\!:\!A'\otimes A''$ and $\Delta'',a : A'\otimes A''\vdash \inputch{a}{v}{}{R'} :: c\!:\!C$ by $\runa{TS-$\otimes$R}$ and $\runa{TS-$\otimes$L}$ such that $\Delta' = \Delta_3,b:A'$. By the premises to these rules we have that $\Delta_3\vdash R'' :: a\!:\!A''$ and $\Delta'',a:A'',v:A'\vdash R' :: c\!:\!C$. This implies $\Delta'',a:A'',b:A'\vdash R'[v\mapsto b] :: c\!:\!C$, and so by $\runa{TS-cut}$ it follows that $\Delta_3,(\Delta'',b : A')\vdash \newvar{a}{(R'' \mid R'[v\mapsto b])} :: c\!:\!C$ and $\Delta = \Delta_3,(\Delta'',b : A')$.
% % % \end{enumerate}

% % % \item[$\runa{R-choice}$] If we reduce the parallel composition by $\runa{R-choice}$ then $P' \mid P'' \equiv a.\texttt{case}\{ l \Rightarrow P_l \}_{l\in L} \mid a.k; R$ for some label and set of labels $k$ and $L$, such that $k\in L$ and $a.\texttt{case}\{ l \Rightarrow P_l \}_{l\in L} \mid a.k; R \longrightarrow P_k \mid R$. We have two cases
% % % \begin{enumerate}
% % %     \item $\Delta'\vdash a.\texttt{case}\{l \Rightarrow P_l\}_{l\in L} :: a\!:\!\&\{l : A_l\}_{l\in L}$ and $\Delta'', a : \&\{l : A_l\}_{l\in L}\vdash a.k; R :: c\!:\!C$ by $\runa{TS-$\&$R}$ and $\runa{TS-$\&$L}$. By the premises of these rules we have that $\Delta' \vdash P_k :: a\!:\!A_k$ and $\Delta'',a : A_k\vdash R :: c\!:\!C$, and so it follows by $\runa{TS-cut}$ that $\Delta',\Delta''\vdash \newvar{a}{(P_k \mid R) :: c\!:\!C}$.
        
% % %     %
    
% % %     \item $\Delta'\vdash a.k; R :: a\!:\!\oplus\{l : A_l\}_{l\in L}$ and $\Delta'',a : \oplus\{l : A_l\}_{l\in L}\vdash a.\texttt{case}\{l\Rightarrow P_l\}_{l\in L} :: c\!:\!C$ by $\runa{TS-$\oplus$R}$ and $\runa{TS-$\oplus$L}$. By the premises of these rules we have that $\Delta'\vdash R :: a\!:\!A_k$ and $\Delta'',a : A_k\vdash P_k :: c\!:\!C$, and so it follows by $\runa{TS-cut}$ that $\Delta',\Delta''\vdash \newvar{a}{(R \mid P_k)} :: c\!:\!C$.
    
% % % \end{enumerate}

% % % \item[$\runa{R-id-1}$]
% % % \item[$\runa{R-id-2}$]
% % % \item[$\runa{R-par}$] If we reduce the parallel composition by $\runa{R-par}$ then $P' \longrightarrow Q'$. Here we can apply induction, as $P'$ cannot be typed as $\Delta' \vdash P' :: a\!:\!A$ and reduce unless it is prefixed by a tick or is wrapped with a restriction (or is structurally congruent to such a process by $\runa{R-struct}$). And so, it follows that $\Delta' \vdash Q' :: a\!:\!A$, such that $\Delta',\Delta'' \vdash \newvar{a}{(Q'\mid P'')} :: c\!:\!C$.
% % % \item[$\runa{R-struct}$] todo: induction (with R-par after).
% % % \end{description}
% % % \item[$\runa{TS-def}$] We have that $P$ is of the form $\newvar{a}{(a\leftarrow f \leftarrow \widetilde{b} \mid P')}$ such that $(\widetilde{d} : \widetilde{B}\vdash f = P :: g\!:\!A) \in \Sigma$, $\Delta',a : A \vdash P' :: c\!:\!C$ and $\Delta',\widetilde{b} : \widetilde{B}\vdash \newvar{a}{(a\leftarrow f \leftarrow \widetilde{b} \mid P') :: c\!:\!C}$. By $\runa{R-res}$ we have that $a\leftarrow f \leftarrow \widetilde{b} \mid P'$ must reduce, for which $\runa{R-par}$ and $\runa{R-struct}$ apply. Note that the parallel composition cannot reduce by $\runa{R-par}$, as  does not  several rules apply.
% % % \begin{description}
% % % \item[$\runa{R-par}$] todo: R-def --> can type with R-cut after.
% % % \item[$\runa{R-struct}$] todo: induction (with R-par after). 
% % % \end{description}
% % % \end{description}


% % \item[$\runa{R-struct}$] Assume that $P$ reduces by $\runa{R-struct}$. Then $P \equiv P'$, $P' \longrightarrow Q'$ and $Q' \equiv Q$. As typability is closed under structural congruence and $\Delta \vdash P :: c\!:\!C$ it follows that $\Delta \vdash P' :: c\!:\!C$. By induction this implies $\Delta \vdash Q' :: c\!:\!C$, and as $Q' \equiv Q$ we have that $\Delta\vdash Q :: c\!:\!C$.
% % \end{description}
% % \end{proof}
% % \end{theorem}

% %

% \begin{theorem}
% If $\Delta\vdash P :: a\!:\!A$ and $P$ reduces to $Q$ by the tick-last strategy using $n$ productive time reductions then $\text{time}(A) \geq n$.
% \begin{proof} By induction on the size of n.
% \begin{description}
% \item[$n = 0$] For any process $P$, context $\Delta$ and session $a:A$ such that $\Delta\vdash P :: a\!:\!A$, we have that $\text{time}(A)\geq 0$, and so this is obtained trivially.

% \item[$n+1$] Assume that $\Delta\vdash P :: a\!:\!A$ and $P$ reduces to $Q$ by the tick-last strategy with $n+1$ productive time reductions. Then we have that $P \leadsto^* P'$ and $P' \Longrightarrow^{-1} Q'$ with $P' \neq Q'$ and $P'\!\not\!\leadsto$ such that $Q'$ reduces to $Q$ by the tick-last strategy with $n$ productive time reductions. By Theorem \ref{theorem:sr} $\Delta\vdash P' :: a\!:\!A$ follows from $\Delta\vdash P :: a\!:\!A$. By Lemma \ref{lemma:timered} $\Delta\vdash P' :: a\!:\!A$ and $P' \Longrightarrow^{-1} Q'$ with $P' \neq Q'$ implies there exists $\hat{\Delta'}[\Delta']=\Delta$ and $\hat{A'}[A']=A$ such that $\hat{\Delta'}[[\Delta']^{-1}_L]\vdash Q' :: a\!:\!\hat{A'}[[A']^{-1}_R]$ and by Lemma \ref{lemma:timegeq} $\text{time}(\hat{A'}[A']) - 1 \geq \text{time}(\hat{A'}[[A']^{-1}_R])$. Then by the induction hypothesis, $\text{time}(\hat{A'}[[A']^{-1}_R]) \geq n$. It follows from $\hat{A'}[A']=A$ that $\text{time}(A) \geq n + 1$.

% \end{description}

% %We do not consider left rules, i.e. those that consume rather than provide sessions, as they hold trivially by induction and the fact that the span of inputs, outputs, external choices and internal choices is either $0$ or the span of their continuations, depending on the environment.
% % \begin{description}
% % %\item[$\runa{TS-$\mathbf{1}$L}$] 

% % \item[$\runa{TS-$\mathbf{1}$R}$] The span of inaction $\mathbf{0}$ is $0$, and by $\runa{TS-$\mathbf{1}R$}$, $\mathbf{0}$ provides a session of type $\mathbf{1}$. We have that $\text{time}(\mathbf{1}) = 0$.

% % %\item[$\runa{TS-$\otimes$L}$]

% % \item[$\runa{TS-$\otimes$R}$] The span of a synchronous output $\outputch{a}{v}{}{P}$ is 0 when it is not in parallel with a corresponding input, and equal to the span of its continuation $P$ when it is. By $\runa{TS-$\otimes$R}$, $\Delta,v : A\vdash\outputch{a}{v}{}{P}:: a\!:\! A\otimes B$ such that $\Delta\vdash P :: a\!:\!B$, and so by induction, $\text{time}(B)$ is an upper bound on the span of $P$. We have that $\text{time}(A \otimes B) = \text{time}(B)$, which is an upper bound in either case.

% % %\item[$\runa{TS-$\multimap$L}$]

% % \item[$\runa{TS-$\multimap$R}$] The span of a synchronous input $\inputch{a}{v}{}{P}$ is 0 when it is not in parallel with a corresponding output, and equal to the span of its continuation $P$ when it is. By $\runa{TS-$\multimap$R}$, $\Delta\vdash\inputch{a}{v}{}{P}:: a\!:\! A\multimap B$ such that $\Delta,v:A\vdash P :: a\!:\!B$, and so by induction, $\text{time}(B)$ is an upper bound on the span of $P$. We have that $\text{time}(A \multimap B) = \text{time}(B)$, which is an upper bound in either case.

% % \item[$\runa{TS-cut}$] The span of a parallel composition depends on the parallelism of reductions of ticks with respect to synchronizations in its two parallel subprocesses, i.e. it depends on the protocols of channels used in the parallel composition. By $\runa{TS-cut}$ we have that $\Delta',\Delta''\vdash \newvar{a}{(P'\mid P'')::c\!:\!C}$ such that $\Delta'\vdash P' :: a\!:\!A$ and $\Delta'',a:A\vdash P'' :: c\!:\!C$. As $P'$ and $P''$ only communicate using one channel (i.e. $a$ when no identity constructs are used, or some fresh name otherwise), the parallelism entirely depends on the session type $A$. By induction $\text{time}(A)$ is an upper bound on the span of $P'$ when session $a:A$ is used. Similarly, $\text{time}(C)$ is a bound on the span of $P''$ including when session $c:C$ is consumed. By Lemma \ref{}, $a:A\vdash P'' :: c\!:\!C$ implies $\text{time}(C) \geq \text{time}(A)$, and so $\text{time}(C)$ is an upper bound on the span of $\newvar{a}{(P'\mid P'')}$.

% % \item[$\runa{TS-id}$] The span of an identity construct $a \leftarrow b$ is 0, as it has no continuation, and so for $b:A\vdash a\leftarrow b :: a\!:\!A$ it trivially holds that $\text{time}(A)$ is an upper bound on the span, as $\text{time}(A) \geq 0$.

% % %\item[$\runa{TS-$\oplus$L}$]

% % \item[$\runa{TS-$\oplus$R}$] The span of an internal choice $a.k; P$ is 0 when it is not in parallel with a corresponding external choice, and equal to the span of its continuation $P$ when it is. By $\runa{TS-$\oplus$R}$, $\Delta\vdash a.k; P ::a\!:\!\oplus\{l:A_l\}_{l\in L}$ such that $k \in L$ and $\Delta\vdash P ::a\!:\!A_k$, and so by induction, $\text{time}(A_k)$ is an upper bound on the span of $P$. We have that $\text{time}(\oplus\{l:A_l\}_{l\in L}) = \text{max}(\text{time}(A_l) \mid l \in L)$. As $k \in L$, $\text{max}(\text{time}(A_l) \mid l \in L)\geq \text{time}(A_k)$, which is an upper bound in either case.

% % %\item[$\runa{TS-$\&$L}$]

% % \item[$\runa{TS-$\&$R}$] The span of an external choice $a.\texttt{case}\{l \Rightarrow P_l\}_{l\in L}$ is 0 when it is not in parallel with a corresponding internal choice, and equal to the maximum span amongst its possible continuations $P_l$ for $l\in L$ when it is. By $\runa{TS-$\&$R}$, $\Delta\vdash a.\texttt{case}\{l \Rightarrow P_l\}_{l\in L} ::a\!:\!\&\{l:A_l\}_{l\in L}$ such that for all $l \in L$ $\Delta\vdash P_l ::a\!:\!A_l$, and so by induction, $\text{time}(A_l)$ is an upper bound on the span of $P_l$. We have that $\text{time}(\&\{l:A_l\}_{l\in L}) = \text{max}(\text{time}(A_l) \mid l \in L)$, which is an upper bound in either case.

% % \item[$\runa{TS-def}$]

% % \item[$\runa{TS-$\ocircle$LR'}$]

% % \item[$\runa{TS-$\ocircle$LR}$]

% % \item[$\runa{TS-$\lozenge$L}$]

% % \item[$\runa{TS-$\lozenge$R}$]

% % \item[$\runa{TS-$\Box$L}$]

% % \item[$\runa{TS-$\Box$R}$]

% % \end{description}
% \end{proof}
% \end{theorem}
\end{document}
%%% Local Variables:
%%% mode: latex
%%% TeX-master: t
%%% End:
