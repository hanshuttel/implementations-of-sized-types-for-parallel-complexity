\subsection{Reducing polynomial constraints to linear constraints}\label{sec:verifyingpolynomial}

Many programs do not run in linear time, and so we cannot type them if we are constrained to just verifying linear constraint judgements. In this section we show how we can reduce certain polynomial constraints to linear constraints, enabling us to use the techniques described above. We first extend our definition of indices such that they can be used to express multivariate polynomials. We assume a normal form for polynomial indices akin to that of linear indices. Terms are now monomials with integer coefficients.
%
\begin{align*}
        I,J ::= n \mid i \mid I + J \mid I - J \mid I J
\end{align*}
%
When reducing normalized constraints with polynomial indices to normalized constraints with linear indices, we wish to construct new linear constraints that are only satisfied if the original polynomial constraint is satisfied. For example, given the constraint ${-i^2 + 10 \leq 0}$, one can see that this polynomial constraint can be simulated using the constraint $-i + \sqrt{10} \leq 0$. We notice that the reason this is possible is that when ${-i^2 + 10 \leq 0}$ holds, i.e. when $i \geq \sqrt{10}$, the value of $i$ can always be increased without violating the constraint. Similarly, when ${-i^2 + 10 \geq 0}$ holds, the value of $i$ can always be decreased until reaching its minimum value of 0 without violating the constraint. We can thus introduce a new simpler constraint with the same properties, i.e. $-i + \sqrt{10} \leq 0$. More specifically, the polynomials of the left-hand side of the two constraints share the same positive real-valued roots as well as the same sign for any value of $i$. In general, limiting ourselves to univariate polynomials, for any constraint whose left-hand side polynomial only has a single positive root, we can simulate such a constraint using a constraint of the form $a \cdot i + c \leq 0$. For describing complexities of programs, we expect to mostly encounter monotonic polynomials with at most a single positive real-valued root.\\

Note that the above has the consequence that we may end up with irrational coefficients for indices of constraints. While such constraints are not usually allowed, they can still represent valid bounds for index variables. As such, we can allow them as constraints in this context. Furthermore, any irrational coefficient can be approximated to an arbitrary precision using a rational coefficient.\\

For finding the roots of a specific polynomial we can use either analytical or numerical methods. Using analytical methods has the advantage of being able to determine all roots with exact values, however, we are limited to polynomials of degree at most four as stated by the Abel-Ruffini theorem \cite{abelruffinitheorem}. With numerical methods, we are not limited to polynomials of a specific degree, however, numerical methods often require a given interval to search for a root and do not guarantee to find all roots. Introducing constraints with false restrictions may lead to an under-approximating type system, and so we must be careful not to introduce such. We must therefore ensure we find all roots to avoid constraints with false restrictions. We can use Descartes rule of signs to get an upper bound on the number of positive real roots of a polynomial. Descarte's rule of signs states that the number of roots in a polynomial is at most the number of sign-changes in its sequence of coefficients.\\

For our application, we decide to only consider constraints whose left-hand side polynomial is univariate and monotonic with a single root. In some cases we may remove safely positive monomials in a normalized constraint to obtain such constraints. We limit ourselves to these constraints both to keep complexity down, as well as because we expect to mainly encounter such polynomials when considering complexity analysis of programs. We use Laguerre's method as a numerical method to find the root of the polynomial, which has the advantage that it does not require any specified interval when performing root-finding. Assuming we can find a root $r$, we add an additional constraint $\pm (i - r) \leq 0$ where the sign depends on whether the original polynomial is increasing or decreasing.\\

Additionally, given a non-linear monomial, we may also treat this as a single unit and construct linear combinations of this by treating the monomial as a single fresh variable. For example, given normalized constraints $C_1 = i^2 + 4 \leq 0$, $C_2 = i - 2 \leq 0$, and $C_3 = 2ij \leq 0$, we may view these as the linear constraints $C_1' = k + 4 \leq 0$, $C_2' = i - 2 \leq 0$, and $C_3' = 2l \leq 0$ where $k = i^2$ and $l = ij$. This may make the feasible region larger than it actually is, meaning we over-approximate when verifying judgements on polynomial constraints.

\begin{examp}
    We want to check if the following judgement holds
    $$\{i, j\}; \{-2i \leq 0, -1i^2 + 1j + 1 \leq 0\} \vDash -2i + 2 \leq 0$$
    
    %We notice that $-2i + 2$ cannot be written as a conical combination of the polynomials $-2i$ and $-1i^2 + 1j + 1$.\\
    
    We first try to generate new constraints of the form $-a \cdot i + r \leq 0$ for some index variable $i$ and some constants $a$ and $r$ based on our two existing constraints using the root-finding method. The first constraint is already of such form, so we can only consider the second. For the second, we first use the subconstraint relation to remove the term $1j$ obtaining $-1i^2 + 1 \leq 0$. Next, we note that $-1i^2 + 1$ is a monotonically decreasing polynomial as every coefficient excluding the constant term is negative. We then find the root $r = 1$ of the polynomial and add a new constraint $-1i + 1 \leq 0$ to our set of constraints. Finally, we can invert the constraint $-2i + 2 \leq 0$ obtaining the constraint $-2i + 1 \geq 0$. We now need to check if the feasible region $\mathcal{M}_{\{i, j, i^2\}}(\{-2i \leq 0, -1i^2 + 1j + 1 \leq 0, -1i + 1 \leq 0, -2i + 1 \geq 0\})$ is empty. Treating the monomial $i^2$ as its own separate variable and solving this as a linear program using an algorithm such as the simplex algorithm, we see that there is indeed no solution, and so the constraint is satisfied. 
\end{examp}

\subsection{Trivial judgements}
We now show how some judgements may be verified without neither transforming constraints into linear constraints nor solving any integer programs. To do so, we consider an example provided by Baillot and Ghyselen \cite{BaillotGhyselen2021}, where we exploit the fact that all coefficients in the normalized constraints are non-positive. Judgements with such constraints can be answered in linear time with respect to the number of monomials in the normalized equivalent of constraint $C$. That is if all coefficients in the normalized constraint are non-positive, we can guarantee that the constraint is always satisfied, recalling that only naturals substitute for index variables. Similarly, if there are no negative coefficients and at least one positive coefficient, we can guarantee that the constraint is never satisfied.\\

In practice, it turns out that we can type check many processes by simply over-approximating constraint judgements using pair-wise coefficient inequality constraints. In Example \ref{example:baillotghyssimple}, we show how all constraint judgements in the typings of both a linear and a polynomial time replicated input can be verified using this approach. 
%
\begin{examp}\label{example:baillotghyssimple}
Baillot and Ghyselen \cite{BaillotGhyselen2021} provide an example of how their type system for parallel complexity of message-passing processes can be used to bound the time complexity of a linear, a polynomial and an exponential time replicated input process. We show that we can verify all judgements on constraints in the typings of the first two processes using normalized constraints. We first define the processes $P_1$ and $P_2$
\begin{align*}
    P_i \defeq\; !\inputch{a}{n,r}{}{\tick\match{n}{\asyncoutputch{r}{}{}}{m}{\newvar{r'}{\newvar{r''}{Q_i}}}}
\end{align*}
for the corresponding definitions of $Q_1$ and $Q_2$
\begin{align*}
    Q_1 \defeq&\; \asyncoutputch{a}{m,r'}{} \mid \asyncoutputch{a}{m,r''}{} \mid \inputch{r'}{}{}{\inputch{r''}{}{}{\asyncoutputch{r}{}{}}}\\
    Q_2 \defeq&\; \asyncoutputch{a}{m,r'}{} \mid \inputch{r'}{}{}{(\asyncoutputch{a}{m,r''}{} \mid \asyncoutputch{r}{}{})} \mid \asyncinputch{r''}{}{}
\end{align*}
We type $Q_1$ and $Q_2$ under the respective contexts $\Gamma_1$ and $\Gamma_2$
\begin{align*}
    \Gamma_1 \defeq&\; a : \forall_0 i.\texttt{oserv}^{i+1}(\texttt{Nat}[0,i],\texttt{ch}_{i+1}()), n : \texttt{Nat}[0,i], m : \texttt{Nat}[0,i-1],\\ &\; r : \texttt{ch}_{i}(),
     r' : \texttt{ch}_{i}(), r'' : \texttt{ch}_i()\\
    %
    \Gamma_2 \defeq&\; a : \forall_0 i.\texttt{oserv}^{i^2+3i+2}(\texttt{Nat}[0,i],\texttt{ch}_{i+1}()), n : \texttt{Nat}[0,i], m : \texttt{Nat}[0,i-1],\\ &\; r : \texttt{ch}_{i}(),
     r' : \texttt{ch}_i(), r'' : \texttt{ch}_{2i-1}()
\end{align*}
Note that in the original work, the bound on the complexity of server $a$ in context $\Gamma_2$ is $(i^2+3i+2)/2$. However, we are forced to use a less precise bound, as the multiplicative inverse is not always defined for our view of indices. Upon typing process $P_1$, we amass the judgements on the left-hand side, with corresponding judgements with normalized constraints on the right-hand side
%
\begin{align*}
    &\{i\};\emptyset\vDash i + 1 \geq 1\kern7.5em\Longleftrightarrow &  \{i\};\emptyset\vDash -i \leq 0\\
   % &\{i\};\{i \geq 1\} \vDash i-1 \leq i \kern6em\Longleftrightarrow & \{i\};\{1-i \leq 0\} \vDash -1 \leq 0\\
    &\{i\};\{i \geq 1\} \vDash i \leq i + 1\kern5em\Longleftrightarrow &  \{i\};\{1-i \leq 0\} \vDash -1 \leq 0\\
    %
    &\{i\};\{i \geq 1\} \vDash i \geq i\kern6.7em\Longleftrightarrow &  \{i\};\{1-i \leq 0\} \vDash 0 \leq 0\\
    &\{i\};\{i \geq 1\} \vDash 0 \geq 0\kern6.45em\Longleftrightarrow &  \{i\};\{1-i \leq 0\} \vDash 0 \leq 0
\end{align*}
%
As all coefficients in the normalized constraints are non-positive, each judgement is trivially satisfied, and we can verify the bound $i + 1$ on server $a$. For process $P_2$ we correspondingly have the trivially satisfied judgements
\begin{align*}
    &\{i\};\emptyset\vDash i + 1 \geq 1\kern10.3em\Longleftrightarrow &  \{i\};\emptyset\vDash -i \leq 0\\
    &\{i\};\{0 \leq 0\}\vDash i \leq i^2 + 3i + 2\kern5.2em\Longleftrightarrow &  \{i\};\{0\leq 0\}\vDash -i^2-2i-2 \leq 0\\
    %
    &\{i\};\{i \geq 1\} \vDash i \leq i + 1\kern7.9em\Longleftrightarrow &  \{i\};\{1-i \leq 0\} \vDash -1 \leq 0\\
    %
    &\{i\};\{i \geq 1\} \vDash 2i-1 \leq i^2 + 3i + 2\kern3.2em\Longleftrightarrow &  \{i\};\{1-i \leq 0\} \vDash -i^2-i-3 \leq 0\\
    &\{i\};\{i \geq 1\} \vDash i \geq i\kern9.7em\Longleftrightarrow &  \{i\};\{1-i \leq 0\} \vDash 0 \leq 0\\
    %
    &\{i\};\{i \geq 1\} \vDash 0 \geq i^2+i\kern7.5em\Longleftrightarrow &  \{i\};\{1-i \leq 0\} \vDash -i^2-i \leq 0\\
    %
    &\{i\};\{i \geq 1\} \vDash i^2+2i \geq i^2+3i+2\kern2.9em\Longleftrightarrow &  \{i\};\{1-i \leq 0\} \vDash -i-2 \leq 0
\end{align*}

\end{examp}

% In the general sense, however, verifying whether judgements on polynomial constraints are satisfied is a difficult problem, as it amounts to verifying that a constraint is satisfied under all interpretations that satisfy our set of known constraints. In Example \ref{example:needconic}, we show how a constraint that is not satisfied for all interpretations can be shown to be covered by a set of two constraints, by utilizing the transitive, multiplicative and additive properties of inequalities to combine the two constraints. More specifically, we can exploit the fact that we can generate new constraints from any set of normalized constraints by taking a \textit{conical} combination of their left-hand side indices, as we shall formalize in the following section.
% %
% \begin{examp}\label{example:needconic}
%     Given the judgement
%     \begin{align*}
%         \{i\};\{i \leq 3, 5 \leq i^2\} \vDash 5i \leq 3i^2
%     \end{align*}
%     we want to verify that constraint $5i \leq 3i^2$ is covered by the set of constraints $\{i\leq 3, 5 \leq i^2\}$. This constraint is not satisfied by all interpretations, as substituting $1$ for $i$ yields $5 \not\leq 3$. However, we can rearrange and scale the constraints $i\leq 3$ and $5\leq i^2$ as follows
%     \begin{align*}
%         i \leq 3 \iff i-3\leq 0 \implies 5i - 15 \leq 0\\
%         %
%         5\leq i^2 \iff 0 \leq i^2-5 \implies 0 \leq 3i^2-15
%     \end{align*}
%     Then it follows that
%     \begin{align*}
%         5i-15 \leq 3i^2-15 \iff 5i \leq 3i^2
%     \end{align*}
%     %More generally, we can use the transitive, multiplicative and additive properties of the inequality relation to construct new constraints from a set of known constraints, thereby verifying that some constraint does not impose new restrictions on interpretations. 
% \end{examp}
% %