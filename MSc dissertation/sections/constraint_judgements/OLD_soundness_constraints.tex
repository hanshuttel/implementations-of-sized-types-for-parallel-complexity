\section{Soundness of constraint verification}

It is worth noting that the Simplex algorithm does not guarantee an integer solution, and so we may get indices in constraints where the coefficients may be non-integer values. However, we can use the fact that any feasible linear programming problem with rational coefficients also has an (optimal) solution with rational values \cite{keller2016applied}. We use this fact and Lemma \ref{lemma:constraintscaling} and \ref{lemma:constraintcommonden} to show that we need not to worry about the solution given by the Simplex algorithm, given a rational linear programming problem. Definition \ref{def:constraintequivalence} defines what it means for constraints to be equivalent.

\begin{defi}[Constraint equivalence]\label{def:constraintequivalence}
    Let $C_1$, $C_2$ and $C\in\Phi$ be linear constraints with integer coefficients and unknowns in $\varphi$. We say that $C_1$ and $C_2$ are equivalent with respect to $\varphi$ and $\Phi$, denoted $C_1 =_{\varphi;\Phi} C_2$, if we have that
    \begin{equation*}
    \mathcal{M}_\varphi(\{C_1\} \cup \Phi) = \mathcal{M}_\varphi(\{C_2\} \cup \Phi) %\mathcal{M}_\varphi(\{C_0\})
\end{equation*}
where $\mathcal{M}_{\varphi'}(\Phi')=\{\rho : \varphi' \rightarrow \mathbb{N} \mid \rho \vDash C\;\text{for}\; C \in \Phi'\}$ is the model space of a set of constraints $\Phi'$ over a set of index variables $\varphi'$.
    %
    %
    %$\varphi;\Phi\vDash C_1$ if and only if $\varphi;\Phi\vDash C_2$.
    %Two normalized constraints $C_1$ and $C_2$ are said to be \textit{equivalent} if for any index valuation $\rho$, we have that $\rho \vDash C_1$ if and only if $\rho \vDash C_2$.
\end{defi}

\begin{lemma}\label{lemma:constraintscaling}
Let $I \leq 0$ be a linear constraint with unknowns in $\varphi$. Then $I \leq 0 =_{\varphi;\Phi} n I \leq 0$ for any $n>0$ and set of constraints $\Phi$.
\begin{proof}
    This follows from the fact that if $I \leq 0$ is satisfied, then the sign of $I$ must be non-positive, and so the sign of $n I$ must also be non-positive as $n > 0$. Conversely, if $I \leq 0$ is not satisfied, then the sign of $I$, must be positive and so the sign of $n I$ must also be positive.
\end{proof}
\end{lemma}

\begin{lemma}\label{lemma:constraintcommonden}
Let $I \leq 0$ be a normalized linear constraint with rational coefficients and unknowns in $\varphi$. Then there exists a normalized linear constraint $I' \leq 0$ with integer coefficients and unknowns in $\varphi$ such that $I \leq 0 =_{\varphi;\Phi} I' \leq 0$ for any set of constraints $\Phi$.% there exists an equivalent constraint $I' = \normlinearindex{n'}{I'}$ where $n', I'_{\alpha_1}, \dots,I'_{\alpha_{m}}$ are integers.
\begin{proof}
    It is well known that any set of rationals has a common denominator, whose multiplication with any rational in the set yields an integer. One is found by multiplying the denominators of all rationals in the set. As the coefficients of $I$ are non-negative, this common denominator must be positive. By Lemma \ref{lemma:constraintscaling}, we have that $I\leq 0 =_{\varphi;\Phi} n I \leq 0$ where $n$ is a positive number and $\Phi$ is any set of constraints.% the constraint $I \leq 0$ is equivalent to $d I \leq 0$.
\end{proof}
\end{lemma}