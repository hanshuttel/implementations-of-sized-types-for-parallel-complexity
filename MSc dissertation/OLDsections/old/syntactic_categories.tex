\section{Preliminaries}
In this section we start by introducing a collection of syntactic categories used throughout the article to define the languagee syntax and formal semantic. First the basic categories are presented, followed by editorial and abstract syntax tree categories.
\subsection*{Base categories}
First we introduce base categories in our language. \textbf{Var} is the set of variable names $x$. \textbf{Con} is the set of all predefined constants in the language. The metavariable $c$ is not expanded upon in this document due to its specificity to a single language.
\begin{align*}
  x\in\, & \textbf{Var} && \quad \text{Variables}\\
  c \in\, & \textbf{Con} && \quad \text{Predefined constants\quad\quad\quad\quad\quad\quad\quad\quad\quad\quad\quad\quad}\\
  %v \in\, & \textbf{Val} && \quad \text{Values resulting from program evaluation}
\end{align*}
%
\subsection*{Editor-calculus categories}
We introduce four syntactic categories for the editor-calculus. $\mathbf{Aep}$ is the set of editor commands that can be evaluated atomically, $\mathbf{Ecd}$ is the set of possible editor conditions and $\mathbf{Edt}$ is the set of editor command expressions. Finally, $\mathbf{Aam}$ is the set of atomic changes that may be applied to an abstract syntax tree, such as the insertion of a breakpoint. We present the compositional rules of these categories in section \ref{EDITCALCULUS}.
\begin{align*}
  \pi \in\, & \textbf{Aep} && \quad \text{Atomic editor prefixes} \\
  \phi \in\, & \textbf{Ecd} && \quad \text{Editor conditions}\\
  E \in\, & \textbf{Edt} && \quad \text{Editor expressions} \\
  D \in\, & \textbf{Aam} && \quad \text{Atomic AST node modifications}
\end{align*}
%
\subsection*{AST-calculus category}
AST nodes are built from the syntactic category \textbf{Ast}. %All possible circumstances are covered by the single metavariable $a$, including the cursor and breakpoints. We present these constructs in section \ref{astcalculus}.


\begin{align*}
  %p \in\, & \textbf{Prg} && \quad \text{Programs with undo-redo contexts}\\
  %r \in\, & \textbf{Sch} && \quad \text{expression of changes induced}\\
  a \in\, & \textbf{Ast} && \quad \text{Abstract syntax tree nodes}\\
\end{align*}
%