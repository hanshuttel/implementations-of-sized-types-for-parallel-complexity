\chapter{Semantic equivalence}\label{app:languageequiv}
\setcounter{theorem}{1}

\begin{lemma}
Let $P$ and $Q$ be processes such that $P \equiv Q$, then also $P \succeq_K Q$ and $Q \succeq_K P$.\\

where $\succeq_{K}$ is the structural preorder from Kobayashi \cite{Kobayashi2000}.
\begin{proof} By induction on the rules defining $\equiv$. We only consider the rules that are unique to $\equiv$.
\begin{description}
    \item[$\runa{SC-par}$] By rule $\runa{SC-par}$ we have that $P \equiv P'$ such that $P \mid Q \equiv P' \mid Q$. By induction $P \succeq_K P'$ and $P' \succeq_K P$. Then it follows from $\runa{SP-par}$ that $P \mid Q \succeq_K P' \mid Q$ and $P' \mid Q \succeq_k P \mid Q$.
    
    \item[$\runa{SC-res}$] By rule $\runa{SC-res}$ we have that $P \equiv Q$ such that $\newvar{x}{P} \equiv \newvar{x}{Q}$. By induction $P \succeq_K Q$ and $Q \succeq_K P$. Then it follows from $\runa{SP-res}$ that $\newvar{x}{P} \succeq_K \newvar{x}{Q}$ and $\newvar{x}{Q} \succeq_k \newvar{x}{P}$.
\end{description}

\end{proof}%\label{lemma:SCImpliesSP}
\end{lemma}

\begin{lemma}[Semantic equivalence]
Let $P$ and $P'$ be processes. We have that if
\begin{enumerate}
    \item $P \longrightarrow Q$ then $P \longrightarrow_{K} Q$
    \item $P \longrightarrow_{K} Q$ and $P' \succeq_K P$ then $P' \longrightarrow R$ and $R \succeq_{K} Q$
\end{enumerate}
where $\longrightarrow_{K}$ and $\succeq_{K}$ are the reduction relation and structural preorder from Kobayashi \cite{Kobayashi2000}, respectively.
\begin{proof} By induction on the rules defining $\longrightarrow$ and $\longrightarrow_{K}$. We consider the two cases separately, and we only show the clauses for non-trivial rules.
\begin{enumerate}
    \item 
    \begin{description}
        \item[\runa{R-rep}] Assume that $\bang\inputch{a}{\widetilde{v}}{}{P} \mid \outputch{a}{\widetilde{e}}{}{Q} \longrightarrow\; \bang\inputch{a}{\widetilde{v}}{}{P} \mid P[\widetilde{v}\mapsto\widetilde{e}] \mid Q$. We have that $\bang\inputch{a}{\widetilde{v}}{}{P} \mid \outputch{a}{\widetilde{e}}{}{Q}\succeq_K\; \bang\inputch{a}{\widetilde{v}}{}{P} \mid \inputch{a}{\widetilde{v}}{}{P} \mid \outputch{a}{\widetilde{e}}{}{Q}$, and so it follows from application of $\runa{K-struct}$ and $\runa{K-comm}$ that $\bang\inputch{a}{\widetilde{v}}{}{P} \mid \outputch{a}{\widetilde{e}}{}{Q} \longrightarrow_K\; \bang\inputch{a}{\widetilde{v}}{}{P} \mid P[\widetilde{v}\mapsto\widetilde{e}] \mid Q$.
        %
        %\item[\runa{R-comm}] Assume that $\inputch{a}{\widetilde{v}}{}{P} \mid \outputch{a}{\widetilde{e}}{}{Q} \longrightarrow P[\widetilde{v}\mapsto\widetilde{e}] \mid Q$. It trivially follows from application of $\runa{K-comm}$ that $\inputch{a}{\widetilde{v}}{}{P} \mid \outputch{a}{\widetilde{e}}{}{Q} \longrightarrow_K P[\widetilde{v}\mapsto\widetilde{e}] \mid Q$, as $\succeq_K$ is reflexive.
        %
        %\item[\runa{R-zero}] Assume that $\match{e}{P}{x}{Q} \longrightarrow P$
        %
        %\item[\runa{R-succ}]
        %
        %\item[\runa{R-par}]
        %
        %\item[\runa{R-res}]
        
        \item[\runa{R-struct}] Assume that $P \longrightarrow Q$ by rule $\runa{R-struct}$. Then we have that $P \equiv P_1$, $P_1 \longrightarrow P_1'$ and $P_1' \equiv Q$. By Lemma \ref{lemma:SCImpliesSP} it follows that $P \succeq_K P_1$ and $P_1' \succeq_K Q$. By induction $P' \longrightarrow_K Q'$, and so by rule $\runa{K-struct}$, $P \longrightarrow_K Q$.
            %\begin{description}
            %    \item[\runa{SC-par}]
            %    
            %    \item[\runa{SC-res}]
            %\end{description}
        
        
    \end{description} 
    
    \item
    \begin{description}
        %\item[\runa{K-comm}]
        %    
        %\item[\runa{K-zero}]
        %
        %\item[\runa{K-succ}]
        %
        %\item[\runa{K-par}]
        %
        %\item[\runa{K-res-1}]
        %
        %\item[\runa{K-res-2}]
        %
        \item[\runa{K-struct}] Assume that $P \longrightarrow_K Q$ by rule $\runa{K-struct}$ and $P' \succeq_K P$. Then we have that $P \succeq_K P_2$, $P_2 \longrightarrow_K P_2'$ and $P_2' \succeq_K Q$. We show that $P'$ can match this reduction by rule $\runa{R-struct}$ such that $P' \longrightarrow R$ and $R \succeq_K Q$. Rule $\runa{R-struct}$ implies $P' \equiv P_1$ such that $P_1 \longrightarrow P_1'$ and $P_1' \equiv R$. By Lemma \ref{lemma:SCImpliesSP}, $P' \equiv P_1$ implies $P' \succeq_K P_1$ and $P_1 \succeq_K P'$. Thus by the assumption, it must be that $P_1 \succeq_K P$, and by transitivity $P \succeq_K P_2$ implies $P_1 \succeq_K P_2$. By induction $P_1 \succeq_K P_2$ and $P_2 \longrightarrow_K P_2'$ implies $P_1 \longrightarrow P_1'$ and $P_1' \succeq_K P_2'$. Now, it suffices to show that $P_1' \equiv R$ and $P_2' \succeq_K Q$ implies $R \succeq_K Q$. By Lemma \ref{lemma:SCImpliesSP}, $P_1' \equiv R$ implies $P_1' \succeq_K R$ and $R \succeq_K P_1'$, and so it follows from $P_1' \succeq_K P_2'$ that $R \succeq_K P_2'$. By transitivity $P_2' \succeq_K Q$ implies $R \succeq_K Q$, concluding the proof.
        
        
        %By the definition of $\succeq_K$, we have that $P \succeq_K P'' %\mid P_1 \mid \dots \mid P_n$ such that $P' \succeq P''$ and %$P_1,\dots,P_n$ are copies of replicated inputs in $P'$.
        %    \begin{description}
        %        \item[\runa{SP-rep-1}]
        %        
        %        \item[\runa{SP-par}]
        %        
        %        \item[\runa{SP-res}]
        %        
        %        \item[\runa{SP-rep-2}]
        %    \end{description}
            
    \end{description}
\end{enumerate}
\end{proof}
\end{lemma}
%
%%
%
% \begin{lemma}
% Let $P$ and $Q$ be processes such that $P \succeq_{K} Q$. We have that if
% \begin{enumerate}
%     \item $P \longrightarrow R$ then $Q \longrightarrow C$ and $R \succeq_K C$
%     \item $Q \longrightarrow R$ then $P \longrightarrow C$ and $C \succeq_K R$
%     \item $P \longrightarrow_{K} R$ then $Q \longrightarrow_K C$ and $R \succeq_K C$ or $C \succeq_K R$
%     \item $Q \longrightarrow_{K} R$ then $P \longrightarrow_{K} R$
% \end{enumerate}
% where $\longrightarrow_{K}$ and $\succeq_{K}$ is the reduction relation and structural preorder from Kobayashi \cite{Kobayashi2000}, respectively.
% \begin{proof}
% Proof by induction on the rules defining $\longrightarrow$ and $\longrightarrow_{K}$. We consider each case separately.
% \begin{enumerate}
%     \item
%         \begin{description}
%             \item[\runa{R-rep}]
            
%             \item[\runa{R-comm}]
            
%             \item[\runa{R-zero}]
            
%             \item[\runa{R-succ}]
            
%             \item[\runa{R-par}]
            
%             \item[\runa{R-res}]
            
%             \item[\runa{R-struct}]
            
%         \end{description}
        
%     \item 
%         \begin{description}
%             \item[\runa{R-rep}]
            
%             \item[\runa{R-comm}]
            
%             \item[\runa{R-zero}]
            
%             \item[\runa{R-succ}]
            
%             \item[\runa{R-par}]
            
%             \item[\runa{R-res}]
            
%             \item[\runa{R-struct}]
            
%         \end{description}
%     \item 
%         \begin{description}
%             \item[\runa{K-comm}]
            
%             \item[\runa{K-zero}]
            
%             \item[\runa{K-succ}]
            
%             \item[\runa{K-par}]
            
%             \item[\runa{K-res-1}]
            
%             \item[\runa{K-res-2}]
            
%             \item[\runa{K-struct}]
            
%         \end{description}
        
%     \item As $P \succeq_K Q$, this trivially holds by applying rule $\runa{K-struct}$ with a corresponding reduction, such that $P \longrightarrow_K R$.
% \end{enumerate}
% \end{proof}
% \end{lemma}